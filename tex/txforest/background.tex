\section{Background}
\label{sec:Background}

%%\item Background
%%\begin{itemize}
%%\item Forest Language (primitives, load, store)
%%\item Running example 
%%\item Problems with naive semantics
%%\item Transactions to the rescue
%%\end{itemize}

The Forest language includes primitives for describing files,
directories, symbolic links, and associated meta-data.  Meta-data
includes names, owners, permissions, sizes, and timestamps.  File
contents may be represented as simple strings or as structured data
using Pads descriptions~\cite{pads}.  We implemented Forest using
Haskell's quasi-quotation mechanism~\cite{haskell-quasi-quote}.
Figure~\ref{fig:SWAT-description} shows a simple Forest
description of the running example we will use throughout this paper.

Given such a description, the Forest compiler generates functions for
(lazily) loading the contents of the filestore into a Haskell data
structure and for writing a Haskell data structure back to disk.
Writing structures to disk is a two-step process.  In step one, a
\textit{manifest} function writes the structure into a temporary space
and notes any errors.  In step two, a \textit{store} function copies
the temporary store into the correct location.  This two-step process
allows Forest to detect errors without corrupting the mainline
filestore and lets users determine whether the errors should halt the
writing process.

The initial version of Forest did not attempt to ensure that 
concurrent loads and stores did not cause inconsistencies or
corruption.  As the number of users manipulating the file store grows,
this laissez-faire approach becomes untenable.  To address this
weekness, this paper integrates transactions into Forest.  With
Transactional Forest (TxForest) we ensure that all accesses to the
filestore mediated by a Forest description will see and maintain a
consistent view by aborting and restarting transactions that would
otherwise have observed a conflict.

To explain the design of TxForest, we will use the following running
example, drawn from the field of agriculture science.  Specifically,
there is a large community centered on SWAT, a Soil and Water
Assessment Tool~\cite{SWAT}(\url{http://swat.tamu.edu}).  Members of
this community use SWAT to explore tradeoffs related to different uses
of land in a given watershed.  The model includes data related to the
topology of the watershed, current land use of regions within the
watershed, historic precipitation and temperature levels, measurements
of water purity at various locations, etc.  This data is stored in a
large collection of files and directories in the filesystem (XXX:how
many files? how much data?).

An example query that researchers using this tool might ask is
``what type of land use assignment to a given area of a watershed
keeps corn yield above a threshold, maintains housing capacity above
another threshold, and minimizes nitrate levels in nearby streams.''
The SWAT approach to solving such queries involves a concurrent black-box
optimization process in which each thread reads the current values of
all relevant parameters from the file system, computes the current
value of the optimization function, and makes local changes, and re-runs
the optimization function. If the new result is higher than the old
one, the tool writes those changes back into the file system.  Figure
~\ref{fig:SWAT-opt-code} shows Forest code that replicates this process.

\begin{figure}
\begin{code}
Update with relevant parts of SWAT forest description    
[forest|
 \kw{type} Stats = \kw{Directory}
   \{ last :: File Last, topk :: File Topk \}
 \kw{type} Dat   = [ s :: Site | s <- \kw{matches} site ]
 \kw{type} Site  = [ d :: Log  | d <- \kw{matches} time ]
 \kw{data} Log = \kw{Directory}
   \{ log \kw{is} coralwebsrv :: Gzip (File CoralLog) \} |]
\end{code}
\caption{Forest SWAT description. }
\label{fig:SWAT-description}
\end{figure}

\begin{figure}
\begin{code}
Update with relevant parts of SWAT optimization code
\end{code}
\caption{Forest SWAT description. }
\label{fig:SWAT-opt-code}
\end{figure}



