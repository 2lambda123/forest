\begin{abstract}
\pads{} is a declarative language used to describe the
syntax and semantic properties of {\em ad hoc data sources}
such as financial transactions, server logs and scientific
data sets.  The \pads{} compiler reads these descriptions 
and generates a suite of useful data processing tools
such as format translators, parsers, printers and even a query engine,
all customized to the ad hoc data format in question.
Recently, however, to further improve the productivity of
programmers that manage ad hoc data sources,
we have turned to using \pads{} as an {\em intermediate language} in a system
that first infers a \pads{} description directly from example data
and then passes that description to the original compiler for tool generation.
A key subproblem in the inference engine
is the {\em token ambiguity problem} --- the
problem of determining which substrings in the example data
correspond to complex tokens such as dates, URLs, or comments.
In order to solve the token ambiguity problem, the paper studies the
relative effectiveness of three different statistical
models
%, Hidden Markov Models, Hierarchical Maximum Entropy Models,
%and Support Vector Machines, 
for tokenizing ad hoc data.  It also shows 
how to incorporate these models into
a general and effective format inference algorithm.
In addition to using a declarative language (\pads{}) as
a key intermediate form, we have implemented
the system as a whole in \ml{}. 
%Finally,
%it describes an effective new heuristic for simplifying formats inferred
%by the system.
\end{abstract}
