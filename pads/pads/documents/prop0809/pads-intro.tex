
\subsection{The Prototype Description Language}
To understand further how the \pads{} will be used,
let us look at an example of a simple ad hoc data source:
a tiny fragment of data in the Common Log Format (CLF) that web
servers use to log client requests~\cite{wpp}.  
This ASCII format consists of a sequence of
records, each of which has seven fields: the host name or IP address
of the client making the request, the account associated with the
request on the client side, the name the user provided for
authentication, the time of the request, the actual request, the
\textsc{http} response code, and the number of bytes returned as a
result of the request.  The actual request has three parts: the
request method (\eg, \texttt{GET}, \texttt{PUT}), the requested
\textsc{uri}, and the protocol version.  In addition, the second and
third fields are often recorded only as a '-' character to indicate
the server did not record the actual data.  \figref{figure:clf-records}
shows a couple of typical records.



\begin{figure*}
\begin{footnotesize}
%\begin{center}
\begin{verbatim}
207.136.97.49 - - [15/Oct/1997:18:46:51 -0700] "GET /tk/p.txt HTTP/1.0" 200 30
234.200.68.71 - - [15/Oct/1997:18:53:33 -0700] "GET /tr/img/gift.gif HTTP/1.0" 200 409
240.142.174.15 - - [15/Oct/1997:18:39:25 -0700] "GET /tr/img/wool.gif HTTP/1.0" 404 178
188.168.121.58 - - [16/Oct/1997:12:59:35 -0700] "GET / HTTP/1.0" 200 3082
tj62.aol.com - - [16/Oct/1997:14:32:22 -0700] "POST /spt/dd@grp.org/cfm HTTP/1.0" 200 941
214.201.210.19 ekf - [17/Oct/1997:10:08:23 -0700] "GET /img/new.gif HTTP/1.0" 304 -
\end{verbatim}
\caption{Tiny example of Common Log Format records. }
\label{figure:clf-records}
%\end{center}
\end{footnotesize}
\end{figure*}

With this example, we can examine how to use the prototype 
\pads{} language to describe 
the {\em physical layout} and 
{\em semantic properties} of our ad hoc data source. 
The language provides a type-based model:
basic types describe atomic data such as integers, characters, 
strings, dates, urls, \etc, while
structured types describe compound data built from simpler pieces.

In a bit more detail,
the \pads{} prototype provides a collection of broadly useful base
types.  Examples include 8-bit unsigned integers (\cd{Puint8}), 32-bit
integers (\cd{Pint32}), dates (\cd{Pdate}), strings (\cd{Pstring}),
and IP addresses (\cd{Pip}).  Semantic conditions for such base types
include checking that the resulting number fits in the indicated
space, \ie, 16-bits for \cd{Pint16}.  By themselves, these base types
do not provide sufficient information to allow parsing because they do
not specify how the data is coded, \ie{}, in ASCII, EBCDIC, or binary.
To resolve this ambiguity, \pads{} uses the \textit{ambient} coding,
which the programmer can set.  By default, \pads{} uses ASCII.  
% To
% specify a particular coding, the description writer can select base
% types which indicate the coding to use.  Examples of such types
% include ASCII 32-bit integers (\cd{Pa_int32}), binary bytes
% (\cd{Pb_int8}), and EBCDIC characters (\cd{Pe_char}).  In addition to
% these types, users can define their own base types to specify more
% specialized forms of atomic data.

To describe more complex data, the prototype provides a collection of
structured types loosely based on \C{}'s type structure.  In
particular, the prototype has \kw{Pstruct}s, \kw{Punion}s, and \kw{Parray}s
to describe record-like structures, alternatives, and sequences,
respectively.  \kw{Penum}s describe a fixed collection of literals,
while \kw{Popt}s provide convenient syntax for optional data.  Each of
these types can have an associated predicate that indicates whether a
value calculated from the physical specification is indeed a legal
value for the type.  For example, a predicate might require that two
fields of a \kw{Pstruct} are related or that the elements of a
sequence are in increasing order.  Programmers can specify such
predicates using \pads{} expressions and functions, written using a
\C{}-like syntax.  Finally, \pads{} \kw{Ptypedef}s can be used to
define new types that add further constraints to existing types.

\pads{} types can be parameterized by values.  This mechanism serves
both to reduce the number of base types and to permit the format and
properties of later portions of the data to depend upon earlier
portions.  For example, the base type \cd{Puint_FW(:x:)} specifies
an unsigned integer physically represented by exactly \cd{x}
characters, where \cd{x} is a value that has been read earlier in the
parse.  The type \cd{Pstring(:SPACE:)} describes a string
terminated by a space (when \texttt{SPACE} is defined to be \texttt{' '}).  
Parameters can be used with compound types like arrays and unions to
specify the size of an array or which branch of a union should be
taken.  This parameterization is what makes PADS a {\em dependently-typed}
language and substantially different from languages based on
context-free grammars or regular expressions.

\figref{figure:clf} gives a \pads{} description for the Common Log Format
data.  
We will use this example to illustrate some of the basic
features of the current \pads{} language.  
In \pads{} descriptions, types are declared before they are used, 
so the type that describes the totality of the data source appears 
at the bottom of the description.  
In this case,
the type \texttt{clf\_t}  describes the entirety of the
CLF data source (the \texttt{Psource} type qualifier indicates
this fact explicitly).  

\kw{Pstruct}s describe fixed sequences of data with unrelated types.
In the CLF description, the type declaration for
\cd{version_t} illustrates a simple \kw{Pstruct}. It starts with a 
string literal that matches the constant \cd{HTTP/} in the data source.  It 
then has two unsigned integers recording the major and minor version numbers
separated by the literal character \kw{'.'}.  \pads{} supports character, string,
and regular expression literals, which are interpreted with the ambient character 
encoding. The type \cd{request_t} 
similarly describes the request portion of a CLF record.  In addition
to physical format information, this \kw{Pstruct} includes a semantic constraint
on the \cd{version} field.  Specifically, it requires that obsolete methods
\cd{LINK} and \cd{UNLINK} occur only under HTTP/1.1.  This constraint illustrates
the use of predicate functions and the fact 
that earlier fields are in scope during the processing of later fields, as the 
constraint
refers to both the \cd{meth} and \cd{version} fields in the \kw{Pstruct}.

\kw{Punion}s describe variation in the data format.  For example, the
\cd{client_t} type in the CLF description indicates that the first
field in a CLF record can be either an IP address or a hostname.
During parsing, the branches of a \kw{Punion} are tried in order; the
first branch that parses without error is taken.  The \cd{auth_id_t}
type illustrates the use of a constraint: the branch \cd{unauthorized}
is chosen only if the parsed character is a dash.  \pads{} also
supports a \textit{switched} union that uses a selection expression to
determine the branch to parse.  Typically, this expression depends
upon already-parsed portions of the data source.

\pads{} provides \kw{Parray}s to describe varying-length sequences of
data all with the same type.  The \cd{clf_t} declaration  uses a
\kw{Parray} to indicate that a CLF file is a sequence of \cd{entry\_t}
records.  This particular array terminates when the data source is
exhausted. In general, \pads{} provides a rich
collection of array-termination conditions: reaching a maximum size,
finding a terminating literal (including end-of-record), or satisfying a
user-supplied predicate over the already-parsed portion of the \kw{Parray}. 
\pads{} also has convenient syntax for 
specifying separators that appear between elements of an array and
declaring inter-element constraints including sorting.

The
\kw{Penum} type \cd{method_t} describes a collection of data literals.
During parsing, \pads{} interprets these constants using the ambient
character encoding.  The \kw{Ptypedef} \cd{response_t} describes
possible server response codes in CLF data by adding the constraint
that the three-digit integer must be between 100 and 600.

Finally, the \kw{Precord} annotations deserve comment. It
indicates that the annotated type constitutes a record.  
The notion of a record varies depending upon the data encoding.  
ASCII data typically uses new-line characters to delimit 
records, binary sources tend to have fixed-width records, while 
COBOL sources usually store the length of each record before the actual data.
\pads{} supports each of these encodings of records and allows users to define
their own encodings.  By default, \pads{} assumes records are new-line terminated.
Before parsing, however, the user can direct \pads{} to use a different record
definition.
\texttt{Precords} have error-recovery semantics -- if errors 
in the data cause the parser to become seriously confused,
it will attempt to recover to a record boundary.
In practice, we have found this to be a very robust recovery mechanism
for ad hoc data.  

\begin{figure}
\begin{small}
\begin{code}
\kw{Punion} client\_t \{
  Pip       ip;      /- 135.207.23.32
  Phostname host;    /- www.research.att.com
\};
\mbox{}
\kw{Punion} auth\_id\_t \{
  Pchar unauthorized : unauthorized == '-';
  Pstring(:' ':) id;
\};
\mbox{}
\kw{Pstruct} version\_t \{
  "HTTP/";
  Puint8 major; '.';
  Puint8 minor;
\};
\mbox{}
\kw{Penum} method\_t \{
    GET,    PUT,  POST,  HEAD,
    DELETE, LINK, UNLINK
\};
\mbox{}
bool chkVersion(version\_t v, method\_t m) \{
  \kw{if} ((v.major == 1) && (v.minor == 1)) \kw{return} true;
  \kw{if} ((m == LINK) || (m == UNLINK)) \kw{return} false;
  \kw{return} true;
\};
\mbox{}
\kw{Pstruct} request\_t \{
  '\\"';   method\_t       meth;
  ' ';    Pstring(:' ':) req\_uri;
  ' ';    version\_t      version :
                  chkVersion(version, meth);
  '\\"';
\};
\mbox{}
\kw{Ptypedef} Puint16\_FW(:3:) response\_t :
         response\_t x => \{ 100 <= x && x < 600\};
\mbox{}
\kw{Precord} \kw{Pstruct} entry\_t \{
         client\_t       client;
   ' ';  auth\_id\_t      remoteID;
   ' ';  auth\_id\_t      auth;
   " ["; Pdate(:']':)   date;
   "] "; request\_t      request;
   ' ';  response\_t     response;
   ' ';  Puint32        length;
\};
\mbox{}
\kw{Psource} \kw{Parray} clf\_t \{
  entry\_t [];
\}
\end{code}
\end{small}
\caption{\pads{} description for Web Log data.}
\label{figure:clf}
\end{figure}
