\begin{figure}
\begin{small}
\begin{verbatim}
discover (chunks) {
  1. Guess the top-level type constructor 
     by analyzing chunks.

  2a. If Guess is BASE TYPE b then  
        return b;

  2b. If Guess is a STRUCT with the shape 
         c0, b1, c1, b2, c2, ..., bk, ck 
         where b1, b2, ..., bk are base types
           and c0, c1, c2, ..., ck are the 
               token sequences found between 
               b1, b2, ..., bk   
      then
        T0 = discover(c0);
        T1 = discover(c1);
        ...
        Tk = discover(ck);
        return (struct {T0; b1; 
                        T1; b2; 
                        T2; ...; bk; Tk;});

  2c. If Guess is an ARRAY with elements
         c0, c1, ..., ck-1, ck
      then
        Tfirst = discover(c0);
        Tbody  = discover(merge [c1,c2,...,ck-1]);
        Tlast  = discover(ck);
        return (struct {Tfirst; 
                        array {Tbody}; 
                        Tlast;});

  2d. If Guess is a UNION partitioning chunks into 
         c0, c1, ..., ck
      then
        T0 = discover(c0);
        T1 = discover(c1);
        ...
        Tk = discover(ck);
        return (union {T0;T1;...;Tk});
}
\end{verbatim}
\end{small}
\caption{The structure discovery algorithm. Needs formatting!!!
         And maybe rewriting to functional style?}
\label{fig:structure-discovery}
\end{figure}
