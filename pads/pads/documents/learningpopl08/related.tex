Researchers have been studying {\em grammar induction}, the process of
inferring descriptions of text-based data, for decades.  Nevertheless,
the work we present in this paper represents an important and novel 
contribution to the field for three key reasons:

\begin{enumerate}
\item Our system solves {\em a new end-to-end problem} not treated in
past work --- the problem of generating an extensible suite of fully
functional data processing tools directly from ad hoc data.  
%%We can
%%currently generate an XML translator, a normalizing reformatter, a
%%graphing tool, a full query engine allowing users to write arbitrary XQueries
%%against the ad hoc data, an accumulator tool, and
%%programming libraries for parsing, printing and data validation.
Generating this suite requires the combination of three elements:
grammar induction, automatic intermediate representation generation
and type-directed programming.  A key contribution of this work is the
conception, development and evaluation of this end-to-end system.

%%After surveying
%%experts at the CAGI 2007 workshop on grammar induction, where we
%%presented a two-page overview of our system~\cite{burke+:cagi07}, and
%%searching the literature, we could find no existing system that
%%provides this end-to-end functionality.

\item Past work on grammar induction has focused primarily on
either (1) theoretical problems, (2) natural language processing, 
(3) web page analysis, or
(4) XML typing.  Our work tackles an understudied domain, that of complex system
logs and other ad hoc data sources.  Since ad hoc data has
different characteristics from the previously studied domains, naive
adaptations of the existing algorithms are unlikely to be %the most
effective.  
%%As the evaluation in this paper shows, our system is tuned
%%to perform well on ad hoc data, particularly system logs and
%%networking data.  
%%One of the conclusions of the chair of the CAGI 2007
%%workshop, presented in the final discussion session of the workshop,
%%was that ``ad hoc data'' was indeed a new domain for the study of
%%grammar induction and that more research in this area was an important
%%future direction for the community.

\item  From a technical standpoint, we developed a new top-down 
structure-discovery algorithm and showed how to combine that 
productively with a classic bottom-up rewriting system based on 
the minimum description length principle. We demonstrate that our
new algorithm has good practical properties on ad hoc data sources:  
it usually infers correct descriptions on a small amount of training
data and its performance scales linearly relative to the amount of training
data used.
\end{enumerate}

\noindent
%We presented a two-page overview of our system~\cite{burke+:cagi07} at
%the CAGI 2007 workshop on grammar induction. 
In the rest of this section, we analyze
the most closely related work in more depth.

\paragraph*{Traditional Grammar Induction.}
Classic grammar induction algorithms \cite{vidal:gisurvey} 
can be divided into two classes: those that require both
positive and negative examples to discover a grammar and those that
only require positive examples. The problem our system solves is the latter;
negative examples of ad hoc data sources are not available in
practice.  Consequently, effective theoretical algorithms for learning
from both positive and negative
examples such as RPNI~\cite{rpni}
%~\cite{lemay+:tree-transducers,rpni,raeymaekers+:learning-tree-languages},
are not applicable in our context.

Unfortunately, an early result by \citet{gold:inference} showed
that perfect grammar induction is impossible for any superfinite class
of languages when the algorithm has no access to negative examples.  A
{\em superfinite} class of languages is any set of languages that
includes all finite languages and at least one infinite
language. Hence, all the most familiar classes of languages, including
regular expressions, context free grammars and PADS are superfinite.
There are two main tactics one can use to avoid this negative
result: 
(1) use domain knowledge to explicitly limit the class of languages to a
non-superfinite class, or
(2) give up on perfect language identification and instead settle for {\em approximate
identification}~\cite{wharton:approximate-language-identification}
through the use of probabilistic language models.

Examples of non-trivial, non-superfinite
language classes with known inference algorithms include
k-reversible languages~\cite{angluin:revesible-language-inference},
%k-testable regular languages~\cite{garcia+:k-testable-languages},
SOREs and CHAREs~\cite{bex+:dtd-inference}.
None of these languages and the associated algorithms 
are a good fit for inferring PADS descriptions (even the
regular subset of PADS without dependencies and constraints).  
For example, ad hoc data is unlikely to be reversible and hence
k-reversible languages are not relevant.  
%K-testable regular languages are
%more relevant, but algorithms for inferring them
%operate by finding a finite automaton and converting that 
%automaton into a regular expression.  Unfortunately, the conversion process
%often leads to overly verbose regular expressions, sometimes 
%exponential in the size of the automaton~\cite{bex+:dtd-inference}. 
SOREs are a subset of the k-testable
regular languages with a linear-size translation from automata to
regular expressions, but they carry the restriction that each symbol
in the regular expression appear at most once.  A cursory glance at
our hand-written PADS descriptions reveals that many such descriptions
include repeated use of the same symbol.  Finally, it appears that
CHAREs restrict the nesting of regular expression operators too severely to 
be of much use to us.  For example, when $a$, $b$, and $c$ are atomic symbols,
even the simple expression $(ab + c)*$ is not a CHARE.

Given the difficulty of finding useful non-superfinite language classes,
it is reasonable to turn to algorithms for approximate
inference that use probabilistic models.    
Classic examples of such procedures include work by~\citet{stolcke94inducing} 
%%{\em insert other references here -- see Hong thesis related work
%%for other work...} 
and 
\citet{hong:thesis}.  These and a number of other algorithms
operate by repeatedly rewriting a candidate grammar (or set of candidate
grammars) until an objective function is optimized.
If the training data for the learning system is the strings
$s_1$, $s_2$, $\ldots$, $s_n$, these algorithms normally start their
process using the grammar $s_1 + s_2 + \cdots + s_n$.  Consequently,  
an enormous number of different rewrites may apply to the
initial candidate grammar.  Our structure refinement
phase avoids these problems 
because it is preceded by a highly efficient
histogram-based structure-discovery algorithm 
that identifies a good candidate grammar from which to start the search.  
%%Another interesting, non-standard element of our algorithm is the way 
%%it is tuned to include specialized rules for finding constraints and 
%%rewrite tokens.
%%These rules are very useful in the domain of ad hoc data; different
%%considerations are appropriate in other domains such as XML or HTML.

%% is also tuned in a variety of ways to make it effective

%% The effectiveness of structure-discovery allows us to 
%% simplify our search algorithm and cut down the search space we 
%% look at substantially.  In addition to worrying about
%% performance considerations, we tuned our structure refinement phase
%% specifically for ad hoc data by including domain-specific rules
%% for finding constraints and 
%% dependencies as well as those for introducing constants, enumerations and
%% good basic types/tokens that cannot be found effectively at earlier stages.
%% Some of these rules are needed in our system, but not other systems
%% that work in different domains, because
%% tokenization is highly ambiguous in ad hoc data.
%% Our initial tokenization and structure-discovery algorithms often 
%% over-generalize and this over-generalization must be undone during
%% the rewriting phase.  {\em NOTE: end of that paragraph was highly  run-on}

Another category of algorithms are those that learn various kinds of
automata as opposed to regular expressions or 
grammars~\cite{denis:learning-regular-languages,rpni,raeymaekers+:learning-tree-languages}.  
One difficulty with adapting these algorithms to our task is that 
we would need to convert the inferred automata into a 
grammatical representation so that we can
present the result to users and funnel it
to our tool-generation infrastructure.
Unfortunately, in theory, conversion from automata into
regular expressions can result in an exponential blowup in the
size of the representation.
Moreover, a substantial blowup appears to be relatively common in
practice~\cite{bex+:dtd-inference}.  Consequently, these algorithms
are not appropriate for our domain.



%% developed another system for information extraction from web pages
%% based on learning ($k$,$l$)-Contextual Tree Languages.  They show that
%% these tree languages can be learned from positive examples 
%% (from which one may infer they are not superfinite) and apply
%% their techniques to the problem of information extraction.  One of the
%% difficulties they face involves estimating the parameters involved in

  
%% {\em NOTE: I'm leaving out reference to denis:learning-regular-languages (see
%% pads.bib file).
%% Vincent Danos mentioned it but it contains no references to any real data.
%% It's interesting theoretical result that infers a new kind of automaton.
%% This automaton will likely have the same potential difficulties 
%% (possibly exponential explosion -- I haven't prove that though) 
%% when conversion to regular expressions happens. Anyway, I just didn't
%% want to bother studying the paper because they're so much other stuff
%% that is more relevant.  basically, I just couldn't figure out how to cram
%% the reference in elegantly.  I actually couldn't even figure out when I skimmed
%% the paper whether or not it uses both positive and negative examples.
%% It gets compared to RPNI, so I think it must use negative examples.
%% }

%% One disadvantage of such
%% techniques is that the initial state is large (representing
%% the entire training data set explicitly) and the search space is 
%% enormous.  Nevertheless, bottom-up state-merging is often used because
%% it has been difficult to find an effective state-splitting algorithm.
%% Our histogram-based structure-discovery procedure is a new state-splitting
%% algorithm that appears to work well on ad hoc data when coupled with
%% bottom-up rewriting.


%% The classic grammar induction problem~\cite{vidal:gisurvey} requires we find an
%% algorithm that discovers a grammar $G$ given a set of
%% positive examples $R+$ (example strings in the language to be inferred)
%% and a set of negative examples $R-$ (example strings {\em not}
%% in the language to be inferred).  To be more specific, in the limit,
%% as the sets of positive and negative examples grow, the
%% algorithm is expected to converge on the language that defines them.  
%% Unfortunately, very early on,
%% Gold~\cite{gold:inference} proved a key negative result about this problem:  If
%% the algorithm is presented with no negative examples, grammar
%% induction for any super-finite class of languages is impossible.
%% A {\em super-finite} class of languages is any set of languages
%% that includes all finite languages and at least one infinite language.
%% All the most familiar classes of languages, including regular expressions, 
%% context free grammars and PADS, fall into this class.

%% Traditional
%% Some traditional grammar induction algorithms assume that
%% both positive and negative training data
%% One way to categorize research in traditional
%% grammar induction is to ask whether the research in question
%% assumes that both positive and negative training data is available
%% or whether only positive training data is available.

%% analyze the assumptions made
%% about the training data.  
%% Very early in the study of grammar induction, Gold proved a key
%% negative result:


%% Other researchers have defined grammar induction algorithms that use
%% bottom-up rewriting to search through description space for an optimal
%% description.  Many of these techniques, such as 
%% require the availability of both
%% positive and negative examples.  In our context, negative examples
%% never exist, making such techniques inapplicable.
%% % since Gold's early result proved the
%% %impossibility of {\em perfect} grammar induction for any useful family of
%% %languages when no negative examples are
%% %available~\cite{gold:inference}.  
%% However, others, such as Stolcke and
%% Omohundro~\cite{stolcke94inducing} and Hong~\cite{hong01using}, do not
%% assume the existence of negative examples.  These and a number of other systems
%% search through solution space using
%% state-merging rewriting rules.  One disadvantage of such
%% techniques is that the initial state is large (representing
%% the entire training data set explicitly) and the search space is 
%% enormous.  Nevertheless, bottom-up state-merging is often used because
%% it has been difficult to find an effective state-splitting algorithm.
%% Our histogram-based structure-discovery procedure is a new state-splitting
%% algorithm that appears to work well on ad hoc data when coupled with
%% bottom-up rewriting.

% State-merging rewriting rules seem to be
% more popular than 
% that use bottom-up rewriting to find good grammars may suffer from the
% problem of running into local maxima.  The rewriting component of our
% algorithm can also run into a local maximum, but because we start with
% a relatively good candidate generated from our recursive, top-down
% algorithm, this does not appear to be much of a problem for us.  We
% also believe that combining top-down structure-discovery with
% bottom-up rewriting has the potential to deal with larger data sources
% than a pure bottom-up approach.  Our empirical experiments demonstrate
% that the top-down structure-discovery phase is extremely efficient
% when compared with the cost of rewriting.  However, proposals for
% bottom-up-only inference techniques use the (possibly enormous) data
% source itself as the first description.  We are unaware of other
% systems that combine two techniques similar to ours.


%% ; De La Higuera
%% surveys some recent trends~\cite{higuera01current}.  However,
%% our system is unique in two important ways.  First, our inference
%% algorithm does not stand alone; it is part of the more general \pads{}
%% programming environment.  The fusion of the
%% \pads{} system, including its automatic data representation generation,
%% its error detection facilities, its generic programming environment, 
%% and its powerful tool suite, together with grammar induction
%% is one of our key contributions.  Second, many researchers have
%% focused either on grammar induction for natural language processing or
%% for information extraction from \xml{} or \html{} documents.  In
%% contrast, we focus on ad hoc data sources such as system logs and
%% scientific data sets. Ad hoc data is substantially less
%% structured syntactically than \xml{}, and yet, unlike natural language, it is
%% possible to assign our data sources accurate, compact descriptions. After
%% searching the literature and consulting
%% with experts in grammar induction at the CAGI 2007
%% workshop, where we presented a two page overview of our system~\cite{burke+:cagi07},
%% we could find no existing work comparable to ours.

% Third, from a
% technical standpoint, we developed a new top-down structure-discovery
% algorithm and showed how to combine that productively with a
% classic bottom-up rewriting systems based on the minimum description
% length principle.  In what follows, we compare our system more
% specifically to the most closely related work of other researchers.

\paragraph*{Information Extraction.}

The basic goal of an information extraction system is to find and
separate the interesting and relevant bits of information (the
needles) from a haystack of data.  Such systems are fundamentally
different from ours, in that they choose which bits of information to
extract, while we learn a description of the entirety of a data
source, leaving the choice about which pieces are interesting to
down-stream applications.  Of course, this option is only feasible
because we target ad hoc data, which is fairly structured and dense in
useful information, rather than web pages or free text, which are the
usual targets for information extraction systems. 

A common approach to information extraction involves an inductive
learning process in which a user manually tags the relevant data in sample documents.
An example might be highlighting product names and prices on a
collection of shopping web pages from a particular site.  The learning
system then uses these labelled documents in two ways: first, to
decide which bits of information should be extracted from the page
(\ie, product names and prices), and second, to construct a
\textit{wrapper} function to extract those bits of information from
similar pages.  Soderland's WHISK system (\citeyear{soderland:whisk}) is an
example of such an extraction system.  It is particularly general as
it makes few assumptions about the form of the source text,
operating over structured data, stylized text such as Craig's List
descriptions, or free-form text.  WHISK differs from our system in
that it requires user labeling and then only extracts a collection of
tuples from the data source rather than returning the complete
structure of the data source.


Kushmerick and
colleagues (\citeyear{kushmerick-phd1997,KushmerickWD97:Wrapper}) focus on
more structured data to reduce the amount of labeling required during
training.  In particular, this work assumes the labelled pages conform
to one of six different templates, the most well-developed of which
has the form of a header, followed by a sequence of K-tuples each of
which is flanked by a pair of begin and end tags, followed by a
trailer.  For such documents, the system generates a wrapper to
extract the K-tuples.  
% To limit the amount of labeling, the system
% has provisions to automatically tag the desired tuples using {\it
% recognizers}, which are imperfect, but reusable heuristics for finding
% atomic pieces of data such as country names or phone numbers.  Such
% recognizers mean the user has to select which set of recognizers to
% use for a particular extraction task instead of labeling pages by
% hand.  The system requires one recognizer to be {\it perfect}, meaning
% it generates neither false positives nor false negatives.  It then
% uses a process called {\it corroboration} to correct the mistakes of
% the other recognisers.  The system is not robust in the presence of
% missing data, and it is not clear how it would handle multiple
% instances of the same kind of data within a single tuple. This
% approach differs from ours in that it requires the data to comform to
% one of a fixed collection of templates.  In addition, the 
% templates that support corroboration will only return relational data,
% whereas our system will return semi-structured data.
The use of fixed templates and the primary focus on relational data makes this
work quite different from ours.

\citet{muslea+:active-learning} tackle a similar
problem, but strive to reduce the amount of labeling by having the
learning system chose which documents to have the user label,
selecting documents by their probative value.  \citet{borkar+:text-segmentation} uses hand-labelled training
examples and a user-specified set of desired features to train Hidden
Markov Models to select the desired features from similar documents.
This work is quite successful at learning to select the relevant
features of addresses and bibliographic citations from a variety of
input formats. 
% Various researchers have leveraged the syntactic
% regularity and verbosity of XML/HTML to reduce the amount of user
% annotations required to train information extraction systems targetted
% at web pages~\cite{Ambite+:ariadne,doorenbos+:shopbot}.
% Work by Ireson {\em et al.}~\cite{ireson+:ml-evaluation} investigates
% how information extraction systems should be evaluated.  Soderland's
% WHISK paper~\cite{soderland:whisk} and Kushmerick's
% theis~\cite{kushmerick-phd1997} both contain detailed descriptions of
% other information extraction systems.  
In general, systems that depend
upon labeling are unlikely to be helpful in our context; rather than
spending time explicitly labeling documents, the user might as well
write a PADS description by hand.

% Another type of information extraction system strives to provide a
% high-level semantic characterization of the content of natural
% language documents to guide information retrieval
% queries~\cite{gubanov+:structural-text-search,rus+:information-capture}.
% This work differs from ours in that it is building a semantic rather
% than a syntatic description of the source data.

More closely related are various efforts to identify tabular data 
either from free-form text~\cite{Ng+:texttables,Pinto+:texttables} or
from web pages~\cite{Lerman+:webtables}.  These approaches typically
use hand-labelled examples to train machine learning systems to
identify the tables.  They then use heuristics specific to tabular
data to extract the tuples contained within those tables.  The portion
of this work related to identifying structured data from within more
free-form documents is complementary to ours.  The portion responsible
for deconstructing the identified tables uses more specific
domain-knowledge related to the form of tables than we do.

Web pages generated in response to queries tend to be formed by
sloting the resulting tuples into a standard template.  Another line
of work aims to separate such templates from the payload
data~\cite{arasu+:sigmod03,Cresenzi+:roadrunner}.  
Arasu and Garcia-Molina %~\cite{arasu+:sigmod03}
use a top-down grammar induction
algorithm somewhat similar to our rough structure-inference phase
(though it does not use histograms),
but has no description-rewriting engine.  
%However, in certain ways, Arasu has a much easier task than we do as html
%documents have far more regular structure than ad hoc data sources do.
This algorithm exploits the hierarchical nesting
structure of \xml{} documents in essential ways
and so cannot be applied directly to ad hoc data.  
%For example,
%we use histograms to summarize the contents of data chunks whereas
%Arasu does not.  In addition, a substantial portion of our system
%is a description rewriting engine, which Arasu seems not to need.  






% For further reading on
% information extraction from web pages, Hong's
% thesis~\cite{hong:thesis} includes an informative survey.  Though,
% Arasu's work and TSIMMIS appear more closely related to our work than
% the others Hong mentions.


\paragraph*{XML Type Inference.}
Many researchers have studied the problem of learning
a schema such as a DTD or XSchema from a collection 
of XML
documents~\cite{bex+:dtd-inference,bex+:inferring-xml-schema,fernau:learning-xml,garofalakis+:xtract}.  
At a high level, this task is similar to the format inference component of our system.  
However, the details differ because XML has different characteristics
from ad hoc data.  One difference is that XML documents come in a well-nested tree 
shape, with obvious delimiters defining the structure.  
A second important difference is that the appropriate tokenization for
a given ad hoc data source is often not known in advance.  
%%One of our strategies for dealing with
%%these ambiguities is to define simple approximate tokens for use
%%in the tokenization phase, but then to employ a collection of 
%%rules to improve token ({\em i.e.}, base type) choices in the rewriting phase
%%when more contextual information is available.  
In contrast,
tokens in XML documents are clearly demarcated using angle bracket syntax.
%%A third difference is that XML documents are often organized such that
%%the structure of a child node is dependent on its parent or grandparent.
%%In contrast, in the flatter ad hoc data we have considered, dependencies
%%generally arise between siblings -- some data item to the left influences
%%the structure of data to the right.  
As a result of these differences,
XML inference algorithms cannot be used ``off-the-shelf'' for understanding
the structure of ad hoc data.  They must be modified, tuned and
empirically evaluated on this new task.

One line of research on schema inference for XML makes use of the 
observation that 99\% of the content models for XML nodes are defined as
SOREs or CHAREs~\cite{martens+:expressiveness-xml-schema}. 
%(recall, these
%are heavily restricted forms of regular expressions).  
This observation allows \citet{bex+:dtd-inference} to define
an efficient algorithm for inferring concise DTDs.  Later 
\citet{bex+:inferring-xml-schema} build on this work 
by showing how to infer $k$-local XML Schema definitions also based on
SORES.  A $k$-local definition allows node content to depend on the parent
tag, grandparent tag, etc. (up to $k$ levels for some fixed $k$).
As mentioned earlier, hand-written PADS descriptions do not generally obey
the SOREs or CHAREs restriction, nor are they generally arranged with a nesting
structure that suggests $k$-local inference will be particularly useful.
The successful application of these techniques to XML data reinforces 
the idea that the ad hoc data we analyze has quite different characteristics
from XML, and therefore the ad hoc data inference problem merits study
independent of the XML inference problem.

XTRACT~\cite{garofalakis+:xtract} is another system for inferring DTDs
for XML documents.  It operates in three phases: generalization,
factoring and MDL optimization.  The first phase plays a role similar to
our structure discovery phase in that it generates a
collection of candidate structures from a series of XML examples.
This generalization phase searches for patterns in XML
data; it is tuned using the authors' knowledge of common DTD
structures.  Factoring decreases the size of generated candidate DTDs;
some of the factoring rules resemble our rewriting rules.
Finally, they tackle the MDL optimization problem by mapping the
problem into an instance of the NP-complete Facility Location Problem,
which they solve using a quadratic approximation algorithm.
Our MDL-guided rewriting problem considers a more general set of
rewriting rules and hence we cannot reuse their technique.

%% Another related problem of great interest in the XML and database world
%% involves finding a mapping between two data sources with different schema.
%% \citet{doan+:disparate-data-sources} is one example amongst
%% many which attempts to solve this problem using a machine learning approach.
%% While some of our PADS tools do involve translations between
%% different formats, our learning system does not attempt to discover
%% translation tools for which the output is guaranteed to match 
%% the characteristics of a second data set.


\paragraph*{Other work.}
Potter's Wheel~\cite{raman+:potterwheel} is a system that attempts to
help users find and purge errors from
relational data sources.  It does so through the use of a spread-sheet
style interface, but in the background, a grammar inference algorithm
infers the structure of the input data, which may be ``ad hoc,'' 
somewhat like ours.  This inference algorithm operates by
enumerating all possible sequences of base types that appear
in the training data.  
%As in our work,
%users can specify custom base types, and search for a description
%is based on the minimum description length principle.  
Since Potter's Wheel is aimed at processing
relational data, they only infer \cd{struct} types
as opposed to enumerations, arrays, switches or unions.  

The TSIMMIS project~\cite{chawathe+:tsimmis} aims to
allow users to manage and query collections of heterogeneous, ad hoc
data sources.  TSIMMIS sits on top of the Rufus
system~\cite{shoens+:rufus}, which supports automatic classification
of data sources based on features such as the presence of certain
keywords, magic numbers appearing at the beginning of files and file
type.  
%The sources are classified using categories such as ``email''
%and ``C program.''  
This sort of classification is materially
different from the syntactic analysis we have developed.
