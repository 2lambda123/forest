%\documentclass{acm_proc_article-sp-sigmod06}
\documentclass{sig-alternate-sigmod06}
\usepackage{url}
\usepackage{stmaryrd}
\usepackage{epsfig}
\usepackage{alltt}
\usepackage{times}
\usepackage{code}
\usepackage{xspace}

% \pdfpagewidth=8.5in
% \pdfpageheight=11in

\renewcommand{\floatpagefraction}{0.9}
\renewcommand{\dbltopfraction}{0.9}
\renewcommand{\dblfloatpagefraction}{0.9}
\addtolength{\textfloatsep}{-8pt}
\addtolength{\dbltextfloatsep}{-8pt}

\newcommand{\cut}[1]{}

\newcommand{\appref}[1]{Appendix~\ref{#1}}
\newcommand{\secref}[1]{Section~\ref{#1}}
\newcommand{\tblref}[1]{Table~\ref{#1}}
\newcommand{\figref}[1]{Figure~\ref{#1}}
\newcommand{\listingref}[1]{Listing~\ref{#1}}
%\newcommand{\pref}[1]{{page~\pageref{#1}}}

\newcommand{\eg}{{\em e.g.}}
\newcommand{\cf}{{\em cf.}}
\newcommand{\ie}{{\em i.e.}}
\newcommand{\etc}{{\em etc.\/}}
\newcommand{\naive}{na\"{\i}ve}
\newcommand{\role}{r\^{o}le}
\newcommand{\forte}{{fort\'{e}\/}}
\newcommand{\appr}{\~{}}

\newcommand{\bftt}[1]{{\ttfamily\bfseries{}#1}}
\newcommand{\kw}[1]{\bftt {#1}}
\newcommand{\Pthen}{\kw{Pthen}}
\newcommand{\pads}{\textsc{pads}}
\newcommand{\padsl}{\textsc{padsl}}
\newcommand{\padst}{\textsc{pads/t}}
\newcommand{\datatype}{\textsc{PADS/T}}
%\newcommand{\datatype}{\textsc{DataType}}
\newcommand{\C}{\textsc{C}}
\newcommand{\perl}{\textsc{Perl}}
\newcommand{\ml}{\textsc{ml}}
\newcommand{\sml}{\textsc{sml}}
\newcommand{\smlnj}{\textsc{sml/nj}}
\newcommand{\java}{\textsc{java}}
\newcommand{\ddl}{\textsc{ddl}}
\newcommand{\xml}{\textsc{xml}}
\newcommand{\datascript}{\textsc{DataScript}}
\newcommand{\packettypes}{\textsc{PacketTypes}}
\newcommand{\erlang}{\textsc{Erlang}}

\newcommand{\Core}{Ad hoc}
\newcommand{\core}{ad hoc}
\newcommand{\pvalue}{\core{} value}
\newcommand{\ppat}{\core{} pattern}
\newcommand{\ptype}{\core{} type}

\newcommand{\padsc}{\textsc{pads}/\C{}}
\newcommand{\padsml}{\textsc{pads}/\ml{}}

\newcommand{\dibbler}{Sirius}
\newcommand{\ningaui}{Altair}
\newcommand{\darkstar}{Regulus}

\newcommand{\pdgood}{{\tt G}}
\newcommand{\pdbad}{{\tt B}}
\newcommand{\pdnest}{{\tt N}}
\newcommand{\pdsem}{{\tt S}}
\newcommand{\ptypes}{T}
\newcommand{\patreadpd}[2]{{\tt #1<<#2>>}}
\newcommand{\btm}{\cd{BOT}}


\newcommand{\lsem}{{[\![}}
\newcommand{\rsem}{{]\!]}}


\newcommand{\figHeight}[4]{\begin{figure}[tb]
	\centerline{
	            \epsfig{file=#1,height=#4}}
	\caption{#2}
	\label{#3}
	\end{figure}}

%% Environment for typesetting BNF grammars. Uses display math mode.
\newenvironment{bnf}
     {%% local command definitions:
        %% BNF definition symbol
      \def\->{\rightarrow}
%%      \def\::={{::=} &}
      \def\::={\bnfdef &}
      \def\|{\bnfalt}
      \newcommand{\name}[1]{\text{##1}}
        %% non-terminal
      \newcommand{\nont}[1]{{##1}}
      \newcommand{\meta}[1]{& ##1 &}
      \newcommand{\descr}[1]{& \text{// ##1}}
      \newcommand{\opt}[1]{ [##1] }
      \newcommand{\opnon}[1]{\opt{\nont{##1}}}
      \newcommand{\none}{\epsilon}
      \newcommand{\nwln}{\\ &&&}
      \newcommand{\nlalt}{\\ && \| &}
      \[\begin{array}{lrlll}
     }
     {\end{array}\]}

\newcommand{\mcd}[1]{\mathtt{#1}}
\newcommand{\ppair}[3]{#1{:}#2 \mathrel{**} #3}
\newcommand{\parray}[3]{#1\;\mcd{Parray}(#2,#3)}
\newcommand{\pset}[3]{\{#1{:}#2\,|\,#3\}}
\newcommand{\pstream}[1]{#1\;\mcd{stream}}
\newcommand{\precord}[1]{\{\{#1\}\}}

\newcommand{\mono}[1]{\texttt{#1}}
\newcommand{\dibbler}{Sirius}
\newcommand{\ningaui}{Altair}
\newcommand{\darkstar}{Regulus}

\newcommand{\abstractdm}{abstract data model}
\newcommand{\concretedm}{concrete data model}
\newcommand{\typeddm}{type-specialized concrete data model}

\title{PADS: An End-to-end System for Processing Ad Hoc Data}

\numberofauthors{3} 
\author{\alignauthor Mark Daly\\
\affaddr{Princeton University}\\
\email{mdaly@princeton.edu}
\alignauthor Yitzhak Mandelbaum and David Walker\\
\affaddr{Princeton University}\\
\email{yitzhakm@cs.princeton.edu\\dpw@cs.princeton.edu}
\alignauthor Mary Fern\'andez and Kathleen Fisher\\
\affaddr{AT\&T Labs Research} \\
\email{mff@research.att.com\\kfisher@research.att.com}}

\eat{
\additionalauthors{Robert Gruber (Google, 
  {\texttt{gruber@google.com}}), while at AT\&T Labs and Xuan Zheng (Univ. of Michigan, 
  {\texttt{xuanzh@eecs.umich.edu}}), supported by 
	 AT\&T Labs and NSF DMS 0354600.}}

\date{\today}

\begin{document}

\maketitle
\eat{
\begin{abstract}
Enormous amounts of data exist in ``well-behaved'' formats such as
relational tables and XML, which come equipped with extensive tool
support.  However, vast amounts of data also exist in non-standard or
\textit{ad hoc} data formats, which often lack standard or extensible
tools. This deficiency forces data analysts to implement
their own tools for parsing, querying, and analyzing their ad hoc
data.  The resulting tools typically interleave parsing, querying, and
analysis, obscuring the semantics of the data format and making it
nearly impossible for others to resuse the tools.

This proposal describes \pads{}, an end-to-end system for processing
ad hoc data sources.  The core of \pads{} is a declarative
language for describing ad hoc data sources and a data-description
compiler that produces customizable libraries for parsing the ad hoc
data.  A suite of tools built around this core include statistical
data-profiling tools, a query engine that permits viewing ad hoc
sources as XML and for querying them with XQuery, and an interactive
front-end that helps users produce \pads{} descriptions quickly.

Details about the \pads{} system are reported in technical
papers~\cite{fernandez+:padx,fisher+:pldi05,fisher+:popl06}.  A
shorter version of this proposal was accepted to the PLAN-X 2006
workshop~\cite{daly+:launchpads}.  An open-source implementation of
\pads{} is available for download~\cite{padsmanual}.
\end{abstract}}

\section{Introduction}
\label{sec:intro}

{\em Data description languages} are a class of domain specific
languages for specifying {\em ad hoc data formats}, from billing 
records to TCP packets to scientific data sets to server logs.  Examples 
of such languages include 
\bro~\cite{paxson:bro}, \datascript{}~\cite{gpce02}, \demeter~\cite{lieberherr+:class-dictionaries},
\packettypes{}~\cite{sigcomm00}, \padsc{}~\cite{fisher+:pads}, 
\padsml{}~\cite{mandelbaum+:padsml}  and
\xsugar~\cite{xsugar2005}, among others.  All of these languages
generate parsers from data descriptions.  In addition, and unlike
conventional parsing tools such as Lex and Yacc, many also automatically
generate auxiliary tools ranging from printers to \xml{} converters to
visitor libraries to visualization and editor tools.

In previous work, we developed the {\em Data Description Calculus}
(\ddcold{}), a calculus of simple, orthogonal type constructors,
designed to capture the core features of many existing type-based data
description languages~\cite{fisher+:next700ddl,fisher+:ddcjournal}.
This calculus had a multi-part denotational semantics that interpreted
the type constructors as (1) parsers the transform external bit
strings into internal data representations and {\em parse descriptors}
(representations of parser errors), (2) types for the data
representations and parse descriptors, and (3) types for the parsers
as a whole.  We proved that this multi-part semantics was coherent in
the sense that the generated parsers always have the expected types
and generate representations that satisfy an important {\em
canonical forms} lemma.

The \ddcold{} has been very useful already, helping us debug and
improve several aspects of \padsc{}~\cite{fisher+:pads}, and serving
as a guide for the design of \padsml{}~\cite{mandelbaum+:padsml}.
However, this initial work on the \ddcold{} told only a fraction of the
semantic story concerning data description languages.  As mentioned
above, many of these languages not only provide parsers, but
also other tools.  Amongst the most common auxiliary tools
are printers, as reliable communication between programs, either through
the file system or over the Web, depends upon both input (parsing) 
and output (printing).

In this work, we begin to address the limitations of
\ddcold{} by specifying a printing semantics for the
various features of the calculus.  We also
prove a collection of theorems for the new semantics that serve as
duals to our theorems concerning parsing.  This new printing semantics
has many of the same practical benefits as our older parsing 
semantics: We can
use it as a check against the correctness of our printer
implementations and as a guide for the
implementation of future data description languages.  


% First, we extend \ddcold{} with
% abstractions over types, which provides a basis for specifying the
% semantics of \padsml{}. In the process, we also improve upon the
% \ddcold{} theory by making a couple of subtle changes. For example, we
% are able to eliminate the complicated ``contractiveness'' constraint
% from our earlier work. Second, .

% The main practical benefit of the calculus has been as a guide for our
% implementation. Before working through the formal semantics, we
% struggled to disentangle the invariants related to polymorphism. After
% we had defined the calculus, we were able to implement type
% abstractions as \ocaml{} functors in approximately a week.  Our new
% printing semantics was also very important for helping us define and
% check the correctness of our printer implementation.  We hope the
% calculus will serve as a guide for implementations of \pads{} in
% other host languages.  

% In summary, this work makes the following key contributions:
% \begin{itemize}
% \item We simultaneously specify both a parsing and a printing semantics
%   for the \ddc{}, a calculus of polymorphic, dependent types.
% \item We prove that \ddc{} parsers and printers are type safe
%   and well-behaved as defined by a canonical forms theorem.
% \end{itemize}

In this extended abstract, we give an brief overview of the calculus,
it's dual semantics and their properties.  A companion technical
report contains a complete formal
specification~\cite{fisher+:popl-sub-long}.  In comparison to our
previous work on the \ddcold{} at POPL 06~\cite{mandelbaum+:padsml},
the calculus we present here has been streamlined in several subtle,
but useful ways.  It has also been improved through the addition of
polymorphic types.  We call this new polymorphic variant
\ddc{}.  These improvements and extensions, together with
proofs, appear in Mandelbaum's thesis~\cite{mandelbaum:thesis} and in
a recently submitted journal article~\cite{fisher+:ddcjournal}.
This abstract reviews the \ddc{} and extends all the previous 
work with a printing semantics and appropriate theorems.
To be more specific,
sections~\ref{sec:ddc-syntax} through \ref{sec:ddc-sem} present the
extended \ddc{} calculus, focusing on the semantics of polymorphic
types for parsing and the key elements of the printing semantics.
Then, \secref{sec:meta-theory} shows that both parsers and
printers in the \ddc{} are type correct and furthermore that parsers
produce pairs of parsed data and parse descriptors in {\em canonical
  form}, and that printers, given data in canonical form, print
successfully. We briefly discuss related work in \secref{sec:related}, and
conclude in \secref{sec:conc}.

%%% Local Variables: 
%%% mode: latex
%%% TeX-master: "paper"
%%% End: 

\subsection{The Pads Language, Compiler and Automated Tool Generation}

In preparation for this proposal, we have developed a prototype
domain-specific language called
\pads{}~\cite{fisher+:pads,fisher+:popl06,mandelbaum+:pads-ml}, 
which is capable of describing
the formats of individual binary files and application log files.
\pads{} specifications are a form of extended type declaration that
simultaneously describe (1) the syntax of a data format, (2) additional
semantic properties of the data (such as value ranges or other constraints)
and (3) an internal, parsed representation of the data.  
%Figure~\ref{figure:clf} presents a small fragment of a \pads{} 
%description for a web log to give the reader what our domain specific
%language is like.

% \begin{figure}
% \begin{small}
% \begin{code}
% \kw{Punion} client\_t \{
%   Pip       ip;      /- 135.207.23.32
%   Phostname host;    /- www.research.att.com
% \};
% \mbox{}
% \kw{Punion} auth\_id\_t \{
%   Pchar unauthorized : unauthorized == '-';
%   Pstring(:' ':) id;
% \};
% \mbox{}
% \kw{Pstruct} version\_t \{
%   "HTTP/";
%   Puint8 major; '.';
%   Puint8 minor;
% \};
% \mbox{}
% \kw{Penum} method\_t \{
%     GET,    PUT,  POST,  HEAD,
%     DELETE, LINK, UNLINK
% \};
% \mbox{}
% bool chkVersion(version\_t v, method\_t m) \{
%   \kw{if} ((v.major == 1) && (v.minor == 1)) \kw{return} true;
%   \kw{if} ((m == LINK) || (m == UNLINK)) \kw{return} false;
%   \kw{return} true;
% \};
% \mbox{}
% \end{code}
% \end{small}
% \caption{Fragment of \pads{} description for Web Log data.}
% \label{figure:clf}
% \end{figure}

These specifications have a well-defined semantics and can play a
valuable role as documentation of the data produced, managed and
accumulated by monitoring systems.  More importantly, though, these
specifications can serve as inputs to a compiler that automatically
generates high-performance and reliable modules that perform key data
processing tasks, such as parsing, printing, error detection, data
traversal, format translation and statistical profiling.  Such
generated libraries can subsequently be linked to other components of
the monitor system, including the traffic sniffer and CoMon
visualization and query engine.  The end result we envision is a 
highly customized, automatically generated and evolvable
suite of high-performance network monitoring
tools.  However, in order to realize this vision, we must engage
in a number of important research subtasks. 

\paragraph*{Algorithms for Efficient Context-free and Non-context-free Data Processing.}
In order to make \pads{} a truly {\em universal language} for describing the data used by
networked applications, we must combine the expressive power of conventional 
{\em context-free} languages with various {\em non-context-free} features.  
For instance, most binary formats used by networked applications
contain length fields (fields, which when parsed,
describe the size or length of other fields in the data), a non-context-free feature,
or computed constraints (such as integer or floating point relations) another non-context-free
feature.  Currently, despite a wealth of research in automatic parser generation over the years,
handling such a combination of features efficiently is an important open problem of tremendous
theoretical and practical importance.  We will design new grammatical specifications
and automaton-based algorithms to solve this problem.

\paragraph*{Design of Archive Specification Language.}  
Our prototype \pads{} language can describe the syntax and semantic properties
of individual files.  However, diagnosing problems or simply monitoring
the health of applications in networked systems
will usually involve navigating and analyzing {\em sets of files},
not just individual logs.  After all, even individual
applications can generate complex sets of log files. 
As an example, consider the Coral content
distribution system~\cite{coral}, a typical distributed application.
Coral is currently running on the PlanetLab system~\cite{planetlab}, a
testbed with hundreds of machines distributed world-wide.
Coral generates a complex set of log files that
reside in a multi-level archive.  At the top-level, a series of subdirectories
contain information pertinent to each PlanetLab machine.  At the next level down,
another series of directories contains information for each time slice.  In the
time slice directories themselves sit four different log files, each containing different 
kinds of basic Coral information.

We propose to design a specification language for {\em entire archives} of 
system and application data, such as those used by Coral and other similar
applications.  This specification language will allow us to describe the structure and
properties expected of multi-level file systems, including the file hierarchy, the
naming conventions of directories (and the meta-data contained within those names), data 
ownership, permissions and formats of the files.

\paragraph*{Tool Generation for Archive Specifications.}  In addition to our archive
specification language design, we will develop a compiler capable of automatically
generating archive processing libraries, interfaces and tools from these specifications.  
These tools will 
include tools for ingesting, traversing and processing data residing in an archive
as well as tools to support archive querying, forensic analysis and visualization through CoMon.

\noindent
\textbf{Additional Authors.}
Robert Gruber ({\small{\texttt{gruber@google.com}}}), Google,
while at AT\&T Labs.
Xuan Zheng ({\small{\texttt{xuanzh@eecs.umich.edu}}}), Univ. of Michigan, 
supported by AT\&T Labs and NSF DMS 0354600.

\bibliographystyle{abbrv}
\bibliography{../pads,../galax}

\end{document}

