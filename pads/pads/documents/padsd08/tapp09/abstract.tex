The term {\em ad hoc data} refers to the billions of bytes of
non-standard and continuously evolving data spread across all computer
systems.  Such data includes server logs, distributed system
performance and debugging data, telephone call records, financial data
and online repositories of scientific data.
% including genetic information, cosmology data and
%global weather systems.

This paper presents \padsd{}, a system that generates monitoring,
analysis and transformation tools for distributed ad hoc data from
declarative specifications.  The generated tools include an archiver,
a database loading system, a statistical analyzer, an alert system, 
an RSS feed generator, and
debugging tools.  In addition, the system generates libraries for
application developers, including modules for parsing, printing, error
management, data traversal and transformation which developers can use
to create their own application-specific tools.  Advanced users can
build new generic tools applicable to any collection of data sources.

The \padsd{} data description language allows data analysts to specify
{\em where} their ad hoc data is located, {\em how} to obtain it, {\em
when} to get it (or give up trying), and {\em what}
preprocessing the system should do when it arrives.  As its name
suggests, \padsd{} is layered on top of the \pads{} sublanguage,
developed in previous research efforts, for specifying the {\em
format} of the data sources.  We illustrate the expressiveness of
\padsd{} by giving descriptions for several different distributed
systems including CoMon, the monitoring system for PlanetLab, and a
monitoring system for a web hosting service provided by AT\&T.  We
define a formal semantics for the language, describe our
implementation, and evaluate its performance.  We show our system is
capable of scaling to distributed systems the size of CoMon,
the current monitor for Planetlab's 800+ nodes.

% CoMon 
% the Safari web
% cache to SDSS star charts {\em [dpw: correct me ... cites]} to
% PlanetLab's CoMon network monitoring system to the log files for the
% Coral content distribution network.  We also provide a denotational
% semantics for the language, which is helpful in understanding the
% language's many features.

% Descriptions written in \padsd{} are compiled into O'Caml libraries
% with clean interfaces that support both polymorphic and monomorphic
% programming.  We illustrate how to use the polymorphic interfaces to
% program description-independent applications, including a selector
% tool that can extract key elements of a data source as well as an
% accumulator tool that collects simple statistical information about
% the data.  The monomorphic interfaces allow users to develop
% ``one-of'' data source-specific applications.

