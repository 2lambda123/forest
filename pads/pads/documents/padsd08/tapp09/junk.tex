
The system can generate all of these tools from \padsd{} descriptions
and declarative tool configuration specifications.  Thus for common
tasks, users can manage distributed data sources simply by writing
high-level declarative specifications.  It is quick and it is easy. There
are relatively few concepts to learn, no complex interfaces and no
tricky boilerplate a programmer needs to master to initialize 
the system or thread together tool libraries.  Because it's so
easy to use, we refer to the act of writing simple specifications
and using pre-defined tools as the {\em quick-and-dirty} \padsd{} programming
mode.  However, to avoid sacrificing flexibility and to support extensibility,
\padsd{} supports two other {\em modes of use} in addition to the 
quick-and-dirty.

The second mode is
for the {\em single-minded implementer}, who needs to build a new
application for a {\em specific} collection of distributed data
sources.  Such users need more than the built-in set of tools, 
and consequently, the system provides support for
creating new tools by automatically generating libraries for fetching
data, for parsing and printing, for performing type-safe data
traversal, and for stream processing using classic functional
programming paradigms such as \cd{map}, \cd{fold} and \cd{iterate}.
These generated libraries make it straightforward to create custom tools
specific to particular data sources.  However, there is a steeper learning
curve in this mode than in the quick-and-dirty because a variety of
interfaces must be learned.  The average functional programmer may find
these interfaces relatively intuitive, but the computational scientist
who is not interested in functional programming may prefer to stick with
the quick-and-dirty. 

The third mode is for the {\em generic programmer}.  Generic
programmers may observe that they (or their colleagues or fellow domain
experts) need to perform some task over and over again on different
data sets.  Rather than writing a program specific to a particular
data set, they use a 
separate set of interfaces supplied by the \padsd{} system to write a
single generic program to complete the task.  For example, the RRD database
loader is generic, because it is possible to load data from any
specified source into the RRD tool without additional ``programming.''
We used the generic support to build the various tools described earlier.
As more and more generic tools are built, the life of the quick-and-dirty
hacker becomes easier and easier.

