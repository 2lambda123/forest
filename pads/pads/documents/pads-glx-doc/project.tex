\documentclass{article}

%% \usepackage{latin-abbrevs,math-cmds,math-envs,code,latexsym,amsmath,amstext,amssymb,verbatim}
\usepackage{xspace}
\newcommand{\cut}[1]{}
\newcommand{\reminder}[1]{{\it #1 }}
\newcommand{\edcom}[1]{\textbf{{#1}}}
\newcommand{\poplversion}[1]{#1}
\newcommand{\trversion}[1]{}

\newcommand{\appref}[1]{Appendix~\ref{#1}}
\newcommand{\secref}[1]{Section~\ref{#1}}
\newcommand{\tblref}[1]{Table~\ref{#1}}
\newcommand{\figref}[1]{Figure~\ref{#1}}
\newcommand{\listingref}[1]{Listing~\ref{#1}}
%\newcommand{\pref}[1]{{page~\pageref{#1}}}

\newcommand{\eg}{{\em e.g.}}
\newcommand{\cf}{{\em cf.}}
\newcommand{\ie}{{\em i.e.}}
\newcommand{\etal}{{\em et al}}
\newcommand{\etc}{{\em etc.\/}}
\newcommand{\naive}{na\"{\i}ve}
\newcommand{\role}{r\^{o}le}
\newcommand{\forte}{{fort\'{e}\/}}
\newcommand{\appr}{\~{}}

\newcommand{\bftt}[1]{{\ttfamily\bfseries{}#1}}
\newcommand{\kw}[1]{\bftt{#1}}
\newcommand{\pads}{\textsc{pads}}
\newcommand{\padsc}{\textsc{pads/c}}
\newcommand{\padx}{\textsc{padx}}
\newcommand{\ipads}{\textsc{ipads}}
\newcommand{\ir}{\textsc{IR}}
\newcommand{\padsl}{\textsc{padsl}}
\newcommand{\padsml}{\textsc{pads/ml}}
%\newcommand{\padsd}{\textsc{pads/d}}
\newcommand{\learnpads}{{\textsc{learnpads}}}
\newcommand{\padsd}{\textsc{Gloves}}
\newcommand{\blt}{\textsc{blt}}
\newcommand{\ddc}{\textsc{ddc}}
\newcommand{\ddl}{\textsc{ddl}}
\newcommand{\C}{\textsc{C}}
\newcommand{\perl}{\textsc{Perl}}
\newcommand{\ml}{\textsc{ml}}
\newcommand{\smlnj}{\textsc{sml/nj}}
\newcommand{\ocaml}{\textsc{OCaml}\xspace}
\newcommand{\haskell}{\textsc{haskell}\xspace}
\newcommand{\ocamlbig}{\textsc{OCAML}\xspace}
\newcommand{\java}{\textsc{java}}
\newcommand{\xml}{\textsc{xml}}
\newcommand{\html}{\textsc{html}}
\newcommand{\xpath}{\textsc{xpath}}
\newcommand{\xquery}{\textsc{xquery}}
\newcommand{\datascript}{\textsc{datascript}}
\newcommand{\packettypes}{\textsc{packettypes}}
\newcommand{\erlang}{\textsc{Erlang}}
\newcommand{\camlp}{\cd{Camlp4}}
\newcommand{\ocamlnet}{\cd{Ocamlnet} \cd{2}}

\newcommand{\totalcost}[2]{\textsc{Cost}(#1,#2)}
\newcommand{\costdescription}[1]{\textsc{CT}(#1)}
\newcommand{\normcostdescription}{\textsc{NCT}}
\newcommand{\costdata}[2]{\textsc{CD}(#2 \; | \; #1)}
\newcommand{\acostdata}[2]{\textsc{ACD}(#2 \; | \; #1)}
\newcommand{\adc}[2]{\textsc{CD'}(#2 \; | \; #1)}
\newcommand{\cardt}{\textsc{Card}}
\newcommand{\costvar}[1]{\textsc{CV}(#1)}
\newcommand{\costchar}[1]{\textsc{CA}(#1)}
\newcommand{\coststring}[1]{\textsc{CS}(#1)}
\newcommand{\costint}[1]{\textsc{CI}(#1)}
\newcommand{\costparam}[1]{\textsc{CP}(#1)}
\newcommand{\costconst}[1]{\textsc{CC}(#1)}

\newcommand{\dibbler}{Sirius}
\newcommand{\ningaui}{Altair}
\newcommand{\darkstar}{Regulus}

\newcommand{\vizGems}{Arrakis}

\newcommand{\comon}{CoMon\xspace}
\newcommand{\planetlab}{PlanetLab\xspace}
\newcommand{\monall}{Monall\xspace}
%% \newcommand{}{}


%% \newcommand{\IParray}[4]{{\tt Parray} \; #1 \; \[#2, #3, #4\]}

\newcommand{\figHeight}[4]{\begin{figure}[tb]
	\centerline{
	            \epsfig{file=#1,height=#4}}
	\caption{#2}
	\label{#3}
	\end{figure}}

\newcommand{\myalt}{\ensuremath{\; | \;}}
\newcommand{\normal}[1]{\ensuremath{\bar{#1}}}
\newcommand{\relativee}[2]{\ensuremath{{\cal R}(#1 \; || \; #2)}}
\newcommand{\srelativee}[2]{\ensuremath{{\cal S}(#1 \; || \; #2)}}
\newcommand{\addh}[2]{\ensuremath{#1 \oplus #2}}

\newcommand{\irstruct}[1]{{\tt struct}\{#1\}}
\newcommand{\irunion}[1]{{\tt union}\{#1\}}
\newcommand{\irenum}[1]{{\tt enum}\{#1\}}
\newcommand{\irarray}[1]{{\tt array}\{#1\}}
\newcommand{\irarrayFW}[2]{{\tt arrayFW}\{#1\}[#2]}
\newcommand{\irswitch}[2]{{\tt switch}(#1)\{#2\}}
\newcommand{\iroption}[1]{{\tt option}\{#1\}}
\newcommand{\setof}[1]{\lsem #1 \rsem}
\newcommand{\goto}{\Rightarrow}
\newcommand{\Pvoid}{{\tt Pvoid}}
\newcommand{\Pempty}{{\tt Pempty}}
\newcommand{\sskip}{\hspace*{5mm}}
\newcommand{\shrink}{\vspace*{-4mm}}

% Semantics
\newcommand{\setalt}{{\; | \;}}
\newcommand{\denote}[1]{\lsem #1 \rsem}
\newcommand{\lsem}{{[\![}}
\newcommand{\rsem}{{]\!]}}
\newcommand{\turn}{\vdash}
\newcommand{\meta}{m}
\newcommand{\nested}{n}
\newcommand{\mytime}[1]{#1.t}
\newcommand{\myds}[1]{#1.ds}
\newcommand{\myval}[1]{#1.nest}
\newcommand{\generatedloc}{\ensuremath{\mathtt{nowhere}}}
\newcommand{\environment}{E}
\newcommand{\universe}{U}
\newcommand{\selectOne}{\ensuremath{\mathsf{earliest}}}
% core feed semantics
\newcommand{\csemantics}[3]{{\cal C}\lsem #1 \rsem_{{#2} \, {#3}}}
% feed semantics
\newcommand{\semantics}[3]{{\cal F}\lsem #1 \rsem_{{#2} \, {#3}}}
% expression semantics
\newcommand{\esemantics}[2]{{\cal E}\lsem #1 \rsem_{{#2}}}
%\newcommand{\esemantics}[2]{#2(#1)}

% Host language types
\newcommand{\ty}{\ensuremath{\tau}}
\newcommand{\basety}{\ensuremath{b}}
\newcommand{\arrow}{\rightarrow}
\newcommand{\optionty}[1]{\ensuremath{#1 \; \mathsf{option}}}
\newcommand{\listty}[1]{\ensuremath{#1 \; \mathsf{list}}}
\newcommand{\setty}[1]{\ensuremath{#1 \; \mathsf{set}}}
\newcommand{\feedty}[1]{\ensuremath{#1 \; \mathsf{feed}}}
\newcommand{\corety}[1]{\ensuremath{#1 \; \mathsf{core}}}
\newcommand{\schedulety}{\ensuremath{\mathsf{sched}}}
\newcommand{\timety}{\ensuremath{\mathsf{time}}}
\newcommand{\locty}{\ensuremath{\mathsf{loc}}}
\newcommand{\boolty}{\ensuremath{\mathsf{bool}}}
\newcommand{\unitty}{\ensuremath{\mathsf{unit}}}
\newcommand{\stringty}{\ensuremath{\mathsf{string}}}
\newcommand{\metatype}[1]{\ensuremath{\mathsf{meta}(#1)}}
\newcommand{\nestedtype}[1]{\ensuremath{\mathsf{nest}(#1)}}
\newcommand{\dsty}{\ensuremath{\mathsf{ds}}}

\newcommand{\dom}{\ensuremath{\mathsf{dom}}}
\newcommand{\ueq}[3]{\ensuremath{#1 =_{#2} #3}}
\newcommand{\fsubset}[3]{\ensuremath{#1 \subseteq_{#2} #3}}
\newcommand{\feq}[3]{\ensuremath{#1 =_{#2} #3}}

% Expressions
\newcommand{\expression}{e}
\newcommand{\constant}{c}
\newcommand{\ds}{\ensuremath{ds}}
\newcommand{\boolf}{\ensuremath{\mathtt{false}}}
\newcommand{\boolt}{\ensuremath{\mathtt{true}}}
\newcommand{\loc}{\ensuremath{\ell}}
\newcommand{\feed}{\ensuremath{F}}
\newcommand{\corefeed}{\ensuremath{C}}
\newcommand{\generalvar}{\ensuremath{x}}
\newcommand{\feedvar}{\ensuremath{x}}
\newcommand{\itemvar}{\ensuremath{x}}
\newcommand{\data}{\ensuremath{v}}
\newcommand{\atime}{\ensuremath{t}}
\newcommand{\astring}{\ensuremath{w}}
\newcommand{\unit}{\ensuremath{()}}
\newcommand{\schedule}{\ensuremath{s}}
\newcommand{\parser}{\ensuremath{p}}
\newcommand{\none}{\ensuremath{\mathtt{None}}}
\newcommand{\some}[1]{\ensuremath{\mathtt{Some}\; #1}}
\newcommand{\inl}[1]{\ensuremath{\mathtt{inl}\; #1}}
\newcommand{\inr}[1]{\ensuremath{\mathtt{inr}\; #1}}
\newcommand{\casedata}[2]{{\tt switch}(#1)\{#2\}}
%\newcommand{\nillist}{\ensuremath{\mathtt{nil}}}
\newcommand{\nillist}{\ensuremath{[\,]}}
%\newcommand{\conslist}[2]{\ensuremath{\mathtt{cons} (#1,#2)}}
\newcommand{\conslist}[2]{\ensuremath{[#1,\ldots,#2]}}
\newcommand{\nilstream}{\ensuremath{\mathtt{done}}}
\newcommand{\consstream}[2]{\ensuremath{\mathtt{next} (#1,#2)}}


% Feeds
\newcommand{\comprehensionfeed}[3]{\ensuremath{\mathtt{\{|} #1 \; \mathtt{|}\; #2 \leftarrow #3 \mathtt{|\}}}}
\newcommand{\computed}[3]{\ensuremath{\mathtt{[} #1 \; \mathtt{|}\; #2 \in #3 \mathtt{]}}}
\newcommand{\letfeed}[3]{\ensuremath{\mathtt{let}\; #1 \; \mathtt{=}\; #2 \; \mathtt{in} \; #3}}
\newcommand{\allfeed}[5]{
  \ensuremath{
    \mathtt{all \{ format=} #1; 
    \mathtt{src=} #2;
    \mathtt{sched=} #3;
    \mathtt{pp=} #4;
    \mathtt{win=} #5;
  \mathtt{\}}}}
\newcommand{\existsfeed}[5]{
  \ensuremath{
    \mathtt{any \{ format=} #1; 
    \mathtt{src=} #2;
    \mathtt{sched=} #3;
    \mathtt{pp=} #4;
    \mathtt{win=} #5;
  \mathtt{\}}}}
\newcommand{\filterfeed}[2]{
  \ensuremath{
    \mathtt{filter} \; #1 \; \mathtt{with}\; #2}}
\newcommand{\remapfeed}[2]{
  \ensuremath{
    \mathtt{redirect} \; #1 \; \mathtt{with}\; #2}}
\newcommand{\ppfeed}[2]{
  \ensuremath{
    \mathtt{pp} \; #1 \; \mathtt{with}\; #2}}
\newcommand{\foreachupdate}[3]{
  \ensuremath{
    \mathtt{foreach{*}{*}}\; #1 \;
    \mathtt{in}\; #2 \;
    \mathtt{update}\; #3}}
\newcommand{\foreachcreate}[3]{
  \ensuremath{
    \mathtt{foreach*}\; #1 \;
    \mathtt{in}\; #2 \;
    \mathtt{create}\; #3}}
\newcommand{\remap}[2]{\ensuremath{\mathtt{redirect}\; #1 \; \mathtt{with} \; #2}}
\newcommand{\stutterfeed}[2]{\ensuremath{\mathtt{stutter}\; #1 \; \mathtt{on} \; #2}}
\newcommand{\refeed}[2]{\ensuremath{\mathtt{reschedule}\; #1 \; \mathtt{to} \; #2}}
\newcommand{\emptyfeed}{\ensuremath{\emptyset}}
\newcommand{\onefeed}[2]{\ensuremath{\mathtt{One}}(#1,#2)}
\newcommand{\sfeed}[1]{\ensuremath{\mathtt{SchedF}}(#1)}
\newcommand{\lfeed}[1]{\ensuremath{\mathtt{ListF}}(#1)}
\newcommand{\unionfeed}{\ensuremath{\cup}}
\newcommand{\sumfeed}{\ensuremath{+}}
\newcommand{\spairfeed}{\; \ensuremath{\mathtt{\&} \; }}
\newcommand{\allpairfeed}{\; \ensuremath{{*}{*}} \; }

\newcommand{\Time}{\ensuremath{\mathtt{Time}}}
\newcommand{\Set}{\ensuremath{\mathtt{Set}}}

% this is used for the translations equal
\newcommand{\transeq}{\stackrel{def}{=} }
\newcommand{\ai}{{\tt wl}}



% BNF
%\newcommand{\bnfalt}{\ |\ }


\setlength{\topmargin}{-0.25in}
\setlength{\oddsidemargin}{0.25in}
\setlength{\textheight}{8.5in}
\setlength{\textwidth}{6in}

\title{PADS/Galax: Enabling XQuery with Ad-hoc Data Sources}
\author{
  Yitzhak Mandelbaum \\ 
  AT\&T Research
}
\date{August 2004}

\begin{document}
\maketitle
\begin{abstract}
  XQuery is a language for querying data stored in an XML format.
  While there are many implementations of XQuery for traditional
  database systems, there are none that support ad-hoc, or arbitrary,
  data sources. PADS is a language for describing ad-hoc data sources,
  automatically generating a C library to process the data based on
  the description. However, the actual processing programs must be
  written in C.
  
  The PADS/Galax project is designed to bring to users the benefits of
  both languages, by integrating the PADS implementation with Galax -
  an open-source XQuery implementation. With the resulting system,
  users can combine declarative, functional data queries written in
  XQuery with a declarative data description to achieve results
  usually achieved with a difficult-to-maintain, imperative program
  written in C.  Furthermore, PADS/Galax supports data sources ranging
  in size from bytes to Gigabytes.
  
  The design of PADS/Galax involved addressing three major problems:1)
  how to represent PADS-described data as XML, 2) how to bring data
  elements into memory, or discard them, as needed by Galax, and 3)
  how to provide these features with little or no user involvement. In
  this talk, I will describe my work here this summer, which focused
  on the latter two problems, while touching on the first. I will
  explain the goals of my work and describe the extent to which we
  have achieved those goals, including descriptions of our solutions
  to the problems described above.

  We describe the continued integration of the \pads and \galax
  systems. Previous work focused on exporting non-XML data sources,
  described by a \pads specification, as XML to the \galax XQuery
  engine. We have improved upon this work with a new interface between
  the \galax \pads driver and the \pads backend, and extended the work
  with support for very large data sources (those that cannot fit in
  main memory). We explain the goals of our work and describe the
  extent to which we have achieved those goals. We conclude with
  observations on the project and a list of potential future tasks.
\end{abstract}

\section{Introduction}
Introduction to PADS and Galax. ...  

Explanation of the idea of connecting PADS and Galax. description of
benefits.

However, we cannot always assume that the PADS data source will fit
into memory in its entirety, whether due to restrictions on available
memory, the size of the data source, or both. Furthermore, we cannot
rely upon the virtual memory system to manage our memory for us, as
the data source might in fact be larger than even the size of the swap
space allotted to virtual memory.  Rather, we need a mechanism for
examining data in parts, never requiring that that the whole data
source be present in memory at one time.

% Roadmap for rest of paper. Don't just list contents of
% sections. Description should tell reader what they should expect out
% of the rest of the paper, and why they should keep reading.

\begin{itemize}
\item Smart arrays
  \begin{itemize}
  \item Arrays vs. other datatypes. Choice of arrays.
  \item Expected use.
  \item Details - tmap, mem managemenet, read management - see text below.
  \item References into smart array. Validation design.
  \end{itemize}
\item PADS/Galax description.
  \begin{itemize}
  \item interface
  \item node architecture
  \item relationship to smart arrays.
  \end{itemize}
\item Polymorphism
  \begin{itemize}
  \item void * polymorphism vs. compiler
  \item overview of division (what's in the libraries vs. what's in
  the compiler)
  \end{itemize}
\item Technical Details
\item Applications ??
\item Future Work
\item Appendix: File Directory Structure
\end{itemize}

\section{Smart Arrays}
\label{sec:smart-arrays}

The key motivating examples for our work are data sources
which consist of a long sequence of records, possibly preceded by a
header. Such data sources are most logically mapped into an array.
There are surely examples of very large data sources containing a
complex structure, and, therefore, most appropriately mapped to
structures, or even other datatypes. However, we have seen few, if
any, such examples, and have therefore chosen not to focus on them.

In a standard PADS application, the programmer would handle such data
sources by manually looping over the data source, one record at a
time. For PADS/Galax, however, we don't have this option as the user
interacts with Galax, via XQuery, and should not be burdened with
designing and implementing PADS applications. Furthermore, this
approach does not address the need to reaccess previously viewed
elements of the array that are longer available in memory. Similarly,
even for PADS itself, such an approach makes it impossible for the
programmer to use any of the advanced array features, such as complex
termination conditions and error checking and handling.

Therefore, we designed an infrastructure that can automatically ensure
that array elements are available when they are needed - no matter the
order in which they are requested - and supports (nearly) all of the
advanced features supported by standard arrays {\em Once we've
  implemented the ADT for use predicates, we can remove the nearly
  --ed.}.
Arrays with the element management infrastructure enabled are termed
{\em smart arrays}.


\subsection{Core Operations}
\label{subsec:core-features}

We can conceptually break down the implementation of smart arrays into
three components.: element management, memory management
and read management, with the first depending on the second two.

\paragraph{Element Management}

Each smart node contains one \emd, which stores meta-data for the elements of
the array. The most important meta-data are the offset of the element
in the data source and pointers to the in-memory representation of the
data and its parse-descriptor, called the \rep and \pd, respectively.
When a particular element is requested from the smart node, the smart
node uses the \emd to find the element and then performs one of two
actions. If the element is available (that is, in memory), then the
smart node immediately returns the element's representation and
parse-descriptor. If the element is unavailable, then the smart node
finds the element in the data source based on the offset in the
meta-data, and rereads/reparses it from the stream.

However, element management does not operate alone. It depends on
support for reading and rereading array elements as well as support
for allocating space for element data and effectively choosing
which node to evict when no space is available for allocation.

\paragraph{Read Management}

Reading and rereading data from the stream can be complicated due to
the nature of \pads arrays. Therefore, we need support for reading any
arbitrary element in the array data stream from the \pads compiler and
library.

The generated code for reading a pads array can be potentially quite
complicated. The complexity arises do to a multiplicity of possible
error conditions that can arise combined with a wide variety of
methods for determining the termination of an array. Given this
complexity, reading or rereading elements is rarely a simple matter of
calling the element type's read function. For this reason, we have
(re)designed the array reading functionality so that at its core is a
read\_one function designed to read at most one element (possibly
none). in addition, we provide a reread function which will recreate a
given element and its parse-descriptor exactly as they were originally
created when first read from the stream.

With incremental reading, smart reading becomes almost
straightforward. All reading is done through the read and reread
functions. Before an element is read for the first time, the offset in
the io stream is recorded in that element's meta-data.{\em when do we
  introduce the tmap? It definitely belongs in this section.}  Then,
upon any future request for the element, the smart array rewinds the
stream to the recorded offset for that element. The key difficulties
left in read management are distinguishing between first-time reads
and rereads and ensuring that we never lose track of the head of the
stream. These problems are addressed by saving both the furthest known
position in the stream as well as the index of the furthest read
element.

\paragraph{Memory Management}

When the element manager needs space to read an element from the data
source, it requests the space from the memory manager. The memory
manager will first check if there is memory available for
allocation. If sufficient memory is available, then the manager will
allocate space for the element and return it to the element
manager. If sufficient memory is unavailable, then the manager will
choose an element to eject from memory, update its meta-data
appropriately, and return the memory to the element manager for reuse.

An important part of the memory manager is the policy it uses to
decided which element to replace. While the current implementation
uses a simple FIFO policy, the system is designed so that the policy
can be changed without significantly affecting the rest of the system.

\subsection{Element References}
\label{subsec:elt-refs}

Due to the ephemeral nature of elements, references to their
representations and parse-descriptors (and their children) present a
serious difficulty. On the one hand, we'd like to avoid dangling
references to nodes that are no longer in memory. On the other hand,
we cannot afford to leave all referenced elements in
memory. Therefore, the smart array cannot simply provide clients with
direct references to elements. Instead, some form of weak reference
must be used, whereby the validity of the reference is ensured before
every access.

The simplest approach would be to provide the client with the index of
the requested element as the weak reference. Then, validition of the
reference would reduce to rerequesting the element from the smart node
based on its index. While this approach works well if the client only
stores references to the element itself, it becomes terribly
inefficient if the client wishes to maintain references to
substructures within the element. To support such internal references,
the weak reference would additionally need to include some form of
path indicating how to reach the substructure starting from the
element pointer itself. More importantly, this path would need to be
traversed with every validation of the reference, a potentially
expensive operation.

To address this problem, that is, to support efficient weak references
to element substructures, we introduce the concept of {\em data
  generations}. Each time an element is read into memory, its data
will (potentially) reside in a different place in memory. Therefore,
despite the equivalence of the data itself, we can consider each
instantiation of the data to be different version, or {\em
  generation}, of the element due to its (potentially) different
location. 

Now, for each element, we can explicitly record in its meta-data the
generation of its data. Then, we can use this generation as a simple
guard for direct references to the data.  A weak reference can record
a direct reference to the data as well as the generation of its direct
reference. Then, as long as the generation recorded by the client is
the same as that listed for the element, the client knows that its
direct reference is valid. Notice that as data in a smart array in
inserted and ejected a whole-element at a time, references to element
substructures can be validated with the same generation counter as
references to the element itself.

The final aspect of weak references that we must address is what
happens when the reference is broken, that is, an element's data is
ejected from memory. As we have written, any direct references to the
object will be invalidated, based on the generation. How does the
client revalidate those references? More specifically, how does the
client get from references to the element representation and
parse-descriptor to references to the substructure's representation
and parse-descriptor? One solution, and the one we adopt, is to
encode, in some form, the path from the element to the substructure to
which the weak reference refers. Then, once an element has been
returned to memory, we use this path to traverse the data until we
find the new pointers to the substructure's representation and
parse-descriptor.

To summarize the structures described, the meta-data for an element
consists of an offset, a generation, a \mask reference, a \rep
reference, and a \pd reference.  The weak reference held by clients of
the array consists of \mask, \rep and \pd references, a generation, an
index of or pointer to the corresponding element's meta-data, and some
recording of the path from the element root to the substructure. Note
that masks are not brought in and out of memory, as all elements of a
smart array share the same mask. Therefore, the mask pointer is always
valid.

\section{\pglx}
\label{sec:pads-galax}

We have described the design of smart arrays on the conceptual
level. However, no {\pads}-only implementation smart arrays
exist. Instead, smart arrays have been implemented in the context of the \pglx node
interface, which is the interface between \galax and \pads. In this
section we describe this interface in detail and the of smart arrays,
or, more generally, {\em smart nodes}, within it.

\subsection{XML and the data model interface}
\label{subsec:xml-dm}

In order for \galax to query a \pads data source, that source must be
presented to \galax as XML. The most obvious way to do so would be to
simply convert the data source to an XML document, and then run \galax
on this document. However, this conversion is rarely desireable.
First, \galax may only examine a small portion of the data source, in
which case converting the whole thing is wasteful.  Second, there
might not be sufficient space for the converted document. Therefore, a
more desireable alternative is to join \pads and \galax at the
application level, through an API that allows \pads to export its data
in datastructures that represent XML data.

A simple XML node should have a name, a kind and a (possibly empty)
list of children, which are themselves nodes. Some nodes should
additionally support requests for a string representation of their value
as well as requests for a typed representation of their value.  The
essential challenge, then, is to find an appropriate name and kind for
each \pads datatype and to appropriately map the datatypes
substructures (if any) to children nodes.

\subsection{The Node Architecture}
\label{subsec:node-arch}

In order to export \pads data as XML, we wrap data in {\em node}
datastructures containing all the information relevant to XML. Handles
to these nodes are then passed to \galax which can invoke very simple
queries on the node, such as ``what is your name? give me a list of
your children, etc.''  Within \galax, the interface with pads is
wrapped in a module that exports the \galax Data Model interface to
the \galax query engine.

While opaque to \galax, in \pads each node contains the XML name of
the data, its XML kind, and references to its mask, parse-descriptor
and representation.  The representations and parse-descriptors are
used to create children nodes in a type-specific manner.  For example,
for structures, the children are the parse-descriptor and the fields
of the structure.  For arrays, the children are the parse-descriptor,
the array length, and a list of nodes with name ``elt,'' corresponding
to the elements of the array. Other datatypes are analagous.

It is important to note that the XML representations of \pads data
integrate each datastructure's parse-descriptor with the data,
exporting it as a child of the datastructure itself. This design
allows queries to select (or ignore) nodes based on the contents of
their parse-descriptors in an easy and efficient manner.


\subsection{Smart Nodes}
\label{subsec:smart-nodes}

One of the key assumptions underlying the management of \pads nodes is
that the references to the node's parse-descriptor and representation
are always valid, and that the contents are complete. However, smart
arrays clearly violate the second of these assumptions, and
substructures of smart arrays violate the first. Therefore, these
cannot be mapped into the standard implementation for nodes. Instead,
we need to add support for nodes that manage their own memory and
parsing in the manner of smart arrays and for nodes that reference the
substructures smart arrays. We call such nodes {\em smart nodes} and
{\em snd (smart-node-descendant) nodes}, respectively.

The implemenation of smart nodes integrates the design of smart
arrays, discussed in Section~\ref{sec:smart-arrays}, with the design
of nodes, discussed above. A smart node appears to \galax as would any
other node. It has a name, a kind, and children, as well as other
relevant information. The only significant difference - with regard to
XML - between a smart
array and a normal one is that the ``parse-descriptor'' and ``length''
children come after the elements of the array, rather than before
them. Internally, however, the smart node extends the standard node
structure with the information used to manage the smart array, including
the \emd. Also, the smart node ensures that its element children are
all initialized as smart node descendants.

Similary, smart node descendants appear to XML as any other node.
Internally, though, they extend the standard node structure with the
information needed to maintain a weak reference to their
parse-descriptor's and representations.

%%%%%%%%%%%%%%%%%%%%%%%%%%%%%%%%%

% Purpose: give reader necessary technical background to understand
% later discussion of projects. This should include only as much detail
% as is absolutely necessary for later sections.

% intro blah, blah,...
% software can be broken into the following major portions:
% \begin{itemize}
% \item Pads 
%   \begin{itemize}
%   \item compiler - standard and pads/galax
%   \item libraries
%   \end{itemize}
% \item Pads-Galax libraries
% \item Galax Data Model
% \item pads\_pglx.c,etc.: glue code
% \item assorted driver programs, tests, data, etc.
% \end{itemize}

% Between these portions we have these interfaces:
% \begin{itemize}
% \item pads\_dm.mli: OCaml interface between galax and pads\_dm, based on the
%   DM (data model) interface.
% \item Pads\_c: IDL interface between pads\_dm.ml OCaml code and
%   pads\_pglx.c glue code.
% \item pglx.h: C interface between pads\_pglx.c glue code and Pads-Galax libraries.
% \end{itemize}

\section{Technical Details}
\begin{itemize}
\item Pads library
\item Pglx library
  \begin{itemize}
  \item node architecture
    \begin{itemize}
    \item interface 
      + Note: in kthChild functions, passing k > num-children is not an
      error. The correct semantics is to return NULL in such a
      situation, just as the children function used to return a
      null-terminated list.
    \item (Why is the parse-descriptor tree integrated into the node
      tree?  (Given the integration of the trees, we don't need array
      pds to be smart, because you can never get to the element pds
      via a given pd. A given pd node only contains the children
      directly relevant to itself.))
    \item virtual tables
    \item node structure
    \item node memory management
      \begin{itemize}
      \item reference counting
      \item free list
      \item interaction with ocaml
      \end{itemize}
    \end{itemize}
  \item extensions
    \begin{itemize}
    \item review design decision of one-size-fits-all.
    \item cached nodes
      \begin{itemize}
      \item use static constructors for each type
      \item use dynamic destructors for each type.
      \item initialize the cache in static constructor, which will know
        the cache size.
      \item Cached children can never be freed based on reference
        counting, because the parent/child\_cache pointers form a
        cycle. So, cached nodes won't use reference counting and 
        their free function will be a noop.
      \item Mention PDCI\_error\_cachedNode\_init
      \item duplicate kth-child functions to support caching.
      \end{itemize}
    \end{itemize}
  \item smart node data structures, including compiler-generated smart\_array\_info.
  \item read semantics for smart node
    \begin{itemize}
    \item forward, backward, etc.
    \item pd and length fields force entire array to be read.
    \end{itemize}
  \item node validation - where, when, how.  
    For PDCI\_sequenced\_pd and PDCI\_structured\_pd, the presence of
    child "loc" depends on field errCode >= 100. To know the value of
    errCode, we need to force the elt into memory (in
    sndNode\_kthChild). Is there a better approach?  Also for base type
    kthChild functions. The presence of the val node depends on there
    having been no serious errors.
  \item paths and path walking
  \item node ids
  \item MOVE TO: section of incremental reading.
    offsets - taken before seperator.
  \item error handling
  \item cstr's. see ToDo list.
  \item structure of pglx.c
  \end{itemize}
\item Compiler
  \begin{itemize}
  \item Use P\_PARTIAL flag to indicate that array is done.
  \end{itemize}
\item Macros
\item Tests
\end{itemize}

intro ....


\subsection{The PGLX Interface}
\label{sec:pglx-interface}

The \pglx interface defines the functions that are available to the
\pads \dm implementation for use on nodes. The relevant portions of
this interface are shown below.
\begin{verbatim}
nodeRep      PGLX_generic_parent      (nodeRep n)

const char*  PGLX_generic_name        (nodeRep n)
const char*  PGLX_generic_kind        (nodeRep n)
item         PGLX_generic_typed_value (nodeRep n)
const char*  PGLX_generic_string_value(nodeRep n)

nodeRep      PGLX_generic_kth_child         (nodeRep n, childIndex idx)
nodeRep      PGLX_generic_kth_child_named   (nodeRep n, childIndex idx, 
                                             const char *name)
\end{verbatim}

The {\tt kth-child} function returns the kth child of the node
starting from zero. The {\tt kth-child-named} function returns the node's kth
child with the specified name. This function is most relevant to
arrays, where all of the elements are given the same name ``elt.''
For both functions, {\tt NULL} is a valid result, indicating that the
requested child does not exist.

\subsection{The PGLX Library}
\label{sec:pglx-lib}

At the core of the \pglx library are the node data structures. These
structures contain all the information \galax needs to manipulate a
given portion of \pads data as a node in an XML document.

The node architecture is designed for limited extensibility in two
dimensions. One dimension is the type of the data represented. \pads
nodes can represent all possible generated \pads types as well as all
of the \pads base types. The second dimension is the type of the nodes
themselves. Different node types can display different behaviours, and
the architecture supports adding new node types to an extent.

In order to support the full range of \pads datatypes (dimension one),
the node data structure is designed using ``{\tt void *}
polymorphism.'' That is, anywhere that the node architecture is not
specific to a particular type, we use a {\tt void *} type. In this
way, we can write type-neutral code to manipulate data nodes.

However, the \pads functions generated by the compiler are type
specific. As the type-neutral and type-specific code must interact, we
need a convenient interface between them. 
%% One choice would be to
%% include tags in each node indicating the type and allow each
%% type-neutral function to switch on the tag when needing to perform a
%% type-specific action. However, this can potentially 
We have chosen to define a {\em virtual table} containing a set of
functions that are typically implemented in a type-specific manner.
For each type, the compiler generates a virtual table for nodes of
that type. We include a virtual table pointer in each node, and
initialize it to point to the correct virtual table when the node is
created. Below we show the types of the virtual table and the
functions it contains.
\begin{verbatim}
PDCI_node_t * (* PDCI_cachedNode_init_fn) (PDCI_node_t *node) 
PDCI_node_t * (* PDCI_kth_child_fn)       (PDCI_node_t *node, 
                                           PDCI_childIndex_t idx) 
PDCI_node_t * (* PDCI_kth_child_named_fn) (PDCI_node_t *node, 
                                           PDCI_childIndex_t idx, 
                                           const char *name) 
void          (* PDCI_free_fn)            (PDCI_node_t *node)
PDCI_id_t     (* PDCI_get_id_fn)          (PDCI_node_t *node)
item          (* PDCI_typed_value_fn)     (PDCI_node_t *node) 
const char *  (* PDCI_string_value_fn)    (PDCI_node_t *node)

struct PDCI_vtable_s {
  PDCI_cachedNode_init_fn   cachedNode_init;
  PDCI_kth_child_fn         kth_child;
  PDCI_kth_child_named_fn   kth_child_named;
  PDCI_free_fn              free;
  PDCI_get_id_fn            get_id;
  PDCI_typed_value_fn       typed_value;
  PDCI_string_value_fn      string_value;
}
\end{verbatim}

The {\tt kth\_child, kth\_child\_named, typed\_value}, and {\tt
  string\_value} functions correspond to those functions with
corresponding names in the \pglx interface. The others will be
discussed later.

While the virtual-table approach complicates adding new type-specific
functions, it makes using them quite simple.  The type-neutral code
merely looks up the desired function in the virtual table and calls
it. As long as the node was initialized correctly, the appropriate
function for that type will be called. Within such a function, the
node's data fields will be downcast to the required type, and then
manipulated in a type-specific manner.

%% Of course, this approach is a poor-man's polymorphism, opening many
%% opportunities for error. Therefore, one must be very careful when
%% manipulating data typed as {\tt void *}.

The basic contents of the node data structure are shown below:
\begin{verbatim}
struct PDCI_node_s {
  const char              *name;
  const char              *kind;
  PDCI_node_t             *parent;

  void                    *m;
  void                    *pd;
  void                    *rep;

  const PDCI_vtable_t     *vt;
  P_t                     *pads;
  ...
}
\end{verbatim}

All nodes have a name and a kind, stored as C-strings. Additionally,
all nodes but the document node have a parent in the document, hence
the parent field that points to another node. Next we have the mask,
representation and parse-descriptor fields. As these elements of the
node are type-specific, they are all typed as \vptr. 

Next we have the pointer to the node's virtual table, {\tt vt}. The
virtual functions will often need a valid \pads handle to execute
correctly. Therefore, we also include a pads handle field in each
node.

\paragraph{Node Memory Management}

Node's are dynamically allocated and freely exchanged between
\pads and \galax, and, so, we must take care to avoid dangling
pointers and leaked memory. We have implemented a simple memory
management scheme that both allows us to free nodes no longer needed
and allocate new nodes with reasonable efficiency. 

The key difficulty in freeing nodes is that \galax is not guaranteed
to hold a unique pointer a node that it uses, as aliases to the node
can exist within \pads. The main source of this aliasing is that each
node (except for the root, document node) maintains a reference to its
parent node. Therefore, any node with live children cannot be freed,
even if \galax no longer needs it. To address this problem, we
reference-count each node, storing the counter in the {\tt rc} field
of the node itself, show below. 
\begin{verbatim}
  unsigned long            rc
\end{verbatim}
The memory manager ensures that nodes are freed if and only if their
reference count is zero.

While reference counting protects against dangling pointers, it does
not initiate node deallocation.  The initiative must come from outside
of \pads, as \pads has no knowledge of how \galax uses the node
references provided it. For this reason, we have included a node
deallocation function in the \pglx interface:
\begin{verbatim}
void        PGLX_node_free           (nodeRep n)
\end{verbatim}
When the \galax garbage collector finds an unreachable a node reference,
it calls this function to inform \pads, which proceeds as appropriate
based on the reference count of the particular node.

While reference counting helps ensure the safety of the system, we
also desire that node allocation and deallocation be as efficient as
possible.  Nodes can be relatively short-lived, as \galax does not
retain nodes that it is not actively querying. For this reason, we
decided to create a node free-list for recycling nodes. As each node
is of the same size, no matter its type, node allocation is simple
matter of removing the first node on the list, merely involving the
adjustment of one free list pointer, rather than a potentially
expensive search by the C memory manager for an appropriately sized
block of memory. The disadvantage of this approach is that it ties-up
memory that might be needed by other portions of the program.
Therefore, in order to provide control over memory consumption, the
node memory manager is parameterized by the maximum size of the free
list.

\subsection{Extensions}
\label{sec:node-ext}

We can extend the basic nodes described above in an object-oriented
manner to create new node types.  First, we extend the node datatype
itself, adding new fields as appropriate.  Then, we override old
virtual table functions with new ones, as necessary. In our current
implementation, we have three extensions to the basic node. For
simplicity, though, we have not defined new structures for each type,
instead placing all of the extended fields in the same {\tt
  PDCI\_node\_s} structure. The difference between the types lies
instead in the implementation of their virtual tables.

For example, we have extended basic nodes with caching nodes. Each
time a child is requested of a basic node, a new node is created to
represent that child. However, the information in that node never
changes. Future requests for that child could safely reuse the
allocated node. Therefore, we've designed the caching node to cache
pointers to children nodes that it creates.  To do so, we first added
a cache to the node structure:
\begin{verbatim}
  PDCI_node_t           **child_cache 
\end{verbatim}

At this point, however, we ran into two problems. First, how can we
initialize the child cache? The standard node allocation and
initialization is unaware of the cache. We need a special
initialization function. However, each type has a different number of
children. To solve this, we need to either add the number of children
to the virtual table, or add a function that has hard-coded the number
of children for the type. We choose the later option for consistency,
naming it the {\tt cachedNode\_init} function. Each type defines its
own initialization function to convert a regular node into a cached
nodes. For node types that cannot be converted, including cached nodes
themselves, we set this slot in the table to {\tt
PDCI\_error\_cachedNode\_init}.

Our second problem relates to freeing nodes. Cached nodes should never
be freed. So, we can't use the same deallocation function for all node
types. We need to make the free function virtual as well, hence the
{\tt free} field in the virtual table. These two new function types
appear above with the rest of the virtual table function types.

The next step in extending the basic node is to define a new {\tt
  kth\_child} function for caching nodes that checks whether the child
node is in the cache before allocating a new node, and saves it in the
cache in the event of allocation. However, the cached node \kcfun is
signifcantly simpler than the original, as it is implemented using the
original function. For all uncached children, \cnkc calls the node's
\kcfun and then initializes the returned node using that node's
\cninitfun.  The \kcnfun, however, does not need to be overriden, as
it is implemented using the virtual \kcfun.


\subsection{Smart Node Implementation}
\label{sec:sn-details}


\paragraph{Meta-Data Structures}

Smart nodes and smart-node-descendants are also implemented as
extensions of the basic node type. Smart nodes extend the node type
itself with a single field:
\begin{verbatim}
  PDCI_smart_node_t       *snExt    
\end{verbatim}
pointing to additional data needed by the smart node. We chose not to
implement that data directly into the node type as we foresee there being at
most one smart node in any given pads program. It did not, therefore,
seem space-efficient for every node to carry those fields. At the same
time, the added cost of allocating the extension seperately is
trivial, due to its infrequency.

Here is the \verb|PDCI_smart_node_t| type:
\begin{verbatim}
struct PDCI_smart_node_s {
  PDCI_smart_elt_read_fn     elt_read;
  PDCI_smart_elt_alloc_fn    elt_alloc;
  PDCI_smart_elt_free_fn     elt_free;
  PDCI_path_walk_fn          elt_path_walk;
  PDCI_smart_failure_fn      handle_failure;

  void                      *elt_state;  // for arrays, this is PDCI_smart_array_t 
}
\end{verbatim}

The structure contains five functions, four for manipulating elements
and one for failure reporting. This last one is deprecated, so we
will not discuss it. We present the types of the other four functions
below:

\begin{verbatim}
Pread_res_t (*PDCI_smart_elt_read_fn)(PDCI_node_t *smartNode, P_t *pads, 
                                      PDCI_smart_elt_info_t *info)


Perror_t (*PDCI_smart_elt_alloc_fn)(PDCI_node_t *smartNode, P_t *pads,
                                    void **elt_pd, void **elt_rep)

Perror_t (*PDCI_smart_elt_free_fn)(P_t *pads, PDCI_smart_elt_info_t *info)


Perror_t (* PDCI_path_walk_fn) (P_t *pads, void *m, void *pd, void *rep, 
                                PDCI_path_t path,
                                void **m_out, void **pd_out, void **rep_out)
\end{verbatim}


The function \verb|elt_read| tells the smart node to bring an element
into memory, based on the element meta-data referenced by parameter
\verb|info|. Next, \verb|elt_alloc| attempts to allocate new memory.
This function is virtual so that different smart arrays can use
different allocation schemes. This call is guaranteed to fail if the
maximum number of elements are already in-memory. Function
\verb|elt_free| is currently unused, as element memory is never freed,
only recycled. However, a future element management system might take
advantage of free in order to dynamically adjust the amount of memory
consumed by the smart array. Also, free will be necessary for
implementing a function to cleanup a smart array once it is no longer
needed. Finally, \verb|elt_path_walk| walks the element representation
(rep), parse-descriptor (pd), and mask (m) according to the provided
path, in order to find the desired subcomponent of the element.
Pointers to the substructure's representation, parse-descriptor, and
mask are returned in the corresponding out-pointers. We will discuss
path walking in more detail below.

The last field of the smart node is a pointer to type-specific smart
node state (the field name \verb|elt_state| is outdated). As we only
support smart arrays at this point, all structures referenced by
\verb|elt_state| are variants of the structure shown below.
\begin{verbatim}
struct PDCI_smart_array_info_s {
  PDCI_smart_elt_info_t     *tmap;
  unsigned int               max_elts;

  PDCI_childIndex_t          next_idx_read;    
  PDCI_childIndex_t          next_idx_create;  

  // parse state
  Sfoff_t                    first_offset;
  Sfoff_t                    next_offset;

  ...  
}
\end{verbatim}
As discussed above, the \emd is an array of element meta-data. Field
\verb|max_elts| is the maximum number of allocated elements. This
field is set at initialization time, based on a parameter provided by
the smart node client.

If the smart array has not yet been fully read, the two
\verb|next_idx| fields track the next index of the \emd in need of
reading or initialization. Otherwise, these fields indicate the length
of the smart array. In the current implementation, they are kept
equal. However, we keep track of them independently so that an
implementation could jump ahead in the stream, upon request, and read
an element, without initializing the unrequested elements that were
skipped. 

Field \verb|first_offset| is the starting offset of the array. Field
\verb|next_offset| is used to save current the location in the data
source in the event that the smart array needs to reread an element
from earlier in the stream. Therefore, \verb|next_offset| contains the
offset of the next unread byte in the stream if the smart array last's
read was of a previously read element. If the last read was at the
head of the stream (i.e. a new element), then the contents of
\verb|next_offset| are irrelevant.

The aforementioned fields are common to all smart arrays. However,
some smart arrays have additional state requirements, as the
parameters passed to the smart array's \verb|read_one_init|,
\verb|read_one| and \verb|reread_one| function are all stored here. Those
parameters can include the location of the beginning of the array, any
user-specified parameters to the type and storage for any regular
expressions. For each smart array, the compiler generates an extended
version of the above structure with all the additional fields
necessary for the that type.

In order to ensure that all of the generated structures can be safely
cast to a \verb|PDCI_smart_array_info_s|, we have the compiler
generate the structure in two steps. First, it generates a seperate
strucuture containing only the additional parameters. This structure
is named \verb|ty_ro_params_s|, where \verb|ty| is the name of the
smart array type. Then, the extended array meta-data structure always
looks like this:
\begin{verbatim}
struct ty_array_info_t_s {
  PDCI_smart_array_info_t base;
  ty_ro_params_t params;
}
\end{verbatim}
In addition, this approach guarantees that any changes made to the
\verb|PDCI_smart_array_info_t| type will be reflected in the generated
types.


\paragraph{Reading Semantics}
Clients of a smart array can request any index at any point and the
array will attempt to provide that index. The core functions
supporting such requests are \kc and \verb|elt_read|. When a request
comes through \kc, the smart array first checks whether space has been
allocated in the \emd for that element. Under the current
implementation, this check is equivalent to a check of whether the
element has been previously read and its meta-data initialized. If the
element meta-data has been initialized, then a new node is created
based on that meta-data. If no space has been allocated, then an
allocation is attempted for all elements between the last element seen
and the requested one, followed by a call to \verb|elt_read|. If both
the allocation and read are successful, then a new node is created
based on the newly-created meta-data. Otherwise, we return 0,
indicating that the element does not exist. Note that if the element
access fails for reasons other than the element's non-existence (such
as some form of IO error), than the error responsible should be
reported through the appropriate channels.

For previously unseen elements, the \kc function requests
the \verb|elt_read| function to read the desired element. However, the
\verb|elt_read| function can also be called for an existing element
that needs to be brought back into memory. Therefore, one of the first
steps taken is to distinguish between the two cases. If a reread is
desired, then we save the current position in the stream (if
necessary), seek to the desired position and reread the element. If a
new read is desired, then move to the head of the stream (if we're not
their already) and read until the desired element is found, an error
occurs, or we reach the end of the array. Note that for new reads,
the process described above will read through all unseen elements
until the desired one is found without keeping those nodes in
memory. However, it will initialize the meta-data - including offsets
- of all such intermediate nodes.

As mentioned in Section~\ref{subsec:smart-nodes}, the {\em pd} and
{\em length} children of a smart array are returned as the last two
children of the array. This feature requires them to be treated
specially by \kc and \verb|elt_read|, as they do not have a fixed
index. Instead, once the array has terminated, we compare the
requested index to the length of the array. If equal, then the child
returned is {\em pd}. If the index is one greater than the length,
then the child returned is {\em length}.

Another function provided for accessing a smart array is the \kcnfun.
While this function usually simply translates an index and name into
simple child index, it behaves differently for the smart array. As the
indeces of the {\em pd} and {\em length} children are unknown before
the array has been seen in its entirety, this \kcnfun forces the array
to be read to completion in the case that either of these fields are
requested. Once the array has read to completion, the length of the
array is known, and the indeces of these fields can be calculated
simply, as described above.

\paragraph{Node Validation}
One of the important design decisions for node validation, is where to
place the validations. We could eagerly validate nodes upon entry to
the \pglx library, however that might cause unnecessary validations if
a nodes data is never examined. However, lazy validation can cause
unnecessary work as well. For a given node, if its element's data will
be accessed through descendant nodes, then it would be more efficient
to validate the node itself and then give all children valid data pointers,
then to pass null pointers to the children, forcing each descendant to
update its pointers independently.  We have chosen a compromise
between lazy and eager. In the case of new elements, we validate
eagerly, becuase we do not know whether the element in fact
exists. Therefore, it is necessary to force a read of the element to
find out. Otherwise, we only force the data into memory if we actually
need it.

However, identifying when data is needed is a little subtle. While
functions such as \tv obviously touch the data, many \kc functions do
as well. For example. for PDCI\_sequenced\_pd and
PDCI\_structured\_pd, the existence of the child {\em loc} depends on
the value of the child {\em errCode}. Therefore, despite the fact that
we are not accessing the {\em value} of the {\em loc} child, we need
to bring the element into memory to check the value of child {\em
  errCode}.  Similarly, for base-type \kc functions, the existence of
the {\em val} child, which represents the actual value of the node,
depends on there having been no serious parsing errors, as indicated
in the parse-descriptor. Once again, we must force the element into
memory in order to access the contents of its parse-descriptor.

\paragraph{Paths}


\paragraph{Node Identifiers}


\paragraph{C-strings}


\subsection{Error Handling}
\label{sec:error-handling}

\subsection{Node Revalidation}

  A smart node (currently, only smart arrays) is conceptually divided 
into elements, which serve as the level of granularity at which data is 
brought into and out of memory. Every descendant of a smart node is 
grouped according to the element from which it descends. Each descendant 
node remembers to which group it belongs, and the system uses this 
information when revalidating the node. However, knowing the ancestor 
element of a given node is not enough to revalidate it. In addition, we 
need to know exactly which portion of the element is represented by the 
given node. In order to record this second piece of information, we 
introduce the notion of paths.

 In essence, a path is a set of directions leading from (the pointers 
to) an element's representation, mask and parse descriptor (henceforth: 
data) to the particular node's data. For a given element, a path 
uniquely determines a child of that element.

  Node revalidation, then, consists of two key steps. First we ensure 
that the data of the node's ancestor element is available in memory. 
Then, we retrieve the node's path, and, using it and the pointers to the 
element data, we locate the node's data in memory. In the current 
system, node revalidation occurs in the \verb+PDCI_sndNode_make_valid+
function, located in libpglx/smart.c. In order to better understand node 
revalidation, we walk through the function, below:

\begin{verbatim}
Perror_t PDCI_sndNode_make_valid(PDCI_node_t *node){
  Pread_res_t res;
  Perror_t result;
\end{verbatim}

The first step is to determine the node's ancestor. This step is 
accomplished via a function call, as a pointer to the ancestor is not 
stored directly in the node.

\begin{verbatim}
  PDCI_smart_elt_info_t  *ancestor = PDCI_get_ancestor(node);
  PDCI_smart_node_t      *sn       = ancestor->parent->snExt; // Alias 
the smart node.
  PDCI_path_t             path;
\end{verbatim}

Next, we check the status of the ancestor's data. The if-statement below 
ensures that the data is in memory. We will not go into the details here.

\begin{verbatim}
  if (ancestor->rep == NULL) {
    /* must read from IO stream first */
    res = sn->elt_read(ancestor->parent, node->pads, ancestor);
    if (res != P_READ_OK_DATA) {
      /*
       * the read failed when we
       * were expecting it to succeed.
       */
      return P_ERR;
    }
  }
\end{verbatim}

At this point, we know that the ancestor's data is available in memory. 
Now, we need to retrieve the pointers to our node's data. First, we 
calculate the path from the ancestor to the node (the details of this 
calculation are discussed below).

\begin{verbatim}
  path = PDCI_node_getPath(node,0);
\end{verbatim}

Now, using this path and the ancestor's data pointers, we ask the smart 
node to walk the path, which returns the node's data pointers in the 
last three arguments of the path walk function. The function called is 
virtual, but there is currently only one macro implementing this 
function, located in pglx-codegen-macros.h. We will discuss this macro 
below.

\begin{verbatim}
  /* update pointers and generation */
  result = sn->elt_path_walk(node->pads,ancestor->m, ancestor->pd, 
ancestor->rep, path,
              &(node->m), &(node->pd), &(node->rep));

  PDCI_PATH_FREE(path);
\end{verbatim}

At this point, the node should have valid data pointers. However, in 
case there was a problem with the path, we do a sanity check here on the 
result of the path walk function.

\begin{verbatim}
  if (result == P_ERR) {
    // Something's wrong:
    //sn->handle_failure(node->pads,sn,ancestor,ERROR_WARNING,"failed to 
find element in path walk");
    return P_ERR;
  }
\end{verbatim}

Now, everything should be ok and the node has been successfully 
revalidated. As a final step, we update the node's \verb+ptr_gen+ indicating 
that its data pointers are in sync with its ancestors data pointers.

\begin{verbatim}
  node->ptr_gen = ancestor->gen;

  return P_OK;
}
\end{verbatim}

In order to better understand the \verb+make_valid+ function, we'll now
discuss a few of the functions that it depends on. First, though,
we'll describe the current implementation of paths. As we said above,
a path needs to be a set of directions that uniquely identify a node
within an element. Our solution records a list of the child indexes
that, passed in succession to \verb+get_kth_child+ - starting with the
element node - would return the desired node. Since, for a given node,
a single index uniquely determines a child of that node, a list of
such indexes uniquely determines a descendant of that node.

  The \verb+PDCI_node_getPath+ function, therefore, constructs (a copy of) a 
given node's path. Once again, for efficiency, the path is not stored 
directly in the node and hence the function call is needed to recreate 
the path. Once constructed, the path contains a reverse-ordered list of 
the sequence of indexes from the element to the node and a value 
indicating the length of the path. See include/path\_walk.h for details 
of the data structure implementing a path.

  With this list, we call the smart-node path walk function. Here is the 
current (macro) implementation:

\begin{verbatim}
#define 
SN_ELT_PATH_WALK_BODY(ty,eltTy,eltPdTy,eltMaskTy,padsIN,mIN,pdIN,repIN,pathIN,m_outIN,pd_outIN,rep_outIN)
  eltTy    *c_rep = (eltTy *)(repIN);
  eltPdTy  *c_pd  = (eltPdTy *)(pdIN);
  eltMaskTy  *c_m   = (eltMaskTy *)(mIN);
/* END_MACRO */

#define SN_ELT_PATH_WALK_RET(eltTy,padsIN,pathIN,m_outIN,pd_outIN,rep_outIN)
eltTy ## 
_node_pathWalk((padsIN),c_m,c_pd,c_rep,(pathIN),(m_outIN),(pd_outIN),(rep_outIN))
/* END_MACRO */
\end{verbatim}

This function simply calls the \verb+path_walk+ function specific to the type 
of the smart node's element, inserting the appropriate casts. This 
simplicity is due to the fact that for smart arrays, all elements have a 
common type, and, hence, a common path walk function.

  While the exact implemention of the \verb+path_walk+ function depends on the 
type for which it is generated, its basic structure depends on the pads 
type (e.g. struct, array, union, etc.). The differences in structure 
reflect the structural differences of the \verb+kth_child+ functions, so 
discussing the \verb+path_walk+ function for one pads type should be enough to 
explain it for all pads types. Below we show the Pstruct \verb+path_walk+ 
macro, from pglx-codegen-macros.h:

\begin{verbatim}
#define STR_NODE_PATH_WALK_BODY_BEGIN()
  Perror_t res = P_ERR;
  PDCI_childIndex_t idx;
 
  if (path.length > 0){
\end{verbatim}

If the path length is non-zero then we have not yet arrived at the 
destination node.  Therefore, this branch will require that we 
(recursively) call another \verb+path_walk+ function.
First, we remove the first index from the path's index list.

\begin{verbatim}
    /* modifies path */
    idx = PDCI_PATH_GET(path);
\end{verbatim}

Now, we use the index to determine which child to visit next.

\begin{verbatim}
    switch(idx){
    case 0:
\end{verbatim}

If we're visiting the pd, then we know that there is no mask. So, we can 
already set \verb+*m_out+ to NULL.

\begin{verbatim}
      *m_out = NULL;
\end{verbatim}

As we're in the Pstruct macro, the pd is a structure pd. For an array, 
it would be a sequenced pd. Etc. Notice that as the path variable was 
already modified, we pass it directly to the child \verb+path_walk+ call - no 
further modification is required.

\begin{verbatim}
      res = PDCI_structured_pd_node_pathWalk(pads,(PDCI_structured_pd 
*)pd,path,pd_out,rep_out);
      break
/* END_MACRO */
\end{verbatim}

Now, what happens if the path length is zero? Then, there are no more 
children to visit and we have arrived at our destination. Let's take a look:

\begin{verbatim}
#define STR_NODE_PATH_WALK_BODY_END()
    }
  }else{
\end{verbatim}

We set all the out variables to the corresponding values from the node 
we are visiting.

\begin{verbatim}
    *rep_out = rep;
    *pd_out = pd;
    *m_out = m;
\end{verbatim}

Then, we indicate that the walk was successful, and we're done.
\begin{verbatim}
    res = P_OK;
  }
/* END MACRO */
\end{verbatim}

  An additional note: for each child other than the pd, a corresponding 
case statement is inserted into the \verb+path_walk+ function, via a call to 
the \verb+NODE_PW_CASE+ macro, in the switch statement of the if-branch:

\begin{verbatim}
#define NODE_PW_CASE(fieldNumIN,fieldTy,fieldNameIN)
    case fieldNumIN:
      res = fieldTy ## 
_node_pathWalk(pads,&(m->fieldNameIN),&(pd->fieldNameIN),&(rep->fieldNameIN),path,m_out,pd_out,rep_out);      

      break
/* END_MACRO */
\end{verbatim}

The field cases are quite similar to the pd case. The only thing to note 
is that a field (may) have a mask, and so the \verb+m_out+ variable is passed 
on to the field's \verb+path_walk+ function.

In addition to the path walk macros, there are concrete
\verb+path_walk+ functions in libpglx/path\_walk.c for all base types,
pd types, \verb+Ploc+, \verb+Ppos+, and \verb+PDCI_cstr+. Their
structure is essentially the same, so I will not discuss them in
detail. The only thing to note is that some of them contain only
terminating cases, with no recursive call to child \verb+path_walk+
functions. This structure occurs in node types that cannot contain
children.

The final function of interest is 
\begin{verbatim}
PDCI_node_getPath(PDCI_node_t *node, int curLength). 
\end{verbatim}
While a path could be stored in its entirety for each 
node, doing so can consume a fair amount of space. As the only 
difference between a node's path and its parent's path is the node's 
index with respect to its parent, that index is the only thing we store 
in each node. Then, when we need the path, the \verb+PDCI_node_getPath+
function recursively climbs up from the node to its ancestor, recording 
the length of the path as it climbs. Upon reaching the ancestor it 
allocates a path of the correct length, and then returns down the path 
(via function call return), filling in the path as it does.

\appendix


\section{Directory Structure}
\end{document}
