\chapter{Histogram}
\label{chap:histogram}
\cutname{histogram.html}

A histogram is a piecewise-constant approximation of an observed 
data distribution. It is used as a small space, approximate
synopsis of the underlying data distribution which are ofter too
large to be stored precisely. Histograms are built for each meaningful 
piece of a \pads{} description. \figref{figure:wsl-hist-report-1}
is an example report for the length field of a web server log data.     
%
\begin{figure*}
\begin{small}
\begin{verbatim}
*** Histogram Result *** 
From 0 to 397, with height 4016. 
From 398 to 398, with height 36122. 
From 399 to 423, with height 5584. 
From 424 to 424, with height 33250. 
From 425 to 499, with height 3126. 
*** Histogram Result *** 
From 0 to 196, with height 3286. 
From 197 to 197, with height 30430. 
From 198 to 313, with height 3233. 
From 314 to 314, with height 30430. 
From 315 to 499, with height 3655. 
*** Histogram Result *** 
From 0 to 242, with height 3686. 
From 243 to 243, with height 36122. 
From 244 to 441, with height 3720. 
From 442 to 442, with height 26074. 
From 443 to 499, with height 7035. 
*** Histogram Result *** 
From 0 to 93, with height 4204. 
From 94 to 94, with height 36122. 
From 95 to 206, with height 3000. 
From 207 to 210, with height 21496. 
From 211 to 499, with height 3890. 
. . . . . . . . . . . . . . . . . . . . . . 
\end{verbatim}
\end{small}
\caption{Portion of histogram report for length field of web server
log data.}
\label{figure:wsl-hist-report-1}
\end{figure*}
%
In this particular run, optimal 5-bucket histogram is built for 
every 500 values seen in the data source.

\section{Operations}
\figref{figure:histogram} shows the histogram functions declared 
for a \pads{} type.
\begin{figure}
\inputCode{hand_code/hist-declare}
\caption{Histogram functions generated for the \texttt{entry\_t} type.}
\label{figure:histogram}
\end{figure}
%
These functions have the following behaviors:
\begin{description}
\item[\cd{entry\_t\_hist\_init}] Initializes histogram data
  structure. This function must be called before any data can be added
  to the histogram.
\item[\cd{entry\_t\_hist\_setPara}] Customizes histogram data
  structure. For the two conversion functions (specified below), user
  needs to set the corresponding fields in the template program. This
  function must be called to make any customization effected.  
\item[\cd{entry\_t\_hist\_reset}] Reinitializes histogram data
  structure. This function can be used to set any point of the data
  source as the start point of a new run. But it can't be used to 
  reset any previous defined parameters.
\item[\cd{entry\_t\_hist\_cleanup}] Deallocates all memory associated
  with histogram.
\item[\cd{entry\_t\_hist\_add}] Inserts a data value. This function 
  is called once a new record is coming. Any data type with an
  associated mapping function to \cd{Pfloat64} is considered as a 
  meaningful type. This function tracks fields with meaningful type
  and legal values only. The output parameter isFull will be set 
  nonzero, if the current data is the last one of a portion.
\item[\cd{entry\_t\_hist\_reportFull2io}] Writes summary report for
  finished histograms to \cd{*outstr}. In most cases, when this
  function is called, all stored histograms will be reported and the
  space will be released, while the current one won't.
\item[\cd{entry\_t\_hist\_reportAll2io}] Writes summary report for
  all histograms to \cd{*outstr}. When this function is called, all 
  histograms will be reported and the space will be released. 
\item[\cd{entry\_t\_hist\_reportFull}] Writes summary report for
  finished histograms to screen.
\item[\cd{entry\_t\_hist\_reportAll}] Writes summary report for
  all histograms to screen. 
\end{description}
\figref{figure:wsl-hist-hand} illustrates a sample use of histogram
functions for printing a summary of CLF \cd{entry_t}s.  
\begin{figure}
\inputCode{hand_code/histogram}
\caption{Simple use of histogram functions for the
  \texttt{entry\_t} type from CLF data.}
\label{figure:wsl-hist-hand}
\end{figure}

\section{Customization}
\label{sec:histograms-customization}
Users are allowed to customize various aspects of histogram by 
setting the appropriate field in the histogram data structure, 
which contains: 

\begin{description}
\item[\cd{INIT\_N}] is a \cd{Puint64} denoting the number of values
  for histogram to summarize. If the number of values in the data
  source exceeds \cd{INIT\_N}, histograms will be built on each
  \cd{INIT\_N} data values respectively, until the end of data
  source is reached. 

\item[\cd{INIT\_B}] is a \cd{Puint32} denoting the number of buckets
  in final histogram. As \cd{INIT\_B} increases, accuracy of final
  histogram approximation is increased, while more time and space is
  consumed. The default value is 10.

\item[\cd{INIT\_M}] is a \cd{Pint64} denoting an upper bound of data
  values in data source. Time consumed increases in poly-logrithm of
  \cd{INIT\_M}, so \cd{INIT\_M} can be set very large if little about
  data values in data source is known. 

\item[\cd{INIT\_ISE}] is a \cd{Pint8} denoting whether buckets in the final
  histogram are required to be of the same width. If \cd{INIT\_ISE} is
  set to be non-zero, all buckets have equal width. In this case,
  the time needed is in linear-\cd{INIT\_N}, and only constant space will
  be used. However, the result histogram will have less accuracy. 

\item[\cd{INIT\_ISO}] is a \cd{Pint8} denoting whether final histogram
  is required to be optimal or not. This parameter is valid only
  when \cd{INIT\_ISE} is zero. If \cd{INIT\_ISO} is set to be
  non-zero, the result histogram will be the most accurate one among
  all \cd{INIT\_B} bucket histograms. However, the time needed is in
  cubic-\cd{INIT\_N}, which could be extremely slow, and the space
  needed is in linear-\cd{INIT\_N}, since all data values in each
  \cd{INIT\_N} section are required to be stored.

\item[\cd{INIT\_n}] is a \cd{Pint8} denoting whether norm 1 or norm 2
  is used to measure accuracy of final histogram. Currently, norm 1
  measurement is supported only when all the data values are
  stored. In other words, it is supported only when \cd{INIT\_ISE} is
  zero, and \cd{INIT\_ISO} is non-zero. 

\item[\cd{INIT\_e}] is a \cd{Pfloat64} denoting error tolerance of the
  final histogram. This parameter is valid only when non-optimal
  result is allowed, namely both \cd{INIT\_ISE} and \cd{INIT\_ISO} are
  zeroes. The final histogram will be guaranteed to be no worse than
  poly- (1+\cd{INIT\_e}) times of the optimal one, but the time and
  space needed increase as the error tolerance decreases.  

\item[\cd{INIT\_scale}] is a \cd{Pint64} denoting scale factor for
  each data value. This parameter is important for computing, but will
  not affect the final result. For example, if the data source can
  take values up to \cd{64} bits, the overall \cd{SSE} could need
  as many as \cd{128} bits, which exceeds the representation limit of
  \pads{}. In this case, \cd{INIT\_scale} is needed. 

\item[\cd{INIT\_maxPortion}] is a \cd{Pint8} denoting the maximal
  number of stored histograms. If user forgets to report a finished
  histogram, or the inner histogram is finished while the outer one is
  not in nested cases, the finished histogram will go into a stored
  histogram list. If a new histogram is required to build, while the
  number of stored histograms is already \cd{INIT\_maxPortion}, the
  list will be cleared and a warning will be reported.

\item[\cd{entry\_t\_toFloat}] is a function pointer, taking
  \cd{entry\_t} as input parameter, and returning corresponding
  \cd{Pfloat64}. Histograms will handle \cd{Pfloat64} type data value 
  only. Any type with a well-defined conversion function to
  \cd{Pfloat64} is considered as a meaningful type, and could be summarized
  correctly by histograms. By default, all base types other than \cd{Pstring} in \pads{} have
  conversion functions to \cd{Pfloat64}. Users are allowed to write their
  own conversion function for each field by defining macro \cd{EXTRA_INIT_CODE}. If zero is assigned to this pointer, those default
  conversion functions will be used.
   
\item[\cd{entry\_t\_fromFloat}] is a function pointer, taking
  \cd{Pfloat64} as input parameter, and returning corresponding
  \cd{entry\_t} type. Any type without a well-defined conversion
  function from \cd{Pfloat64} may not be printed correctly. By
  default, all base types other than \cd{Pstring} in \pads{} have
  conversion functions from \cd{Pfloat64}. Users are allowed to write their
  own conversion function for each field by defining macro \cd{EXTRA_INIT_CODE}. If zero is assigned to this pointer, those default
  conversion functions will be used.

\end{description}

\section{Template Program}
Because generating a histogram report from a \pads{} description is a
very routine task, \pads{} provides a template program to automate the
task for common data formats. In particular, the template applies to
data that can be viewed as an optional header followed by a sequence
of records. Note that any data source that can be read entirely into
memory fits this pattern by considering the source to have no header
and a single body record. 

When instantiated, the template program takes an optional command-line
argument specifying the path to the data source. If no argument is
given, it uses a default location for the data specified by the
template user. The template first reads the optional header, then
reads each record and inserts the value of each meanful field into
histogram until either the data source is exhuasted or the end of a
portion is reached, at which point it prints the histogram report to
standard io. The following list describes the macros used by
histogram template:

\begin{description}

\item[\cd{DEF\_INPUT\_FILE}] If defined, this macro specifies a string
  representation of the path to the default data source. If no path to
  the data is supplied at the command-line, this is the location used
  for input data. 
\item[\cd{EXTRA\_BEGIN\_CODE}] If defined, this macro points to a \C{}
  statement that will be executed after all initialization code is
  performed, but before the optional header is read.
\item[\cd{EXTRA\_DECLS}] This optional macro defines additional \C{}
  declarations that proceed all template code.
\item[\cd{EXTRA\_DONE\_CODE}] If defined, this macro points to a \C{}
  statement that will be executed after generating the accumulator report.
\item[\cd{EXTRA\_INIT\_CODE}] This optional macro defines additional \C{}
  codes that customize histogram data structure for different fields.
\item[\cd{EXTRA\_READ\_ARGS}] If the type of the repeated record was
  parameterized, this macro allows the user to supply corresponding
  parameters. 
\item[\cd{IO\_DISC\_MK}] If defined, this macro specifies the
  interpretation of \Precord{} by indicating which IO discpline the
  system should install. It specifies the discipline by naming the
  function to create the discipline. \secref{sec:io-discipline}
  describes the available IO discipline creation functions.  If the
  user does not define this macro, the system installs the IO
  discipline corresponding to  new-line terminated ASCII records.
\item[\cd{PADS\_HDR\_TY}] Intuitively, this macro defines the type of
  the header record in the data source. This macro need only be
  defined if the data source has a header record. It defines a function used by the template program to
  generate the various function and type names derived from the name
  of the header record type, $i.e.$, the type of the associated in-memory
  representation, mask, parse descriptor, read function, etc.
\item[\cd{PADS\_TY}] Intuitively, this macro defines the type of the repeated
  record in the data source, $i.e.$, the type of the value to be
  summarized. This macro must be defined to use the histogram
  template. It defines a function used by the template program to
  generate the various function and type names derived from the name
  of the record type, $i.e.$, the type of the associated in-memory
  representation, mask, parse descriptor, read function, etc.
\item[\cd{READ\_MASK}] This macro specifies the mask to use in reading
  the repeated record. If not defined by the user, the template uses
  the value \cd{P\_CheckAndSet}.

\end{description}
