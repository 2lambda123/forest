\chapter{Tutorial}
\label{chap:example}
\cutname{tutorial.html}
In this chapter, we use examples to give an overview of \pads{}.
The examples used in this chapter appear in the \pads{} distribution in the 
\texttt{pads/demo} directory.


\section{Example data formats}
We start by briefly describing the data formats we will use as running examples. 

\subsection{CLF: Common log format}
\label{sec:example:common-log-format}
One of the formats that web servers use to log client
requests is the Common Log Format (CLF)~\cite{wpp}.  Researchers use such logs to measure 
properties of web workloads and to evaluate protocol changes
by ''replaying'' the user activity recorded in the log.
This ASCII format consists of a sequence of
records, each of which has seven fields: the host name or IP address
of the client making the request, the account associated with the
request on the client side, the name the user provided for
authentication, the time of the request, the actual request, the
\textsc{http} response code, and the number of bytes returned as a
result of the request.  The actual request has three parts: the
request method (\eg, \texttt{GET}, \texttt{PUT}), the requested
\textsc{uri}, and the protocol version.  In addition, the second and
third fields are often recorded only as a '-' character to indicate
the server did not record the actual data.  \figref{figure:clf-records}
shows a couple of typical records.

\subsection{Provisioning data}

In the telecommunications industry, the term \textit{provisioning}
refers to the steps necessary to convert an order for phone service
into the actual service.  To track AT\&T's provisioning process, the
\dibbler{} project compiles weekly summaries of the state of certain
types of phone service orders.  These ASCII summaries store the
summary date and one record per order.  Each order record contains a
header followed by a sequence of events.  The header has 13 pipe
separated fields: the order number, AT\&T's internal order number, the
order version, four different telephone numbers associated with the
order, the zip code of the order, a billing identifier, the order
type, a measure of the complexity of the order, an unused field, and
the source of the order data.  Many of these fields are optional, in
which case nothing appears between the pipe characters.  The billing
identifier may not be available at the time of processing, in which
case the system generates a unique identifier, and prefixes this value
with the string ``no\_ii'' to indicate the number was generated. The
event sequence represents the various states a service order goes
through; it is represented as a new-line terminated, pipe separated
list of state, timestamp pairs.  There are over 400 distinct states
that an order may go through during provisioning.  The sequence is
sorted in order of increasing
timestamps. \figref{figure:dibbler-records} shows a small example of
this format.


\begin{figure*}[t!]
\begin{footnotesize}
\begin{verbatim}
207.136.97.49 - - [15/Oct/1997:18:46:51 -0700] "GET /tk/p.txt HTTP/1.0" 200 30
tj62.aol.com - - [16/Oct/1997:14:32:22 -0700] "POST /scpt/dd@grp.org/confirm HTTP/1.0" 200 941
\end{verbatim}
\end{footnotesize}
\caption{Tiny example of web server log data.}
\label{figure:clf-records}
\end{figure*}

\begin{figure*}
\begin{footnotesize}
\begin{verbatim}
0|1005022800
9152|9152|1|9735551212|0||9085551212|07988|no_ii152272|EDTF_6|0|APRL1|DUO|10|1000295291
9153|9153|1|0|0|0|0||152268|LOC_6|0|FRDW1|DUO|LOC_CRTE|1001476800|LOC_OS_10|1001649601
\end{verbatim}
\end{footnotesize}
\caption{Tiny example of \dibbler{} provisioning data.}
\label{figure:dibbler-records}
\end{figure*}

\begin{figure}
\inputCode{code/wsl}
\caption{\pads{} description for web server log data.}
\label{figure:wsl}
\end{figure}

\begin{figure}
\inputCode{code/sirius}
\caption{Partial \pads{} description for \dibbler{} provisioning data.}
\label{figure:dibbler}
\end{figure}



\section{\padsl{} descriptions}
\label{sec:example:padsl-description}
\figref{figure:wsl} gives the \pads{} description for CLF web server
logs,  while \figref{figure:dibbler} gives the description for the
\dibbler{} provisioning data.  We use these examples to illustrate
various features of the \pads{} language.  In \pads{} descriptions,
types are declared before they are used, so the type that describes
the totality of the data source appears at the bottom of the
description.

\kw{Pstruct}s describe fixed sequences of data with unrelated types.
In the CLF description, the type declaration for \cd{version_t}
illustrates a simple \kw{Pstruct}. It starts with a string literal
that matches the constant \cd{HTTP/} in the data source.  It then has
two unsigned integers recording the major and minor version numbers
separated by the literal character \literal{\cd{'.'}}.  \pads{} supports
character, string, and regular expression literals, which are
interpreted with the ambient character encoding. The type
\cd{request_t} similarly describes the request portion of a CLF
record.  In addition to physical format information, this \kw{Pstruct}
includes a semantic constraint on the \cd{version} field.
Specifically, it requires that obsolete methods \cd{LINK} and
\cd{UNLINK} occur only under HTTP/1.1.  This constraint illustrates
the use of predicate functions and the fact that earlier fields are in
scope during the processing of later fields, as the constraint refers
to both the \cd{meth} and \cd{version} fields in the \kw{Pstruct}.
\chapref{chap:structs} describes \kw{Pstruct}s in detail.

\kw{Punion}s describe variation in the data source.  For example, the
\cd{client_t} type in the CLF description indicates that the first
field in a CLF record can be either an IP address or a hostname.
During parsing, the branches of a \kw{Punion} are tried in order; the
first branch that parses without error is taken.  The \cd{auth_id_t}
type illustrates the use of a constraint: the branch \cd{unauthorized}
is chosen only if the parsed character is a dash.  \pads{} also
supports a \textit{switched} union that uses a selection expression to
determine the branch to parse.  Typically, this expression depends
upon already-parsed portions of the data source.

\pads{} provides \kw{Parray}s to describe varying-length sequences of
data all with the same type.  The \cd{eventSeq_t} declaration in the
\dibbler{} data description uses a \kw{Parray} to characterize the
sequence of events an order goes through during processing.  This
declaration indicates that the elements in the sequence have type
\cd{event_t}.  It also specifies that the elements will be separated
by vertical bars, and that the sequence will be terminated by an
end-of-record marker (\kw{Peor}).  In general, \pads{} provides a rich
collection of array-termination conditions: reaching a maximum size,
finding a terminating literal (including end-of-record and
end-of-source), or satisfying a user-supplied predicate over the
already-parsed portion of the \kw{Parray}.  Finally, this type
declaration includes a \kw{Pwhere} clause to specify that the sequence
of timestamps must be in sorted order.  It uses the \kw{Pforall}
construct to express this constraint.  In general, the body of a
\kw{Pwhere} clause can be any boolean expression.  In such a context
for arrays, the pseudo-variable \cd{elts} is bound to the in-memory
representation of the sequence and \cd{length} to its length.

Returning to the CLF description in \figref{figure:wsl}, the
\kw{Penum} type \cd{method_t} describes a collection of data literals.
During parsing, \pads{} interprets these constants using the ambient
character encoding.  The \kw{Ptypedef} \cd{response_t} describes
possible server response codes in CLF data by adding the constraint
that the three-digit integer must be between 100 and 600.

The \cd{order_header_t} type in the \dibbler{} data description
contains several anonymous uses of the \kw{Popt} type.  This type is
syntactic sugar for a stylized use of a \kw{Punion} with two branches:
the first with the indicated type, and the second with the ``void''
type, which always matches but never consumes any input.



\section{Generated library}
\label{sec:example:generated-library}
From a description, the \pads{} compiler generates a \C{} library
for parsing and manipulating the associated data source.  
\setcounter{totalnumber}{1}
\setcounter{dbltopnumber}{1}
\renewcommand{\topfraction}{0.85}
\renewcommand{\textfraction}{0.1}
\renewcommand{\floatpagefraction}{0.75}
\begin{figure*}
\inputCode{code/dibbler_library}
\caption{Selected portions of the library generated for the \texttt{entry\_t}
  declaration from \dibbler{} data description.}
\label{figure:library}
\end{figure*}
To give a feeling for the library that \pads{} generates, 
\figref{figure:library} includes selected portions of the generated 
library for the \dibbler{} \cd{entry_t} declaration.


\subsection{Example library use}
\label{sec:example:library-use}
\figref{figure:sirius-filter} shows a simple use of 
the generated \dibbler{} library to filter out ill-formed records and normalize the representation of optional phone numbers. 
The code first initializes the \pads{} library handle, \texttt{p}, by invoking the core library function \texttt{P\_open}.  The second argument to this function allows the user to customize behavior in the \pads{} library by passing a (pointer to a) \texttt{Pdisc\_t} \textit{discipline}. With this discipline, the user can specify properties such as the endianness of the data, the default character set (ASCII or EBCIDC), error handling, \etc{}  The third argument to the \texttt{P\_open} function specifies a (pointer to a) \texttt{Pio\_disc\_t} discipline, which allows the user to describe how to detect record boundaries, \ie{}, are records new-line terminated, fixed width, EBCDIC-style, \etc{} 
Passing zero in these argument positions triggers default behavior, which 
means ASCII encoded, little endian, new-line terminated records.
\chapref{chap:library-customization} describes the use of disciplines in more detail. 

After initializing the \pads{} handle, the code uses the core library function \texttt{P\_io\_fopen} to open the data source by specify the path to the data on disk.  The code then uses generated functions to initialize the mask, parse descriptor, and in-memory representations for the header and the entry types.
The specified mask values instruct the parsing code to check all conditions in the \dibbler{} description except the sorting of the timestamps.  The \texttt{P\_Set} flag instead instructs the compiler to simply set the in-memory representation for the timestamp sequence.

After reading the header, 
the code echoes error records to one file and cleaned ones to another.
The raw data has two different representations of unavailable phone numbers:
simply omitting the number altogether, which corresponds to the \cd{NONE}
branch of the \kw{Popt}, or having the value \cd{0} in the data.  
The function \cd{cnvPhoneNumbers} unifies these two representations 
by converting the zeroes into \cd{NONE}s.  The function \cd{entry_t_verify}
ensures that our computation hasn't broken any of the semantic properties
of the in-memory representation of the data.

Finally, the core library functions \texttt{P\_io\_close} and \texttt{P\_close} close the IO stream and \pads{} library, respectively.

\begin{figure*}
\inputCode{code/sirius-filter}
\caption{Fragment of a program to filter and normalize \dibbler{} data.}
\label{figure:sirius-filter}
\end{figure*}



\subsection{Accumulators}

Before using a data source, analysts must develop an understanding 
of both the layout and the meaning of the data.  
Because documentation is usually
incomplete or out-of-date, this understanding must be developed 
through exploring the data itself.  Typical questions include:
how complete is the description of the syntax of the data source,
how many different representations for ``data not available" are there,
what is the distribution of values for particular fields, \etc{}
\pads{} addresses these kinds of questions with the notion of an accumulator. 
For each type in a \pads{} description, accumulators track the number of good values, the number of bad values, and the 
distribution of legal values.  Selected functions from this portion of the library appear in \figref{figure:library}; more detailed information appears in \chapref{chap:accumulators}.  

We can of course use these functions by hand to write a program to
compute the statistical profile of any \pads{} data source.  However,
ad hoc sources are often simply a sequence of records, perhaps
prefixed by a header.  For example, both the web server log and the
\dibbler{} data sources exhibit this pattern,  as does any data format
that can be read in one bulk read. 
As a convenience, \pads{} supplies a program
template to construct accumulator programs for such data sources.
\pads{} provides this template as a \C{} include file and supports
customization via \C{}'s macro system. 
Using this template, we can write an accumulator program by
specifying only the names of the optional header type and the record type.  
\figref{figure:wsl-accum} contains the entirety of the user-specified accumulator program text.  

\begin{figure*}
\inputCode{code/wsl-accum}
\caption{Accumulator program to construct statistical profiles of web server log data.}
\label{figure:wsl-accum}
\end{figure*}



The accumulator report for the length field
of the web server data that results when run on a data set used
in several studies of web traffic~\cite{clf-cluster, clf-adaptation}
appears in \figref{figure:wsl-accum-report}.
%
\begin{figure*}
\begin{small}
\begin{verbatim}
<top>.length : uint32
+++++++++++++++++++++++++++++++++++++++++++
good: 53544   bad: 3824    pcnt-bad: 6.666
min: 35  max: 248591  avg: 4090.234
top 10 values out of 1000 distinct values:
tracked 99.552% of values
 val:  3082 count:  1254  %-of-good:  2.342
 val:   170 count:  1148  %-of-good:  2.144
 val:    43 count:  1018  %-of-good:  1.901
 val:  9372 count:   975  %-of-good:  1.821
 val:  1425 count:   896  %-of-good:  1.673
 val:   518 count:   893  %-of-good:  1.668
 val:  1082 count:   881  %-of-good:  1.645
 val:  1367 count:   874  %-of-good:  1.632
 val:  1027 count:   859  %-of-good:  1.604
 val:  1277 count:   857  %-of-good:  1.601
. . . . . . . . . . . . . . . . . . . . . . 
 SUMMING    count:  9655  %-of-good: 18.032
\end{verbatim}
\end{small}
\caption{Portion of accumulator report for length field of web server
  log data.}
\label{figure:wsl-accum-report}
\end{figure*}
%
By default, accumulators track the first 1000 distinct
values seen in the data source and report the frequency
of the top ten values.  In this particular run, 99.552\%
of all values were tracked.  When generating the accumulator
program (or when using the library directly), \pads{} users can specify 
the number of distinct values to track and the number 
of values to print in the report.  Details about customizing accumulator programs appear in \chapref{chap:accumulators}.

Perhaps surprisingly, the report shows that 6.66\% of the length
fields contained errors.  A glance at the error log generated
by the program (which contains all records flagged as errors) 
reveals that web servers occasionally store the '-' character
rather than the actual number of bytes returned, a possibility
not mentioned in the documentation~\cite{wpp}.
Accumulators often serve as a quick tool for iteratively
refining a \pads{} description until only genuine errors remain.




\subsection{Formatting}
To support converting ad hoc data into a delimited format, the \pads{}
library generates a formatting function for each type.  This function,
an example of which appears in \figref{figure:library}, takes a delimiter
list as an argument.  At each field boundary, it prints the first delimiter.
At each nested type boundary, it advances the delimiter list unless the list
is exhausted, in which case it reuses the last delimiter.  The mask argument
allows the user to suppress printing of portions of the data.  Programmers
can use the library directly to write formatting programs by hand.  However, 
as in the accumulator case, \pads{} can generate a formatting program for 
commonly occurring data patterns given only the header type (optional), record type, and a delimiter string.  Users can further customize the generated program by specifying an output format for dates and mask values as illustrated in 
\figref{figure:wsl-fmt}.
\begin{figure*}
\begin{small}
\inputCode{code/wsl-fmt}
\caption{PADS program for formatting CLF records using formatting template.}
\label{figure:wsl-fmt}
\end{small}
\end{figure*}
Given the delimiter
string \cd{"|"} and the output date format \cd{"\%D:\%T"}, the generated
web server log formatting program yields
the output shown in \figref{figure:clf-records-formatted} when applied to the
sample data in \figref{figure:clf-records}. 
\begin{figure*}
\begin{small}
\begin{verbatim}
207.136.97.49|-|-|10/16/97:01:46:51|GET|/tk/p.txt|1|0|200|30
tj62.aol.com|-|-|10/16/97:21:32:22|POST|/scpt/dd@grp.org/confirm|1|0|200|941
\end{verbatim}
\caption{Formatted CLF records.}
\label{figure:clf-records-formatted}
\end{small}
\end{figure*}


To support customization, \pads{} allows users to provide their own formatting functions for any type.  More information on formatting functions can be found in \chapref{chap:formatting}.


\subsection{Conversion to \xml{}}
\pads{} also supports converting ad hoc data into XML by providing a canonical mapping from \pads{} descriptions into XML.  This mapping is quite natural, as both \pads{} and XML are languages for describing semi-structured data.
One interesting aspect of the mapping is that we embed not just the in-memory representation of \pads{} values, but also the parse descriptors in cases where the data was buggy.  This choice allows users to explore the error portions
of their data sources, which can be the most interesting parts of the data.
Given a \pads{} specification, the \pads{} compiler generates an XML Schema describing the canonical embedding for that data source.  
As an example, 
\figref{figure:sirius-xsd} shows 
the portion of the \xml{} Schema generated from the \dibbler{} data description
related to the 
\cd{eventSeq} type.

\begin{figure*}
\begin{small}
\begin{verbatim}
<xs:complexType name="eventSeq_pd">
<xs:sequence>
<xs:element name="pstate" type="Pflags_t"/>
<xs:element name="nerr" type="Puint32"/>
<xs:element name="errCode" type="PerrCode_t"/>
<xs:element name="loc" type="Ploc_t"/>
<xs:element name="neerr" type="Puint32"/>
<xs:element name="firstError" type="Puint32"/>
<xs:element name="elt" type="Puint32" 
    minOccurs="0" maxOccurs="unbounded"/>
</xs:sequence>
</xs:complexType>

<xs:complexType name="eventSeq">
<xs:sequence>
<xs:element name="elt" type="event" 
    minOccurs="0" maxOccurs="unbounded"/>
<xs:element name="length" type="Puint32"/>
<xs:element name="pd" type="eventSeq_pd" 
    minOccurs="0" maxOccurs="1"/>
</xs:sequence>
</xs:complexType>
\end{verbatim} 
\end{small}
\caption{Portion of \xml{} Schema generated from \dibbler{} data description.}
\label{figure:sirius-xsd}
\end{figure*}
The \pads{} compiler will put the generated schema in the same directory as the generated library.  The schema file will share the name of the \pads{} description and have the extension \texttt{xsd}.


The \pads{} compiler generates a \cd{write_xml_2io} function for each type, an example of which is shown in \figref{figure:library}.  Programmers can use these functions by hand to write conversion programs, or they can use 


 \pads{} also provides a template for generating such conversion programs for data that can be read entirely into memory.\footnote{
  The \xml{} template program also permits the programmer to specify a header type followed by a body record type.  However, the generated \xml{} does not include the appropriate top-level declarations to conform to the generated \xml{} schema.  We are working on addressing this problem.
} %
\figref{figure:sirius-xml} contains an example of such a program.
More information about converting to \xml{} can be found in \chapref{chap:xml}.
\begin{figure*}
\begin{small}
\inputCode{code/sirius-xml}
\end{small}
\caption{Program to convert \dibbler{} data into \xml{} using \pads{} template.}
\label{figure:sirius-xml}
\end{figure*}











