\chapter{Cluster}
\label{chap:cluster}
\cutname{cluster.html}

Clustering program divides data into several groups based on
certain distribution. It summarizes the data by recording specified
features of each group. Clustering is built for each meaningful 
piece of a \pads{} description. \figref{figure:wsl-cluster-report-1}
is an example report for a web server log data.     
%
\begin{figure*}
\begin{small}
\begin{verbatim}
[Describing each tag arm of <top>.host]

=====================================================================================================
<top>.host.resolved : array nIP of Puint8
=====================================================================================================
Array lengths: 
Clustering based distribution: User defined distribution. 
mean 4, and variance 0, containing 4 elements. 
=====================================================================================================
Possible anormality based on probability 0.010000: 
Possible anormality based on clustering elements number 0.100000: 

-----------------------------------------------------------------------------------------------------
allArrayElts : uint8
-----------------------------------------------------------------------------------------------------

Clustering based distribution: User defined distribution. 
mean 128, and variance 77, containing 8 elements. 
mean 136, and variance 0, containing 4 elements. 
mean 97, and variance 0, containing 4 elements. 
=====================================================================================================
Possible anormality based on probability 0.010000: 
Data (around): 49 
Data (around): 207 
Data (around): 49 
Data (around): 207 
Data (around): 50 
Data (around): 207 
Data (around): 50 
Possible anormality based on clustering elements number 0.100000: 

=====================================================================================================
<top>.host.symbolic : array sIP of Pstring_SE
=====================================================================================================
Array lengths: 
Clustering based distribution: User defined distribution. 
mean 4, and variance 0, containing 7 elements. 
=====================================================================================================
Possible anormality based on probability 0.010000: 
Possible anormality based on clustering elements number 0.100000: 

-----------------------------------------------------------------------------------------------------
allArrayElts : string
-----------------------------------------------------------------------------------------------------

Clustering based distribution: User defined distribution. 
mean non defined., and variance non defined., containing 28 elements. 
=====================================================================================================
Possible anormality based on probability 0.010000: 
Possible anormality based on clustering elements number 0.100000: 
. . . . . . . . . . . . . . . . . . . . . . 
\end{verbatim}
\end{small}
\caption{Portion of clustering report for web server log data.}
\label{figure:wsl-cluster-report-1}
\end{figure*}
%
In this particular run, maximal 3 clusterings are built for all the
data values seen in the data source.

\section{Operations}
\figref{figure:cluster} shows the clustering functions declared 
for a \pads{} type.
\begin{figure}
\inputCode{hand_code/cluster-declare}
\caption{Clustering functions generated for the \texttt{entry\_t} type.}
\label{figure:cluster}
\end{figure}
%
These functions have the following behaviors:
\begin{description}
\item[\cd{entry\_t\_hist\_init}] Initializes clustering data
  structure. This function must be called before any data can be added
  to the programme.
\item[\cd{entry\_t\_hist\_setPara}] Customizes clustering data
  structure. For the distribution function and two conversion
  functions (specified below), user needs to set the corresponding 
  fields explicitly. This function must be called to make any 
  customization effected.  
\item[\cd{entry\_t\_hist\_reset}] Reinitializes clustering data
  structure. This function can be used to set any point of the data
  source as the start point of a new run. But it can't be used to 
  reset any previous defined parameters.
\item[\cd{entry\_t\_hist\_cleanup}] Deallocates all memory associated
  with clustering.
\item[\cd{entry\_t\_hist\_add}] Inserts a data value. This function 
  is called once a new record is coming. Any data type with an
  associated mapping function to \cd{Pfloat64} is considered as a 
  meaningful type. This function tracks fields with meaningful type
  and legal values only.
\item[\cd{entry\_t\_hist\_report2io}] Writes summary report for
  clustering \cd{c} to \cd{*outstr}. 
\item[\cd{entry\_t\_hist\_report}] Writes summary report for
  clustering \cd{c} to screen. 
\end{description}
\figref{figure:wsl-cluster-hand} illustrates a sample use of clustering
functions for printing a summary of CLF \cd{entry_t}.  
\begin{figure}
\inputCode{hand_code/cluster}
\caption{Simple use of clustering functions for the
  \texttt{entry\_t} type from CLF data.}
\label{figure:wsl-cluster-hand}
\end{figure}

\section{Customization}
\label{sec:clusterings-customization}
Users are allowed to customize various aspects of clustering by 
setting the appropriate field in the clustering data structure, 
which contains: 

\begin{description}
\item[\cd{INIT\_CTYPE}] is an enumeration denoting the type of the
  underlying distribution for each clustering. Built-in distributions
  include K\_mean, Gaussian distribution, Exponential distribution and
  Laplace distribution. Users are allowed to add any distributions,
  which could be fully characterized by mean and variance, by setting
  the field \cd{Distri\_fn}, which will be specified later.

\item[\cd{INIT\_K}] is a \cd{Puint32} denoting the maximal number of
  clusterings users want to use to divide the date source. Together
  with \cd{INIT\_CTYPE}, it decides the underlying model of the data
  source. 

\item[\cd{INIT\_OPEN}] is a \cd{Pfloat64} denoting the probability
  threshold for opening a new clustering. A new clustering will be
  opened for a coming data value, if and only if, the number of current
  clusterings is less than \cd{INIT\_K} and the probabilities it falls
  in all current clusterings are less than \cd{INIT\_OPEN}.

\item[\cd{INIT\_INITVAR}] is a \cd{Pfloat64} denoting the initial
  variance for each clustering. It takes effect only before the second
  data item is inserted. After that, the variance of each clustering
  will be fully decided by its elements.

\item[\cd{INIT\_ANORM\_POS}] is a \cd{Pfloat64} denoting the
  probability threshold for detecting anormality. A data value will
  be reported as anormality if no new clustering is opened for it, and the
  probabilities it falls in all existing clusterings are less than
  \cd{INIT_ANORM_POS}. The data value detected later in the data source
  is expected to be more accurate than the one detected at the
  beginning of the data source.

\item[\cd{INIT\_ANORM\_NUM}] is a \cd{Pfloat64} denoting the element
  number threshold for detecting anormality. A whole clustering will be
  reported as anormality if the number of its elements is less than
  \cd{INIT\_ANORM\_NUM} of the total number of data items in the data source.

\item[\cd{entry\_t\_probFn}] is a function pointer, taking mean,
  variance and data value as input, and returning corresponding
  probability of that data value, according to mean and variance. Users
  could define their own distribution for each clustering, as long as the
  distribution is fully specified by mean and variance. Doing this,
  they need: first, set \cd{INIT\_CTYPE} to be \cd{OTHERS}; then, use
  \cd{EXTRA_INIT_CODE} to define their own distribution function, and
  assign them to this pointer. If \cd{OTHERS} is set to \cd{INIT\_CTYPE},
  and zero is set to this pointer, Gaussian distribution will be used.

\item[\cd{entry\_t\_toFloat}] is a function pointer, taking
  \cd{entry\_t} as input parameter, and returning corresponding
  \cd{Pfloat64}. Clusterings will handle \cd{Pfloat64} type data value 
  only. Any type with a well-defined conversion function to
  \cd{Pfloat64} is considered as a meaningful type, and could be summarized
  correctly by clusterings. By default, all base types other than \cd{Pstring} in \pads{} have
  conversion functions to \cd{Pfloat64}. Users are allowed to write their
  own conversion function for each field by defining macro \cd{EXTRA_INIT_CODE}. If zero is assigned to this pointer, those default
  conversion functions will be used.
   
\item[\cd{entry\_t\_fromFloat}] is a function pointer, taking
  \cd{Pfloat64} as input parameter, and returning corresponding
  \cd{entry\_t} type. Any type without a well-defined conversion
  function from \cd{Pfloat64} may not be printed correctly. By
  default, all base types other than \cd{Pstring} in \pads{} have
  conversion functions from \cd{Pfloat64}. Users are allowed to write their
  own conversion function for each field by defining macro \cd{EXTRA_INIT_CODE}. If zero is assigned to this pointer, those default
  conversion functions will be used.

\end{description}

\section{Template Program}
Because generating a clustering report from a \pads{} description is a
very routine task, \pads{} provides a template program to automate the
task for common data formats. In particular, the template applies to
data that can be viewed as an optional header followed by a sequence
of records. Note that any data source that can be read entirely into
memory fits this pattern by considering the source to have no header
and a single body record. 

When instantiated, the template program takes an optional command-line
argument specifying the path to the data source. If no argument is
given, it uses a default location for the data specified by the
template user. The template first reads the optional header, then
reads each record and inserts the value of each meanful field into
clustering until either the data source is exhuasted or the end of a
portion is reached, at which point it prints the clustering report to
standard io. The following list describes the macros used by
clustering template:

\begin{description}

\item[\cd{DEF\_INPUT\_FILE}] If defined, this macro specifies a string
  representation of the path to the default data source. If no path to
  the data is supplied at the command-line, this is the location used
  for input data. 
\item[\cd{EXTRA\_BEGIN\_CODE}] If defined, this macro points to a \C{}
  statement that will be executed after all initialization code is
  performed, but before the optional header is read.
\item[\cd{EXTRA\_DECLS}] This optional macro defines additional \C{}
  declarations that proceed all template code.
\item[\cd{EXTRA\_DONE\_CODE}] If defined, this macro points to a \C{}
  statement that will be executed after generating the accumulator report.
\item[\cd{EXTRA\_INIT\_CODE}] This optional macro defines additional \C{}
  codes that customize clustering data structure for different fields.
\item[\cd{EXTRA\_READ\_ARGS}] If the type of the repeated record was
  parameterized, this macro allows the user to supply corresponding
  parameters. 
\item[\cd{IO\_DISC\_MK}] If defined, this macro specifies the
  interpretation of \Precord{} by indicating which IO discpline the
  system should install. It specifies the discipline by naming the
  function to create the discipline. \secref{sec:io-discipline}
  describes the available IO discipline creation functions.  If the
  user does not define this macro, the system installs the IO
  discipline corresponding to  new-line terminated ASCII records.
\item[\cd{PADS\_HDR\_TY}] Intuitively, this macro defines the type of
  the header record in the data source. This macro need only be
  defined if the data source has a header record. It defines a function used by the template program to
  generate the various function and type names derived from the name
  of the header record type, $i.e.$, the type of the associated in-memory
  representation, mask, parse descriptor, read function, etc.
\item[\cd{PADS\_TY}] Intuitively, this macro defines the type of the repeated
  record in the data source, $i.e.$, the type of the value to be
  summarized. This macro must be defined to use the clustering
  template. It defines a function used by the template program to
  generate the various function and type names derived from the name
  of the record type, $i.e.$, the type of the associated in-memory
  representation, mask, parse descriptor, read function, etc.
\item[\cd{READ\_MASK}] This macro specifies the mask to use in reading
  the repeated record. If not defined by the user, the template uses
  the value \cd{P\_CheckAndSet}.

\end{description}
