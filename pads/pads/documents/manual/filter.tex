\chapter{Filters}
\label{chap:filter}
\cutname{filters.html}
Filters partition an input file into two output files: one with data
conforming to the specification, the other with data containing
errors.  Filters apply to data formats that contain an optional header
followed by a sequence of records.  


\section{Template Program}
Because generating a filter from a \pads{} description is
a very routine task, \pads{} provides a template program to automate
the task for common data formats.  In particular, the template applies
to data that can be viewed as an optional header followed by a
sequence of records.  

When instantiated, the template program takes an optional command-line
argument specifying the path to the data source. If no argument is
given, it uses a default location for the data specified by the
template user.  The location for the clean and error records can be
set in the template program.

The template first reads the optional header, echoing it to either the
clean or the error file, depending upon the resulting parse
descriptor. It then reads each record, echoing it to the approporiate
file until the data source is exhausted.

Like the accumulator template, the filter template is a \C{} header
file parameterized by a number of macros that permit the user to customize
the template by defining appropriate values for these macros.  
The following list describes the macros used by the
filter template:

\begin{description}
\cut{
\item[\cd{COPY\_STRINGS}] If defined, this macros indicates that
  string values should be copied into the in-memory representation
  rather than being shared.
  \secref{sec:library-customization-copy-strings} describes this
  choice in more detail.  }
\item[\cd{DATE\_IN\_FMT}] If defined, this macro sets the default
  input format for dates described by \cd{Pdate}.  See
  \secref{sec:library-customization-input-formats} for more
  information.
\item[\cd{DATE\_OUT\_FMT}] If defined, this macro sets the default
  output format for \cd{Pdate} and \cd{Pdate\_explicit}.  See
  \secref{sec:library-customization-output-formats} for more information.
\item[\cd{DEF\_INPUT\_FILE}] If defined, this macros specifies a
  string representation of the path to the default data source.  If no
  path to the data is supplied at the command-line, this is the
  location used for input data.
\item[\cd{EXTRA\_BAD\_READ\_CODE}] If defined, this macro points to a \C{}
  statement that will be executed after any body record containing an
  error.
\item[\cd{EXTRA\_BEGIN\_CODE}] If defined, this macro points to a \C{}
  statement that will be executed after all initialization code is
  performed, but before the optional header is read.
\item[\cd{EXTRA\_DECLS}] This optional macro defines additional \C{}
  declarations that proceed all accumulator code.
\item[\cd{EXTRA\_DONE\_CODE}] If defined, this macro points to a \C{}
  statement that will be executed after generating the accumulator report.
\item[\cd{EXTRA\_GOOD\_READ\_CODE}] If defined, this macro points to a \C{}
  statement that will be executed after any body record not containing an
  error.
\item[\cd{EXTRA\_HEADER\_READ\_ARGS}] If the type of the header record
  was parameterized, this macro allows the user to supply
  corresponding  parameters. 
\item[\cd{EXTRA\_READ\_ARGS}] If the type of the repeated record was
  parameterized, this macro allows the user to supply corresponding
  parameters. 
\item[\cd{IN\_TIME\_ZONE}] If set, this macro specifies the input time
  zone of date types that do not include time zone information.
  See \secref{sec:library-customization-input-time-zone} for more detail.
\item[\cd{IO\_DISC\_MK}] If defined, this macro specifies the
  interpretation of \Precord{} by indicating which IO discpline the
  system should install.  It specifies the discipline by naming the
  function to create the discipline. \secref{sec:io-discipline}
  describes the available IO discipline creation functions.  If the
  user does not define this macro, the system installs the IO
  discipline corresponding to  new-line terminated ASCII records.
\item[\cd{MAX\_RECS}] If defined, this macro specifies an integer that
  limits the number of repeated records that the accumulator program
  should read.
\item[\cd{OUT\_TIME\_ZONE}] If set, this macro specifies the output
  time zone of date types.
  See \secref{sec:library-customization-output-time-zone} for more detail.
\item[\cd{PADS\_HDR\_TY}]  Intuitively, this macro defines the 
  type of the header record in the data source.  This macro need only
  be defined if the data source has a header record.
  It defines a function used by the template
  program to generate the various function and type names derived from
  the name of the header record type, \ie{}, the type of the associated
  in-memory representation, mask, parse descriptor, read function,
  \etc{}
\item[\cd{PADS\_TY}]  Intuitively, this macro defines the 
  type of the repeated record in the data source, \ie{}, the type of
  the value to be accumulated.  This macro must be defined to use the
  accumulator template.  It defines a function used by the template
  program to generate the various function and type names derived from
  the name of the record type, \ie{}, the type of the associated
  in-memory representation, mask, parse descriptor, read function,
  \etc{}
\item[\cd{READ\_MASK}] This macro specifies the mask to use in reading
  the repeated record.  If not defined by the user, the template uses
  the value \cd{P\_CheckAndSet}.
\item[\cd{TIME\_IN\_FMT}] If defined, this macro sets the default
  input format for \cd{Ptime}.  See
  \secref{sec:library-customization-input-formats} for more
  information.
\item[\cd{TIME\_OUT\_FMT}] If defined, this macro sets the default
  output format for \cd{Ptime} and \cd{Ptime\_explicit}.  See
  \secref{sec:library-customization-output-formats} for more information.
\item[\cd{TIMESTAMP\_IN\_FMT}] If defined, this macro sets the default
  input format for \cd{Ptimestamp}.  See
  \secref{sec:library-customization-input-formats} for more
  information.
\item[\cd{TIMESTAMP\_OUT\_FMT}] If defined, this macro sets the default
  output format for the \pads{} types \cd{Ptimestamp} and \cd{Ptimestamp\_explicit}.  See
  \secref{sec:library-customization-output-formats} for more information.
\item[\cd{WSPACE\_OK}] If defined, this macro indicates that leading
  white space for variable-width ASCII integers is okay, as well as
  leading and trailing white space for fixed-width ASCII integers.

\end{description}
