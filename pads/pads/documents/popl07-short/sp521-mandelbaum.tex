%\documentclass[fleqn]{article}
%\documentclass[nocopyrightspace]{sigplanconf}
\documentclass{sigplanconf}

\usepackage{xspace,pads,amsmath,math-cmds,
            math-envs,inference-rules,times,
            verbatim,alltt,multicol,proof,url}

\usepackage{code} 
\usepackage{epsfig}

%\setlength{\oddsidemargin}{0in}
%\setlength{\evensidemargin}{0in}
%\setlength{\textwidth}{6.5in}
%\setlength{\textheight}{8.5in}

\begin{document}
\conferenceinfo{POPL'07} {January 17--19, 2007, Nice, France.}

\copyrightyear{2007}

\copyrightdata{1-59593-575-4/07/0001} 

\title{PADS/ML: A Functional Data Description Language}
\authorinfo{Yitzhak Mandelbaum$^\ast$, Kathleen Fisher$^\dagger$, David Walker$^\ast$, Mary
  Fernandez$^\dagger$, Artem Gleyzer$^\ast$}
{$^\ast$Princeton University \ \ \ \ \ \ \ $^\dagger$AT\&T Labs Research}
 {\mono{yitzhakm,dpw,agleyzer@CS.Princeton.EDU} \ \ \ \ \mono{kfisher,mff@research.att.com}}

\newcommand{\cut}[1]{}
\newcommand{\reminder}[1]{{\it #1 }}
\newcommand{\poplversion}[1]{#1}
\newcommand{\trversion}[1]{}
\newcommand{\edcom}[1]{\textbf{*** #1}}

\newcommand{\qqquad}{\quad \qquad}

\newcommand{\appref}[1]{Appendix~\ref{#1}}
\newcommand{\secref}[1]{Section~\ref{#1}}
\newcommand{\tblref}[1]{Table~\ref{#1}}
\newcommand{\figref}[1]{Figure~\ref{#1}}
\newcommand{\listingref}[1]{Listing~\ref{#1}}
%\newcommand{\pref}[1]{{page~\pageref{#1}}}

\newcommand{\eg}{{\em e.g.}}
\newcommand{\cf}{{\em cf.}}
\newcommand{\ie}{{\em i.e.}}
\newcommand{\etc}{{\em etc.\/}}
\newcommand{\naive}{na\"{\i}ve}
\newcommand{\role}{r\^{o}le}
\newcommand{\forte}{{fort\'{e}\/}}
\newcommand{\appr}{\~{}}

%\newcommand{\bftt}[1]{{\ttfamily\bfseries{}#1}}
\newcommand{\kw}[1]{\bftt{#1}}
\newcommand{\pads}{\textsc{pads}}
\newcommand{\padsc}{\textsc{pads/c}}
\newcommand{\ipads}{\textsc{ipads}}
\newcommand{\padsl}{\textsc{padsl}}
\newcommand{\blt}{\textsc{blt}}
\newcommand{\ddc}{\textsc{ddc}$^{\alpha}$}
\newcommand{\ddcold}{\textsc{ddc}}
\newcommand{\padsml}{\textsc{pads/ml}}
\newcommand{\padsmlbig}{\textsc{PADS/ML}}
\newcommand{\ddl}{\textsc{ddl}}
\newcommand{\C}{\textsc{c}}
\newcommand{\perl}{\textsc{perl}}
\newcommand{\ml}{\textsc{ml}}
\newcommand{\smlnj}{\textsc{sml/nj}}
\newcommand{\ocaml}{\textsc{o'caml}}
\newcommand{\ocamlbig}{\textsc{O'CAML}}
\newcommand{\java}{\textsc{java}}
\newcommand{\xml}{\textsc{xml}}
\newcommand{\xquery}{\textsc{xquery}}
\newcommand{\datascript}{\textsc{datascript}}
\newcommand{\packettypes}{\textsc{packettypes}}
\newcommand{\lex}{\textsc{Lex}}
\newcommand{\yacc}{\textsc{YACC}}
\newcommand{\erlang}{\textsc{Erlang}}

\newcommand{\dibbler}{Sirius}
\newcommand{\ningaui}{Altair}
\newcommand{\darkstar}{Regulus}

%% \newcommand{\IParray}[4]{{\tt Parray} \; #1 \; \[#2, #3, #4\]}

\newcommand{\figHeight}[4]{\begin{figure}[tb]
	\centerline{
	            \epsfig{file=#1,height=#4}}
	\caption{#2}
	\label{#3}
	\end{figure}}




\maketitle{}

\begin{abstract}  

  Massive amounts of useful data are stored and processed in
  \textit{ad hoc} formats for which common tools like parsers,
  printers, query engines and format converters are not readily 
  available.  In this paper,
  we explain the design and implementation of \padsml{},
  a new language and system that facilitates the
  generation of data processing tools for ad hoc formats.
  The \padsml{} design includes features such as
  dependent, polymorphic and recursive datatypes, which allow
  programmers to describe
  the syntax and semantics of ad hoc data in a concise,
  easy-to-read notation.
  The \padsml{} implementation compiles these descriptions
  into \ml{} structures and functors
  that include types for parsed data, functions for parsing
  and printing, and auxiliary support
  for user-specified, format-dependent and format-independent 
  tool generation.
%%  \edcom {sentence here about \padsml{} being based on theory?}
%% dpw -- couldn't think of anything useful to say.

\end{abstract}

\category{D.3.2}{Language Classifications}{Applicative (functional) languages}

\terms
Languages

\keywords
Data description languages, domain-specific languages, functional programming,
dependent types, ML, modules, parsing, printing


%\section{Introduction}
\label{sec:intro}

{\em Data description languages} are a class of domain specific
languages for specifying {\em ad hoc data formats}, from billing 
records to TCP packets to scientific data sets to server logs.  Examples 
of such languages include 
\bro~\cite{paxson:bro}, \datascript{}~\cite{gpce02}, \demeter~\cite{lieberherr+:class-dictionaries},
\packettypes{}~\cite{sigcomm00}, \padsc{}~\cite{fisher+:pads}, 
\padsml{}~\cite{mandelbaum+:padsml}  and
\xsugar~\cite{xsugar2005}, among others.  All of these languages
generate parsers from data descriptions.  In addition, and unlike
conventional parsing tools such as Lex and Yacc, many also automatically
generate auxiliary tools ranging from printers to \xml{} converters to
visitor libraries to visualization and editor tools.

In previous work, we developed the {\em Data Description Calculus}
(\ddcold{}), a calculus of simple, orthogonal type constructors,
designed to capture the core features of many existing type-based data
description languages~\cite{fisher+:next700ddl,fisher+:ddcjournal}.
This calculus had a multi-part denotational semantics that interpreted
the type constructors as (1) parsers the transform external bit
strings into internal data representations and {\em parse descriptors}
(representations of parser errors), (2) types for the data
representations and parse descriptors, and (3) types for the parsers
as a whole.  We proved that this multi-part semantics was coherent in
the sense that the generated parsers always have the expected types
and generate representations that satisfy an important {\em
canonical forms} lemma.

The \ddcold{} has been very useful already, helping us debug and
improve several aspects of \padsc{}~\cite{fisher+:pads}, and serving
as a guide for the design of \padsml{}~\cite{mandelbaum+:padsml}.
However, this initial work on the \ddcold{} told only a fraction of the
semantic story concerning data description languages.  As mentioned
above, many of these languages not only provide parsers, but
also other tools.  Amongst the most common auxiliary tools
are printers, as reliable communication between programs, either through
the file system or over the Web, depends upon both input (parsing) 
and output (printing).

In this work, we begin to address the limitations of
\ddcold{} by specifying a printing semantics for the
various features of the calculus.  We also
prove a collection of theorems for the new semantics that serve as
duals to our theorems concerning parsing.  This new printing semantics
has many of the same practical benefits as our older parsing 
semantics: We can
use it as a check against the correctness of our printer
implementations and as a guide for the
implementation of future data description languages.  


% First, we extend \ddcold{} with
% abstractions over types, which provides a basis for specifying the
% semantics of \padsml{}. In the process, we also improve upon the
% \ddcold{} theory by making a couple of subtle changes. For example, we
% are able to eliminate the complicated ``contractiveness'' constraint
% from our earlier work. Second, .

% The main practical benefit of the calculus has been as a guide for our
% implementation. Before working through the formal semantics, we
% struggled to disentangle the invariants related to polymorphism. After
% we had defined the calculus, we were able to implement type
% abstractions as \ocaml{} functors in approximately a week.  Our new
% printing semantics was also very important for helping us define and
% check the correctness of our printer implementation.  We hope the
% calculus will serve as a guide for implementations of \pads{} in
% other host languages.  

% In summary, this work makes the following key contributions:
% \begin{itemize}
% \item We simultaneously specify both a parsing and a printing semantics
%   for the \ddc{}, a calculus of polymorphic, dependent types.
% \item We prove that \ddc{} parsers and printers are type safe
%   and well-behaved as defined by a canonical forms theorem.
% \end{itemize}

In this extended abstract, we give an brief overview of the calculus,
it's dual semantics and their properties.  A companion technical
report contains a complete formal
specification~\cite{fisher+:popl-sub-long}.  In comparison to our
previous work on the \ddcold{} at POPL 06~\cite{mandelbaum+:padsml},
the calculus we present here has been streamlined in several subtle,
but useful ways.  It has also been improved through the addition of
polymorphic types.  We call this new polymorphic variant
\ddc{}.  These improvements and extensions, together with
proofs, appear in Mandelbaum's thesis~\cite{mandelbaum:thesis} and in
a recently submitted journal article~\cite{fisher+:ddcjournal}.
This abstract reviews the \ddc{} and extends all the previous 
work with a printing semantics and appropriate theorems.
To be more specific,
sections~\ref{sec:ddc-syntax} through \ref{sec:ddc-sem} present the
extended \ddc{} calculus, focusing on the semantics of polymorphic
types for parsing and the key elements of the printing semantics.
Then, \secref{sec:meta-theory} shows that both parsers and
printers in the \ddc{} are type correct and furthermore that parsers
produce pairs of parsed data and parse descriptors in {\em canonical
  form}, and that printers, given data in canonical form, print
successfully. We briefly discuss related work in \secref{sec:related}, and
conclude in \secref{sec:conc}.

%%% Local Variables: 
%%% mode: latex
%%% TeX-master: "paper"
%%% End: 


\section{Introduction}
\label{sec:intro}
\cut{
{\em
To do:
\begin{itemize}
\item update text to include description of tools generated using the framework.
\item add size column to data source table.
\end{itemize}
}}
%% WHAT IS AD HOC DATA?

An {\em ad hoc} data format is any semi-structured data format for which
parsing, querying, analysis, or transformation tools are not readily
available.  Despite the existence of standard
formats like \xml{}, ad hoc data sources are ubiquitous,
arising in industries as diverse as finance, health care,
transportation, and telecommunications as well as in scientific
domains, such as computational biology and physics.
\figref{figure:data-sources} summarizes a variety of such formats,
including ASCII, binary, and Cobol encodings, with both fixed and
variable-width records arranged in linear sequences and in tree-shaped
hierarchies.  Snippets of some of these data formats appear in \figref{fig:sample-data}.
Note that even a single format can exhibit a great deal of
syntactic variability.  For example, \figref{fig:sample-data}(c)
contains two records from a network-monitoring application.  
Each record has a different number of fields (delimited by `$|$') and
individual fields contain structured values (\eg{},
attribute-value pairs separated by `=' and delimited by `;').

Common characteristics of ad hoc data make it difficult to perform
even basic data-processing tasks.  To start, data analysts typically
have little control over the format of the data;  it
arrives ``as is,'' and the analysts can only thank the supplier,
not request a more convenient format.  The documentation accompanying
ad hoc data is often incomplete, inaccurate, or missing entirely,
which makes understanding the data format more difficult.
Managing the errors that frequently occur poses another challenge. Common errors include undocumented fields, corrupted or missing data, and multiple representations for missing values.  Sources of errors include
malfunctioning equipment, race conditions on log entry, the
% deleted this cite for race conditions:  ~\cite{wpp}
presence of non-standard values to indicate ``no data available,'' and
human error when entering data.  How to respond to errors is highly 
application-specific: Some need to halt processing and
alert a human operator; others can repair errors by consulting 
auxiliary sources; still others simply filter out erroneous values. 
In some cases, erroneous data is more important than error-free data; 
for example, it may signal where two systems are failing to communicate.
Unfortunately, writing code that reliably handles
both error-free and erroneous data is difficult and tedious.

\begin{figure}
\begin{center}
\scriptsize
\begin{tabular}{@{}|l|l|l|l|l|}
\hline
Name: Use                               & Representation     \\ \hline\hline
Gene Ontology (GO): & Variable-width     \\
Gene Product Information 	           & ASCII records      \\ \hline
SDSS/Reglens Data:  & Floating point numbers,  \\
Weak gravitational lensing analysis     & among others       \\ \hline
Web server logs (CLF):                  & Fixed-column       \\ 
Measuring web workloads                 & ASCII records      \\ \hline
AT\&T Call detail data:                 & Fixed-width        \\
Phone call fraud detection              & binary records     \\ \hline 
AT\&T billing data:                     & Cobol              \\ 
Monitoring billing process              &                    \\ \hline
Newick:                           & Fixed-width ASCII records  \\ 
Immune system response simulation              & in tree-shaped hierarchy   \\ \hline                                
OPRA:                                   & Mixed binary \& ASCII records  \\
Options-market transactions             & with data-dependent unions     \\ \hline
Palm PDA:                               & Mixed binary \& character  \\
Device synchronization                  & with data-dependent constraints \\ \hline
\end{tabular}
\caption{Selected ad hoc data sources.}
\label{figure:data-sources}
\end{center}
\end{figure}

\cut{
\begin{figure*}
\begin{center}
\scriptsize
\begin{tabular}{@{}|l|l|l|l|l|}
\hline
Name: Use                           & Representation    & Processing Problems \\ \hline\hline
Gene Ontology (GO)~\cite{geneontology}:                  & Variable-width    & White-space ambiguities \\
Gene Product Information 	      & ASCII records &  \\ \hline
SDSS/Reglens Data~\cite{mandelbaum+:reglens}:                & Floating point numbers, & Repeated multiplicative error \\
Weak gravitational lensing analysis   & among others & \\ \hline
Web server logs (CLF):                & Fixed-column      & Race conditions on log entry\\ 
Measuring web workloads               & ASCII records     & Unexpected values\\ \hline
AT\&T Call detail data:                          & Fixed-width       & Undocumented data\\
Phone call fraud detection            & binary records  & \\ \hline 
AT\&T billing data:                 & Cobol             &  Unexpected values\\ 
Monitoring billing process          &                   & Corrupted data feeds \\ \hline
Newick:   Immune                    & Fixed-width ASCII records & None \\ 
system response simulation          & in tree-shaped hierarchy &\\ \hline                                
OPRA:                               & Mixed binary \& ASCII records 
                                                       & 100-page informal \\
Options-market transactions         & with data-dependent unions & documentation \\ \hline
Palm PDA:                           & Mixed binary \& character & No high-level  \\
Device synchronization              & with data-dependent constraints & documentation available\\ \hline
\end{tabular}
\caption{Selected ad hoc data sources.}
\label{figure:data-sources}
\end{center}
\end{figure*}
}

% \begin{figure*}
% \begin{center}
% \begin{tabular}{@{}|l|l|l|}
% \hline
% \textbf{Name:} Use & Record Format (Size) 
% %& Size
%            & Common Errors \\ \hline\hline
% \textbf{Web server logs (CLF):}           & Fixed-column ASCII & Race conditions on log entry\\ 
% Measuring Web workloads  & ($\leq$12GB/week)  & Unexpected values\\ \hline
% \textbf{AT\&T provisioning data (\dibbler{}):} & Variable-width ASCII & Unexpected values \\ 
% Monitoring service activation  & (2.2GB/week) & Corrupted data feeds \\ \hline
% \textbf{Call detail:}                   & Fixed-width binary &  Undocumented data\\
% Fraud detection                         &   (\appr{}7GB/day) & \\ \hline 
% \textbf{AT\&T billing data (\ningaui{}):}      & Cobol      & Unexpected values\\ 
% Monitoring billing process  &  ($>$250GB/day) & Corrupted data feeds \\ \hline
% \textbf{IP backbone data (\darkstar{}):}  & ASCII & Multiple representations \\
% {Network Monitoring}       &  ($\ge$ 15 sources,\appr{}15 GB/day)  & of missing values \\
%           & & Undocumented data \\ \hline
% \textbf{Netflow:}               & Data-dependent number of & Missed packets\\ 
% {Network Monitoring}  & fixed-width binary records & \\ 
%                       & ($\ge$1Gigabit/second) & \\ \hline
% \textbf{Gene Ontology data:}    & Variable-width  & \\
% Gene product information & ASCII records & White-space ambiguities\\\hline
% \textbf{Newick data}              & Fixed-width ASCII & Manual entry errors \\
% Immune system response simulation & in tree-shaped hierarchy 
% & \\
% \hline
% \end{tabular}
% \normalsize
% \caption{Selected ad hoc data sources.}
% \label{figure:data-sources}
% \end{center}
% \end{figure*}

\begin{figure*}
  \centering
  \small

\begin{verbatim}
 2:3004092508||5001|dns1=abc.com;dns2=xyz.com|c=slow link;w=lost packets|INTERNATIONAL
 3:|3004097201|5074|dns1=bob.com;dns2=alice.com|src_addr=192.168.0.10; \
 dst_addr=192.168.23.10;start_time=1234567890;end_time=1234568000;cycle_time=17412|SPECIAL
\end{verbatim}  
%  \label{fig:darkstar-records1}
(a) Simplified Regulus network-monitoring data. 

\begin{verbatim}
0|1005022800
9153|9153|1|0|0|0|0||152268|LOC_6|0|FRDW1|DUO|LOC_CRTE|1001476800|LOC_OS_10|1001649601
9152|9151|1|9735551212|0||9085551212|07988|no_ii152272|EDTF_6|0|APRL1|DUO|10|1000295291
\end{verbatim}
%   \label{figure:dibbler-records}
(b) \dibbler{} data used to monitor billing in telecommunications industry.

\begin{verbatim}
(((erHomoC:0.28006,erCaelC:0.22089):0.40998, (erHomoA:0.32304,(erpCaelC:0.58815,((erHomoB: \
0.5807,erCaelB:0.23569):0.03586,erCaelA: 0.38272):0.06516):0.03492):0.14265):0.63594, \
(TRXHomo:0.65866,TRXSacch:0.38791):0.32147, TRXEcoli:0.57336)
\end{verbatim}
% \label{fig:newick}
(c) Newick data used to study immune system responses. 


  \caption{Snippets of a variety of ad hoc data formats. Each `$\backslash$' denotes a newline we inserted to improve readability. }
  \label{fig:sample-data}
\end{figure*}

\cut{
Surprisingly,
few meta-language tools, such as data-description languages or parser
generators, exist to assist in management of ad hoc data.  And
although ad hoc data sources are among the richest for database and
data mining researchers, they often ignore such sources as the work
necessary to clean and vet the data is prohibitively expensive.}

\subsection{\padsmlbig{}}

%% Slight change to emphasize generation of many tools.

\padsml{} is a domain-specific language designed to 
improve the productivity of data analysts, be they computational biologists, physicists, network administrators, healthcare providers or financial analysts.
To use the system, analysts describe their data in the \padsml{} language, capturing both the physical format of the data and any expected semantic constraints.  In return for this investment, analysts reap substantial rewards.
First of all, the description
serves as clear, compact, and formally-specified documentation of 
the data's structure and properties.  In addition, the \padsml{}
compiler can convert the description into a suite of robust, end-to-end
data processing tools and libraries specialized
to the format.  As the analysts' data sources evolve over time,
they can simply update the high-level descriptions
and recompile to produce updated tools.


The type structure of modern functional programming languages inspired the design of the \padsml{} language.  
Specifically, \padsml{} provides dependent, polymorphic recursive datatypes, layered on top of a rich collection of base types, to specify the syntactic structure and semantic properties of data
formats.  Together, these features enable analysts to write concise,
complete, and reusable descriptions of their data.  
We describe the \padsml{} language using examples from several domains
in \secref{sec:padsml-overview}.

We have implemented \padsml{} by compiling descriptions into
\ocaml{} code.  We use a
``types as modules'' implementation strategy in which each \padsml{} type
becomes a module and each \padsml{} type constructor becomes a functor. 
We chose \ml{} as the host language because we believe that 
functional languages lend themselves to data processing tasks more readily than imperative languages such as \C{} or \java{}.  In particular, constructs such as pattern matching and higher-order functions make expressing data transformations particularly convenient. \secref{sec:padsml-impl} describes our ``types as modules''
strategy and shows how \padsml{}-generated modules together
with functional \ocaml{} code can concisely express
common
data-processing tasks such as filtering errors and format transformation.

A key benefit of our approach is the high return-on-investment that
analysts can derive from describing their data in \padsml{}.  In particular, \padsml{} makes it possible to produce automatically a collection of data analysis and processing tools from each description.   
As a start, the \padsml{} compiler generates from each description a parser and a printer for the associated data source.  The parser maps raw 
data into two data
structures: a canonical \textit{representation} of the parsed data and
a \textit{parse descriptor}, a meta-data object detailing properties
of the corresponding data representation.  Parse descriptors provide
applications with programmatic access to errors detected during
parsing.  The printer inverts the process, mapping internal data structures
and their corresponding parse descriptors back into raw data.

In addition to generating parsers and printers, our framework permits
developers to add {\em format-independent} tools without modifying the \padsml{} compiler by specifying \textit{tool generators}.  Such generators need only match a generic interface, specified as an \ml{} signature.  
Correspondingly, for each \padsml{} description, the \padsml{} compiler generates a meta-tool (a functor)
that takes a tool generator and specializes it for use with
the particular description.  \secref{sec:gen-tool} describes the
tool framework and gives examples of three format-independent 
tools that we have
implemented: a data printer useful for description debugging,
an accumulator that keeps track of error information for
each type in a data source, and a formatter that maps data into XML.

To define the semantics of \padsml{}, we extended our earlier work on
the Data Description Calculus (\ddcold{})~\cite{fisher+:next700ddl} to
account for \padsml{}'s polymorphic types.  In the process, we
simplified the original presentation of the parser semantics
substantially, particularly for recursive types.  In addition, we
extended the theory to give a printing semantics. We used this new
semantics to guide the \padsml{} implementation of printing.  We 
also proved a {\em canonical forms} theorem, which states 
that the generated parsers produce well-typed, well-behaved canonical
results, and, conversely, that printers operate correctly on the 
appropriate canonical
inputs. A full treatment of the extended calculus appears in
Mandelbaum's Ph.D. thesis~\cite{mandelbaum-thesis}, while an overview of
the calculus and printing semantics, as well as the associated
metatheory, can be found in our companion technical
report~\cite{fisher+:popl-sub-long}.


\cut{\textbf{***THEN compare to \padsc{} in one paragraph only}}
\padsml{} has evolved from previous work on
\padsc{}~\footnote{We henceforth call the original
\pads{} language \padsc{}.}~\cite{fisher+:pads}, but
\padsml{} differs from \padsc{} in three significant ways.  First, it
is targeted at the \ml{} family of languages.  Using \ml{} as the host
language simplifies many data processing tasks, such as filtering and normalization, 
which benefit from \ml{}'s pattern matching constructs and
high level of abstraction.  Second, unlike \padsc{} types, \padsml{}
types may be parameterized by other types, resulting in more concise
and elegant descriptions through code reuse.  \ml{}-style datatypes and 
anonymous nested tuples also help improve readability by making descriptions more compact.  Third, \padsml{} provides
significantly better support for the development of new tool generators.
In particular, \padsml{} provides a generic interface against which
tool generators can be written.  In \padsc{}, the compiler itself
generates all tools, and, therefore, developing a new tool generator
requires understanding and modifying the compiler.
Mandelbaum's Ph.D. thesis~\cite{mandelbaum-thesis}
contains a full discussion of related work.


\cut{
\textbf{***Then give observations.}
\textbf{Actually, this observation should go in the conclusion.  It's
  distracting here.}
Our ``types as modules'' implementation strategy encountered the
  limits of the \ocaml{} module system in multiple ways. It therefore
  provides a natural, well-motivated challenge example for
  functional-programming researchers in type-directed programming and
  advanced module design.

Therefore, the combination of
  \padsml{} and \ocaml{} is a significant step towards a unified
  language for data description, transformation, and analysis.
}

\cut{\textbf{***Then, if still necessary give a pithy list of contributions}}

In summary, this research makes the following contributions:
\begin{itemize}
\item We have designed and implemented \padsml{}, a novel
data-description language that includes 
dependent, polymorphic, recursive datatypes.  This design
allows data analysts to express the syntactic
structure and semantic properties of data formats from numerous
application domains in a concise, elegant, and easy-to-read notation.  
\item Our \padsml{} implementation employs an effective and
general ``types as modules'' compilation strategy
that produces robust parser and printer functions
as well as auxiliary support for user-specified tool generation.
Our implementation is available
at \url{http://www.padsproj.org/padsml/}.
\end{itemize}


%%% Local Variables: 
%%% mode: latex
%%% TeX-master: "paper"
%%% End: 


% \section{Describing Data in \padsmlbig{}}
\label{sec:padsml-overview}
{\em
To do:
\begin{itemize}
\item Check the details of the Kleinstein reference.
\end{itemize}
}
A \padsml{} description specifies the physical layout and semantic
properties of an ad hoc data source.  These descriptions are composed
of types: base types describe atomic data, while structured types
describe compound data built from simpler pieces.  Examples of base
types include ASCII-encoded, 8-bit unsigned integers (\cd{Puint8}) and
32-bit signed integers (\cd{Pint32}), binary 32-bit integers (\cd{Pbint32}),
dates (\cd{Pdate}), strings (\cd{Pstring}), zip codes (\cd{Pzip}),
phone numbers (\cd{Pphone}), and IP addresses (\cd{Pip}).  Semantic
conditions for such base types include checking 
that the resulting number fits in the indicated space, \ie, 16-bits
for \cd{Pint16}.

Base types may be parameterized by \ml{} values.  This mechanism
reduces the number of built-in base types and permits base types to
depend on values in the parsed data.  For example, the base type
\cd{Puint16_FW(3)} specifies an unsigned two byte integer physically
represented by exactly three characters, and the base type
\cd{Pstring} takes an argument indicating the \textit{terminator
character}, \ie{}, the character in the source that immediately
follows the string.

To describe more complex data, \padsml{} provides a collection of type
constructors derived from the type structure of functional programming
languages like Haskell and ML.  We explain these structured types in
the following subsections using examples drawn from data sources we
have encountered in practice.

% Readers eager to see the complete syntax
% of types should flip forward to Appendix~\ref{app:syntax-dd}.

\subsection{Simple Structured Types}


\begin{figure*}
{\scriptsize
\begin{verbatim}
0|1005022800
9152|9151|1|9735551212|0||9085551212|07988|no_ii152272|EDTF_6|0|APRL1|DUO|10|1000295291
9153|9153|1|0|0|0|0||152268|LOC_6|0|FRDW1|DUO|LOC_CRTE|1001476800|LOC_OS_10|1001649601
\end{verbatim}
}
  \caption{Miniscule example of \dibbler{} data.}
  \label{figure:dibbler-records}
\end{figure*}

The bread and butter of a \padsml{} description are the simple
structured types: tuples and records for specifying ordered data;
lists for specifying homogeneous sequences of data; sum types for
specifying alternatives; and singletons for specifying the occurrence
of literal characters in the data.  We describe each of these
constructs as applied to the \dibbler{} data presented in
\figref{figure:dibbler-records}.

\dibbler{} data summarizes orders for phone service placed with AT\&T.
Each \dibbler{} data file starts with a timestamp followed by one
record per phone service order.  Each order consists of a header and a
sequence of events.  The header has 13 pipe separated fields: the
order number, AT\&T's internal order number, the order version, four
different telephone numbers associated with the order, the zip code of
the order, a billing identifier, the order type, a measure of the
complexity of the order, an unused field, and the source of the order
data.  Many of these fields are optional, in which case nothing
appears between the pipe characters.  The billing identifier may not
be available at the time of processing, in which case the system
generates a unique identifier, and prefixes this value with the string
``no\_ii'' to indicate that the number was generated. The event
sequence represents the various states a service order goes through;
it is represented as a new-line terminated, pipe separated list of
state, timestamp pairs.  There are over 400 distinct states that an
order may go through during provisioning.  The sequence is sorted in
order of increasing timestamps.  From this description, it is apparent
that English is a poor language for describing data formats!

\begin{figure}
\begin{code}\scriptsize
\kw{ptype} Semicolon = Pcharlit(';')
\kw{ptype} Vbar = Pcharlit('|')
\mbox{}
\kw{ptype} Pn\_t = Puint64
\kw{ptype} Pzip = Puint32
\mbox{}
\kw{ptype} Summary\_header = "0|" * Puint32 * Peor
\mbox{}
\kw{pdatatype} Dib\_ramp = 
  Ramp of Pint 
| GenRamp of "no\_ii" * Pint
\mbox{}
\kw{ptype} Order\_header = \{ 
    order\_num : Pint;  
'|'; att\_order\_num : [i:Pint | i < order\_num];  
'|'; ord\_version : Pint;  
'|'; service\_tn : Pn\_t Popt;
'|'; billing\_tn : Pn\_t Popt;  
'|'; nlp\_service\_tn : Pn\_t Popt;  
'|'; nlp\_billing\_tn : Pn\_t Popt;  
'|'; zip\_code : Pzip Popt;  
'|'; ramp : Dib\_ramp;  
'|'; order\_sort : Pstring('|');  
'|'; order\_details : Pint;
'|'; unused : Pstring('|');  
'|'; stream : Pstring('|'); 
'|'
\} 
\mbox{}
\kw{ptype} Event  = Pstring('|') * '|' * Puint32
\kw{ptype} Events = (Event,Vbar,Peor) Plist
\mbox{}
\kw{ptype} Order  = Order\_header * Events
\kw{ptype} Orders = (Order, Peor, Peof) Plist
\mbox{}
\kw{ptype} Source = Summary\_header * Orders\end{code}

  \caption{\padsml{} description for \dibbler{} provisioning data.}
  \label{figure:sirius_pml}
\end{figure}

\figref{figure:sirius_pml} contains the \padsml{} description for the
\dibbler{} data format.  The description is a sequence of type
definitions.  Type definitions precede uses, therefore the description
should be read bottom up.
The type \cd{Source} (Line~27) describes a complete \dibbler{} data
file and denotes an ordered tuple containing a
\cd{Summary\_header} value followed by an \cd{Orders} value.
Other tuple types are defined on Lines~1, 23, and~25.

The type \cd{Orders} (Line~26) uses the list type constructor
\cd{Plist} to describe a homogenous sequence of values in a data
source.  The \cd{Plist} constructor takes three parameters: on the
left, the type of elements in the list; on the right, a literal
\emph{separator} that delimits elements in the list, and a literal
\emph{terminator}.  In this example, the type \cd{Orders} is a list of
\cd{Order} elements, separated by a newline, and terminated by
\cd{peof}, a special literal that describes the \emph{end-of-file
  marker}.  Similarly, the \cd{Events} type (Line~24) denotes a
sequence of \cd{Event} values separated by vertical bars and
terminated by a newline.

Literal characters in type expressions denote singleton types.  For
example, the \cd{Event} type (Line~23) is a string terminated by a
vertical bar, followed by a vertical bar, followed by a timestamp.  The
singleton type \cd{'|'} means that the data source must contain the
character \cd{'|'} at this point in the input stream.  String,
character, and integer literals can be embedded in a description and
are interpreted as singleton types, \eg{}, the singleton type
\cd{"0|"} in the \cd{Summary\_header} type (Line~1) 
denotes the string literal \cd{"0|"}.

The type \cd{Order\_header} is a record type, \ie{}, a tuple type in
which each field may have an associated name.  The named field
\cd{att\_order\_num} (Line~9) illustrates two other features of
\padsml: dependencies and constraints.  Here, \cd{att\_order\_num}
depends on the previous field \cd{order\_num} and is constrained to be
less than that value.  In practice, constraints may be complex, have
multiple dependencies, and can specify, for example, the sorted order
of records in a sequence.  Constrained types have the form \cd{[x:T |
e]} where \cd{e} is an arbitrary pure boolean expression.  Data
satisfies this description if it satisfies \cd{T} and boolean \cd{e}
evaluates to true when the parsed representation of the data is
substituted for \cd{x}.  If the boolean expression evaluates to false,
the data contains a \textit{semantic} error.

The datatype definition of \cd{Dib\_ramp} (Line~3) specifies two
alternatives for a data fragment, either one integer or the fixed
string \cd{"no\_ii"} followed by one integer.  The order of
alternatives is significant, that is, the parser attempts to parse the
first alternative and only if it fails, it attempts to parse the
second alternative.  This semantics differs from similar constructs in
regular expressions and context-free grammars, which
non-deterministically choose between alternatives.
\cut{Fortunately, we have yet to come across an ad hoc data
source where we wish we had nondeterministic choice.\footnote{\padsml{}
  can recognize string data based on regular expressions.
  Non-determinism here has been useful, but as it has been confined to
  parsing elements of the \cd{Pstring} base type, it has had no impact
  on the overall parsing algorithm.}
}
\begin{figure}
\begin{code}\scriptsize
\kw{ptype} Semicolon = Pcharlit(';')
\kw{ptype} Rparen= Pcharlit(')')
\mbox{}
\kw{ptype} Entry = \{
  name : Pstring(':'); ':'; dist : Pfloat32
\}
\mbox{}
\kw{pdatatype} Tree =
  Interior of '(' * (Tree, Semicolon, Rparen) Plist * ')'
| Leaf of Entry\end{code}

%%% Local Variables: 
%%% mode: latex
%%% TeX-master: "paper"
%%% End: 

Tiny fragment of Newick data:

{
\begin{verbatim}
(((erHomoC:0.28006,erCaelC:0.22089):0.40998,
(erHomoA:0.32304,(erpCaelC:0.58815,((erHomoB:0.5807,
erCaelB:0.23569):0.03586,erCaelA:0.38272):0.06516):
0.03492):0.14265):0.63594,(TRXHomo:0.65866,TRXSacch:
0.38791):0.32147,TRXEcoli:0.57336)
\end{verbatim}
}
  \caption{Simplified tree-shaped Newick data}
  \label{fig:newick}
\end{figure}

\subsection{Recursive Types}

\padsml{} can describe data sources with recursive structure.  One
example of recursive data is the Newick format, a flat representation
of trees used by biologists~\cite{newick} that uses properly nested
parentheses to specify a tree's hierarchy.  A leaf node is a string
label followed by a colon and a number.  An interior node contains a
sequence of children nodes, delimited by parentheses, followed by a
colon and a number.  The numbers represent the ``distance'' that
separates a child node from its parent.

\figref{fig:newick} contains a concise and elegant description of Newick
and a small fragment of example data.  In this example, the string
labels are gene names and the distance denotes the number of mutations
that occur in the antibody receptor genes of B lymphocytes.  This
data is provided by Steven Kleinstein, program coordinator of
Princeton's Picasso project for interdisciplinary research in
computational sciences.  \edcom{Is this true anymore? Should it be in
  past tense? Should it be here at all?}. Kleinstein is building a simulator to study
the proliferation of B lymphocytes during an immune response.

\begin{figure}
  \centering
  \small
\begin{verbatim}
 2:3004092508||5001|dns1=abc.com;dns2=xyz.com|c=slow link;w=lost packets|INTERNATIONAL
 3:|3004097201|5074|dns1=bob.com;dns2=alice.com|src_addr=192.168.0.10;
 dst_addr=192.168.23.10;start_time=1234567890;end_time=1234568000;cycle_time=17412|SPECIAL
\end{verbatim}  
  \caption{Simplified network-monitoring data. We inserted the newline
    after the ';' to improve legibility.}
  \label{fig:darkstar-records1}
\end{figure}

\subsection{Polymorphic Types}

The last key feature of \padsml{} is parameterized types.  Type
parameterization makes descriptions more concise and allows
programmers to define convenient libraries of reusable components. The
description in Figure~\ref{fig:darkstar-ml} illustrates types
parameterized by both types and values.  This description specifies
the format of alarm data recorded by a network-link monitor used by
the \darkstar{} project at AT\&T.  Sample \darkstar{} data is in
\figref{fig:darkstar-records1}.

Each alarm signals some problem with a network link; At this point, we
could write an English description of the \darkstar{} data. However,
English is a terrible specification language for ad hoc data, both
verbose and imprecise.  Hence, we dispense with the informal English
and head straight for the \padsml{} description, shown in
\figref{fig:darkstar-ml}.

\begin{figure}
  \centering
  \begin{code}\scriptsize
(* Pstring terminated by ';' or '|'. *)
\kw{ptype} SVString = Pstring\_SE("/;|\\\\|/")
\mbox{}
(* Generic name value pair. Accepts predicate 
   to validate name as argument.� *)
\kw{ptype} (Alpha) Pnvp(p : string -> bool) =
      \{ name : [name : Pstring('=') | p name]; 
            '='; 
       value : Alpha \}
\mbox{}
(* Name value pair with name specified. *)
\kw{ptype} (Alpha) Nvp(name:string) = 
   Alpha Pnvp(fun s -> s = name)
\mbox{}
(* Name value pair with any name. *)
\kw{ptype} Nvp\_a = SVString Pnvp(fun \_ -> true)
\mbox{}
\kw{ptype} Details = \{
      source      : Pip Nvp("src\_addr");
';';  dest        : Pip Nvp("dest\_addr");
';';  start\_time  : Ptimestamp Nvp("start\_time");
';';  end\_time    : Ptimestamp Nvp("end\_time");
';';  cycle\_time  : Puint32 Nvp("cycle\_time")
\}

\kw{pdatatype} Info(alarm\_code : int) =
  match alarm\_code with
    5074 -> Details of Details
  | \_    -> Generic of Nvp\_a Plist(';','|')

\kw{pdatatype} Service = 
    DOMESTIC      of "DOMESTIC" 
  | INTERNATIONAL of "INTERNATIONAL" 
  | SPECIAL       of "SPECIAL"

\kw{ptype} Alarm = \{
       alarm    : [i : Puint32 | i = 2 or i = 3];
 ':';  start    : Ptimestamp Popt;
 '|';  clear    : Ptimestamp Popt;
 '|';  code     : Puint32;
 '|';  src\_dns  : SVString Nvp("dns1");
 ';';  dest\_dns : SVString Nvp("dns2");
 '|';  info     : Info(code);
 '|';  service  : Service
\} 

\kw{ptype} Source = Alarm Plist('\\n',\kw{peof})\end{code}
%
%\kw{let} checkCorr ra = ...
%\kw{ptype} Alarm = [x:Raw\_alarm | checkCorr x]



  \caption{Description of \darkstar{} data.}
  \label{fig:darkstar-ml}
\end{figure}

This data format has several variants of name-value pairs. The
\padsc{} description of this format (shown in~\appref{app:regulus-padsc})
must define a different type for each variant. In contrast, the
polymorphic types of \padsml{} allow us to define the type \cd{Pnvp}
(Line~2), which takes both type and value parameters to encode all the
variants. Type parameters appear to the left of the type name, as is
customary in \ml{}.  Value parameters and their \ml{} types appear to
the right of the type name.  In the definition of \cd{Pnvp}, there is
one type parameter named \cd{Alpha} and one value parameter named
\cd{p}.  Informally, \cd{Alpha Pnvp(p)} is a name-value pair where the
value is described by \cd{Alpha} and the name must satisfy the
predicate \cd{p}.

The \cd{Nvp} type (Lines~6--7) is defined in terms of \cd{Pnvp}, in
which the name argument must match a particular string, but the type
parameter remains abstract.  The \cd{Nvp\_a} (Line~8) is also defined
in terms of \cd{Pnvp}.  In this case, any name is permitted, but the
value must be a \cd{SVString}, \ie{}, string terminated by a semicolon
or vertical bar.  Elsewhere in the description, the type parameter to
\cd{Nvp} is instantiated with IP addresses, timestamps, and integers.

% The source type is an array of \cd{alarm}s, where each alarm is a
% \cd{raw\_alarm}, constrained to ensure that the alarm number is
% properly correlated with the timestamps.  We check this correlation
% with the function \cd{checkCorr}.  The type \cd{raw\_alarm} closely
% follows the description above. We highlight a few important features.
% First, we note that the type of the field \cd{info} depends on the
% alarm code, reflecting the text above. More interestingly, the type
% \cd{info} is implemented with a switched datatype, deciding how to
% parse based on the parameter \cd{alarm\_code}.  Next, we note that the
% description includes five different types of name-value pairs. We take
% advantage of both the type and value parameterization of types to
% encode all of these pair types based on one common description,
% \cd{pnvp}. This type is polymorphic in the type of the value and takes
% an arbitrary constraint \cd{c} as an argument. The type \cd{nvp} is
% polymorphic in the type of the value, but takes the expected name of
% the string as an argument. 

The \darkstar{} description also illustrates the use of
\textit{switched} datatypes.  A switched datatype selects a variant
based on the value of a user-specified \ocaml{} expression, which
typically references parsed data from earlier in the data source.  For
example, the switched datatype \cd{Info} (Lines~16--19) chooses a
variant based on the value of its \cd{alarm\_code} parameter.  More
specifically, if the alarm code is \cd{5074}, the format specification
given by the \cd{Details} constructor will be used to parse the
current data.  Otherwise, the format given by the \cd{Generic}
constructor will be used to parse the current data.

The last construct in the \darkstar{} description is the type
qualifier \cd{omit}.  In the \cd{Service} datatype (Lines~20--23),
\cd{omit} specifies that the parsed string literal should be omitted
in the internal data representation, because the literal can be
determined by the datatype constructor.

\cut{We can do this because we can discern from the
datatype constructor which string was found in the data source.}

%%% Local Variables: 
%%% mode: latex
%%% TeX-master: "../thesis.tex"
%%% End: 

\section{Describing Data in \padsmlbig{}}
\label{sec:padsml-overview}

A \padsml{} description specifies the physical layout and semantic
properties of an ad hoc data source.  These descriptions are composed
of types: base types describe atomic data, while structured types
describe compound data built from simpler pieces.  Examples of base
types include ASCII-encoded, 8-bit unsigned integers (\cd{Puint8}) and
32-bit signed integers (\cd{Pint32}), binary 32-bit integers (\cd{Pbint32}),
dates (\cd{Pdate}), strings (\cd{Pstring}), zip codes (\cd{Pzip}),
phone numbers (\cd{Pphone}), and IP addresses (\cd{Pip}).  Semantic
conditions for such base types include checking 
that the resulting number fits in the indicated space, \ie, 16-bits
for \cd{Pint16}.

Base types may be parameterized by \ml{} values.  This mechanism
reduces the number of built-in base types and permits base types to
depend on values in the parsed data.  For example, the base type
\cd{Puint16_FW(3)} specifies an unsigned two byte integer physically
represented by exactly three characters, and the base type
\cd{Pstring} takes an argument indicating the \textit{terminator
character}, \ie{}, the character in the source that
follows the string.

To describe more complex data, \padsml{} provides a collection of type
constructors derived from the type structure of functional programming
languages like Haskell and ML.  We explain these structured types in
the following subsections using examples drawn from data sources we
have encountered in practice.

% Readers eager to see the complete syntax
% of types should flip forward to Appendix~\ref{app:syntax-dd}.

\subsection{Simple Structured Types}

\cut{
\begin{figure*}
{\scriptsize
\begin{verbatim}
0|1005022800
9152|9151|1|9735551212|0||9085551212|07988|no_ii152272|EDTF_6|0|APRL1|DUO|10|1000295291
9153|9153|1|0|0|0|0||152268|LOC_6|0|FRDW1|DUO|LOC_CRTE|1001476800|LOC_OS_10|1001649601
\end{verbatim}
}
  \caption{Miniscule example of \dibbler{} data.}
  \label{figure:dibbler-records}
\end{figure*}
}

The bread and butter of a \padsml{} description are the simple
structured types: tuples and records for specifying ordered data,
lists for specifying homogeneous sequences of data, sum types for
specifying alternatives, and singletons for specifying the occurrence
of literal characters in the data.  We describe each of these
constructs as applied to the \dibbler{} data presented in
\figref{fig:sample-data}(b).

\dibbler{} data summarizes orders for phone service placed with AT\&T.
Each \dibbler{} data file starts with a timestamp followed by one
record per phone service order.  Each order consists of a header and a
sequence of events.  The header has 13 pipe separated fields: the
order number, AT\&T's internal order number, the order version, four
different telephone numbers associated with the order, the zip code of
the order, a billing identifier, the order type, a measure of the
complexity of the order, an unused field, and the source of the order
data.  Many of these fields are optional, in which case nothing
appears between the pipe characters.  The billing identifier may not
be available at the time of processing, in which case the system
generates a unique identifier, and prefixes this value with the string
``no\_ii'' to indicate that the number was generated. The event
sequence represents the various states a service order goes through;
it is represented as a newline-terminated, pipe-separated list of
state, timestamp pairs.  There are over 400 distinct states that an
order may go through during provisioning.  The sequence is sorted in
order of increasing timestamps.  Clearly
English is a poor language for describing data formats!

\begin{figure}
\begin{code}\scriptsize
\kw{ptype} Semicolon = Pcharlit(';')
\kw{ptype} Vbar = Pcharlit('|')
\mbox{}
\kw{ptype} Pn\_t = Puint64
\kw{ptype} Pzip = Puint32
\mbox{}
\kw{ptype} Summary\_header = "0|" * Puint32 * Peor
\mbox{}
\kw{pdatatype} Dib\_ramp = 
  Ramp of Pint 
| GenRamp of "no\_ii" * Pint
\mbox{}
\kw{ptype} Order\_header = \{ 
    order\_num : Pint;  
'|'; att\_order\_num : [i:Pint | i < order\_num];  
'|'; ord\_version : Pint;  
'|'; service\_tn : Pn\_t Popt;
'|'; billing\_tn : Pn\_t Popt;  
'|'; nlp\_service\_tn : Pn\_t Popt;  
'|'; nlp\_billing\_tn : Pn\_t Popt;  
'|'; zip\_code : Pzip Popt;  
'|'; ramp : Dib\_ramp;  
'|'; order\_sort : Pstring('|');  
'|'; order\_details : Pint;
'|'; unused : Pstring('|');  
'|'; stream : Pstring('|'); 
'|'
\} 
\mbox{}
\kw{ptype} Event  = Pstring('|') * '|' * Puint32
\kw{ptype} Events = (Event,Vbar,Peor) Plist
\mbox{}
\kw{ptype} Order  = Order\_header * Events
\kw{ptype} Orders = (Order, Peor, Peof) Plist
\mbox{}
\kw{ptype} Source = Summary\_header * Orders\end{code}

  \caption{\padsml{} description for \dibbler{} provisioning data.}
  \label{figure:sirius_pml}
\end{figure}

\figref{figure:sirius_pml} contains the \padsml{} description for the
\dibbler{} data format.  The description is a sequence of type
definitions.  Type definitions precede uses, therefore the description
should be read bottom up.
The type \cd{Source} describes a complete \dibbler{} data
file and denotes an ordered tuple containing a
\cd{Summary\_header} value followed by an \cd{Orders} value.

The type \cd{Orders} uses the list type constructor
\cd{Plist} to describe a homogenous sequence of values in a data
source.  The \cd{Plist} constructor takes three parameters: on the
left, the type of elements in the list; on the right, a literal
\emph{separator} that separates elements in the list and a literal
\emph{terminator} that marks the end of the list.  
In this example, the type \cd{Orders} is a list of
\cd{Order} elements, separated by a newline, and terminated by
\cd{peof}, a special literal that describes the \emph{end-of-file
  marker}.  Similarly, the \cd{Events} type denotes a
sequence of \cd{Event} values separated by vertical bars and
terminated by a newline.

String, character, and integer literals can be embedded in a
description and are interpreted as singleton types.  For example, the
\cd{Event} type is a string terminated by a vertical bar, followed by
a vertical bar, followed by a timestamp.  The singleton type \cd{'|'}
means that the data source must contain the character \cd{'|'} at this
point in the input stream. Correspondingly, the generated parser reads
\cd{'|'} and the generated printer writes \cd{'|'}.  These literals do
not appear in the generated data representations.

The type \cd{Order\_header} is a record type, \ie{}, a tuple type in
which each field may have an associated name.  The named field
\cd{att\_order\_num} illustrates two other features of
\padsml: dependencies and constraints.  Here, \cd{att\_order\_num}
depends on the previous field \cd{order\_num} and is constrained to be
less than that value.  In practice, constraints may be complex, have
multiple dependencies, and can specify, for example, the sorted order
of records in a sequence.  Constrained types have the form \cd{[x:T |
e]} where \cd{e} is an arbitrary pure boolean expression.  Data
satisfies this description if it satisfies \cd{T} and \cd{e}
evaluates to \cd{true} when the parsed representation of the data is
substituted for \cd{x}.  If the boolean expression evaluates to false,
the data contains a \textit{semantic} error.

The datatype \cd{Dib\_ramp} specifies two
alternatives for a data fragment: either one integer or the fixed
string \cd{"no\_ii"} followed by one integer.  The order of
alternatives is significant, that is, the parser attempts to parse the
first alternative and only if it fails, it attempts to parse the
second alternative.  This semantics differs from similar constructs in
regular expressions and context-free grammars, which
non-deterministically choose between alternatives.
\cut{Fortunately, we have yet to come across an ad hoc data
source where we wish we had nondeterministic choice.\footnote{\padsml{}
  can recognize string data based on regular expressions.
  Non-determinism here has been useful, but as it has been confined to
  parsing elements of the \cd{Pstring} base type, it has had no impact
  on the overall parsing algorithm.}
}
\cut{
\begin{figure}
\begin{code}\scriptsize
\kw{ptype} Semicolon = Pcharlit(';')
\kw{ptype} Rparen= Pcharlit(')')
\mbox{}
\kw{ptype} Entry = \{
  name : Pstring(':'); ':'; dist : Pfloat32
\}
\mbox{}
\kw{pdatatype} Tree =
  Interior of '(' * (Tree, Semicolon, Rparen) Plist * ')'
| Leaf of Entry\end{code}

%%% Local Variables: 
%%% mode: latex
%%% TeX-master: "paper"
%%% End: 

Tiny fragment of Newick Standard data:

{
\begin{verbatim}
(((erHomoC:0.28006,erCaelC:0.22089):0.40998,
(erHomoA:0.32304,(erpCaelC:0.58815,((erHomoB:
0.5807,erCaelB:0.23569):0.03586,erCaelA:
0.38272):0.06516):0.03492):0.14265):0.63594,
(TRXHomo:0.65866,TRXSacch:0.38791):0.32147,
TRXEcoli:0.57336)
\end{verbatim}
}
  \caption{Simplified tree-shaped Newick Standard data.  Newlines
     inserted to improve legibility.}
  \label{fig:newick}
\end{figure}
}


\subsection{Recursive Types}

\padsml{} can describe data sources with recursive structure.  An
example of such data is the Newick Standard format, a flat representation
of trees used by biologists~\cite{newick}.
Example Newick Standard data provided by Steven Kleinstein appears in
\figref{fig:sample-data}(c). 
The format uses properly nested
parentheses to specify a tree hierarchy.  A leaf node is a string
label followed by a colon and a number.  An interior node contains a
sequence of children nodes, delimited by parentheses, followed by a
colon and a number.  The numbers represent the ``distance'' that
separates a child node from its parent. 
In this example,  the string labels are gene names and the distances denotes the number of mutations that occur in the antibody receptor genes of B lymphocytes. \cut{Kleinstein uses this kind of data to study
the proliferation of B lymphocytes during an immune response.}
The following \padsml{} code 
describes this format:
\begin{code}\scriptsize
\kw{ptype} Semicolon = Pcharlit(';')
\kw{ptype} Rparen= Pcharlit(')')
\mbox{}
\kw{ptype} Entry = \{
  name : Pstring(':'); ':'; dist : Pfloat32
\}
\mbox{}
\kw{pdatatype} Tree =
  Interior of '(' * (Tree, Semicolon, Rparen) Plist * ')'
| Leaf of Entry\end{code}

%%% Local Variables: 
%%% mode: latex
%%% TeX-master: "paper"
%%% End: 

 
\cut{
\begin{figure}
  \centering
  \small
\begin{verbatim}
2:3004092508||5001|dns1=abc.com;dns2=xyz.com|
c=slow link;w=lost packets|INTERNATIONAL
3:|3004097201|5074|dns1=bob.com;dns2=alice.com|
src_addr=192.168.0.10;dst_addr=192.168.23.10;
start_time=1234567890;end_time=1234568000;
cycle_time=17412|SPECIAL
\end{verbatim}  
  \caption{Simplified network-monitoring data. Newlines
     inserted to improve legibility.}
  \label{fig:darkstar-records1}
\end{figure}
}

\subsection{Polymorphic Types and Advanced Datatypes}

Polymorphic types enable more concise descriptions and allow
programmers to define convenient libraries of reusable descriptions. The
description in \figref{fig:darkstar-ml} illustrates types
parameterized by both types and values.  It specifies
the format of alarm data recorded by a network-link monitor used in
the \darkstar{} project at AT\&T.  \figref{fig:sample-data}(a) contains corresponding example data. We describe the format in tandem with describing its \padsml{} description.
\begin{figure}
  \centering
  \begin{code}\scriptsize
(* Pstring terminated by ';' or '|'. *)
\kw{ptype} SVString = Pstring\_SE("/;|\\\\|/")
\mbox{}
(* Generic name value pair. Accepts predicate 
   to validate name as argument.� *)
\kw{ptype} (Alpha) Pnvp(p : string -> bool) =
      \{ name : [name : Pstring('=') | p name]; 
            '='; 
       value : Alpha \}
\mbox{}
(* Name value pair with name specified. *)
\kw{ptype} (Alpha) Nvp(name:string) = 
   Alpha Pnvp(fun s -> s = name)
\mbox{}
(* Name value pair with any name. *)
\kw{ptype} Nvp\_a = SVString Pnvp(fun \_ -> true)
\mbox{}
\kw{ptype} Details = \{
      source      : Pip Nvp("src\_addr");
';';  dest        : Pip Nvp("dest\_addr");
';';  start\_time  : Ptimestamp Nvp("start\_time");
';';  end\_time    : Ptimestamp Nvp("end\_time");
';';  cycle\_time  : Puint32 Nvp("cycle\_time")
\}

\kw{pdatatype} Info(alarm\_code : int) =
  match alarm\_code with
    5074 -> Details of Details
  | \_    -> Generic of Nvp\_a Plist(';','|')

\kw{pdatatype} Service = 
    DOMESTIC      of "DOMESTIC" 
  | INTERNATIONAL of "INTERNATIONAL" 
  | SPECIAL       of "SPECIAL"

\kw{ptype} Alarm = \{
       alarm    : [i : Puint32 | i = 2 or i = 3];
 ':';  start    : Ptimestamp Popt;
 '|';  clear    : Ptimestamp Popt;
 '|';  code     : Puint32;
 '|';  src\_dns  : SVString Nvp("dns1");
 ';';  dest\_dns : SVString Nvp("dns2");
 '|';  info     : Info(code);
 '|';  service  : Service
\} 

\kw{ptype} Source = Alarm Plist('\\n',\kw{peof})\end{code}
%
%\kw{let} checkCorr ra = ...
%\kw{ptype} Alarm = [x:Raw\_alarm | checkCorr x]



  \caption{Description of \darkstar{} data.}
  \label{fig:darkstar-ml}
\end{figure}

This data format has several variants of name-value pairs. The
\padsc{} description of this format~\cite{fisher+:popl-sub-long} must
define a different type for each variant. In contrast, the polymorphic
types of \padsml{} allow us to define the type \cd{Pnvp}, which takes
both type and value parameters to encode all the variants. As is
customary in \ml{}, type parameters appear to the left of the type
name, while value parameters and their \ml{} types appear to the
right.  The type \cd{Pnvp} has one type parameter named \cd{Alpha} and
one value parameter named \cd{p}.  Informally, \cd{Alpha Pnvp(p)} is a
name-value pair where the value is described by \cd{Alpha} and the
name must satisfy the predicate \cd{p}.

The \cd{Nvp} type reuses the \cd{Pnvp} type to define a name-value pair
whose name must match the argument string \texttt{name} but whose
value can have any type.  The \cd{Nvp\_a} type also uses the type
\cd{Pnvp}. It defines a name-value pair that permits any name, but
requires the value to have type \cd{SVString} (a string terminated by
a semicolon or vertical bar).  Later in the description, the type
parameter to \cd{Nvp} is instantiated with IP addresses, timestamps,
and integers.

% The source type is an array of \cd{alarm}s, where each alarm is a
% \cd{raw\_alarm}, constrained to ensure that the alarm number is
% properly correlated with the timestamps.  We check this correlation
% with the function \cd{checkCorr}.  The type \cd{raw\_alarm} closely
% follows the description above. We highlight a few important features.
% First, we note that the type of the field \cd{info} depends on the
% alarm code, reflecting the text above. More interestingly, the type
% \cd{info} is implemented with a switched datatype, deciding how to
% parse based on the parameter \cd{alarm\_code}.  Next, we note that the
% description includes five different types of name-value pairs. We take
% advantage of both the type and value parameterization of types to
% encode all of these pair types based on one common description,
% \cd{pnvp}. This type is polymorphic in the type of the value and takes
% an arbitrary constraint \cd{c} as an argument. The type \cd{nvp} is
% polymorphic in the type of the value, but takes the expected name of
% the string as an argument. 

The \darkstar{} description also illustrates the use of
\textit{switched} datatypes.  A switched datatype selects a variant
based on the value of a user-specified \ocaml{} expression, which
typically references parsed data from earlier in the data source.  For
example, the switched datatype \cd{Info} chooses a
variant based on the value of its \cd{alarm\_code} parameter.  More
specifically, if the alarm code is \cd{5074}, the format specification
given by the \cd{Details} constructor will be used to parse the
current data.  Otherwise, the format given by the \cd{Generic}
constructor will be used.

% The last construct in the \darkstar{} description is the type
% qualifier \cd{omit}.  In the \cd{Service} datatype,
% \cd{omit} specifies that the parsed string literal should be omitted
% in the internal data representation because the literal can be
% determined by the datatype constructor.

\cut{We can do this because we can discern from the
datatype constructor which string was found in the data source.}

%%% Local Variables: 
%%% mode: latex
%%% TeX-master: "../thesis.tex"
%%% End: 

%\section{From \padsmlbig{} to \ocamlbig{}}
\label{sec:padsml-impl}

{\em
ToDo: Move detail about Traverse functor to Generic tools section.
}

We have implemented \padsml{} for use with \ocaml{}. The \padsml{}
compiler generates libraries in \ocaml{} source code that can then be
used by any \ocaml{} program. In this section, we describe the
contents of the generated libraries followed by some examples
demonstrating their use.


\subsection{Generated Libraries}
\label{sec:gen-code}

From each \padsml{} description, we generate a collection of types and
functions in \ocaml{}, including:
\begin{itemize}
\item The types of two data structures: one to contain parsed data in
  memory and the other to hold meta-data about the parsing process.
  These data structures are respectively called the
  \emph{representation} and the \emph{parse descriptor}.
\item A parsing function, which parses a data source to produce a
  representation and parse descriptor for the data.
\item A generic tool generator, based on the new tool development
  framework for \padsml{}. This framework is discussed in
  \secref{sec:gen-tool}.
\end{itemize} 

In general, the representation and parse-descriptor type definitions
are designed to closely resemble the original description.
The aim is to minimize the amount of effort a user must invest in
order to understand and use the data structures returned by the
parser.

Furthermore, the type of parse descriptor mimics the type of the
representation so that the parse descriptor can provide a parsing
report for every element of a corresponding representation. Parse
descriptors have two components: a header and a body. The header
reports on the parsing process that produced the representation. It
includes an error count that indicates the number of subcomponents
with errors; an error code that indicates the type of error, if any;
and the location of the data within the original data source. The body
of the parse descriptor contains the parse descriptors (if any) for
subcomponents of corresponding representations. The body for a
base-type parse descriptor is always of type \cd{unit}.

Below is a simple \padsml{} description of a character
and integer separated by a vertical bar.
\begin{code}\scriptsize
  \kw{ptype} Pair = Pchar * '|' * Pint\end{code} 
Here is a partial listing of the elements generated from that description.
\begin{code}\scriptsize
\kw{type} rep = Pchar.rep * Pint.rep
\kw{type} pd_body = Pchar.pd  * Pint.pd
\kw{type} pd = Pads.pd_header * pd_body

\kw{val} parse : Pads.handle -> rep * pd\end{code} 
This sample code and others that follow make use of a module
\cd{Pads} that contains types and functions that commonly occur in
generated and base-type modules. In particular, the above declarations use
\cd{Pads.pd_header}, which is the type of all parse-descriptor
headers, and \cd{Pads.handle}, which is the type of the (abstract)
handles used for data sources.
Note the close correspondence between the structure of the description
and that of the \cd{rep} and \cd{pd_body} types. In addition, we see
that the type of the parse function is defined in terms of the
\cd{rep} and \cd{pd} types.

Given the close relationship between the elements generated from a
description, it is natural to collect them together in a module. For
each named type, therefore, we generate a module with definitions like
those shown in the above example.
% For all generated modules, \cd{rep}, \cd{pd_body},\cd{pd} define the
% types of the data's representation, parse-descriptor body, and parse
% descriptor, respectively. The parsing function is named \cd{parse}.
In general, all types with base kind (i.e. those that are not
parameterized) match the following signature,
\cd{Type.S}:
\begin{code}\scriptsize
\kw{type} rep
\kw{type} pd\_body
\kw{type} pd = Pads.pd_header * pd_body

\kw{val} parse : Pads.handle -> rep * pd\end{code}

Modules, then, become the building blocks of the \padsml{} system.
Base types, too, are implementated with modules. Polymorphic types,
which map types to types, are implemented as functors from (type)
modules to (type) modules. It would even be appropriate to map
recursive types into recursive modules. Unfortunately, this approach
fails due to the limitations of the \ocaml{} implementation of
recursive modules. We would need support for inclusion of functors in
recursive modules in order to take this approach.

Given the signature \cd{Type.S} for types of base kind, we can now
show an example signature for a polymorphic type.
\begin{code}\scriptsize
\kw{ptype} (Alpha,Beta) ABPair = Alpha * '|' * Beta\end{code}
becomes
\begin{code}\scriptsize
\kw{module} ABPair (Alpha : Type.S) (Beta : Type.S) :
\kw{sig}
  \kw{type} rep = Alpha.rep * Beta.rep
  \kw{type} pd\_body = (Pads.pd_header * Alpha.pd\_body) * 
                 (Pads.pd_header * Beta.pd\_body)
  \kw{type} pd = Pads.pd_header * pd\_body

  \kw{val} parse : Pads.handle -> rep * pd
\kw{end}\end{code}

Once a description has been compiled into an \ocaml{} module, that
module can be used like any other.  More specifically, each named type
in a description file is mapped into an \ocaml{} module of the same
name.  The collection of modules is grouped together into a
single file (compilation unit) with a name corresponding to the name
of the original description file. For example, a description file
named ``foo.pml'' with three types inside results in a file ``foo.ml''
with three submodules, each corresponding to one named type.  In the
remainder of this section, we will demonstrate a number of uses of
generated modules, highlighting data processing, transformation, and
filtering.

\subsection{Example: Data Processing}
\label{sec:ex-process}

We begin with a simple example in which we process a triple of
integers. Below is their description:
\begin{code}\scriptsize
\kw{ptype} Source = Pint * '|' * Pint * '|' * Pint\end{code} Next, we
show a complete \ocaml{} program that finds the average of the three
integers. (Note that we assume that the name of the description file
is ``intTriple.pml,'' resulting in an \ocaml{} module \cd{IntTriple}.)
\begin{code}\scriptsize
\kw{open} Pads
\kw{let} ((i1,i2,i3),pd) = 
    parse_source IntTriple.Source.parse "input.txt"
\kw{let} avg = match get_pd_hdr pd with
    \{error_code = Good\} -> (i1 + i2 + i3)/3
  | _ -> 0\end{code}

In this program, we parse the triple, check that it is valid and then
average its elements. The function \cd{parse_source} takes a parsing
function for a data source and a file name in which the data is
stored, and parses the source. In order to ensure that the data is
valid, the program projects the parse descriptor header from the parse
descriptor \cd{pd} and checks that the error code is set to \cd{Good}.
This error code is defined in the \cd{Pads} module, and indicates that
the data is syntactically and semantically valid.

Notice that checking the parse descriptor of the triple is enough to
guarantee that there are no errors in any of the triple's
subcomponents. This property is generally true of all representations
and corresponding parse descriptors. That is, if the header of a parse
descriptor reports no errors, then none of its subcomponents will
report errors. In this way, we support a ``pay-as-you-go'' approach to
application error handling, as the parse descriptor for valid data
need only be consulted once, no matter the size of the corresponding
data. Only if there are errors within the structure does the user then
need to continue consulting the parse descriptor until the error is
located.

\subsection{Example: Filtering}
\label{sec:ex-filter}

An important set of tasks relating to ad hoc data are those
related to errors, including error analysis, repair, and removal.
Programmers might want to clean their data, \ie{}, filter out data
containing errors. In this case, they need only access parse
descriptors to facilitate this task.

\begin{figure}
\begin{code}\scriptsize
\kw{open} Pads
   ...
\kw{let} split_entry (entry,pd) =
   match get\_pd\_hdr pd with
     \{error_code = Good\} -> write_valid entry
   | _ => write_invalid entry\end{code}
\caption{Error filter for \dibbler{} data}
\label{fig:ex-data-clean}
\end{figure}

\figref{fig:ex-data-clean} provides a partial demonstration of
splitting a standard data source into two separate sources, one with
valid records and the other (potentially) invalid records.  The valid
entries may then be further processed or loaded into a database
without corrupting the valuable data therein.  A human might examine
the bad entries off-line to determine the cause of errors or to figure
out how to fix the corrupted entries.

We assume that functions \cd{write_valid} and \cd{write_invalid} are
defined elsewhere to write an entry to a stream of valid and invalid
entries, respectively. The \cd{split_entry} function, then, receives
an entry and its parse descriptor, and, based on the parse descriptor,
writes the entry to the appropriate stream.

\subsection{Example: Transformation}
\label{sec:ex-trans}

\begin{figure}
  \centering
  \begin{code}\scriptsize
...
\kw{ptype} Header = \{
       alarm : [ a : Puint32 | a = 2 or a = 3];
 ':';  start :  Timestamp Popt;
 '|';  clear :  Timestamp Popt;
 '|';  code: Puint32;
 '|';  src\_dns  :  Nvp("dns1");
 ';';  dest\_dns :  Nvp("dns2");
 '|';  service  : service
\}
\mbox{}
\kw{ptype} D\_alarm = \{
       header   : header;
 '|';  details  : details
 \}
\mbox{}
\kw{ptype} G\_alarm = \{
       header   : header;
 '|';  generic  : (Nvp\_a,Semicolon,Vbar) Plist
\}\end{code}
\caption{Normalized format for \darkstar{} data. All named types not
  explicitly included in this figure are unchanged from the original
  \darkstar{} description.}
\label{fig:normal-darkstar}
\end{figure}

\begin{figure}
\begin{code}\scriptsize
\kw{open} Regulus
\kw{open} RegulusNormal
\kw{module} RA = Raw\_alarm
\kw{module} DA = D\_alarm
\kw{module} GA = G\_alarm
\kw{module} Header = H

\kw{let} splitAlarm ra =
    let h = 
       \{H.alarm=ra.RA.alarm; H.start=ra.RA.start; 
         H.clear=ra.RA.clear; H.code=ra.RA.code;
         H.src\_dns=ra.RA.src\_dns; H.dest\_dns=ra.RA.dest\_dns;
         H.service=ra.RA.service\};
    in match ra with
        \{info=Details(d)\} -> 
        (Some \{DA.header = h; DA.details = d\}, None)
      | \{info=Generic(g)\} ->
        (None, Some \{GA.header = h; GA.generic = g\})    
  \end{code}
  \caption{Shredding \darkstar{} data based on the {\tt info} field.}
  \label{fig:ex-no-err-check}
\end{figure}

Once a data source has been parsed and cleaned, a natural desire is to
transform such data to make it more amenable to further analysis.  For
example, analysts often need to convert ad hoc data into a form
suitable for loading into an existing system, such as a relational
database or statistical analysis package. Desired transformations
include removing extraneous literals, inserting delimiters, dropping
or reordering fields, and normalizing the values of fields (\eg{}
converting all times into a specified time zone).

Because relational databases typically cannot store unions directly,
another important transformation is to convert data with variation
(\ie{}, datatypes) into a form that such systems can handle.
Typically, there are two choices for such a transformation.  The first
is to chop the data into a number of relational tables: one table for
each variation.  This approach is called \textit{shredding}. The
second is to create an ``uber'' table, with one ``column'' for each
field in any variation.  If a given field is not in a particular
variation, it is marked as missing. 

The description fragment in \figref{fig:normal-darkstar} and code
fragment in \figref{fig:ex-no-err-check} demonstrate shredding
\darkstar{} data with \padsml{} and \ocaml{}. We shred the data into
two different tables based on the \cd{info} field of \cd{Alarm}
records. In the process, we also reorder the fields, putting the
\texttt{service} field into the common \texttt{header}. Notice that, in
the normalized format, \cd{Alarm} has been replaced with \cd{D\_alarm}
and \cd{G_alarm}, neither of which contain any fields with variable
type.

\begin{figure}
  \centering
  \begin{code}\scriptsize
\kw{let} normalizeTimeToGMT t = 
    match t with
      \{time=t;timezone="GMT"\} => t
    | \{time=t;timezone="EST"\} => t + (5 * 60 * 60)
    | \{time=t;timezone="PST"\} => t + (8 * 60 * 60)
    | ... \end{code}
  \caption{Normalizing timestamps}
  \label{fig:ex-normalize}
\end{figure}

In \figref{fig:ex-normalize}, we show an additional example of data
transformation, where we normalize timestamp-timezone pairs into
simple timestamps in GMT time.

%%% Local Variables: 
%%% mode: latex
%%% TeX-master: "paper"
%%% End: 

\section{From \padsmlbig{} to \ocamlbig{}}
\label{sec:padsml-impl}
The \padsml{} compiler takes descriptions and generates \ocaml{}
modules that can be used by any \ocaml{} program.  In this section, we
describe the generated modules and illustrate their use.

\subsection{Types as Modules}
\label{sec:gen-code}

We use the \ocaml{} module system to structure the libraries generated
by the \padsml{} compiler.  Each \padsml{} base type is implemented as
an \ocaml{} module.  For each \padsml{} type in a description, the
\padsml{} compiler generates an \ocaml{} module containing the types,
functions, and nested modules that implement the \padsml{} type.  All
the generated modules are grouped into one module that implements the
complete description.  For example, a \padsml{} description named
\texttt{sirius.pml}, which contains three named types, will result in the
\ocaml{} file \texttt{sirius.ml} defining the module \cd{Sirius},
which will contain three submodules, each corresponding to one named type.

Namespace management alone is sufficient motivation to employ a
``types as modules'' approach, but the power of the \ml{} module
system provides substantially more.  We implement polymorphic
\padsml{} types as functors from (type) modules to (type) modules.
Ideally, we would like to map recursive \padsml{} types into recursive
modules.  Unfortunately, this approach currently is not possible, because
\ocaml{} prohibits the use of functors within recursive modules,
and the output of the \padsml{} compiler includes a functor for each
type.  Instead, we implement recursive types as modules containing
recursive datatypes and functions.  As there is no theoretical reason
to prevent recursive modules from containing functors~\cite{dreyer-thesis}, we
pose our system as a challenge to implementers of module systems.

The module generated for any monomorphic \padsml{} type 
matches the signature \cd{S}:
\begin{code}\scriptsize
\kw{module} \kw{type} S = \kw{sig}
  \kw{type} rep
  \kw{type} pd\_body
  \kw{type} pd = Pads.pd_header * pd_body
  \kw{val}  parse : Pads.handle -> rep * pd
  \kw{val}  print : rep -> pd -> Pads.handle -> unit
  (* Functor for tool generator ... *)
  \kw{module} Traverse ...
\kw{end}\end{code}% 
\noindent 
The \emph{representation} (\cd{rep}) type describes the
in-memory representation of parsed data, while the
\emph{parse-descriptor} (\cd{pd}) type describes meta-data collected
during parsing.
The parsing function converts the raw data into an in-memory
representation and parse descriptor for the representation.
The printing function performs the reverse operation.
The module also contains a generic tool generator implemented as a functor; we defer a description of this functor to \secref{sec:gen-tool}.
The module \cd{Pads} contains the
built-in types and functions that occur in base-type and generated
modules.  The type \cd{Pads.pd_header} is the type of all parse-descriptor
headers and \cd{Pads.handle} is an abstract type containing the
private data structures \padsml{} uses to manage data sources.

The structure of the representation and parse-descriptor types
resembles the structure of the corresponding \padsml{} type, making it
easy to see the correspondence between parsed data, its internal
representation, and the corresponding meta-data.  
For example, given the \padsml{} type \cd{Pair} describing a character
and integer separated by a vertical bar:
\begin{code}\scriptsize
  \kw{ptype} Pair = Pchar * '|' * Pint\end{code}%
the compiler generates a module with the signature:
\begin{code}\scriptsize
\kw{module} \kw{type} Pair_sig = \kw{sig}
  \kw{type} rep     = Pchar.rep * Pint.rep
  \kw{type} pd_body = Pchar.pd  * Pint.pd
  \kw{type} pd      = Pads.pd_header * pd_body
  \kw{val}  parse   : Pads.handle -> rep * pd
  \kw{val}  print   : rep -> pd -> Pads.handle -> unit
  ...
\kw{end}\end{code}%
\noindent 
The parse-descriptor header reports on the parsing
process that produced the corresponding representation.  It includes
the location of the data in the source, an error code
describing the first error encountered, and the number of
subcomponents with errors.  The body contains the parse
descriptors for subcomponents.
Parse descriptors for base types have a body of type \cd{unit}.

The signature for a polymorphic \padsml{} type uses the signature
\cd{S} for monomorphic types, defined above.  
Given the polymorphic \padsml{} type \cd{ABPair}:
\begin{code}\scriptsize
\kw{ptype} (Alpha,Beta) ABPair = Alpha * '|' * Beta\end{code}%
the compiler generates a module with the signature:
\begin{code}\scriptsize
\kw{module} \kw{type} ABPair_sig (Alpha : S) (Beta : S) = 
\kw{sig}
  \kw{type} rep      = Alpha.rep * Beta.rep
  \kw{type} pd\_body = Alpha.pd * Beta.pd
  \kw{type} pd       = Pads.pd_header * pd\_body
  \kw{val}  parse    : Pads.handle -> rep * pd
  \kw{val}  print    : rep -> pd -> Pads.handle -> unit
  ...
\kw{end}\end{code}%

\subsection{Using the Generated Libraries}
Common data management tasks like filtering and normalization are easy to express in \ocaml{}.  In the remainder of this section, we illustrate this point by giving \ocaml{} programs to compute properties of ad hoc data, to filter it, and to transform it.

\cut{In contrast, \padsc{} users fall off an
abstraction cliff when they shift from declaratively describing data
to manipulating the generated data structures in \C{}, where they have
to worry about details such as manipulating \C{} strings and worrying
about memory management.  }

\subsubsection{Example: Computing Properties}
\label{sec:ex-process}
Given the \padsml{} type:
\begin{code}\scriptsize
\kw{ptype} IntTriple = Pint * '|' * Pint * '|' * Pint\end{code}%
the following \ocaml{} expression computes the average of the three integers in the file \texttt{input.data}:
\begin{code}\scriptsize
\kw{let} ((i1,i2,i3), (pd_hdr, pd_body)) = 
 Pads.parse_source IntTriple.parse "input.data" \kw{in}
\kw{match} pd_hdr \kw{with}
  \{error_code = Pads.Good\} -> (i1 + i2 + i3)/3
{|} _ -> \kw{raise} Pads.Bad_file \end{code}%
\noindent
The \cd{parse_source} function takes a parsing function and a file
name, applies the parsing function to the data in the specified file,
and returns the resulting representation and parse descriptor.  To
ensure the data is valid, the program examines the error code in the parse-descriptor header.  The error code \cd{Good} indicates that the data
is syntactically and semantically valid. Other error codes include
\cd{Nest}, indicating an error in a subcomponent, \cd{Syn}, indicating
that a syntactic error occurred during parsing, and \cd{Sem},
indicating that the data violates a semantic constraint. The
expression above raises an exception if it encounters any of these
error codes.

Checking the top-level parse descriptor for errors is sufficient to
guarantee that there are no errors in any of the subcomponents.  This
property holds for all representations and corresponding parse
descriptors.  This design supports a ``pay-as-you-go'' approach to
error handling. The parse descriptor for valid data need only be
consulted once, no matter the size of the corresponding data. User
code only needs to traverse nested parse descriptors if information
about an error is required.

\subsubsection{Example: Filtering}
\label{sec:ex-filter}

Data analysts often need to ``clean'' their data (\ie{}, remove or
repair data containing errors) before loading the data into a database
or other application.  \ocaml{}'s pattern matching and higher-order
functions can simplify these tasks.  For example, the expression in
\figref{fig:ex-data-clean} partitions \dibbler{} data into valid
orders and invalid orders.  \cut{The valid orders may then be further
processed or loaded into a database without risk of failure during the
load nor of corrupting the valuable data therein.  The invalid orders
can be examined off-line to determine the cause of the errors or to
repair them.

An important set of tasks involving ad hoc data are those related to
errors, including error analysis, repair, and removal.}

\begin{figure}
\begin{code}\scriptsize
\kw{open} Pads

\kw{let} classify_order order (pd\_hdr, pd\_body) (good, bad)=
   \kw{match} pd\_hdr \kw{with}
    \{error_code = Good\} -> (order::good, bad)
   | _                  -> (good, order::bad)

\kw{let} split_orders orders (orders_pd_hdr,order_pds) = 
   List.fold_right2 classify_order orders order_pds ([],[])

\kw{let} ((header, orders),(header_pd, orders_pd)) = 
   parse_source Sirius.parse "input.txt"

\kw{let} (good,bad) = split_orders orders orders_pd\end{code} 
\caption{Error filtering of \dibbler{} data}
\label{fig:ex-data-clean}
\end{figure}

\cut{
The \cd{classify_order} function receives an order, its parse descriptor,
and the lists of good and bad orders. Based on the parse
descriptor, it adds the order to the appropriate list.  The function
\cd{split_orders} simply folds \cd{classify_order} over the lists of
orders and corresponding parse descriptors.
}
\subsubsection{Example: Transformation}
\label{sec:ex-trans}

\begin{figure}
  \centering
  \begin{code}\scriptsize
...
\kw{ptype} Header = \{
       alarm : [ a : Puint32 | a = 2 or a = 3];
 ':';  start :  Ptimestamp Popt;
 '|';  clear :  Ptimestamp Popt;
 '|';  code  :  Puint32;
 '|';  src\_dns  :  Nvp("dns1");
 ';';  dest\_dns :  Nvp("dns2");
 '|';  service  : Service
\}
\mbox{}
\kw{ptype} D\_alarm = \{
       header : Header;
 '|';  info   : Details
 \}
\mbox{}
\kw{ptype} G\_alarm = \{
       header : Header;
 '|';  info   : Nvp\_a Plist(';','|')
\}\end{code}
\caption{Listing of \texttt{\darkstar{}Normal.pml}, a normalized format for
  \darkstar{} data. All named types not explicitly included in this
  figure are unchanged from the original \darkstar{} description.}
\label{fig:normal-darkstar}
\end{figure}

\begin{figure}
\begin{code}\scriptsize
\kw{open} Regulus
\kw{open} RegulusNormal
\kw{module} A = Alarm
\kw{module} DA = D\_alarm
\kw{module} GA = G\_alarm
\kw{module} Header = H

\kw{type} ('a,'b) Sum = Left of 'a | Right of 'b

\kw{let} split_alarm ra =
  \kw{let} h = 
    \{H.alarm=ra.A.alarm; H.start=ra.A.start; 
     H.clear=ra.A.clear; H.code=ra.A.code;
     H.src\_dns=ra.A.src\_dns; H.dest\_dns=ra.A.dest\_dns;
     H.service=ra.A.service\}
  \kw{in} \kw{match} ra \kw{with}
      \{info=Details(d)\} -> 
      Left \{DA.header = h; DA.info = d\}
    | \{info=Generic(g)\} ->
      Right \{GA.header = h; GA.info = g\}

\kw{let} split_alarm_pd pd = ... (* mirrors split_alarm *)

\kw{let} process_alarm pads [pads_D; pads_G] = 
  \kw{let} a,a_pd = Alarm.parse pads in
    \kw{match} (split_alarm a, split_alarm_pd a_pd) \kw{with}
     (Left  da, Left  da_p) -> DA.print da da_p pads_D
    |(Right ga, Right ga_p) -> GA.print ga ga_p pads_G
    | _ -> ... (* Bug! *)

\kw{let} _ = process_source process_alarm 
              "input.data" ["d_out.data";"g_out.data"]
  \end{code}
  \vskip -2ex
  \caption{Shredding \darkstar{} data based on the {\tt info} field.}
  \label{fig:ex-no-err-check}
\end{figure}

Once a data source has been parsed and cleaned, a common task is to
transform the data into formats required by other tools, like a
relational database or a statistical analysis package.
Transformations include removing extraneous literals, inserting
delimiters, dropping or reordering fields, and normalizing the values
of fields (\eg{}, converting all times into a specified time zone).
Because relational databases typically cannot store unions directly,
one common transformation is to convert data with variants (\ie{},
datatypes) into a form that such systems can handle.  One option is to
partition or ``shred'' the data into several relational tables, one
for each variant.  A second option is to create an universal table,
with one column for each field in any variant.  If a given field does
not occur in a particular variant, its value is marked as missing.

\figref{fig:normal-darkstar} shows a partial listing of
\texttt{\darkstar{}Normal.pml}, a normalized version of the
\darkstar{} description from \secref{sec:padsml-overview}. In this
shredded version, \cd{Alarm} has been split into two top-level types
\cd{D\_alarm} and \cd{G_alarm}.  The type \cd{D\_alarm} contains all
the information concerning alarms with the detailed payload, while
\cd{G\_alarm} contains the information for generic payloads.  In the
original description, the \cd{info} field identified the type of its
payload.  In the shredded version, the two different types of records
appear in two different data files. Since neither of these formats
contains a union, they can be easily loaded into a relational
database.

The code fragment in \figref{fig:ex-no-err-check} shreds \darkstar{}
data in the format described by \texttt{\darkstar{}.pml} into the
formats described in \texttt{\darkstar{}Normal.pml}.  It uses the
\cd{info} field of \cd{Alarm} records to partition the data. 
Notice that the code invokes the \cd{print} functions generated for the \cd{G_alarm} and \cd{D_alarm} types to output the shredded data.

\cut{
\begin{figure}
  \centering
  \begin{code}\scriptsize
\kw{let} normalizeTimeToGMT t = 
    match t with
      \{time=t;timezone="GMT"\} => t
    | \{time=t;timezone="EST"\} => t + (5 * 60 * 60)
    | \{time=t;timezone="PST"\} => t + (8 * 60 * 60)
    | ... \end{code}
  \caption{Normalizing timestamps}
  \label{fig:ex-normalize}
\end{figure}

In \figref{fig:ex-normalize}, we show an additional example of data
transformation, where we normalize timestamp-timezone pairs into
simple timestamps in GMT time.
}
%%% Local Variables: 
%%% mode: latex
%%% TeX-master: "paper"
%%% End: 


%\section{The Generic Tool Framework}
\label{sec:gen-tool}

\begin{itemize}
\item show signature of traversal functor.
\item show interface - explain that one module/constructor and base types.
\item constrast records and one other module.
\item x Make it 3 tools: accum, XML, debuger.
\item x Focus on accumulator. describe in more detail.
\item show implementations,  including base types to show how they differ.
\item x then, describe XML, debugger as pretty printers.
\item use ``generic tool'' in consistent way. I like it better than
  tool generator.
\end{itemize}

An essential benefit of \padsml{} is that it can provide the users
with a high return-on-investment for describing their data. While the
generated parser alone is often enough to justify the user's effort,
we aim to increase the return by producing a large suite of data
analysis tools for every data source described in \padsml{}. However,
there is a limit, both in resources and expertise, to the range of
tool generators that we can develop. Indeed, new and interesting data
analysis tools are constantly being developed, and we have no hope of
integrating even a fraction of them into the \padsml{} system
ourselves.

Fortunately, we don't have to. A large class of data analysis tools
share a common data processing method while differing in the details
of how they transform data. These tools traverse the data
representation and parse descriptor in a depth-first, left-to-right
manner, often carrying some auxiliary state.  For each element
visited, they perform some action involving the auxiliary state,
either before, after or between visiting the element's subcomponents.
Often, the most interesting computations occur at the leaves, where
the computation is based on the leaf's type.  

To the experienced functional programmer, these tools essentially
perform a generalized fold over the representation and parse
descriptor, viewed as trees of components. Therefore, we have created
a generic tool framework modeled after \ml{}'s fold: we decouple the
mechanism of data traversal from the operations performed on
individual data items. A format-dependent traversal mechanism is
generated by the \padsml{} compiler. \emph{Generic tools}, which
specify how to process individual data elements, is developed
separately. The two interact through a signature that every generic
tool must match.

Unlike lists, the elements of data representations and PD's do not
have a uniform type. Therefore, the generic tool must provide
different processing functions for different types. The generated
traversal mechanism, which is format specific, can then choose the
correct function to apply to each element of the data. Therefore, the
generic-tool signature specifies a particular collection of types and
functions for every construct in \padsml{}.

The new, generic tool architecture of \padsml{} delivers a number of
benefits over the fixed architecture of \padsc{}. In \padsc{}, all
tools are generated from within the compiler. Therefore, developing a
new tool generator requires understanding and modifying the compiler.
Furthermore, the set of tools to be generated is chosen by the user
when compiling the description.  In \padsml{}, tool generators can be
developed independent of the compiler and they can be developed more
rapidly, as the ``boilerplate'' code to traverse data need not be
replicated for each tool generator. In addition, the user controls
which tools to ``generate'' for a given data format, and the choice
can differ on a program-by-program basis.

\subsection{The Generic-Tool Interface}
\label{sec:gentool-interface}

\begin{figure}
\begin{code}\scriptsize
\kw{module} \kw{type} S = \kw{sig}
  \kw{type} state
  ...
  \kw{module} Record : \kw{sig}
    \kw{type} partial_state
    \kw{val}  init          : (string * state) list -> state
    \kw{val}  start         : state -> Pads.pd_header 
                         -> partial_state
    \kw{val}  project       : state -> string -> state
    \kw{val}  process_field : partial_state -> string
                         -> state -> partial_state
    \kw{val}  finish        : partial_state -> state
  \kw{end}

  \kw{module} Datatype : \kw{sig}
    \kw{type} partial_state
    \kw{val}  init            : unit -> state
    \kw{val}  start           : state -> Pads.pd_header 
                           -> partial_state
    \kw{val}  project         : state -> string -> state option
    \kw{val}  process_variant : partial_state -> string 
                           -> state -> state
  \kw{end}
   ...
\kw{end}
\end{code}
\caption{An excerpt of the generic-tool interface.}
\label{fig:gentool-interface}
\end{figure}

The generic tool interface specifies a type for abstract state that is
threaded through the traversal, and, for every type constructor in
\padsml{}, a set of types and functions -- grouped together in a
module -- that a generic tool must implement. Every module has an
\cd{init} function to create an initial state object for data
processed by that module. In addition, a \cd{project} function
retrieves the state of a subcomponent from the state of an element. As
processing an element can occur before, after, or between processing
an element's children, the signature includes functions corresponding
to each of these events. The function \cd{start} begins processing the
element, \cd{process_...} to process a subcomponent, and \cd{finish}
to complete processing the element. For type constructs with only one
subcomponent, the \cd{process} and \cd{finish} functions are combined.

% The generated traversal function processes the data in a
% depth-first, left-to-right manner. In general, given an abstract
% state object for a a data element, it is traversed as follows.
% First, it provides the generic tool with the state and the PD header
% and receive a new, ``in-progress'' state object in return.  Next,
% for each child of the node, it will ask the tool for the child's
% state based on the node's state, recursively process the child node
% with the given state, and then give the child's new state to the
% tool.  For nodes with more than one child, such as records and
% tuples, it will tell the tool that it has finished, at which point
% the tool converts the ``in-progress'' state into a new state object.

An excerpt of the generic tool interface is shown in
\figref{fig:gentool-interface}. It includes the signatures of the
\cd{Record} and \cd{Datatype} modules. Both modules include a type
\cd{partial_state} that allows tools to represent intermediate state
in a different form than the general state while processing
subcomponents. In the \cd{Record} module, \textbf{I am here.}

\subsection{Example Tools}
\label{sec:gentool-motivation-ex}

% \begin{figure}
%   \centering
%   \small
% \begin{verbatim}
% 122Joe|Wright|450|95|790
% n/aEd|Wood|10|47|31
% 124Chris|Nolan|80|93|85
% 125Tim|Burton|30|82|71
% 126George|Lucas|32|62|40
% \end{verbatim}  
%   \caption{A fictitious data fragment in the Movie-director Bowling
%     Score (MBS) format. Note that the first record contains semantic
%     errors (in that the minimum is larger than the maximum and the
%     average is larger than both the minimum and the maximum).}
%   \label{fig:gentool-mbs-sample}
% \end{figure}

\begin{figure}
  \centering
  \scriptsize
\begin{verbatim}
<Order>
   <summary>
      <errors>1</errors> <total>2</total>        
   </summary>
   <Order_header>
      <summary>
         <errors>1</errors> <total>2</total>        
      </summary>
      <order_num>
         <errors>0</errors> <total>2</total>        
      </order_num>
      <att_order_num>
         <summary>
            <errors>1</errors> <total>2</total>        
         </summary>
         <val>
            <errors>0</errors> <total>2</total>                
         </val>
      </att_order_num>
      <ord_version>
         <errors>0</errors> <total>2</total>                
      </ord_version>
      ...
   </Order_header>
</Order>
\end{verbatim}  
  \caption{A fragment of the accumulator output for \dibbler{}. The
    output is encoded in XML.}
  \label{fig:gentool-acc-output}
\end{figure}

We have implemented three generic tools that illustrate important
features of the framework: a tool for generating statistical
overviews of the data, a data printer for debugging, and an XML
formatter.

A common desire of a data analyst upon receiving a new data source is
to get a sense of the quality of the data. In particular, they might
be interested in knowing what percentage of the source has errors, or
which fields are the most problematic. For this purpose, \padsc{}
provides an \emph{accumulator} tool for \padsc{} that statistically
summarizes data sources. The accumulator is designed for data sources
whose basic structure is series of records of the same type.
\emph{another sentence or two here. explain notion of one
  acccumulator, many records}

We have implemented a generic tool that provides some of the basic
features of \padsc{} accumulators. Our accummulator counts the number
of errors and the total number of values for every element of a
description. In \figref{fig:gentool-acc-output}, we show a sample
portion of accumulator output for the \dibbler{} data
from~\figref{figure:dibbler-records}.

The accumulator data shows that 1 out of the 2 \cd{Order}s has an
error. Investigating further, we notice that the problem lies in the
\cd{Order_header}, in particular within the \cd{att_order_num} field.
This field has a constraint on it, and one of the values violates the
constraint. A quick glance at the data fragment reveals that the
second order is contains the offending field. In general, specific
information like this could be found in the parse descriptor.

\begin{figure}
\begin{code}\scriptsize
\kw{type} baseAcc = int * int
\kw{type} acc = baseAcc G.metadata
\kw{type} state = acc

\kw{module} Record = \kw{struct}
  \kw{type} partial_state = baseAcc * acc Table.t
  ... 
  \kw{let} start state header =
    \kw{match} state \kw{with}
      G.RecordData ((errs, total), accs) ->
	\kw{let} errs' = if header.nerr > 0
                    then errs + 1 else errs
	\kw{in} ((errs', total + 1), accs))
    | _ -> \kw{raise} ...
	  
  \kw{let} project state label = \kw{match} state \kw{with}
      G.RecordData (_, accs) -> (\kw{try} Table.find accs label
                                 \kw{with} _ -> \kw{raise} ...)
    | _ -> \kw{raise} ...

  \kw{let} process_field (bAcc, accs) label acc =
    (bAcc, Table.update accs label acc)
      
  \kw{let} finish (bAcc, accs) = G.RecordData (bAcc, accs)
\kw{end}
    
\kw{module} Datatype = \kw{struct}
  \kw{type} partial_state = baseAcc * acc Table.t
	
  \kw{let} init () = G.DatatypeData (initBaseAcc, Table.empty)

  \kw{let} start state header = 
    \kw{match} state \kw{with}
      G.DatatypeData ((errs, total), accs) ->
	\kw{let} errs' = if header.nerr > 0
                    then errs + 1 else errs
	\kw{in} ((errs', total + 1), accs))
    | _ -> \kw{raise} ...
	  
  \kw{let} project state label = \kw{match} state \kw{with}
    G.DatatypeData (_, accs) -> (\kw{try} Some (Table.find accs label)
                                 \kw{with} _ -> None)
  | _ -> \kw{raise} ...
	  
  \kw{let} process_variant (bAcc, accs) label acc =
    G.DatatypeData (bAcc, Table.update accs label acc)
\kw{end}
...
\end{code}
\caption{Excerpts from the implementation of the accumulator.}
\label{fig:gentool-accum-code}
\end{figure}

\emph{Discuss implementation of accum briefly here.}

In addition to the accumulator, we have implemented two different
kinds of pretty printers for parsed data.  One formats the data in
XML. The other prints the data in a simple text format that is helpful
for debugging descriptions. Importantly, both tool's output corresponds to the in-memory
representation of the data rather than its original format (which may,
for example, have delimiters that are not present in the
representation).

% \begin{figure}
% \begin{code}\scriptsize
%   \kw{type} state = Xml.xml \kw{exception} Tool_error \kw{of} state
%   * string \kw{let} init () = () ...  \kw{module} Record =
%   \kw{struct} (* Partial state is a list of XML objects *
%   representing the fields processed so far * tagged with the
%   record's PD header *) \kw{type} partial_state = Pads.pd_header *
%   (Xml.xml list)

%   \kw{let} init named_states =
%     ... (* Create initial "blank" state with no data *)

%   \kw{let} start state header =
%     ... (* Initialize processing of a new record *)

%   \kw{let} project state field_name =
%     ... (* Extract named field's state *)

%   \kw{let} process_field (header, fields) field_name state =
%     (header, fields@[Xml.Element 
%     	(field_name, [], [state])])

%   \kw{let} finish pstate =
%     ... (* Convert a partial state into a final state *)
% \kw{end}
% ...
% \end{code}
% \caption{Excerpts from the implementation of the generic XML formatter
%   tool in {\tt xml\_formatter.ml}.}
% \label{fig:gentool-xmlf-code}
% \end{figure}

%%% Local Variables: 
%%% mode: latex
%%% TeX-master: "paper"
%%% End: 

\section{The Generic Tool Framework}
\label{sec:gen-tool}

An essential benefit of \padsml{} is that it can provide users with a
high return-on-investment for describing their data. While the
generated parser and printer alone are enough to justify the user's
effort, we aim to increase the return by enabling users to easily
construct data analysis tools.
% However, there is a limit, both in resources and expertise, to the
% range of tool generators that we can develop.  Indeed, new and
% interesting data analysis tools are constantly being developed, and
% we have no hope of integrating even a fraction of them into the
% \padsml{} system ourselves.
To this end, we provide a simple framework for others to develop 
format-independent tools.

The techniques of type-directed programming, known variously as
\textit{generic} or \textit{polytypic} programming, provide a
convenient conceptual starting point in designing a tool framework.
In essence, any format-independent tool is a function from a description to 
a concrete realization of that tool. 
As \padsml{} descriptions are (dependent) types, a 
format-independent tool is a type-directed function.

% Support for some form of generic programming over data representations
% and parse descriptors is an essential first step in supporting the
% development of tool generators. 

Some modern functional programming languages,
Generic Haskell~\cite{hinze+:generic-haskell}, in particular,
have many features that support type-directed programming, and
hence would support development of format-independent tools quite nicely. 
% would be
% useful in this context, 
\ocaml{}, however, lacks any specific, built-in generic programming facility.
Fortunately, we can still achieve many of the benefits of generic 
programming idioms by having the \padsml{} compiler generate
well-designed, format-specific libraries at compile time
and then linking those libraries to format-independent routines.

To be specific, for each format description, \padsml{} generates a
format-dependent traversal mechanism that implements a generalized
fold over the representations and parse descriptors that correspond to
the description. Independently, tool developers write format-independent
routines that specify the behaviour of a tool over each
\padsml{} type constructor. When users need a {\em specific tool} for a
{\em specific format}, they link the format-dependent 
traversal to the format-independent routines via 
functor application.

In principle, different tools might require different sorts of traversals.  
However, many of the tools we have encountered in practice so far, both in 
implementing \padsml\ and \pads,
perform their computations in a single pass over the representation
and corresponding parse descriptor, visiting each value in the data
with a left-to-right, pre-, post-, or in-order traversal.  This 
paradigm arises naturally as it scales to very large data sets. 
Hence, the \padsml\ compiler generates an implementation of a such traversal
for each data description. 
%abstracted in a manner similar to the generic functions of Lammel and
%Peyton-Jones~\cite{lammel+:syb}. 

% The generic tool architecture of \padsml{} delivers a number of
% benefits over the fixed architecture of \padsc{}. In \padsc{}, all
% tools are generated from within the compiler. Therefore, developing a
% new tool generator requires understanding and modifying the compiler.
% Furthermore, the user selects the set of tools to generate when compiling the description.  In \padsml{}, tool generators can be
% developed independent of the compiler and they can be developed more
% rapidly because the ``boilerplate'' code to traverse data need not be
% replicated for each tool generator. In addition, the user controls
% which tools to ``generate'' for a given data format, and the choice
% can differ on a program-by-program basis.

\subsection{The Generic-Tool Interface}
\label{sec:gentool-interface}

\begin{figure}
\begin{code}\scriptsize
\kw{module} \kw{type} S = \kw{sig}
 \kw{type} state
 ...
 \kw{module} Record : \kw{sig}
   \kw{type} partial_state
   \kw{val}  init          : (string * state) list -> state
   \kw{val}  start         : state -> Pads.pd_header 
                        -> partial_state
   \kw{val}  project       : state -> string -> state
   \kw{val}  process_field : partial_state -> string
                        -> state -> partial_state
   \kw{val}  finish        : partial_state -> state
 \kw{end}

 \kw{module} Datatype : \kw{sig}
   \kw{type} partial_state
   \kw{val}  init            : unit -> state
   \kw{val}  start           : state -> Pads.pd_header 
                          -> partial_state
   \kw{val}  project         : state -> string -> state option
   \kw{val}  process_variant : partial_state -> string 
                          -> state -> partial_state
   \kw{val}  finish          : partial_state -> state
 \kw{end}
  ...
\kw{end}
\end{code}
\vskip -2ex
\caption{Excerpt of generic-tool interface \texttt{Generic\_tool.S}.}
\label{fig:gentool-interface}
\end{figure}

The interface between format-specific traversals and generic tools is
specified as an \ocaml{} signature.  For every type constructor in
\padsml{}, the signature describes a sub-module that implements the
generic tool for that type constructor.  In addition, it specifies an
(abstract) type for auxiliary state that is threaded through the
traversal. \figref{fig:gentool-interface} contains an
excerpt of the signature that includes the signatures of the
\cd{Record} and \cd{Datatype} modules.  The signatures of other
modules are quite similar.

The \cd{Record} module includes a type \cd{partial_state} that allows
tools to represent intermediate state in a different form than the
general state. The \cd{init} function forms the state of the record
from the state of its fields. The \cd{start} function receives the PD
header for the data element being traversed and begins processing the
element. Function \cd{project} takes a record's state and the name of
a field and returns that field's state. Function \cd{process_field}
updates the intermediate state of the record based on the name and
state of a field, and \cd{finish} converts the finished intermediate
state into general tool state.  Note that any of these functions could
have side effects.

Although the \cd{Datatype} module is similar to the \cd{Record} module,
there are some important differences. The \cd{Datatype} \cd{init} function 
does not
start with the state of all the variants. Instead, a variant's state is
added during processing so that only variants that have been
encountered will have corresponding state. For this reason,
\cd{project} returns a \cd{state option}, rather than a \cd{state}.
This design is essential for supporting recursive datatypes as trying
to initialize the state for all possible variants of the datatype
would cause the \cd{init} function to loop infinitely.
\cut{
\begin{figure}
\begin{code}\scriptsize
\kw{module} Traverse (Tool : Generic_tool.S) :
\kw{sig}
  \kw{val} init : unit -> Tool.state
  \kw{val} traverse : rep -> pd -> Tool.state -> Tool.state
\kw{end}
\end{code}
\caption{The signature of the Traversal functor within the signature \texttt{Type.S}.}
\label{fig:traversal-interface}
\end{figure}
}

The following code snippet gives the signature of the
traversal functor as it would appear in the signature \cd{S} from
\secref{sec:padsml-impl}. 
\begin{code}\scriptsize
\kw{module} Traverse (Tool : Generic_tool.S) :
\kw{sig}
  \kw{val} init : unit -> Tool.state
  \kw{val} traverse : rep -> pd -> Tool.state -> Tool.state
\kw{end}
\end{code}%
\vskip -2ex
\noindent
The functor takes a generic tool generator and produces
a format-specific tool with two functions: \cd{init}, to create the initial
state for the tool, and \cd{traverse}, which traverses the
representation and parse descriptor for the type and updates the given
tool state.

\subsection{Example Tools}
\label{sec:gentool-motivation-ex}

We have used this framework to implement a variety of tools useful for processing ad hoc data, including an \xml{} formatter, an accumulator tool for generating statistical overviews of the data, and a data printer for debugging.  We briefly describe these tools to illustrate the flexibility of the framework.

The \xml{} formatter converts any data with a \padsml{} description
into a canonical \xml{} format.  This conversion is useful because it
allows analysts to exploit the many useful tools that exist for
manipulating data in \xml{}. 

The accumulator tool provides a statistical summary of data.
Such summaries are useful for developing a quick understanding of data quality.  In particular, after receiving a new batch of data, analysts might want to know the frequency of errors, or which fields are the most corrupted. 
The accumulator tool tracks the distribution of the top $n$
distinct legal values and the percentage of errors. 
It operates over data sources
whose basic structure is a series of records of the same type, 
providing a summary based on viewing many records
in the data source.    More complex accumulator programs and a number of other statistical algorithms can easily be implemented using the tool generation infrastructure.

Finally, as an aid in debugging \padsml{} descriptions, we have implemented a simple printing tool.  In contrast to the printer generated by the \padsml{} compiler, the output of this tool corresponds to the in-memory
representation of the data rather than its original format, which may have delimiters or literals that are not present in the representation.  
This format is often more readable than the raw data.

%%% Local Variables: 
%%% mode: latex
%%% TeX-master: "../thesis.tex"
%%% End: 


\section{Conclusions}
\label{sec:future}

\padsml{} is a high-level, domain-specific language and system
designed to help improve the productivity of the legions of data analysts
who work with ad hoc data on a regular basis.  
Inspired by the type structure of functional
programming languages, \padsml{} uses dependent,
polymorphic and recursive data types to describe the syntax and 
the semantic properties of ad hoc data sources.  The language 
is compact and expressive, capable of describing data from diverse 
domains including networking, computational biology, finance, and physics. 
The \padsml{} compiler uses a ``types as modules'' compilation strategy
in which every \padsml{} type definition is compiled into
an \ocaml{} module containing types for data representations
and functions for data processing.  Functional programmers
can use the generated modules to write clear and concise {\em format-dependent}
data processing programs.  Furthermore, our system design
allows external tool developers to write new {\em format-independent} tools
simply by supplying a module that matches the appropriate generic
signature.  The latest release of our implementation is available
at \url{http://www.padsproj.org/padsml/}.

The next step in our long-term agenda is to build a new generation of
format-independent data analysis tools.  While our current tools
perform some simple syntactic analysis and transformation, we intend
our next generation toolkit to perform deeper semantic analysis and
more sophisticated transformations.  For example, we may explore
specification-driven, content-based search, clustering, test data
generation, machine learning, security, and data visualization.  We
believe that if we can automatically generate stand-alone, end-to-end
tools that perform these functions over arbitrary data, we can have a
substantial impact on the productivity of researchers in a broad array
of scientific fields ranging from computational biology through
computer science to cosmology and beyond.


\paragraph*{Acknowledgments.}
We would like to thank Derek Dreyer for discussions and advice on
advanced module systems.  Some of this material is based upon work 
supported by NSF awards 0615062 and 0612147.  
Any opinions, findings, and conclusions 
or recommendations expressed in the material are those of the authors 
and do not necessarily reflect the views of the NSF.

\bibliographystyle{abbrv}
%\bibliography{pads}
\begin{thebibliography}{1}

\bibitem{dreyer-thesis}
D.~Dreyer.
\newblock {\em Understanding and Evolving the ML Module System}.
\newblock PhD thesis, CMU, May 2005.

\bibitem{fisher+:pads}
K.~Fisher and R.~Gruber.
\newblock {PADS}: A domain specific language for processing ad hoc data.
\newblock In {\em {ACM} Conference on Programming Language Design and
  Implementation}, pages 295--304. {ACM} Press, June 2005.

\bibitem{fisher+:next700ddl}
K.~Fisher, Y.~Mandelbaum, and D.~Walker.
\newblock The next 700 data description languages.
\newblock In {\em {ACM SIGPLAN-SIGACT} Symposium on Principles of Programming
  Languages}, pages 2 -- 15, Jan. 2006.

\bibitem{mandelbaum-thesis}
Y.~Mandelbaum.
\newblock {\em The Theory and Practice of Data Description}.
\newblock PhD thesis, Princeton University, September 2006.

\bibitem{fisher+:popl-sub-long}
Y.~Mandelbaum, K.~Fisher, D.~Walker, M.~Fernandez, and A.~Gleyzer.
\newblock {PADS/ML: A} functional data description language.
\newblock Technical Report TR-761-06, Princeton University, July 2006.

\bibitem{newick}
Tree formats. {Workshop} on molecular evolution.
\newblock \url{http://workshop.molecularevolution.org/resources/}\\\noindent
  \url{fileformats/tree}$\_$\url{formats.php}.

\end{thebibliography}

\end{document}

%%% Local Variables:
%%% mode: outline-minor
%%% End:

