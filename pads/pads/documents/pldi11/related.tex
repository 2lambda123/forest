\section{Related Work}
\label{sec:related}

The work in this paper builds upon ideas developed in the \pads{} 
project~\cite{fisher+:pads,fisher+:toplas}. \pads{} uses extended
type declarations to describe the grammar of a document
and simultaneously to generate types for parsed data and a suite 
of data-processing tools.  The obvious difference between 
\pads{} (and other parser generators) and
\forest{} is that 
\pads{} generates infrastructure for processing strings (the insides
of a single file) whereas \forest{} generates infrastructure for 
processing entire file systems.
\forest{} (and \padshaskell) is architecturally superior to 
previous versions of \pads{} in the tight integration with its host
language and
in its support for third-party generic programming and tool construction.

More generally, \forest{} shares high-level goals with other systems
that seek to make data-oriented programming simpler and more productive.
For example, Microsoft's LINQ~\cite{linq} extends the .NET languages
to enable querying
any data source that supports the \cd{IEnumerable} interface using
a simple, convenient syntax.  
LINQ differs in that it does not provide support for
declaratively specifying the structure of, and then ingesting, 
\filestores{}. {\em Type Providers}~\cite{syme+:type-providers}, an
experimental feature 
of F\#, help programmers materialize standard data sources equipped with
predefined schema (such as \xml{} documents or databases) in memory in
an F\# program.  Type Providers and \forest{} descriptions
are complementary language
features.  In fact, it may be possible to define a new F\# Type Provider
capable of interpreting \forest{} file system schema and ingesting
the described data, thereby making any \forest{}-described data available
in F\#.

In the databases community, a number of XML-based
description languages have been defined for specifying
file formats, file organization and file locations.
One example of such a language is XFiles~\cite{xml-file-sys}.
XFiles has many features in common with \forest{}.  It can
describe file locations, permissions, ownership and other
attributes.  It can also specify the name of an application capable of
parsing the files in question.  The main difference between
a language like XFiles and \forest{} is that \forest{} is
tightly integrated into a general-purpose, conventional programming
language.  \forest{} declarations generate types, functions
and data structures that materialize the data within
a surrounding \haskell{} program. XFiles does not interoperate
directly with conventional programming languages. 
