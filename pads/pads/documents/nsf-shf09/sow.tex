\subsection{Overview of Planned Research and Collaboration}
\label{ssec:sow}

The expected results from this project are: 
(1) a general framework for accurately specifying, composing and 
    verifying higher-level properties for low-level software; 
(2) a prototype implementation of certifying compiler for the Microsoft Common
Intermediate Language (CIL) and
    (together with Intel\footnote{
Intel Labs has an ``Open Runtime Platform'' project, one component
of which is
an open-source research JIT compiler for the Microsoft Common Intermediate
Language
interface.  We are now beginning a collaboration with
researchers in this project to make this JIT ``type-preserving'' so
that it can generate proof-carrying code.  Intel plans to publish
all its results and make all the source code available
under a BSD-style open-source license.  
All of our own work will be unrestricted
by proprietary licenses.}) 
a realistic industrial-strength 
    certifying JIT compiler for CIL; and
(3) a set of general recipes for writing certified libraries.
% and a type-safe dynamic linker.
We divide our work into the following three areas:
%
{\renewcommand{\labelenumi}{\bf\theenumi.}
\begin{enumerate} \itemsep 0pt
\item {\bf High-level specifications for low-level software.} 
Work in this area focuses on basic research. Our goal is to tackle the
fundamental problem in constructing high-assurance software
components, i.e., how to accurately specify, compose, and verify
high-level specifications for low-level programs.
We will work on {\bf{}Logic-based type systems (LTS)} at Yale and Princeton
to specify and verify low-level 
      constructs such as stack allocation, mutable recursive
      data structures, exception handling, data layout 
      with alignment and flexible tagging, 
      and array-bounds checking elimination.
We will extend the current {\bf{}FPCC infrastructure} at Princeton
 to support new and more
      powerful type systems, develop semantic models for LTS, and build a
      highly efficient FPCC verifier.
We will develop 
{\bf{}LTS/FPCC interface}
technologies for translating compiler intermediate languages
annotated with LTS types into foundational proof-carrying code.

\item{\bf High-assurance virtual machine.}
Work in this area focuses on the technology transfer of results from
Part 1 into realistic CLR implementations. We plan to extend our
FLINT certifying compiler to support Microsoft Common Intermediate 
Language. We'll also work with people at Intel Labs, 
advising them on design of low-level intermediate representations 
and type systems to use in their JIT compilers.  
%
At Yale we'll develop a prototype 
certifying {\bf FLINT/CIL} compiler that compiles CIL bytecodes into the
      logic-based typed assembly language (which uses LTS developed in
      Part 1).  
At Princeton we'll 
      develop a prototype {\bf CIL PCC} generator that translates FLINT/CIL 
      assembly code into typed machine code. 
And we'll all work with Intel Labs
      to transfer our technologies developed under our prototype
      implementation into realistic industrial-strength 
{\bf Certifying JIT compilers}. 
%
\item{\bf Certified libraries with application-specific 
           properties.}
Large software systems often have very specific rules and constraints. 
To achieve seamless interoperability, we must specify
and verify these advanced system properties. 
At Princeton and Yale we'll
      develop technologies for specifying, composing, and
      verifying various kinds of {\bf{}memory management APIs}  
      (e.g., 
      reference counting, garbage collection, region-based memory  
      management).
\ignore{We'll also
      develop technologies for {\bf type-safe dynamic linking},
build a set of libraries for enforcing consistent
      {\bf metadata protocols and type-safe reflection},
and develop the basic methodologies for building libraries that
      enforce {\bf 
%advanced security protocols and 
accurate resource usage}.
}%
\end{enumerate}}
%
\noindent{}We give a detailed description for each of these three
areas in Sections~\ref{ssec:hlspec}--\ref{ssec:certapi}. 
We then summarize the broader impact (see Section~\ref{ssec:impact}) 
and compare our work with other related research 
(see Section~\ref{ssec:related}). Finally we present the PIs' results 
from prior NSF supported grants (see Section~\ref{ssec:results}).

%%% Local Variables: 
%%% mode: latex
%%% TeX-master: "proposal"
%%% End: 
