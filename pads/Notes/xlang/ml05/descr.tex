\section{Describing Data in \datatypebig{} }
\label{sec:data-description}

A \datatype{} description specifies the physical layout and semantic
properties of an ad hoc data source.  
These descriptions are composed of types: 
base types describe atomic data, while structured
types describe compound data built from simpler pieces.  Examples of
base types include 8-bit unsigned integers (\cd{Puint8}), 32-bit
integers (\cd{Pint32}), binary 32-bit integers (\cd{Pbint32}), dates
(\cd{Pdate}), strings (\cd{Pstring}), and IP addresses (\cd{Pip}).
Semantic conditions for such base types include checking that the
resulting number fits in the indicated space, \ie, 16-bits for
\cd{Pint16}.

Base types (and other types) may be parameterized by values.  This
mechanism serves both to reduce the number of base types and to permit
the format and properties of later portions of the data to depend upon
earlier portions.  For example, the base type \cd{Puint16_FW(3)}
specifies an unsigned two byte integer physically represented by
exactly three characters. The base type \cd{Pstring} takes 
a string indicating the set of acceptable terminator characters. Hence, 
\cd{Pstring(";|")} can be terminated by either a semicolon or a
vertical bar.


To describe more complex data, \datatype{} provides a collection of
type constructors derived from the type structure of functional
programming languages such as Haskell and ML.  The following
subsections will explain these structured types through a series 
of real-world examples.  Readers eager to see the complete syntax
of types should flip forward to Appendix~\ref{app:syntax-dd}.

\subsection{Simple Structured Descriptions}

The bread and butter of any \datatype{} description are the
simple structured types: tuple, record and array types for expressing
sequences of data; sum types, realized here as datatypes, for
expressing alternatives; and singleton types for expressing
the placement of particular characters in the data.  In this section, we
explain each of these items by building a description 
of the \dibbler{} data presented in \figref{figure:dibblerml}.

The \dibbler{} data source is used to record summaries of
phone orders produced at AT\&T.  Each summary
involves a date and one record per order.
Each order record contains a header followed by a sequence of events.
The header has 13 pipe separated fields: the order number, AT\&T's
internal order number, the order version, four different telephone
numbers associated with the order, the zip code of the order, a
billing identifier, the order type, a measure of the complexity of the
order, an unused field, and the source of the order data.  Many of
these fields are optional, in which case nothing appears between the
pipe characters.  The billing identifier may not be available at the
time of processing, in which case the system generates a unique
identifier, and prefixes this value with the string ``no\_ii'' to
indicate the number was generated. The event sequence represents the
various states a service order goes through; it is represented as a
new-line terminated, pipe separated list of state, timestamp pairs.
There are over 400 distinct states that an order may go through during
provisioning.  The sequence is sorted in order of increasing timestamps. 
%\figref{figure:dibbler-records} shows a small example of
%this format.
%156 different states for one order
%-rw-r--r--    1 angusm   dibbler   2187472314 Jun  9  2003 /fs/dibblerd/tlf/data/out_sum.stream
%2171.364u 31.379s 40:41.54 90.2% 0+0k 2+0io 2pf+0w
%53 had trailing t or } after zip code
%It may be apparent from this paragraph that English is a poor
%language for describing data formats!


\begin{figure*}
\begin{small}
%\begin{center}
\begin{verbatim}
0|1005022800
9152|9152|1|9735551212|0||9085551212|07988|no_ii152272|EDTF_6|0|APRL1|DUO|
9153|9153|1|0|0|0|0||152268|LOC_6|0|FRDW1|DUO|
\end{verbatim}
\caption{Miniscule example of \dibbler{} data.}
\label{figure:dibbler-records}
%\end{center}
\end{small}
\end{figure*}

\suppressfloats

\begin{figure}
\begin{code}
\kw{type} summary\_header = <"0|"> * Puint32 * <NL>
\mbox{}
\kw{datatype} dib\_ramp = 
  Ramp of Puint64 
| GenRamp of <"no\_ii"> * Puint64
\mbox{}
\kw{type} order\_header = \{
       order\_num      : Puint32;
 '|';  att\_order\_num  : Puint32;             
 '|';  ord\_version    : Puint32;         
 '|';  service\_tn     : pn\_t \kw{Popt};
 '|';  billing\_tn     : pn\_t \kw{Popt};          
 '|';  nlp\_service\_tn : pn\_t \kw{Popt};
 '|';  nlp\_billing\_tn : pn\_t \kw{Popt};
 '|';  zip\_code       : Pzip \kw{Popt};
 '|';  ramp           : dib\_ramp; 
 '|';  order\_sort     : Pstring('|');
 '|';  order\_details  : Puint32;             
 '|';  unused         : Pstring('|');
 '|';  Pstring('|')   : stream;
 '|';
\}
\mbox{}
\kw{type} event  = Pstring('|') *  <'|'> * Puint32
\mbox{}
\kw{type} events = event Parray('|',NL)
\mbox{}
\kw{type} source = summary\_header * (order\_header * events) Parray(NL,EOF)
\end{code}
\caption{\datatype{} description for \dibbler{} provisioning data.}
\label{figure:dibblerml}
\end{figure}


\figref{figure:dibblerml} gives a \datatype{} description for the
\dibbler{} phone order summaries in our syntax.  Overall, the description 
is a sequence of type definitions.  It is probably
easiest to understand the data source by reading these definitions
bottom up.

The last type definition \cd{source} is intended to be a definition of
an entire \dibbler{} data source.  It states that a
\cd{source} is a \cd{summary\_header} followed by a sequence of
objects made up of an \cd{order\_header} followed by \cd{events}.  The
tuple type constructor (\cd{T1 ** T2}) and the array type constructor
(\cd{T Parray(sep,term)}) both specify sequences of objects in a data
source.  The \cd{Parray} type depends upon two value parameters,
(\cd{sep} and \cd{term}).  The first parameter describes the syntactic
separators that may be found between elements of the array.  In this
case \cd{Peor}, the {\em end-of-record} character, may be found 
between each element
of the array.  In this case, the end-of-record character 
should be set to a newline character as there is one record per
line in the file.   
The second parameter is the
terminator for the array.  In this case, the terminator is \cd{Peof}, the
end-of-file marker.  It is often necessary to have additional
termination conditions for arrays, such as termination dependent on
the number of elements read so far, but we omitted this additional option 
for expository purposes.

The definition of \cd{events} indicates that this part of the
\dibbler{} data will contain a sequence of \cd{event}s separated by
vertical bars and terminated by an end-of-record character.  
Each \cd{event} is a
string terminated by a vertical bar, followed by a vertical bar and
ending with an unsigned 32-bit integer.  The interesting part of this
sequence is the presence of the type \cd{'|'}.  In type-theoretic
terms, this is a {\em singleton type}.  It states that one should
expect exactly the character \cd{'|'} in the input stream at this
point.  Other singletons appear in the summary header type as
\cd{"0|"} and \cd{Peor}.  Any expression that appears embedded in a type
as a singleton must be effect-free.

The type \cd{order\_header} is a record type that indicates
the data format involves the sequence of items described by
the fields of the record.  Notice that there are two different
sorts of fields: anonymous fields (a second form of singleton type) 
containing directives to parse a particular character (\cd{'|'}) or
string (\cd{"0|"})
and fields with names.  
% The second named field,
% \cd{att\_order\_num}, reveals two other proposed features of 
% \datatype: dependency and constraints.  Here,
% \cd{att\_order\_num} is constrained to be less than
% \cd{order\_num}, the value parsed in an earlier field.
% This is a relatively simple constraint on the correctness of the
% ad hoc data format.  In practice, constraints can become very rich
% involving properties such as sortedness of records in an array,
% definitions of expected characters,
% restrictions on date and time ranges, constraints on IP address
% domains, restrictions on phone number area codes and virtually 
% infinite variety of other possibilities. 

The last interesting feature in the \dibbler{} example is the
datatype definition of \cd{dib\_ramp}.  It describes
two alternatives for a portion of data, either an integer alone
or the fixed string \cd{"no\_ii"} followed by an integer.
To parse data in this format, the parser will
attempt to parse the first branch and only if it
fails will it attempt to parse the second branch.
Notice how this semantics differs from similar constructs in
regular expressions and context-free grammars, which
non-deterministically choose between alternatives.
By making alternatives deterministic, we avoid
the need to implement general backtracking search.  
Fortunately, we have yet to come across an ad hoc data source
where we wish we had nondeterminisc choice.\footnote{\pads{}
can recognize string data based on regular expressions.
Non-determinism here has been useful, but as it
has been confined to parsing elements of the \cd{Pstring} 
base type, it has had no impact on the overall parsing 
algorithm.}

\begin{figure}
\begin{code}
val COMMA  = ','
val COLON  = ':'
val SEMI   = ';'
val LPAREN = '('
val RPAREN = ')'
\mbox{}
type entry = Pstring(COLON) ** COLON ** Pfloat32
\mbox{}
datatype tree =
    Tree of LPAREN ** tree Parray(COMMA,RPAREN) ** "):" ** Puint32
  | Tip of entry
\mbox{}
type source = (tree ** SEMI) Parray(NL,EOF)
\mbox{}
{\rm Tiny fragment of Newick data:} 
\mbox{}
(((erHomoC:0.28006,erCaelC:0.22089):0.40998,(erHomoA:0.32304,
(erpCaelC:0.58815,((erHomoB:0.5807,erCaelB:0.23569):0.03586,
erCaelA:0.38272):0.06516):0.03492):0.14265):0.63594,(TRXHomo:0.65866,
TRXSacch:0.38791):0.32147,TRXEcoli:0.57336);
\end{code}
\caption{Simplified tree-shaped Newick data}
\label{fig:newick}
\end{figure}

\subsection{Recursive Descriptions}

The original \pads{} design contained analogues of all the 
simple type constructors described in the previous section.
Hence, although the syntax of types we just saw
is more compact and more elegant than the C-style syntax 
of \pads, we have yet to add any expressive power. 
The first real step forward for \datatype{} is the introduction
of recursion through the data type mechanism.  While
recursion never appeared in the telecommunications and 
networking data sources \pads{} was initially designed for, it
appears essential for describing a number of biological
data sources we have encountered.
  
A representative example of such data comes courtesy of Steven
Kleinstein, program coordinator of Princeton's Picasso project for
interdisciplinary research in computational sciences.  Kleinstein is
in the process of building a simulator to study the
proliferation of B lymphocytes during an immune response.  Data
needed for his simulations is represented in a variant of the
Newick format, which is a flat
representation of trees used by 
many biologists~\cite{newick}.  In Newick, 
leaves of the tree are string labels followed by a colon and a number.
A parent node in the tree introduces a collection of children by
placing a sequence of trees within parens.  Following the parens is a
colon and a number, as is the case for the leaf node.
The numbers represent the ``distance'' 
that separates the child from the parent.  In
Kleinstein's case, the distance is the number of mutations that occur
in the antibody receptor genes of B lymphocytes.   Each line
of a file may contain a different tree, terminated by a semi-colon.

\figref{fig:newick} gives a description of Newick and a short fragment 
of example data.  Despite the relative complexity of the structure of the data,
the description is remarkably concise and elegant.  Notice that the data type
definition of \cd{tree} is recursive --- we have not been able to find any
effective description of this data source without it.

\begin{figure}
  \centering
  \small
\begin{verbatim}
 2:3004092508||5001|dns1=abc.com;dns2=xyz.com|c=slow link;w=lost packets
 |INTERNATIONAL
 3:|3004097201|5074|dns1=bob.com;dns2=alice.com|src_addr=192.168.0.10;
 dst_addr=192.168.23.10;start_time=1234567890;end_time=1234568000;
 cycle_time=17412|SPECIAL
\end{verbatim}  
  \caption{Simplified network monitoring data.  This 
data containts two alarm records.  Extra newlines 
were inserted mid-record so the data would fit on a page.}
  \label{fig:darkstar-records}
\end{figure}

\subsection{Parameterized Descriptions}

The last key feature of \datatype{} is the ability to define
parameterized descriptions.  In \pads, description parameters
were limited to values.  In \datatype, descriptions
maybe parameterized by both values and types.  Type parameterization
allows descriptions to be
reused at a variety of different types.  This feature 
helps make data descriptions
more concise and allows programmers to define convenient libraries
of reusable description components.

Our final example, shown in
Figure~\ref{fig:darkstar-records}, will illustrate 
the usefulness of descriptions parameterized by
both types and values and introduce a couple of other
useful features as well.   This example displays
alarm data recorded by a network link monitor
used by the \darkstar{} project at AT\&T.  Each alarm signals some
problem with a network link.  

Now, at this point, we could write out an English description
of the \darkstar{} data, and indeed, in an earlier version of this paper
we did so.  However, English is a terrible specification language
for ad hoc data, and every description we have attempted to give has not 
only been imprecise, it has also made for some very 
boring reading.  Hence, we dispense with the informal
English and head straight for for the \datatype{} description,
which is shown in \figref{fig:darkstar-ml}.

% In our example,
% there are two classes of alarms, indicated with the integers 2 and 3.
% The alarm class is specified first, followed by a colon. When a class
% 2 alarm occurs, the ``start'' timestamp is recorded, while when a
% class 3 alarm occurs, the ``clear'' timestamp is recorded.  Next, we
% have an integer indicating the exact type of the alarm.  Following
% this, we expect two pairs of strings, with the strings of each pair
% separated by '='. The first string is a name and the second a string
% value associated with that name. The pairs are separated by a
% semicolon.  The first pair has name ``dns1'' and the second ``dns2'',
% indicating the DNS names of the parties on either end of the network
% link. Next, the record contains information about the alarm. The
% format of this information differs depending on the type of alarm. For
% alarm 5074, we include a specific set of details about the failure.
% For other alarms, there appears a sequence of name-string pairs of
% unpspecfied length.  Finally, the alarm indicates the service class of
% the network. This value is one of three strings: ``DOMESTIC'',
% ``INTERNATIONAL,'' and ``SPECIAL.'' Each field of the alarm is
% separated by a vertical bar, except where otherwise specified.

% The details included for alarm 5074 are as follows: a source and
% detination IP address, both encoded as name-IP pairs, with names
% ``src\_addr'' and ``dest\_addr.'' These are followed by two
% name-timestamp pairs, with names ``start\_time'' and ``end\_time,''
% indicating the start and end times of the problem. The timestamps are
% 10-digits indicating the number of seconds since the epoch, in
% GMT. The last field is a name-integer pair. Note that each field 
% is separated by a semicolon.

% At this point, the reader might consider for themselves whether or not
% English is an effective specification language for ad hoc data.

\begin{figure}
  \centering
  \small
  \begin{code}
type timestamp = Ptimestamp_explicit_FW(10, "%S", GMT)
\mbox{}
type 'a pnvp(c : Pstring -> bool) =
      \{ name : \{name : Pstring("=") | c name\};
        '=';
        val : 'a \}
type 'a nvp(name:string) = 'a pnvp(fun c(s:string):bool = (s = name))
type nsp(name:string) = Pstring(";|") nvp(name)
type nsp_a = Pstring(";|") pnvp(fun c(s:string):bool = true)
\mbox{}
type details = \{
      source      : Pip nvp("src_addr");
';';  dest        : Pip nvp("dest_addr");
';';  start_time  : timestamp nvp("start_time");
';';  end_time    : timestamp nvp("end_time");
';';  cycle_time  : Puint32 nvp("cycle_time")
\}
\mbox{}
datatype info(alarm_code : Puint32) =
  case alarm_code of 
    5074 => Details of details
  | _    => Generic of nsp_a Parray(';', '|')
\mbox{}
datatype service =
    DOMESTIC
  | INTERNATIONAL
  | SPECIAL
\mbox{}
type raw_alarm = \{
       alarm    :  \{ alarm : Puint32 | alarm == 2 orelse alarm == 3\};
 ':';  start :  timestamp Popt;
 '|';  clear :  timestamp Popt;
 '|';  code: Puint32;
 '|';  src_dns  :  nsp("dns1");
 ';';  dest_dns :  nsp("dns2");
 '|';  info     :  info(code);
 '|';  service  :  service;
\}
\mbox{}
fun checkCorr(ra:raw_alarm):bool =
  case ra of 
    \{alarm=2; start=SOME(_); clear=NONE;...\} => true
  | \{alarm=3; start=NONE;    clear=SOME(_);...\} => true
  |  _ => false
\mbox{}
type alarm = \{x: raw_alarm | checkCorr x\}
\mbox{}
type source = alarm Parray('\\n', Peof)
\end{code}

  \caption{Description of \darkstar{} data.}
  \label{fig:darkstar-ml}
\end{figure}

One of the interesting facets of this data source is the fact that
it contains multiple different types of name-value pairs.  In the
second definition from the top, \cd{pnvp}, we take
advantage of both the type and value parameterization of types to
encode all the different 
variations.\footnote{This fragment of code contains both
types \cd{Pstring} and \cd{string}.  These types are subtly different
as the first is the type of an \pvalue{} containing both a string 
representation and a parse descriptor whereas the second is an 
ordinary string.  The reader can safely ignore the distinction for now.}    
The type parameters
are specified to the left of type name, as is customary in ML.
To the right of the type name in parentheses, we give the
value parameter and its type. 
In the case of \cd{pnvp}, there is a type parameter name \cd{'a}
and a value parameter named \cd{f}.  Informally,
\cd{'a pnvp(f)} is a name-value pair where the value has type \cd{'a}
and the name must satisfy the predicate \cd{f}.  More precisely,
\cd{pnvp} is defined to be
a record with three fields.  The first field of the record
has been given a constrained type.  In general, constrained types have the
form \cd{\{x:T | M\}} where \cd{M} is an arbitrary pure boolean 
expression.  Data \cd{d} satisfies this description if it satisfies
\cd{T} and boolean \cd{M} evaluates to true when the \pvalue{} \cd{d}
is substituted for \cd{x}.  If the boolean evaluates to false, the
data contains a semantic error.  

The \cd{pnvp} type is used in both of the following type definitions.
In the case of the \cd{nvp} definition, the predicate is instantiated
with a test for a specific string but the type parameter remains 
abstract.  In the \cd{nvp\_a} 
definition, the name can be arbitrary, but the value must have 
type string. Later in the description, \cd{nvp}'s
are used with ip addresses, timestamps and integers. 

% The source type is an array of \cd{alarm}s, where each alarm is a
% \cd{raw\_alarm}, constrained to ensure that the alarm number is
% properly correlated with the timestamps.  We check this correlation
% with the function \cd{checkCorr}.  The type \cd{raw\_alarm} closely
% follows the description above. We highlight a few important features.
% First, we note that the type of the field \cd{info} depends on the
% alarm code, reflecting the text above. More interestingly, the type
% \cd{info} is implemented with a switched datatype, deciding how to
% parse based on the parameter \cd{alarm\_code}.  Next, we note that the
% description includes five different types of name-value pairs. We take
% advantage of both the type and value parameterization of types to
% encode all of these pair types based on one common description,
% \cd{pnvp}. This type is polymorphic in the type of the value and takes
% an arbitrary constraint \cd{c} as an argument. The type \cd{nvp} is
% polymorphic in the type of the value, but takes the expected name of
% the string as an argument. 

The \darkstar{} description also introduces two new forms of
datatypes.  The first new form appears in the \cd{info} type.
This datatype is parameterized by an \cd{alarm\_code}.
Rather than 
determine the branch of the datatype to choose
based on the data about to be parsed, the decision is made based on
the \cd{alarm\_code}, data which has likely been extracted from some previous
point in the source.  More specifically, if the alarm code is
\cd{5074}, the format specification given by the 
\cd{Details} constructor will be used to parse
the current data.  Otherwise, the format given by the \cd{Generic} constructor
will be used to parse the current data.  

The second new form of datatype appears in the  \cd{service} type.
Here, the constructors specify no types for their arguments.  
Our convention in this case
is to look for string data that matches the name of the
constructor. For example, the constructor \cd{DOMESTIC} 
is associated with the string
``DOMESTIC.'' This type could have been specified (more verbosely)
using the ordinary form for datatypes together with singletons:
\begin{code}
datatype service =
    DOMESTIC      of "DOMESTIC"
  | INTERNATIONAL of "INTERNATIONAL"
  | SPECIAL       of "SPECIAL"
\end{code}


\subsection{Complete Syntax}

Appendix~\ref{app:syntax-dd} presents the complete syntax of
\datatype{} data descriptions.

\cut{
%\subsection{The \pads{} Weltanshauung}
\subsection{Error Representation}

The data description primitives of the \pads{} language are types.
However, more than describing data alone, these types describe a
transformation from data in an external format to data in an internal
format.  Yet, in \pads{}, data does not live alone. A critical benefit
of the parsing is the meta-data gained during the transformation
process.  Therefore, the result of a parse is a pair consisting of a
canonical in-memory representation of the data and meta-data for the
parse.  We refer to the meta-data as the {\em parse descriptor} (PD).
The parse descriptor may hold a variety of bits of information
including the position of the data in the file and a characterization
of the possible errors in the data.  There are several different sorts
of errors that may arise.  Syntactic errors occur when a parser cannot
read a valid item of the right type from the file (eg: of parser
attempts to read an integer to find the character '?'  instead).
Semantic errors occur when a parser can read an item but it does not
satisfy the appropriate semantic condition (eg: the parse finds an
integer in the file, but the integer is $-1$ when it should be greater
than zero).  Furthermore, the \pads{} system does not view errors as
fatal. Through the parse descriptor, \pads{} is able to record the
presence of errors and continue (performing recovery as necessary),
instead of throwing an exception or halting the parse as may be done
in more conventional, ``data-only'' systems. Here, the parse
descriptors play a critical role in the functioning of the system,
beyond their otherwise ``passive,'' descriptive role.
}
%%% Local Variables: 
%%% mode: latex
%%% TeX-master: "paper"
%%% End: 
