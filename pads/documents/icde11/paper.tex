\documentclass{sig-alternate}

\usepackage{xspace,pads,amsmath,math-cmds,
            math-envs,inference-rules,times,
            verbatim,alltt,multicol,proof,url}
\usepackage{epsfig}
\usepackage{code} 
%\setlength{\oddsidemargin}{0in}
%\setlength{\evensidemargin}{0in}
%\setlength{\textwidth}{6.5in}
%\setlength{\textheight}{8.5in}

\begin{document}
\title{The PADS Project: An Overview}
\conferenceinfo{ICDT}{2011 Uppsala, Sweden}
\numberofauthors{2}
\author{
\alignauthor
Kathleen Fisher\\
  \affaddr{AT\&T Labs Research}\\
  \email{kfisher@research.att.com}
\alignauthor
David Walker\\
  \affaddr{Princeton University}\\
  \email{dpw@cs.princeton.edu}}

\newcommand{\cut}[1]{}
\newcommand{\reminder}[1]{{\it #1 }}
\newcommand{\edcom}[1]{\textbf{{#1}}}
\newcommand{\poplversion}[1]{#1}
\newcommand{\trversion}[1]{}

\newcommand{\appref}[1]{Appendix~\ref{#1}}
\newcommand{\secref}[1]{Section~\ref{#1}}
\newcommand{\tblref}[1]{Table~\ref{#1}}
\newcommand{\figref}[1]{Figure~\ref{#1}}
\newcommand{\listingref}[1]{Listing~\ref{#1}}
%\newcommand{\pref}[1]{{page~\pageref{#1}}}

\newcommand{\eg}{{\em e.g.}}
\newcommand{\cf}{{\em cf.}}
\newcommand{\ie}{{\em i.e.}}
\newcommand{\etal}{{\em et al}}
\newcommand{\etc}{{\em etc.\/}}
\newcommand{\naive}{na\"{\i}ve}
\newcommand{\role}{r\^{o}le}
\newcommand{\forte}{{fort\'{e}\/}}
\newcommand{\appr}{\~{}}

\newcommand{\bftt}[1]{{\ttfamily\bfseries{}#1}}
\newcommand{\kw}[1]{\bftt{#1}}
\newcommand{\pads}{\textsc{pads}}
\newcommand{\padsc}{\textsc{pads/c}}
\newcommand{\padx}{\textsc{padx}}
\newcommand{\ipads}{\textsc{ipads}}
\newcommand{\ir}{\textsc{IR}}
\newcommand{\padsl}{\textsc{padsl}}
\newcommand{\padsml}{\textsc{pads/ml}}
%\newcommand{\padsd}{\textsc{pads/d}}
\newcommand{\learnpads}{{\textsc{learnpads}}}
\newcommand{\padsd}{\textsc{Gloves}}
\newcommand{\blt}{\textsc{blt}}
\newcommand{\ddc}{\textsc{ddc}}
\newcommand{\ddl}{\textsc{ddl}}
\newcommand{\C}{\textsc{C}}
\newcommand{\perl}{\textsc{Perl}}
\newcommand{\ml}{\textsc{ml}}
\newcommand{\smlnj}{\textsc{sml/nj}}
\newcommand{\ocaml}{\textsc{OCaml}\xspace}
\newcommand{\haskell}{\textsc{haskell}\xspace}
\newcommand{\ocamlbig}{\textsc{OCAML}\xspace}
\newcommand{\java}{\textsc{java}}
\newcommand{\xml}{\textsc{xml}}
\newcommand{\html}{\textsc{html}}
\newcommand{\xpath}{\textsc{xpath}}
\newcommand{\xquery}{\textsc{xquery}}
\newcommand{\datascript}{\textsc{datascript}}
\newcommand{\packettypes}{\textsc{packettypes}}
\newcommand{\erlang}{\textsc{Erlang}}
\newcommand{\camlp}{\cd{Camlp4}}
\newcommand{\ocamlnet}{\cd{Ocamlnet} \cd{2}}

\newcommand{\totalcost}[2]{\textsc{Cost}(#1,#2)}
\newcommand{\costdescription}[1]{\textsc{CT}(#1)}
\newcommand{\normcostdescription}{\textsc{NCT}}
\newcommand{\costdata}[2]{\textsc{CD}(#2 \; | \; #1)}
\newcommand{\acostdata}[2]{\textsc{ACD}(#2 \; | \; #1)}
\newcommand{\adc}[2]{\textsc{CD'}(#2 \; | \; #1)}
\newcommand{\cardt}{\textsc{Card}}
\newcommand{\costvar}[1]{\textsc{CV}(#1)}
\newcommand{\costchar}[1]{\textsc{CA}(#1)}
\newcommand{\coststring}[1]{\textsc{CS}(#1)}
\newcommand{\costint}[1]{\textsc{CI}(#1)}
\newcommand{\costparam}[1]{\textsc{CP}(#1)}
\newcommand{\costconst}[1]{\textsc{CC}(#1)}

\newcommand{\dibbler}{Sirius}
\newcommand{\ningaui}{Altair}
\newcommand{\darkstar}{Regulus}

\newcommand{\vizGems}{Arrakis}

\newcommand{\comon}{CoMon\xspace}
\newcommand{\planetlab}{PlanetLab\xspace}
\newcommand{\monall}{Monall\xspace}
%% \newcommand{}{}


%% \newcommand{\IParray}[4]{{\tt Parray} \; #1 \; \[#2, #3, #4\]}

\newcommand{\figHeight}[4]{\begin{figure}[tb]
	\centerline{
	            \epsfig{file=#1,height=#4}}
	\caption{#2}
	\label{#3}
	\end{figure}}

\newcommand{\myalt}{\ensuremath{\; | \;}}
\newcommand{\normal}[1]{\ensuremath{\bar{#1}}}
\newcommand{\relativee}[2]{\ensuremath{{\cal R}(#1 \; || \; #2)}}
\newcommand{\srelativee}[2]{\ensuremath{{\cal S}(#1 \; || \; #2)}}
\newcommand{\addh}[2]{\ensuremath{#1 \oplus #2}}

\newcommand{\irstruct}[1]{{\tt struct}\{#1\}}
\newcommand{\irunion}[1]{{\tt union}\{#1\}}
\newcommand{\irenum}[1]{{\tt enum}\{#1\}}
\newcommand{\irarray}[1]{{\tt array}\{#1\}}
\newcommand{\irarrayFW}[2]{{\tt arrayFW}\{#1\}[#2]}
\newcommand{\irswitch}[2]{{\tt switch}(#1)\{#2\}}
\newcommand{\iroption}[1]{{\tt option}\{#1\}}
\newcommand{\setof}[1]{\lsem #1 \rsem}
\newcommand{\goto}{\Rightarrow}
\newcommand{\Pvoid}{{\tt Pvoid}}
\newcommand{\Pempty}{{\tt Pempty}}
\newcommand{\sskip}{\hspace*{5mm}}
\newcommand{\shrink}{\vspace*{-4mm}}

% Semantics
\newcommand{\setalt}{{\; | \;}}
\newcommand{\denote}[1]{\lsem #1 \rsem}
\newcommand{\lsem}{{[\![}}
\newcommand{\rsem}{{]\!]}}
\newcommand{\turn}{\vdash}
\newcommand{\meta}{m}
\newcommand{\nested}{n}
\newcommand{\mytime}[1]{#1.t}
\newcommand{\myds}[1]{#1.ds}
\newcommand{\myval}[1]{#1.nest}
\newcommand{\generatedloc}{\ensuremath{\mathtt{nowhere}}}
\newcommand{\environment}{E}
\newcommand{\universe}{U}
\newcommand{\selectOne}{\ensuremath{\mathsf{earliest}}}
% core feed semantics
\newcommand{\csemantics}[3]{{\cal C}\lsem #1 \rsem_{{#2} \, {#3}}}
% feed semantics
\newcommand{\semantics}[3]{{\cal F}\lsem #1 \rsem_{{#2} \, {#3}}}
% expression semantics
\newcommand{\esemantics}[2]{{\cal E}\lsem #1 \rsem_{{#2}}}
%\newcommand{\esemantics}[2]{#2(#1)}

% Host language types
\newcommand{\ty}{\ensuremath{\tau}}
\newcommand{\basety}{\ensuremath{b}}
\newcommand{\arrow}{\rightarrow}
\newcommand{\optionty}[1]{\ensuremath{#1 \; \mathsf{option}}}
\newcommand{\listty}[1]{\ensuremath{#1 \; \mathsf{list}}}
\newcommand{\setty}[1]{\ensuremath{#1 \; \mathsf{set}}}
\newcommand{\feedty}[1]{\ensuremath{#1 \; \mathsf{feed}}}
\newcommand{\corety}[1]{\ensuremath{#1 \; \mathsf{core}}}
\newcommand{\schedulety}{\ensuremath{\mathsf{sched}}}
\newcommand{\timety}{\ensuremath{\mathsf{time}}}
\newcommand{\locty}{\ensuremath{\mathsf{loc}}}
\newcommand{\boolty}{\ensuremath{\mathsf{bool}}}
\newcommand{\unitty}{\ensuremath{\mathsf{unit}}}
\newcommand{\stringty}{\ensuremath{\mathsf{string}}}
\newcommand{\metatype}[1]{\ensuremath{\mathsf{meta}(#1)}}
\newcommand{\nestedtype}[1]{\ensuremath{\mathsf{nest}(#1)}}
\newcommand{\dsty}{\ensuremath{\mathsf{ds}}}

\newcommand{\dom}{\ensuremath{\mathsf{dom}}}
\newcommand{\ueq}[3]{\ensuremath{#1 =_{#2} #3}}
\newcommand{\fsubset}[3]{\ensuremath{#1 \subseteq_{#2} #3}}
\newcommand{\feq}[3]{\ensuremath{#1 =_{#2} #3}}

% Expressions
\newcommand{\expression}{e}
\newcommand{\constant}{c}
\newcommand{\ds}{\ensuremath{ds}}
\newcommand{\boolf}{\ensuremath{\mathtt{false}}}
\newcommand{\boolt}{\ensuremath{\mathtt{true}}}
\newcommand{\loc}{\ensuremath{\ell}}
\newcommand{\feed}{\ensuremath{F}}
\newcommand{\corefeed}{\ensuremath{C}}
\newcommand{\generalvar}{\ensuremath{x}}
\newcommand{\feedvar}{\ensuremath{x}}
\newcommand{\itemvar}{\ensuremath{x}}
\newcommand{\data}{\ensuremath{v}}
\newcommand{\atime}{\ensuremath{t}}
\newcommand{\astring}{\ensuremath{w}}
\newcommand{\unit}{\ensuremath{()}}
\newcommand{\schedule}{\ensuremath{s}}
\newcommand{\parser}{\ensuremath{p}}
\newcommand{\none}{\ensuremath{\mathtt{None}}}
\newcommand{\some}[1]{\ensuremath{\mathtt{Some}\; #1}}
\newcommand{\inl}[1]{\ensuremath{\mathtt{inl}\; #1}}
\newcommand{\inr}[1]{\ensuremath{\mathtt{inr}\; #1}}
\newcommand{\casedata}[2]{{\tt switch}(#1)\{#2\}}
%\newcommand{\nillist}{\ensuremath{\mathtt{nil}}}
\newcommand{\nillist}{\ensuremath{[\,]}}
%\newcommand{\conslist}[2]{\ensuremath{\mathtt{cons} (#1,#2)}}
\newcommand{\conslist}[2]{\ensuremath{[#1,\ldots,#2]}}
\newcommand{\nilstream}{\ensuremath{\mathtt{done}}}
\newcommand{\consstream}[2]{\ensuremath{\mathtt{next} (#1,#2)}}


% Feeds
\newcommand{\comprehensionfeed}[3]{\ensuremath{\mathtt{\{|} #1 \; \mathtt{|}\; #2 \leftarrow #3 \mathtt{|\}}}}
\newcommand{\computed}[3]{\ensuremath{\mathtt{[} #1 \; \mathtt{|}\; #2 \in #3 \mathtt{]}}}
\newcommand{\letfeed}[3]{\ensuremath{\mathtt{let}\; #1 \; \mathtt{=}\; #2 \; \mathtt{in} \; #3}}
\newcommand{\allfeed}[5]{
  \ensuremath{
    \mathtt{all \{ format=} #1; 
    \mathtt{src=} #2;
    \mathtt{sched=} #3;
    \mathtt{pp=} #4;
    \mathtt{win=} #5;
  \mathtt{\}}}}
\newcommand{\existsfeed}[5]{
  \ensuremath{
    \mathtt{any \{ format=} #1; 
    \mathtt{src=} #2;
    \mathtt{sched=} #3;
    \mathtt{pp=} #4;
    \mathtt{win=} #5;
  \mathtt{\}}}}
\newcommand{\filterfeed}[2]{
  \ensuremath{
    \mathtt{filter} \; #1 \; \mathtt{with}\; #2}}
\newcommand{\remapfeed}[2]{
  \ensuremath{
    \mathtt{redirect} \; #1 \; \mathtt{with}\; #2}}
\newcommand{\ppfeed}[2]{
  \ensuremath{
    \mathtt{pp} \; #1 \; \mathtt{with}\; #2}}
\newcommand{\foreachupdate}[3]{
  \ensuremath{
    \mathtt{foreach{*}{*}}\; #1 \;
    \mathtt{in}\; #2 \;
    \mathtt{update}\; #3}}
\newcommand{\foreachcreate}[3]{
  \ensuremath{
    \mathtt{foreach*}\; #1 \;
    \mathtt{in}\; #2 \;
    \mathtt{create}\; #3}}
\newcommand{\remap}[2]{\ensuremath{\mathtt{redirect}\; #1 \; \mathtt{with} \; #2}}
\newcommand{\stutterfeed}[2]{\ensuremath{\mathtt{stutter}\; #1 \; \mathtt{on} \; #2}}
\newcommand{\refeed}[2]{\ensuremath{\mathtt{reschedule}\; #1 \; \mathtt{to} \; #2}}
\newcommand{\emptyfeed}{\ensuremath{\emptyset}}
\newcommand{\onefeed}[2]{\ensuremath{\mathtt{One}}(#1,#2)}
\newcommand{\sfeed}[1]{\ensuremath{\mathtt{SchedF}}(#1)}
\newcommand{\lfeed}[1]{\ensuremath{\mathtt{ListF}}(#1)}
\newcommand{\unionfeed}{\ensuremath{\cup}}
\newcommand{\sumfeed}{\ensuremath{+}}
\newcommand{\spairfeed}{\; \ensuremath{\mathtt{\&} \; }}
\newcommand{\allpairfeed}{\; \ensuremath{{*}{*}} \; }

\newcommand{\Time}{\ensuremath{\mathtt{Time}}}
\newcommand{\Set}{\ensuremath{\mathtt{Set}}}

% this is used for the translations equal
\newcommand{\transeq}{\stackrel{def}{=} }
\newcommand{\ai}{{\tt wl}}



% BNF
%\newcommand{\bnfalt}{\ |\ }


\maketitle{}

\begin{abstract}  
  The goal of the \pads{} project, which started in 2001, is to make
  it easier for data analysts to extract useful information from ad
  hoc data files.  This paper does not report new results, but rather
  gives an overview of the project
  and how it helps bridge
the gap between the unmanaged world of ad hoc data and
the managed world of typed programming languages and
databases.
  In particular, the paper reviews the design of \pads{}
  data description languages, describes the generated parsing
  tools and discusses the importance of meta-data.  It also sketches the formal
  semantics, discusses useful tools and how can they can be
  generated automatically from \pads{} descriptions, and describes
  an inferencing system that can learn useful \pads{}
  descriptions from positive examples of the data format.
\end{abstract}

\category{D.3.m}{Programming languages}{Miscellaneous}

\terms
Languages, Algorithms

\keywords
Data description languages, ad hoc data,  domain-specific languages


\section {Why PADS}
\label{sec:intro}

Traditional databases and \xml{}-based systems provide rich
infrastructure for managing well-behaved data, but provide little
support for \textit{ad hoc formats}, which we define to be any
semi-structured data format for which parsing, querying, analysis, or
transformation tools are not readily available.  Vast amounts of
useful data are stored and processed in such ad hoc formats, despite
the existence of standard formats for semi-structured data.  Examples
arise from a myriad of domains, including finance, health care,
transportation, telecommunications, and the sciences ~\cite{fisher+:pads}. 

For a number of reasons, processing ad hoc data is challenging.  Ad
hoc data typically arrives ``as is'': analysts are lucky to get the
data at all and have little chance of getting suppliers to standardize
its format.  Documentation for the format is often incomplete,
out-of-date, or completely non-existent.  Such data frequently contain
errors for a variety of reasons, including malfunctioning equipment,
non-standard representations of missing values, human error in
entering data \etc{}  Handling such errors is challenging because of
the variety of errors and because the appropriate response is 
application-dependent. Some applications require the data to be error
free and need to halt processing if any errors are detected.  Others
can simply discard erroneous values, while still others want to study
the errors in detail.  All of these challenges are exacerbated by the
fact that ad hoc data sources often have high volume.  For example, 
AT\&T's \ningaui{} project accumulates billing data at a rate
of 250-300GB/day, with occasional spurts of 750GBs/day.

Finally, someone has to write a parser for a new format before anyone
can use the data. People tend to use \C{}, \perl{}, or \python{} for
this task.  This approach is tedious and error-prone, complicated by
the lack of documentation, convoluted encodings designed to save
space, the need to produce efficient code, and the need to handle
errors robustly to avoid corrupting down-stream data.  Moreover, the
parser writers' hard-won understanding of the data ends up embedded in
parsing code, making long-term maintenance difficult for the original
writers and sharing the knowledge with others nearly impossible.

The \pads{} project started with the observation that an appropriately
designed, declarative data-description language could 
help bridge the gap
between the {\em unmanaged} world of ad hoc data and the {\em managed} world
of strongly typed programming languages and databases and thereby
help alleviate many of the concerns mentioned above.  To this end,
the language we have designed permits analysts to describe ad hoc 
data \textit{as it is}, not
how one might want it to be. The descriptions are concise enough to
serve as documentation and flexible enough to describe most of the
data formats we have seen in practice, including ASCII, binary, Cobol,
and mixed data formats.  From these descriptions, a compiler can
produce data structure declarations for representing the data in the
host language of the data description language as well as parsing and
printing routines.  Because the compiler is generating software
artifacts used to manipulate the data, analysts have to keep the data
description up to date, ensuring it can serve as living documentation.

\begin{figure*}[!ht]
\begin{alltt}
\begin{small}
171.64.78.97 - kfisher [31/Oct/2010:23:55:36 -0700] "GET /padsproj.org HTTP/1.1" 200 982
quantum.com  - - [31/Oct/2010:24:55:36 -0700] "GET /padsproj.org/logo.png HTTP/1.1" 200 2326
\end{small}
\end{alltt}
\vskip -.3in
\caption{A fragment of data in Apache Common Log Format.}
\label{fig:clf}
\end{figure*}

The declarative nature of \pads{} descriptions facilitates the
insertion of error handling code.  The generated parsing code checks
all possible error cases.  Because these checks appear only in
generated code, they do not clutter the high-level declarative
description of the data source.  The result of a parse is a pair
consisting of a canonical in-memory representation of the data and a
parse descriptor. The parse descriptor precisely characterizes both
the syntactic and the semantic errors that occurred during parsing.
This structure allows analysts to choose how to respond to errors in
application-specific ways.

Finally, a \pads{} description gives enough information about the
structure of the data that it is possible to generate automatically a
wide variety of useful tools customized to the particular format.
Examples of such tools include statistical analyses, format
converters, data adaptors to connect to query engines like
XQuery~\cite{fernandez+:padx}, and visualizers.  Using generic
programming techniques~\cite{fernandez+:padl,Lammel+:syb,mandelbaum+:pads-ml},
third party developers can design tools that will work for any \pads{}
description.



This paper does not describe new research, but rather collects and
summarizes the work done in the \pads{} project. The interested reader
is invited to read more about individual aspects of the project in the
original papers.  The project is the work of many people over a long
period of time; all contributors are listed in the acknowledgements
section of the paper.

The rest of the paper is organized as follows. \secref{sec:language}
explains what a \pads{} data description language looks like by
working through a simple example. \secref{sec:compiler} describes the
output of a \pads{} compiler.  \secref{sec:formal} describes a formal
semantics for data description languages, which serves as a
specification for how \pads{} data description languages should
behave, regardless of the host language in which they are embedded.
\secref{sec:tools} describes the kinds of tools we can generate from
\pads{} descriptions as well as how generic programming makes third-party
tool generation possible.  \secref{sec:inference} describes how we can
leverage large quantities of data to learn \pads{} descriptions rather
than having to write them entirely by hand.  \secref{sec:related} briefly
reviews related work and \secref{sec:conclusion}
concludes.



\section{What you say}
\label{sec:language}



In this section, we briefly describe what a \pads{} data description
language looks like by working through a simple example. 
\figref{fig:clf} gives a small fragment of a web access log in Apache
Common Log Format (CLF)~\cite{apacheclf}.  Each time an Apache web server
receives a request, it writes such an entry to its access log.  The
first field denotes the IP address or host name of the client making
the request.  The next two columns denote the identity of the client
user, the first as determined by the \texttt{identd} function on the
client machine and the second as determined by HTTP authentication.
If the information is not available, the server writes a \texttt{-}
instead. The next field is the time stamp of the request, in brackets.
It is followed by the request itself in double quotes. The request has
three pieces: the HTTP method used by the client, the requested url,
and the version of HTTP used for the transaction.  The format ends
with two integers, the first of which is the three-digit response code
sent back to the client and the second is the number of bytes
returned.  If no bytes were returned, this last number will be a dash
instead.  

\pads{} uses a type metaphor to describe ad hoc data.  Base types
describe atomic pieces of data and type constructors describe how to
build compound descriptions from simpler ones.  Each \pads{} type
plays two roles: it defines a grammar for parsing the data
\textit{and} a format-specific data structure in which to store the
result of the parse.  We have developed versions of \pads{} for
\C{}~\cite{fisher+:pads} and \ml{}~\cite{mandelbaum+:pads-ml}, both of
which are available for download on the web with an open-source
license~\cite{padsweb}. We are currently developing a version for
\haskell{}, which we will use as the example language in this paper.
For each language binding, we re-use the types of the host language in
the data description language.  \pads{} also uses predicate expressions
in the host language to describe semantic properties of the data, such
as the range of integer values or correlations between data items.


\begin{figure}[t]
\begin{code}
[pads| 
  \kw{type} CLF_t = [Line Entry_t]               
  \kw{data} Entry_t = 
    \{       host       :: Source_t, 
      ' ',  identdID   :: ID_t, 
      ' ',  httpID     :: ID_t, 
      ' ',  time       :: TimeStamp_t, 
      ' ',  request    :: Request_t,
      ' ',  response   :: Response_t,
      ' ',  contentLen :: ContentLength_t 
    \}
  \mbox{}
  \kw{data} Source_t = IP IP_t | Host Host_t  
  \mbox{}
  \kw{data} ID_t = Missing '-' | Id (Pstring ' ')
  \mbox{}
  \kw{data} Request_t = 
    \{ '"',  method  :: Method_t,       
      ' ',  url     :: Pstring ' ', 
      ' ',  version :: Version_t, 
      '"' \}  
  \mbox{}
  \kw{data} Version_t  = 
    \{"HTTP/", major :: Pint, '.', minor :: Pint \}  
  \mbox{}
  \kw{data} Method_t = GET | PUT | POST | HEAD
                | DELETE | LINK | UNLINK
  \mbox{}
  \kw{type} Response_t = \kw{constrain} r :: Pint 
       \kw{where} <| 100 <= r && r < 600 |> 
  \mbox{}
  \kw{data} ContentLength_t = NotAvailable '-' 
                       | ContentLength Pint 
|]
\end{code}
\vskip -.2in
\caption{\padshaskell{} description of CLF data.}
\label{fig:clf-haskell}
\vskip -.1in
\end{figure}


The \padshaskell{} description in \figref{fig:clf-haskell} describes
the CLF data format.  We embed \pads{} in \haskell{} using \haskell{}'s
quasi-quoting mechanism~\cite{Mainland:quasi}.  All of the code
inside the quasi-quotes \texttt{[pads|...|] }  is \padshaskell{}; the
identifier \texttt{pads} tells the \haskell{} parser which quasi-quoter to use to
process the enclosed code.  Any code written outside of the quasi-quotes is 
ordinary \haskell{}.  Moreover, \haskell{} declarations written
prior to such quotes are in
scope and available in the \padshaskell{} code.  At compile time, the \haskell{}
compiler calls the \pads{} quasi-quoter to convert the \padshaskell{}
code to plain \haskell{} declarations, which are spliced into the source
code at the point of the quasi-quotation.  These declarations are subsequently
in scope for any \haskell{} code that follows.  This mechanism enables us to
completely control the syntax of our data description language while
still inter-operating with the host language.  Indeed, the ability to alternate
back and forth at will between \pads{} and \haskell{} declarations
provides a smooth, flexible and pleasing programming experience --- the two
languages act as one.

The first line of the description declares the type \texttt{CLF\_t},
which describes the entirety of an access log.  It says that such a
log is a list of lines of type \texttt{Entry\_t}.  The next line
declares the type \texttt{Entry\_t}, which is a record describing a
single entry in the log.  It says that an entry is a sequence of seven
fields; each field is given a name and an associated type.  For
example, the \texttt{host} field has type \texttt{Source\_t}, which is
also declared in the figure.  In addition to the named fields, there
are literal characters in the record declaration, which correspond to
literals in the data.  For example, between the \texttt{host} and
\texttt{identdID} fields, there is a space character (\texttt{' '}).

The \texttt{Source\_t} type is an interesting example of a
datatype: a value of this type is \textit{either} an IP address
\texttt{IP\_t} or a \texttt{Host\_t}, where \texttt{IP\_t} and
\texttt{Host\_t} are \pads{} \textit{base types} describing IP
addresses and host names, respectively.  The branches of a datatype
are attempted in order; the parser greedily selects the first branch
that matches.   The labels \texttt{IP} and \texttt{Host} tag the
parsed value as belonging to the first or second branch, respectively.
Type \texttt{ID\_t} is another datatype. In this case, the first
branch corresponds to a data format with the literal \texttt{'-'}. The
type \texttt{Method\_t} is a datatype where the branch labels
correspond to the data, and so the argument type to the branch label
is omitted.  

Base types describe atomic pieces of data.  Examples of base types include integers
(\texttt{Pint}) and floats (\texttt{Pfloat}), single characters
(\texttt{Pchar}) and strings (\texttt{Pstring ' '}), IP addresses
(\texttt{IP\_t}) and host names (\texttt{Host\_t}), dates
(\texttt{Date\_t}), and many others. 
The type
\texttt{Pstring ' '} is an example of a \textit{parameterized type}.
In general, a string could go on forever.  The parameter specifies
when the string should stop, in this case, when the parser encounters
a space.  To account for more general stopping conditions,
\padshaskell{} provides the base type \texttt{PstringME}, which takes
a regular expression as a parameter.  This type matches the longest
string that matches the argument regular expression.

Finally, the type \texttt{Response\_t} is an example of a
\textit{constrained type}. It specifies that a \texttt{Response\_t} is
a \texttt{Pint} between 100 and 599, inclusive. In the declaration,
the variable \texttt{r} is bound to the result of parsing the input as
a \texttt{Pint}, after which the predicate given in the \texttt{where}
clause is evaluated. The type matches if the predicate evaluates to
\texttt{True}.  The brackets \texttt{<|...|>} indicate that
the enclosed code is pure \haskell{} code.



\section{What you get}
\label{sec:compiler}
From a \pads{} specification, the compiler generates a pair of data
structures: one for the in-memory representation of the parsed data
and an isomorphic structure for meta-data such as the number and type
of errors.  The form of the generated representation
type corresponds to the form of the type in the \pads{} description: \pads{}
lists compile to \haskell{} lists, records to records, and data types to
data types. Constrained types map to the representation of the
underlying type.  \figref{fig:haskell-rep} gives a selection of the
representation types generated for the CLF description.
\begin{figure}
\begin{code}
\kw{newtype} CLF_t = CLF_t [Entry_t]
\kw{data} Entry_t = Entry_t  
         \{host       :: Source_t,
          identdID   :: ID_t,
          httpID     :: ID_t,
          time       :: TimeStamp_t,
          request    :: Request_t,
          response   :: Response_t,
          contentLen :: ContentLength_t\}
\kw{data} Source_t = IP IP_t | Host Host_t
\kw{data} ID_t = Missing | Id Pstring
 ...
\kw{newtype} Response_t = Response_t Pint
\end{code}
\vskip -.2in
\caption{Generated representation types.}
\label{fig:haskell-rep}
\end{figure}

\begin{figure}[ht!]
\begin{code}
\kw{type} CLF_t_md   = (Base_md, [Entry_t_md]),
\kw{type} Entry_t_md = (Base_md,  Entry_t_inner_md),
\kw{data} Entry_t_inner_md = Entry_t_inner_md 
         \{host_md       :: Source_t_md,
          identdID_md   :: ID_t_md,
          httpID_md     :: ID_t_md,
          time_md       :: TimeStamp_t_md,
          request_md    :: Request_t_md,
          response_md   :: Response_t_md,
          contentLen_md :: ContentLength_t_md\}
\kw{type} Source_t_md = (Base_md, Source_t_inner_md)
\kw{data} Source_t_inner_md
       = IP_md IP_t_md | Host_md Host_t_md
\kw{data} ID_t = Missing | Id Pstring
\kw{data} ID_t_inner_md
       = Missing_md Base_md 
       | Id_md (Base_md, Base_md)
  ...
\kw{newtype} Response_t = Response_t Pint
\end{code}
\vskip -.2in
\label{fig:haskell-md}
\caption{Generated meta-data types.}
\end{figure}


The meta-data types, shown in 
\figref{fig:haskell-md}, have a similar structure.
These declarations make use of the type
\texttt{Base\_md} to describe the number and type of errors detected
during parsing: 
\begin{code}
\kw{data} Base_md = Base_md 
        \{ numErrors :: Int,
          errInfo   :: Maybe ErrInfo \}
\kw{data} ErrInfo = ErrInfo 
       \{ msg      :: ErrMsg,
         position :: Maybe Position \}
\end{code}



Each generated meta-data type pairs a generic \texttt{Base\_md} with a
type-specific meta-data structure.  The \texttt{Base\_md} type
summarizes the errors that occurred within the corresponding
structure while the type-specific meta-data localizes error
information.  This structure allows analysts to handle errors in
application-specific ways.  By checking the top-level
\texttt{Base\_md} value, analysts can determine whether there were
any errors during parsing. If the error count is zero, they know 
the data parsed without any errors and all the semantic predicates
held.  If the error count is non-zero, they can traverse the
meta-data structure to determine precisely where the errors occurred
and what caused them, allowing the analysts to decide whether to discard,
repair, or study the errors. This design also means that the
representation is not cluttered with option types indicating that each
value could be absent because of an error during parsing.  If an error
occurs while parsing a base type, the compiler fills in an appropriate
default value and marks the meta-data accordingly.


Finally, the compiler generates a function that parses
an input file into a pair of a representation and a meta-data
structure.  In \haskell{} terms, each generated representation (\texttt{rep}),
meta-data (\texttt{md}) pair belongs to the \texttt{Pads} type class and provide
a definition for the \texttt{parseFile} method:
\begin{code}
parseFile :: Pads rep md => 
             FilePath -> IO (rep, md)
\end{code}
The return type \texttt{IO(rep,md)} indicates that the function
produces a value of type \texttt{(rep,md)} while causing side-effects
such as opening and closing file handles.  A particular instance of
this function parses values for the type \texttt{CLF\_t}:
\begin{code}
parseFile :: FilePath -> IO (CLF_t, CLF_t_md)
\end{code}
The generated parser is a simple recursive-descent parser.  This
parsing strategy makes it easy for values early in the parse to affect
down-stream choices, for example, to read an integer that determines
the length of an upcoming list or a tag that predicts what form the
body of a record will take.  While generally satisfactory, recursive
descent parsers cannot parse left-recursive grammars and
can require exponential time (if there is a lot of
backtracking).  Developing always-efficient parsers suitable for
powerful \pads{}-like grammars is currently an active research question~\cite{Jim+:yakker}.

Although not yet implemented in \padshaskell{}, \padsc{} and
\padsml{} both also generate a pretty printing function that takes a
pair of a representation and meta-data structure and serializes the
representation to a file.  In Haskell, this function will have the
signature:
\begin{code}
printFile :: Pads rep md => 
             (rep, md) -> FilePath -> IO ()
\end{code}



\section{What it means}
\label{sec:formal}
The close correspondence between \pads{} descriptions and the type
structure of the host language makes the meaning of \pads{}
descriptions relatively intuitive, but it does not suffice to
precisely define their semantics.  To address this deficiency, we
developed a formal calculus, called \ddca{}\hskip -5pt,\hskip 5pt based on dependent type
theory~\cite{Fisher+:ddca}.  

We defined a denotational semantics for \ddca{} that interprets
each term in multiple ways.  In the first interpretation, each \ddca{}
term $\tau$ is mapped to a type that we call its ``representation''
type, $\tau_{rep}$.  This type describes the data structure that stores the
host-language representation of the parsed value.  In the second
interpretation, each \ddca{} term $\tau$ is mapped to a type that we
call its ``meta-data'' type, $\tau_{md}$.  This type describes the data
structure that stores the host-language representation of the
meta-data generated during parsing.  In the third interpretation, each
\ddca{} term $\tau$ is mapped to a parsing function.  This parsing
function takes as input a string to be parsed and returns a pair with
type $(\tau_{rep}, \tau_{md})$.

We precisely characterized the \textit{canonical} relationship between a
representation value and a meta-data value for the two to be
meaningfully paired.  This canonical relation enforces the property
that the meta-data structure captures the errors in the
representation.  We then showed that for every \ddca{} term $\tau$,
the generated parser returns a representation and a meta-data value
that are related via the canonical relation, guaranteeing that the
meta-data structure returned by the parser precisely captures the
errors detected during parsing. 

In addition, we showed how to translate \pads{} declarations into terms of
\ddca{} to document precisely the semantics of those declarations.
This process allowed us to find several bugs in the \padsc{}
implementation and guided the design of later versions of \pads{}
including \padsml{} and \padshaskell{}.  Moreover,
the \ddca{} calculus is general enough that it also allowed us to
define formal semantics for interesting elements of other data description languages,
including \packettypes{}~\cite{sigcomm00} and
\datascript{}~\cite{gpce02}. 

We eventually extended \ddca{} to add a fourth semantic function,
corresponding to a printing function.  We explored under what
conditions parsing followed by printing or printing followed by
parsing is the identity function~\cite{fisher+:ddc-printing}.  This
question is non-trivial because various parsing functions throw away
information from the input, such as the number of white space
characters between two values.  This loss makes it impossible to
precisely regenerate the output in all cases.  Of course, it would always 
be possible to change
the parser to retain enough information to ensure round-tripping laws for
parsing and printing, but it is unclear whether the practical price is worth the theoretical
gain.



\section{What else you get}
\label{sec:tools}
A key insight behind the \pads{} project is that once someone has
written a description, it is possible to generate a wide variety of
additional tools besides a parser and a printer because the description
tells the computer a lot about the data.  Each version of \pads{}
can generate a number of such tools fully automatically from any
description.  We describe a few of the most useful tools below.


\paragraph{Accumulator} With large data sets, it can be difficult to get a
  ``bird's eye'' view of the data, which requires developing a sense
  of what the data ``usually'' looks like, what fields have a lot of
  variation, what the representations for missing values are, \etc{} 
  The accumulator tool is designed to help with this problem.  
  It runs over large volumes of data with the designated format,
  accumulating a variety of different statistics for each part of the
  structure.  When all the relevant data has been
  processed, the accumulator generates an informative statistical report.
  For base types, the accumulator reports
  information relevant to that type.  For example, for integers, the
  tool reports the minimum, maximum, and average values, as well as a
  histogram of the most commonly seen values, precisely tracking all
  values up to a customizable limit.  For strings, the accumulator
  reports the observed lengths of the strings and a histogram of
  observed values.  For structured types, the accumulator tool reports
  summaries of the components of the types.  For lists, it reports the
  various lengths of the list observed in the input.  For datatypes,
  it reports the relative frequencies of the various branches.  

A common use of the accumulator tool is in writing \pads{}
descriptions.  It is typical to write an initial description that
covers a representative sample of the data source, using string base
types to specify poorly-understood portions of the data.  From this
description, the analyst generates and runs the accumulator tool,
which reports both the records in the input that do not match the
description and distributions on the place-holder strings.  Both
pieces of information allow the analyst to refine the description and
iterate.  This process helps in developing descriptions where the data
file is large and has variation throughout the file, making it
impossible for human beings to see all the variation without automated
assistance. 

A demo of a \pads{}  accumulator is available from the
website \url{http://www.padsproj.org/learning-demo.cgi}.

\paragraph{XML Converter} \pads{} descriptions typically describe
semi-structured data, which makes it natural to represent the same
information in XML.  Because XML is a standard representation for
semi-structured data, there are many tools available to manage XML
data.  To leverage this infrastructure, we developed a tool to convert
any \pads{} description into a corresponding, format-specific XML
Schema and any data matching the \pads{} description into XML that
matches the generated Schema. The \pads{} website also has a demo of
this tool. 

\paragraph{Relational Converter} Although not all \pads{} descriptions
describe essentially relational data, some do, and for such
descriptions, it can be useful to convert the raw data into a
``cleaned-up'' form suitable for loading into a relational database.
The common log format we saw in \secref{sec:language} is an example of
such a data source.  Its raw form is difficult to include via a
typical database import function because of the extraneous
punctuation, but conceptually it is a simple table.  The \pads{}
relational converter tool maps the raw data into a delimited column
form, where the user can specify the delimiter.  We have used this
tool at AT\&T to import data into the Daytona database system. 
Again, the \pads{} website has a demo of this tool. 

\paragraph{XQuery Integration}
An obviously useful tool for ad hoc data sources is the ability to
query the data.  Inventing an entirely new query language for what is
essentially semi-structured data seemed like reinventing the wheel.
Instead, we decided to develop a tool that would allow analysts to
query any data source with a \pads{} description using
\xquery{}~\cite{xquery}.  An obvious approach to this integration
would be to use the XML converter to translate the original data into
XML and then run an XQuery implementation on the resulting document.
However, the large amount of extra space required to represent the
data in XML, typically a factor of eight, led to unacceptable
performance with this approach.  Instead, we were able to leverage the
abstract data model provided by the Galax~\cite{galax} implementation
of \xquery{} to enable Galax to query \pads{} data directly.  The
original implementation of this tool~\cite{fernandez+:padx} only
allowed query results to be returned in XML, but a subsequent extension
allowed results to be mapped back into the original
form~\cite{fernandez+:padl}. 

\paragraph{Harmony integration}
The Harmony synchronization framework~\cite{harmony-web} allows
two replicas of a document to be synchronized with each other.
Internally, Harmony works on unordered trees.  Synchronizing
particular data formats requires writing viewers to map between the
on-disk representation and Harmony's tree model.  To avoid having to
write such viewers by hand, we wrote a tool to automatically convert
any data with a \pads{} description into the required format and
back. Together, \pads{} and Harmony allow effective, semantics-preserving synchronization of arbitrary ad hoc data sets.

\subsection{Implementing tools}
The best way to implement these description-specific tools has been
the subject of on-going research.  In the original implementation of
\pads{}, the compiler generated these tools.  This approach gives a
lot of flexibility and is fairly straightforward to implement, but it
means that only compiler writers can add new tools, a significant
limitation.  

To address this problem, the \padsml{} compiler generates
generic ``traversal functions''~\cite{fernandez+:padl,mandelbaum+:pads-ml} 
in addition to the standard type declarations and parsing functions.
Using this infrastructure, third-party developers can write tools
without having to change the compiler.  However, the downside
is that the interface to the set of generic
functions is extremely complex. The complexities in the interface
arose because the host language for \padsml,
\ocaml{}, does not provide direct support for generic programming,
In order to obtain the generic programming facilities we required,
we were forced to implement sophisticated type-directed algorithms in the
structures and functors supplied by \ocaml{}'s powerful
module system.  The result was a system that is expressive enough to
accomplish the desired tasks, but useable only by extreme experts.  

One
of the motivations for building a version of \pads{} in \haskell{} is
that \haskell{} does provide a lot of support for generic
programming~\cite{Lammel+:syb,Wadler+:typeclass}.  We anticipate that
writing third party tools in the \padshaskell{} framework will be
significantly easier.

\section{Something for free}
\label{sec:inference}
The time and expertise required to write a \pads{} description from
scratch can be a significant impediment to using the system.
Depending on the complexity of a data source, it can take hours to
days to produce a comprehensive \pads{} description.  To shorten this
process, we have developed a system that automatically infers a
\pads{} description from multiple positive examples of the data
format~\cite{Fisher+:dirttoshovels}.  This learning process can be
connected to the \pads{} tool infrastructure to automatically produce
tools to generate accumulator reports or XML representations of ad hoc
data sources without any human intervention.  A demo of this
capability is available on the web
\url{http://www.padsproj.org/learning-demo.cgi}. 

The inference system works in a series of stages.  In the first stage,
the input data is broken into chunks, each of which is a positive
instance of the data format to be learned.  We require the user to
tell us how to do this division.  Typical examples include breaking a
file on newline boundaries or treating each file in a collection of
files as an instance.  
%It may be possible to infer how to do this
%chunking automatically, but that is a difficult research question and
%in all the examples we have considered so far, it is trivial for the
%user to determine this information. 

In the second stage, we convert each chunk into a sequence of tokens,
where the collection of possible tokens is specified using regular
expressions.  By default, the system provides tokens for integers,
floats, various kinds of strings, white space, and punctuation.  It
also provides domain-specific tokens for systems-like data, such as IP
addresses, MAC addresses, email addresses, dates, and times.  The
intuition is that this collection should include atomic pieces of data
that a human would glance at and know what it means with 100\%
confidence.   The system is parameterized so users can provide their
own set of regular expressions. 

In the third stage, the system computes a histogram for each token,
counting the number of records in which the token appears zero times,
one time, two times, \etc{}  Tokens with similar histograms are
clustered, based on a similarity metric.  The cluster that ``best
describes'' the data is selected, using a heuristic that rewards high
coverage, meaning the cluster appears in almost every record, and
narrowness, meaning that the tokens in the cluster appear with the
same frequency in almost all records.  For example, if every record
had exactly one comma and two quotation characters, then the comma and
two quotation tokens would be clustered and that cluster would be
selected.  The system next partitions the input based on the selected
cluster, with one partition for each observed order for the cluster
tokens and an extra partition for the records that do not contain all
the tokens in the cluster.  This collection of partitions will
correspond to a datatype in the eventual description, with one branch
for each non-empty partition.  (In the case where there is only one
partition, this datatype is omitted from the inferred description.)
Within a partition, all records have all the tokens in the cluster in
the same order.  Each such partition will correspond to a record type
declaration in the generated description. This record type contains
each of the tokens in the appropriate order.  To infer the description
for the data between these tokens, the system divides each input
record into the tokens before the first cluster token, between the
first and second cluster token, \etc{} Each of these groups is then
recursively analysed to produce a description that is slotted into the
top-level record declaration.

In the fourth stage, we greedily search for the best possible description
in the nearby area.
This search is executed by successively applying rewriting 
rules to the inferred description and scoring the results of the
rewrites.  The search continues until it is no longer possible to
rewrite the description in such a way as to obtain a new description
with a superior score.  The scoring function itself
uses an information-theoretic measure called
the Minimum Description Length (MDL)~\cite{mdlbook} to evaluate the
quality of any description.  This measure counts the complexity of the description
and the complexity of the data given the description, thereby
penalizing both simplistic descriptions (like \texttt{String}) that
cover the data without adding any information as well as overly
complex descriptions, such as the description that specifies each
character in order.  

\subsection{Learning tokenizations}
The learning algorithm is very sensitive to how the input is
tokenized.  For certain basic types, like filepaths, the regular
expression that defines legal file paths is very general.  Almost
every string of characters is a file path, but that does not mean it
is \textit{likely} that every string of characters is a file path.  It
is fairly easy for human beings to look at a data set and identify the
filepaths.  We explored whether machine-learning techniques could help
us develop a tokenizer that could capture this concept 
effectively~\cite{Xi+:padl}. The result of this study was that the
learning system required a lot of data and the inference process was
slower, but the quality of the inferred descriptions improved.  Still,
more research in this area may well improve the quality of descriptions
relative to the time required to learn them.

\subsection{Incremental Inference}
The original inferencing algorithm produces a description exclusively
from a relatively small, single input data set.  Unfortunately,
this original design made it impossible to use inference
to improve an existing description, to process streaming data
or to process large data sets.  To address these weaknesses, we
are in the process of developing an incremental version of the
learning system~\cite{Zhu+:wasl}.  In this version, the learning system optionally
takes an existing description as input in addition to the data.
The output of the system is a new description that covers all the new
data while diverging from the original description as little as
possible.  

This architecture allows us to scale to larger data sets by
iterating the inference process.  We can either start with a supplied
description or use the original learning algorithm to produce a
description from a subset of the supplied data.  We then divide the
remaining data into groups of appropriate size and iteratively apply
the incremental algorithm to these groups, eventually returning a
description that covers the entirety of the original data set.
We are in the process of evaluating the effectiveness of this approach
and how well it scales.

\subsection{Putting humans in the loop}
Even if format inference were perfect, we would still need human
involvement to produce the high-quality descriptions.  At the very least, it
is not possible for the computer to infer meaningful labels for fields
in records.  For example, the computer cannot tell if a given IP
address is a source or a destination.  In the end, it is likely that the
best system for inferring descriptions will use a combination of
machine learning techniques and a high-power user interface that lets
humans explore the data and edit descriptions effectively.

One kind of human-friendly interface we have explored in detail
is a new sort of {\em markup language} for raw text~\cite{xi+:anne}.  This
markup language, called \anne{}, helps users 
interactively generate \pads{} descriptions with human-readable names
and the exact structure desired.  Given a new ad
hoc data source, an \anne{} programmer edits the document to add
annotations that are somewhat akin to XML tags, yet contain bits of
grammatical information that serve to specify the syntactic
structure of the document. 
These annotations include elements that specify constants,
optional data, alternatives, enumerations, sequences, tabular data,
and recursive patterns. The \anne{} system uses a combination of
user annotations, smart defaults, and the raw data itself to extract a \pads{}
description from the document. This \pads{} description can then
be used to parse the data and transform it into an XML parse tree,
which may be viewed through a browser for analysis or debugging
purposes.  The description can also be saved as documentation
or used to generate any of the other \pads{} tools.  Like other
languages in the \pads{} family, \anne{} has a formal semantics.
This time, the semantics is based on concepts drawn from
Relevance Logic~\cite{anderson+:relevance-logic} as opposed to type
theory.  We used the semantics to prove a number of
interesting properties concerning the expressiveness of 
\anne{} and the conditions under which it is able 
to extract a desired context-free grammar from a document.

One of the inspirations for \anne{} was an earlier system called
\launchpads{}~\cite{planx06:launchpads}. This tool used a graphical user
interface to help human beings write descriptions.  The tool 
presents users with sample input and provided a tool palette for
introducing structure such as lists, datatypes, and records.  An
integration of a visual tool like \launchpads{} or a text-based
annotation language like \anne{} and the incremental
inferencing algorithm is likely the best approach to producing
high-quality descriptions quickly.
   

\section{What others have done}
\label{sec:related}
There is a vast literature on parsing, so here we only briefly review
related work on data description languages.  Our paper defining the
semantics of \pads{} contains a detailed discussion of a large body of
related work~\cite{Fisher+:ddca}.  The interested reader is invited to
consult that paper for a more detailed discussion.  In addition, each
of the \pads{} papers mentioned here discusses the relevant related
work.  For more information on particular aspects of the \pads{}
project, consult the relevant paper.

There are many data description languages for designing data formats,
including ASN.1~\cite{asn} and ASDL~\cite{ASDL}, or, more recently,
Google Protobufs~\cite{protobufs} and Apache Avro~\cite{avro}.  These declarative
languages allow programmers to describe the \textit{logical representation} of
data and then automatically generate a \textit{physical
  representation} and functions to map between the two representations.
Although useful for many purposes, such tools are of little use when
the physical representation is already fixed, which is the domain that
\pads{} targets.

Traditional parsing systems such as \textsc{YACC} generate parsers from
declarative specifications; however, they are not particularly well
suited for writing data descriptions.  In particular, such systems
generally require users to write a separate lexer and construct
in-memory data structures by hand.  They typically only work on ASCII
data and do not allow data-dependent parsing.

The languages and systems that are most closely related to \pads{} are
\packettypes{}~\cite{sigcomm00} and \datascript{}~\cite{gpce02}.  
Each of these systems allow declarative
descriptions of physical data, motivated respectively by the goals of
parsing \textsc{TCP/IP} packets and \java{} jar-files.  As
with \pads{}, these languages all use a type-directed approach to
describing physical data formats and permit the user to specify
semantic constraints in a host language.  These systems differ from
\pads{} in focusing only on binary data and assuming that the data is
error free, halting if an error is detected.  In addition, these
systems focus on the parsing problem, rather than also providing a
body of auxiliary tools.  None of these systems attempt to infer
descriptions from raw data. 



\section{Where we go from here}
\label{sec:conclusion}
Various problems remain open.  
Developing new tools is still difficult.  We anticipate that working
in the context of \haskell{}, which has an active research community
in generic and type-directed programming, will help in this area.  
Format inferencing is reasonably successful, but we believe the
inferencing process can be further improved by incorporating more
advanced machine-learning techniques and by including the user in
strategic decisions.  Finally, to get widespread adoption of the
technology, it is likely necessary to integrate a version of \pads{}
more tightly into standard tools for manipulating data. 

%  \newpage

\section*{Acknowledgments}
The \pads{} project has been a collaboration involving a large number of
people from a variety of institutions, including:
Andy Adams-Moran,
David Burke,
Mark Daly,
Zach DeVito,
Pamela Dragosh,
Mary Fern\'andez,
Andrew Forrest,
J. Nathan Foster,
Artem Gleyzer,
Joel Gottlieb,
Michael Greenberg,
Robert Gruber,
Vikas Kedia,
John Launchbury,
Yitzhak Mandelbaum,
Ricardo Medel,
Frances Spalding,
%David Walker,
Peter White,
Qian Xi,
Xuan Zheng, and
Kenny Q. Zhu.

Portions of this material are based upon work 
supported under NSF grants CCF-1016937,  0238328,  0612147, and 0615062 and
supported by DARPA under grant FA8750-07-C-0014.
Any opinions, findings, and conclusions or recommendations
   expressed in this material are those of the authors and do not
   necessarily reflect the views of the NSF or DARPA.

%\newpage

%\bibliographystyle{plainnat}
\bibliographystyle{abbrv}
\bibliography{pads}


\end{document}

%%% Local Variables:
%%% mode: outline-minor
%%% End:

