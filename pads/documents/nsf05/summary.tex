%%%%%%%%%%%%%%%%%%%%%%%%%%%%%%
% \centerline{
% \begin{tabular}{cc}
%           \\[-1ex]
% \multicolumn{2}{c}{\Large{} Tools for Processing Ad Hoc Data Sources} \\
%           & \\[-1ex]
% Kathleen Fisher & David Walker (PI)\\
% AT\&T  & Princeton University\\[2ex]
% \end{tabular}}
%%%%%%%%%%%%%%%%%%%%%%%%%%%%%%

An {\em ad hoc data format} is any nonstandard data format
for which parsing, querying, analysis or transformation
tools are not readily available.
There are vast amounts of useful data 
stored in traditional databases and \xml\ formats, but there is just
as much in ad hoc formats.
In the networking and telecommunications
domain, we find ad hoc formats for web server logs~\cite{wpp}, 
netflows capturing internet traffic~\cite{netflow}, 
log files characterizing IP backbone resource utilization,
wire formats for legacy telecommunication billing systems, 
and telephone call detail data~\cite{hancock-toplas}, to name just a few.
The ability to parse, query, monitor and transform this ad hoc data
is a crucial part of maintaining and providing security for
our communications infrastructure.

Processing ad hoc data is challenging for a variety of
reasons. First, ad hoc data arrives ``as is:'' an analyst might want the
data delivered in \xml, but unfortunately, he or she has just got to deal
with it as it is.
Second, documentation for the format may not exist at all, or it may be
out of date.  
Third, ad hoc data frequently contain errors, for a variety of
reasons: malfunctioning equipment, race conditions on log
entry~\cite{wpp}, malicious agents attempting to exploit
software vulnerabilities, human error in entering data, and others. 
The appropriate response to such errors depends on the application. 
% Some applications require the data to be error free: 
% if an error is detected, processing needs to stop immediately and a human
% must be alerted.  Other applications can repair the data, while still
% others can simply discard erroneous or unexpected values.  
For some applications, errors or anomalous data can be the most 
interesting part because they can signal potential network
vulnerabilities, opportunities or ongoing fraud in telephony systems, 
or failure to communicate between systems.  A fourth challenge is that 
ad hoc data sources can be high volume.  For instance,
AT\&T's call-detail stream contains roughly 300~million calls per day
requiring approximately 7GBs of storage space.

Work has begun on developing a high-level specification 
language called PADS for describing ad hoc data formats.  From a
PADS specification it is currently possible to generate a number of
tools for processing that data automatically including (1)
efficient parsers for reading ad hoc data, (2) an interface to a 
general-purpose data query engine, (3) an analyzer that generates 
statistical summaries of data read by the parser, and (4)
a translator from the ad hoc format into \xml.  

While the work to date demonstrates the feasibility of the PADS
approach, the PADS design and implementation are still in their infancy:
We have discovered many common, real-world data formats that the current PADS 
infrastructure is incapable of describing, parsing or analyzing. 
To address these deficiencies, we propose to explore a number of different ways
of extending the basic PADS specification language and related
infrastructure.  In addition, we propose to augment PADS with
facilities for programming common data transformations at a
high level of abstraction.  These new transformational capabilities
will allow us to fix data errors when they are detected, filter
important information out of the data source and integrate basic data 
processing steps with encryption or compression algorithms.
Finally, in order to improve the reliability of the infrastructure,
we will formalize PADS and prove important correctness properties
of our formal model.  Overall, this proposal involves challenging research in
language design, efficient systems implementation and theoretical
semantic analysis, all aimed at solving real-world data processing 
problems.

\paragraph*{Broader Impacts:}  We are collaborating with researchers at
AT\&T who will be able to use our tools to address real problems such as
telephone fraud detection at AT\&T.  In addition, our tools will be 
freely available and open source so that other industrial partners
may use them.  Moreover, ad hoc data is everywhere, not just the
telecommunications industry.  Part of our mission will be to work with 
biologists, chemists and physicists at Princeton and the broader community
to help them with their data 
processing needs.  We have already been meeting with Olga Troyanskaya,
who works in Princeton's Institute for Integrative Genomics 
on pathway modeling and analysis of protein-protein interactions.  Many of our proposed
extensions to PADS were inspired directly by her data processing needs.
The collaboration between CS and Genomics will also be an excellent platform
for developing interdisciplinary undergraduate research projects.
