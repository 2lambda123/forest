%%%%%%%%%%%%%%%%%%%%%%%%%%%%%%
% \centerline{
% \begin{tabular}{cc}
%           \\[-1ex]
% \multicolumn{2}{c}{\Large{} Tools for Processing Ad Hoc Data Sources} \\
%           & \\[-1ex]
% Kathleen Fisher & David Walker (PI)\\
% AT\&T  & Princeton University\\[2ex]
% \end{tabular}}
%%%%%%%%%%%%%%%%%%%%%%%%%%%%%%

\paragraph*{Intellectual Merits:} 
An {\em ad hoc data format} is any nonstandard data format for which
parsing, querying, analysis or transformation tools are not readily
available.  \xml{} is not an ad hoc data format --- there are hundreds
of tools for reading, querying, and transforming \xml{}.   However,
network administrators, programmers,
healthcare and airline information systems managers,
corporate IT professionals, 
Wall Street traders, microbiologists, chemists, and astrophysicists 
deal with ad hoc
data in a myriad of complex formats on a daily basis.
This data, which is often unpredictable, poorly documented, and
filled with errors,
poses tremendous challenges to its users and the software
that manipulates it.  Often, before anything can be done with
ad hoc data one must clense it of as many errors as possible,
normalize its shape and transform it into a standardized format
that can be loaded directly into a database.
The goal of our research is to help software developers 
create efficient, reliable and easy-to-maintain software for
for processing ad hoc data.  We will achieve our goal
through the design and implementation of \datatype,
a revolutionary new language with intrinsic support for 
ad hoc data.  Our comprehensive research agenda
consists of the following components.

\begin{enumerate}
\item Design, theory and implementation of a 
{\em Universal Data Description Language.}  
We will aim to develop a 
high-level, easy-to-understand description language for \datatype, capable of
concisely and accurately describing the physical format of
any ad hoc data source whether it be a stream of binary data such as 
packets from various networking protocols, a recursive hierarchy
of (ASCII) geneological data, a table of stock ticker information, a
web log from a web server, the output of a tcsh shell command, a (Cobol) 
entry for telephone call billing data or any other bit of electronic data
in any other format one can imagine.
 
\item Design, theory and implementation of an expressive and efficient 
{\em Ad Hoc Data Transformation Language.}
\datatype{} will support analysis, manipulation and transformation
of ad hoc data in domains ranging from genomics and microbiology to
networking and telecommunications, on scales ranging from kilobytes to 
gigabytes or more, and in the presence of both syntactic and semantic
data errors.  
\datatype{} will use the universal data descriptions as directives for
printing and parsing ad hoc data and as types for describing
representations of ad hoc data within the programming environment.  

\item Applications and Evaluation.  We will evaluate the expressiveness and 
performance of \datatype{}
by implementing applications in at least two completely 
different domains.  First, we will
develop tools for cleaning, normalizing, analyzing and mining
telecommunications data supplied by AT\&T.  This will allow
us to evaluate the performance of our tools on the truly
massive data sets (gigabytes or even terabytes in size).  
Second, we will collaborate with
Olga Troyanskaya from Princeton's Lewis-Sigler Institute for 
Integrative Genomics and with Steven Kleinstein, program coordinator
for Princeton's
Picasso project for interdisciplinary computational science
and provide them with tools for manipulating
genomics and microbiology data.
\end{enumerate}

% Processing ad hoc data is challenging for a variety of
% reasons. First, ad hoc data arrives ``as is:'' an analyst might want the
% data delivered in \xml, but unfortunately, he or she has just got to deal
% with it as it is.
% Second, documentation for the format may not exist at all, or it may be
% out of date.  
% Third, ad hoc data frequently contain errors, for a variety of
% reasons: malfunctioning equipment, race conditions on log
% entry~\cite{wpp}, malicious agents attempting to exploit
% software vulnerabilities, human error in entering data, and others. 
% The appropriate response to such errors depends on the application. 
% % Some applications require the data to be error free: 
% % if an error is detected, processing needs to stop immediately and a human
% % must be alerted.  Other applications can repair the data, while still
% % others can simply discard erroneous or unexpected values.  
% For some applications, errors or anomalous data can be the most 
% interesting part because they can signal potential network
% vulnerabilities, opportunities or ongoing fraud in telephony systems, 
% or failure to communicate between systems.  A fourth challenge is that 
% ad hoc data sources can be high volume.  For instance,
% AT\&T's call-detail stream contains roughly 300~million calls per day
% requiring approximately 7GBs of storage space.

\paragraph*{Broader Impacts:}  
Ad hoc data is ubiquitous. Consequently our research has the 
potential for far-ranging impact on corporate and
governmental data processing as well as
on the progress of research in the natural sciences including
biology, genomics, chemistry and physics.  For instance, in the governmental
arena, Newt Gingrich and Hillary Clinton have recently proposed integrating 
our national health records and providing a single electronic service
for accessing them.  Our technology will be of incalcuable value in helping
move old health records from legacy formats into modern systems.
In industry, our tools could be used to facilitate mergers or
data exchange between corporate parteners.
In academia, our tools will help improve the productivity of
computational biologists who need to collect data from a vast
array of online sources in their search to understand the functioning of our
genes and their relations to disease.

% We are collaborating with researchers at AT\&T who will be able to use
% our tools to address real problems such as telephone fraud detection.
% In addition, our tools will be freely available to researchers and
% scientists over the web.  Moreover, part of our mission will be to
% work with biologists, chemists and physicists at Princeton and the
% broader community to help them with their data processing needs.  We
% have already been meeting with Olga Troyanskaya, who works in
% Princeton's Lewis-Sigler Institute for Integrative Genomics on pathway
% modeling and analysis of protein-protein interactions.  Many of our
% proposed extensions to \pads{} were inspired directly by her data
% processing needs.  The collaboration between Computer Science and
% Genomics will also be an excellent platform for developing
% interdisciplinary undergraduate research projects.

