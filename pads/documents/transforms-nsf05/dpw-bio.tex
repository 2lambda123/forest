\documentclass[11pt]{article}
        % Use font 11pt Roman.
%\documentstyle{article}

%\def\references{\newpage\section*{References}}
%\newcounter{refcount}
%\def\lit#1{~[\ref{#1}]}
%\def\cit#1#2#3#4{#1\quad#2\quad{\it #3}\quad #4\vskip.15in }
%\def\citb#1#2#3{#1\quad{\it #2}.\quad#3.}
%\def\citu#1#2{#1\quad#2.}


\newcommand{\comment}[1]{}


% No header or foot; top 2.0in, left 0.6in, right 0.9in, bottom 0.5in

%\pagestyle{empty}
%\topmargin=-.5in
%\textheight=8.8in
%\leftmargin=-1.0in
%\oddsidemargin=.25in
%\oddsidemargin=.25in
%\textwidth=6.2in

\setlength{\textheight}{22.8true cm}
%\setlength{\textwidth}{15.4true cm}
\setlength{\textwidth}{15.9true cm}
\setlength{\oddsidemargin}{.55true cm}
\setlength{\evensidemargin}{.55true cm}
\setlength{\topmargin}{-1.0cm}
\pagestyle{empty}

\newlength{\oldparindent}
\setlength{\oldparindent}{\parindent}
\setlength{\parindent}{0pt}

% No indentation at the beginning of paragraphs.

% Don't hyphenate.
%\hyphenpenalty=10000


% Rsection: a resume section environment.
\newenvironment{Rsection}[1]{
  \noindent{\bf #1}
  \begin{list}{}{\topsep=-\parskip
                 \leftmargin=-0.0in}
    \item[]
}{
  \end{list}
%  \vspace \baselineskip
}
                           
% Section: a resume section environment.
\newenvironment{Section}[1]{
  \centerline{\bf #1}
  \begin{list}{}{\topsep=-\parskip
                 \leftmargin=-0.5in}
    \item[]
}{
  \end{list}
%  \vspace \baselineskip
}

% edlist: a resume education list environment.
\newenvironment{edlist}{
  \begin{list}{}{\topsep=-\parskip
                 \itemsep=0.5\baselineskip
                 \leftmargin=0.25in}
}{
  \end{list}
}


% edlist2: a resume education list environment.
\newenvironment{edlist2}{
  \begin{list}{}{\topsep=-\parskip
                 \itemsep=0.5\baselineskip
                 \leftmargin=0.25in
                 \setlength{\rightmargin}{\leftmargin}
                 \textwidth=4in}
}{
  \end{list}
}
                           

% honorlist: a resume honor list environment.
\newenvironment{honor2list}{
  \begin{list}{}{\topsep=-\parskip
                 \itemindent=-0.in
                 \itemsep=0.0\baselineskip
                 \leftmargin=.25in}
}{
  \end{list}
}  

\newenvironment{honor2enum}{
  \begin{enumerate}{}{\topsep=-\parskip
                 \itemindent=-0.in
                 \itemsep=0.0\baselineskip
                 \leftmargin=.25in}
}{
  \end{enumerate}
}  

\newenvironment{honor3enum}{
  \begin{enumerate}{}{\topsep=-\parskip
                 \itemindent=-0in
                 \itemsep=0.0\baselineskip
                 \leftmargin=.25in}
}{
  \end{enumerate}
}  



% honorlist: a resume honor list environment.
\newenvironment{honorlist}{
  \begin{list}{}{\topsep=-\parskip
                 \itemindent=-0.in
                 \itemsep=0.2\baselineskip
                 \leftmargin=.25in}
}{
  \end{list}
}  


% joblist: a resume job list environment.
\newenvironment{joblist}{
  \begin{list}{}{\topsep=-\parskip
                 \itemsep=0.5\baselineskip
                 \leftmargin=0.5in}
}{
  \end{list}
}

                                      
% job: a job entry for a joblist environment
\newcommand{\job}[3]{
  \item[]
  \begin{tabbing}
    \hspace{-0.5in}\=\hspace{1.5in}\=\kill
    \>    {\it #1:}       \> #2   \\
    \>                    \> #3   \\
  \end{tabbing}
  \vspace{-\baselineskip}
}

              
%%%%%%%%%%%%%%%%%%%%%%%%%%%%%%%%%%%%%%%%%%%%%%%%%%%%%%%%%%%%%%%%%%%%%%%%%%%%%%%




%%%%%%%%%%%%%%%%%%%%%%%%%%%%%%%%%%%%%%%%%%%%%%%%%%%%%%%%%%%%%%%%%

%\leftmargin=-1in
\begin{document}


\centerline{\Large \bf {\LARGE \bf D}AVID {\LARGE \bf W}ALKER}
\vskip .3in

\normalsize
\it
\begin{tabular}{@{\hspace{0in}}l@{\hspace{1.4in}}l}
35 Olden Street   & Phone: (609) 258--7654  \\
Princeton University   & Email: {\rm dpw@cs.princeton.edu} \\
Princeton, NJ 08544

\end{tabular}
\rm
\vskip.3in

\begin{Rsection}{\Large \bf {Professional Preparation}}
\vskip.1in
\begin{edlist}

\item {\bf Queens University, Kingston, Ontario, Canada }
  \begin{honor2list}
  \item {Computer Science, B.Sc. Honors, 1995.}  

  \end{honor2list}
\item {\bf Cornell University}
\begin{honor2list}
\item {Computer Science, M.S. 1999.}
\item {Computer Science, Ph.D. 2001.}
\end{honor2list}
\end{edlist}
\end{Rsection}


\vskip.3in
\begin{Rsection}{\Large \bf {Appointments}}
\vskip.1in
\begin{edlist}
\item {\bf Princeton University}\\
{\bf  Department of Computer Science} \\
%\vskip .05in
\vspace{-1ex}
\begin{honor2list}
\item Assistant Professor, Feb. 2002--present.
\end{honor2list}

\item {\bf Carnegie Mellon University}\\
{\bf  Department of Computer Science} \\ 
%\vskip .05in
\vspace{-1ex}
\begin{honor2list}
\item Postdoctoral Fellow, Oct. 2000--Oct. 2001
\end{honor2list}


%\newpage
\end{edlist}
\end{Rsection}


\vskip.3in
\begin{Rsection}{{\Large \bf {Selected Publications}} 
}
\vskip.1in
\begin{edlist}

\item {\bf Selected papers of greatest relevance (in chronological order)}
\begin{honor2enum}
\vskip .1in

\item Composing Security Policies in Polymer.  
Lujo Bauer, Jarred Ligatti and David Walker.
To appear in the ACM SIGPLAN Conference on Programming Language Design and 
Implementation.  June 2005.

\item Edit Automata: Enforcement Mechanisms for Run-time Security Policies.  Jay Ligatti, Lujo Bauer and David Walker.  
International Journal of Information Security. Electronic version
published October 2004.  Hardcopy anticipated February 2005.

\item Dynamic Typing with Dependent Types (extended abstract).
Xinming Ou, Gang Tan, Yitzhak Mandelbaum, and David Walker.  In the
3rd IFIP International Conference on Theoretical Computer Science,
August, 2004.

\item An Effective Theory of Type Refinements.  Yitzhak Mandelbaum, 
David Walker and Robert Harper.  In the ACM SIGPLAN International 
Conference on Functional Programming, August 2003. 


\item A Type System for Expressive Security Policies.  David Walker. 
Twenty-Seventh ACM SIGPLAN Symposium on Principles of Programming Languages. 
pages 254-267, Boston, January 2000

%


\end{honor2enum}

\newpage
\item {\bf Selected other publications (in chronological order):}
\begin{honor2enum}

\item Substructural Type Systems.  David Walker.
Chapter 1 of Benjamin Pierce, ed.,
Advanced Topics in Type Systems for Programming Languages.  
MIT Press, January 2005.

\item Reasoning about Hierarchical Storage.  Amal Ahmed, Limin Jia and David Walker.  IEEE Symposium on Logic in Computer Science, pp. 33-44. Ottawa, Canada, June 2003.

\item 
Greg Morrisett, Karl Crary, Neal Glew, and David Walker. Stack-based
Typed Assembly Language.  Journal on Functional Programming,
12(1):43-88, January 2002.


\item 
Typed Memory Management via Static Capabilities. 
David Walker, Karl Crary, and Greg Morrisett. ACM Transactions on Programming Languages and
Systems, 22(4):701-771, July 2000.   A previous version of this paper
appeared in the 26th ACM Symposium on Principles of Programming Languages.

\item 
From System F
to Typed Assembly Language. 
Greg Morrisett, David Walker, Karl Crary, and Neal Glew. 
ACM Transactions on Programming Languages
and Systems, 21(3):528-569, May 1999.  A previous version of this paper
appeared in the 25th ACM Symposium on Principles of Programming Languages.


\end{honor2enum}

\end{edlist}
\end{Rsection}



\vskip.3in
\begin{Rsection}{\Large \bf {Synergistic Activities}}
\begin{honor2list}
\vskip .1in
\begin{itemize}
\item Alfred P. Sloan Fellow.  Sept 2004-Sept 2006. 
\item Summer School on Reliable Computing.  Program
Co-Chair and Organizer.  July 2005.  
A 10-day summer school on type systems, program analysis,
software model checking, and domain specific languages
aimed at teaching the latest research on improving software reliability. 
\item Summer School on Software Security: Theory to Practice.  Program
Co-Chair and Organizer.  
June 2004.  Supported by ACM SIGPLAN, NSF, and Microsoft.
A 10-day summer school that taught the fundamentals of
software security.  Over 70 participants (Masters students,
Ph.D. students, academic and industrial researchers). 
\item Publicity chair for ACM SIGPLAN-SIGACT Symposium on
Principles of Programming Languages, 2003--2006.
Program Chair for FOAL 2005, SPACE 2006.  Program committee
for ICFP 2004, TLDI 2005.  Periodic program chair and host for
the New Jersey Programming Languages Seminar, 2002-now.
\end{itemize}



\end{honor2list}
\end{Rsection}
\vskip.3in
\begin{Rsection}{\Large \bf {Collaborators and Other Affiliations}}
\begin{itemize}
\vskip .1in
\item Collaborators: 
Amal Ahmed (Harvard),
Andrew Appel (Princeton),
Zena Ariola (U of Oregon),
Lujo Bauer (Carnegie Mellon),
Iliano Cervesato (Tulane),
%Karl Crary (Carnegie Mellon University),
%Daniel Dantas (Princeton University),
%Ed Felten (Princeton University),
Kathleen Fisher (AT\&T),
Neal Glew (Intel),
Daniel Grossman (U of Washington),
Robert Harper (Carnegie Mellon),
%Limin Jia (Princeton University),
%Jarred Ligatti (Princeton University),
%Yitzhak Mandelbaum (Princeton University),
Greg Morrisett (Harvard),
Benjamin Pierce (U of Pennsylvania),
Frank Pfenning (Carnegie Mellon),
%Fred Smith (MathWorks),
%Frances Spalding (Princeton),
%Daniel Wang (Princeton),
Geoffrey Washburn (U of Pennsylvania),
Kevin Watkins (Carnegie Mellon),
Stephanie Weirich (U of Pennsylvania),
Steve Zdancewic (U of Pennsylvania)

\item Graduate (Ph.D.) Advisor: 
Greg Morrisett (Cornell University, now Harvard University)
\item Thesis Advisees (graduated): none
\item Thesis Advisees (current): Dan Dantas, Limin Jia, Jarred Ligatti, 
Yitzhak Mandelbaum, Frances Spalding
\end{itemize}
\end{Rsection}



% \iffalse
% \vskip.3in
% \begin{Rsection}{\Large \bf {Selected Awards and Fellowships}}
% \begin{itemize}
% \vskip .1in
% \end{itemize}
% \end{Rsection}
% \fi




%\newpage

%\newpage





%%%%%$$^{\ref{fn:second}}$







%\vskip.3in




\end{document}

