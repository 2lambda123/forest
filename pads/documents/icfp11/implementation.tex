\section{Implementation}
\label{sec:implementation}
\cut{
\reviewer{I don't understand how laziness is used to avoid loading
  unneeded parts of the filestore.  Perhaps this uses some standard
  Haskell feature that I'm unaware of, but I'd like to see more
  discussion, probably in the "Implementation" section.  What's the
  complexity of the operation to save changes to a filestore?  Is it
  proportional to the change size or the size of the whole filestore?}

\reviewer{What is the performance overhead of Forest? I suspect it
  would be quite minimal since the matching does not do any intense
  computation, but it would be great if the authors can state it
  explicitly.}

\reviewer{It's unclear exactly how Forest gets translated into
  Haskell, and if this process depends crucially on the kind of
  approach to generics used (TemplateHaskell vs SYB vs..).}
}

The current implementation of \forest{} is available from the project web
site:~\url{forestproj.org}. 

\haskell{} provides powerful language features and libraries that
greatly facilitated implementation of \forest{}.  The most obvious
of these features is the quasiquotation mechanism~\cite{Mainland:quasi}
that we used to embed \forest{} into \haskell{}.  This mechanism
provided the benefits of being an embedded domain-specific
language without having to sacrifice the flexibility of
defining our own syntax. To use quasiquoting, we defined a \haskell{}
value \cd{forest} of type \cd{QuasiQuoter} which specifies how to
convert an input string representing a sequence of \forest{} declarations into the
\template{}~\cite{Sheard+:templatehaskell} 
data structures that represent the syntax of the
corresponding collection of \haskell{} declarations.
The Haskell compiler calls the \cd{forest} ``compilation'' function 
during the compilation of any Haskell source file containing a \forest{} quasiquotation.
The quasiquoted syntax \cd{[forest| <input> |]}
is legal anywhere the identifier \cd{forest} is in scope.
When the \haskell{} compiler processes this declaration, it first
passes \cd{<input>} as a string to the \cd{forest} quasiquoter, and
then it compiles the resulting \template{} data structures as if the
corresponding \haskell{} code had appeared in the input at the
location of the quasiquote.  Early versions of quasiquoting
supported quoting only expression and pattern forms.  Simon Peyton
Jones extended the mechanism to permit declaration and type
quasiquoting partly to enable the \forest{} implementation.  We used
this same approach to implement \padshaskell{}, which we built
concomitantly.  

Note that in implementing \forest{}, we had to use \template{}
rather than any of the other libraries that support generic programming,
both because that is what the quasiquote library expects and
because we need to generate type and datatype declarations (and to do
so at compile time).  
Other available generic libraries do not support the latter functionality.
%declarations can only be generated at compile time.

\paragraph{Parsing.}  
We used the parsec 3.1.0 parser
combinator library~\cite{LeijenMeijer:parsec} to implement the \forest{} parser.  One key element
of the \forest{} design is to allow arbitrary \haskell{} expressions
in various places inside \forest{} descriptions.  We did not want to
reimplement the grammar for \haskell{} expressions, which is quite
complicated.  Instead, we structured the \forest{} grammar so we could
always determine the extent of any embedded \haskell{} code.  We then
used the Haskell Source Extension package~\cite{haskell-src-exts} to parse
these fragments.  The data structure that this library returns is
unfortunately not the data structure that \template{} requires, so we
used yet another library, the Haskell Source Meta
package~\cite{haskell-src-meta}, that does this translation. 


\paragraph{Type checking.}
We would like to give users high-quality error messages if there
are type errors in their \forest{} declarations.  At the moment, 
typechecking occurs, but only after the \forest{} declarations have
been expanded to the corresponding \haskell{} code.  Although these
error messages can be quite informative, it is sub-optimal to report
errors in terms of generated code.  Type checking the \forest{} source
is complicated by the embedded fragments of \haskell{}.  As with the
syntax, we do not want to reimplement the \haskell{} typechecker!  
There is an active proposal~\cite{extend-templates} to extend the \template{} infrastructure
with functions that would enable us to ask the native \haskell{}
typechecker for the types of embedded expressions and to extend the
current type environment with type bindings for new identifiers.  With
this combination of features, we would be able to type check \forest{}
sources directly.
%, yielding a fully satisfactory embedded
%domain-specific language.

\paragraph{Runtime.}
Although \forest{} facilitates treating the file system as a
persistent store, it does not provide the ACID guarantees familiar from
databases.  None of the \filestore{}s we
have encountered in practice are implemented in a system that provides
such support; users instead have extra-linguistic mechanisms to make
sure they do not corrupt their data with ill-timed concurrent reads
and writes. That said, the \forest{} language does not preclude an
implementation from providing such guarantees. We consider this issue
very interesting future work. 

\forest{} uses Haskell's \cd{unsafeInterleaveIO} to load each portion
of a \filestore{} only when needed by an application program.
We have not systematically measured the performance overhead of using
\forest{}.  However, we have used our
mostly-unoptimized implementation to manipulate filestores on the
order of many gigabytes and found the performance acceptable
for many applications.  

The running time of storing operations is proportional to the
``footprint'' of the described \filestore{}. However, the \forest{}
compiler generates \cd{load} and \cd{manifest} functions for each
named type in a description. Thus, updates may be made at any
granularity for which there is a named type, which is typically at the
level of individual files. We plan to investigate better support for
incremental updates in future work.



