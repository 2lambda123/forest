\section{Core Calculus}
\label{sec:theory}

\begin{figure}
\[
\begin{array}{@{}r@{\;}r@{\;}r@{\;}l@{}}
\multicolumn{4}{@{}l}{\fbox{\textit{Basic definitions}}}\\[1.5ex]
\textit{Integers}     & n    & \in & \mathbb{Z}\\[1ex]
\textit{Strings}      & u    & \in & \Sigma^{\ast}\\[1ex]
\textit{Values}       & v    & \bnfdef & u \bnfalt n \bnfalt \True \bnfalt \False \bnfalt r \bnfalt () \bnfalt (v_1,v_2)\\
                      &      & \bnfalt &\Just{v} \bnfalt \Nothing \bnfalt \List{v_1,\tdots,v_n}\\[1ex]
\textit{Types}        & \tau & \bnfdef & \typ{String} \bnfalt \typ{Int} \bnfalt \typ{Bool} \bnfalt \typ{Path} \bnfalt \typ{()} \bnfalt (\tau_1,\tau_2) \\
                      &      & \bnfalt & \typ{Maybe}~\tau \bnfalt \List{\tau}\\[1ex]
\textit{Environments} & \Env & \bnfdef & \ENil \bnfalt \Env,x\mapsto v\\[1ex]
\textit{Expressions}  & e    & \bnfdef & x \bnfalt \lambda x.~e \bnfalt e_1~e_2 \bnfalt \dots \\[2.5ex]
%
\multicolumn{4}{@{}l}{\fbox{\textit{Forest definitions}}}\\[1.5ex]
\textit{Paths}          & r    & \bnfdef & \PNil \mid \PCons{r}{u}\\[1ex]
\textit{Attributes}     & \att & \bnfdef & v\\[1ex]
\textit{Contents}       & T    & \bnfdef & \File{u} \bnfalt \Link{r} \bnfalt \Dir{ \Set{u_1,\tdots,u_n} }\\[1ex]
\textit{File systems}   & F    & \bnfdef & \Map{ \! r_1 \mapsto (\att_1,T_1), \tdots, r_n \mapsto (\att_k,T_n) \! }\\[1ex]
\textit{Specifications} & s    & \bnfdef & \Sk \bnfalt \SPath{e}{s} \bnfalt \SPair{x}{s_1}{s_2} \bnfalt \SComp{s}{x}{e} \bnfalt \SPred{e} \bnfalt \SOption{s}
\end{array}
\]
\caption{\forest{} calculus syntax}
\label{fig:csyntax}
\end{figure}

This section presents a core calculus for \forest{}, which formalizes
the essential features of the language in a simple setting. The
calculus is based on classical (\ie{}, not separating, substructural,
or ambient) unordered tree logics, but has a number of features
tailored to file systems. We used it to study various features of
\forest{} as we were developing it, and to prove key theorems like the
round-tripping properties described at the end of the section.

\subsection{Data Model}

\forest{} calculus specifications manipulate instances of a simple
file system model, which is given in Figure~\ref{fig:csyntax}. A path
$r$ is a sequence of strings $u$,\footnote{For simplicity, we do not
  model special path elements such as ``..'' and ``.''.  It would be
  easy to add these at the cost of complicating the semantics.} and a
file system $F$ is a finite map from paths to pairs $(\att,T)$ of
attributes $\att$ and file system contents $T$.  We do not specify the
syntax of attributes precisely, but expect that they will contain the
usual fields including owner, group, permissions, modification date
and time, {\it etc.} A special attribute $\defaultatt$ contains
default values for all fields. The contents $T$ of a node in the file
system is either a file $\File{u}$, where $u$ is the string contents
of the file; a symbolic link $\Link{r}$, where $r$ is the path pointed
to by the link; or a directory $\Dir{\Set{u_1,\tdots,u_n}}$, where
$u_1$ to $u_n$ are the names of the elements of the file system
located directly beneath the current path. We write $\dom{F}$ for the
set of paths $F$ is defined on, $F(r)$ for the contents at $r$, $F(r)
= \bot$ when $r$ is not in $\dom{F}$, and $\FUpdate{F}{r}{(\att,T)}$
or $\FDelete{F}{r}$ for the file systems obtained from $F$ by
overwriting the mapping for $r$ to $(\att,T)$ or deleting the mapping
for $r$ respectively.

A file system $F$ is {\em well-formed} if it encodes a tree with
directories at the internal nodes and files and symbolic links at the
leaves. More formally, $F$ is well-formed if the following conditions
hold:
%
\begin{itemize}
\item $\dom{F}$ is prefix-closed,
\item $F(r) = (\att,\Dir{ \Set{u_1,\tdots,u_n} })$ implies
  $\PCons{r}{u_i} \in \dom{F}$ for all $i$ from $1$ to $n$, and
\item $F(r) = (\att,\File{u})$ or $F(r) = (\att,\Link{r'})$ implies
  $\PCons{r}{u'} \not\in \dom{F}$ for all $u'$.
\end{itemize}
%
Note that although these conditions imply that the structure of a
well-formed file system $F$ must be tree-shaped, cycles can be
expressed using symbolic links that point ``upwards'' in the file
system.

\subsection{Specifications}

The syntax of specifications $s$ is given in
Figure~\ref{fig:csyntax}. The calculus models the most important
features of the full \forest{} language in the simplest possible way,
using orthogonal, independent constructs. The set of specifications is
parameterized over a collection of constants $\Sk$, 
where $\tau_1$ is the type of the generated representation and $\tau_2$ is
the type of the generated metadata. We omit the type annotations in
examples.  For the purpose of illustration, we
will assume that constants for describing files $\pfile$, directories
$\pdir$, and symbolic links $\plink$ all exist, as well as constants
$\SAdhoc{p}$ parameterized on \padshaskell{} descriptions $p$ for
describing the contents of files in more detail. Path specifications
$\SPath{e}{s}$ are parameterized on an expression $e$, which must
evaluate to a string (to be appended to the current path), and a
specification $s$ that describes the fragment of the file system at
the extended path. We leave the syntax of expressions $e$ abstract but
assume that it contains the features of a simple functional language
(of course, in the full \forest{} language, expressions can be
arbitrary fragments of Haskell code). We will assume a semantic
function $\eval{e}{\Env}{\tau}$ that evaluates $e$ in environment $\Env$, yielding
a value $v$ of type $\tau$. Dependent pairs are written
$\SPair{x}{s_1}{s_2}$ where $s_1$ and $s_2$ are specifications that
describe possibly overlapping fragments of the file system. When a
dependent pair is evaluated, the variables $x$ and $x_{\MD}$ are bound
to the value and metadata computed for $s_1$ and may be used in
$s_2$. Comprehensions are written $\SComp{s}{x}{e}$, where $e$ is an
expression that evaluates to a set of values, and $s$ is a
specification that describes a fragment of the file system for each
value of $x$.  Predicates $\SPred{e}$ succeed when the expression $e$
evaluates to \True{} and fail when it evaluates to \False{} under the
current environment.  Finally, the type $\SOption{s}$ describes
either (1) a filesystem not containing the current path, or (2)
a file system containing the current path and satisfying $s$. 

To develop some intuitions about these constructs, let us encode a few
of the more complicated features of the full \forest{} language in the
calculus. In the full language, records and paths are specified using
a single construct:
%
\begin{centercode}
\kw{Directory}
  \{ c \kw{is} "c.txt" :: C, 
    d \kw{is} "d.txt" :: D c, 
    e \kw{is} "e.csv" :: E d \}
\end{centercode}
%
The calculus, however, only has dependent pairs, not full-blown
records, and has a separate construct for describing paths. Thus, to
encode the specification above, we use the following specification:
%
\[
\SPair{c\,}{\,(\SPath{\texttt{"}\mathrm{c.txt}\texttt{"}}{\mathrm{C}})\;}
           {\SPair{d\,}{(\SPath{\texttt{"}\mathrm{d.txt}\texttt{"}}{\mathrm{D}\; c})\;}
                       {(\SPath{\texttt{"}\mathrm{e.csv}\texttt{"}}{\mathrm{E}\; d})}} 
\]
%
Similarly, a comprehension written as
%
\begin{center}
\cd{[c :: C | c <- \kw{matches} (GL "*")]} 
\end{center}
%
in the full language is encoded in the calculus as
%
\[
\SPair{d}{\pdir~}{\SComp{\;\SPath{c}{\mathrm{C}}\,}{c}{d\;}}
\]
%
The value returned by the $\pdir$ constant is the set of names of the
immediate children of the directory. Finally, in the full language,
constraints are written as
%
\begin{center}
\cd{s \kw{where} e}
\end{center}
%
but are encoded in the calculus using a dependent pair and a
predicate:
%
\[
\SPair{\mathit{this}}{s}{\SPred{e}}
\]
%
When evaluated, the predicate encodes the success or failure of the
constraint expressed by $e$ in the metadata for the second component
of the pair.

\subsection{Semantics}
%
The semantics of the \forest{} calculus is organized into four
separate definitions, one for each of the major artifacts generated by
the compiler. These artifacts include a type for the representations,
a type for metadata, a function $\loadkname{}$ for loading data from
the file system, and a function $\storekname{}$ for storing it
back. They are carefully designed to fit together in a particular
way---\eg, the $\loadkname{}$ and $\storekname{}$ functions manipulate
representations and metadata of the appropriate type and are
guaranteed (under certain precisely identified circumstances) to
preserve data on round trips.

\paragraph*{Types.}
%
The definitions of the types for the representations ($\repty{s}$) and
metadata ($\mdty{s}$) generated for each specification $s$ appear in
Figure~\ref{fig:csemantics}. 
The representation type for constants $\Sk$ is simply
read off from $\tau_1$. For other specifications, representation types
are built out of the types for sub-specifications in the obvious
way---\eg, for pairs $\SPair{x}{s_1}{s_2}$, the type is a product
$(\repty{s_1}, \repty{s_2})$. 
The types for metadata are more
interesting.  In the \forest{} calculus we represent metadata using
structured datatypes containing a boolean value at each level of
structure. Intuitively, this value indicates whether there were any
errors during loading. The $\mathit{Md}$ constructor provides a
uniform representation for these structures,
\[
\begin{array}{lcl}
\typ{MdHeader} & = & \typ{Bool} \\
\md~{\tau} & = & (\typ{MdHeader}, \tau),
\end{array}
\]
and the function $\valid{d}$ extracts the boolean value, returning
$\True$ if there are no errors in the structure, and $\False$
otherwise. Each of the the $\loadkname{}$ functions generated from
specifications maintain the invariant that the the boolean value at
each level of structure is $\True$ if and only if all of the nested
values are also $\True$. Thus, one can test for errors during loading
simply by examining the top-level boolean value.  Note that it would
be simple to augment this structure with additional information, such
as the number of errors, an optional error message, or file system
attributes, as in Figure~\ref{fig:forest-md}; one would simply have to
change the $\mathit{MdHeader}$ type and the $\valid{d}$ function.

\paragraph*{Load Functions.}
%
The most common use of \forest{} is as a tool for loading data from
the file system into an in-memory representation. The functions
defined by the inference rules in Figure~\ref{fig:csemantics}~(a)
implement this task.  Formally, the judgment
$\load{\Env}{F}{r}{s}{v,d}$ holds if loading the specification $s$
from the file system $F$ at path $r$ in environment $\Env$ yields a
pair $(v,d)$ of representation and metadata. This judgment may be seen
as describing a total function from $\Env$, $r$, $s$ and $F$ to
$(v,d)$. The fact that the load function is total is useful---it
allows the programmer to explore fragments of the file system that do
not match $s$ exactly. When $F$ does not match $s$, the function puts
default values in the representation $v$ and records an error in the
metadata $d$.

Let us examine the inference rules that define the load functions in
detail. We assume that each constant $\Sk$ has an associated
$\loadkname{k}$ function. For example, the $\loadkname{}$ function for
the \pfile{} construct, which describes any file (but not symbolic
links or directories), takes an environment $\Env$, a file system $F$,
and a path $r$ as arguments, and either returns the contents and
attributes of the file at path $r$, or ``$\,$'' and default attributes
if $F(r)$ is undefined or not a file.
%
\[
\loadkname{\pfile}~(\Env,F,r) =
\begin{cases}
  (u,(\True,\att)), \mathrm{if } F(r) = (\att,\File{u}) \\
  (\text{``}{\,}\text{''},(\False,\defaultatt)), \mathrm{otherwise}\\
\end{cases}
\]
%
The $\loadkname{}$ function for symbolic links is similar:
%
\[
\loadkname{\plink}~(\Env,F,r) =
\begin{cases}
  (r',(\True,\att)), \mathrm{if } F(r) = (\att,\Link{r'}) \\
  (\PNil,(\False,\defaultatt)), \mathrm{otherwise}\\
\end{cases}
\]
%
Note that this function only returns the path $r'$ contained in the
link and not the directory or file pointed to by it. To access the
contents of the file system at $r'$, the programmer could bind the
path value returned by $\plink$ to a variable, and use the path
construct to navigate to that path, as in the following specification:
%
\[
\SPair{x\,}{\,\plink\;}{(\SPath{x}{\pfile})\;}
\]
%
Alternatively, one could design a different constant for symbolic
links that implements a ``deep'' lookup in the file system.

The load function for path specifications $\SPath{e}{s}$ evaluates the
expression $r/e$ to a path $r'$, and invokes the load function for $s$
from $r'$. The load function for dependent pairs $\SPair{x}{s_1}{s_2}$
first invokes the load function for $s_1$, yielding a representation
$v_1$ and metadata $d_1$, and then invokes the load function for $s_2$
in an extended environment where $x$ is bound to $v_1$ and $x_{\MD}$
to $d_1$, yielding $v_2$ and $d_2$. It returns $(v_1,v_2)$ and
%
$(\valid{d_1} \wedge \valid{d_2},(d_1,d_2))$
%
as the result. The load function for comprehensions $\SComp{s}{x}{e}$
is similar. It evaluates the expression $e$ to a list
$[w_1,\tdots,w_n]$, invokes the load function for $s$ $n$ times, in a
series of environments with $x$ bound to each $w_i$, and collects up
the results into lists of representations $[v_1,\tdots,v_n]$ and
metadata $[d_1,\tdots,d_n]$.

The load function for predicates $\SPred{e}$ tests whether the
expression $e$ is holds in the environment $\Env$. It returns $()$ as
the representation and $(b,())$ as the metadata, where $b$ is
$\eval{e}{\Env}{\mathit{bool}}$. The final load function, for options,
$\SOption{s}$ is defined using two inference rules. The first handles
the case where the current path $r$ exists in the file system. It uses
$s$'s load function to obtain a representation and metadata and
injects both into the $\typ{Maybe}$ using the $\mathsf{Just}$
constructor. The second handles the case where $r$ does not exist in
the file system. It returns $\Nothing$ for both the representation and
metadata.

\paragraph*{Store Functions.}
%
Just as they can be used for loading, \forest{} specifications can
also be used to store data back to the file system. The inference
rules in Figure~\ref{fig:csemantics}~(b) define the judgment
$\store{\Env}{F}{r}{v,d}{s}{F',\phi'}$, which holds if storing $(v,d)$
into $F$ at $r$ using specification $s$ with respect to $\Env$ yields
a modified file system $F'$ and a {\em validator} $\phi'$.  This
last element, the validator, is a
predicate on file systems that tests for inconsistencies during
storing --- it will be described in greater detail momentarily. As with load functions, the store judgment may be read as a total
function; it maps $\Env$, $r$, $s$, $F$, and $(v,d)$ to $(F',\phi')$.

Intuitively, it should be obvious that there are numerous ways that
storing data back to the file system could go wrong. For example, the
representation $v$ and metadata $d$ might not be consistent with the
existing information on the file system, or with each other. Even
worse, storing $v$ and $d$ back to $F$ could involve overwriting the
same file (or link, or directory) with multiple pieces of data---a
conflict. Thus, to reliably use store functions to manipulate the file
system, we need a way way to keep track of the intricate constraints
on the data in the representation, metadata, and file system implied
by the specification.

The validator $\phi'$ produced by the store function does exactly
this. It keeps track of the conditions needed to ensure that storing
the representation and metadata back to $F$ will accurately reflect
all of the information contained in them. As a simple example to show
why validators are needed, consider the specification
%
\[ s = \SPair{x}{\pfile}{\pfile}
\]
% 
and suppose that we invoke $s$'s load function with $\Env$, $F$, and
$r$ where $F(r)$ is $(\att,\File{u})$. According to the semantics of
the load functions described above, the representation $v$ will be a
pair $(u,u)$ containing two copies of the file contents at $r$ and the
metadata $d$ will also be a pair $((\True,\att),(\True,\att))$
containing two copies of the attributes associated to that file. Now
suppose that we change the representation to $(u,u')$, where $u \neq
u'$, and store the result back to the file system. Because this new
representation does not satisfy the dependency between the two
components of the pair implied by $s$, the store function cannot
produce a new file system containing the information in both $u$ and
$u'$. It must store one of the strings and discard the other. The
validator $\phi'$ detects this inconsistency. In this case, the
validator will be equivalent to the following predicate on file
systems:
%
\[
\lambda F'.~ (F'(r) = (\att,\File{u})) \wedge (F'(r) = (\att,\File{u'}),
\]
%
which is clearly not satisfiable unless $u = u'$.

The store functions also use validators to detect internal
inconsistencies between the representation and metadata. For example,
consider the specification $s$ defined as $\SOption{s_1}$, and
suppose that invoking $s$'s load function with $\Env$, $F$, $r$ where
$F(r)$ is undefined. The representation and metadata will be
$\Nothing$ and $(\True,\Nothing)$ respectively. Now suppose that we
change the representation to $\Just{u'}$, and invoke $s$'s store
function. Even though the pair $(\Just{u'}, (\True,\Nothing))$ would
never be produced by $s$'s load function, the store function must
still do something reasonable with it. One reasonable choice---the one
used in our semantics---is to update the file system contents to
$(\File{u'},\defaultatt)$, but produce a validator $\phi' = \lambda
F'.~ \False$ that records the internal inconsistency between the
representation and metadata. Among other things, this informs the
programmer that if they invoke the load function on the new file
system, the representation and metadata may not be preserved (\eg,
loading $u'$ using $s_1$ could result in errors).

Now let us examine the inference rules in detail. We assume that each
constant defines a $\storekname{k}$ function. The $\storekname{}$
function for \pfile{} is defined as follows:
%
\[
\begin{array}{l}
\storekname{\pfile}~(\Env,F,r,v,d) =\\[1ex]
\quad 
\left\{
\begin{array}{ll}
  (\FUpdate{F}{r}{(\att,\File{v})},            
  \; \lambda F'.~ (F'(r) = (\att,\File{v})))\\
  \hfill\text{if}~d = (\True,\att)\\[1ex]
  (\FDelete{F}{r},                             
  \; \lambda F'.~\False) \\
  \hfill\text{if}~d = (\False,\att)~\wedge F(r) = (\us,\File{\us})\\[1ex]
  (F,              
  \; \lambda F'.~ (v=\text{``}{\,}\text{''} \wedge a=\defaultatt \wedge F'(r) \neq (\us,\File{\us}))\\
  \hfill\text{if}~d = (\us,\att)
\end{array}\right.
\end{array}\]
%
It overwrites the contents of the file system $F$ at path $r$ with
$(\att,\File{v})$ if $d$ is valid, deletes the contents of $F$ at $r$
if $d$ is not valid but $F(r)$ contains a file, and returns $F$
unchanged otherwise. The validator $\phi'$ also has three cases: in
the first, it tests whether $F'(r)$ is $(\att,\File{v})$; in the
second, it is always $\False$; and in the third it tests whether
$F'(r)$ is not a file and $(v,\att)$ is the pair of default values
$(\text{``}{\,}\text{''}, \defaultatt)$ generated by the load
function. These constraints are necessary to prove the round-tripping
properties described at the end of this section.

The rule for path specifications $\SPath{e}{s}$ simply passes off
control to the store function for $s$ after replacing the current path
$r$ with $\eval{\PCons{r}{e}}{\Env}{\mathit{path}}$. The rule for
dependent pairs $\SPair{x}{s_1}{s_2}$ is more interesting. Given a
pair $(v_1,v_2)$ as the representation and $(b,(d_1,d_2))$ as the
metadata, it invokes the store function for $s_1$ on $(v_1,d_1)$,
yielding an updated file system $F_1'$ and validator $\phi_1'$, and
then invokes the store function for $s_2$ on $(v_2,d_2)$ in an
extended environment where $x$ is bound to $v_1$ and $x_{\MD}$ is
bound to $d_1$, yielding another updated file system $F_2'$ and
validator $\phi_2'$. It combines the updated file systems using the
right-biased file system concatenation operator defined as follows:
%
\[
\begin{array}{l}
(F_1 \FAppend F_2)(r) = \\
\qquad 
\begin{cases}
  (\att_2,\Dir{N_1 \cup N_2}) & \text{if}~F_1(r) = (\att_1,\Dir{N_1}) \; \wedge \\
                              & \;\;\; F_2(r) = (\att_2,\Dir{N_2})\\
  F_1(r) & \text{if}~F_2(r) = \bot\\
  F_2(r) & \text{otherwise}\\
\end{cases}
\end{array}
\]
%
Finally, it conjoins the two validators $\phi_1'$ and $\phi'_2$. The
result is a file system that contains the consistent changes made to
the file system by the store functions for $s_1$ and $s_2$ as well as
a validator that checks for the consistency of all the changes. The
store function for comprehensions is similar.

The store function for predicates $\SPred{e}$ returns the input file
system $F$ unchanged and a validator $\phi'$ that checks whether $e$
evaluates to $b$, the boolean value stored in the metadata. The store
function for options $\SOption{s_1}$ is defined by three inference
rules. The first handles the case where the representation is
$\Just{v_1}$ and the metadata is $\Just{d_1}$ by simply unpacking the
encapsulated values and invoking $s_1$'s store function. The second
handles the case where the representation is $\Nothing$. It deletes
the file and returns a validator that checks whether the metadata $d$
is also $\Nothing$ and the file system is undefined on the path
$r$. The third rule handles the case where the representation is
$\Just{v_1}$ and the metadata is $\Nothing$. It invokes $s_1$'s store
function on $v_1$ and default metadata and returns a validator that
always evaluates to $\False$, reflecting the inconsistency in the
representation and metadata.

\begin{figure}
\[
\begin{array}{l|l|l}
\hline
& & \\[-.85em]
s & \repty{s} & \mdty{s} \\[.5ex]
\hline
& & \\[-.85em]
\Sk & \tau_1 &  \md{\tau_2} \\[.5ex]
\SPath{e}{s} & \repty{s} & \mdty{s} \\[.5ex]
\SPair{x}{s_1}{s_2} & (\repty{s_1}, \repty{s_2}) & \md{(\mdty{s_1}, \mdty{s_2})} \\[.5ex]
\SComp{s}{x}{e} & \List{\repty{s}} & \md{\List{\mdty{s}}}\\[.5ex]
\SPred{e} & () & \md{()}\\[.5ex]
\SOption{s} & \typ{Maybe}~\repty{s} & \md{(\typ{Maybe}~\mdty{s})} 
\end{array}
\]
\caption{\forest{} calculus representation and metadata types}
\label{fig:ctypes}
\end{figure}

\begin{figure*}
\begin{minipage}[t]{.45\textwidth}
\vspace*{3em}
\fbox{$\load{\Env}{F}{r}{s}{v,d}$}
\vspace*{1em}
\[
\begin{array}{c}
\inferrule
{ }
{ \load{\Env}{F}{r}{\Sk}{ \loadk{k}{\Env}{F}{r} } }
\\[2em]
\inferrule
{ \load{\Env}{F}{\eval{\PCons{r}{e}}{\Env}{\mathit{Path}}}{s}{v,d} }
{ \load{\Env}{F}{r}{\SPath{e}{s}}{v,d} }
\\[2em]
\inferrule
{ \load{\Env}{F}{r}{s_1}{ v_1,d_1 } \\\\
  \load{(\Env,x_{\REP} \mapsto v_1,x_{\MD} \mapsto d_1)}{F}{r}{s_2}{ v_2,d_2 }\\\\
  b = \valid{d_1} \wedge \valid{d_2} }
{ \load{\Env}{F}{r}{\SPair{x}{s_1}{s_2}}{ (v_1,v_2),(b, (d_1,d_2)) } }
\\[2em]
\inferrule
{ \eval{e}{\Env}{\List{\tau}} = [w_1,\tdots,w_k ]\\\\
  \forall i \in \{1,\tdots,k\}.~\load{(\Env,x \mapsto w_i)}{F}{r}{s}{v_i,d_i}\\\\
  b = \textstyle\bigwedge_i^k \valid{d_i} \quad \mathit{vs} = [v_1,\tdots,v_k] \quad \mathit{ds} = [d_1,\tdots,d_k] }
{ \load{\Env}{F}{r}{ \SComp{s}{x}{e} }{ \mathit{vs},(b,\mathit{ds}) } }
\\[2em]
\inferrule
{ b = \eval{e}{\Env}{\mathit{Bool}} }
{ \load{\Env}{F}{r}{ \SPred{e} }{ (),(b,()) } }
\\[2em]
\inferrule
{ r \not\in \dom{F} }
{ \load{\Env}{F}{r}{ \SOption{s_1} }{ \Nothing,(\True,\Nothing) } }
\\[2em]
\inferrule
{ r \in \dom{F} \quad \load{\Env}{F}{r}{s_1}{v_1,d_1} }
{ \load{\Env}{F}{r}{ \SOption{s_1} }{ \Just{v_1},(\valid{d_1},\Just{d_1}) } }
\end{array}
\]
\end{minipage}\hfill\vrule\hfill\begin{minipage}[t]{.5\textwidth}
\fbox{$\store{\Env}{F}{r}{v,d}{s}{F',\phi'}$}\\
\vspace*{1em}
\[
\begin{array}{c}
\inferrule
{ }
{ \store{\Env}{F}{r}{v,d}{\Sk}{ \storek{k}{\Env}{F}{r}{v,d} } }
\\[2em]
\inferrule
{ \store{\Env}{F}{\eval{\PCons{r}{e}}{\Env}{\mathit{Path}}}{v,d}{s}{F',\phi'} }
{ \store{\Env}{F}{r}{v,d}{\SPath{e}{s}}{F',\phi' } }
\\[2em]
\inferrule
{ \store{\Env}{F}{r}{ v_1,d_1 }{s_1}{ F_1',\phi_1' } \\\\
  \store{(\Env,x_{\REP} \mapsto v_1, x_{\MD} \mapsto d_1)}{F}{r}{ v_2,d_2 }{s_2}{ F_2',\phi_2' } \\\\
  \phi' = \lambda F'.~ (b = \valid{d_1} \wedge \valid{d_2}) \wedge \phi_1'(F') \wedge \phi_2'(F') }
{ \store{\Env}{F}{r}{ (v_1,v_2),(b,(d_1,d_2)) }{\SPair{x}{s_1}{s_2}}{ F_1' \FAppend F_2',\phi' } }
\\[2em]
\inferrule
{ \mathit{vs} = [v_1,\tdots,v_j] \quad \mathit{ds} = [d_1,\tdots,d_l]\\\\
  \eval{e}{\Env}{\List{\tau}} = [w_1,\tdots,w_m ] \quad  k = \min{(j,l,m)} \\\\
  \forall i \in \{1,\tdots,k\}.~\store{(\Env,x \mapsto w_i)}{F}{r}{v_i,d_i}{s}{F_i',\phi_i'}\\\\
  \phi' = \lambda F'.~ (j = l = m) \wedge (b = \textstyle\bigwedge_i^k \valid{d_i}) \wedge (\textstyle\bigwedge_i^k \phi_i'(F')) }
{ \store{\Env}{F}{r}{\mathit{vs},(b,\mathit{ds})}{ \SComp{s}{x}{e} }{ F_1' \FAppend \tdots \FAppend F_k',\phi' } }
\\[2em]
\inferrule
{ \phi' = \lambda F'.~ (b = \eval{e}{\Env}{\mathit{Bool}}) }
{ \store{\Env}{F}{r}{(),(b,())}{ \SPred{e} }{ F,\phi' } }
\\[2em]
\inferrule
{ \store{\Env}{F}{r}{ v_1,d_1 }{ s_1 }{ F',\phi_1' }\\\\
  \phi' = \lambda F'.~ (b = \valid{d_1}) \wedge (r \in \dom{F'}) \wedge \phi_1'(F') }
{ \store{\Env}{F}{r}{ \Just{v_1},(b,\Just{d_1}) }{ \SOption{s_1} }{ F',\phi' } }
\\[2em]
\inferrule
{ \phi' = \lambda F'.~(d = \Nothing) \wedge b \wedge r \not\in \dom{F'} }
{ \store{\Env}{F}{r}{ \Nothing,(b,d) }{ \SOption{s_1} }{ \FDelete{F'}{r},\phi' } }
\\[2em]
\inferrule
{ \store{\Env}{F}{r}{ v_1,\defaultmd{s_1} }{ s_1 }{ F',\phi_1' }\\\\
  \phi' = \lambda F'.~ \False }
{ \store{\Env}{F}{r}{ \Just{v_1},(b,\Nothing) }{ \SOption{s_1} }{ F',\phi' } }
\\[2em]
\end{array}
\]
\end{minipage}
\centerline{\hfill (a) \hfill\hfill \hspace*{.05\textwidth} (b) \hfill}
\caption{\forest{} calculus semantics for (a) loading and (b) storing}
\label{fig:csemantics}
\end{figure*}

\subsection{Formal Properties}

The semantics of \forest{} calculus specifications is carefully
designed to ensure some essential correctness properties. The first is
a basic type safety property, which states that the load function for
specifications $s$ generates representations and metadata belonging to
$\repty{s}$ and $\mdty{s}$ respectively.
%
\begin{proposition}[Load Type Safety]
  If $\load{\Env}{F}{r}{s}{v,d}$ and $\repty{s} = \tau_{\mathcal{R}}$
  and $\mdty{s} = \tau_{\mathcal{M}}$ then $\turn v :
  \tau_{\mathcal{R}}$ and $\turn d : \tau_{\mathcal{M}}$,
\end{proposition}
%
This property demonstrates that our type definitions are properly
aligned with the semantics of loading.  

To ensure that the semantics of loading is aligned with the semantics
of storing, we also prove the following two round-tripping properties.
%
\deferproof{theorem}{loadstore}{LoadStore}{%
  Let $\Env$ be an environment, $F$ a file system, $r$ a path, $s$ a
  specification, $v$ a representation, and $d$ metadata.  If
\[
\begin{array}{c}
\load{\Env}{F}{r}{s}{v,d} \\
\store{\Env}{F}{r}{v,d}{s}{F',\phi}
\end{array}
\]
then $F = F'$ and $\phi'(F')$.}
%
{ The proof is by induction on $s$. For every constant $\Sk$, we
  assume that $\loadkname{k}$ and $\storekname{k}$ satisfy the theorem. 
\begin{description}
%%%
%%% CONST
%%%
\item[Case:] $s = \Sk$\\[1ex]
%
By the definitions of the load and store functions we have 
\[ 
\begin{array}{rcl}
v,d  &=& \loadk{k}{\Env}{F}{r}\\
F',\phi' &=& \storek{k}{\Env}{F}{r}{v,d}\\
\end{array}
\]
By the assumptions about the behavior of $\loadkname{k}$ and
$\storekname{k}$, we have $F' = F$ and $\phi'(F')$, which finishes the
case.

%%%
%%% PATH
%%%
\item[Case:] $s=\SPath{e}{s_1}$\\[1ex]
%
By the definitions of the load and store functions we have 
%
\[ 
\begin{array}{l}
\load{\Env}{F}{r'}{s_1}{v,d}\\
\store{\Env}{F}{r'}{v,d}{s_1}{F',\phi'}
\end{array} \hspace*{1cm}
\begin{array}{rcl}
r' & = & \eval{\PCons{r}{e}}{\Env}{\mathit{path}}\\
\\
\end{array}
\]
%
By the induction hypothesis applied to $s_1$ we have $F' = F$ and
$\phi'(F')$, which finishes the case.

%%%
%%% Pair
%%%
\item[Case:]$s = \SPair{x}{s_1}{s_2}$\\[1ex]
%
By the definitions of the load and store functions we have 
\[ 
\begin{array}{l}
\load{\Env}{F}{r}{s_1}{v_1,d_1}\\
\load{\Env'}{F}{r}{s_2}{v_2,d_2}\\
\store{\Env}{F}{r}{v_1,d_1}{s_1}{F_1',\phi_1'}\\
\store{\Env'}{F}{r}{v_2,d_2}{s_2}{F_2',\phi_2'}\\
\end{array} \hspace*{1cm}
\begin{array}{rcl}
\Env' &=& \Env,x_{\REP} \mapsto v_1, x_{\MD} \mapsto d_1\\
b &= & \valid{d_1} \wedge \valid{d_2}\\
v &=& (v_1,v_2)\\
d &=& (b,(d_1,d_2))\\
F' &=& F_1' \FAppend F_2'\\
\phi' &=& \lambda F'.~(b = \valid{d_1} \wedge \valid{d_2}) \wedge \phi_1'(F') \wedge \phi_2'(F')\\
\end{array}
\]
By the induction hypothesis applied to $s_1$, we have $F_1' = F$ and
$\phi_1'(F_1')$. Likewise, by the induction hypothesis applied to
$s_2$, we have $F_2' = F$ and $\phi_2'(F_2')$. We immediately have
$\phi'(F')$ and $F' = F$ as (\FAppend) is idempotent, which finishes
the case.

%%%
%%% Comprehension
%%%
\item[Case:] $s = \SComp{s_1}{x}{e}$\\[1ex]
%
By the definitions of the load and store functions we have 
\[ 
\begin{array}{l}
\eval{e}{\Env}{\List{tau}} = [w_1,\tdots,w_k ]\\
\forall i \in \{1,...,k\}.\,\load{(\Env, x \mapsto w_i)}{F}{r}{s_1}{v_i,d_i}\\
\forall i \in \{1,...,k\}.\,\store{(\Env, x \mapsto w_i)}{F}{r}{v_i,d_i}{s}{F_i',\phi_i'}\\
\end{array}\hspace*{.75cm}\begin{array}{rcl}
v &=& [v_1,\tdots,v_k]\\
b &= &\bigwedge_i^k \valid{d_i}\\
d &=& (b,[d_1,\tdots,d_k])\\
F' &=& F_1' \FAppend \tdots \FAppend F_k'\\
\phi' &=& \lambda F'.~(b = \bigwedge_i^k \valid{d_i}) \wedge (\bigwedge_i^k \phi_i'(F'))\\
\end{array}
\]
By the induction hypothesis applied to $s$ ($k$ times), we have $F_i'
= F$ and $\phi'_i(F_i')$ for $i$ from $1$ to $k$. We immediately have
$\phi'(F')$ and $F' = F$ as (\FAppend) is idempotent, which finishes
the case.

%%%
%%% Predicate
%%%
\item[Case:] $s = \SPred{e}$\\[1ex]
%
By the definition of the load and store functions we have
\[
\begin{array}{rcl}
v &=& ()\\
d &=& (b,())\\
b &=& \eval{e}{\Env}{\mathit{bool}}\\
\end{array}\hspace*{1cm}\begin{array}{rcl}
F' &=& F\\
\phi' &=& \lambda F'.~(b = \eval{e}{\Env}{\mathit{bool}})\\
\end{array}
\]
Thus, we immediately have $F' = F$ and $\phi'(F')$.

%%%
%%% Option
%%%
\item[Case:] $s = \SOption{s_1}$\\[1ex]
%
We analyze two subcases:
\begin{description}
\item[Subcase:] $r \in \dom{F}$\\[1ex]
%
By the definition of the load and store functions we have
\[ 
\begin{array}{l}
\load{\Env}{F}{r}{s}{v_1,d_1}\\
\store{\Env}{F}{r}{v_1,d_1}{s}{ F',\phi_1' }\\
\end{array}\hspace*{2cm}\begin{array}{rcl}
v &=& \Just{v_1}\\
d &=& (b,\Just{d_1})\\
b &=& \valid{d_1}\\
\phi' &=& \lambda F'.~(b = \valid{d_1}) \wedge (r \in \dom{F}) \wedge \phi_1'(F')\\
\end{array}
\]
%
By the induction hypothesis applied to $s_1$ we have $F' = F$ and
$\phi_1'(F')$. We immediately have $\phi'(F')$.

\item[Subcase:] $r \not\in \dom{F}$\\[1ex]
%
By the definition of the load and store functions we have 
\[ 
\begin{array}{rcl}
v &=& \Nothing\\
d &=& (b,d_1)\\
d_1 &=& \Nothing\\
b &=& \True\\
\end{array}\hspace*{1cm}\begin{array}{rcl}
F' &=& \FDelete{F}{r}\\
\phi' &=& \lambda F'.~(d_1 = \Nothing) \wedge b \wedge  (r \not\in \dom{F'})
\end{array}
\]
We immediately have $F' = F$ and $\phi'(F')$, which finishes the case
and the inductive proof. \hfill \qed
\end{description}
\end{description}
}

\deferproof{theorem}{storeload}{StoreLoad}{ 
%
  Let $\Env$ be an environment, $F$ and $F'$ file systems, $r$ a path,
  $s$ a specification, $v$ a representation, $d$ and $d'$ metadata,
  and $\phi'$ a validator. If 
\[
\begin{array}{c}
\store{\Env}{F}{r}{v,d}{s}{F',\phi'} \qquad \phi'(F')\\
\load{\Env}{F'}{r}{s}{v',d'}
\end{array}
\]
then $v' = v$ and $\valid{d} = \valid{d'}$.}
%
{ We will prove a slightly stronger result which implies the theorem:
  for all environments $\Env$, file systems $F$, $G_1$, $G_2$, and
  $F'$, paths $r$, specifications $s$, representations $v$ and $v'$,
  metadata $d$ and $d'$, and constraints $\phi'$, if 
%
\[
\begin{array}{c}
  \store{\Env}{F}{r}{v,d}{s}{F',\phi'}\\
  \phi'(G_1 \FAppend F' \FAppend G_2)\\
  \load{\Env}{(G_1 \FAppend F' \FAppend G_2)}{r}{s}{v',d'}
\end{array}
\]
%
then $v' = v$ and $\valid{d} = \valid{d'}$. 

The proof is by induction on $s$. For every constant $\Sk$, we assume
that $\loadkname{k}$ and $\storekname{k}$ satisfy the strengthened
property.
\begin{description}
%%%
%%% Constant
%%%
\item[Case]: $s = \Sk$\\[1ex]
%
By the definitions of the load and store functions we have 
\[ 
\begin{array}{rcll}
F',\phi &=& \storek{k}{\Env}{F}{r}{v,d}\\
v',d'  &=& \loadk{k}{\Env}{(G_1 \FAppend F' \FAppend G_2)}{rs}\\
\end{array}
\]
By assumptions about the behavior of $\loadkname{k}$ and
$\storekname{k}$, we have $v' = v$ and $\valid{d} = \valid{d'}$,
which finishes the case. 

%%%
%%% Path
%%%
\item[Case:] $s=\SPath{e}{s_1}$\\[1ex]
%
By the definitions of the load and store functions we have
\[
\begin{array}{l}
\store{\Env}{F}{r'}{v,d}{s_1}{F',\phi'}\\
\load{\Env}{(G_1 \FAppend F' \FAppend G_2)}{r'}{s_1}{v',d'}
\end{array}\hspace*{1cm}\begin{array}{rcl}
r' &=& \eval{\PCons{r}{e}}{\Env}{\mathit{path}}\\
\\
\end{array}
\]
%
By the induction hypothesis applied to $s_1$ we have $v' = v$ and
$\valid{d} = \valid{d'}$, which finishes the case.

%%%
%%% Pair
%%%
\item[Case:] $s = \SPair{x}{s_1}{s_2}$\\[1ex]
%
By the definition of the load function we have
\[
\begin{array}{l}
\store{\Env}{F}{r}{v_1,d_1}{s_1}{F_1',\phi_1'}\\
\store{(\Env, x \mapsto v_1)}{F}{r}{v_2,d_2}{s_2}{F_2',\phi_2'}\\
\end{array}\hspace*{1cm}\begin{array}{rcl}
v &=& (v_1,v_2)\\
d &=& (b,(d_1,d_2))\\
F' &=& F_1' \FAppend F_2'\\
\phi' &=& \lambda F'.~(b = \valid{d_1} \wedge \valid{d_2}) \wedge \phi_1'(F') \wedge \phi_2'(F')\\
\end{array}
\]
By $\phi'(G_1 \FAppend F' \FAppend G_2)$ we have
\[
\begin{array}{l}
b = \valid{d_1} \wedge \valid{d_2}\\
\phi_1'(G_1 \FAppend F' \FAppend G_2)\\
\phi_2'(G_1 \FAppend F' \FAppend G_2)\\
\end{array}
\]
%
As $(\FAppend)$ is associative we also have,
\[
\begin{array}{rcl}
(G_1\FAppend (F_1\FAppend F_2)\FAppend G_2) &=& (G_1\FAppend F_1\FAppend (F_2\FAppend G_2))\\
(G_1\FAppend (F_1\FAppend F_2)\FAppend G_2) &=& ((G_1\FAppend F_1)\FAppend F_2\FAppend G_2),\\
\end{array}
\]
and hence:
\[
\begin{array}{l}
\phi_1'(G_1\FAppend F_1\FAppend (F_2\FAppend G_2))\\
\phi_2'((G_1\FAppend F_1)\FAppend F_2\FAppend G_2)\\
\end{array}
\]
By the definition of the load function we have 
\[ 
\begin{array}{rcll}
\load{\Env}{(G_1 \FAppend F_1 \FAppend (F_2 \FAppend G_2))}{r}{s_1}{v_1',d_1'}\\
\load{(\Env, x \mapsto v_1)}{((G_1 \FAppend F_1) \FAppend F_2 \FAppend G_2)}{r}{s_2}{v_2',d_2'}\\
\end{array}\hspace*{1cm}\begin{array}{rcl}
v' &=& (v_1',v_2')\\
b' &= & \valid{d_1} \wedge \valid{d_2}\\
d' &=& (b',(d_1',d_2'))\\
\end{array}
\]
%
By the induction hypothesis applied to $s_1$ and $s_2$, we have 
\[
\begin{array}{l}
v_1' = v_1\\
v_2' = v_2\\
\end{array}\hspace*{1cm}\begin{array}{l}
\valid{d_1} = \valid{d_1'}\\
\valid{d_2} = \valid{d_2'}\\
\end{array}
\]
It follows that $(v_1,v_2) = (v_1',v_2')$ and $b = b'$, which
finishes the case.

%%%
%%% Comprehension
%%%
\item[Case:] $s = \SComp{s}{x}{e}$\\[1ex]
By the definition of the store function we have
\[
\begin{array}{l}
\forall i \in \{1,\tdots,k\}.\,\store{(\Env, x \mapsto w_i)}{F}{r}{v_i,d_i}{s}{F_i',\phi_i}\\
\eval{e}{\Env}{\List{tau}} = [w_1,\tdots,w_m ]\\
\end{array}\hspace*{.75cm}\begin{array}{rcl}
v &=& [v_1,\tdots,v_j]\\
d &=& [d_1,\tdots,d_l]\\
k &=& \min{(j,l,m)}\\
F' &=& F_1' \FAppend \tdots \FAppend F_k'\\
\phi &=& \lambda F'.~(j = k = l) \wedge (b = \bigwedge_i^k \valid{d_i}) \bigwedge_i^k \phi_i(F')\\
\end{array}
\]
By $\phi'(G_1 \FAppend F' \FAppend G_2)$ we have
\[
\begin{array}{l}
k = j = l = m\\
b = (\bigwedge_i^k \valid{d_i})\\
\end{array}\hspace*{1cm}\begin{array}{l}
\forall i \in \{1,\tdots,k\}.\,\phi_i'(G_1 \FAppend F' \FAppend G_2) \\
\end{array}
\]
%
Let
\[
H_i = ((G_1\FAppend F_1\FAppend \tdots \FAppend F_{i-1}) \FAppend F_i \FAppend (F_{i+1} \FAppend \tdots \FAppend F_k \FAppend G_2))
\]
for $i$ from $1$ to $k$. As $(\FAppend)$ is associative we have,
\[
(G_1\FAppend (F_1\FAppend \tdots \FAppend F_k)\FAppend G_2) = H_i \quad \text{for}~i \in \{1,\tdots,k\}
\]
and hence:
\[
\phi_i'(H_i) \quad \text{for}~i \in \{1,\tdots,k\}
\]
%
By the definition of the load function we also have 
\[ 
\begin{array}{l}
\forall i \in \{1,\tdots,k\}\,\load{(\Env, x \mapsto w_i)}{H_i}{r}{s_1}{v_i,d_i'}\\
\end{array}\hspace*{1cm}\begin{array}{rcl}
v' &=& [v_1',\tdots,v_k']\\
b' &=& \bigwedge_i^k \valid{d_i'}\\
d' &=& (b',[d_1',\tdots,d_k'])\\
\end{array}
\]
%
By the induction hypothesis applied to $s$ ($k$ times), we have $v_i'
= v_i$ and $\valid{d_i'} = \valid{d_i}$ for = $i$ from $1$ to
$k$. It follows that $[v_1,\tdots,v_k] = [v_1',\tdots,v_k']$ and $b =
b'$, which finishes the case.

%%%
%%% Predicate
%%%
\item[Case:] $s = \SPred{e}$\\[1ex]
%
By the definitions of the store and load functions we have
\[
\begin{array}{rcl}
v &=& ()\\
d &=& ((),b)\\
F' &=& F\\
\phi' &=& \lambda F'.~(b = \eval{e}{\Env}{\mathit{bool}})\\
\end{array}\hspace*{1cm}\begin{array}{rcl}
v' &=& ()\\
d' &=& (b',())\\
b' &=& \eval{e}{\Env}{\mathit{bool}}\\
\end{array}
\]
By $\phi'(G_1 \FAppend F' \FAppend G_2)$ we have $b =
\eval{e}{\Env}{\mathit{bool}}$. It follows that $v = v'$ and $b =
b'$, which finishes the case.

%%%
%%% Option
%%%
\item[Case:] $s = \SOption{s_1}$\\[1ex]
%
We analyze several subcases:
\begin{description}
\item[Subcase:] $v = \Just{v_1}$ and $d = (b,\Just{d_1})$\\[1ex]
%
By the definition of the store function we have
\[
\begin{array}{l}
\store{\Env}{F}{r}{ v_1,d_1 }{s_1}{F',\phi_1'}\\
\end{array}\hspace*{1cm}\begin{array}{rcl}
\phi' &=& \lambda F'.~(b = \valid{d_1}) \wedge (r \in \dom{F}) \wedge \phi_1'(F')\\
\end{array}
\]
%
By $\phi'(G_1 \FAppend F' \FAppend G_2)$ we have $b = \valid{d_1}$ and
$r \in \dom{G_1 \FAppend F' \FAppend G_2}$.  By the definition of the
load function we also have
\[ 
\begin{array}{rcll}
\load{\Env}{(G_1 \FAppend F' \FAppend G_2)}{r}{s_1}{v_1',d_1'}\\
\end{array}\hspace*{1cm}\begin{array}{rcl}
v' &=& \Just{v_1'}\\
b' &=& \valid{d_1'}\\
d' &=& (b',\Just{d_1'})\\
\end{array}
\]
By the induction hypothesis applied to $s_1$ we have $v_1' = v_1$ and
$b' = b$. It follows that $v' = v$.

\item[Subcase:] $v = \Nothing$\\[1ex]
%
By the definition of the store and load functions we have
\[
\begin{array}{rcl}
F' &=& \FDelete{F}{r}\\
\phi' &=& \lambda F'.~(d = \Nothing) \wedge b \wedge (r \not\in \dom{F'})\\
\end{array}\hspace*{1cm}\begin{array}{rcl}
v' &=& \Nothing\\
b' &=& \True\\
d' &=& (b',\Nothing)\\
\end{array}
\]
As $\phi'(G_1 \FAppend F' \FAppend G_2)$ we have $d
= \Nothing$ and $b$ and $r \not\in \dom{G_1 \FAppend F' \FAppend
  G_2}$. Thus, we immediately have $v' = v$ and $b' = b$. 
\item[Subcase:] $v = \Just{v_1}$ and $d = \Nothing$\\[1ex]
%
Vacuously holds: by the definition of the store function we have
$\phi' = \lambda F'.~\False$, which contradicts the assumption that
$\phi'(G_1 \FAppend F' \FAppend G_2)$.
\end{description}
Thus, in each subcase we have $v' = v$ and $\valid{d} = \valid{d'}$,
which finishes the subcase and the inductive proof. \hfill \qed
\end{description}}

\smallskip

\noindent The first theorem states that loading from a file system $F$
and immediately storing the resulting representation and metadata
yields the original file system and, moreover, that this file system
will satisfy the validator produced by the store function. This
guarantees that the store function will not disturb information in the
file system if possible, such as the information outside of the
fragment of the file system described by the specification. It also
establishes that the validator is not the trivial predicate on file
systems that always returns false.

The second theorem states that storing an arbitrary representation 
and then loading the resulting file system yields the same
representation and equally valid metadata,
provided the stored file system satisfies the validator.  
 This theorem ensures that the
store function reflects all of the information contained in the
representation in the updated file system.

These properties are based on the general correctness conditions that
have been proposed for bidirectional transformations in the context of
lenses~\cite{lenses}, but are generalized here to accommodate the
inconsistencies that can arise when working with imperfect, ad hoc
data. The proofs of these theorems can be found in the accompanying
technical report.

