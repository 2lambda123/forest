A {\em filestore} is a structured collection of data files housed in a
conventional hierarchical file system. Many applications use file
systems as a poor-man's database, and the correct execution of these
applications requires that the collection of files, directories, and
symbolic links stored on disk satisfy a variety of precise
invariants. Moreover, all of these structures must have acceptable
ownership, permission, and timestamp attributes. Unfortunately,
current programming languages do not provide support for documenting
assumptions about filestores, detecting errors, or safely loading from
and storing to them.

This paper describes the design, implementation, and semantics of
\forest{}, a novel domain-specific language for describing
filestores. The language uses a type-based metaphor to specify the
expected structure, attributes, and invariants of filestores.
\forest{} generates loading and storing functions that make it easy to
connect data on disk to an isomorphic representation in memory that
can be manipulated as if it were any other data structure.  \forest{}
also generates metadata that describes the degree to which the
structures on the disk conform to the specification, making error
detection easy. Hence, in a nutshell, \forest{} extends the
rigorous discipline of typed 
programming languages and many of their benefits to the untyped 
world of file systems.

We have implemented \forest{} as an embedded domain-specific language
in \haskell{}. In addition to generating infrastructure for reading,
writing and checking file systems, our implementation generates a
type class instances that make it easy to build generic tools that
operate over arbitrary \filestores.  We illustrate the utility of
this infrastructure by building a file system visualizer, a file access
checker, a generic query interface, description-directed variants of 
several standard UNIX shell tools and (circularly) a simple \forest{}
description inference engine.   Finally, we formalize a core fragment 
of \forest{} in a
semantics inspired by classical tree logics and prove round-tripping
laws showing that the loading and storing functions behave sensibly.
