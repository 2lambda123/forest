\documentclass[10pt]{article}

\usepackage{times}
\usepackage{code} 
\newcommand{\cut}[1]{}

\newcommand{\appref}[1]{Appendix~\ref{#1}}
\newcommand{\secref}[1]{Section~\ref{#1}}
\newcommand{\tblref}[1]{Table~\ref{#1}}
\newcommand{\figref}[1]{Figure~\ref{#1}}
\newcommand{\listingref}[1]{Listing~\ref{#1}}
%\newcommand{\pref}[1]{{page~\pageref{#1}}}

\newcommand{\eg}{{\em e.g.}}
\newcommand{\cf}{{\em cf.}}
\newcommand{\ie}{{\em i.e.}}
\newcommand{\etc}{{\em etc.\/}}
\newcommand{\naive}{na\"{\i}ve}
\newcommand{\role}{r\^{o}le}
\newcommand{\forte}{{fort\'{e}\/}}
\newcommand{\appr}{\~{}}

\newcommand{\bftt}[1]{{\ttfamily\bfseries{}#1}}
\newcommand{\kw}[1]{\bftt {#1}}
\newcommand{\Pthen}{\kw{Pthen}}
\newcommand{\pads}{\textsc{pads}}
\newcommand{\padsl}{\textsc{padsl}}
\newcommand{\padst}{\textsc{pads/t}}
\newcommand{\datatype}{\textsc{PADS/T}}
%\newcommand{\datatype}{\textsc{DataType}}
\newcommand{\C}{\textsc{C}}
\newcommand{\perl}{\textsc{Perl}}
\newcommand{\ml}{\textsc{ml}}
\newcommand{\sml}{\textsc{sml}}
\newcommand{\smlnj}{\textsc{sml/nj}}
\newcommand{\java}{\textsc{java}}
\newcommand{\ddl}{\textsc{ddl}}
\newcommand{\xml}{\textsc{xml}}
\newcommand{\datascript}{\textsc{DataScript}}
\newcommand{\packettypes}{\textsc{PacketTypes}}
\newcommand{\erlang}{\textsc{Erlang}}

\newcommand{\Core}{Ad hoc}
\newcommand{\core}{ad hoc}
\newcommand{\pvalue}{\core{} value}
\newcommand{\ppat}{\core{} pattern}
\newcommand{\ptype}{\core{} type}

\newcommand{\padsc}{\textsc{pads}/\C{}}
\newcommand{\padsml}{\textsc{pads}/\ml{}}

\newcommand{\dibbler}{Sirius}
\newcommand{\ningaui}{Altair}
\newcommand{\darkstar}{Regulus}

\newcommand{\pdgood}{{\tt G}}
\newcommand{\pdbad}{{\tt B}}
\newcommand{\pdnest}{{\tt N}}
\newcommand{\pdsem}{{\tt S}}
\newcommand{\ptypes}{T}
\newcommand{\patreadpd}[2]{{\tt #1<<#2>>}}
\newcommand{\btm}{\cd{BOT}}


\newcommand{\lsem}{{[\![}}
\newcommand{\rsem}{{]\!]}}


\newcommand{\figHeight}[4]{\begin{figure}[tb]
	\centerline{
	            \epsfig{file=#1,height=#4}}
	\caption{#2}
	\label{#3}
	\end{figure}}

%% Environment for typesetting BNF grammars. Uses display math mode.
\newenvironment{bnf}
     {%% local command definitions:
        %% BNF definition symbol
      \def\->{\rightarrow}
%%      \def\::={{::=} &}
      \def\::={\bnfdef &}
      \def\|{\bnfalt}
      \newcommand{\name}[1]{\text{##1}}
        %% non-terminal
      \newcommand{\nont}[1]{{##1}}
      \newcommand{\meta}[1]{& ##1 &}
      \newcommand{\descr}[1]{& \text{// ##1}}
      \newcommand{\opt}[1]{ [##1] }
      \newcommand{\opnon}[1]{\opt{\nont{##1}}}
      \newcommand{\none}{\epsilon}
      \newcommand{\nwln}{\\ &&&}
      \newcommand{\nlalt}{\\ && \| &}
      \[\begin{array}{lrlll}
     }
     {\end{array}\]}

\newcommand{\mcd}[1]{\mathtt{#1}}
\newcommand{\ppair}[3]{#1{:}#2 \mathrel{**} #3}
\newcommand{\parray}[3]{#1\;\mcd{Parray}(#2,#3)}
\newcommand{\pset}[3]{\{#1{:}#2\,|\,#3\}}
\newcommand{\pstream}[1]{#1\;\mcd{stream}}
\newcommand{\precord}[1]{\{\{#1\}\}}

\setlength{\textwidth}{6.5in}
\setlength{\textheight}{9in}
\setlength{\oddsidemargin}{0in}
\setlength{\evensidemargin}{0in}
\setlength{\parskip}{5pt}

\title{\vskip -.5in \pads{}: Processing Arbitrary Data Streams}
\date{\today}
\author{
  Kathleen Fisher\\
  AT\&T Labs --- Research \\
  \small\texttt{kfisher@research.att.com}
\and
  Robert E. Gruber \\
  AT\&T Labs --- Research \\
  \small\texttt{gruber@research.att.com}
}

\begin{document}

\maketitle
\thispagestyle{empty}
\newcommand{\smallersp}{\vskip -2ex}
\newcommand{\smallerend}{\vskip -5ex}
\section{Introduction}
\smallersp{}
Transactional data streams, such as sequences of stock-market buy/sell orders,
credit-card purchase records, web server entries, and electronic fund
transfer orders, can be mined very profitably.  As an example,
researchers at AT\&T have built customer profiles from streams of
call-detail records to significant financial effect~\cite{kdd98,kdd99,kdd00}.   

Often such streams are high-volume: AT\&T's call-detail stream contains
roughly 300~million calls per day requiring approximately 7GBs of
storage space.  Typically, such stream data arrives ``as is'' in
\textit{ad hoc} formats with poor documentation.  In addition, the
data frequently contains errors.  The appropriate response to such
errors is application-specific. Some applications can simply discard
unexpected or erroneous values and continue processing.  For other
applications, however, errors in the data can be the most interesting
part of the data.  

Understanding a new data stream and producing a suitable parser are
crucial first steps in any use of stream data.  Unfortunately, writing
parsers for such data is a difficult task, both tedious and
error-prone. It is complicated by lack of documentation, convoluted
encodings designed to save space, the need to handle errors
robustly, and the need to produce efficient code to cope with the
scale of the stream.  Often, the hard-won understanding of the data
ends up embedded in parsing code, making long-term maintenance
difficult for the original writer and sharing the knowledge with
others nearly impossible.

The goal of the PADS project is to provide languages and tools for
simplifying data stream analysis.  We have a preliminary design of a
declarative data-description language, \padsl{}, expressive enough to
describe the data feeds we see at AT\&T in practice, including ASCII,
binary, EBCDIC, Cobol, and mixed data formats.  From
\padsl{} we generate a tunable \C{} library with functions for
parsing, manipulating, and summarizing the data.   

\smallerend{}
\section{\pads{} language}
\smallersp
Intuitively, a \padsl{} description specifies complete information
about the physical layout and semantic constraints for the associated
data stream.  Most type declarations in \padsl{} are analogous to type
declarations in \C{}.
\padsl{} has an extensible set of base types that
specify how to read and verify atomic pieces of data such as ASCII
32-bit integers (\cd{Pa_int32}) and binary bytes (\cd{Pb_int8}).
Verification conditions for such base types include checking that the
resulting number fits in the indicated space, \ie, 16-bits for
\cd{Pa_int16}.  \padsl{} has 
\kw{Pstruct}s, \kw{Punion}s, and \kw{Parray}s to describe
record-like structures, alternatives, and sequences, respectively.
Each of these
types can have an associated predicate that indicates whether a
value calculated from the physical specification is indeed a legal
value for the type.  For example, a predicate might require that two
fields of a \kw{Pstruct} are related or that the elements
of a sequence are in increasing order.  Programmers can specify such
predicates using \padsl{} expressions or functions.
\padsl{} \kw{Ptypedef}s can be used
to define new types that add further constraints to existing types.

In addition, \padsl{} types can be parameterized by values.
This mechanism
serves both to reduce the number of base types and to permit the
format of later portions of the data to depend upon earlier portions.
For example, 
the base type \cd{Pa_uint32_FW(:3:)} specifies an unsigned integer
physically represented by exactly 3 ASCII characters, while the type
\cd{Pa_string(:' ':)} 
describes an ASCII string terminated by a space.  Parameters can be 
used with compound types to specify the size of an array or which
branch of a union should be taken.

As an example, consider the common log format for Web server logs.  A
typical record looks like the following:
\begin{verbatim}
207.136.97.49 - - [15/Oct/1997:18:46:51 -0700] "GET /tk/p.txt HTTP/1.0" 200 30
\end{verbatim}

\noindent
recording the IP address of the requester; either a dash or the owner
of the TCP session; either a dash or the login of the requester; the
date; the actual request, which consists of the HTTP method, the
requested URL, the HTTP version number; a response code; and the
number of bytes returned.  A \padsl{} type describing the request
portion is
\begin{code}
\kw{Pstruct} http_request_t \{
  '\\"'; http_method_t    meth;           /- Method used during request
  ' ';  Pa_string(:' ':) req_uri;        /- Requested uri.
  ' ';  http_v_t         version : checkVersion(version, meth);
                                         /- HTTP version number of request 
  '\\"';
\};
\end{code}
This \kw{Pstruct} uses (omitted) auxiliary types \cd{http_method_t} and
\cd{http_v_t} to describe
the HTTP method and version formats, respectively.
It uses character literals (\cd{'\\"'} and \cd{' '}) to consume
the quotes and 
spaces from the physical representation. 
The \cd{version} field has a constraint predicate \cd{checkVersion}
which ensures that obsolete HTTP methods \cd{LINK} and \cd{UNLINK} 
are only used with HTTP version \cd{1.0}.

\smallerend{}
\section{Generated library}
\smallersp
From each type in a \padsl{} description, we generate C declarations for
(1) an in-memory representation, 
(2) a {\em checkset}\/-mask, which allows users to specify 
which portions of the data are relevant to their applications,
(3) an error-description, which we use to describe physical and
semantic errors detected during parsing, 
(4) a parse function, and 
(5) utility functions.
The parse function takes as arguments pointers to a checkset-mask, an
in-memory representation, and an error-description.  The generated
library maintains the invariant that if the checkset-mask requests
that a data item be verified and set, and if the error description
indicates no error, then the in-memory representation satisfies the
semantic constraints on the data.

The checkset-mask allows the user to specify with fine granularity
which constraints the parser should check and which portions of the
in-memory representation it should fill in.  This control allows the
description-writer to specify all known constraints about the data
without worrying about the run-time cost of verifying potentially
expensive constraints for time-critical applications.

Appropriate error-handling can be as important as processing
error-free data.  The error descriptor marks which portions of the
data contain errors and characterizes the detected errors.
Depending upon the nature of the errors and the desired application,
programmers can take the appropriate action: halting the program,
discarding parts of the data, or repairing the errors.

By supporting multiple entry-points, we accommodate larger-scale data.
For a small file, programmers can define a \padsl{} type that describes
the entire file and use that type's parsing function to read the whole
file with one call.  For larger-scale data, programmers can sequence
calls to parsing functions that read manageable portions of the file,
\eg{}, reading a record at a time in a loop.

\smallerend{}
\section{Related work}
\smallersp
There are many tools for describing data formats. For example,
\textsc{ASN.1}~\cite{asn} and \textsc{ASDL}~\cite{asdl} are both
systems for declaratively describing data and then generating
libraries for manipulating that data.  In contrast to \pads{},
however, both these systems specify the {\em logical\/} representation
and automatically generate a {\em physical\/} representation.
Although useful for many purposes, this technology does not help
process data that arrives in predetermined, \textit{ad hoc} formats.


More closely related work allows declarative descriptions of physical
data~\cite{sigcomm00,erlang-bit-syntax,gpce02}, motivated by parsing
\textsc{TCP/IP} packets and \java{} jar-files.  In contrast to our
work, these systems only handle binary data and assume the data is
error-free or halt parsing if an error is detected.  

\smallerend{}
\section{Conclusion}
\smallersp
Further information and a source-code distribution of \pads{} is
available from:
\begin{centercode}
http://www.research.att.com/projects/pads
\end{centercode}
\bibliographystyle{jhr-alpha} 
\bibliography{mpds}
\end{document}
