\documentclass[10pt]{article}

\usepackage{times}
\usepackage{code} 
\newcommand{\cut}[1]{}

\newcommand{\appref}[1]{Appendix~\ref{#1}}
\newcommand{\secref}[1]{Section~\ref{#1}}
\newcommand{\tblref}[1]{Table~\ref{#1}}
\newcommand{\figref}[1]{Figure~\ref{#1}}
\newcommand{\listingref}[1]{Listing~\ref{#1}}
%\newcommand{\pref}[1]{{page~\pageref{#1}}}

\newcommand{\eg}{{\em e.g.}}
\newcommand{\cf}{{\em cf.}}
\newcommand{\ie}{{\em i.e.}}
\newcommand{\etc}{{\em etc.\/}}
\newcommand{\naive}{na\"{\i}ve}
\newcommand{\role}{r\^{o}le}
\newcommand{\forte}{{fort\'{e}\/}}
\newcommand{\appr}{\~{}}

\newcommand{\bftt}[1]{{\ttfamily\bfseries{}#1}}
\newcommand{\kw}[1]{\bftt {#1}}
\newcommand{\Pthen}{\kw{Pthen}}
\newcommand{\pads}{\textsc{pads}}
\newcommand{\padsl}{\textsc{padsl}}
\newcommand{\padst}{\textsc{pads/t}}
\newcommand{\datatype}{\textsc{PADS/T}}
%\newcommand{\datatype}{\textsc{DataType}}
\newcommand{\C}{\textsc{C}}
\newcommand{\perl}{\textsc{Perl}}
\newcommand{\ml}{\textsc{ml}}
\newcommand{\sml}{\textsc{sml}}
\newcommand{\smlnj}{\textsc{sml/nj}}
\newcommand{\java}{\textsc{java}}
\newcommand{\ddl}{\textsc{ddl}}
\newcommand{\xml}{\textsc{xml}}
\newcommand{\datascript}{\textsc{DataScript}}
\newcommand{\packettypes}{\textsc{PacketTypes}}
\newcommand{\erlang}{\textsc{Erlang}}

\newcommand{\Core}{Ad hoc}
\newcommand{\core}{ad hoc}
\newcommand{\pvalue}{\core{} value}
\newcommand{\ppat}{\core{} pattern}
\newcommand{\ptype}{\core{} type}

\newcommand{\padsc}{\textsc{pads}/\C{}}
\newcommand{\padsml}{\textsc{pads}/\ml{}}

\newcommand{\dibbler}{Sirius}
\newcommand{\ningaui}{Altair}
\newcommand{\darkstar}{Regulus}

\newcommand{\pdgood}{{\tt G}}
\newcommand{\pdbad}{{\tt B}}
\newcommand{\pdnest}{{\tt N}}
\newcommand{\pdsem}{{\tt S}}
\newcommand{\ptypes}{T}
\newcommand{\patreadpd}[2]{{\tt #1<<#2>>}}
\newcommand{\btm}{\cd{BOT}}


\newcommand{\lsem}{{[\![}}
\newcommand{\rsem}{{]\!]}}


\newcommand{\figHeight}[4]{\begin{figure}[tb]
	\centerline{
	            \epsfig{file=#1,height=#4}}
	\caption{#2}
	\label{#3}
	\end{figure}}

%% Environment for typesetting BNF grammars. Uses display math mode.
\newenvironment{bnf}
     {%% local command definitions:
        %% BNF definition symbol
      \def\->{\rightarrow}
%%      \def\::={{::=} &}
      \def\::={\bnfdef &}
      \def\|{\bnfalt}
      \newcommand{\name}[1]{\text{##1}}
        %% non-terminal
      \newcommand{\nont}[1]{{##1}}
      \newcommand{\meta}[1]{& ##1 &}
      \newcommand{\descr}[1]{& \text{// ##1}}
      \newcommand{\opt}[1]{ [##1] }
      \newcommand{\opnon}[1]{\opt{\nont{##1}}}
      \newcommand{\none}{\epsilon}
      \newcommand{\nwln}{\\ &&&}
      \newcommand{\nlalt}{\\ && \| &}
      \[\begin{array}{lrlll}
     }
     {\end{array}\]}

\newcommand{\mcd}[1]{\mathtt{#1}}
\newcommand{\ppair}[3]{#1{:}#2 \mathrel{**} #3}
\newcommand{\parray}[3]{#1\;\mcd{Parray}(#2,#3)}
\newcommand{\pset}[3]{\{#1{:}#2\,|\,#3\}}
\newcommand{\pstream}[1]{#1\;\mcd{stream}}
\newcommand{\precord}[1]{\{\{#1\}\}}

\setlength{\textwidth}{6.5in}
\setlength{\textheight}{9in}
\setlength{\oddsidemargin}{0in}
\setlength{\evensidemargin}{0in}
\setlength{\parskip}{5pt}


\title{\vskip -1in \pads{}: Processing Arbitrary Data Streams}
\date{\today}
\author{
  Kathleen Fisher\\
  AT\&T Labs --- Research \\
  \small\texttt{kfisher@research.att.com}
\and
  Robert E. Gruber \\
  AT\&T Labs --- Research \\
  \small\texttt{gruber@research.att.com}
}

\begin{document}

\maketitle
\thispagestyle{empty}

\section{Introduction}
Transactional data streams, such as stock-market buy/sell orders,
credit-card purchase records, web server entries, and electronic fund
transfer orders, can be mined very profitably.  As an example,
researchers at AT\&T have extracted information from streams of
call-detail records to significant effect~\cite{kdd98,kdd99,kdd00}.   

Often such streams are high-volume: the call-detail stream contains
roughly 300~million calls per day requiring approximately 7GBs of
storage space.  Typically, such stream data arrives ``as is'' in
\textit{ad hoc} formats with poor documentation.  In addition, the
data frequently contains errors.  The appropriate response to such
errors is application-specific. Some applications can simply discard
unexpected or erroneous values and continue processing.  For other
applications, however, errors in the data can be the most interesting
part of the data.  

Understanding a new data stream and producing a suitable parser are
crucial first steps in any use of stream data.  Unfortunately, writing
parsers for such data is a difficult task, both tedious and
error-prone. It is complicated by lack of documentation, convoluted
encodings designed to save disk space, the need to handle errors
robustly, and the need to produce efficient code to cope with the
scale of the stream.  Often, the hard-won understanding of the data
ends up embedded in parsing code, making long-term maintenance
difficult for the original writer and sharing the knowledge with
others nearly impossible.

The goal of the PADS project is to provide languages and tools for
simplifying data stream analysis.  We have a preliminary design of a
declarative data-description language, PADSL, expressive enough to
describe the data feeds we see at AT\&T in practice, including both ASCII,
binary, and mixed data.  From PADSL we generate a tunable C library with
functions for parsing, manipulating, and summarizing the data.  

\section{\pads{} language}
Intuitively, a \pads{} description specifies complete information
about the physical layout and semantic constraints for the associated
data stream.  The structure of the language is analogous to \C{}'s
type structure.  \pads{} has an extensible set of base types that
specify how to read and verify atomic pieces of data such as ASCII
integers (\cd{aint32}) and binary bytes (\cd{bint8}).
Verification conditions for such base types include checking that the
number produced fits in the indicated space, \ie, 32-bits for
\cd{aint32}.  \pads{}
\kw{pstruct}s allow users to describe record-like structures, 
\kw{punion}s alternatives, and \kw{array}s sequences.  Each of these
types can have a predicate associated with it that indicates whether a
value calculated from the physical specification is indeed a legal
value for the type.  Such conditions might be that two fields of a
\cd{pstruct} are related via a particular function or that the elements
of a sequence are in increasing order.  \pads{} \cd{typedef}s allow
users to add additional constraints to existing types. 

In addition, each of these types can be parameterized by values, which
serves both to reduce the number of base types and permit later
portions of the data to depend upon earlier portions.  For example,
base type \cd{auint32FW(:3:)} specifies an unsigned ASCII integer
represented on disk with 3 characters, while \cd{astring(:' ':)}
describes an ASCII string terminated by a space.  Parameters can be 
used with compound types to specify the size of an array or which
branch of a union should be taken.

As an example, consider the common log format for web server logs.  A
typical record looks like the following:
\begin{verbatim}
207.136.97.49 - - [15/Oct/1997:18:46:51 -0700] "GET /tk/p.gif HTTP/1.0" 200 30
\end{verbatim}
recording the IP address of the requester; either a dash or the owner
of the TCP session; either a dash or the login of the requester; the
date; the actual request, which consists of the HTTP method, the
requested URL, the HTTP version number; a response code, and the
number of bytes returned.  The \pads{} description of the request is 
\begin{code}
\kw{pstruct} http_request_t \{
  '\\"'; http_method_t  meth;            /- Method used during request
  ' ';  astring(:' ':) req_uri;         /- Requested uri.
  ' ';  http_v_t       version : checkVersion(version, meth);
                                        /- HTTP version number of request 
  '\\"';
\};
\end{code}
The \cd{version} field of this \cd{pstruct} has a constraint predicate
which ensures that obsolte versions \cd{LINK} and \cd{UNLINK} do not
appear after HTTP version \cd{1.0}.

\section{Generated Library}
From each type in a \pads{} description, we generate 
1) an in-memory representation, 
2) an error-mask, which allows users to specify
which portions of the data are relevant to their applications,
3) an error-description, which we use to describe physical and
semantic errors detected during parsing, 
4) a parsing function, and 
5) utility functions.
Each parse function takes as an argument an error mask and returns
both an error description and the in-memory representation of the
physical data.  
The invariant maintained by the library is that if the error mask
requests a portion of the data and the error description indicates no
error, then the in-memory representation satisfies all of the semantic
constraints of the data. 

By supporting multiple entry-points, we accomodate larger-scale data.
For a small file, the user can call a single parse-function that will
read the entire contents of a file.  For larger-scale data, programmers
can sequence calls to parser functions for portions of the data,
perhaps parsing each record in a file in a loop, for example. 


\section{Related work}
\textsc{ASN.1}~\cite{asn} and \textsc{ASDL}~\cite{asdl} are both
systems for describing data declaratively and then generating
libraries for manipulating that data.  In contrast to \pads{},
however, both these systems specify the logical representation and
automatically generate a physical representation.  Although useful for
many purposes, this technology does not help process data that arrives
in an already-determined format. 

More closely related work allows declarative descriptions of binary
data~\cite{sigcomm00,erlang-bit-syntax,gpce02}, motivated by parsing
\textsc{TCP/IP} packets and \java{} jar-files.  In contrast to our
work, these systems only handle binary data and assume the data is
error-free or halt parsing if an error is detected. 

\section{Conclusion}
Source code for \pads{} will be available for download shortly with
a non-commercial use license from: 
\begin{centercode}
http://www.research.att.com/projects/pads
\end{centercode}
\bibliographystyle{jhr-alpha} 
\bibliography{mpds}
\end{document}
