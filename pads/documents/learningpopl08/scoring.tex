We use an information theoretic scoring function to assess the quality of 
our inferred descriptions and to decide whether to apply
rewriting rules to refine candidate descriptions.
Intuitively, a good description is one that is both {\em compact} and
{\em precise}.  There are trivial descriptions of any data source
that are highly compact (\eg{}, the description that says
the data source is a string terminated by end of file) or
perfectly precise (\eg{}, the data itself abstracts nothing and
therefore serves as its own description).  A good scoring function
balances these opposing goals.  As is common
in machine learning, we have defined a scoring function based on the
{\em Minimum Description Length Principle} (MDL), which states that
a good description is one that minimizes the cost (in bits) of transmitting
the data.  Mathematically,
if $T$ is a description and $d_1,\ldots,d_k$ are representations of
the $k$ chunks in our training set, parsed according to $T$, then the 
total cost in bits is:
\[
\totalcost{T}{d_1,\ldots,d_k} = \costdescription{T} + \costdata{T}{d_1,\ldots,d_k}
\]
where $\costdescription{T}$ is the number of bits to transmit the
description and $\costdata{T}{d_1,\ldots,d_k}$ is the number of bits
to transmit the data {\em given the description}.

%\paragraph*{The cost of transmitting a type.}
Intuitively, the cost in bits of transmitting a description
is the cost of transmitting the sort of description (\ie{}, \cd{struct},
\cd{union}, \cd{enum}, \etc{}) plus the cost of transmitting all of its
subcomponents.  
%Figure~\ref{fig:cost-type} defines this cost function.
For example, the cost of transmitting a struct type
$\costdescription{\mathtt{struct \{} T_1;\ldots ;T_k; \mathtt{\}}}$ is 
$\cardt + \sum_{i=1}^{k} \costdescription{T_i}$
%$b(p_1,\ldots,p_k)$ is $ \cardt + \sum_{i=1}^{k} \costparam{p_i}$,
where $\cardt$ is the log of the number of different sorts of type 
constructors (24 of them in the \ir{} presented in this paper).
%and $\costparam{p}$ is the cost of encoding the parameter $p$.
%% \figref{fig:cost-type} does not show the
%% cost of encoding variables, constants and parameters, but it is
%% straightforward.  For example, the cost of 
%% encoding an ASCII character is 8 bits.  The cost of encoding a string
%% is the cost of encoding its length (assumed to be a 32-bit quantity)
%% plus the cost of encoding each character in turn.
We have defined the recursive cost function mathematically in full,
but space limitations preclude giving that definition here.

The cost of encoding data relative to selected types is shown in 
Figure~\ref{fig:cost-data}.  The top of the figure defines
the cost of encoding all data chunks relative to the type $T$; it is
simply the sum of encoding each individual chunk relative to $T$.  

In the middle of the figure,
we define the cost of encoding a chunk relative to one of the integer base 
types; other base types are handled similarly.  Notice that the cost of 
encoding an integer relative to the constant type \cd{PintConst} is
zero because the type itself contains all information
necessary to reconstruct the integer-- no data need be transmitted.
The cost of encoding data relative to \cd{Pint32} or \cd{Pint64} types 
is simply 32 or 64 bits, respectively.  Finally, we artificially set the cost of
ranged types \cd{PintRanged}$(p_{min},p_{max})$ to be infinity because
our experiments reveal that attempting to define integer types with
minimum and maximum values usually leads to overfitting of the data.\footnote{
We nevertheless retain \cd{PintRanged} types in our \ir{} to encode
the range of values found during the value-space analysis.  During the
rewriting phase, we use this range information to rewrite \cd{PintRanged}
into other integer types.  Since the
cost of encoding \cd{PintRanged} is so high, the appropriate rewriting 
is guaranteed to be applied.  In the future, we may emit this range information
as comments in the generated descriptions.}

The last section of Figure~\ref{fig:cost-data} presents the cost of
encoding data relative to selected type constructors.  
The cost of encoding a \cd{struct} is the sum of the costs of encoding
its component parts.  The cost of encoding a \cd{union} is the cost
of encoding the branch number ($log(k)$ if the union has $k$ branches)
plus the cost of encoding the branch itself.  The cost of encoding
an \cd{enum} is the cost of encoding its tag only -- given the tag,
the underlying data is determined by the type.  The cost of encoding
a \cd{switch} is the cost of encoding the branch only -- the tag need not
be encoded because it is determined by the type and earlier data.

%% \begin{figure}
%% Miscellaneous definitions:
%% \[
%% \begin{array}{@{\,}l@{\,}c@{\,}l}
%% \cardt &=& \mbox{log of the number of type constructors in the \ir{}}
%% \end{array}
%% \]
%% Cost of transmitting constants and parameters:
%% \[
%% \begin {array} {@{}l@{\,}c@{\,}l}
%% \costchar{a}  &=& \mbox{Cost of transmitting character } a \\
%% \coststring{s}  &=& \mbox{Cost of transmitting string } s \\
%% \costint{i}  &=& \mbox{Cost of transmitting integer } i \\
%% \costconst{c}  &=& \mbox{Cost of transmitting constant } c \\
%% \costvar{x} &=& \mbox{Cost of transmitting variable } \\
%% \costparam{p}  &=& \mbox{Cost of transmitting parameter } p \\
%% \end{array}
%% \]
%% Cost of transmitting a type:
%% \[
%% \begin{array}{@{}l@{\,}c@{\,}l}
%% \costdescription{b(p_1,\ldots,p_k)} &=& 
%%   \cardt + \sum_{i=1}^{k} \costparam{p_i} \\
%% \costdescription{x{:}b(p_1,\ldots,p_k)} &=& 
%%   \cardt + \costvar{x} + \sum_{i=1}^{k} \costparam{p_i} \\
%% \costdescription{\mathtt{struct \{} T_1;\ldots ;T_k; \mathtt{\}}} &=& 
%%   \cardt + \sum_{i=1}^{k} \costdescription{T_i} \\
%% \costdescription{\mathtt{union \{} T_1; \ldots ;T_k; \mathtt{\}}} &=& 
%%   \cardt + \sum_{i=1}^{k} \costdescription{T_i} \\
%% \costdescription{\mathtt{enum \{} c_1,\ldots,c_k \mathtt{\}}} &=& 
%%   \cardt + \sum_{i=1}^{k} \costconst{c_i} \\
%% \end{array}
%% \]
%% \caption {Cost of transmitting a type, selected rules}
%% \label{fig:cost-type}
%% \end{figure}

\begin{figure}
Cost of encoding all training data relative to a type:
\[
\begin{array}{@{}lcl}
\costdata{T}{d_1,\ldots,d_k} &=& \sum_{i=1}^{k} \adc{T}{d_i} \\
\end{array}
\]
Cost of encoding a single chunk relative to selected base types:
\[
\begin{array}{@{}lcl}
\adc{\mathtt{PintConst}(p)}{i}      &=& 0 \\
\adc{\mathtt{Pint32}}{i}            &=& 32 \\
\adc{\mathtt{Pint64}}{i}            &=& 64 \\
\adc{\mathtt{PintRanged}(p_{min},p_{max})}{i} &=& \infty \\
\end{array}
\]
Cost of encoding a single chunk relative to selected types:

\begin{tabbing}
XXXX\=XXXX\=\+\kill
$\adc{\mathtt{struct \{} T_1;\ldots T_k; \mathtt{\}}}{(d_1,\ldots,d_k)}$ \\
  \> $= \sum_{i=1}^{k} \adc{T_i}{d_i}$ \\[1.5ex] 
$\adc{\mathtt{union \{} T_1;\ldots T_k; \mathtt{\}}}{in_i(d)}$ \\            
  \> $= \mathtt{log}(k) + \adc{T_i}{d}$ \\[1.5ex]
$\adc{\mathtt{enum \{} c_1;\ldots c_k; \mathtt{\}}}{in_i(c)}$ \\            
  \> $= \mathtt{log}(k)$ \vspace{3pt} \\[1.5ex]
$\adc{
\mathtt{switch}\; x \; 
  \mathtt{of \{} 
    c_1 \mathtt{=>} T_1; \ldots  
    c_k \mathtt{=>} T_k; 
  \mathtt{\}}
}{in_i(d)}$ \\ 
  \> $= \adc{T_i}{d}$ 
\end{tabbing}
\caption {Cost of transmitting data relative to a type, selected rules}
\label{fig:cost-data}
\end{figure}