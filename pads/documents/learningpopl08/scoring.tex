The goal of the scoring function is to assess the quality of any
inferred description.  This quality metric guides decisions concerning
which rewriting rules to use to refine and transform a candidate
description, as explained in the following section.

Intuitively, a good description is one that is both {\em compact} and
yet {\em precise}.  There are trivial descriptions of any data source
that are highly compact ({\em e.g.,} the description that says
the data source is a string terminated by end of file) or
perfectly precise ({\em e.g.,} the data itself abstracts nothing and
therefore serves as its own description).  A good scoring function
carefully balances these two opposing constraints.  As is common
in machine learning, we have defined a scoring function based on the
{\em Minimum Description Length} (MDL) principal, which states that
a good description is one that minimizes the cost in bits of transmitting
the data in question by minimizing the sum of the cost of transmitting
the syntax of the description ($\costdescription{T}$)
plus the cost of transmitting the data 
relative to the information known given the description
($\costdata{T}{d_1,\ldots,d_k}$).  Mathematically,
if $T$ is a description and $d_1,\ldots,d_k$ are representations of
the $k$ chunks in our training set, parsed according to $T$, then the 
total cost in bits is:
\[
\totalcost{T}{d_1,\ldots,d_k} = \costdescription{T} + \costdata{T}{d_1,\ldots,d_k}
\]

%\paragraph*{The cost of transmitting a type.}
The cost of transmitting a type, measured in bits, is defined in 
Figure~\ref{fig:cost-type}.  In general, the cost of transmitting
of type is the cost of transmitting the sort of type ({\em i.e.,}\cd{struct},
\cd{union}, \cd{enum}, etc.) plus the cost of transmitting all subcomponents
of the type.  For example, the cost of transmitting any base type
$b(p_1,\ldots,p_k)$ is $ \cardt + \sum_{i=1}^{k} \costparam{p_i}$,
where $\cardt$ is the log of the number of different sorts of type 
constructors (24 of them in the \ir{} presented in this paper)
and $\costparam{p}$ is the cost of encoding the parameter $p$.
The cost of encoding variables, constants and parameters is not
shown in the figure, but is perfectly natural.  For instance, the cost of
encoding an ASCII character is 8 bits.  The cost of encoding a string
is the cost of encoding its length (assumed to be a 32-bit quantity)
plus the cost of encoding each character in turn.

The cost of encoding data relative to selected types is presented in 
Figure~\ref{fig:cost-data}.  At the top of the figure,
we present the cost of encoding all data
chunks relative to the type $T$; it is simply the sum of encoding
each individual chunk relative to $T$.  

In the middle of the figure,
we present the cost of encoding a chunk relative to one of the integer base 
types; other base types are handled similarly.  Notice that the cost of 
encoding an integer relative to the constant type \cd{PintConst} is
zero.  The reason is that the type itself contains all information
necessary to reconstruct the integer -- no data at all need be encoded.
The cost of encoding data relative to \cd{Pint32} or \cd{Pint64} types 
is simply 32 or 64 bits.  Finally, we artificially set the cost of
ranged types \cd{PintRanged}$(p_{min},p_{max})$ to be infinity.
Our experiments reveal that attempting to define integer types with
minimum and maximum values usually leads to overfitting of the data.
We nevertheless retain \cd{PintRanged} types in our \ir{} to encode
the range of values found during the value-space analysis.  During the
rewriting phase, this range information is used to rewrite \cd{PintRanged}
into other integer types.  Since the
cost of encoding \cd{PintRanged} is so high, the appropriate rewriting is 
guaranteed to be applied.

The last section of Figure~\ref{fig:cost-data} presents the cost of
encoding data relative to selected type constructors.  For example,
the cost of encoding a \cd{struct} is the sum of the costs of encoding
it's component parts.  The cost of encoding a \cd{union} is the cost
of encoding the branch number ($log(k)$ if the union has $k$ branches)
plus the cost of encoding the branch itself.  The cost of encoding
an \cd{enum} is the cost of encoding it's tag only -- given the tag,
the underlying data is determined by the type.  The cost of encoding
a \cd{switch} is the cost of encoding the branch only -- the tag need not
be encoded because it is determined.

\begin{figure}
Miscellaneous definitions:
\[
\begin{array}{@{\,}l@{\,}c@{\,}l}
\cardt &=& \mbox{log of the number of type constructors in the \ir{}}
\end{array}
\]
Cost of transmitting constants and parameters:
\[
\begin {array} {@{}l@{\,}c@{\,}l}
\costchar{a}  &=& \mbox{Cost of transmitting character } a \\
\coststring{s}  &=& \mbox{Cost of transmitting string } s \\
\costint{i}  &=& \mbox{Cost of transmitting integer } i \\
\costconst{c}  &=& \mbox{Cost of transmitting constant } c \\
\costvar{x} &=& \mbox{Cost of transmitting variable } \\
\costparam{p}  &=& \mbox{Cost of transmitting parameter } p \\
\end{array}
\]
Cost of transmitting a type:
\[
\begin{array}{@{}l@{\,}c@{\,}l}
\costdescription{b(p_1,\ldots,p_k)} &=& 
  \cardt + \sum_{i=1}^{k} \costparam{p_i} \\
\costdescription{x{:}b(p_1,\ldots,p_k)} &=& 
  \cardt + \costvar{x} + \sum_{i=1}^{k} \costparam{p_i} \\
\costdescription{\mathtt{struct \{} T_1;\ldots ;T_k; \mathtt{\}}} &=& 
  \cardt + \sum_{i=1}^{k} \costdescription{T_i} \\
\costdescription{\mathtt{union \{} T_1; \ldots ;T_k; \mathtt{\}}} &=& 
  \cardt + \sum_{i=1}^{k} \costdescription{T_i} \\
\costdescription{\mathtt{enum \{} c_1,\ldots,c_k \mathtt{\}}} &=& 
  \cardt + \sum_{i=1}^{k} \costconst{c_i} \\
\end{array}
\]

\caption {Cost of transmitting a type, selected rules}
\label{fig:cost-type}
\end{figure}

\begin{figure}
Cost of encoding all training data relative to a type:
\[
\begin{array}{@{}lcl}
\costdata{T}{d_1,\ldots,d_k} &=& \sum_{i=1}^{k} \adc{T}{d_i} \\
\end{array}
\]
Cost of encoding a single chunk relative to selected base types:
\[
\begin{array}{@{}lcl}
\adc{\mathtt{PintConst}(p)}{i}      &=& 0 \\
\adc{\mathtt{Pint32}}{i}            &=& 32 \\
\adc{\mathtt{Pint64}}{i}            &=& 64 \\
\adc{\mathtt{PintRanged}(p_{min},p_{max})}{i} &=& \infty \\
\end{array}
\]
Cost of encoding a single chunk relative to selected types:

{\em needs more space in between definitions... cannot figure out how
to add it properly without another whole new line
in between definitions...}

\begin{tabbing}
XXXX\=XXXX\=\+\kill
$\adc{\mathtt{struct \{} T_1;\ldots T_k; \mathtt{\}}}{(d_1,\ldots,d_k)}$ \\
  \> $= \sum_{i=1}^{k} \adc{T_i}{d_i}$ \\ 
$\adc{\mathtt{union \{} T_1;\ldots T_k; \mathtt{\}}}{in_i(d)}$ \\            
  \> $= \mathtt{log}(k) + \adc{T_i}{d}$ \\
$\adc{\mathtt{enum \{} c_1;\ldots c_k; \mathtt{\}}}{in_i(c)}$ \\            
  \> $= \mathtt{log}(k)$ \vspace{3pt} \\
$\adc{
\mathtt{switch}\; x \; 
  \mathtt{of \{} 
    c_1 \mathtt{=>} T_1; \ldots  
    c_k \mathtt{=>} T_k; 
  \mathtt{\}}
}{in_i(d)}$ \\ 
  \> $= \adc{T_i}{d}$ \\
\end{tabbing}


\caption {Cost of transmitting data relative to a type, selected rules}
\label{fig:cost-data}
\end{figure}