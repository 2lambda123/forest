\section{Using \pads{} to Access Ad Hoc Data}
\label{section:pads}
In this section, we give a brief overview of \pads{}, focusing on its
data description language and the portions of the libraries it generates
that are relevant to \padx{}. More information about \pads{} is
available~\cite{pldi05,padsmanual}.   


\subsection{\pads{}: The language}
\label{subsection:pads-language}
A \pads{} specification describes the physical layout and 
semantic properties of an ad hoc data source. 
The language provides a type-based model:
basic types specify atomic data such as integers, strings, dates, \etc{}, while
structured types describe compound data built from simpler pieces.
The \pads{} library provides a collection of useful base types.
Examples include
8-bit signed integers (\cd{Pint8}),
32-bit unsigned integers (\cd{Puint32}),
IP addresses (\cd{Pip}), 
dates (\cd{Pdate}), and strings (\cd{Pstring}).
By themselves, these base types do not provide sufficient information for parsing
because they do not indicate how the data is coded, \ie{}, in ASCII, EBCDIC, or binary.  
To resolve this ambiguity, \pads{} uses the \textit{ambient} coding.
By default, the ambient coding is ASCII, but programmers can customize
it as appropriate.

To describe more complex data, \pads{} provides a collection of 
structured types loosely based on \C{}'s type structure.
In particular, \pads{} has 
\kw{Pstruct}s, \kw{Punion}s, and \kw{Parray}s to describe
record-like structures, alternatives, and sequences, respectively.
\kw{Penum}s describe a fixed collection of literals, while \kw{Popt}s 
provide convenient syntax for optional data.
A type may have an associated predicate that determines
whether a parsed value is indeed a legal
value for the type.  For example, a predicate might require that one
field of a \kw{Pstruct} is bigger than another or that the elements
of a sequence are sorted.  Programmers can specify such
predicates using \pads{} expressions and functions, 
written in a \C{}-like syntax.
Finally, \pads{} \kw{Ptypedef}s allow programmers
to define new types that add further constraints to existing types.

\pads{} types can be parameterized by values.
This mechanism reduces the number of base types and  permits the
format and properties of later portions of the data to depend upon earlier portions.
For example, 
the base type \cd{Puint16_FW(:3:)} specifies an unsigned two byte integer
physically represented by exactly three characters, while the type
\cd{Pstring(:'|':)} 
describes a string terminated by a vertical bar.  Parameters can be 
used with compound types to specify the size of an array or the
appropriate branch of a union.

\cut{We will use the example in \figref{figure:dibbler} to illustrate
various features of the \pads{} language.  In \pads{} descriptions,
types are declared before they are used, so the type that describes
the totality of the data source appears at the bottom of the description.}

\kw{Pstruct}s describe fixed sequences of data with unrelated types.
In the \dibbler{} example in \figref{figure:dibbler}, the type declaration for \cd{order\_t}
illustrates a simple \kw{Pstruct}.  It starts with an
\cd{order\_header} followed by a timestamp \cd{tstamp}, separated by
the literal character \kw{'|'}.
\pads{} supports character, string, and regular expression literals,
which are interpreted with the ambient character  encoding.

\kw{Punion}s describe alternatives in the data format.  For example,
the \cd{dib\_ramp\_t} type in the \dibbler{} description indicates
that the \cd{ramp} field in a \dibbler{} record can be is either a
\cd{Puint\_64} or a string \cd{"no\_ii"} followed by a \cd{Puint\_64}.
During parsing, the branches of a \kw{Punion} are tried in order; the
first branch that parses without error is taken.  \cut{\pads{} also
supports a \textit{switched} union that uses a selection expression to
determine the branch to parse.  In this case, the expression typically
depends upon already-parsed portions of the data source.}

The \cd{order_header_t} type in the \dibbler{} data description
contains several anonymous uses of the \kw{Popt} type.  This type is
syntactic sugar for a stylized use of a \kw{Punion} with two branches:
the first with the indicated type, and the second with the ``void''
type, which always matches but never consumes any input.

\pads{} provides \kw{Parray}s to describe varying-length sequences of data all 
with the same type.  The \cd{event_seq_t} declaration in the \dibbler{} data description
uses a \kw{Parray} to characterize the sequence of events an
order goes through during processing.  This declaration indicates that the elements
in the sequence have type \cd{event_t}.  It also specifies that the elements will
be separated by vertical bars, and that the sequence will be terminated by 
an end-of-record marker (\kw{Peor}).  In general, \pads{} provides a rich collection
of array-termination conditions: reaching a maximum size, finding a terminating
literal (including end-of-record and end-of-source),
or satisfying a user-supplied predicate over the already-parsed portion of 
the \kw{Parray}.  

Finally, the \kw{Precord} and \kw{Psource} annotations deserve comment.  The first
indicates that the annotated type constitutes a record,
while the second means that the type constitutes the totality of a data source.  
The notion of a record varies depending upon the data encoding.  
ASCII data typically uses new-line characters to delimit 
records, binary sources tend to have fixed-width records, while 
COBOL sources usually store the length of each record before the actual data.
\pads{} supports each of these encodings of records and allows users to define
their own encodings.  \cut{By default, \pads{} assumes records are new-line terminated.
Before parsing, however, the user can direct \pads{} to use a different record
definition.}

\subsection{\pads{}: The generated library}
From a description, the \pads{} compiler generates a \C{} library
for parsing and manipulating the associated data source.  
From each type in a \pads{} description, the compiler generates 
\begin{itemize}
\setlength{\itemsep}{0ex plus0.2ex}
\item an in-memory representation, 
\item a parse descriptor, which describes syntactic and semantic errors detected during parsing, 
\item parsing and printing functions, 
\item a mask, which allows users to customize generated functions, and
\item a collection of utility functions.
\end{itemize}
%
\setcounter{totalnumber}{1}
\setcounter{dbltopnumber}{1}
\renewcommand{\topfraction}{0.85}
\renewcommand{\textfraction}{0.1}
\renewcommand{\floatpagefraction}{0.75}
\begin{figure*}
\begin{tiny}
\begin{code}
\kw{typedef} \kw{struct} \{
  Pbase\_m compoundLevel;   // Struct-level controls, eg., check Pwhere clause
  order\_header\_t\_m h;
  eventSeq\_t\_m events;
\} entry\_t\_m;
\mbox{}
\kw{typedef} \kw{struct} \{
  Pflags\_t pstate;         // Normal, Partial, or Panicking 
  Puint32 nerr;            // Number of detected errors.
  PerrCode\_t errCode;      // Error code of first detected error
  Ploc\_t loc;              // Location of first error
  order\_header\_t\_pd h;     // Nested header information
  eventSeq\_t\_pd events;    // Nested event sequence information
\} entry\_t\_pd;
\mbox{}
\kw{typedef} \kw{struct} \{
  order\_header\_t h;
  eventSeq\_t events;
\} entry\_t;
\end{code}
\begin{code}
/* Core parsing library */
Perror\_t entry\_t\_read (P\_t *pads,entry\_t\_m *m,entry\_t\_pd *pd,entry\_t *rep);
ssize\_t entry\_t\_write2io (P\_t *pads,Sfio\_t *io,entry\_t\_pd *pd,entry\_t *rep);
\end{code}
\caption{Selected portions of the library generated for the \texttt{entry\_t}
  declaration from \dibbler{} data description.}
\label{figure:library}
\end{tiny}
\end{figure*}
To give a feeling for the library that \pads{} generates, 
\figref{figure:library} includes selected portions of the generated 
library for the \dibbler{} \cd{entry_t} declaration.

The \C{} declarations for the in-memory representation, the mask, 
and the parse descriptor all share the structure of the \pads{}
type declaration.  The mapping to \C{} for each is straightforward: 
\kw{Pstruct}s map to \C{} structs with appropriately mapped fields, 
\kw{Punion}s map to tagged unions coded as \C{} structs with a tag field 
and an embedded 
union, \kw{Parray}s map to a \C{} struct with a length field and a 
dynamically allocated sequence, \kw{Penum}s map to \C{} enumerations, \kw{Popt}s 
to tagged unions, and \kw{Ptypedef}s to \C{} typedefs.  Masks include
auxiliary fields to control behavior at the level of a structured
type, and parse descriptors include extra fields to record the 
state of the parse, the number of detected errors, 
the error code of the first detected error, and the location of that error.

The parser takes a mask as an argument and returns an
in-memory representation and a parse descriptor.  
The mask allows the user to specify 
which constraints the parser should check and which portions of the
in-memory representation it should fill in.  This control allows the
description-writer to specify all known constraints about the data
without worrying about the run-time cost of verifying potentially
expensive constraints for time-critical applications.

Appropriate error-handling can be as important as processing
error-free data.  The parse descriptor marks which portions of the
data contain errors and characterizes the detected errors.
Depending upon the nature of the errors and the desired application,
programmers can take the appropriate action: halting the program,
discarding parts of the data, or repairing the errors.
If the mask requests
that a data item be verified and set, and if the parse descriptor
indicates no error, then the in-memory representation satisfies the
semantic constraints on the data.

Because we generate a parsing function for each type in a \pads{} description,
we support multiple-entry point parsing, which allows us to 
accommodate larger-scale data.
For a small file, programmers can define a \pads{} type that describes
the entire file and use that type's parsing function to read the whole
file with one call.  For larger-scale data, programmers can sequence
calls to parsing functions that read manageable portions of the file,
\eg{}, reading a record at a time in a loop.  The parsing code generated
for \kw{Parray}s allows users to choose between reading the entire array
at once or reading it one element at a time, again to support parsing
and processing very large data sources.

