\section{Using \pads{} to Access Ad Hoc Data [Kathleen]}
\label{section:pads}
In this section, we give a brief overview of \pads{}.  More
information may be found in~\cite{pldi05,padsmanual}.
A \pads{} specification describes the physical layout and 
semantic properties of an ad hoc data source. 
The language provides a type-based model:
basic types specify atomic data such as integers, strings, dates, \etc{}, while
structured types describe compound data built from simpler pieces.
The \pads{} library provides a collection of useful base types.
Examples include
8-bit signed integers (\cd{Pint8}),
32-bit unsigned integers (\cd{Puint32}),
IP addresses (\cd{Pip}), 
dates (\cd{Pdate}), and strings (\cd{Pstring}).
By themselves, these base types do not provide sufficient information for parsing
because they do not indicate how the data is coded, \ie{}, in ASCII, EBCDIC, or binary.  
To resolve this ambiguity, \pads{} uses the \textit{ambient} coding.
By default, the ambient coding is ASCII, but programmers can customize
it as appropriate.

To describe more complex data, \pads{} provides a collection of 
structured types loosely based on \C{}'s type structure.
In particular, \pads{} has 
\kw{Pstruct}s, \kw{Punion}s, and \kw{Parray}s to describe
record-like structures, alternatives, and sequences, respectively.
\kw{Penum}s describe a fixed collection of literals, while \kw{Popt}s 
provide convenient syntax for optional data.
Each of these
types can have an associated predicate that indicates whether a
value calculated from the physical specification is indeed a legal
value for the type.  For example, a predicate might require that two
fields of a \kw{Pstruct} are related or that the elements
of a sequence are in increasing order.  Programmers can specify such
predicates using \pads{} expressions and functions, 
written using a \C{}-like syntax.
Finally, \pads{} \kw{Ptypedef}s can be used
to define new types that add further constraints to existing types.

\pads{} types can be parameterized by values.
This mechanism serves both to reduce the number of base types and to permit the
format and properties of later portions of the data to depend upon earlier portions.
For example, 
the base type \cd{Puint16_FW(:3:)} specifies an unsigned two byte integer
physically represented by exactly three characters, while the type
\cd{Pstring(:' ':)} 
describes a string terminated by a space.  Parameters can be 
used with compound types to specify the size of an array or which
branch of a union should be taken.


We will use the example in \figref{figure:dibbler} to illustrate
various features of the \pads{} language.  In \pads{} descriptions,
types are declared before they are used, so the type that describes
the totality of the data source appears at the bottom of the description.

\kw{Pstruct}s describe fixed sequences of data with unrelated types.
In the \dibbler{} example, the type declaration for \cd{order\__t}
illustrates a simple \kw{Pstruct}.  It starts with an
\cd{order\_header} followed by a timestamp \cd{tstamp}, separated by
the literal character \kw{'|'}.
\pads{} supports character, string, and regular expression literals,
which are interpreted with the ambient character  encoding.

\subsection{outline}

Focus on expressiveness of data description language.

Generated library : rep, pd, and per-type parsing functions and other
type specific tools (but we're talking about those here). 

Error-aware data processing.  

Many of PADS features are not described here b/c they are not germane
to understand this work.  See PLDI and POPL papers. 
