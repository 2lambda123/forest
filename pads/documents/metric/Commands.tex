% Abbreviations
\newcommand{\cc}{Common Criteria}
\newcommand{\role}{r\^{o}le}
\newcommand{\roles}{r\^{o}les}
\newcommand{\Role}{R\^{o}le}
\newcommand{\Roles}{R\^{o}les}

% Conventions
\newcommand{\selection}[1]{\uline{#1}}
\newcommand{\assignment}[1]{\textit{#1}}

% Miscellaneous
\newcommand{\paralb}[1]{\paragraph{#1} \hspace{1mm} \linebreak} %% Paragraph with line break
\newenvironment{applicationNote}{\begin{verse} \itshape Application Note:}
{\upshape \end{verse}}

\newenvironment{layeringNote}{\vspace{2mm}\itshape \textbf{Layering Note:}}
{\upshape}

\newcommand{\target}[1]{\texttt{\hypertarget{#1}{#1}}}
\newcommand{\link}[1]{\texttt{\hyperlink{#1}{#1}}}

% Haskell stuff
\newcommand{\hs}[1]{\textsf{#1}}

% Math mode stuff
\newcommand{\col}[1]{\ensuremath{\{ \mathit{#1} \}}}
\newcommand{\mita}[1]{$\mathit{#1}$}
\newcommand{\qm}[1]{\hspace{4pt}\mbox{#1}}

%
%      STUFF FROM DISSERTATION
%      STUFF FROM DISSERTATION
%      STUFF FROM DISSERTATION
%

\newcommand{\dom}[1]{\ensuremath{\mathrm{dom}(#1)}}
\newcommand{\cod}[1]{\ensuremath{\mathrm{cod}(#1)}}
\newcommand{\begdef}{\begin{definition}\rm\raggedright\ }
\newcommand{\eeedef}{\vspace{1mm}\end{definition}}
\newcommand{\begter}{\begin{terminology}\rm\raggedright\ }
\newcommand{\edefi}{\hspace{1mm}\ensuremath{\clubsuit} \end{definition}}
\newcommand{\edefii}{\hspace{1mm}\ensuremath{\clubsuit}}
\newcommand{\begex}{\begin{example}\rm\raggedright\ }
\newcommand{\begrem}{\begin{remark}\rm\raggedright\ }
\newcommand{\begthm}{\begin{theorem}\raggedright\ }
\newcommand{\begcor}{\begin{corollary}\raggedright\ }
\newcommand{\beglem}{\begin{lemma}\raggedright\ }
\newcommand{\begnot}{\begin{notation}\rm\raggedright\ }
\newcommand{\begprob}{\begin{problem}\rm\raggedright\ }
\newcommand{\begprim}{\begin{primitive}\rm\raggedright\ }
\newcommand{\begax}{\begin{axiom}\rm\raggedright\ }

\newcommand{\letcat}[1]{Let $\cal #1$ be a category.}
\newcommand{\letsite}[2]{Let $(\cal #1, #2)$ be a site on $\cal #1$.}
\newcommand{\letpresheaf}[2]{Let \onearrow{\ensuremath{{\cal #1}^\mathrm{op}}}{#2}{\Set} be a presheaf on $\cal #1$.}
% \newcommand{\letcats}[2]{Let \mbox{${\cal #1}$} and \mbox{${\cal #2}$}
% be categories.}
\newcommand{\letobj}[2]{Let \mbox{$#1$} be an object of \mbox{${\cal #2}$}.}
\newcommand{\letobjs}[2]{Let \mbox{$#1$} be objects of \mbox{${\cal #2}$}.}
\newcommand{\letinit}[2]{Let \mbox{$#1$} be an initial object of \mbox{${\cal #2}$}.}

\newcommand{\letfunctor}[3]{Let \mbox{${\cal #1} \stackrel{#2}{\rightarrow} {\cal #3}$} be a functor.}
\newcommand{\func}[3]{\mbox{${#1} \stackrel{#2}{\rightarrow} {#3}$}}
\newcommand{\letfunc}[3]{Let \mbox{${#1} \stackrel{#2}{\rightarrow} {#3}$} be a function.}

\newcommand{\letnat}[3]{Let \mbox{${\cal #1} \stackrel{#2}{\rightarrow} {\cal #2}$} be a natural transformation.}

\newcommand{\obcat}[1]{\ensuremath{\mathrm{Ob}({\cal #1})}}
\newcommand{\inobcat}[2]{\ensuremath{#1 \in \obcat{#2}}}
\newcommand{\arcat}[1]{\ensuremath{\mathrm{Ar}({\cal #1})}}
\newcommand{\inarcat}[4]{\ensuremath{\onearrow{#1}{#2}{#3} \in \arcat{#4}}}
\newcommand{\homm}{\ensuremath{\mathrm{hom}}}
\newcommand{\homcat}[1]{\ensuremath{Hom({\cal #1})}}
\newcommand{\homsub}[1]{\ensuremath{Hom_{\cal #1}}}
\newcommand{\homset}[3]{\ensuremath{\homm_{\cal #1}( #2, #3 )}}
\newcommand{\homsettwo}[2]{\ensuremath{\homm( #1, #2 )}}

%% Special command for domain annotations
\newcommand{\ann}{\ensuremath{(n \stackrel{\eta}{\rightarrow} x) \rightarrow a \rightarrow a}}

\newcommand{\onearrow}[3]{\ensuremath{#1 \stackrel{#2}{\rightarrow} #3}}
\newcommand{\onearrowt}[3]{\mbox{#1 $\stackrel{#2}{\rightarrow}$ #3}}
\newcommand{\onearrowl}[3]{\mbox{$#1 \stackrel{#2}{\longrightarrow} #3$}}
\newcommand{\maparrow}[2]{\mbox{$#1 \mapsto #2$}}
\newcommand{\functor}[3]{\mbox{${\cal #1} \stackrel{#2}{\rightarrow} {\cal #3}$}}
\newcommand{\naturalt}[3]{\mbox{$#1 \stackrel{#2}{\rightarrow} #3$}}

\newcommand{\set}[1]{\mbox{$\{ #1 \}\ $}}
\newcommand{\setqual}[2]{\mbox{$\{ #1 \mid $ #2 $\}\ $}}
\newcommand{\family}[1]{\mbox{$\{ #1 \}_{i \in I}\ $}}
\newcommand{\calA}{\ensuremath{{\cal A}\ }}
\newcommand{\calB}{\ensuremath{{\cal B}\ }}
\newcommand{\calBnsp}{\ensuremath{{\cal B}}}
\newcommand{\calC}{\ensuremath{{\cal C}\ }}
\newcommand{\calCnsp}{\ensuremath{{\cal C}}}
\newcommand{\calD}{\ensuremath{{\cal D}\ }}
\newcommand{\calE}{\ensuremath{{\cal E}\ }}
\newcommand{\calF}{\ensuremath{{\cal F}\ }}
\newcommand{\calH}{\ensuremath{{\cal H}\ }}
\newcommand{\calI}{\ensuremath{{\cal I}\ }}
\newcommand{\calM}{\ensuremath{{\cal M}\ }}
\newcommand{\calMnsp}{\ensuremath{{\cal M}}}
\newcommand{\calN}{\ensuremath{{\cal N}\ }}
\newcommand{\calO}{\ensuremath{{\cal O}\ }}
\newcommand{\calOnsp}{\ensuremath{{\cal O}}}
\newcommand{\calP}{\ensuremath{{\cal P}\ }}
\newcommand{\calR}{\ensuremath{{\cal R}\ }}
\newcommand{\calS}{\ensuremath{{\cal S}\ }}
\newcommand{\calT}{\ensuremath{{\cal T}\ }}
\newcommand{\Set}{\mbox{\bf Set}\ }
\newcommand{\Setop}[1]{\ensuremath{\mathbf{Set}^{#1^\mathrm{op}}}}
\newcommand{\Cat}{\mbox{\bf Cat}\ }
\newcommand{\qed}{\ ${\cal Q.E.D.}\ $}
\newcommand{\proof}{\rm\ \linebreak {\raggedleft {\bf PROOF}\hspace{0.5in}}}

\newcommand{\homo}[1]{\mbox{${\cal HOM}_{\ \cal #1}$}}
\newcommand{\homright}[1]{\mbox{${\cal H}_{#1}$}}
\newcommand{\homleft}[1]{\mbox{${\cal H}^{#1}$}}
\newcommand{\homop}[1]{\mbox{${\cal #1}^{op}$}}
\newcommand{\calop}[1]{\mbox{${\cal #1}^{op} \times {\cal #1}$}}

\newcommand{\powercat}[2]{\mbox{${\cal #1}^{\cal #2}$}}
\newcommand{\power}[1]{\mbox{${\cal P}(a)$}}

\newcommand{\bs}{\mbox{$\calB / {S^*}$ }}
\newcommand{\tabref}{table \arabic{chapter}.\arabic{table}}

\newcommand{\limit}[1]{\mbox{$\stackrel{\lim}{\leftarrow} #1$}}

\newcommand{\fover}[2]{\mbox{$\left(\frac{\onearrow{#1}{\empty}{X}}{\textstyle{\em #2}}\right)$}}
\newcommand{\fnover}[2]{\mbox{$\left(\frac{\onearrow{#1}{\empty}{\calN}}{\textstyle{\em #2}}\right)$}}

\newcommand{\pplist}{ \setlength{\rightmargin}{0.0in}
 \setlength{\leftmargin}{1.2in}
 \setlength{\labelwidth}{0.5in}
 \setlength{\labelsep}{0.2in}
}
\newcommand{\pplii}{ \setlength{\rightmargin}{0.0in}}
\newcommand{\ppliii}{ \setlength{\rightmargin}{0.0in}}

\newcommand{\fromzero}[1]{\hat{\mathbf{#1}}}
\newcommand{\centralblock}[1]{\bar{\mathbf{#1}}}
