\newpage
\section{Bottom Line}
\label{sec:BottomLine}

\subsection{Version 1 complexity metric}

The PADS types that will be supported by the complexity metric in
version 1 are listed in table \ref{tab:v1Complexity}. The table
specifies the information that is needed for some of the PADS types.

\begin{longtable}{||l|l|l|}
\caption[Version 1 complexity metric]{Version 1 complexity metric}
\label{tab:v1Complexity}
\\\hline
\hline
PADS type & Length distribution & Value distribution \\\hline\hline

Base type &
\multicolumn{2}{l}{
The list of values of the base type is stored in the abstract
syntax tree resulting from the parse of the data. From this
list of values we can derive a length distribution and/or
a value distribution.
\vspace{0.5mm}} \\\hline

\textbf{Punion} &
\parbox[t]{5cm}{
A histogram of frequencies versus variant in the \textbf{Punion}.
\vspace{0.5mm}} &
\parbox[t]{6cm}{
Each subtype of the union will have its own distribution information
associated with it.
\vspace{0.5mm}} \\\hline

\textbf{Pstruct} &
\parbox[t]{5cm}{
No additional information at this time.
\vspace{0.5mm}} &
\parbox[t]{6cm}{
No additional information at this time.
\vspace{0.5mm}} \\\hline

\textbf{Pre} &
\parbox[t]{5cm}{
Not supported at this time
\vspace{0.5mm}} &
\parbox[t]{6cm}{
Not supported at this time
\vspace{0.5mm}} \\\hline

\textbf{Parray} &
\parbox[t]{5cm}{
A histogram of length of \textbf{Parray} versus frequency.
\vspace{0.5mm}} &
\parbox[t]{6cm}{
The element type will have its own distribution information
associated with it.
\vspace{0.5mm}}

\\\hline
\end{longtable}
