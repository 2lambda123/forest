\section{Tools}
\label{sec:tools}
Third-party developers can use type-directed programming techniques~\cite{??} to
generate tools that will work for any file system structure that has a
\forest{} description.  As a proof of concept, we have written a
number of such tools, which we describe in this section.  We simulated
being third-party users by not changing the code of the \forest{}
compiler to build any of these tools.

\subsection{\fg{}}
\fg{} generates a graphical representation of any directory structure that matches a 
\forest{} specification.  We generated the graphs in
Figures~\ref{fig:student-pic} and \ref{fig:coral-pic} using  
this tool.  In the default configuration, \fg{} uses boxes to denote
directories and ovals to denote files. Borders of varying
thickness distinguish between ASCII and binary files.  
Dashed node boundaries indicate symbolic links and red nodes flag errors.

The core functionality of \fg{} lies in a Haskell function
called \cd{mdToPDF}:
\begin{code}
mdToPDF :: ForestMD md => 
     md -> FilePath -> IO (Maybe String)
\end{code}
The function takes as input any \forest{} meta-data value (\ie{}, any
value of type \cd{md} where \cd{md} belongs to the \forest{}
meta-data class \cd{ForestMD}) and a filepath that specifies where to
put the generated PDF file.  It optionally returns a string (\cd{Maybe
String}); if the option is present, the string contains an error
message.  The \cd{IO} type constructor indicates that there can be
side effects during the execution of the function.  A use of
this function to generate the graph for the Princeton Computer Science
Department looks like:
\begin{code}
 do \{ (cs_rep,cs_md) <- CS_load  "facadm"
    ; mdToPDF cs_md "Output/CS.pdf" 
    \}
\end{code}
Note that \fg{} needs only the meta-data to generate the graph;
laziness means \forest{} will not load the representation in this
case. 

The related function \cd{mdToPDFWithParams} takes an additional
argument that allows the user to specify how to draw the nodes and
edges in the output graph





\begin{itemize}
\item pretty printer
\item Graph representation giving status
\item Permissions checker
\item Shell tools: 
\end{itemize}
