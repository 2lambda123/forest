\section{A Core Calculus for \forest{}}
\label{sec:exp}
\renewcommand{\bnfdef}{\ensuremath{\mathord{::=}}}
\newcommand{\slsh}{\ensuremath{\mathord{\textsf{/}}}}
\newcommand{\File}[1]{\ensuremath{\mathsf{File}(#1)}}
\newcommand{\Dir}[1]{\ensuremath{\mathsf{Dir}(#1)}}
\newcommand{\Link}[1]{\ensuremath{\mathsf{Link}(#1)}}
\newcommand{\Set}[1]{\ensuremath{\{ #1 \}}}
\newcommand{\Map}[1]{\ensuremath{\{\!\mid #1 \mid\!\}}}
\newcommand{\PNil}{\ensuremath{\bullet}}
\newcommand{\PCons}[2]{\ensuremath{#1\,\textsf{/}\,#2}}
\newcommand{\Just}[1]{\ensuremath{\mathsf{Just(#1)}}}
\newcommand{\Nothing}{\ensuremath{\mathsf{Nothing}}}
\newcommand{\True}{\ensuremath{\mathsf{True}}}
\newcommand{\False}{\ensuremath{\mathsf{False}}}
\newcommand{\fn}[2]{\ensuremath{#1(#2)}}
\newcommand{\Sk}{k^{\tau_m}_{\tau_r}}
\newcommand{\SAdhoc}[1]{\ensuremath{\mathsf{Adhoc}({#1}^{\tau_m}_{\tau_r})}}
\newcommand{\SPred}[1]{\ensuremath{\mathsf{Pred}(#1)}}
\newcommand{\SPath}[2]{\ensuremath{#1 \mathrel{::} #2}}
\newcommand{\SPair}[3]{\ensuremath{\langle #1 {:} #2, #3 \rangle}}
\newcommand{\SOption}[1]{\ensuremath{#1?}}
\newcommand{\SSetComp}[3]{\ensuremath{\{ #1 \mid #2 \in #3 \}}}
\newcommand{\Env}{\ensuremath{\mathcal{E}}}
\newcommand{\ENil}{\ensuremath{\bullet}}
\newcommand{\yields}{\ensuremath{\rightsquigarrow}}
\newcommand{\valid}[1]{\ensuremath{\mathit{valid}(#1)}}
\newcommand{\typ}[1]{\ensuremath{\mathit{#1}}}
\newcommand{\meta}{\typ{meta}}
\newcommand{\REP}{\ensuremath{\mathit{REP}}}
\newcommand{\PD}{\ensuremath{\mathit{PD}}}
\newcommand{\pd}[1]{\ensuremath{(#1)~\mathit{pd}}}
\newcommand{\eval}[1]{\ensuremath{\mathsf{eval}(#1)}}
\newcommand{\ptext}{\ensuremath{\mathsf{Txt}}}
\newcommand{\pbin}{\ensuremath{\mathsf{Bin}}}
\newcommand{\pany}{\ensuremath{\mathsf{Any}}}
\newcommand{\att}{\ensuremath{\mathsf{a}}}
\newcommand{\defaultatt}{\ensuremath{\att_{\mathsf{df}}}}
\newcommand{\repty}[1]{{\cal R}[\![ #1 ]\!]}
\newcommand{\mdty}[1]{{\cal M}[\![ #1 ]\!]}
\newcommand{\checkk}[3]{\fn{\mathit{check}}{#1,#2,#3}}

In order to capture the essence of \forest{}'s semantics, we have
defined an idealized core calculus.  This calculus has helped us
design various aspects of the \forest{} language itself and provides
compact means by which we may communicate the central, orthogonal
concepts to others.  It is inspired by classical ({\em i.e.,} not
separating, substructural or ambient) unordered tree logics, though
customized for our particular needs.

\paragraph*{File system model and specification syntax.}
Figure~\ref{fig:calculus-syntax} presents the formal file system model.  
In our model, file paths $r$ are
sequences of string names\footnote{For simplicity, we ignore the special path
elements ``..'' and ``.''.  Adding these elements
requires path normalization in the semantics, which, while easily
achievable, does not shed light on interesting design elements.} 
and file systems $F$
are finite partial maps from paths to pairs of file attributes $\att$
and file system contents $T$.  We leave the attribute records $\att$ 
unspecified, but they should include the usual elements: owner, group,
date modified, {\it etc.}  We write $\defaultatt$ for the dummy or
default attribute record where necessary.
A $T$ may be a file $\File{n}$ (with underlying string contents $n$),
a directory $\Dir{ \Set{n_1,\dots,n_k} }$ (with contents named 
$n_1, \ldots, n_k$) or a symbolic link $\Link{r}$ (where $r$ is the
path linked to).  

A file system model $F$ is only {\em well-formed} when it is tree-shaped,
with directories forming internal nodes and files and symbolic
links the leaves.  More formally, the following conditions must hold:
\begin{itemize}
\item The domain of $F$ must be prefix-closed.
\item If $F(r) = (\att,\Dir{ \Set{n_1,\dots,n_k} })$ then for $i=1,\ldots,k$,
$\PCons{r}{n_i} \in \dom{F}$.
\item  If $F(r) = (\att,\File{n})$ or $(m,\Link{r'})$ then 
there does not exist $n$ such that $\PCons{r}{n} \in \dom{F}$.
\end{itemize}

Figure~\ref{fig:calculus-syntax} also presents the syntax of
a simple computation language $e$ and our
file system specifications $s$.  The computation language $e$
contains values $v$, variables $x$, and some number of other
operations, which we leave unspecified.  An environment
$\Env$ maps variables to values.  A semantic function
$\eval{\Env,F,r,e}$ evaluates an expression $e$ in an
environment $\Env$, file system $F$, and at current path $r$,
producing a value $v$.

The simplest file system specifications are the constants
$k$, which range over basic specifications such as those for
text files (\ptext), binary files (\pbin), or any
file system object at all (\pany). 

\padshaskell{} specifications are modelled as
\SAdhoc{b} where $b$, a parser,  is a total
function from pairs of environments and strings to
values.  

For syntactic reasons, \forest{} itself combines specifications
for records and paths and also comprehensions and paths.
However, the calculus reveals that (dependent) records, paths 
and comprehensions
may all be understood as independent, orthogonal constructs.
The record specifications are modelled as a dependent pair,
written \SPair{x}{s_1}{s_2}, where $x$ may appear in $s_2$.
A path specification is written $\SPath{e}{s}$, where $e$ is
a path name (to be appended to the current path) and 
$s$ specifies the file system fragment at that path.  As an example,
a combined \forest{} record-and-path specification such as
\cd{\{c is "c.txt" :: C, dat is "d.txt" :: D c\}}
is written in the calculus as
$\SPair{x}{(\SPath{\mathtt{"c.txt"}}{C})}{(\SPath{\mathtt{"d.txt"}}{D\; x})}$.
A specification with the form $\SSetComp{s}{x}{e}$ is a comprehension where
$e$ is a set of values, $x$ is bound to each such value in turn, and
$s$, which may contain $x$, specifies a file system fragment for each
such $x$.  Again, a combined \forest{} comprehension-and-path
\cd{[x :: s | x <- e]} is modelled as the composition of two
orthogonal constructors ($\SSetComp{s_1}{x}{e}$ and $s_1 = \SPath{x}{s}$).

A predicate specifications $\SPred{e}$ succeeds
when $e$ evaluates to \True{} and fails when $e$ evaluates
to \False.  A \forest{} constraint of the form \cd{s where e}
is modelled in the calculus as a dependent pair with a predicate:
$\SPair{x}{s}{\SPred{e[x/\mathtt{this}]}}$

Finally, a maybe specification is written $\SOption{s}$ in the calculus.
 
\begin{figure}
\[
\begin{array}{rr@{\;}r@{\;}l}
\textit{Strings}        & n & \in & \Sigma^{\ast} \\[1ex]
\textit{Paths}          & r,s & \bnfdef & \PNil \mid \PCons{r}{n} \\[1ex]
\textit{Attributes}     & \att  & \bnfdef & \dots\\[1ex]
\textit{Filesystem}     & T  & \bnfdef & \File{n} \\
\textit{Contents}       &    & \bnfalt & \Dir{ \Set{n_1,\dots,n_k} }\\
                        &    & \bnfalt & \Link{r}\\[1ex]
\textit{Filesystems}    & F & \bnfdef & \Map{ r_1 \mapsto (\att_1,T_1), \dots r_k \mapsto (\att_k,T_k) }\\[1ex]
\textit{Values}         & v & \bnfdef & \att \bnfalt n \bnfalt r \bnfalt \True \bnfalt \False \bnfalt () \bnfalt (v_1,v_2) \\
                        &   & \bnfalt & \Just{v} \bnfalt \Nothing \bnfalt \{ v_1,\dots,v_k \}\\
\textit{Expressions}    & e & \bnfdef & x \bnfalt v \bnfalt \dots \\[1ex]
\textit{Environments}   & \Env & \bnfdef & \ENil \bnfalt E,x\mapsto v \\[1ex]
\textit{Specifications} & s & \bnfdef & \Sk \\
                        &   & \bnfalt & \SAdhoc{b}\\
                        &   & \bnfalt & \SPath{e}{s}\\
                        &   & \bnfalt & \SPair{x}{s_1}{s_2}\\
                        &   & \bnfalt & \SSetComp{s}{x}{e}\\
                        &   & \bnfalt & \SPred{e}\\
                        &   & \bnfalt & \SOption{s}\\[1ex]
\end{array}
\]
\caption{File systems and their specifications}
\label{fig:calculus-syntax}
\end{figure}

\paragraph*{Calculus Semantics.}
The calculus has a semantics in three parts, which represent the three different
artifacts generated by the \forest{} compiler from each specification.  The
first semantic judgement has the form $\Env;F;r \models s \yields v,d$.
Intuitively, this judgement states that in environment $\Env$ and file system
$F$, specification $s$ matches the file system fragment at current path $r$
and produces file system representation $v$ and metadata $d$.  This judgement
may also be viewed as a total function from $\Env$, $F$, $r$ and $s$ to the
pair $v$ and $d$.  The judgement is total because when file system fragments
fail to match the given specification, defaults are generated for the representation
$v$ and errors are recorded in the metadata $d$.  This design is preferable to
failing outright as it allows a programmer
to explore a file system fragment even when it contains errors, as is common
in ad hoc data repositories.  The second and third
semantic judgements characterize the shape of the representation and metadata
given a specification by giving their types.  They have the form $\repty{s} = \tau$ and
$\mdty{s} = \tau$, respectively.  The three sets of definitions obey the following
basic coherence property, where $\turn v : \tau$ states that value $v$ has type $\tau$.

\begin{proposition}
If $\Env;F;r \models s \yields v,d$ and
$\repty{s} = \tau_r$ and $\mdty{s} = \tau_m$
then $\turn v : \tau_r$ and $\turn d : \tau_r$ 
\end{proposition}


The semantics of our calculus is spelled out in Figure~\ref{fig:calculus-semantics} 
and \ref{fig:calculus-types}.

\begin{figure}
\[
\infrule
{ }
{ \Env;F;r \models \Sk \yields \checkk{\Sk}{F}{r} }
{}
\]

\[
\infrule
{ F(r) = (\att,\File{n}) \qquad 
  b(E,n) = v,d }
{ \Env;F;r \models \SAdhoc{b} \yields v,(\valid{d},(d,\att)) }
{ }
\]

\[
\infrule
{ %\begin{array}{l}
  F(r) = (\att,\Dir{\_}) \text{ or } F(r) = (\att,\Link{\_}) \qquad
  b(E,\epsilon) = (v,d)
  %\end{array} 
}
{ \Env;F;r \models \SAdhoc{b} \yields v,(\False,(d,\att)) }
{}
\]

\[
\infrule
{ 
  r \not\in \dom{F}  \qquad
  b(E,\epsilon) = (v,d)
}
{ \Env;F;r \models \SAdhoc{b} \yields v,(\False,(d,\defaultatt)) }
{}
\]

\[
\infrule
{ \Env;F;\eval{E,F,r,\PCons{r}{e}} \models s \yields v,d }
{ \Env;F;r \models \SPath{e}{s} \yields v,d }
{ }
\]

\[
\infrule
{ \begin{array}{c}
  \Env;F;r \models s_1 \yields v_1,d_1 \\
  \Env[x \mapsto v_1, x_d \mapsto d_1];F;r \models s_2 \yields v_2,d_2 \\
  v = (v_1,v_2) \qquad d = (d_1,d_2) 
  \end{array} }
{ \Env;F;r \models \SPair{x}{s_1}{s_2} \yields v,(\valid{d_1} \wedge \valid{d_2}, d) }
{ }
\]

\[
\infrule
{ \begin{array}{c}
  \eval{\Env,F,r,e} = \{ v_1,\dots,v_k \} \\
  S = \Set{ (v,d) \mid v' \in \{ v_1,\dots,v_k \} \text{ and } \Env[x \mapsto v'];F;r \models s \yields v,d }
  \end{array} }
{ \Env;F;r \models \SSetComp{s}{x}{e} \yields \pi_1~S,(\bigwedge~\valid{\pi_2~S}, \pi_2~S) }
{ }
\]


\[
\infrule
{ }
{ \Env;F;r \models \SPred{e} \yields (),(\eval{(E,F,r,e)},()) }
{ }
\]


\[
\infrule
{ r \not\in \dom{F} }
{ \Env;F;r \models \SOption{s} \yields \Nothing,(\False,\Nothing) }
{ }
\]

\[
\infrule
{ r \in \dom{F} \qquad \Env;F;r \models s \yields v,d }
{ \Env;F;r \models \SOption{s} \yields \Just{v},(\valid{d},\Just{d}) }
{ }
\]


\caption{\forest{} calculus semantics}
\label{fig:calculus-semantics}
\end{figure}

\begin{figure}

\[
\begin{array}{lcl}
\repty{\Sk} & = & \tau_r \\
\repty{\SAdhoc{b}} & = & \tau_r \\
\repty{\SPath{e}{s}} & = & \repty{s} \\
\repty{\SPair{x}{s_1}{s_2}} & = & \repty{s_1} \times \repty{s_2} \\
\repty{\SSetComp{s}{x}{e}} & = & \repty{s}~\typ{list}  \\
\repty{\SPred{e}} & = & \typ{unit} \\
\repty{\SOption{s}} & = & \repty{s}~\typ{option}  \\
\\
\mdty{\Sk} & = & \pd{\tau_m} \\
\mdty{\SAdhoc{b}} & = & \pd{\tau_m} \\
\mdty{\SPath{e}{s}} & = & \mdty{s} \\
\mdty{\SPair{x}{s_1}{s_2}} & = & \pd{\mdty{s_1} \times \mdty{s_2}} \\
\mdty{\SSetComp{s}{x}{e}} & = & \pd{\mdty{s}~\typ{list}}  \\
\mdty{\SPred{e}} & = & \pd{\typ{unit}} \\
\mdty{\SOption{s}} & = & \pd{\mdty{s}~\typ{option}}  \\
\\
\pd{\tau} & = & \typ{header} \times \tau \\
\typ{header} & = & \typ{bool} 
\end{array}
\]

\caption{\forest{} calculus data and metadata types}
\label{fig:calculus-types}
\end{figure}

% \paragraph*{Constants}

% \reminder{I assume we'll only have space to describe the
%   $\fn{Check_k}{}$ functions informally, perhaps with one example, so
%   not TeXing them for now.}

% \reminder{The second rule below is broken: $m$ is not defined in the
%   case where $r \not\in \dom{F}$.}

% \paragraph*{Adhoc}
% \[
% \REP = b_\REP \qquad
% \PD = \pd{b_\PD \times \meta}\\
% \]

% \paragraph*{Predicate}
% \[
% \REP = \typ{unit}\qquad
% \PD = \pd{\typ{unit}}
% \]

% \paragraph*{Path}
% \[
% \REP = s_\REP\qquad
% \PD = s_\PD
% \]

% \paragraph*{Pair}
% \[
% \REP = {s_1}_\REP \times {s_2}_\REP\qquad
% \PD = \pd{{s_1}_\PD \times {s_2}_\PD}
% \]

% \paragraph*{Option}
% \[
% \REP = s_\REP~\typ{option}\qquad
% \PD = \pd{s_\PD~\typ{option}}
% \]

% \reminder{I removed ``DESC'' below. Can it just be $\pi_2~S$?}
% \paragraph*{Comprehension}
% \[
% \REP = s_\REP~\typ{list}\qquad
% \PD = \pd{s_\PD~\typ{list}}
% \]
