\section{Conclusions}
\label{sec:conclusion}
In this paper, we present the design of \forest{}, an embedded
domain-specific language for describing \filestores{}.  A \forest{}
description concisely specifies a collection of files, directories,
and symbolic links as well as expected file system attributes such as
owners and permissions.  From a description, the \forest{} compiler
generates code to lazily load the on-disk data into an isomorphic
in-memory representation, lowering the divide between on-disk and
in-memory data.  The generated metadata identifies errors in the
\filestore{}.  Haskell's generic programming infrastructure makes it
easy for third-party developers to generate tools that work for any 
\filestore{} that has a \forest{} desecription. We have used this
infrastructure already to define a number of useful tools. The
language has a formal semantics based on classical tree logic and is
fully implemented.  We anticipate releasing the source code shortly.
