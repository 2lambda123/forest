Many applications use the file system as a simple persistent data
store.  Although this approach is expedient, imposing almost no
overhead, it is not robust because in general, the overall correctness
of the application will depend on the collection of files,
directories, and symbolic links in the file system having some precise
hierarchical organization and metadata such as file ownership,
permissions, and timestamps but current programming languages do not
provide support for documenting assumptions about the file system. In
addition, actually loading the data from the disk requires writing a
lot of distracting boilerplate code.

This paper describes \forest{}, a new domain-specific language for
describing directory structures embedded in \haskell{}. \forest{}
descriptions use a type-based metaphor to specify portions of the file
system in a simple, declarative manner.  \forest{} makes it easy to
connect data on the disk to an isomorphic representation in memory
that can be manipulated by programmers as if it were any other
strongly-typed data structure in their program.  \forest{} also
generates metadata that can be used to verify that a given portion of
the file system conforms to its specification.  It greatly lowers the
divide between on-disk and in-memory representations of data.

We present our design for \forest{} and describe an implementation of
a full working prototype in \haskell{}. Leveraging the support for
generic programming in \haskell{}, we also describe a framework that
allows third-party developers to build tools for querying,
visualizing, and detecting errors in on-disk data in a generic way. We
present examples illusrtating the use of \forest{} on a number of
real-world directory structures and programming tasks, including
drop-in replacements for a number of standard shell tools. Finally, we
formalize the core elements of the language as a simple calculus based
on classical tree logics.