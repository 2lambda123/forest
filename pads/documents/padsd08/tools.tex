\subsection{Using Tool Configurations}
\label{sec:tools}

The most straight-forward use of the \padsd{} system is to create a feed based on a \padsd{}
description, and then apply a number of built-in tools such as archiver and
accumulator on the feed. \padsd{} provides a very easy-to-use interface, a tool configuration, 
to accomplish this simple task. 
Figure~\ref{fig:toolconfigs} presents an example tool config file.

\begin{figure}[tb]
\centering
\begin{codebox}
feed comon.fml/comon

tool feedaccum
\{
  minalert  = true;
  maxalert  = true;
  lesssig   = Some 3;
  moresig   = Some 3;
  useralert = (fun x -> false);
  slicesize = Some 10;
  slicefile = Some "slice.acc";
  totalfile = Some "total.acc"
\}

tool rss
\{
  title = "Comon RSS";
  link  = "http://www.cs.princeton.edu/~kzhu/rss.xml";
  desc  = "Memory Info of Monall";
  sched = None; 
  path  = ["[?]"; "Mem_info"];
  rssfile = None; 
\}
\end{codebox}
\caption{Example Tool Configuration File}
\label{fig:toolconfigs}
\end{figure}

A tool configuration includes a feed declaration header (the first line in the config),
and a sequence of tool specifications. The header specifies the path to the 
feed description file (comon.fml) and the name of the feed (described in the fml file) 
to be created (comon).
Each tool specification starts with the the keyword {\tt tool} followed by the
name of the tool (feedaccum and rss for example). Only system built-in tools may be 
included in the tool config file. The body of each tool specification is essentially
an O'Caml record. The exact syntax and meanings of these tool specs can be found in
the \padsd{} library. 

Once the tool config is edited, it can be compiled into an O'Caml user program
that calls the \padsd{} tools library properly. The program can then be built with the the
fml description and (optionally) \padsml{} descriptions into the final executable
code. At runtime, the specified feed will be created and archived, and then the
configured tools will be applied to the feed in parallel. This is possible because feeds in
\padsd{} are functional and immutable. Typically the tools process the data items
in the feed, produce an output, and return unit type. We apply the tools in parallel to
achieve maximum throughput and also because some of the tools, such as feedaccum,
operate on a per feed basis, rather than a per item basis. And thus running the tools
in pipeline is not optimal.

In summary, the tool configuration in \padsd{} is a very simple programming 
paradigm where minimum O'Caml language knowledge is required. Feed designers
and system administrators who hardly know any functional programming can still 
do a lot with the feeds. For users who do know O'Caml programming, more control
is available by programming directly against the \padsd{} tools library, which
will be discussed in the next section.

