\subsection{Quick and Dirty}
\label{sec:tools}
The \padsd{} system comes with a suite of useful tools that help users cope with standard data adminstration needs. We now briefly introduce them and discuss their uses. 

%\edcom{Move this bit somewhere?}
%A tool is typically an \ocaml{} module with a standard interface: a tool specification, a feed representation or prefeed compiled from the feed description, and a feed of interest. 

\begin{figure}[th]
\centering
\begin{scodebox}
======================================================
Summary of network transmission errors
======================================================
ErrCode: 1      ErrMsg: Misc HTTP error Count: 12
ErrCode: 5      ErrMsg: Bad message     Count: 27
ErrCode: 6      ErrMsg: No reply        Count: 2

======================================================
Top 10 locations with most network errors
======================================================
Loc: http://planetlab01.cnds.unibe.ch:3121    Count: 2
Loc: http://pepper.planetlab.cs.umd.edu:3121  Count: 2
Loc: http://planetlab3.cs.uchicago.edu:3121   Count: 2
... omitted ...
======================================================
<top>.<listitem>.Cpu_hog.elt1.Some.elt2 : string
======================================================
good vals: 554        bad vals: 0      pcnt-bad: 0.00%
min val: 4   max val: 22   avg val: 11   std dev: 5.29   

Distribution of top 10 values of 47 distinct values:
  Val                Count        Percent of good vals
"nyu_d"              111                        20.04%
"ethzcs_q"           86                         15.52%
"princeton_coblitz"  68                         12.27%
... omitted ...
"root"               12                          2.17%
. . . . . . . . . . . . . . . . . . . . . . . . . . . 
SUMMING              434                        78.34%
\end{scodebox}
\caption{Portion of an accumulator output} \label{fig:acc}
\end{figure} 

%===========================================================================
%Summary of network transmission errors
%===========================================================================
%ErrCode: 1      ErrMsg: Misc HTTP error Count: 12
%ErrCode: 5      ErrMsg: Bad message     Count: 27
%ErrCode: 6      ErrMsg: No reply        Count: 2

%===========================================================================
%Top 10 locations with most network errors
%===========================================================================
%Loc: http://planetlab01.cnds.unibe.ch:3121    Count: 2
%Loc: http://pepper.planetlab.cs.umd.edu:3121  Count: 2
%Loc: http://planetlab3.cs.uchicago.edu:3121   Count: 2
%... omitted ...
%===========================================================================
%<top>.<listitem>.Cpu_hog.elt1.Some.elt2 : string
%===========================================================================
%good vals:      554     bad vals:       0       pcnt-bad:       0.00%
%min val: 4      max val: 22     avg val: 11     std dev: 5.29   

%=> Distribution of the top 10 values out of 47 distinct values:
%    Val: "nyu_d"                   Count: 111     pcnt-of-good-vals: 20.04%
%    Val: "ethzcs_q"                Count: 86      pcnt-of-good-vals: 15.52%
%    Val: "princeton_coblitz"       Count: 68      pcnt-of-good-vals: 12.27%
%    ... omitted ...
%    Val: "root"                    Count: 12      pcnt-of-good-vals: 2.17%
%. . . . . . . . . . . . . . . . . . . . . . . . . . . . . . . . . . . . . .
%    SUMMING                        Count: 434     pcnt-of-good-vals: 78.34%
    
\begin{figure}
{\small
\begin{verbatim}
path :: =  
  "top"
| path.ID  (field/variant name)
| path.INT (branch number (from 1) of a tuple)
| path.[?] (any one elements of array/table) 
| path.[*] (all elements of array/table) 
| path.[INT] (nth element of array (from 0))
| path.[Key] (a table entry indexed by the Key)
\end{verbatim}
}
\caption{Selector path language.}
\label{fig:selector-lang}
\end{figure}

\begin{itemize}
\item {\em Archiver.} The archiver saves the files fetched by a feed into the local file system, organizing the archived files
according to the structure of the feed, with one directory per base feed. 
%One directory is created per base feed, in which the archiver stores all the files fetched by that feed. 
It places a catalog in each directory documenting the source of 
the data in the directory, its scheduled arrival time and
actually arrival time. The user may specify that the archiver compress
the files in each directory after their quantity passes a certain threshold. 

\item {\em Printer.} The printer prints the content of a feed to a file or
multiple files. When it prints to a single file, all data from a feed is
concatenated with an optional separator. When it is configured to print
to multiple files, it outputs the content of each base feed into a separate
file.

\item {\em Profiler.} The profiler monitors the performance of the feeds, reporting the throughputs \edcom{What throughputs?  yhm},  and the average network and system latencies
of the feed over a period of time. Users can specify when to profile the feed and
for how long, using the tool's specification. This tool has been used to produce
some of the experimental results in section \ref{sec:experiments}.

\item {\em Accumulator.} The accumulator builds and maintains a statistical profile of the feed, including its error characteristics. For numerical data, the accumulator maintains aggregated information such as 
average values, max/min values and standard deviations. For other
types of data, like strings, URL's and IP addresses, it maintains the frequency of the top $N$
most common values. For all data, it maintains the error rates and most common errors values and their source.

The user can configure the accumulator to profile the entire feed at once, or in sections. The latter are
useful if the feed is infinite, because they allow the user to continuously monitor the data and compare with historical statistics. 
The accumulator can output either plain text or XML with an
accompanying DTD file. Figure \ref{fig:acc} shows portions of an accumulator
output for the CoMon example, in text format.

\item {\em Alerter.} The alerter allows the user to specify conditions on the data for which, if violated, the user should be notified. The library has several predefined conditions such as value exceeding a max/min or \edcom{missing verb here} $n$ standard deviations away. 
Users can define custom conditions by specifying an arbitrary \ocaml{} predicate. 
This tool has been integrated with the accumulator above. The alerter notifies users of violated conditions by
appending to a file, which can then be piped to other tools such as an email
client.

\item {\em Database loader.} The DB loader allows the user to load numerical
data from a feed into a round robin database (RRD tool). The user specifies
a custom function to transform feed items into numerical values together with other parameters required by the RRD tools, such as datasource
type and sampling rate. The data is indexed in the RRD by arrival time and
older data is discarded to make space for new data.

\item {\em Selector.} The selector allows the user to specify subcomponents of feed elements using a path expression and returns a feed of the selected subcomponents. The path expression language, similar to \xpath{}~\cite{xpath}, is shown in \figref{fig:selector-lang}.

\item {\em RSS feed generator.} The RSS feed generator formats a
\padsd{} feed as an RSS feed. The user specifies
the title, link (source), description, and update schedule of the RSS feed.
The content of the RSS feed is specified by a path expression. The tool uses the selector tool to
project data from the feed based on the path expression and wraps the result in XML for inclusion in the RSS feed.
\end{itemize}

The most straightforward use of \padsd{} is to create a feed 
based on a \padsd{} description, and then apply one or more built-in tools 
to the feed. \padsd{} provides a {\it tool configurations} ({\it configs} for short) to accomplish this task. 
Figure~\ref{fig:toolconfigs} presents an example config.

\begin{figure}[tb]
\centering
\begin{codebox}
\kw{feed} comon.fml/comon

\kw{tool} feedaccum
\{
  minalert  = true;
  maxalert  = true;
  lesssig   = 3;
  moresig   = 3;
  slicesize = 10;
  slicefile = "slice.acc";
  totalfile = "total.acc"
\}

\kw{tool} rss
\{
  title = "CoMon RSS";
  link  = "http://www.cs.princeton.edu/~kzhu/rss.xml";
  desc  = "Memory Info of Monall";
  path  = "top.[?].Mem_info";
\}
\end{codebox}
\caption{Example tool configuration file}
\label{fig:toolconfigs}
\end{figure}
%  path  = ["[?]"; "Mem_info"];

A tool configuration includes a feed declaration header 
(the first line in the config),
and a sequence of tool specifications. The header specifies the path to the 
feed description file (comon.fml) and the name of the feed to be created (comon).
Each tool specification starts with the the keyword \cd{tool} followed by the
name of the tool (\cd{feedaccum} and \cd{rss} for example). The body of each tool specification is a list of name-value pairs, where values are \ocaml{} expressions. Some attributes are optional, with the system providing default values as needed. 

\padsd{} compiles a config into an \ocaml{} program that will create and archive the specified feed, configure the specified tools, and apply those tools to the feed in parallel. Multiple tools can simultaneously consume the same feed possible because feeds in
\padsd{} are functional and immutable. 

%We apply the tools in parallel to
%achieve maximum throughput and also because some of the tools, such as feedaccum,
%operate on a per feed basis, rather than a per item basis. And thus running the tools
%in pipeline is not optimal.

%In summary, the tool configuration in \padsd{} is a very simple programming 
%paradigm where minimum \ocaml{} language knowledge is required. Feed designers
%and system administrators who hardly know any functional programming can still 
%perform day-to-day data management tasks. For users who do know \ocaml{} programming, 
%more control is available by programming directly against the \padsd{} 
%tools library, which will be discussed in the next section.

