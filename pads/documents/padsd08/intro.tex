

An {\em ad hoc data source} is any semistructured data source for
which useful data analysis and transformation tools are not readily
available.  The data that constitutes a single, abstract source 
often comes from many different concrete, physical destinations
distributed across the Internet.  It may also become available
over a range of times and in several, evolving formats.
Before the users can extract the information they need from the data,
it usually must be fetched, archived locally for historical analysis, 
compressed, perhaps encrypted or anonymized, and monitored for errors 
or deviations from the norm.

Managing ad hoc data is a particular bane of the implementers of
distributed systems.  Depending on the size, these systems may have
hundreds or thousands of heterogenous, distributed components.
Keeping all these components up and running is an enourmous continuous
maintenance task.  Consequently, each component in a well-designed
system will produce endless log files that measure its performance and
heath.  As an example, consider the data manipulated by
CoMon~\cite{comon}, a system designed to monitor the health,
performance and security of PlanetLab~\cite{planetlab}.  Every five
minutes, CoMon attempts to contact each of 842 PlanetLab nodes across
416 sites worldwide.\footnote{Data as of writing this manuscript;
PlanetLab membership varies over time.}  When all is running smoothly,
which it never is, each node responds by sending back an ASCII data
file in mail-header format containing information ranging from the
kernel version to the uptime to the memory usage to the id of the user
with the greatest CPU utilization.  CoMon archives this data in
compressed form and its backend processes the information for display
to PlanetLab users.  It is an invaluable resource for Planet users who
need to monitor the health and performance of their applications or
experiments.

Almost all distributed systems have (or should have) similar sorts of
monitoring infrastructure.  Unfortunately, system implementers are
often left to hack ``one-off'' monitoring tools of their own, which
are invariably less reliable, unoptimized, insecure, and difficult or
impossible to evolve when new requirements become known.  A
substantial part of the difficulty simply comes from the diversity,
quality, and volume of data these systems must often handle. Often,
new monitoring systems also face the problem of having to interact
with legacy devices, legacy software and legacy data, leaving
implementers in a situation where they cannot use robust off-the-shelf
data management tools built for standard formats like XML.  XML-based
tools also have the disadvantage of significant bloat (often 8-10 times
the size of a more natural, even uncompressed representation) caused by
using a generic reprensentation.

Somewhat similar problems also appear across the natural and social
sciences, including biology, physics and economics.  For example,
systems such as BioPixie~\cite{biopixie}, Grifn~\cite{grifn} and
Golem~\cite{golem}, built by computational biologists at Princeton,
routinely obtain data supplied from a number of sources scattered
across the net.  The data is often archived and later analyzed or
mined for valuable information about gene structure and regulation.
Likewise, cosmologists need access to data uploaded from major
telescopes~\cite{sdss} and economists can make use of vast data
repositories at FedStats.org amongst other sources.  These and other
selected ad hoc data sources are presented in
Figure~\ref{fig:exampledata}.

\begin{figure*}
\begin{center}
\begin{tabular}{|l|l|l|}
\hline\hline
Name & Use & Properties 
\\\hline\hline
CoMon~\cite{comon} & PlanetLab host monitoring & Multiple data sets in mail-header formats\\
                                       && Archiving every 5 minutes \\
                                       && From evolving set of ~800 nodes \\\hline
CoBlitz~\cite{coblitz} & File transfer system monitoring & Multiple data sets \\
                                       && Archiving every 5 minutes \\
                                       && From evolving set of ~800 nodes \\\hline
CoralCDN~\cite{coral} & Log files from CDN monitoring & Single Format \\
                                       && Periodic archiving \\
                                       && From evolving set of ~250 hosts \\\hline
\vizGems{}       & Website Host Monitoring & Many and varied machines \\
                 &                         & Execute programs remotely\\
                 &                         & to collect data\\\hline
\darkstar{}      & AT\&T network monitoring & Archiving for future analysis \\\hline
\ningaui{}       & AT\&T billing auditing   & Thousands of data sources\\
                 &                          & Archiving and error analysis\\\hline
GO DB (Gene Ontology)~\cite{geneontology} & Gene Function Information & Multiple Formats \\
                                             && Uploaded in daily, weekly, monthly intervals \\\hline
BioGrid~\cite{biogrid} & Curated Gene and Protein Data & XML and Tab-separated Formats \\
          & & multiple data sets $<=$ 50MB each \\
          & & monthly data releases \\\hline
NCBI~\cite{ncbi} & National Center for Biotechnology Information & Links to multiple bioinformatics datasets \\
                                                     && and online databases\\
\hline\hline
\end{tabular}
\end{center}
\caption{Example ad hoc data sources}
\label{fig:exampledata}
\end{figure*}

The purpose of our research is to develop a system that makes it
easier to create, maintain, and evolve tools for processing ad hoc data
generated from a wide array of data sources over periods of time.    
We propose to do so by developing a
domain-specific language, called \padsd, in which software developers
specify a number of key aspects of the data sources they wish to
monitor including any of the following.

\begin{itemize}
\item {\bf where} the data is located.  The data may be in some directory
on the current machine (perhaps placed there by another process) or at some 
remote location or collection of locations.
\item {\bf when} to get the data.  The data may need to be fetched just 
once (right now!) or according to some repeated schedule in time series 
indexed by minutes, days or months.
\item {\bf how} to obtain it.  The data may be accessible through standard 
protocols such as http or ftp or it may be created through remote execution 
of a non-standard script. 
\item {\bf what preprocessing} the system should do when it arrives.  The 
data may be compressed or encrypted and therefore need to be decompressed 
or decrypted before it can be processed.  Privacy considerations may require 
the data be anonymized in some way.
\item {\bf what format} the data source arrives in.  The data may be ASCII 
or binary; it may be tab- or comma-separated.  The data may also be 
represented in some completely ad hoc, non standard format consisting of 
floating point numbers, integers, strings, vertical bars and curly-q's.  
\end{itemize}

These rich, high-level specifications are then compiled into a
collection of programming libraries and end-to-end tools for
distributed systems monitoring.  Our current tool suite includes a
number of useful artifacts, inspired by the common needs we have
observed in ad hoc monitoring systems:

\begin{itemize}
\item {\bf an archiver} that collects distributed data on the specified 
schedule, archives it locally, and maintains a ``table of contents.''
\item {\bf a printing tool} and {\bf performance monitor tool} that helps
debug specifications and easily measure what is going on in the system
\item {\bf a database loader} that takes the data and extracts specified 
pieces to load into the RRD database tool~\cite{rrdtool}.  The data is 
indexed by its arrival time and supports time-based queries.  For 
performance, as more recent data arrives, older data is discarded.
\item {\bf an accumulator tool} that maintains a statistical profile of the 
data and its error characteristics.  For numeric data, information about 
average values and standard deviations are maintained.  For other kinds 
of data, such as strings, urls, ip addresses, times, dates, and ad hoc 
enumerations, information counts of the top $N$ most commonly occurring 
items are maintained.  For all data, error rates and information about 
common errors are maintained.
\item {\bf an alert system} that generates alerts based on programmable 
conditions.
\item {\bf a selector tool} that extracts and records specified 
subcomponents of a larger data source.
\item {\bf an RSS feed generator} that wraps data in the appropriate 
XML headers and creates a single RSS feed from possibly multiple diverse 
ad hoc data sources.
\end{itemize}

These generated end-to-end tools can
be used off-the-shelf by editing (or not) a short configuration file.
Conveniently, any user in a rush need never do any real ``programming,''
if they do not choose to.  

In addition to these standard tools, the system provides support for
creating new tools by automatically generating a collection of
libraries.  The libraries include a run-time system for fetching data,
libraries for parsing data in a specified format and for printing data
in that format.  There is also infrastructure for type-safe data
traversal and stream processing using classic functional programming
paradigms such as map, fold and iterate.  The generated libraries make
it straightforward for programmers to create their own custom tool
specific to a single data source or collection of sources.  

There is also advanced support for creating new, {\em generic}
programs, where a generic program is one that operates correctly over
{\em any} well-specified data source.  For example, the RRDtool loader
is generic, because it is possible to load data from any specified
source into the RRDtool without doing any substantial additional
``programming.''  Likewise, the alert system, selector, RSS feed
generator, parsers, printers and traversal libraries are all generic
programs.

A key result of our overall language and system design is that it
readily support three sorts of users, or perhaps more accurately, {\em
three modes of use}.  The first mode of use might be of use to the
{\em quick-and-dirty hacker}.  In quick-and-dirty mode, a user need
only specify the relevant properties of their data, edit the simple
tool configuration file, and watch the results pour in.  There is no
need to learn any interfaces or stitch together any boilerplate code
-- a real advantage of our domain-specific language design.  On the
other hand, the quick-and-dirty user is limited to the tools that
others have designed and while these tools are useful and relatively
flexible, they are by no means a silver bullet.  The second mode is
for the {\em single-minded implementer }, who needs to build a new
application on top of a {\em specific} collection of distributed data
sources.  This user has found they need more functionality than what
is provided directly through our generic tools so he or she writes a
specification for their data sources and uses the
specification-specific generated interfaces to implement their
application.  Since the application is specific to a particular
collection of sources, it cannot be directly reused by others.  The
third mode is for the {\em generic programmer}.  The generic
programmer observes that they (or their colleagues or fellow domain
experts) need to perform some task over and over again on different
data sets.  Examples such as xml-conversion, database loading, web
page generation, domain-specific search or querying abound.  Rather
than writing a program specific to a particular data set, they use a
separate set of interfaces we supply to write a single generic program
to complete the task.

\paragraph*{Contributions.}  To summarize, this paper makes a number of 
important contributions:

\begin{itemize}
\item It outlines the design of a language for specifying the 
spatial, temporal and auxiliary properties of distributed ad hoc data
sources.  We are aware of no other research effort that has attempted
to design, implement or analyze such a language.

\item It provides a mechanism for automatic generation of 6 different
data processing tools directly from high-level specifications with no
programming required.

\item It supports three modes of use and different sorts of interfaces
for three different kinds of users, 
the quick-and-dirty hacker, the single-minded implementer and the generic 
programmer, thereby optimizing both flexibility and ease-of-use.

\item It provides a formal denotational semantics for the domain-specific
language.

\item It reports on the implementation experience and performance.
\end{itemize}

\paragraph{Outline.}
In the remainder of the paper, we will explain the design and
implementation of our system in further detail.  First, 
in Section~\ref{sec:examples}, we will
outline two running examples we will use for expository purposes
throughout the paper, one involving CoMon~\cite{comon}, a system built at Princeton
to monitor PlanetLab~\cite{planetlab}, 
and a second involving \ningaui{}, built at
AT\&T for monitoring AT\&T's web hosting service.  After introducing
the examples, we will explain how to specify the attributes of the
data sources they depend upon in Section~\ref{sec:informal}. 
The follow section explains the three modes of use of our system
once a specification has been written.  In Section~\ref{sec:language},
we describe the formal semantics of the language and now that helped guide
us in building our implementation.  Section~\ref{sec:implementation}
evaluates the performance of our system.
Section~\ref{sec:related} we explain the relationship with other
research in this area and, finally, in Section~\ref{sec:conclusions} we
will touch upon future work and conclude.

