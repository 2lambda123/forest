

% An {\em ad hoc data source} is any semistructured data source for
% which useful data analysis and transformation tools are not readily
% available.  The data that constitutes a single, abstract source 
% often comes from many different concrete, physical destinations
% distributed across the Internet.  It often becomes available
% over a range of times and in several, evolving formats.
% Before users can extract the information they need from the data,
% it must be fetched, archived locally for historical analysis, 
% compressed, perhaps encrypted or anonymized, and monitored for errors 
% or deviations from the norm.

% Managing ad hoc data is a bane of the implementers of distributed
% systems.  These systems may have hundreds or thousands of
% heterogeneous, distributed components.  Keeping these components
% running smoothly is 

% Many modern applications depend upon ad hoc collections of distributed
% data sources.  For instance, o
One of the primary tasks in developing
a distributed system is keeping it  running smoothly over
long periods of time.  
% Doing so is a continuous maintenance task of significant
% complexity.  
Consequently, well-designed distributed systems include
a subsystem responsible for monitoring the health, security and
performance of its constituent parts.
CoMon~\cite{comon}, designed to monitor PlanetLab~\cite{planetlab},
is an illustrative example.  CoMon
%
%\footnote{Current data; PlanetLab membership varies over time.} %
% 
operates by
attempting to gather a log file from each of 800+ PlanetLab nodes every
five minutes.
When all is well (which it never is) each node responds with
an ASCII data file in mail-header format containing 
the node's kernel version, its uptime, its memory usage, the
ID of the user with the greatest CPU utilization, \etc{}  CoMon archives
this data in compressed form and processes the information
for display to PlanetLab users.  CoMon also tracks various
networking problems, maintains lists 
of ``problem nodes'' and  supports on-going time-indexed 
queries on the data.  These features make CoMon an invaluable resource
for users who need to monitor the health and performance of their
PlanetLab applications or experiments.

Almost all distributed systems should have similar
monitoring infrastructure.  However, the implementors of each new
distributed system currently have to build ``one-off'' monitoring tools,
which takes an enormous amount of time and expertise to do well.  A
substantial part of the difficulty comes from the diversity, quality,
and quantity of data these systems must handle.  In addition,
implementors cannot ignore errors: they must properly handle network 
errors, partial disconnects and corrupted data.  They also cannot ignore 
performance issues:
data must be fetched before it vanishes from remote sites and it must
be archived efficiently in ways that do not burn out hard drives by
causing them to overheat.  Last but not least, new monitoring systems 
must interact with legacy devices, legacy software and legacy data,
often preventing implementers from using robust off-the-shelf data management
tools built for standard formats like XML and RSS. 
% I don't think this sentence really fits here...
% XML-based tools also have
%the disadvantage of significant bloat (often 8-10 times the size of a
%more natural, even uncompressed representation) caused by using a
%generic representation.

Systems researchers are not alone in their struggles with distributed
collections of ad hoc data sources.
Similar problems appear in the natural and social sciences,
including biology, physics and economics.  For example, systems such
as BioPixie~\cite{biopixie}, Grifn~\cite{grifn} and
Golem~\cite{golem}, built by computational biologists at Princeton,
routinely obtain data from a number of sources scattered
across the net.  Often, the data is archived and later analyzed or
mined for information about gene structure and regulation.
%Likewise, cosmologists access data from
%telescopes~\cite{sdss}. 
%Economists make use of the vast data
%repositories at \url{FedStats.org} amongst other sources.  
Figure~\ref{fig:exampledata} summarizes 
selected distributed ad hoc data sources used in these and other
applications.

% \vizGems: Many and varied machines \\

\begin{figure}
\begin{center}
\begin{tabular}{|l|l|}
\hline
{\bf Name/{\em Use}} & {\bf Properties} 
\\\hline\hline
CoMon~\cite{comon} & Multiple data sets \\
{\em ~~ PlanetLab host}              & Archiving every 5 minutes \\
{\em ~~ monitoring}                              & From evolving set of 800+ nodes \\\hline
%CoBlitz~\cite{coblitz}   & Multiple data sets \\
%{\em ~~ File transfer}        & Archiving every 3 minutes \\
%{\em ~~ system monitoring}                                      & From evolving set of 800+ nodes \\\hline
CoralCDN~\cite{coral}         & Single Format \\
{\em ~~ Log files from}  & Periodic archiving \\
{\em ~~ CDN monitoring}                  & From evolving set of 250+ hosts \\\hline
AT\&T \vizGems{}                    & Execute programs remotely to \\
{\em ~~ Website host}                 & collect data \\
{\em ~~  monitoring}                   & Varied fetch frequencies \\\hline
AT\&T \darkstar{}                   & Diverse data sources\\
{\em ~~ Network monitoring}       & Archiving for future analysis \\
				& Per minute, hour, and day fetches\\\hline
AT\&T \ningaui{}                    & Thousands of data sources\\
{\em ~~ Billing auditing}        & Archiving and error analysis\\\hline
GO DB~\cite{geneontology} & Multiple Formats \\
{\em ~~ Gene function info.}     & Uploads daily, weekly, monthly \\\hline
BioGrid~\cite{biogrid}        & XML and Tab-separated Formats \\
{\em ~~ Curated gene and}  & multiple data sets $<=$ 50MB each \\
{\em ~~ protein data}                              & Monthly data releases \\\hline
%NCBI~\cite{ncbi} & Links to multiple bioinformatics  \\
%{\em ~~ Biotechnology info.} & datasets\\
%\hline
\end{tabular}
\end{center}
\caption{Example distributed ad hoc data sources.}
\label{fig:exampledata}
\end{figure}

%\paragraph*{\padsd{}: A specification language for distributed data
%sources.}
We have developed a new domain-specific language and system called
\padsd{} to facilitate the creation, maintenance and evolution of tools
for processing ad hoc data from distributed sources.  The language
allows developers to describe the provenance, syntax and
semantics of data sources they wish to monitor, including:

\begin{itemize}
\item {\bf Where} the data is located.  The data may be in a file
on the current machine (perhaps written by another process), at some 
remote location, or at a collection of locations.
\item {\bf When} to get the data.  The data may need to be fetched just 
once (right now!) or according to some repeating schedule.
\item {\bf How} to obtain it.  The data may be accessible through standard 
protocols such as \cd{http} or \cd{ftp} or it may be created via a
local or remote computation. 
\item {\bf What preprocessing} the system should do when the data arrives.  
The data may be compressed or encrypted.  Privacy considerations may require 
the data be anonymized.
\item {\bf What format} the data source arrives in.  The data may be
  in ASCII, binary, or EBCDIC. It may be tab- or comma-separated or it
  may be in XML.  It may be in the kind of non-standard format that
  \pads{}~\cite{fisher+:pads,mandelbaum+:pads-ml} was designed to 
  describe or for which the user has a well-typed parser.
\end{itemize}

The \padsd{} system compiles these high-level specifications into
a collection of programming libraries and end-to-end tools for
distributed systems monitoring.  Our current tool suite includes a
number of useful artifacts, inspired by the needs we have observed in
a variety of ad hoc monitoring systems including an archiver, provenance
tracking system, database loader and a number of others.

% \begin{itemize}
% \item {\bf An archiver} that collects distributed data on the specified 
% schedule, archives it, and maintains a ``table of contents.''
% \item {\bf A provenance tracking system} that maintains a profile of the 
% data, its error characteristics and its provenance, including its
% originating location, its time of arrival and the other sources upon which it depends.  
% \item {\bf A RRD database loader} that extracts specified 
% pieces of the data and loads them into an Round-Robin
% Database~\cite{rrdtool}.  The database indexes the data by its
% arrival time and supports time-based querying.  
% \cut{
% This information should be in section 4
% For performance, as more recent data arrives, older data is discarded.}
% \cut{This information should be in section 4
% For numeric data, the system
% tracks average values and standard deviations.  For other 
% data, such as strings, urls, ip addresses, times, dates, and ad hoc 
% enumerations, the system maintains information counts of the top $N$ most commonly occurring 
% items.  For all data, the system tracks error rates and information about 
% common errors.}
% \item {\bf An alert system} that generates alarms based on programmable 
% conditions.
% \item {\bf An RSS feed generator} that wraps raw data in the appropriate 
% headers to create an RSS feed.
% \item {\bf A selector} that extracts specified 
% subcomponents of a data source.
% \item {\bf A printer} that fetches, prints and helps debug
% specifications.
% \item {\bf A performance monitor} that measures fetch times and also helps
% debug specifications.
% \end{itemize}

The \padsd{} system can generate all of these tools from declarative descriptions
and tool configuration specifications.  Thus for common
tasks, users can manage distributed data sources simply by writing
high-level declarative specifications.  There
are relatively few concepts to learn, no complex interfaces and no
tricky boilerplate to master to initialize 
the system or thread together tool libraries.  Because there is
so little ``programming'' involved, we refer to the act 
of writing simple specifications
and using pre-defined tools as the {\em off-the-shelf} 
mode of use.  

To provide extensibility, \padsd{} supports two other
modes of use. 
The second mode is
for the {\em single-minded implementer}, who needs to build a new
application for a {\em specific} collection of distributed data
sources.  Such users need more than the built-in set of tools. To meet
this need, the system provides support for
creating new tools by generating libraries for fetching
data, for parsing and printing, for performing type-safe data
traversal, and for stream processing using classic functional
programming paradigms such as \cd{map}, \cd{fold} and \cd{iterate}.
These generated libraries make it straightforward to create custom tools
specific to particular data sources.  The cost of this flexibility is
a steeper learning curve because the programmer must learn a variety
of interfaces. Functional programmers may find
these interfaces intuitive, but computational scientists
may prefer to stick with off-the-shelf uses. 

The third mode is for the {\em generic programmer}.  Generic
programmers may observe that they (or their colleagues) 
need to perform some task over and over again on different
data sets.  Rather than writing a program specific to a particular
data set, they use a 
separate set of interfaces supplied by the \padsd{} system to write a
single generic program to complete the task.  For example, the Round Robin
Database (RRD) loader is generic because it is possible to load data from any
specified source into the RRD tool \cite{rrdtool} 
without additional ``programming.''
The generic programming mode is the most difficult to use as it involves
learning a relatively complex set of interfaces for encoding
Generalized Algebraic Datatypes (GADTs)~\cite{xi:popl03} 
and Higher-Order Abstract Syntax (HOAS). 
These complexities are required to encode 
the dependent features of \padsd{} and to compensate for the lack of
built-in generic programming support in \ocaml{}. 
Still, the reward for building generic tools is high:
as more and more such tools are built, the life of the off-the-shelf
user becomes easier and easier.  We used this infrastructure to
build the off-the-shelf tools described earlier.

% The system can generate all of these tools from \padsd{} descriptions
% and declarative tool configuration specifications.  Thus for common
% tasks, users can manage distributed data sources simply by writing
% declarative specifications.

% In addition to these standard tools, the system provides support for
% creating new tools by automatically generating libraries for fetching
% data, for parsing and printing, for performing type-safe data
% traversal, and for stream processing using classic functional
% programming paradigms such as \cd{map}, \cd{fold} and \cd{iterate}.
% The generated libraries make it straightforward to create custom tools
% specific to particular data sources.

% The system also provides support for creating new, {\em generic}
% programs, where a generic program is one that operates correctly over
% {\em any} well-specified data source.  For example, the RRD database
% loader is generic, because it is possible to load data from any
% specified source into the RRD tool without additional ``programming.''
% We used the generic support to build the various tools described earlier.

% The \padsd{} system supports three sorts of users, or perhaps more
% accurately, {\em three modes of use}.  We might term the first mode of
% use {\em quick-and-dirty}.  In this mode, users need only specify
% the relevant properties of their data, edit the simple tool
% configuration files, and watch the results pour in.  Such users do not
% have to learn programming interfaces or stitch together boilerplate
% code, which makes it very easy to get started.  On
% the other hand, the quick-and-dirty user is limited to the tools that
% others have built.   The second mode is
% for the {\em single-minded implementer}, who needs to build a new
% application for a {\em specific} collection of distributed data
% sources.  Such users need more than the built-in set of tools, 
% so they write specifications of their data sources and use the
% description-specific libraries generated by the system to implement
% their application. Since the application is specific to a particular
% collection of sources, it cannot be directly reused by others.  The
% third mode is for the {\em generic programmer}.  Generic
% programmers may observe that they (or their colleagues or fellow domain
% experts) need to perform some task over and over again on different
% data sets.  Rather than writing a program specific to a particular
% data set, they use a 
% separate set of interfaces supplied by the \padsd{} system to write a
% single generic program to complete the task.

To guide the design and implementation of \padsd{}, we have developed
an idealized, first-order calculus and associated type system to model
its core elements.  We have equipped this calculus with a denotational
semantics that specifies for each data source description the set of
(meta-data, data) pairs that it should produce. 
The semantics allows users to calculate and reason about the data that
they should be receiving.
We have
proven the type system sound with respect to the semantics.
Moreover, we have used the
semantics to prove {\em dependency correctness}, a key theorem
inspired by earlier work on provenance in databases by 
Cheney {\em et al.}~\cite{cheney-dbpl07}.  This theorem
guarantees the correct provenance meta-data is associated with
every data item.  

In addition to being of theoretical interest, the calculus and its meta-theory
have served as a guide for our implementation infrastructure.  In particular,
the compilation strategy for our surface-level language was influenced by
observations about how higher-level constructs could be compiled into
combinators from our calculus.   We also reorganized the way 
earlier versions of our system processed and propagated provenance meta-data
in order to obey the principle of dependency correctness. 

\paragraph*{Contributions.} The paper makes the following
contributions:

\begin{itemize}
\item It describes the design of a
domain-specific language for specifying provenance, syntax and
semantic properties of distributed ad hoc data
sources.  

\item It provides a formal denotational semantics for our language
and proves the key properties of Type Soundness
and Dependency Correctness.

\item It describes the architecture of the system and how it
enables multiple modes of use.

\item It demonstrates the practicality of our architecture and its
  implementation by showing the infrastructure will scale to handle
  systems the size of PlanetLab.
\end{itemize}

\paragraph{Outline.}
In the rest of the paper, we describe the examples we will
use throughout the paper (\secref{sec:examples}), show how to describe
these data sources in \padsd{} (\secref{sec:informal}), describe
the generated tool infrastructure and its modes of use
(\secref{sec:programming}), define a denotational semantics and prove
our key correctness properties (\secref{sec:semantics}),  
discuss the implementation and
evaluate its performance (\secref{sec:implementation}), describe
related work (\secref{sec:related}) and conclude
(\secref{sec:conclusions}). 

