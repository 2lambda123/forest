\section{Tools}
\label{sec:tools}
Third-party developers can use generic programming~\cite{Lammel+:syb} to
generate tools that will work for any file system structure that has a
\forest{} description.  As a proof of concept, we have written a
number of such tools, which we describe in this section.  
%We simulated
%being third-party users by not changing the code of the \forest{}
%compiler to build any of these tools.  The generic programming
%infrastructure provided by Haskell makes writing such tools very easy.

\subsection{Generic Querying }
One simple application of generic programming is querying 
meta-data to find files with a particular collection of attributes. 
The \cd{findFiles} function 
\begin{code}
findFiles :: (ForestMD md) => 
     md -> (FileInfo -> Bool) -> [FilePath]
\end{code}
takes as input any \forest{} meta-data value (\ie{}, any
value of type \cd{md} where \cd{md} belongs to the \forest{}
meta-data class \cd{ForestMD}) and a predicate on \cd{FileInfo}
structures, and returns the list of all \cd{FilePath}s anywhere in the
input meta-data whose associated \cd{FileInfo} satisfies the
predicate.  For example, if \cd{cs_md} is the meta-data associated
with the Princeton Computer Science Department data, then the code 
\begin{code}
dirs  = findFiles cs_md (\textbackslash(r::FileInfo) -> 
                           (kind r) == DirectoryK)
other = findFiles cs_md (\textbackslash(r::FileInfo) -> 
                           (owner r) /= "bwk")
\end{code}
binds \cd{dirs} to the list of all directories in the data set and
\cd{other} to all the directories and files not owned by 
user \cd{"bwk"}.

To implement the \cd{findFiles} function, we use the generic Haskell
function \cd{listify}: 
\begin{code}
findFiles md pred = map fullpath (listify pred md)
\end{code}
The return type of the polymorphic \cd{listify} function is
instantiated to match the argument type of its predicate argument. 
We map the \cd{fullpath} function over the resulting list of
\cd{FileInfo} structures to return only the \cd{FilePaths}.

\subsection{File System Visualization}
\fg{} generates a graphical representation of any directory structure
that matches a \forest{} specification.  We generated the graphs in
Figures~\ref{fig:student-pic} and \ref{fig:coral-pic} using  
this tool.  In the default configuration, \fg{} uses boxes to denote
directories and ovals to denote files. Borders of varying
thickness distinguish between ASCII and binary files.  
Dashed node boundaries indicate symbolic links and red nodes flag errors.

The core functionality of \fg{} lies in the Haskell function \cd{mdToPDF}:
\begin{code}
mdToPDF :: ForestMD md => 
     md -> FilePath -> IO (Maybe String)
\end{code}
The function takes as input any meta-data value and a
filepath that specifies where to 
put the generated PDF file.  It optionally returns a string (\cd{Maybe
String}); if the option is present, the string contains an error
message.  The \cd{IO} type constructor indicates that there can be
side effects during the execution of the function.  A use of
this function to generate the graph for the Princeton Computer Science
Department looks like:
\begin{code}
 do \{ (cs_rep,cs_md) <- CS_load  "facadm"
    ; mdToPDF cs_md "Output/CS.pdf"       \}
\end{code}
Note that this code needs only the meta-data to generate the graph;
laziness means \forest{} will not load the representation in this
case. 

The related function \cd{mdToPDFWithParams} takes an additional
argument that allows the user to specify how to draw the nodes and
edges in the output graph.  Among other things, this parameter
specifies how to map a value of type \cd{Forest\_md} into
\graphviz{}~\cite{haskell-graphviz,Gansner+:graphviz} attributes.  By appropriately setting the
parameter, a user can customize the formatting of each node according
to its owner, group, or permissions, \etc{}, as well as specify global
properties of the graph such as its orientation and size.  \fg{} uses
the Haskell binding of the \graphviz{} library to layout and render
the graphs, so all customization provided by \graphviz{} are
available.

The \cd{listify} function is at the heart of the implementation of
this tool; we use it to convert the input meta-data to the list of
\cd{FileInfo}s in the meta-data.  We then convert this list into a
graph data structure suitable for use with the \graphviz{} library.

\subsection{Permission Checker}
The permission tool is designed to check the permissions on the files
and directories in a \forest{} description on a multi-user machine.
In particular, it enables one user to determine which files a second
user can read, write, or execute.  If the second user cannot access a
file in a particular way, the tool also reports the names of the files
and directories whose permissions have to change to allow the access.
The tool is useful when trying to share files with a colleague.  It
helps the first user ensure that all the necessary permissions
have been set properly to allow the second user access.  The key to
the implementation of this tool is again applying the \cd{listify}
function to the meta-data for the \forest{} description.

\subsection{Shell Tools}
We have implemented analogs of many shell tools that work
over a file system fragment defined by a
\forest{} description:
\begin{code}
ls    :: (ForestMD md) => md -> String -> IO String
grep  :: (ForestMD md) => md -> String -> IO String
tar   :: (ForestMD md) => md -> FilePath -> IO ()
cp    :: (ForestMD md) => md -> FilePath -> IO ()
\end{code}
%rm    :: (ForestMD md) => md -> String -> IO String
%rmdir :: (ForestMD md) => md -> String -> IO String
All of these functions work by extracting the relevant file names from
the argument meta-data structure using \cd{listify} and then calling
out to a shell tool to do the work.  For \cd{ls}, the second argument
gives the command-line arguments to pass to the shell version of
\cd{ls}, and the result is the resulting output. The implementation
uses \cd{xarg} to lift the restriction on the number of files that can
be passed to \cd{ls}. For \cd{grep}, the second argument is the 
search string and result is the output of the shell version of \cd{grep}. For
\cd{tar}, the second argument specifies the location for the resulting
tarball.  The implementation uses a file manifest to allow \cd{tar} to
work regardless of the number of files involved.  The \cd{cp} tool
uses the \cd{tar} tool to move the files mentioned in the meta-data to
the location specified by the second argument \textit{while retaining
the same directory structure}.   
%The module that implements these tools is~80 lines of Haskell code.

\subsection{Description Inference}
This tool allows the user to generate a \forest{} description from the
contents of the file system.   The function
\begin{code}
getDesc :: FilePath -> IO String
\end{code}
takes as an argument the path to the root of the directory structure
to infer.  It returns a string containing the generated
representation.  \figref{fig:generated-description} shows the result
of invoking \cd{getDesc} on the \texttt{classof11} directory.  
The description is not perfect: the label names are generated 
from the file name, but human editing is desirable, \etc, but it can
be easier for a programmer to edit a generated description rather than
starting from scratch.  A more sophisticated version of the tool 
takes width and depth limits. When the number of files in a directory
exceeds the width parameter, files with the same type are collected into
comprehensions.  When the depth of the description exceeds the depth
parameter, directories are given the universal directory type.

The \cd{getDesc} function works by using the universal description to
load the contents of the file system starting from the supplied path.
It then walks over the resulting meta-data to generate a \forest{}
parse tree, which it then pretty prints.


\begin{figure}
\begin{code}
\kw{data} transfer = \kw{Directory} \{
\}

\kw{data} wITHDREW = \kw{Directory} \{
    fingertxt \kw{is} "finger.txt" :: File Ptext
\}

\kw{data} tRANSFER = \kw{Directory} \{
    bEAUCHEMINtxt \kw{is} "BEAUCHEMIN.txt" :: File Ptext,
    vERSTEEGtxt \kw{is} "VERSTEEG.txt" :: File Ptext
\}

\kw{data} bSE11 = \kw{Directory} \{
    transfer \kw{is} "transfer" :: transfer,
    bOZAKtxt \kw{is} "BOZAK.txt" :: File Ptext,
    kESSELtxt \kw{is} "KESSEL.txt" :: File Ptext,
    ssstxt \kw{is} "sss.txt" :: File Ptext
\}

\kw{data} aB11 = \kw{Directory} \{
    kADRItxt \kw{is} "KADRI.txt" :: File Ptext,
    mACARTHERtxt \kw{is} "MACARTHER.txt" :: File Ptext,
    oRRtxt \kw{is} "ORR.txt" :: File Ptext,
    sSSStxt \kw{is} "SSSS.txt" :: File Ptext
\}

\kw{data} classof11 = Directory \{
    aB11 \kw{is} "AB11" :: aB11,
    bSE11 \kw{is} "BSE11" :: bSE11,
    tRANSFER \kw{is} "TRANSFER" :: tRANSFER,
    wITHDREW \kw{is} "WITHDREW" :: wITHDREW
\}
\end{code}
\caption{Generated description. Type \texttt{File Ptext} is synonymous
  with \texttt{Text}.}
\label{fig:generated-description}
\end{figure}



\subsection{The Single-Minded Implementer}

In addition to the built-in tools, \padsd{} includes an API for
manipulating feeds created from a description. The API provides the
user with a feed abstraction representing a potentially infinite
series of elements. This abstraction is related to that of a lazy
list, but extends it with support for data timing and provenance
information. Therefore, the API provided for feeds is modeled on the
list APIs of common functional languages, like \ocaml and \haskell,
but provides two levels of abstraction. One level allows the user to
manipulate feeds like any other lazy list of data elements (ignoring
where they come from), while the other exposes the metadata along with
the data. 

% Need a new name for Feedmain module. I vote Feed and then Feed_core for the lower-level module.

\begin{figure}[tb]
\begin{codebox}
\kw{let} deadline = Time.now() +. 600. \kw{in}
\kw{let} (sample, remainder) = \textit{Feed.split_when} 
   (fun () -> Time.now() > deadline) comon_feed \kw{in}
\kw{let} select_load = \kw{function}
    Some \{Comon_format.Source.
          loads = (_, load::_)\} -> Some load
  | None -> None \kw{in}
\kw{let} loads = \textit{Feed.map} select_load sample \kw{in}
\kw{let} empty_tbl = create () \kw{in}
\kw{let} load_table = \textit{Feed.fold} update empty_tbl loads 
\kw{in} print_top_10 load_table
\end{codebox}
  \caption{Code fragment for sampling \planetlab loads for 10 minutes. Function
  \texttt{print\_top\_10} selects the $10$ lowest loads from the table.}
\label{fig:sample-loads}
%\vskip -2ex
\end{figure}

\begin{figure}[tb]

\begin{codebox}
\kw{let} update tbl idata =
  \kw{let} meta = IData.get_meta idata \kw{in}
  \kw{let} data = IData.get_contents idata \kw{in}
  \kw{match} meta, data \kw{with} 
    (h, Some basemeta), Some load ->
      \kw{let} location = Meta.get_link basemeta \kw{in}
      update tbl location data
  | _ -> tbl \textit{ (* no change to tbl *)}
\kw{in} ...
\kw{let} load_table = \textit{\textbf{Feed.fold_p}} update empty_tbl loads
\kw{in} print_top_10 load_table
\end{codebox}
  \caption{Revised code fragment which exploits provenance metadata. }
\label{fig:sample-loads-prov}
%\vskip -2ex
\end{figure}

For example, consider a \planetlab user looking for a desirable set of
nodes on which to run their experiments. Based on the \comon
description, they can use our API to monitor \planetlab for a few
minutes to find the least loaded nodes. In \figref{fig:sample-loads},
we show an \ocaml code fragment that collects a list of nodes with
lowest average loads over $10$ minutes, and then prints them. We leave
out the exact details on maintaining the table of top values, as it is
orthogonal to our discussion. First, we use \cd{Feed.split_when} to
split the feed when 600 seconds (10 minutes) have elapsed. Then, we
use \cd{Feed.map} to project the loads data from the\comon
elements. Finally, we use \cd{Feed.fold} to collect the load values
into a table after which we can average the data on a per node basis
and extract the $10$ lowest values.

However, this solution is not quite enough. The \comon data format
does not include the host name in the data itself, so the code in
\figref{fig:sample-loads} will only be able to return the lowest
average loads, but not the names of the machines associated with
them. In situations like this, the provenance data is essential.  In
\figref{fig:sample-loads-prov}, we replace the last three lines of
\figref{fig:sample-loads} with a call to the lower-level fold,
\cd{fold_p}, which provides both the data and metadata to the folding
function. We also sketch an \cd{update} function which makes use of 
the metadata.   Notice the use of the lower level \cd{IData} and
\cd{Meta} interfaces to facilitate mangement of both data and 
metadata from the feed. 

It should be clear from these examples that the single-minded implementer
has a number of new interfaces to master relative to the quick-and-dirty
hacker, but gains a correspondingly higher degree of flexibilty and can
still write relatively concise programs.

\subsection{The Generic Programmer}

% Motivate

Occasionally, a user might want to develop a function that can
manipulate {\it any} feed. This desire might arise because the user
has a number of different feeds to process in the same way, or simply
because the user wishes to provide a new tool for other \padsd{}
users. Some such functions can be written simply by designing them to
be parametric in the type of the feed element, much like the feed
library functions discussed above. However, the behavior of many feeds
functions will depends on the structure of the feed and its
elements. Such functions can be viewed as {\it interpretations} of
feed descriptions, so, to support their development, we provide a
framework for writing feed interpreters.

% Examples

Two core examples of feed interpretations are the feed creator and the
feed accumulator. The behavior of both of these tools depend
essentially on the structure of the feed. These functions, and others
like them, require as input a runtime representation of the feed,
complete with all of the details included in the feed description that
they represent. The obvious choice for representing feed descriptions
in \ocaml is a datatype. However, standard \ocaml datatypes are not
sufficiently typeful to express the types of many generic feed
functions. For example, the feed creation function has the type:
\begin{code}
feed_create : 'a prefeed -> 'a feed
\end{code} where \cd{'a prefeed} is an AST of a feed description and feed 
elements have type \cd{'a}.
%
% Tools as interpreters
%
This limitation of datatypes has been widely discussed in the
literature, and various solutions have been 
proposed~\cite{yang:icfp98,weirich:encodingtypecase,hinz:icfp04,padsml-padl}. We have 
chosen to represent our AST using a variant of the Mogensen-Scott
encoding in which higher-order abstract syntax is exploited 
to encode variable binding in feed descriptions.  This implementation strategy 
exploits \ocaml's module system to type the encodings in $F_\omega$. 
A similar strategy is exploited in our earlier work on \padsml~\cite{padsml-padl}, 
but there we only sought to encode the \ocaml type of data, not entire \padsml
description, which is where the higher-order abstract syntax enters the picture.
% To effectively encode the dependency present in feeds
% descriptions, we instead 
% encodings~\cite{mogensen}, employing HOAS to encode variable binding
% in feed descriptions.
% Given that \ocaml is our target language, we follow
% the method used in \padsml{} to encode type
% representations~\cite{padsml-padl}. Because of space limitations, we
% do not provide details here.  

The result of our work is that developers
can interpret feed-description representations by case analysis on
their structure, while still achieving the desired static
guarantees. Moreover, we have successfully used this framework to
develop {\it all} of the tools presented in this paper, including the
feed creator. The only role of the compiler is to infer appropriate
type declarations from feed descriptions, and compile the feed syntax
into our representations.  However, as one might expect, interfaces using
higher-order abstract syntax and Mogensen-Scott encodings are one step more
complex than those involving the more familiar maps and folds.  Consequently, the
learning curve for the generic programmer is one step steeper than
the curve for the single-minded implementor, and two (or perhaps ten) steps steeper
than the curve for the quick-and-dirty hacker.

% There, we represent our AST using a
% Scott-encoded datatype, and leverage \ocaml's module system to, in
% essence, type the Scott encodings in $F_\omega$. However, in \padsml
% we only sought to encode the \ocaml type of data, not entire \padsml
% descriptions. To effectively encode the dependency present in feeds
% descriptions, we instead use a variant of Mogensen-Scott
% encodings~\cite{mogensen}, employing HOAS to encode variable binding
% in feed descriptions.




