Development of a formal semantics for \padsd{} has been an integral
part of our language design process.  It has been a medium through
which we have communicated our ideas to one another precisely and it has 
allowed us to experiment with the nuanced consequences of our decisions.
Its purpose in this paper is to
provide system users with a way to calculate the expected contents of
any feed they would specify, including subtle details of the effects
of timeouts and errors.

The calculus of feeds we will define depends upon a {\em
host language}, in which one may compute locations, schedules,
and constraints.  For the purposes of the
formalization, the host language is the simply-typed lambda calculus.
Figure~\ref{fig:host-language} presents the syntax we will use to
write down host language expressions.  To simplify the semantics,
the host language contains a collection of special constants 
including strings ($\astring$), locations ($\loc$), times ($\atime$),
and schedules ($\schedule$).  We assume times may be added and compared
and let $\infty$ represent a time later than all others.
We treat schedules as sets of times and
often use the notation $\atime \in \schedule$ to refer to a time $\atime$
drawn from the set $\schedule$.  We use a similar notation to refer to elements of a list.
The host language also includes structured types
such as options, pairs, sums, lists and functions, all of which are 
completely standard.  We often omit the typing annoation from the lambda
expression when it can be reconstructed from the context easily.

\begin{figure}[t]
\[
\begin{array}{lrll}
\multicolumn{4}{l}{\mbox{(host-language base types)}}\\ 
\basety & ::= & \multicolumn{2}{l}{\boolty \bnfalt \stringty \bnfalt \locty \bnfalt \timety \bnfalt \schedulety} \\
\\
\multicolumn{4}{l}{\mbox{(host-language types)}}\\ 
\ty & ::= & \basety & \mbox{base types} \\
 & \bnfalt & \optionty{\ty} & \mbox{option types}\\
 & \bnfalt & \ty_1 * \ty_2 & \mbox{pair types} \\
 & \bnfalt & \ty_1 + \ty_2 & \mbox{union types} \\
 & \bnfalt & \listty{\ty} & \mbox{list types}\\
% & \bnfalt & \feedty{\ty} & \mbox{stream types}\\
 & \bnfalt & \ty_1 \arrow \ty_2 & \mbox{function types} \\
\\
\multicolumn{4}{l}{\mbox{(host-language values)}}\\ 
\data & ::=     & \constant & \mbox{generic constant} \\
 & \bnfalt & \boolf \bnfalt \boolt & \mbox{booleans} \\
 & \bnfalt & \astring & \mbox{strings} \\
 & \bnfalt & \loc & \mbox{locations} \\
 & \bnfalt & \atime & \mbox{times} \\
 & \bnfalt & \schedule & \mbox{schedules} \\
 & \bnfalt & \none \bnfalt 
                           \some{\data} & \mbox{optional values}\\
 & \bnfalt & (\data_1,\data_2) & \mbox{pairs} \\
 & \bnfalt & \inl{\data} \bnfalt 
                           \inr{\data} & \mbox{union values} \\
 & \bnfalt & \nillist \bnfalt 
                           \conslist{\data_1}{\data_2} & \mbox{list values} \\
% & \bnfalt & \nilstream \bnfalt 
%                           \consstream{\data_1}{\data_2} & \mbox{stream values} \\

& \bnfalt & \lambda x{:}\ty.\expression & \mbox{function values} \\
\\
\multicolumn{4}{l}{\mbox{(host-language expressions)}}\\ 
\expression & ::= & \generalvar & \mbox{variables} \\
 & \bnfalt & \data & \mbox{data values} \\
 & \bnfalt & \none \bnfalt 
              \some{\expression} & \mbox{option expressions}\\
 & \bnfalt & (\expression_1,\expression_2) \bnfalt e.1 \bnfalt e.2 
    & \mbox{pair expressions} \\
% & \bnfalt & \inl{\expression} \bnfalt 
%             \inr{\expression} & \mbox{union expressions} \\
% & \bnfalt & \expression_1 \; \expression_2 & \mbox{application expression} \\
 & \bnfalt & ... & \mbox{more typed lambda expressions} \\
%\\
%\multicolumn{4}{l}{\mbox{(feed meta-data:  a subset of host language values)}}\\ 
%\multicolumn{4}{l}{\mbox{(a special location (\generatedloc) is used when data is created artificially)}}\\ 
\end{array}
\]
\caption{Host Language Syntax.}
\label{fig:host-language}
\end{figure}


\subsection{Feed Syntax and Typing}

The abstract syntax for our core calculus of feeds appears in
Figure~\ref{fig:syntax} and its typing rules appear in 
Figure~\ref{fig:typing}.  The feed typing judgment has the form
$\Gamma \turn \feed : \feedty{\sigma}$, where $\Gamma$ is a context that maps
variables to host language types $\tau$.  Notice that $\sigma$ is a subset of the
host language types.  Feed typing depends upon a standard judgment for
simple typing of lambda calculus expressions:  $\Gamma \turn e : \ty$.

The elements of the feed language include
the empty feed ($\emptyset$), which contains no elements and serves as
the unit for the union feed ($\feed_1 \union \feed_2$).  The
computed feed ($\computed{e_1}{x}{e_2}$) allows programmers to
generate a feed with schedule $e_2$ and elements $(\lambda x.e_1)\; \atime$,
where $\atime$ is drawn from the schedule. Likewise, the feed 
comprehension ($\comprehensionfeed{e}{x}{\feed}$) allows creation of a 
feed with elements $(\lambda x.e)\; v$ when $v$ is an element of $\feed$.
The filter feed
$\filterfeed{\feed}{e}$ eliminates elements $v$ from $\feed$ when
$e\; v$ is $\boolf$.  Let feeds $\letfeed{x}{e_1}{\feed_2}$
provide a convenient mechanism for binding intermediate computations.

The base forms $\mathtt{all \{ \cdots \}}$ and $\mathtt{any \{ \cdots \}}$
are similar to what we have seen before, except they have five components:
file format ($\mathtt{format}$), source locations ($\mathtt{locs}$), 
schedule ($\mathtt{sched}$), preprocessing function ($\mathtt{pp}$) 
and window ($\mathtt{win}$).
File format, schedule and preprocessing function are self-explanatory.
The location field may contain pseudo-locations that 
model the $\mathtt{proc}$ form found in the implementation.
Instead of expressing time out conditions in the schedule,
the calculus requires such conditions be specified in the window 
field.\footnote{A separate window field simplifies the semantics slightly,
but combining window and schedule together is more appealing to
programmers.  These minor syntactic differences are semantically unimportant.} 
Notice that the preprocessor
and parser both map values with option type to option type.
The {\tt None} option is used to represent networking or data format
errors.  For the sake of simplicity, 
we do not model the variety of different error codes that
the implementation supports. 

The calculus contains a number of binary operators for combining
feeds.  The union takes feeds with the same type and merges them
into a single feed.  On the other hand, the sum feed takes feeds
with (possibly) different types, injects the elements of each feed into
a sum before merging.  

The calculus also contains three different
sorts of pairs, each providing a slightly different way of
combining elements from the subcomponent feeds.  The first pair, 
written $(\feed_1, \feed_2)$ is a {\em synchronous pair} -- 
elements of $\feed_1$ are paired with elements of $\feed_2$
that are {\em scheduled} at exactly the same time (regardless of when those elements
actually {\em arrive}).  Synchronous pairs are most useful when the underlying
subcomponent feeds share the same schedule, as in the Arrakis example
from Section~\ref{sec:informal}.  However, synchronous pairs lack a
dependent variant --- it is not sensible to schedule acquisition of an element
that depends upon another element scheduled at the same time.
There are two forms of dependent pair, {\em continuous} and {\em local}.
In the the continuous variant, each element $x$ of $\feed_1$ is paired 
with all elements of the feed $\feed_2$ that depends on $x$.  In the
local variant, each element $x$ of $\feed_1$
is paired with all elements of the feed $\feed_2$ {\em until the next
element $y$ of $\feed_1$ is scheduled}.  The formal semantics, presented
in the next subsection, makes this idea precise.  The local feed
enables a paradigm in which a programmer defines an infinite feed $\feed_2$
that gets truncated and regenerated whenever a new element in $\feed_1$
is scheduled.

The final elements of the calculus include the $\mathtt{foreach}$ feeds
and the list comprehension feed.
Intuitively, a $\mathtt{foreach}$ is identical to a dependent pair in
which the first element of the pair is omitted from the data representation.
Consequently, there are two forms of $\mathtt{foreach}$ -- one for
each sort of dependent pair.  The notation ${*}$ or ${**}$ serves as
a syntactic mnemonic to remind readers of the connection.  The list comprehension
feed generates a feed of lists where each element of the list is scheduled at the same time.

\begin{figure}[t]
\[
\begin{array}{lll}
\multicolumn{3}{l}{\mbox{(feed types)}}\\ 
\multicolumn{3}{l}{\sigma \ ::= \tau \bnfalt \optionty{\tau} 
  \bnfalt \sigma_1 * \sigma_2
  \bnfalt \sigma_1 + \sigma_2
  \bnfalt \listty{\sigma}
}   \\  
\\
\multicolumn{3}{l}{\mbox{(feed specs)}}\\ 
\multicolumn{3}{l}{\feed \ ::=}   \\  
% & x &  \mbox{feed variable} \\ %% no feed variables now
% & \bnfalt 
& \emptyfeed & \mbox{empty feed} \\
 \bnfalt & \computed{e_1}{x}{e_2} & \mbox{computed feed} \\
 \bnfalt & \comprehensionfeed{e}{x}{\feed} & \mbox{feed comprehension} \\
 \bnfalt & \filterfeed{\feed}{e} & \mbox{filter feed} \\
 \bnfalt & \letfeed{x}{e_1}{\feed_2} & \mbox{let feed} \\
 \bnfalt &     \mathtt{all \{ format=} e_1; & \mbox{all locations}\\ 
& \qquad     \mathtt{locs=} e_2; & \\
& \qquad     \mathtt{sched=} e_3; & \\
& \qquad     \mathtt{pp=} e_4; & \\
& \qquad     \mathtt{win=} e_5; \}  & \\
%\allfeed{e_1}{e_2}{e_3}{e_4}{e_5} &  \\
 \bnfalt &     \mathtt{any \{ format=} e_1; & \mbox{one of several locations}\\ 
& \qquad     \mathtt{locs=} e_2; & \\
& \qquad     \mathtt{sched=} e_3; & \\
& \qquad     \mathtt{pp=} e_4; & \\
& \qquad     \mathtt{win=} e_5; \}  & \\
% \bnfalt & \existsfeed{e_1}{e_2}{e_3}{e_4}{e_5} & \mbox{one of several location} \\
 \bnfalt & \feed_1 \unionfeed \feed_2 & \mbox{union feed} \\
 \bnfalt & \feed_1 \sumfeed \feed_2 & \mbox{sum feed} \\
 \bnfalt & (\feed_1, \feed_2) & \mbox{synchronous pair} \\
% & \bnfalt & \feed_1 cartesian \feed_2 & \mbox{cartesian pair -- use a symbol different from *} \\
% & \bnfalt & \feed_1 * \feed_2 & \mbox{continuous pair} \\
% & \bnfalt & \feed_1 {*}{*} \feed_2 & \mbox{local pair} \\
 \bnfalt & x{:}\feed_1 * \feed_2 & \mbox{dependent continuous pair} \\
 \bnfalt & x{:}\feed_1\, {*}{*} \, \feed_2 & \mbox{dependent local pair} \\
 \bnfalt &     \mathtt{foreach{*}}\; x \; 
    \mathtt{in}\; \feed_1 & \mbox{for each $x$ create continuous $\feed_2$} \\
 &   \quad \mathtt{create}\; \feed_2 \\
 \bnfalt &     \mathtt{foreach{*}{*}}\; x \; 
    \mathtt{in}\; \feed_1 & \mbox{for each $x$ create local $\feed_2$}\\
 &   \quad \mathtt{extend}\; \feed_2 \\
%\foreachcreate{x}{\feed_1}{\feed_2} & \mbox{for each $x$ create continuous $F_2$} \\
% \bnfalt & \foreachupdate{x}{\feed_1}{\feed_2} & \mbox{for each $x$ create local $F_2$} \\
% & \bnfalt & \ppfeed{\feed}{e} & \mbox{preprocess (eg, unzip) data} \\
% & \bnfalt & \remap{\feed}{e} & \mbox{direct feed to different locations/times} \\
% & \bnfalt & \refeed{\feed}{e} & \mbox{adapt feed to new schedule; 
%                                               fill missing entries with ``None''} \\
% & \bnfalt & \stutterfeed{\feed}{e} & \mbox{stutter on new schedule} \\
 \bnfalt & [\feed \bnfalt x \leftarrow e ] & \mbox{list comprehension feed} \\
\end{array}
\]
\caption{Feed Language Syntax.}
\label{fig:syntax}
\end{figure}


\begin{figure}[th]

% \[
% \infer[(\textit{t-var})]
% {\Gamma \turn x : \Gamma(x)}
% {}
% \]

\[
\infer[(\textit{t-empty})]
{\Gamma \turn \emptyfeed : \feedty{\sigma}}
{}
\]

\[
\infer[(\textit{t-compute})]
{\Gamma \turn \computed{e_1}{x}{e_2} : \feedty{\optionty{\ty}}}
{
  \Gamma \turn e_2 : \schedulety &
  \Gamma,x{:}\timety \turn e_1 : \optionty{\ty} 
}
\]

\[
\infer[(\textit{t-comph})]
{\Gamma \turn \comprehensionfeed{e}{x}{\feed} : \feedty{\ty}}
{
  \Gamma \turn \feed :  \feedty{\sigma} &
  \Gamma,x{:}\sigma \turn e : \ty 
}
\]

\[
\infer[(\textit{t-filter})]
{\Gamma \turn \filterfeed{\feed}{e} : \feedty{\sigma}}
{
  \Gamma \turn \feed : \feedty{\sigma} &
  \Gamma \turn e : \sigma \arrow \boolty
}
\]

\[
\infer[(\textit{t-let})]
{\Gamma \turn \letfeed{x}{e_1}{\feed_2} : \feedty{\sigma_2}}
{
  \Gamma \turn e_1 : \ty_1 & 
  \Gamma,x{:}\ty_1 \turn \feed_2 : \feedty{\sigma_2} 
}
\]

\[
\infer[(\textit{t-all})]
{ \begin{array}{l}
  \Gamma \turn 
    \mathtt{all \{ format=} e_1; 
      \mathtt{locs=} e_2;
      \mathtt{sched=} e_3; \\ \qquad\qquad
    \mathtt{pp=} e_4;
    \mathtt{win=} e_5; \} : \feedty{\optionty{\ty}}
 \end{array}
}
{
 \begin{array}{c}
  \Gamma \turn e_1 : \optionty{\stringty} \arrow \optionty{\ty} \\
  \Gamma \turn e_2 : \listty{\locty} \qquad
  \Gamma \turn e_3 : \schedulety \\
  \Gamma \turn e_4 : \optionty{\stringty} \arrow \optionty{\stringty}  \\
  \Gamma \turn e_5 : \timety
 \end{array}
}
\]

\[
\infer[(\textit{t-any})]
{ \begin{array}{l}
    \Gamma \turn 
      \mathtt{any \{ format=} e_1; 
      \mathtt{locs=} e_2;
      \mathtt{sched=} e_3; \\ \qquad\qquad
    \mathtt{pp=} e_4;
    \mathtt{win=} e_5; \}
 : \feedty{\optionty{\ty}}
  \end{array}
}
{
 \begin{array}{c}
  \Gamma \turn e_1 : \optionty{\stringty} \arrow \optionty{\ty} \\
  \Gamma \turn e_2 : \listty{\locty} \qquad
  \Gamma \turn e_3 : \schedulety \\
  \Gamma \turn e_4 : \optionty{\stringty} \arrow \optionty{\stringty}  \\
  \Gamma \turn e_5 : \timety
 \end{array}
}
\]

\[
\infer[(\textit{t-union})]
{\Gamma \turn \feed_1 \unionfeed \feed_2  : \feedty{\sigma}}
{
  \Gamma \turn \feed_1 : \feedty{\sigma} &
  \Gamma \turn \feed_2 : \feedty{\sigma}
}
\]

\[
\infer[(\textit{t-sum})]
{\Gamma \turn \feed_1 \sumfeed \feed_2  : \feedty{\sigma_1 + \sigma_2}}
{
  \Gamma \turn \feed_1 : \feedty{\sigma_1} &
  \Gamma \turn \feed_2 : \feedty{\sigma_2}
}
\]

\[
\infer[(\textit{t-synch-pair})]
{\Gamma \turn (\feed_1, \feed_2)  : \feedty{\sigma_1 * \sigma_2}}
{
  \Gamma \turn \feed_1 : \feedty{\sigma_1} &
  \Gamma \turn \feed_2 : \feedty{\sigma_2}
}
\]

% \[
% \infer[(\textit{t-local-pair})]
% {\Gamma \turn \feed_1 * \feed_2  : \feedty{\sigma_1 * \sigma_2}}
% {
%   \Gamma \turn \feed_1 : \feedty{\sigma_1} &
%   \Gamma \turn \feed_2 : \feedty{\sigma_2}
% }
% \]

\[
\infer[(\textit{t-cont-pair})]
{\Gamma \turn x{:}\feed_1 * \feed_2  : \feedty{\sigma_1 * \sigma_2}}
{
  \Gamma \turn \feed_1 : \feedty{\sigma_1} &
  \Gamma,x{:}\sigma_1 \turn \feed_2 : \feedty{\sigma_2}
}
\]

\[
\infer[(\textit{t-local-pair})]
 {\Gamma \turn x{:}\feed_1 \allpairfeed \feed_2  : \feedty{\sigma_1 * \sigma_2}}
 {
   \Gamma \turn \feed_1 : \feedty{\sigma_1} &
   \Gamma,x{:}\sigma_1 \turn \feed_2 : \feedty{\sigma_2}
 }
\]

\[
\infer[(\textit{t-foreachcont})]
{\Gamma \turn \foreachcreate{x}{\feed_1}{\feed_2}  : \feedty{\sigma_2}}
{
  \Gamma \turn \feed_1 : \feedty{\sigma_1} &
  \Gamma,x{:}\sigma_1 \turn \feed_2 : \feedty{\sigma_2}
}
\]

\[
\infer[(\textit{t-foreachlocal})]
{\Gamma \turn \foreachupdate{x}{\feed_1}{\feed_2}  : \feedty{\sigma_2}}
{
  \Gamma \turn \feed_1 : \feedty{\sigma_1} &
  \Gamma,x{:}\sigma_1 \turn \feed_2 : \feedty{\sigma_2}
}
\]


% \[
% \infer[(\textit{t-pp})]
% {\Gamma \turn \ppfeed{\feed}{e} : \feedty{\ty}}
% {
%   \Gamma \turn \feed : \feedty{\ty} &
%   \Gamma \turn e : ((\locty * \timety) * \stringty) \arrow \stringty
% }
% \]

% \[
% \infer[(\textit{t-redirect})]
% {\Gamma \turn \remapfeed{\feed}{e} : \feedty{\ty}}
% {
%   \Gamma \turn \feed : \feedty{\ty} &
%   \Gamma \turn e : \locty * \timety \arrow \locty * \timety
% }
% \]

%% \[
%% \infer[(\textit{t-reschedule})]
%% {\Gamma \turn \refeed{\feed}{e} : \feedty{\optionty{\ty}}}
%% {
%%   \Gamma \turn \feed : \feedty{\ty} &
%%   \Gamma \turn e : \schedulety
%% }
%% \]

%% \[
%% \infer[(\textit{t-stutter})]
%% {\Gamma \turn \stutterfeed{\feed}{e} : \feedty{\ty}}
%% {
%%   \Gamma \turn \feed : \feedty{\ty} &
%%   \Gamma \turn e : \schedulety
%% }
%% \]


\[
\infer[(\textit{t-listf})]
{\Gamma \turn [\feed \bnfalt x \leftarrow e ]  : \feedty{\listty{\sigma}}}
{
  \Gamma,x{:}\tau \turn \feed : \feedty{\sigma} &
  \Gamma \turn e : \listty{\tau}
}
\]

\caption{Feed Language Typing.}
\label{fig:typing}
\end{figure}

%
\begin{figure*}[t]
\[
\begin{array}{lcl}
\esemantics{e}{\environment} &=& \mbox{denotation of simply-typed term $e$ in environment $\environment$ mapping variables to values.}
\\
\\
\semantics{\feed}{\environment}{\universe} &=& \mbox{denotation of 
$\feed$ in environment $\environment$ mapping variables to values.} \\
 && \mbox{and ``universe'' $\universe$ mapping schedule time and location to 
          arrival time and string data}
\\
% \\
% \semantics{x}{\environment}{\universe} 
%  &=& (\environment(x_t), \environment(x))
% \\
\\
\semantics{\emptyfeed}{\environment}{\universe} 
 &=& \{\;\}
\\\\
\semantics{\computed{e_1}{x}{e_2}}{\environment}{\universe} 
 &=& \{(\atime, \esemantics{(\lambda x.e_1)}{\environment}\; \atime) 
          \setalt \atime \in  \esemantics{e_2}{\environment} 
     \} 
\\\\

%  \begin{array}{l}
    {\cal F}\lsem\mathtt{all \{ format=} e_1; 
%    \allfeed{e_1}{e_2}{e_3}{e_4}
%  \end{array} 
%}{\environment}{\universe} 
 &=& \{(\atime, (\loc, \esemantics{e_1}{\environment} \; (\universe'(\loc,\atime))))
          \setalt \atime \in  \esemantics{e_3}{\environment} 
          \;\mbox{and}\; \loc \in  \esemantics{e_2}{\environment}
     \} 
\\
 \qquad\quad\ \,   \mathtt{locs=} e_2;
&&\quad\mbox{where} \\
 \qquad\quad\ \,    \mathtt{sched=} e_3;
&& \qquad \mathtt{timeout} =  
     \lambda (x_t,(x_{at},s)).
        \mathtt{if}\, x_{at} \leq x_t + \esemantics{e_5}{\environment} \,
        \mathtt{then}\,  s \, \mathtt{else} \, \mathtt{None} \\
 \qquad\quad\ \,    \mathtt{pp=} e_4;
&& \qquad \universe' =
     \lambda (x_{\ell}, x_t). 
           \esemantics{e_4}{\environment}\, 
                 (\mathtt{timeout}\, (x_t,\universe (x_\ell,x_t))) 
 \\
 \qquad\quad\ \,    \mathtt{win=} e_5; \}\rsem_{{\environment} \, {\universe}}
&& \\
% \\
% {\cal F}\lsem\mathtt{exists \{ format=} e_1;
% % \semantics{\existsfeed{e_1}{e_2}{e_3}{e_4}}{\environment}{\universe} 
%  &=& \bigcup_{\atime \in \esemantics{e_2}{\environment}} f\; \atime \\
%  \qquad\qquad\ \ \ \,   \mathtt{locs=} e_2;
% &&\quad\mbox{where} \\
% %\begin{array}{l}
%  \qquad\qquad\ \ \ \,    \mathtt{sched=} e_3;
%   && \qquad \universe' \, = \lambda x{:}\locty * \timety. \esemantics{e_4}{\environment} (x,\universe(x)) \\
%  \qquad\qquad\ \ \ \,    \mathtt{pp=} e_4;
% &&\qquad
%   f\; \atime = \{(\atime,(\loc,\esemantics{e_1}{\environment}\; (\some{v})\} \\
%  \qquad\qquad\ \ \ \,   \mathtt{win=} e_5;
%  \}\rsem_{{\environment} \, {\universe}}
% &&\qquad\qquad \qquad\qquad 
%   \mbox{for some $\loc \in \esemantics{e_2}{\environment}$ such that
%     $\universe'(\atime,\loc) = \some{v}$} \\
% &&\qquad
%   f \; \atime = \{(\atime,(\loc,\esemantics{e_1}{\environment}\; \none\} \\
% &&\qquad\qquad \qquad\qquad 
%   \mbox{if no such $\loc$ exists} \\
% %\end{array}
% \\
\\
{\cal F}\lsem\mathtt{exists \{ format=} e_1;
% \semantics{\existsfeed{e_1}{e_2}{e_3}{e_4}}{\environment}{\universe} 
 &=& \bigcup_{\atime \in \esemantics{e_2}{\environment}} f\; \atime \\
 \qquad\qquad\ \ \ \,   \mathtt{locs=} e_2;
&&\quad\mbox{where} \\
%\begin{array}{l}
 \qquad\qquad\ \ \ \,    \mathtt{sched=} e_3;
  && 
\qquad f = \lambda \atime. \{(\atime,(\ell,\mathtt{Some} \; s))\}\ \mbox{for some $\ell$ and $s$ such that:}\\
 \qquad\qquad\ \ \ \,    \mathtt{pp=} e_4;
&&\qquad\quad
 \mbox{$(\atime,(\ell,\mathtt{Some} \; s)) \in \semantics{\allfeed{e_1}{e_2}{e_3}{e_4}{e_5}}{\environment}{\universe}$}
  \\
 \qquad\qquad\ \ \ \,   \mathtt{win=} e_5;
 \}\rsem_{{\environment} \, {\universe}}
&&\qquad
  f = \lambda \atime. \{(\atime,(\loc, \none))\}\  
  \mbox{for some $\loc \in \esemantics{e2}{\environment}$
if there exists no $\ell$ and $s$ such that:} \\
&&\qquad\quad 
 \mbox{$(\atime,(\ell,\mathtt{Some} \; s)) \in \semantics{\allfeed{e_1}{e_2}{e_3}{e_4}{e_5}}{\environment}{\universe}$}
\\

%\end{array}
\\
\\
\semantics{\feed_1 \unionfeed \feed_2}{\environment}{\universe} 
 &=& \semantics{\feed_1}{\environment}{\universe} 
     \bigcup
     \semantics{\feed_2}{\environment}{\universe} 
\\\\
\semantics{\feed_1 \sumfeed \feed_2}{\environment}{\universe} 
 &=& \{
      (\atime,\inl{v}) \setalt 
        (\atime,v) \in \semantics{\feed_1}{\environment}{\universe} 
     \}
     \bigcup
     \{
      (\atime,\inr{v}) \setalt 
        (\atime,v) \in \semantics{\feed_2}{\environment}{\universe}
     \}
\\\\
\semantics{\feed_1 \spairfeed \feed_2}{\environment}{\universe} 
 &=&
 \{(\atime,(v_1,v_2)) \setalt 
     (\atime,v_1) \in \semantics{\feed_1}{\environment}{\universe} 
     \; \mbox{and} \; 
     (\atime,v_2) \in \semantics{\feed_2}{\environment}{\universe}
  \}
\\\\
% \semantics{\feed_1 * \feed_2}{\environment}{\universe} 
%  &=&
%  \{(\atime_2,(v_1,v_2)) \setalt 
%      (\atime_1,v_1) \in \semantics{\feed_1}{\environment}{\universe} 
%      \; \mbox{and} \; 
% \\&&\qquad\qquad\qquad\ \ \,
%      (\atime_2,v_2) \in \semantics{\feed_2}{\environment}{\universe}
%      \; \mbox{and} \;
% \\&&\qquad\qquad\qquad\ \ \,
%      ((\atime_1',v_1') \in \semantics{\feed_1}{\environment}{\universe} 
%       \; \mbox{implies} \; (t_1' \leq t_1 \; \mbox{or} \; t_1' > t_2)) 
%   \}
% \\\\
\semantics{x{:}\feed_1 * \feed_2}{\environment}{\universe} 
 &=&
 \{(\atime_2,(v_1,v_2)) \setalt 
     (\atime_1,v_1) \in \semantics{\feed_1}{\environment}{\universe} 
     \; \mbox{and} \; 
\\&&\qquad\qquad\qquad\ \ \,
     (\atime_2,v_2) \in \semantics{\feed_2}{(\environment,x\mapsto{}v_1)}{\universe}
     \; \mbox{and} \; \atime_2 > \atime_1
  \}
\\\\
\semantics{x{:}\feed_1 \, {*}{*} \, \feed_2}{\environment}{\universe} 
 &=&
 \{(\atime_2,(v_1,v_2)) \setalt 
     (\atime_1,v_1) \in \semantics{\feed_1}{\environment}{\universe} 
     \; \mbox{and} \; 
\\&&\qquad\qquad\qquad\ \ \,
     (\atime_2,v_2) \in \semantics{\feed_2}{(\environment,x\mapsto{}v_1)}{\universe}
     \; \mbox{and} \; \atime_2 > \atime_1
\\&&\qquad\qquad\qquad\ \ \,
     ((\atime_1',v_1') \in \semantics{\feed_1}{\environment}{\universe} 
      \; \mbox{implies} \; (t_1' \leq t_1 \; \mbox{or} \; t_1' > t_2)) 
  \}
\\\\
{\cal F}\lsem
\mathtt{foreach{*}}\; x \; \mathtt{in}\; \feed_1 
%\semantics{\foreachupdate{x}{\feed_1}{\feed_2}}{\environment}{\universe} 
 &=&
 \{(\atime_2,v_2) \setalt 
     (\atime_1,v_1) \in \semantics{\feed_1}{\environment}{\universe} 
     \; \mbox{and} \; 
\\
\qquad\qquad\quad\ \, \mathtt{create}\; \feed_2 \rsem_{{\environment} \, {\universe}}
&&\qquad\qquad\ \,
     (\atime_2,v_2) \in \semantics{\feed_2}{(\environment,x\mapsto{}v_1)}{\universe}
     \; \mbox{and} \;
%\\&&\qquad\qquad\ \,
     \atime_2 > \atime_1 
  \}
\\\\
{\cal F}\lsem
\mathtt{foreach{*}{*}}\; x \; \mathtt{in}\; \feed_1 
%\semantics{\foreachcreate{x}{\feed_1}{\feed_2}}{\environment}{\universe} 
 &=&
 \{(\atime_2,v_2) \setalt 
     (\atime_1,v_1) \in \semantics{\feed_1}{\environment}{\universe} 
     \; \mbox{and} \; 
\\
\qquad\qquad\quad\ \, \mathtt{extend}\; \feed_2 \rsem_{{\environment} \, {\universe}}
&&\qquad\qquad\ \,
     (\atime_2,v_2) \in \semantics{\feed_2}{(\environment,x\mapsto{}v_1)}{\universe}
     \; \mbox{and} \; t_2 > t_1 \; \mbox{and} \;
\\&&\qquad\qquad\ \,
     ((\atime_1',v_1') \in \semantics{\feed_1}{\environment}{\universe} 
      \; \mbox{implies} \; (t_1' \leq t_1 \; \mbox{or} \; t_1' > t_2))      
  \}
\\\\
\semantics{\filterfeed{\feed}{e}}{\environment}{\universe} 
 &=&
\{(t,v) \setalt (t,v) \in \semantics{\feed}{\environment}{\universe} \; \mbox{and} \;
            \esemantics{e}{\environment}\; v = \mathtt{true}
\}
\\\\
% \semantics{\ppfeed{\feed}{e}}{\environment}{\universe} 
%  &=&
% \semantics{\feed}{\environment}{
%   (\lambda x{:}\locty * \timety. \esemantics{e}{\environment} (x,\universe(x)))} 
% \\\\
% \semantics{\remapfeed{\feed}{e}}{\environment}{\universe} 
%  &=&
% \semantics{\feed}{\environment}{(\universe \circ \esemantics{e}{\environment})}
% \\\\
\semantics{\refeed{\feed}{e}}{\environment}{\universe} 
 &=&
\{(\atime,\some{v}) \setalt 
   (\atime,v) \in \semantics{\feed}{\environment}{\universe} \; \mbox{and} \;
   \atime \in \esemantics{e}{\environment}
\} \bigcup
\\&&
\{(\atime,\none) \setalt
   (\atime,\_) \not\in \semantics{\feed}{\environment}{\universe} \; \mbox{and} \;
   \atime \in \esemantics{e}{\environment}
\}
\\\\
\semantics{\stutterfeed{\feed}{e}}{\environment}{\universe} 
 &=&
\{(\atime,v) \setalt 
   (\atime,v) \in \semantics{\feed}{\environment}{\universe} \; \mbox{and} \;
   \atime \in \esemantics{e}{\environment}
\} \bigcup
\\&&
\{(\atime,v) \setalt 
   (\atime',v) \in \semantics{\feed}{\environment}{\universe} \; \mbox{and} \;
   \atime \in \esemantics{e}{\environment}  \; \mbox{and} \;
\\&&\qquad\qquad\qquad\ \ \,
    \mbox{for all $\atime''$ such that $\atime' < \atime'' \leq \atime$,} \;
   (\atime'',\_) \not\in \semantics{\feed}{\environment}{\universe} \; 
\}

\end{array}
\]
\caption{Feed Language Semantics.}
\label{fig:semantics}
\end{figure*}


\subsection{Feed Semantics}

The calculus of feeds has a denotational semantics that allows
programmers to calculate the contents of their feeds relative to
the uncertainty present in the networking environment.  The
principal semantic function is written
$\semantics{\feed}{\environment}{\universe}$,
where  $\environment$ is an {\em environment} mapping variables to values
and $\universe$ is a {\em universe} mapping pairs of
schedule time and location to arrival time and string option.
Intuitively, the universe models the network.
When $\universe (\atime_1, \ell) = (\atime_2, \mathtt{Some} \; \astring)$,
 the interpretation is that if the run-time system requests data
from location $\ell$ at time $\atime_1$ then string data $\astring$
will be returned at time $\atime_2$.  The time $\atime_2$ must be
later than $\atime_1$.
When $\universe (\atime_1, \ell) = (\infty, \mathtt{None})$,
networking errors have simply made location $\ell$ unreachable.

The result of the semantic function 
$\semantics{\feed}{\environment}{\universe}$
is a set of metadata/data pairs, where the 
metadata is coded as follows.
%   The metadata 
% we model is somewhat simplified relative to the implementation.
% The metadata for base feeds includes both the scheduled time of the
% associated data item and its location of origin -- such 
% information is also often referred to as the data's 
% {\em provenance}~\cite{provenancestuff}.  As shown
% below, the metadata for pair 
% and sum feeds is structured similarly to the data items themselves.  

\[
\begin{array}{lcll} 
\meta & ::=     
& (\atime,\loc) & \mbox{base metadata} \\
& \bnfalt & (\atime,(\meta,\meta)) & \mbox{pair metadata} \\
& \bnfalt & (\atime,\inl{\meta}) & \mbox{sum metadata} \\
& \bnfalt & (\atime,\inr{\meta}) & \mbox{sum metadata} \\
& \bnfalt & (\atime,[\meta_1,\ldots,\meta_k]) & \mbox{list metadata} \\
\end{array}
\]  

Since every metadata item contains a top-level time $\atime$, that
time can be used to serialize the set of items as a stream, and 
our implementation does just that.  Items scheduled at the
same time may appear in any order in the implementation's
serialized stream.  To refer to the top-level time in any metadata
item $\meta$, we write $\mytime{\meta}$. 

Our semantics for feeds depends upon a semantics for the simply-typed
host language, which we write $\esemantics{e}{\environment}$. We assume that
given environment $\environment$ with type $\Gamma$ 
and expression $e$ with type $\tau$ in $\Gamma$, 
$\esemantics{e}{\environment} = v$
and $\turn v : \tau$.

Figure~\ref{fig:semantics} presents the semantic definitions for feeds themselves.  In these definitions,
we use conventional set-theoretic notation to define the elements
that inhabit the feed.
The first definition shows that the empty feed is the empty set. The
second gives the definition of a computed feed.  Notice that a
dummy location ($\generatedloc$) occupies the position of location
metadata.  The feed comprehensions are defined in a similar way. 
The filter feed and the feed comprehensions together serve to model
the full power of the feed comprehensions found in our implementation.


The semantics of {\tt all} and {\tt any} feeds are somewhat more complex
because they describe the consequences of networking errors and time outs.
In the case for {\tt all}, a function {\tt timeout} is defined.  This
function checks the item arrival time $x_{at}$ and compares it to its
scheduled time ($x_t$) plus its window ($\esemantics{e_5}{\environment}$).  If
the arrival time exceeds its window, {\tt None} is returned. Otherwise,
its data argument ($x_s$, which may be {\tt None} due to other networking 
errors) is returned.  The {\tt timeout} is used in definition of an 
alternate universe $\universe'$, that retrieves data from the outside world,
checks for timeout, and applies the preprocessor 
($\esemantics{e_4}{\environment}$) before returning its final result.
The elements of the feed itself are pairs of base metadata $(\atime,\ell)$
with data defined by $\esemantics{e_1 \; (\universe'(\loc,\atime))}{\environment}$
where $e_1$ is the \pads{}-generated parser.
The semantics of an {\tt any} feed is defined by selecting one of the
good (non-error) elements of the corresponding {\tt all} feed at the same time,
if any exist.  If there are no good elements in 
 the corresponding {\tt all} feed, then the {\tt any} feed will contain a
{\tt None} item at the given time, and the dummy location $\generatedloc$
occupies the location slot in the metadata.

The two forms of disjunctive feed are easy to define.  The
union feed is simply the set-theorectic union of
its constituent feeds.  The sum feed injects the elements of its constituent
feeds into a sum and likewise takes their union.

More care must be taken when defining the semantics of pairs.
The synchronous pair $(\feed_1,\feed_2)$
is formed by finding all elements of $\feed_1$
at a given time (including erroneous elements) 
and all elements of  $\feed_2$ (again including erroneous elements) 
at that time and
generating their Cartesian product.  Notice that if schedules 
do not intersect, a synchronous pair will wind up isomorphic to the
empty feed.  Metadata from constituent feeds is paired together.
The scheduled time of the composite item is noncontroversial in this case --
it is the same as the scheduled times of the two underlying feeds.

The elements of the continuous dependent pair ($x{:}\feed_1 * \feed_2$)
are calculated by first determining the elements of $\feed_1$. 
For each element $v$ in $\feed_1$, it is possible to calculate
the elements of $\feed_2$ by binding $x$ to $v$ in $\feed_2$'s definition.
From this point, the elements of the composite feed include all elements
$x$ of $\feed_1$ paired with the corresponding elements of $\feed_2$ scheduled
later than $x$.  Like in the case for synchronous pairs, metadata from 
constitute feeds is paired together in the composite feed.  {\em Unlike}
in the case for synchronous pairs, the two elements of the pair do
not share a common schedule time.  We adopt the later time ({\em i.e.,}
the time of the element from $\feed_2$) as the schedule time for the
composite feed element.

The difference between the continuous pair and the local pair
comes in the third line of the definition of the local pair.
Here, the existence of another element $(\meta_1',v_1')$
in $\feed_1$ in addition to $(\meta_1,v_1)$ implies that the 
schedule time for the item under consideration in $\feed_2$
must fall between the schedule times for $v_1$ and $v_1'$
(otherwise that item from $\feed_2$ is not included in the
final composite feed). 

As discussed earlier, the semantics of the two variants of $\mathtt{foreach}$
follows the semantics of dependent pairs closely.  Notice, for instance,
the similarity between the conditions on the local {\tt foreach} with those
for the local dependent pair.  

The semantics of the list comprehension feeds are closely connected to the semantics of
the synchronous pairs in that each list in the generated feed of lists contains elements
scheduled for the same time.

As a sanity check on our semantic definitions, we have proven
that the values contained in each feed are pairs of meta-data and data
values with the appropriate type.  More specifically, if the feed typing
rules give feed $\feed$ type $\sigma$, then its data has type $\sigma$ and
its meta data has type $\metatype{\sigma}$ where $\metatype{\sigma}$ is
defined as follows.
\[
\begin {array} {lcl}
\metatype{\ty} & = & \timety * \locty \\
\metatype{\optionty{\ty}} & = & \timety * \locty \\
\metatype{\sigma_1 * \sigma_2} & = & \timety * (\metatype{\sigma_1} * \metatype{\sigma_2}) \\
\metatype{\sigma_1 + \sigma_2} & = & \timety * (\metatype{\sigma_1} + \metatype{\sigma_2}) \\
\metatype{\listty{\sigma}} & = & \timety * (\listty{\metatype{\sigma}}) \\
\end{array}
\]
%We state this property formally as follows.
\begin{theorem}[Semantic Soundness]
If $\Gamma \turn \feed : \feedty{\sigma}$ and
for all $x$ in $\dom(\Gamma)$, $\turn \environment(x) : \Gamma(x)$
and $\turn \universe : \timety * \locty \arrow \timety * (\optionty{\stringty})$
then
for all $(\meta,v) \in \semantics{\feed}{\environment}{\universe}$,
$\turn (\meta,v) : \metatype{\sigma} * \sigma$. 
\end{theorem}
The proof follows by induction on the structure of $\feed$.

\newcommand{\rb}[1]{\raisebox{6ex}[0pt]{#1}}

\begin{figure*}[t]
\[
\begin{array}{lcl}

    {\cal C}\lsem\mathtt{\{ src=} e_{src}; 
 &=& (S, \{((\atime,\loc), \esemantics{e_f\; (\universe'(\loc,\atime))}{\environment}))
          \setalt \atime \in S
          \;\mbox{and}\; \loc \in  \esemantics{e_{src}}{\environment}
     \})
\\
 \quad\ \   \mathtt{sched=} e_{sched};
&&\quad\mbox{where} \\
 \quad\ \  \mathtt{win=} e_{win};
&& \qquad S = \esemantics{e_{sched}}{\environment} \\
 \quad\ \  \mathtt{pp=} e_{pp};
&& 
\qquad \mathtt{timeout} =  
     \lambda (x_t,(x_{at},x_s)).
        \mathtt{if}\, x_{at} \leq x_t + \esemantics{e_{win}}{\environment} \,
        \mathtt{then}\,  x_s \, \mathtt{else} \, \mathtt{None} 
 \\
 \quad\ \  \mathtt{format=} e_{f}; \}\rsem_{{\environment} \, {\universe}}
&& \qquad \universe' =
     \lambda (x_{\ell}, x_t). 
           \esemantics{e_{pp}}{\environment}\, 
                 (\mathtt{timeout}\, (x_t,\universe (x_\ell,x_t))) 
\\\\

\semantics{\mathtt{all}\ \corefeed}{\universe}{\environment} 
&=& 
A\ \ \ \mbox{where}\ (S,A) = \csemantics{C}{\universe}{\environment}
\\\\


%New version: any rule
\semantics{\mathtt{any}\ \corefeed}{\universe}{\environment} 
& = & \{ i_t\ |\ \atime \in S\}\\
&&
\begin{array}{l}
 \begin{array}{ll@{\hspace{1ex}}c@{\hspace{1ex}}l}
 \mbox{where} & (S,A)   & = &\csemantics{C}{\environment}{\universe}\\
              & A_\atime & = & \{(\meta,\some{v})\ | \ (\meta, \some{v}) \in A\ \mbox{and} \ \mytime{\meta} = \atime\}\\
              & i_\atime & = & \left\{ \begin{array}{lll}
                                           \mbox{\selectOne}(A_\atime) & \mbox{if} & |A_\atime| > 0\\
                                           ((\atime,\generatedloc), \none) & \mbox{if} & |A_\atime| = 0 \\
                                           \end{array} \right.\\
 \end{array}
\end{array} 
%%End New version: any rule
\\\\

\semantics{\emptyfeed}{\environment}{\universe} 
 &=& \{\;\}
\\\\
\semantics{\computed{e_1}{x}{e_2}}{\environment}{\universe} 
 &=& \{((\atime,\generatedloc), \esemantics{(\lambda x.e_1) \; \atime}{\environment}) 
          \setalt \atime \in  \esemantics{e_2}{\environment} 
     \} 
\\\\
\semantics{\comprehensionfeed{e}{x}{\feed}}{\environment}{\universe} 
 &=& \{((\mytime{\meta},\generatedloc), \esemantics{(\lambda x.e) \; v}{\environment}) 
          \setalt (\meta,v) \in  \semantics{\feed}{\environment}{\universe}  
     \} 
\\\\
\semantics{\filterfeed{\feed}{e}}{\environment}{\universe} 
 &=&
\{(\meta,v) \setalt (\meta,v) \in \semantics{\feed}{\environment}{\universe} \; \mbox{and} \;
            \esemantics{e \; v}{\environment} = \mathtt{true}
\}
\\\\
\semantics{\letfeed{x}{e_1}{\feed_2}}{\environment}{\universe} 
 &=& \semantics{\feed_2}{(\environment,x\mapsto\esemantics{e_1}{\environment})}{\universe} 
\\\\

\semantics{\feed_1 \unionfeed \feed_2}{\environment}{\universe} 
 &=& \semantics{\feed_1}{\environment}{\universe} 
     \bigcup
     \semantics{\feed_2}{\environment}{\universe} 
\\\\
\semantics{\feed_1 \sumfeed \feed_2}{\environment}{\universe} 
 &=& \{
      ((\mytime{\meta},\inl{\meta}),\inl{v}) \setalt 
        (\meta,v) \in \semantics{\feed_1}{\environment}{\universe} 
     \}
     \bigcup
     \{
      ((\mytime{\meta},\inr{\meta}),\inr{v}) \setalt 
        (\meta,v) \in \semantics{\feed_2}{\environment}{\universe}
     \}
\\\\
\semantics{(\feed_1, \feed_2)}{\environment}{\universe} 
 &=&
 \{((\mytime{\meta_1},(\meta_1,\meta_2)),(v_1,v_2)) \setalt 
     (\meta_1,v_1) \in \semantics{\feed_1}{\environment}{\universe} 
     \; \mbox{and} \; 
     (\meta_2,v_2) \in \semantics{\feed_2}{\environment}{\universe}
     \; \mbox{and} \; 
     \mytime{\meta_1} = \mytime{\meta_2}
  \}
\\\\
% \semantics{\feed_1 * \feed_2}{\environment}{\universe} 
%  &=&
%  \{(\atime_2,(v_1,v_2)) \setalt 
%      (\atime_1,v_1) \in \semantics{\feed_1}{\environment}{\universe} 
%      \; \mbox{and} \; 
% \\&&\qquad\qquad\qquad\ \ \,
%      (\atime_2,v_2) \in \semantics{\feed_2}{\environment}{\universe}
%      \; \mbox{and} \;
% \\&&\qquad\qquad\qquad\ \ \,
%      ((\atime_1',v_1') \in \semantics{\feed_1}{\environment}{\universe} 
%       \; \mbox{implies} \; (t_1' \leq t_1 \; \mbox{or} \; t_1' > t_2)) 
%   \}
% \\\\
\semantics{x{:}\feed_1 * \feed_2}{\environment}{\universe} 
 &=&
 \{(\mytime{\meta_2},(\meta_1,\meta_2)),(v_1,v_2)) \setalt 
     (\meta_1,v_1) \in \semantics{\feed_1}{\environment}{\universe} 
     \; \mbox{and} \; 
\\&&\qquad\qquad\qquad\qquad\qquad\qquad\ \ \,
     (\meta_2,v_2) \in \semantics{\feed_2}{(\environment,x\mapsto{}v_1)}{\universe}
     \; \mbox{and} \; \mytime{\meta_2} > \mytime{\meta_1}
  \}
\\\\
\semantics{x{:}\feed_1 \, {*}{*} \, \feed_2}{\environment}{\universe} 
 &=&
 \{(\mytime{\meta_2},(\meta_1,\meta_2)),(v_1,v_2)) \setalt 
     (\meta_1,v_1) \in \semantics{\feed_1}{\environment}{\universe} 
     \; \mbox{and} \; 
\\&&\qquad\qquad\qquad\qquad\qquad\qquad\ \ \,
     (\meta_2,v_2) \in \semantics{\feed_2}{(\environment,x\mapsto{}v_1)}{\universe}
     \; \mbox{and} \; \mytime{\meta_2} > \mytime{\meta_1}
\\&&\qquad\qquad\qquad\qquad\qquad\qquad\ \ \,
     ((\meta_1',v_1') \in \semantics{\feed_1}{\environment}{\universe} 
      \; \mbox{implies} \; (\mytime{\meta_1'} \leq \mytime{\meta_1} 
            \; \mbox{or} \; \mytime{\meta_1'} > \mytime{\meta_2})) 
  \}
\\\\
%%OLD foreach update
%%{\cal F}\lsem
%%\mathtt{foreach{*}}\; x \; \mathtt{in}\; \feed_1 
%%%\semantics{\foreachupdate{x}{\feed_1}{\feed_2}}{\environment}{\universe} 
%% &=&
%% \{(\meta_2,v_2) \setalt 
%%     (\meta_1,v_1) \in \semantics{\feed_1}{\environment}{\universe} 
%%     \; \mbox{and} \; 
%%\\
%%\qquad\qquad\ \ \mathtt{create}\; \feed_2 \rsem_{{\environment} \, {\universe}}
%%&&\qquad\qquad\ \,
%%     (\meta_2,v_2) \in \semantics{\feed_2}{(\environment,x\mapsto{}v_1)}{\universe}
%%     \; \mbox{and} \;
%%%\\&&\qquad\qquad\ \,
%%     \mytime{\meta_2} > \mytime{\meta_1} 
%%  \}
%%\\\\




%% New foreach *
{\cal F}\lsem
\mathtt{foreach{*}}\; x \; \mathtt{in}\; \feed_1 
%\semantics{\foreachupdate{x}{\feed_1}{\feed_2}}{\environment}{\universe} 
 &=&
   \{(\meta_2,v_2) \setalt (\atime,(\meta_1,\meta_2)),(v_1,v_2)) \in 
       \semantics{x{:}\feed_1 * \feed_2}{\environment}{\universe} \}
%% \{(\meta_2,v_2) \setalt 
%%     (\meta_1,v_1) \in \semantics{\feed_1}{\environment}{\universe} 
%%     \; \mbox{and} \; 
\\
\qquad\qquad\ \ \mathtt{create}\; \feed_2 \rsem_{{\environment} \, {\universe}}
%% End New foreach *
\\\\

%% Old foreach **
%%{\cal F}\lsem
%%\mathtt{foreach{*}{*}}\; x \; \mathtt{in}\; \feed_1 
%%%\semantics{\foreachcreate{x}{\feed_1}{\feed_2}}{\environment}{\universe} 
%% &=&
%% \{(\meta_2,v_2) \setalt 
%%     (\meta_1,v_1) \in \semantics{\feed_1}{\environment}{\universe} 
%%     \; \mbox{and} \; 
%%\\
%%\qquad\qquad\ \ \ \mathtt{update}\; \feed_2 \rsem_{{\environment} \, {\universe}}
%%&&\qquad\qquad\ \,
%%     (\meta_2,v_2) \in \semantics{\feed_2}{(\environment,x\mapsto{}v_1)}{\universe}
%%     \; \mbox{and} \; \mytime{\meta_2} > \mytime{\meta_1} \; \mbox{and} \;
%%\\&&\qquad\qquad\ \,
%%     ((\meta_1',v_1') \in \semantics{\feed_1}{\environment}{\universe} 
%%      \; \mbox{implies} \; (\mytime{\meta_1'} \leq \mytime{\meta_1} 
%%           \; \mbox{or} \; \mytime{\meta_1'} > \mytime{\meta_2}))      
%%  \}
%%\\\\
%% End Old foreach **

%% New Foreach **
{\cal F}\lsem
\mathtt{foreach{*}{*}}\; x \; \mathtt{in}\; \feed_1 
 &=&
   \{(\meta_2,v_2) \setalt (\atime,(\meta_1,\meta_2)),(v_1,v_2)) \in 
       \semantics{x{:}\feed_1\, {**}\, \feed_2}{\environment}{\universe} \}\\
\qquad\qquad\ \ \ \mathtt{update}\; \feed_2 \rsem_{{\environment} \, {\universe}}
\\\\
%% end new foreach **

% \semantics{\ppfeed{\feed}{e}}{\environment}{\universe} 
%  &=&
% \semantics{\feed}{\environment}{
%   (\lambda x{:}\locty * \timety. \esemantics{e}{\environment} (x,\universe(x)))} 
% \\\\
% \semantics{\remapfeed{\feed}{e}}{\environment}{\universe} 
%  &=&
% \semantics{\feed}{\environment}{(\universe \circ \esemantics{e}{\environment})}
% \\\\
%% \semantics{\refeed{\feed}{e}}{\environment}{\universe} 
%%  &=&
%% \{(\atime,\some{v}) \setalt 
%%    (\atime,v) \in \semantics{\feed}{\environment}{\universe} \; \mbox{and} \;
%%    \atime \in \esemantics{e}{\environment}
%% \} \bigcup
%% \\&&
%% \{(\atime,\none) \setalt
%%    (\atime,\_) \not\in \semantics{\feed}{\environment}{\universe} \; \mbox{and} \;
%%    \atime \in \esemantics{e}{\environment}
%% \}
%% \\\\
%% \semantics{\stutterfeed{\feed}{e}}{\environment}{\universe} 
%%  &=&
%% \{(\atime,v) \setalt 
%%    (\atime,v) \in \semantics{\feed}{\environment}{\universe} \; \mbox{and} \;
%%    \atime \in \esemantics{e}{\environment}
%% \} \bigcup
%% \\&&
%% \{(\atime,v) \setalt 
%%    (\atime',v) \in \semantics{\feed}{\environment}{\universe} \; \mbox{and} \;
%%    \atime \in \esemantics{e}{\environment}  \; \mbox{and} \;
%% \\&&\qquad\qquad\qquad\ \ \,
%%     \mbox{for all $\atime''$ such that $\atime' < \atime'' \leq \atime$,} \;
%%    (\atime'',\_) \not\in \semantics{\feed}{\environment}{\universe} \; 
%% \}
\semantics{[\feed \bnfalt x \leftarrow e]}{\environment}{\universe} 
 &=&
 \{((\atime,[\meta_1,\ldots,\meta_k]),[v_1,\ldots,v_k]) \setalt 
    \exists \atime.\forall i:1\ldots k.
     (\meta_i,v_i) \in \semantics{\feed}{\environment[x\mapsto z_i]}{\universe} 
     \; \mbox{and} \; 
     \mytime{\meta_i} =\atime
  \} \\
&&\quad\mbox{where} \quad\mbox{$[z_1,\ldots,z_k] = \esemantics{e}{\environment}$}
\\
\end{array}
\]
\caption{Feed Language Semantics.}
\label{fig:semantics}
\end{figure*}



