Development of a formal semantics for \padsd{} has been an integral
part of our language design process.  It has been a medium through
which we have communicated our ideas to one another precisely. It has also
allowed us to experiment with the nuanced consequences of our decisions in the
small before implementing them.  Its purpose in this paper is to
provide system users with a way to calculate the expected contents of
any feed they would specify, including subtle details of the effects
of timeouts and errors.

The abstract syntax for our core calculus of feeds appears in
Figure~\ref{fig:syntax}.  The calculus of feeds depends upon a {\em
host language}, in which one may compute locations, schedules,
and constraints.  For the purposes of the
formalization, the host language is the simply-typed lambda calculus.
Figure~\ref{fig:host-language} presents the syntax we will use to
write down host language expressions.  To simplify the semantics,
the host language contains a collection of special constants 
including parsers ($\parser$), locations ($\loc$), times ($\atime$),
and schedules ($\schedule$).  It naturally also includes structured types
such as pairs, sums, lists, streams and functions, all of which are 
completely standard.

\begin{figure}[t]
\[
\begin{array}{lll}
\multicolumn{3}{l}{\mbox{(feed specs)}}\\ 
\multicolumn{3}{l}{\feed \ ::=}   \\  
% & x &  \mbox{feed variable} \\ %% no feed variables now
% & \bnfalt 
& \emptyfeed & \mbox{empty feed} \\
 \bnfalt & \computed{e_1}{x}{e_2} & \mbox{computed feed} \\
 \bnfalt & \letfeed{x}{e_1}{\feed_2} & \mbox{let feed} \\
 \bnfalt &     \mathtt{all \{ format=} e_1; & \mbox{all locations}\\ 
& \qquad     \mathtt{locs=} e_2; & \\
& \qquad     \mathtt{sched=} e_3; & \\
& \qquad     \mathtt{pp=} e_4; & \\
& \qquad     \mathtt{win=} e_5; \}  & \\
%\allfeed{e_1}{e_2}{e_3}{e_4}{e_5} &  \\
 \bnfalt &     \mathtt{any \{ format=} e_1; & \mbox{one of several locations}\\ 
& \qquad     \mathtt{locs=} e_2; & \\
& \qquad     \mathtt{sched=} e_3; & \\
& \qquad     \mathtt{pp=} e_4; & \\
& \qquad     \mathtt{win=} e_5; \}  & \\
% \bnfalt & \existsfeed{e_1}{e_2}{e_3}{e_4}{e_5} & \mbox{one of several location} \\
 \bnfalt & \feed_1 \unionfeed \feed_2 & \mbox{union feed} \\
 \bnfalt & \feed_1 \sumfeed \feed_2 & \mbox{sum feed} \\
 \bnfalt & (\feed_1, \feed_2) & \mbox{synchronous pair} \\
% & \bnfalt & \feed_1 cartesian \feed_2 & \mbox{cartesian pair -- use a symbol different from *} \\
% & \bnfalt & \feed_1 * \feed_2 & \mbox{continuous pair} \\
% & \bnfalt & \feed_1 {*}{*} \feed_2 & \mbox{local pair} \\
 \bnfalt & x{:}\feed_1 * \feed_2 & \mbox{dependent continuous pair} \\
 \bnfalt & x{:}\feed_1\, {*}{*} \, \feed_2 & \mbox{dependent local pair} \\
 \bnfalt &     \mathtt{foreach{*}}\; x \; 
    \mathtt{in}\; \feed_1 & \mbox{for each $x$ create continuous $\feed_2$} \\
 &   \quad \mathtt{create}\; \feed_2 \\
 \bnfalt &     \mathtt{foreach{*}{*}}\; x \; 
    \mathtt{in}\; \feed_1 & \mbox{for each $x$ create local $\feed_2$}\\
 &   \quad \mathtt{extend}\; \feed_2 \\
%\foreachcreate{x}{\feed_1}{\feed_2} & \mbox{for each $x$ create continuous $F_2$} \\
% \bnfalt & \foreachupdate{x}{\feed_1}{\feed_2} & \mbox{for each $x$ create local $F_2$} \\
 \bnfalt & \filterfeed{\feed}{e} & \mbox{filter out some elements} \\
% & \bnfalt & \ppfeed{\feed}{e} & \mbox{preprocess (eg, unzip) data} \\
% & \bnfalt & \remap{\feed}{e} & \mbox{direct feed to different locations/times} \\
% & \bnfalt & \refeed{\feed}{e} & \mbox{adapt feed to new schedule; 
%                                               fill missing entries with ``None''} \\
% & \bnfalt & \stutterfeed{\feed}{e} & \mbox{stutter on new schedule} \\
\end{array}
\]
\caption{Feed Language Syntax.}
\label{fig:syntax}
\end{figure}


\begin{figure}[t]
\[
\begin{array}{lrll}
\multicolumn{4}{l}{\mbox{(host-language base types)}}\\ 
\basety & ::= & \multicolumn{2}{l}{\boolty \bnfalt \stringty \bnfalt \locty \bnfalt \timety \bnfalt \schedulety} \\
\\
\multicolumn{4}{l}{\mbox{(host-language types)}}\\ 
\ty & ::= & \basety & \mbox{base types} \\
 & \bnfalt & \ty_1 * \ty_2 & \mbox{pair types} \\
 & \bnfalt & \optionty{\ty} & \mbox{option types}\\
 & \bnfalt & \ty_1 + \ty_2 & \mbox{union types} \\
 & \bnfalt & \listty{\ty} & \mbox{list types}\\
 & \bnfalt & \feedty{\ty} & \mbox{stream types}\\
 & \bnfalt & \ty_1 \arrow \ty_2 & \mbox{function types} \\
\\
\multicolumn{4}{l}{\mbox{(host-language values)}}\\ 
\data & ::=     & \constant & \mbox{generic constant} \\
 & \bnfalt & \parser & \mbox{parser (generated from PADS/ML)} \\
 & \bnfalt & \loc & \mbox{locations} \\
 & \bnfalt & \atime & \mbox{times} \\
 & \bnfalt & \schedule & \mbox{schedules} \\
 & \bnfalt & \none \bnfalt 
                           \some{\data} & \mbox{optional values}\\
 & \bnfalt & (\data_1,\data_2) & \mbox{pairs} \\
 & \bnfalt & \inl{\data} \bnfalt 
                           \inr{\data} & \mbox{union values} \\
 & \bnfalt & \nillist \bnfalt 
                           \conslist{\data_1}{\data_2} & \mbox{list values} \\
 & \bnfalt & \nilstream \bnfalt 
                           \consstream{\data_1}{\data_2} & \mbox{stream values} \\

& \bnfalt & \lambda x{:}\ty.\expression & \mbox{function values} \\
\\
\multicolumn{4}{l}{\mbox{(host-language expressions)}}\\ 
\expression & ::= & \generalvar & \mbox{variables} \\
 & \bnfalt & \data & \mbox{data values} \\
 & \bnfalt & \none \bnfalt 
              \some{\expression} & \mbox{option expressions}\\
 & \bnfalt & (\expression_1,\expression_2) \bnfalt e.1 \bnfalt e.2 
    & \mbox{pair expressions} \\
% & \bnfalt & \inl{\expression} \bnfalt 
%             \inr{\expression} & \mbox{union expressions} \\
% & \bnfalt & \expression_1 \; \expression_2 & \mbox{application expression} \\
 & \bnfalt & ... & \mbox{more typed lambda expressions} \\
\\
\multicolumn{4}{l}{\mbox{(feed meta-data:  a subset of host language values)}}\\ 
%\multicolumn{4}{l}{\mbox{(a special location (\generatedloc) is used when data is created artificially)}}\\ 
\meta & ::=     
& (\atime,\loc) & \mbox{base metadata} \\
& \bnfalt & (\atime,(\meta,\meta)) & \mbox{pair metadata} \\
& \bnfalt & (\atime,\inl{\meta}) & \mbox{sum metadata} \\
& \bnfalt & (\atime,\inr{\meta}) & \mbox{sum metadata} \\
\end{array}
\]
\caption{Host Language Syntax.}
\label{fig:host-language}
\end{figure}


\begin{figure*}[t]

% \[
% \infer[(\textit{t-var})]
% {\Gamma \turn x : \Gamma(x)}
% {}
% \]

\[
\infer[(\textit{t-empty})]
{\Gamma \turn \emptyfeed : \feedty{\ty}}
{}
\]

\[
\infer[(\textit{t-compute})]
{\Gamma \turn \computed{e_1}{x}{e_2} : \feedty{\optionty{\ty}}}
{
  \Gamma \turn e_2 : \schedulety &
  \Gamma,x{:}\timety \turn e_1 : \optionty{\ty} 
}
\]

\[
\infer[(\textit{t-let})]
{\Gamma \turn \letfeed{x}{e_1}{\feed_2} : \feedty{\ty_2}}
{
  \Gamma \turn e_1 : \ty_1 & 
  \Gamma,x{:}\ty_1 \turn \feed_2 : \feedty{\ty_2} 
}
\]

\[
\infer[(\textit{t-all})]
{\Gamma \turn \allfeed{e_1}{e_2}{e_3}{e_4}{e_5} : \feedty{\optionty{\ty}}}
{
 \begin{array}{c}
  \Gamma \turn e_1 : \optionty{\stringty} \arrow \optionty{\ty} \qquad
  \Gamma \turn e_2 : \listty{\locty} \\
  \Gamma \turn e_3 : \schedulety \qquad
  \Gamma \turn e_4 : \optionty{\stringty} \arrow \optionty{\stringty}  \qquad
  \Gamma \turn e_5 : \timety
 \end{array}
}
\]

\[
\infer[(\textit{t-any})]
{\Gamma \turn \existsfeed{e_1}{e_2}{e_3}{e_4}{e_5} : \feedty{\optionty{\ty}}}
{
 \begin{array}{c}
  \Gamma \turn e_1 : \optionty{\stringty} \arrow \optionty{\ty} \qquad
  \Gamma \turn e_2 : \listty{\locty} \\
  \Gamma \turn e_3 : \schedulety \qquad
  \Gamma \turn e_4 : \optionty{\stringty} \arrow \optionty{\stringty}  \qquad
  \Gamma \turn e_5 : \timety
 \end{array}
}
\]

\[
\infer[(\textit{t-union})]
{\Gamma \turn \feed_1 \unionfeed \feed_2  : \feedty{\ty}}
{
  \Gamma \turn \feed_1 : \feedty{\ty} &
  \Gamma \turn \feed_2 : \feedty{\ty}
}
\]

\[
\infer[(\textit{t-sum})]
{\Gamma \turn \feed_1 \sumfeed \feed_2  : \feedty{\ty_1 + \ty_2}}
{
  \Gamma \turn \feed_1 : \feedty{\ty_1} &
  \Gamma \turn \feed_2 : \feedty{\ty_2}
}
\]

\[
\infer[(\textit{t-synch-pair})]
{\Gamma \turn \feed_1 \spairfeed \feed_2  : \feedty{\ty_1 * \ty_2}}
{
  \Gamma \turn \feed_1 : \feedty{\ty_1} &
  \Gamma \turn \feed_2 : \feedty{\ty_2}
}
\]

% \[
% \infer[(\textit{t-local-pair})]
% {\Gamma \turn \feed_1 * \feed_2  : \feedty{\ty_1 * \ty_2}}
% {
%   \Gamma \turn \feed_1 : \feedty{\ty_1} &
%   \Gamma \turn \feed_2 : \feedty{\ty_2}
% }
% \]

\[
\infer[(\textit{t-cont-pair})]
{\Gamma \turn x{:}\feed_1 * \feed_2  : \feedty{\ty_1 * \ty_2}}
{
  \Gamma \turn \feed_1 : \feedty{\ty_1} &
  \Gamma,x{:}\ty_1 \turn \feed_2 : \feedty{\ty_2}
}
\]

\[
\infer[(\textit{t-local-pair})]
 {\Gamma \turn x{:}\feed_1 \allpairfeed \feed_2  : \feedty{\ty_1 * \ty_2}}
 {
   \Gamma \turn \feed_1 : \feedty{\ty_1} &
   \Gamma,x{:}\ty_1 \turn \feed_2 : \feedty{\ty_2}
 }
\]

\[
\infer[(\textit{t-foreachcont})]
{\Gamma \turn \foreachcreate{x}{\feed_1}{\feed_2}  : \feedty{\ty_2}}
{
  \Gamma \turn \feed_1 : \feedty{\ty_1} &
  \Gamma,x{:}\ty_1 \turn \feed_2 : \feedty{\ty_2}
}
\]

\[
\infer[(\textit{t-foreachlocal})]
{\Gamma \turn \foreachupdate{x}{\feed_1}{\feed_2}  : \feedty{\ty_2}}
{
  \Gamma \turn \feed_1 : \feedty{\ty_1} &
  \Gamma,x{:}\ty_1 \turn \feed_2 : \feedty{\ty_2}
}
\]


\[
\infer[(\textit{t-filter})]
{\Gamma \turn \filterfeed{\feed}{e} : \feedty{\ty}}
{
  \Gamma \turn \feed : \feedty{\ty} &
  \Gamma \turn e : \ty \arrow \boolty
}
\]

% \[
% \infer[(\textit{t-pp})]
% {\Gamma \turn \ppfeed{\feed}{e} : \feedty{\ty}}
% {
%   \Gamma \turn \feed : \feedty{\ty} &
%   \Gamma \turn e : ((\locty * \timety) * \stringty) \arrow \stringty
% }
% \]

% \[
% \infer[(\textit{t-redirect})]
% {\Gamma \turn \remapfeed{\feed}{e} : \feedty{\ty}}
% {
%   \Gamma \turn \feed : \feedty{\ty} &
%   \Gamma \turn e : \locty * \timety \arrow \locty * \timety
% }
% \]

\[
\infer[(\textit{t-reschedule})]
{\Gamma \turn \refeed{\feed}{e} : \feedty{\optionty{\ty}}}
{
  \Gamma \turn \feed : \feedty{\ty} &
  \Gamma \turn e : \schedulety
}
\]

\[
\infer[(\textit{t-stutter})]
{\Gamma \turn \stutterfeed{\feed}{e} : \feedty{\ty}}
{
  \Gamma \turn \feed : \feedty{\ty} &
  \Gamma \turn e : \schedulety
}
\]

\caption{Feed Language Typing.}
\label{fig:typing}
\end{figure*}

%
\begin{figure*}[t]
\[
\begin{array}{lcl}
\esemantics{e}{\environment} &=& \mbox{denotation of simply-typed term $e$ in environment $\environment$ mapping variables to values.}
\\
\\
\semantics{\feed}{\environment}{\universe} &=& \mbox{denotation of 
$\feed$ in environment $\environment$ mapping variables to values.} \\
 && \mbox{and ``universe'' $\universe$ mapping schedule time and location to 
          arrival time and string data}
\\
% \\
% \semantics{x}{\environment}{\universe} 
%  &=& (\environment(x_t), \environment(x))
% \\
\\
\semantics{\emptyfeed}{\environment}{\universe} 
 &=& \{\;\}
\\\\
\semantics{\computed{e_1}{x}{e_2}}{\environment}{\universe} 
 &=& \{(\atime, \esemantics{(\lambda x.e_1)}{\environment}\; \atime) 
          \setalt \atime \in  \esemantics{e_2}{\environment} 
     \} 
\\\\

%  \begin{array}{l}
    {\cal F}\lsem\mathtt{all \{ format=} e_1; 
%    \allfeed{e_1}{e_2}{e_3}{e_4}
%  \end{array} 
%}{\environment}{\universe} 
 &=& \{(\atime, (\loc, \esemantics{e_1}{\environment} \; (\universe'(\loc,\atime))))
          \setalt \atime \in  \esemantics{e_3}{\environment} 
          \;\mbox{and}\; \loc \in  \esemantics{e_2}{\environment}
     \} 
\\
 \qquad\quad\ \,   \mathtt{locs=} e_2;
&&\quad\mbox{where} \\
 \qquad\quad\ \,    \mathtt{sched=} e_3;
&& \qquad \mathtt{timeout} =  
     \lambda (x_t,(x_{at},s)).
        \mathtt{if}\, x_{at} \leq x_t + \esemantics{e_5}{\environment} \,
        \mathtt{then}\,  s \, \mathtt{else} \, \mathtt{None} \\
 \qquad\quad\ \,    \mathtt{pp=} e_4;
&& \qquad \universe' =
     \lambda (x_{\ell}, x_t). 
           \esemantics{e_4}{\environment}\, 
                 (\mathtt{timeout}\, (x_t,\universe (x_\ell,x_t))) 
 \\
 \qquad\quad\ \,    \mathtt{win=} e_5; \}\rsem_{{\environment} \, {\universe}}
&& \\
% \\
% {\cal F}\lsem\mathtt{exists \{ format=} e_1;
% % \semantics{\existsfeed{e_1}{e_2}{e_3}{e_4}}{\environment}{\universe} 
%  &=& \bigcup_{\atime \in \esemantics{e_2}{\environment}} f\; \atime \\
%  \qquad\qquad\ \ \ \,   \mathtt{locs=} e_2;
% &&\quad\mbox{where} \\
% %\begin{array}{l}
%  \qquad\qquad\ \ \ \,    \mathtt{sched=} e_3;
%   && \qquad \universe' \, = \lambda x{:}\locty * \timety. \esemantics{e_4}{\environment} (x,\universe(x)) \\
%  \qquad\qquad\ \ \ \,    \mathtt{pp=} e_4;
% &&\qquad
%   f\; \atime = \{(\atime,(\loc,\esemantics{e_1}{\environment}\; (\some{v})\} \\
%  \qquad\qquad\ \ \ \,   \mathtt{win=} e_5;
%  \}\rsem_{{\environment} \, {\universe}}
% &&\qquad\qquad \qquad\qquad 
%   \mbox{for some $\loc \in \esemantics{e_2}{\environment}$ such that
%     $\universe'(\atime,\loc) = \some{v}$} \\
% &&\qquad
%   f \; \atime = \{(\atime,(\loc,\esemantics{e_1}{\environment}\; \none\} \\
% &&\qquad\qquad \qquad\qquad 
%   \mbox{if no such $\loc$ exists} \\
% %\end{array}
% \\
\\
{\cal F}\lsem\mathtt{exists \{ format=} e_1;
% \semantics{\existsfeed{e_1}{e_2}{e_3}{e_4}}{\environment}{\universe} 
 &=& \bigcup_{\atime \in \esemantics{e_2}{\environment}} f\; \atime \\
 \qquad\qquad\ \ \ \,   \mathtt{locs=} e_2;
&&\quad\mbox{where} \\
%\begin{array}{l}
 \qquad\qquad\ \ \ \,    \mathtt{sched=} e_3;
  && 
\qquad f = \lambda \atime. \{(\atime,(\ell,\mathtt{Some} \; s))\}\ \mbox{for some $\ell$ and $s$ such that:}\\
 \qquad\qquad\ \ \ \,    \mathtt{pp=} e_4;
&&\qquad\quad
 \mbox{$(\atime,(\ell,\mathtt{Some} \; s)) \in \semantics{\allfeed{e_1}{e_2}{e_3}{e_4}{e_5}}{\environment}{\universe}$}
  \\
 \qquad\qquad\ \ \ \,   \mathtt{win=} e_5;
 \}\rsem_{{\environment} \, {\universe}}
&&\qquad
  f = \lambda \atime. \{(\atime,(\loc, \none))\}\  
  \mbox{for some $\loc \in \esemantics{e2}{\environment}$
if there exists no $\ell$ and $s$ such that:} \\
&&\qquad\quad 
 \mbox{$(\atime,(\ell,\mathtt{Some} \; s)) \in \semantics{\allfeed{e_1}{e_2}{e_3}{e_4}{e_5}}{\environment}{\universe}$}
\\

%\end{array}
\\
\\
\semantics{\feed_1 \unionfeed \feed_2}{\environment}{\universe} 
 &=& \semantics{\feed_1}{\environment}{\universe} 
     \bigcup
     \semantics{\feed_2}{\environment}{\universe} 
\\\\
\semantics{\feed_1 \sumfeed \feed_2}{\environment}{\universe} 
 &=& \{
      (\atime,\inl{v}) \setalt 
        (\atime,v) \in \semantics{\feed_1}{\environment}{\universe} 
     \}
     \bigcup
     \{
      (\atime,\inr{v}) \setalt 
        (\atime,v) \in \semantics{\feed_2}{\environment}{\universe}
     \}
\\\\
\semantics{\feed_1 \spairfeed \feed_2}{\environment}{\universe} 
 &=&
 \{(\atime,(v_1,v_2)) \setalt 
     (\atime,v_1) \in \semantics{\feed_1}{\environment}{\universe} 
     \; \mbox{and} \; 
     (\atime,v_2) \in \semantics{\feed_2}{\environment}{\universe}
  \}
\\\\
% \semantics{\feed_1 * \feed_2}{\environment}{\universe} 
%  &=&
%  \{(\atime_2,(v_1,v_2)) \setalt 
%      (\atime_1,v_1) \in \semantics{\feed_1}{\environment}{\universe} 
%      \; \mbox{and} \; 
% \\&&\qquad\qquad\qquad\ \ \,
%      (\atime_2,v_2) \in \semantics{\feed_2}{\environment}{\universe}
%      \; \mbox{and} \;
% \\&&\qquad\qquad\qquad\ \ \,
%      ((\atime_1',v_1') \in \semantics{\feed_1}{\environment}{\universe} 
%       \; \mbox{implies} \; (t_1' \leq t_1 \; \mbox{or} \; t_1' > t_2)) 
%   \}
% \\\\
\semantics{x{:}\feed_1 * \feed_2}{\environment}{\universe} 
 &=&
 \{(\atime_2,(v_1,v_2)) \setalt 
     (\atime_1,v_1) \in \semantics{\feed_1}{\environment}{\universe} 
     \; \mbox{and} \; 
\\&&\qquad\qquad\qquad\ \ \,
     (\atime_2,v_2) \in \semantics{\feed_2}{(\environment,x\mapsto{}v_1)}{\universe}
     \; \mbox{and} \; \atime_2 > \atime_1
  \}
\\\\
\semantics{x{:}\feed_1 \, {*}{*} \, \feed_2}{\environment}{\universe} 
 &=&
 \{(\atime_2,(v_1,v_2)) \setalt 
     (\atime_1,v_1) \in \semantics{\feed_1}{\environment}{\universe} 
     \; \mbox{and} \; 
\\&&\qquad\qquad\qquad\ \ \,
     (\atime_2,v_2) \in \semantics{\feed_2}{(\environment,x\mapsto{}v_1)}{\universe}
     \; \mbox{and} \; \atime_2 > \atime_1
\\&&\qquad\qquad\qquad\ \ \,
     ((\atime_1',v_1') \in \semantics{\feed_1}{\environment}{\universe} 
      \; \mbox{implies} \; (t_1' \leq t_1 \; \mbox{or} \; t_1' > t_2)) 
  \}
\\\\
{\cal F}\lsem
\mathtt{foreach{*}}\; x \; \mathtt{in}\; \feed_1 
%\semantics{\foreachupdate{x}{\feed_1}{\feed_2}}{\environment}{\universe} 
 &=&
 \{(\atime_2,v_2) \setalt 
     (\atime_1,v_1) \in \semantics{\feed_1}{\environment}{\universe} 
     \; \mbox{and} \; 
\\
\qquad\qquad\quad\ \, \mathtt{create}\; \feed_2 \rsem_{{\environment} \, {\universe}}
&&\qquad\qquad\ \,
     (\atime_2,v_2) \in \semantics{\feed_2}{(\environment,x\mapsto{}v_1)}{\universe}
     \; \mbox{and} \;
%\\&&\qquad\qquad\ \,
     \atime_2 > \atime_1 
  \}
\\\\
{\cal F}\lsem
\mathtt{foreach{*}{*}}\; x \; \mathtt{in}\; \feed_1 
%\semantics{\foreachcreate{x}{\feed_1}{\feed_2}}{\environment}{\universe} 
 &=&
 \{(\atime_2,v_2) \setalt 
     (\atime_1,v_1) \in \semantics{\feed_1}{\environment}{\universe} 
     \; \mbox{and} \; 
\\
\qquad\qquad\quad\ \, \mathtt{extend}\; \feed_2 \rsem_{{\environment} \, {\universe}}
&&\qquad\qquad\ \,
     (\atime_2,v_2) \in \semantics{\feed_2}{(\environment,x\mapsto{}v_1)}{\universe}
     \; \mbox{and} \; t_2 > t_1 \; \mbox{and} \;
\\&&\qquad\qquad\ \,
     ((\atime_1',v_1') \in \semantics{\feed_1}{\environment}{\universe} 
      \; \mbox{implies} \; (t_1' \leq t_1 \; \mbox{or} \; t_1' > t_2))      
  \}
\\\\
\semantics{\filterfeed{\feed}{e}}{\environment}{\universe} 
 &=&
\{(t,v) \setalt (t,v) \in \semantics{\feed}{\environment}{\universe} \; \mbox{and} \;
            \esemantics{e}{\environment}\; v = \mathtt{true}
\}
\\\\
% \semantics{\ppfeed{\feed}{e}}{\environment}{\universe} 
%  &=&
% \semantics{\feed}{\environment}{
%   (\lambda x{:}\locty * \timety. \esemantics{e}{\environment} (x,\universe(x)))} 
% \\\\
% \semantics{\remapfeed{\feed}{e}}{\environment}{\universe} 
%  &=&
% \semantics{\feed}{\environment}{(\universe \circ \esemantics{e}{\environment})}
% \\\\
\semantics{\refeed{\feed}{e}}{\environment}{\universe} 
 &=&
\{(\atime,\some{v}) \setalt 
   (\atime,v) \in \semantics{\feed}{\environment}{\universe} \; \mbox{and} \;
   \atime \in \esemantics{e}{\environment}
\} \bigcup
\\&&
\{(\atime,\none) \setalt
   (\atime,\_) \not\in \semantics{\feed}{\environment}{\universe} \; \mbox{and} \;
   \atime \in \esemantics{e}{\environment}
\}
\\\\
\semantics{\stutterfeed{\feed}{e}}{\environment}{\universe} 
 &=&
\{(\atime,v) \setalt 
   (\atime,v) \in \semantics{\feed}{\environment}{\universe} \; \mbox{and} \;
   \atime \in \esemantics{e}{\environment}
\} \bigcup
\\&&
\{(\atime,v) \setalt 
   (\atime',v) \in \semantics{\feed}{\environment}{\universe} \; \mbox{and} \;
   \atime \in \esemantics{e}{\environment}  \; \mbox{and} \;
\\&&\qquad\qquad\qquad\ \ \,
    \mbox{for all $\atime''$ such that $\atime' < \atime'' \leq \atime$,} \;
   (\atime'',\_) \not\in \semantics{\feed}{\environment}{\universe} \; 
\}

\end{array}
\]
\caption{Feed Language Semantics.}
\label{fig:semantics}
\end{figure*}



\newcommand{\rb}[1]{\raisebox{6ex}[0pt]{#1}}

\begin{figure*}[t]
\[
\begin{array}{lcl}

    {\cal C}\lsem\mathtt{\{ src=} e_{src}; 
 &=& (S, \{((\atime,\loc), \esemantics{e_f\; (\universe'(\loc,\atime))}{\environment}))
          \setalt \atime \in S
          \;\mbox{and}\; \loc \in  \esemantics{e_{src}}{\environment}
     \})
\\
 \quad\ \   \mathtt{sched=} e_{sched};
&&\quad\mbox{where} \\
 \quad\ \  \mathtt{win=} e_{win};
&& \qquad S = \esemantics{e_{sched}}{\environment} \\
 \quad\ \  \mathtt{pp=} e_{pp};
&& 
\qquad \mathtt{timeout} =  
     \lambda (x_t,(x_{at},x_s)).
        \mathtt{if}\, x_{at} \leq x_t + \esemantics{e_{win}}{\environment} \,
        \mathtt{then}\,  x_s \, \mathtt{else} \, \mathtt{None} 
 \\
 \quad\ \  \mathtt{format=} e_{f}; \}\rsem_{{\environment} \, {\universe}}
&& \qquad \universe' =
     \lambda (x_{\ell}, x_t). 
           \esemantics{e_{pp}}{\environment}\, 
                 (\mathtt{timeout}\, (x_t,\universe (x_\ell,x_t))) 
\\\\

\semantics{\mathtt{all}\ \corefeed}{\universe}{\environment} 
&=& 
A\ \ \ \mbox{where}\ (S,A) = \csemantics{C}{\universe}{\environment}
\\\\


%New version: any rule
\semantics{\mathtt{any}\ \corefeed}{\universe}{\environment} 
& = & \{ i_t\ |\ \atime \in S\}\\
&&
\begin{array}{l}
 \begin{array}{ll@{\hspace{1ex}}c@{\hspace{1ex}}l}
 \mbox{where} & (S,A)   & = &\csemantics{C}{\environment}{\universe}\\
              & A_\atime & = & \{(\meta,\some{v})\ | \ (\meta, \some{v}) \in A\ \mbox{and} \ \mytime{\meta} = \atime\}\\
              & i_\atime & = & \left\{ \begin{array}{lll}
                                           \mbox{\selectOne}(A_\atime) & \mbox{if} & |A_\atime| > 0\\
                                           ((\atime,\generatedloc), \none) & \mbox{if} & |A_\atime| = 0 \\
                                           \end{array} \right.\\
 \end{array}
\end{array} 
%%End New version: any rule
\\\\

\semantics{\emptyfeed}{\environment}{\universe} 
 &=& \{\;\}
\\\\
\semantics{\computed{e_1}{x}{e_2}}{\environment}{\universe} 
 &=& \{((\atime,\generatedloc), \esemantics{(\lambda x.e_1) \; \atime}{\environment}) 
          \setalt \atime \in  \esemantics{e_2}{\environment} 
     \} 
\\\\
\semantics{\comprehensionfeed{e}{x}{\feed}}{\environment}{\universe} 
 &=& \{((\mytime{\meta},\generatedloc), \esemantics{(\lambda x.e) \; v}{\environment}) 
          \setalt (\meta,v) \in  \semantics{\feed}{\environment}{\universe}  
     \} 
\\\\
\semantics{\filterfeed{\feed}{e}}{\environment}{\universe} 
 &=&
\{(\meta,v) \setalt (\meta,v) \in \semantics{\feed}{\environment}{\universe} \; \mbox{and} \;
            \esemantics{e \; v}{\environment} = \mathtt{true}
\}
\\\\
\semantics{\letfeed{x}{e_1}{\feed_2}}{\environment}{\universe} 
 &=& \semantics{\feed_2}{(\environment,x\mapsto\esemantics{e_1}{\environment})}{\universe} 
\\\\

\semantics{\feed_1 \unionfeed \feed_2}{\environment}{\universe} 
 &=& \semantics{\feed_1}{\environment}{\universe} 
     \bigcup
     \semantics{\feed_2}{\environment}{\universe} 
\\\\
\semantics{\feed_1 \sumfeed \feed_2}{\environment}{\universe} 
 &=& \{
      ((\mytime{\meta},\inl{\meta}),\inl{v}) \setalt 
        (\meta,v) \in \semantics{\feed_1}{\environment}{\universe} 
     \}
     \bigcup
     \{
      ((\mytime{\meta},\inr{\meta}),\inr{v}) \setalt 
        (\meta,v) \in \semantics{\feed_2}{\environment}{\universe}
     \}
\\\\
\semantics{(\feed_1, \feed_2)}{\environment}{\universe} 
 &=&
 \{((\mytime{\meta_1},(\meta_1,\meta_2)),(v_1,v_2)) \setalt 
     (\meta_1,v_1) \in \semantics{\feed_1}{\environment}{\universe} 
     \; \mbox{and} \; 
     (\meta_2,v_2) \in \semantics{\feed_2}{\environment}{\universe}
     \; \mbox{and} \; 
     \mytime{\meta_1} = \mytime{\meta_2}
  \}
\\\\
% \semantics{\feed_1 * \feed_2}{\environment}{\universe} 
%  &=&
%  \{(\atime_2,(v_1,v_2)) \setalt 
%      (\atime_1,v_1) \in \semantics{\feed_1}{\environment}{\universe} 
%      \; \mbox{and} \; 
% \\&&\qquad\qquad\qquad\ \ \,
%      (\atime_2,v_2) \in \semantics{\feed_2}{\environment}{\universe}
%      \; \mbox{and} \;
% \\&&\qquad\qquad\qquad\ \ \,
%      ((\atime_1',v_1') \in \semantics{\feed_1}{\environment}{\universe} 
%       \; \mbox{implies} \; (t_1' \leq t_1 \; \mbox{or} \; t_1' > t_2)) 
%   \}
% \\\\
\semantics{x{:}\feed_1 * \feed_2}{\environment}{\universe} 
 &=&
 \{(\mytime{\meta_2},(\meta_1,\meta_2)),(v_1,v_2)) \setalt 
     (\meta_1,v_1) \in \semantics{\feed_1}{\environment}{\universe} 
     \; \mbox{and} \; 
\\&&\qquad\qquad\qquad\qquad\qquad\qquad\ \ \,
     (\meta_2,v_2) \in \semantics{\feed_2}{(\environment,x\mapsto{}v_1)}{\universe}
     \; \mbox{and} \; \mytime{\meta_2} > \mytime{\meta_1}
  \}
\\\\
\semantics{x{:}\feed_1 \, {*}{*} \, \feed_2}{\environment}{\universe} 
 &=&
 \{(\mytime{\meta_2},(\meta_1,\meta_2)),(v_1,v_2)) \setalt 
     (\meta_1,v_1) \in \semantics{\feed_1}{\environment}{\universe} 
     \; \mbox{and} \; 
\\&&\qquad\qquad\qquad\qquad\qquad\qquad\ \ \,
     (\meta_2,v_2) \in \semantics{\feed_2}{(\environment,x\mapsto{}v_1)}{\universe}
     \; \mbox{and} \; \mytime{\meta_2} > \mytime{\meta_1}
\\&&\qquad\qquad\qquad\qquad\qquad\qquad\ \ \,
     ((\meta_1',v_1') \in \semantics{\feed_1}{\environment}{\universe} 
      \; \mbox{implies} \; (\mytime{\meta_1'} \leq \mytime{\meta_1} 
            \; \mbox{or} \; \mytime{\meta_1'} > \mytime{\meta_2})) 
  \}
\\\\
%%OLD foreach update
%%{\cal F}\lsem
%%\mathtt{foreach{*}}\; x \; \mathtt{in}\; \feed_1 
%%%\semantics{\foreachupdate{x}{\feed_1}{\feed_2}}{\environment}{\universe} 
%% &=&
%% \{(\meta_2,v_2) \setalt 
%%     (\meta_1,v_1) \in \semantics{\feed_1}{\environment}{\universe} 
%%     \; \mbox{and} \; 
%%\\
%%\qquad\qquad\ \ \mathtt{create}\; \feed_2 \rsem_{{\environment} \, {\universe}}
%%&&\qquad\qquad\ \,
%%     (\meta_2,v_2) \in \semantics{\feed_2}{(\environment,x\mapsto{}v_1)}{\universe}
%%     \; \mbox{and} \;
%%%\\&&\qquad\qquad\ \,
%%     \mytime{\meta_2} > \mytime{\meta_1} 
%%  \}
%%\\\\




%% New foreach *
{\cal F}\lsem
\mathtt{foreach{*}}\; x \; \mathtt{in}\; \feed_1 
%\semantics{\foreachupdate{x}{\feed_1}{\feed_2}}{\environment}{\universe} 
 &=&
   \{(\meta_2,v_2) \setalt (\atime,(\meta_1,\meta_2)),(v_1,v_2)) \in 
       \semantics{x{:}\feed_1 * \feed_2}{\environment}{\universe} \}
%% \{(\meta_2,v_2) \setalt 
%%     (\meta_1,v_1) \in \semantics{\feed_1}{\environment}{\universe} 
%%     \; \mbox{and} \; 
\\
\qquad\qquad\ \ \mathtt{create}\; \feed_2 \rsem_{{\environment} \, {\universe}}
%% End New foreach *
\\\\

%% Old foreach **
%%{\cal F}\lsem
%%\mathtt{foreach{*}{*}}\; x \; \mathtt{in}\; \feed_1 
%%%\semantics{\foreachcreate{x}{\feed_1}{\feed_2}}{\environment}{\universe} 
%% &=&
%% \{(\meta_2,v_2) \setalt 
%%     (\meta_1,v_1) \in \semantics{\feed_1}{\environment}{\universe} 
%%     \; \mbox{and} \; 
%%\\
%%\qquad\qquad\ \ \ \mathtt{update}\; \feed_2 \rsem_{{\environment} \, {\universe}}
%%&&\qquad\qquad\ \,
%%     (\meta_2,v_2) \in \semantics{\feed_2}{(\environment,x\mapsto{}v_1)}{\universe}
%%     \; \mbox{and} \; \mytime{\meta_2} > \mytime{\meta_1} \; \mbox{and} \;
%%\\&&\qquad\qquad\ \,
%%     ((\meta_1',v_1') \in \semantics{\feed_1}{\environment}{\universe} 
%%      \; \mbox{implies} \; (\mytime{\meta_1'} \leq \mytime{\meta_1} 
%%           \; \mbox{or} \; \mytime{\meta_1'} > \mytime{\meta_2}))      
%%  \}
%%\\\\
%% End Old foreach **

%% New Foreach **
{\cal F}\lsem
\mathtt{foreach{*}{*}}\; x \; \mathtt{in}\; \feed_1 
 &=&
   \{(\meta_2,v_2) \setalt (\atime,(\meta_1,\meta_2)),(v_1,v_2)) \in 
       \semantics{x{:}\feed_1\, {**}\, \feed_2}{\environment}{\universe} \}\\
\qquad\qquad\ \ \ \mathtt{update}\; \feed_2 \rsem_{{\environment} \, {\universe}}
\\\\
%% end new foreach **

% \semantics{\ppfeed{\feed}{e}}{\environment}{\universe} 
%  &=&
% \semantics{\feed}{\environment}{
%   (\lambda x{:}\locty * \timety. \esemantics{e}{\environment} (x,\universe(x)))} 
% \\\\
% \semantics{\remapfeed{\feed}{e}}{\environment}{\universe} 
%  &=&
% \semantics{\feed}{\environment}{(\universe \circ \esemantics{e}{\environment})}
% \\\\
%% \semantics{\refeed{\feed}{e}}{\environment}{\universe} 
%%  &=&
%% \{(\atime,\some{v}) \setalt 
%%    (\atime,v) \in \semantics{\feed}{\environment}{\universe} \; \mbox{and} \;
%%    \atime \in \esemantics{e}{\environment}
%% \} \bigcup
%% \\&&
%% \{(\atime,\none) \setalt
%%    (\atime,\_) \not\in \semantics{\feed}{\environment}{\universe} \; \mbox{and} \;
%%    \atime \in \esemantics{e}{\environment}
%% \}
%% \\\\
%% \semantics{\stutterfeed{\feed}{e}}{\environment}{\universe} 
%%  &=&
%% \{(\atime,v) \setalt 
%%    (\atime,v) \in \semantics{\feed}{\environment}{\universe} \; \mbox{and} \;
%%    \atime \in \esemantics{e}{\environment}
%% \} \bigcup
%% \\&&
%% \{(\atime,v) \setalt 
%%    (\atime',v) \in \semantics{\feed}{\environment}{\universe} \; \mbox{and} \;
%%    \atime \in \esemantics{e}{\environment}  \; \mbox{and} \;
%% \\&&\qquad\qquad\qquad\ \ \,
%%     \mbox{for all $\atime''$ such that $\atime' < \atime'' \leq \atime$,} \;
%%    (\atime'',\_) \not\in \semantics{\feed}{\environment}{\universe} \; 
%% \}
\semantics{[\feed \bnfalt x \leftarrow e]}{\environment}{\universe} 
 &=&
 \{((\atime,[\meta_1,\ldots,\meta_k]),[v_1,\ldots,v_k]) \setalt 
    \exists \atime.\forall i:1\ldots k.
     (\meta_i,v_i) \in \semantics{\feed}{\environment[x\mapsto z_i]}{\universe} 
     \; \mbox{and} \; 
     \mytime{\meta_i} =\atime
  \} \\
&&\quad\mbox{where} \quad\mbox{$[z_1,\ldots,z_k] = \esemantics{e}{\environment}$}
\\
\end{array}
\]
\caption{Feed Language Semantics.}
\label{fig:semantics}
\end{figure*}



