To give a feel for our feed declaration language, we work through a
series of examples of increasing sophistication. We start by defining
a simple CoMon statistics feed in \figref{fig:simplecomon}.  This
version of the CoMon description specifies the \cd{simple\_comon} feed
using the \kw{base} feed constructor.  The \kw{sources} field
indicates that data for the feed comes from all of the locations
listed in \cd{sites}.  The \kw{schedule} field specifies that relevant
data is available from each source every minute, starting immediately.
When trying to fetch such data, the system may occasionally fail,
either because a remote machine is down or because of network
problems. To manage such errors, the schedule specifies that the
system should try to collect the data from each source for 10 seconds.
If the data does not arrive within that window, the system should give
up. Finally, the \kw{format} field indicates that the fetched data
conforms to the \padsml{}~\cite{mandelbaum+:pads-ml} descrition named
\cd{Source} defined in the file \cd{comon_format}.

\begin{figure}
\begin{code}
(* Simple CoMon distributed data description *)
let sites = 
  [
    "http://pl1.csl.utoronto.ca:3121";
    "http://plab1-c703.uibk.ac.at:3121";
    "http://planet-lab1.cs.princeton.edu:3121"
  ] 
\kw{feed} simple_comon =
  \kw{base} \{|
    \kw{sources}  = \kw{all} sites;
    \kw{schedule} = every 1 min, starting now, 
               timeout 10.0 sec; 
    \kw{format}   = Comon_format.Source;  
  |\}
\end{code}
\vskip -2ex
\caption{Simple CoMon feed that fetches data from all sites every minute.}
\label{fig:simplecomon}
\end{figure}

In contrast to the \cd{simple\_comon} feed, which returns data from
all sites, the \cd{comon\_1} feed defined in \figref{fig:comon_1}
returns only a single value per time slice: that of the first site to
supply a complete set of data.
%
\begin{figure}
\begin{code}
\kw{feed} comon_1 =
  \kw{base} \{|
    \kw{sources}  = \kw{any} sites;
    \kw{schedule} = every 1 min, lasting 2 hours;
    \kw{format}   = Comon_format.Source;
  |\}
\end{code}
\vskip -2ex
\caption{CoMon feed fragment that fetches data from a single source.}
\label{fig:comon_1}
\end{figure}
%
This feature is particularly useful when monitoring the
behavior of replicated systems, such as those using
state machine replication, consensus protocols, or even
loosely-coupled ones such as Distributed Hash Tables (DHTs).
In these systems, the same data will be available from any
of the functioning nodes, so receiving results from the first
available node is sufficient. These kinds of monitoring systems
are useful in the face of partial network unreachability or
machine failure. Specifying this behavior at the language level
provides a simpler implementation than network-centric approaches such
as anycast. 

The schedule for \cd{comon\_1} indicates the system should fetch data
every minute for two hours.  It omits the \cd{starting} and
\cd{timeout} specifications, in which case the system uses default
settings of \cd{now} for the start time and 30 seconds for the
timeout.  

The \cd{simple\_comon} example hard-codes the set of locations from
which to gather performance data.  In reality, the CoMon system has an
internet-addressable configuration file that contains a list of hosts
to be queried, one per non-comment line. \figref{fig:feedcomon}
uses this configuration information to define the \cd{comon} feed as a
\textit{dependent} feed: dependent upon the auxiliary feed \cd{nodes}
that fetches data conforming to the \padsml{} description
\cd{Nodelist.Source} from \cd{config\_locations} every two minutes.   


\begin{figure}
\begin{code}
(* CoMon distributed data description *)

(* Helper values and functions *)
\kw{let} config_locations = 
  ["http://summer.cs.princeton.edu/status/ \\
    tabulator.cgi?table=slices/ \\
    table_princeton_comon&format=nameonly"]

\kw{let} makeURL (Nodelist.Data x) = 
     "http://" ^ x ^ ":3121"

\kw{let} old_locs = ref []
\kw{let} current list_opt =
  \kw{match} list_opt \kw{with}
    Some l ->  old_locs := l; l
  | None   -> !old_locs

(* Feed of nodes to query *)
\kw{feed} nodes =  
  \kw{base} \{|
    \kw{sources}  = \kw{all} config_locations;
    \kw{schedule} = every 2 min, starting now, 
               lasting 2 hour; 
    \kw{format}   = Nodelist.Source;
  |\}

(* Dependent CoMon feed of node statistics *)
\kw{feed} comon =
  \kw{foreach} nodelist \kw{in} nodes 
  \kw{create}
    \kw{base} \{|
      \kw{sources}  = \kw{all} (List.map makeURL 
                     (List.filter Nodelist.isNode 
                     (current nodelist)));
      \kw{schedule} = once, timeout 60.0 sec; 
      \kw{format}   = Comon_format.Source;
    |\}
\end{code}
\vskip -2ex
\caption{Comon program that uses a feed of node locations to drive
  data collection.}
\label{fig:feedcomon}
\end{figure}

