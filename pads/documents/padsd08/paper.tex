%\documentclass[fleqn]{article}
\documentclass[nocopyrightspace]{sigplanconf}

\usepackage{xspace,amsmath,math-cmds,
            math-envs,inference-rules,times,
            verbatim,alltt,multicol,proof,url}
\usepackage{epsfig}
\usepackage{code} 
%\setlength{\oddsidemargin}{0in}
%\setlength{\evensidemargin}{0in}
%\setlength{\textwidth}{6.5in}
%\setlength{\textheight}{8.5in}

\begin{document}
\title{Language Support for Processing Distributed Ad Hoc Data}

\authorinfo{
Daniel S. Dantas$^1$ \quad
Kathleen Fisher$^2$ \quad
Limin Jia$^1$ \\
Yitzhak Mandelbaum$^2$ \quad
David Walker$^1$ \quad
Kenny Q. Zhu$^1$
}{
$^1$ Princeton University \\
$^2$ AT\&T Labs Research
}{}

%% \authorinfo{Kathleen Fisher}{
%% 	   AT\&T Labs Research}
%%        {\mono{kfisher@research.att.com}}
%% \authorinfo{Yitzhak Mandelbaum}{
%% 	   AT\&T Labs Research}
%%        {\mono{yitzhakm@research.att.com}}
%% \authorinfo{David Walker}{
%% 	   Princeton University}
%%        {\mono{dpw@CS.Princeton.EDU}}
%% \authorinfo{Kenny Q. Zhu}{
%%            Princeton University}
%%        {\mono{kzhu@CS.Princeton.EDU}}

\newcommand{\cut}[1]{}
\newcommand{\reminder}[1]{{\it #1 }}
\newcommand{\edcom}[1]{\textbf{{#1}}}
\newcommand{\poplversion}[1]{#1}
\newcommand{\trversion}[1]{}

\newcommand{\appref}[1]{Appendix~\ref{#1}}
\newcommand{\secref}[1]{Section~\ref{#1}}
\newcommand{\tblref}[1]{Table~\ref{#1}}
\newcommand{\figref}[1]{Figure~\ref{#1}}
\newcommand{\listingref}[1]{Listing~\ref{#1}}
%\newcommand{\pref}[1]{{page~\pageref{#1}}}

\newcommand{\eg}{{\em e.g.}}
\newcommand{\cf}{{\em cf.}}
\newcommand{\ie}{{\em i.e.}}
\newcommand{\etal}{{\em et al}}
\newcommand{\etc}{{\em etc.\/}}
\newcommand{\naive}{na\"{\i}ve}
\newcommand{\role}{r\^{o}le}
\newcommand{\forte}{{fort\'{e}\/}}
\newcommand{\appr}{\~{}}

\newcommand{\bftt}[1]{{\ttfamily\bfseries{}#1}}
\newcommand{\kw}[1]{\bftt{#1}}
\newcommand{\pads}{\textsc{pads}}
\newcommand{\padsc}{\textsc{pads/c}}
\newcommand{\padx}{\textsc{padx}}
\newcommand{\ipads}{\textsc{ipads}}
\newcommand{\ir}{\textsc{IR}}
\newcommand{\padsl}{\textsc{padsl}}
\newcommand{\padsml}{\textsc{pads/ml}}
%\newcommand{\padsd}{\textsc{pads/d}}
\newcommand{\learnpads}{{\textsc{learnpads}}}
\newcommand{\padsd}{\textsc{Gloves}}
\newcommand{\blt}{\textsc{blt}}
\newcommand{\ddc}{\textsc{ddc}}
\newcommand{\ddl}{\textsc{ddl}}
\newcommand{\C}{\textsc{C}}
\newcommand{\perl}{\textsc{Perl}}
\newcommand{\ml}{\textsc{ml}}
\newcommand{\smlnj}{\textsc{sml/nj}}
\newcommand{\ocaml}{\textsc{OCaml}\xspace}
\newcommand{\haskell}{\textsc{haskell}\xspace}
\newcommand{\ocamlbig}{\textsc{OCAML}\xspace}
\newcommand{\java}{\textsc{java}}
\newcommand{\xml}{\textsc{xml}}
\newcommand{\html}{\textsc{html}}
\newcommand{\xpath}{\textsc{xpath}}
\newcommand{\xquery}{\textsc{xquery}}
\newcommand{\datascript}{\textsc{datascript}}
\newcommand{\packettypes}{\textsc{packettypes}}
\newcommand{\erlang}{\textsc{Erlang}}
\newcommand{\camlp}{\cd{Camlp4}}
\newcommand{\ocamlnet}{\cd{Ocamlnet} \cd{2}}

\newcommand{\totalcost}[2]{\textsc{Cost}(#1,#2)}
\newcommand{\costdescription}[1]{\textsc{CT}(#1)}
\newcommand{\normcostdescription}{\textsc{NCT}}
\newcommand{\costdata}[2]{\textsc{CD}(#2 \; | \; #1)}
\newcommand{\acostdata}[2]{\textsc{ACD}(#2 \; | \; #1)}
\newcommand{\adc}[2]{\textsc{CD'}(#2 \; | \; #1)}
\newcommand{\cardt}{\textsc{Card}}
\newcommand{\costvar}[1]{\textsc{CV}(#1)}
\newcommand{\costchar}[1]{\textsc{CA}(#1)}
\newcommand{\coststring}[1]{\textsc{CS}(#1)}
\newcommand{\costint}[1]{\textsc{CI}(#1)}
\newcommand{\costparam}[1]{\textsc{CP}(#1)}
\newcommand{\costconst}[1]{\textsc{CC}(#1)}

\newcommand{\dibbler}{Sirius}
\newcommand{\ningaui}{Altair}
\newcommand{\darkstar}{Regulus}

\newcommand{\vizGems}{Arrakis}

\newcommand{\comon}{CoMon\xspace}
\newcommand{\planetlab}{PlanetLab\xspace}
\newcommand{\monall}{Monall\xspace}
%% \newcommand{}{}


%% \newcommand{\IParray}[4]{{\tt Parray} \; #1 \; \[#2, #3, #4\]}

\newcommand{\figHeight}[4]{\begin{figure}[tb]
	\centerline{
	            \epsfig{file=#1,height=#4}}
	\caption{#2}
	\label{#3}
	\end{figure}}

\newcommand{\myalt}{\ensuremath{\; | \;}}
\newcommand{\normal}[1]{\ensuremath{\bar{#1}}}
\newcommand{\relativee}[2]{\ensuremath{{\cal R}(#1 \; || \; #2)}}
\newcommand{\srelativee}[2]{\ensuremath{{\cal S}(#1 \; || \; #2)}}
\newcommand{\addh}[2]{\ensuremath{#1 \oplus #2}}

\newcommand{\irstruct}[1]{{\tt struct}\{#1\}}
\newcommand{\irunion}[1]{{\tt union}\{#1\}}
\newcommand{\irenum}[1]{{\tt enum}\{#1\}}
\newcommand{\irarray}[1]{{\tt array}\{#1\}}
\newcommand{\irarrayFW}[2]{{\tt arrayFW}\{#1\}[#2]}
\newcommand{\irswitch}[2]{{\tt switch}(#1)\{#2\}}
\newcommand{\iroption}[1]{{\tt option}\{#1\}}
\newcommand{\setof}[1]{\lsem #1 \rsem}
\newcommand{\goto}{\Rightarrow}
\newcommand{\Pvoid}{{\tt Pvoid}}
\newcommand{\Pempty}{{\tt Pempty}}
\newcommand{\sskip}{\hspace*{5mm}}
\newcommand{\shrink}{\vspace*{-4mm}}

% Semantics
\newcommand{\setalt}{{\; | \;}}
\newcommand{\denote}[1]{\lsem #1 \rsem}
\newcommand{\lsem}{{[\![}}
\newcommand{\rsem}{{]\!]}}
\newcommand{\turn}{\vdash}
\newcommand{\meta}{m}
\newcommand{\nested}{n}
\newcommand{\mytime}[1]{#1.t}
\newcommand{\myds}[1]{#1.ds}
\newcommand{\myval}[1]{#1.nest}
\newcommand{\generatedloc}{\ensuremath{\mathtt{nowhere}}}
\newcommand{\environment}{E}
\newcommand{\universe}{U}
\newcommand{\selectOne}{\ensuremath{\mathsf{earliest}}}
% core feed semantics
\newcommand{\csemantics}[3]{{\cal C}\lsem #1 \rsem_{{#2} \, {#3}}}
% feed semantics
\newcommand{\semantics}[3]{{\cal F}\lsem #1 \rsem_{{#2} \, {#3}}}
% expression semantics
\newcommand{\esemantics}[2]{{\cal E}\lsem #1 \rsem_{{#2}}}
%\newcommand{\esemantics}[2]{#2(#1)}

% Host language types
\newcommand{\ty}{\ensuremath{\tau}}
\newcommand{\basety}{\ensuremath{b}}
\newcommand{\arrow}{\rightarrow}
\newcommand{\optionty}[1]{\ensuremath{#1 \; \mathsf{option}}}
\newcommand{\listty}[1]{\ensuremath{#1 \; \mathsf{list}}}
\newcommand{\setty}[1]{\ensuremath{#1 \; \mathsf{set}}}
\newcommand{\feedty}[1]{\ensuremath{#1 \; \mathsf{feed}}}
\newcommand{\corety}[1]{\ensuremath{#1 \; \mathsf{core}}}
\newcommand{\schedulety}{\ensuremath{\mathsf{sched}}}
\newcommand{\timety}{\ensuremath{\mathsf{time}}}
\newcommand{\locty}{\ensuremath{\mathsf{loc}}}
\newcommand{\boolty}{\ensuremath{\mathsf{bool}}}
\newcommand{\unitty}{\ensuremath{\mathsf{unit}}}
\newcommand{\stringty}{\ensuremath{\mathsf{string}}}
\newcommand{\metatype}[1]{\ensuremath{\mathsf{meta}(#1)}}
\newcommand{\nestedtype}[1]{\ensuremath{\mathsf{nest}(#1)}}
\newcommand{\dsty}{\ensuremath{\mathsf{ds}}}

\newcommand{\dom}{\ensuremath{\mathsf{dom}}}
\newcommand{\ueq}[3]{\ensuremath{#1 =_{#2} #3}}
\newcommand{\fsubset}[3]{\ensuremath{#1 \subseteq_{#2} #3}}
\newcommand{\feq}[3]{\ensuremath{#1 =_{#2} #3}}

% Expressions
\newcommand{\expression}{e}
\newcommand{\constant}{c}
\newcommand{\ds}{\ensuremath{ds}}
\newcommand{\boolf}{\ensuremath{\mathtt{false}}}
\newcommand{\boolt}{\ensuremath{\mathtt{true}}}
\newcommand{\loc}{\ensuremath{\ell}}
\newcommand{\feed}{\ensuremath{F}}
\newcommand{\corefeed}{\ensuremath{C}}
\newcommand{\generalvar}{\ensuremath{x}}
\newcommand{\feedvar}{\ensuremath{x}}
\newcommand{\itemvar}{\ensuremath{x}}
\newcommand{\data}{\ensuremath{v}}
\newcommand{\atime}{\ensuremath{t}}
\newcommand{\astring}{\ensuremath{w}}
\newcommand{\unit}{\ensuremath{()}}
\newcommand{\schedule}{\ensuremath{s}}
\newcommand{\parser}{\ensuremath{p}}
\newcommand{\none}{\ensuremath{\mathtt{None}}}
\newcommand{\some}[1]{\ensuremath{\mathtt{Some}\; #1}}
\newcommand{\inl}[1]{\ensuremath{\mathtt{inl}\; #1}}
\newcommand{\inr}[1]{\ensuremath{\mathtt{inr}\; #1}}
\newcommand{\casedata}[2]{{\tt switch}(#1)\{#2\}}
%\newcommand{\nillist}{\ensuremath{\mathtt{nil}}}
\newcommand{\nillist}{\ensuremath{[\,]}}
%\newcommand{\conslist}[2]{\ensuremath{\mathtt{cons} (#1,#2)}}
\newcommand{\conslist}[2]{\ensuremath{[#1,\ldots,#2]}}
\newcommand{\nilstream}{\ensuremath{\mathtt{done}}}
\newcommand{\consstream}[2]{\ensuremath{\mathtt{next} (#1,#2)}}


% Feeds
\newcommand{\comprehensionfeed}[3]{\ensuremath{\mathtt{\{|} #1 \; \mathtt{|}\; #2 \leftarrow #3 \mathtt{|\}}}}
\newcommand{\computed}[3]{\ensuremath{\mathtt{[} #1 \; \mathtt{|}\; #2 \in #3 \mathtt{]}}}
\newcommand{\letfeed}[3]{\ensuremath{\mathtt{let}\; #1 \; \mathtt{=}\; #2 \; \mathtt{in} \; #3}}
\newcommand{\allfeed}[5]{
  \ensuremath{
    \mathtt{all \{ format=} #1; 
    \mathtt{src=} #2;
    \mathtt{sched=} #3;
    \mathtt{pp=} #4;
    \mathtt{win=} #5;
  \mathtt{\}}}}
\newcommand{\existsfeed}[5]{
  \ensuremath{
    \mathtt{any \{ format=} #1; 
    \mathtt{src=} #2;
    \mathtt{sched=} #3;
    \mathtt{pp=} #4;
    \mathtt{win=} #5;
  \mathtt{\}}}}
\newcommand{\filterfeed}[2]{
  \ensuremath{
    \mathtt{filter} \; #1 \; \mathtt{with}\; #2}}
\newcommand{\remapfeed}[2]{
  \ensuremath{
    \mathtt{redirect} \; #1 \; \mathtt{with}\; #2}}
\newcommand{\ppfeed}[2]{
  \ensuremath{
    \mathtt{pp} \; #1 \; \mathtt{with}\; #2}}
\newcommand{\foreachupdate}[3]{
  \ensuremath{
    \mathtt{foreach{*}{*}}\; #1 \;
    \mathtt{in}\; #2 \;
    \mathtt{update}\; #3}}
\newcommand{\foreachcreate}[3]{
  \ensuremath{
    \mathtt{foreach*}\; #1 \;
    \mathtt{in}\; #2 \;
    \mathtt{create}\; #3}}
\newcommand{\remap}[2]{\ensuremath{\mathtt{redirect}\; #1 \; \mathtt{with} \; #2}}
\newcommand{\stutterfeed}[2]{\ensuremath{\mathtt{stutter}\; #1 \; \mathtt{on} \; #2}}
\newcommand{\refeed}[2]{\ensuremath{\mathtt{reschedule}\; #1 \; \mathtt{to} \; #2}}
\newcommand{\emptyfeed}{\ensuremath{\emptyset}}
\newcommand{\onefeed}[2]{\ensuremath{\mathtt{One}}(#1,#2)}
\newcommand{\sfeed}[1]{\ensuremath{\mathtt{SchedF}}(#1)}
\newcommand{\lfeed}[1]{\ensuremath{\mathtt{ListF}}(#1)}
\newcommand{\unionfeed}{\ensuremath{\cup}}
\newcommand{\sumfeed}{\ensuremath{+}}
\newcommand{\spairfeed}{\; \ensuremath{\mathtt{\&} \; }}
\newcommand{\allpairfeed}{\; \ensuremath{{*}{*}} \; }

\newcommand{\Time}{\ensuremath{\mathtt{Time}}}
\newcommand{\Set}{\ensuremath{\mathtt{Set}}}

% this is used for the translations equal
\newcommand{\transeq}{\stackrel{def}{=} }
\newcommand{\ai}{{\tt wl}}



% BNF
%\newcommand{\bnfalt}{\ |\ }


\maketitle{}

\begin{abstract}  
Many applications use the file system as a simple persistent data
store.  Although this approach is expedient, imposing almost no
overhead, it is not robust because in general, the overall correctness
of the application will depend on the collection of files,
directories, and symbolic links in the file system having some precise
hierarchical organization and metadata such as file ownership,
permissions, and timestamps but current programming languages do not
provide support for documenting assumptions about the file system. In
addition, actually loading the data from the disk requires writing a
lot of distracting boilerplate code.

This paper describes \forest{}, a new domain-specific language for
describing directory structures embedded in \haskell{}. \forest{}
descriptions use a type-based metaphor to specify portions of the file
system in a simple, declarative manner.  \forest{} makes it easy to
connect data on the disk to an isomorphic representation in memory
that can be manipulated by programmers as if it were any other
strongly-typed data structure in their program.  \forest{} also
generates metadata that can be used to verify that a given portion of
the file system conforms to its specification.  It greatly lowers the
divide between on-disk and in-memory representations of data.

We present our design for \forest{} and describe an implementation of
a full working prototype in \haskell{}. From a single compact
description, the \forest{} implementation generates a useful
collection of \haskell{} types and functions for manipulating,
checking, and analyzing file system data.   In addition, \forest{}
generates type class definitions that make it possible to
exploit powerful
generic programming paradigms that
allow third-party developers to build tools for querying,
visualizing, and debugging on-disk data in a generic way. We
present examples illustrating the use of \forest{} on a number of
real-world directory structures and programming tasks, including
drop-in replacements for a number of standard shell tools. Finally, we
formalize the core elements of the language as a simple calculus based
on classical tree logics.
\end{abstract}


\section {Introduction}
\label{sec:intro}

An {\em ad hoc data source} is any semistructured data source for
which useful data analysis and transformation tools are not readily
available.  The data that constitutes a single, abstract source 
often comes from many different concrete, physical destinations
distributed across the Internet.  It may also become available
over a range of times and in several, evolving formats.
Before the users can extract the information they need from the data,
it usually must be fetched, archived locally for historical analysis, 
compressed, encrypted, and monitored for errors or deviations from the norm.

There are all kinds of users of such data ranging from scientists to
system administrators to financial analysts.  To get more of a flavour
of one sort of distributed ad hoc data source, consider the data
manipulated by CoMon~\cite{comon}, a system designed to monitor the
health, performance and security of PlanetLab~\cite{planetlab}.  Every
five minutes, CoMon attempts to contact each of 842 PlanetLab nodes
across 416 sites worldwide.\footnote{Data as of writing this manuscript;
PlanetLab membership varies over time.}  
When all is running smoothly, which it
never, ever is, each node responds by sending back a data file
consisting of a number of valuable bits of information ranging from
the kernel version to the uptime to the memory usage to the id of the
user with the greatest CPU utilization.  CoMon archives this data in
compressed form and its backend processes the information for display
to PlanetLab users.  It is an invaluable resource for Planet users
who need to monitor the health and performance of their applications
or experiments.  Almost all distributed systems have (or should
have) similar sorts of monitoring infrastructure.  Unfortunately,
system implementers are often left
to hack ``one-off'' monitoring tools of their own, which are
invariably less reliable, unoptimized, insecure, and difficult or
impossible to evolve when new requirements become known.
A substantial part of the difficulty simply comes from the diversity, quality,
and volume of data these systems must often handle. Often, new
monitoring systems also face the problem of having to interact with legacy
devices, legacy software and legacy data, leaving implementers in a
situation where they cannot use robust off-the-shelf data management
tools built for standard formats like XML.

Somewhat similar problems also appear
across the natural and social sciences, including biology,
physics and economics.  For example, systems such as BioPixie~\cite{biopixie}, Grifn~\cite{grifn} and Golem~\cite{golem}, built by
computational biologists at Princeton,
routinely obtain data supplied from a number of sources scattered 
across the net.  The data is often archived and later analyzed or mined 
for valuable information about gene structure and regulation.  Likewise, 
cosmologists need access to data uploaded from major telescopes~\cite{sdss}
and economists can make use of vast data repositories at FedStats.org
amongst other sources.  These and other selected ad hoc data sources 
are presented in Figure~\ref{fig:exampledata}.

\begin{figure*}
\begin{center}
\begin{tabular}{|l|l|l|}
\hline\hline
Name & Use & Properties 
\\\hline\hline
CoMon~\cite{comon} & PlanetLab host monitoring & Multiple data sets in mail-header formats\\
                                       && Archiving every 5 minutes \\
                                       && From evolving set of ~800 nodes \\\hline
CoBlitz~\cite{coblitz} & File transfer system monitoring & Multiple data sets \\
                                       && Archiving every 5 minutes \\
                                       && From evolving set of ~800 nodes \\\hline
CoralCDN~\cite{coral} & Log files from CDN monitoring & Single Format \\
                                       && Periodic archiving \\
                                       && From evolving set of ~250 hosts \\\hline
\vizGems{}       & Website Host Monitoring & Many and varied machines \\
                 &                         & Execute programs remotely\\
                 &                         & to collect data\\\hline
\darkstar{}      & AT\&T network monitoring & Archiving for future analysis \\\hline
\ningaui{}       & AT\&T billing auditing   & Thousands of data sources\\
                 &                          & Archiving and error analysis\\\hline
GO DB (Gene Ontology)~\cite{geneontology} & Gene Function Information & Multiple Formats \\
                                             && Uploaded in daily, weekly, monthly intervals \\\hline
BioGrid~\cite{biogrid} & Curated Gene and Protein Data & XML and Tab-separated Formats \\
          & & multiple data sets $<=$ 50MB each \\
          & & monthly data releases \\\hline
NCBI~\cite{ncbi} & National Center for Biotechnology Information & Links to multiple bioinformatics datasets \\
                                                     && and online databases\\
\hline\hline
\end{tabular}
\end{center}
\caption{Example ad hoc data sources}
\label{fig:exampledata}
\end{figure*}

\section{Running Examples}
\label{sec:examples}
In this section, we describe two examples that we will use throughout
the paper to motivate and explain our system.

At Princeton, Vivek Pai and KyoungSoo Park have developed
CoMon~\cite{comon}, a system for monitoring the health and status of
PlanetLab~\cite{planetlab}.  Every five minutes, CoMon attempts to
fetch data from each of PlanetLab's 800+ nodes.  This data ranges from
the node uptime to memory usage to kernel version.  
%The CoMon system takes this
%raw data and transforms it into two different forms, one of which is a
%per-node collection of statistics and the other of which is a
%per-slice ({\em i.e.,} per-application) collection of statistics.
CoMon displays the data to users in tabular form and allows them to
perform a number of simple queries to find, for instance, lightly
loaded nodes, nodes with drifting clocks or nodes with little
remaining disk space.  CoMon also monitors nodes for various different
sorts of problems and alerts users of deviant machines or slices.
Finally, the data is archived so PlanetLab users can perform their own
custom analyses of historical data.

AT\&T provides a web hosting service.  The infrastructure for this
service includes a variety of hardware components including routers,
firewalls, load balanacing machines, actual web servers, and
databases, replicated and geographically distributed.  Hence, a given
web site may be distributed across a variety of machines running a
variety of operating systems in a variety of locations.  When a
customer signs up for AT\&T's hosting service, part of the contract
specifies what kinds of monitoring AT\&T will provide for the site.
The \vizGems{} infrastructure provides this monitoring
service.  It tracks a variety of resources using a wide array of
measures, including network
bandwidth, packet loss, cpu utilization, disk utilization, memory
usage, load averages, \etc{} For each machine in the hosting service
and for each such resource, the monitoring system archives the values at
regular intervals and issues alerts when the values exceed resource-
and contract-specific levels.  The archive is used to track long-term
behavior of the service, allowing engineers to determine when more
resources need to be provisioned, for example, adding additional cpus,
memory, or disk space.  It also allows engineers to understand the
``normal'' behavior for a particular site such as daily or seasonal
cycles for a particular site.


%% Notes on the visgems example.
%smaug:/fs/swift/proj/vg/4yitzhak
% inventory file:
% labems-test-inv.txt
%    for each asset, defines its type: linux, ip address, password
%    url1, url2, ip,
%    systype(url,url-win,win32.i386,vmware, solaris.sun4,linux.i386,cisco,cisco3750, alteonsw, alteon,...), 
%    user, password, 
%    snmpcommunity(public,CompuLert,monitor, MT1HostingMgmt!,R1cd4Win+g1A, private), 
%    sysfunc(client,ems), 
%    servicelevel(os,man,mon,soss), 
%    need_tags(eastcoast), nets(ip/port), weight (1000,300),
%    ticketmodel(keep)
%    realid(esxhost-122)
%    scopeinv_port22(22), scopeinv_port443(443), scopeinv_port80(80)
%    implappend_protSNMP(version=1)
% class file:
% parameter.txt
%    for each asset, what type of info to collect and how
%    including what kind of scope machine (windows, linux) to use
%    bindings from inventory file are in scope in single brackets
%    what are double brackets: [[scopeinv_cpu]]?
%       the single brackets mean if there's an inventory entry with
%       that key, find it and replace the thing in brackets with the
%       value. if there's no entry, abort processing that metric
%       rule. the double brackets are similar except that the thing in
%       brackets is assumed to be a prefix. so in the above, the tool
%       searches the inventory for entries with key == scopeinv_cpu*
%       and for each one found, it generates a metric collection entry
%       in the schedule. this is how monitoring of multiple
%       filesystems, or multiple cpus is implemented. the scopes query
%       the assets and collect info about filesystems, and cpus which
%       are sent back to the main server that adds them to the
%       inventory. 
%          scopeinv_cpu
%          scopeinv_fs
%          scopeinv_iface
%          scopeinv_port

%    what are counts: count=10, count=5?
%       these are collection type specific. for example, in PING
%       rules, it means send 5 packets.  in calls to vmstat / mpstat /
%       etc, means collect 5 samples. 


%    what are inst parameters (inst=_total), etc
%       inst goes with the 'var' attribute: var=cpu_used and inst=0
%       would collect data for cpu usage on cpu #0 and return it as
%       metric: cpu_used.0 

%    what are labels used for?
%      CPU Used ([[scopeinv_cpu!All]])
%      CPU System
%      CPU User
%      Number of Threads
%      Pages In
%      Pages Out
%      Run Queue
%      Swap In
%      Swap Out
%      Used Memory
%      they are used for tools like WMI where it's simpler to override
%      the label of the returned stats instead of generating them on
%      the scope. 


%    what are val fields
%       val=* */%v *
%       collection specific, in this case it's a regular expression
%       that means the value is the text after a '/' and before a
%       space. 

%    What are file fields
%       file=loadavg
%       file=vmstat
%       file=stat
%       collection specific, in this case it tells the tool to look for /proc/loadavg etc

%    what are exclude fields?
%       exclude=*:top
%       collection specific, in this case it tells the top tool to not
%       include itself in the top process discovery. 


%    pipe separated
%    servicelevel: man, os, soss, mon, colo
%       monitor fewer things for less expensive levels of support
%    asset machine type: linux.i386
%    scope machine type: linux.i386
%    collected info
%       ping_loss (_main)
%       ping_time (_main)
%       cpu_free
%       cpu_sys
%       cpu_used
%       cpu_usr
%       cpu_wait
%       fs_used
%       memory_free (_total)
%       memory_total (_total)
%       memory_used (_total)
%       os_loadavg (_main)
%       os_nproc (_total)
%       os_nthread (_total)
%       os_nuser (_total)
%       os_pagein (_total)
%       os_pageout (_total)
%       os_runqueue (_total)
%       os_swapin (_total)
%       os_swapout (_total)
%       proc_topcpu (1)
%       swap_free (_total)
%       swap_total (_total)
%       swap_used (_total)
%       tcpip_inpkt
%       tcpip_outpkt
%       tcpip_inerrpkt
%       tcpip_outerrpkt
%       url_avail (_main)
%       url_time (_main)
%       port_avail
%       port_time
%       log.hardware
%       log.console
%       log.application
%       log.system
%       host_cpuused,....
%       pool_cpumax,...
%       guest_numvcpu,...
%       collection

%       any instance starting with '_' is meant to be special, as in
%       'overall' or 'main' metric instance. so you may have
%       cpu_used.0, cpu_used.1, ..., for each cpu and also
%       cpu_used._total that is the average of the individual ones.


%    y/n
%       the y/n is a boolean that says to report or not report the
%       stat value back to the server. you'd set it to 'n' when the
%       metric is't important, but you either need it to generate
%       another metric (using the CALC methods), or to generate an
%       alarm. for example, for network interfaces, we don't really
%       care to chart the in/out errors and discards since they are
%       usually 0. but we still monitor them and when errors do occur
%       we create an alarm. 

%    command: 
%      what is distinction between raw, cooked, and embedded?
%         - raw means run a simple command and return the output,
%         e.g. collect SNMP oid .a.b.c.d and return its value.
%         - cooked means runs a more elaborate tool that interacts with
%         the remote side. for example, most SSH collections are like
%         that because they run either multiple commands or need to
%         parse the results and perform calculations. 
%         - embedded is similar to cooked except that the remote end
%         is assumed to not be a full POSIX shell environment, so the
%         mechanism for collection needs to be a little
%         different. this happens for network switches that support a
%         limited shell type environment. 

%      PING:..., 
%        loss, time
%      SSH:...
%        mpstat, df, free, uptime, top, proc, uptime, netstat, sar,
%        swap, ibmhmc, vmwarei, vmwarevires
%      CALC:...
%        [[!scopeinv_cpu]]
%      PORT:...
%      URL:
%         url=
%      WMI:
%      NOOP
%      SNMP:
%         community
%         version
%         var, label, unit, helper, unique
%    units:
%        %,ms,GB,<empty>, pkts, mbps
%    number, counter
%    alarm spec:  >=100:1:2/2:1/3600:CLEAR:5:2/2
%       <vrange>:<severity>:<m hits/n collections>:<alarm refresh count/time>
%     or
%       CLEAR:<severity>:<m clears/n collections>
%     vrange can be >= v, <= v, [v1,v2], (v1,v2) (inclusive / exclusive intervals)
%     1/3600 means resend this alarm once every 3600 secs, e.g. 1hr.
%     so the above means: alarm if the value is >= 100 for 2 consecutive intervals,
%     refresh the alarm every hour while the condition persists, and clear the alarm
%     if you get 2 intervals < 100.



% scopemgr script assigns a scope based on inventory and class files
% and generates a schedule for the asset.  
% schedules are grouped by customer and scope
%  asset schedule file:
%  labems-test-sched.scope3.txt
% all schedules for a scope are concatentated it single schedule for
% scope

% vg_collector invoked with segment of scheduler for given asset
% examples:
%  ssh-schedule.txt
%  snmp-schedule.txt
% "XML code between <cfg> tags
% "vars" table with variables to collect and how to do it
% "alarms" table with threshold limits for these variables.

% scopes come in linux and windows flavors because hard to monitor
% windows machines from non-windows machines.

% scopes are assigned based on network reachablity and grouping by
% tags:
%  a scope will be assigned to an asset if their tag sets intersect
%  assignment also considers "cost" which accounts for bandwidth and
%  load issues.

% if a scope fails, tasks are reassigned to other scopes until it
% comes back on-line


\section{\padsd{}: An Informal Introduction}
\label{sec:informal}
The \padsd{} language allows users to describe streams of data and
meta-data that we refer to as {\em feeds}.  
To introduce the central features of the language,
we work through a series of examples 
drawn from the CoMon and \vizGems{} monitoring systems.

\begin{figure}[t]
\begin{code}
\kw{let} sites = 
  [
    "http://pl1.csl.utoronto.ca:3121";
    "http://plab1-c703.uibk.ac.at:3121";
    "http://planet-lab1.cs.princeton.edu:3121"
  ] 
\kw{feed} simple_comon =
  \kw{base} \{|
    \kw{sources}  = \kw{all} sites;
    \kw{schedule} = every 5 min, starting now, 
               timeout 60.0 sec; 
    \kw{format}   = Comon_format.Source;  
  |\}
\end{code}
\vskip -1.5ex
\caption{Simple CoMon feed (\texttt{simple\_comon.fml}).}
\label{fig:simplecomon}
\end{figure}

\begin{figure}[t]
\begin{code}
\kw{feed} comon_1 =
  \kw{base} \{|
    \kw{sources}  = \kw{any} sites;
    \kw{schedule} = every 5 min, lasting 2 hours;
    \kw{format}   = Comon_format.Source;
  |\}
\end{code}
\vskip -2ex
\caption{Description fragment for data from one of many sites (\texttt{sites.fml}).}
\label{fig:comon_1}
\end{figure}


\subsection{CoMon Feeds}
\figref{fig:simplecomon} presents our first attempt to define
a simple CoMon statistics feed.  This description
specifies the \cd{simple\_comon} feed
using the \kw{base} feed constructor.  The \kw{sources} field
indicates that data for the feed comes from \kw{all} of the locations
listed in \cd{sites}.  The \kw{schedule} field specifies that relevant
data is available from each source every five minutes, starting immediately.
When trying to fetch such data, the system may occasionally fail,
either because a remote machine is down or because of network
problems. To manage such errors, the schedule specifies that the
system should try to collect the data from each source for 60 seconds.
If the data does not arrive within that window, the system should give
up. 

The last field in a base feed constructor is the \kw{format} field,
which specifies the syntax of the fetched
data by supplying a parser for it.  In this case, \cd{Comon_format.Source} is 
actually a parser generated from another specification file 
(\cd{comon_format.pml}, which we have omitted because of space
constraints) written
in \padsml{}, a parser generator developed in earlier 
work~\cite{mandelbaum+:pads-ml}.  While it is not strictly necessary for
\padsd{} programmers to use \padsml{} specifications in their descriptions,
and the key ideas in this paper can be understood without a deep knowledge
of \padsml{}, the two languages have been designed to fit together elegantly.
Moreover, several of our generated tools exploit the common underlying 
infrastructure to perform useful data analyses and transformations over
collections of data files.  

% \padsml{}~\cite{mandelbaum+:pads-ml} description named
% \cd{Source} defined in the file \cd{comon_format}.

A simple variation of our first description is shown in \figref{fig:comon_1}.
Here, in contrast to \cd{simple\_comon}, which returns data from
{\em all} sites per time slice, \cd{comon\_1} returns data from just 
{\em one} site per time slice.  This difference between the two is
specified using the \kw{any} constructor instead of the \kw{all}.
%
%
This feature is particularly useful when monitoring the
behavior of replicated systems, such as those using
state machine replication, consensus protocols, or even
loosely-coupled ones such as Distributed Hash Tables (DHTs) 
\cite{Balakrishnan+03:dht}.
In these systems, the same data will be available from any
of the functioning nodes, so receiving results from the first
available node is sufficient. 
%These kinds of monitoring systems
%are useful in the face of partial network unreachability or
%machine failure. Specifying this behavior at the language level
%provides a simpler implementation than network-centric approaches such
%as anycast \cite{anycast}. 

The schedule for \cd{comon\_1} indicates the system should fetch data
every five minutes for two hours, using the \cd{lasting} field to indicate
the duration of the feed.  It omits the \cd{starting} and
\cd{timeout} specifications, causing the system to use default
%settings of \cd{now} for the start time and 30 seconds for the
%timeout.  
settings for the start time and the timeout window.  


\begin{figure}
\begin{code}
(* Ocaml helper values and functions *)
\kw{let} config_locations = 
  ["http://summer.cs.princeton.edu/status/ \\
    tabulator.cgi?table=slices/ \\
    table_princeton_comon&format=nameonly"]

(* Feed of nodes to query *)
\kw{feed} nodes =  
  \kw{base} \{|
    \kw{sources}  = \kw{all} config_locations;
    \kw{schedule} = every 5 min;
    \kw{format}   = Nodelist.Source;
  |\}

\kw{let} makeURL (Nodelist.Data x) = 
     "http://" ^ x ^ ":3121"

\kw{let} old_locs = ref []
\kw{let} current list_opt =
  \kw{match} list_opt \kw{with}
    Some l ->  old_locs := l; l
  | None   -> !old_locs

(* Dependent CoMon feed of node statistics *)
\kw{feed} comon =
  \kw{foreach} nodelist \kw{in} nodes 
  \kw{create}
    \kw{base} \{|
      \kw{sources}  = \kw{all} (List.map makeURL 
                     (List.filter Nodelist.is_node 
                     (current (value nodelist))));
      \kw{schedule} = once, timeout 60.0 sec; 
      \kw{format}   = Comon_format.Source;
    |\}
\end{code}
\vskip -2ex
\caption{Node location feed drives
  data collection (\texttt{comon.fml}).}
\label{fig:feedcomon}
\end{figure}



\begin{figure}
\begin{code}
\kw{ptype} nodeitem =
  Comment of '#' * pstring_SE(peor)
| Data of pstring_SE(peor)

\kw{let} is_node item = 
  \kw{match} item \kw{with}
  Data _ -> true
  | _ -> false

\kw{ptype} source = 
    nodeitem precord plist (No_sep, No_term)
\end{code}
\vskip -2ex
\caption{\texttt{nodelist.pml}: \padsml{} description for CoMon configuration 
  files, which contain one host name per non-commented line.}
\label{fig:nodepml}
\end{figure}


So far, our simple examples have hard-coded the set of locations from
which to gather performance data.  In reality, however, the CoMon system has an
Internet-addressable configuration file that contains a list of hosts
to be queried, one per non-comment line. This list is periodically
updated to reflect the set of active nodes in PlanetLab. 
\figref{fig:feedcomon} specifies a version of the \cd{comon} feed that
depends upon this configuration information.  To do so, the
description includes an auxiliary feed called \cd{nodes} that describes the
configuration information: it is available from the
\cd{config_location}, it should be fetched every five minutes, and its
format is described by the \padsml{} description \cd{source} given in
the file \cd{nodelist.pml}, which appears in \figref{fig:nodepml}.

The \padsml{} description in \figref{fig:nodepml} 
specifies that \cd{source} is a list of
new-line terminated records, each containing a \cd{nodeitem}.  In
turn, a \cd{nodeitem} is either a \cd{'#'} character followed by a
comment string, which should be tagged with the \cd{Comment}
constructor, or a host name, which should be tagged as
\cd{Data}. The description also defines a helper function \cd{is_node},
which returns true if the data item in question is a host name
rather than a comment.  Given this specification, the \cd{nodes} feed
logically yields a list of host names and comments every five minutes.
In fact, because of the possibility of errors, the feed actually
delivers a {\em list option} every five minutes: \cd{Some} if the list is
populated with data, \cd{None} if the data was unavailable at the
given time-slice.

Using the \cd{nodes} specification, we are subsequently able to define 
the \cd{comon} feed using the notation 
\kw{foreach} \cd{nodelist} \kw{in} \cd{nodes} \kw{create} \cd{...}.
In this declaration, each element of \cd{nodes} is bound to the variable
\cd{nodelist} for use in generation of the new feed declared in ``\cd{...}''
The final result of the \kw{foreach} is the union of all such newly
generated feeds.
Importantly, each element of \cd{nodes} is actually a pair of
provenance meta-data and computed data value, either of which may be used
to direct creation of the dependent feed.  In the example we are
studying here, the data component is projected from the pair
using the \cd{value} function and the meta-data is ignored.\footnote{The
meta-data may be obtain by applying the function \cd{meta} to a feed
element.}  

To complete the construction of the \cd{comon} feed,
a small amount of functional programming allows the user 
to manage errors and strip out comment fields.  Any such simple
transformations may be written directly in \ocaml{}, the host language 
into which we have embedded \padsd{}.  In particular, here,
the \cd{current} function checks if
the \cd{nodelist} value is \cd{Some l}, in which case it caches \cd{l}
before returning it as a result.  Otherwise, if the \cd{nodelist} value
is \cd{None} (indicating an error), the most recently cached list of nodes 
is used instead.  The
rest of the \kw{sources} specification filters out comment fields, and then
converts the host names to URLs with the required port using the
auxiliary function \cd{makeURL}.

% The \kw{foreach ... create} construct merges the resulting data from
% each machine into a single feed.  As
% before, the format of data fetched from each node matches the
% description \cd{Comon\_format.Source}.  

With this specification, we
expect to get data from all the active machines listed in the
configuration file every five minutes.  We further expect the system to
notices changes in the configuration file within five minutes.


The previous examples all showcased feeds that contained a single type
of data.  \padsd{} also provides a datatype mechanism that allows us
to construct compound feeds containing data of different sorts.  As an
example where such a construct is useful, the CoMon system includes a
number of administrative data sources.  One example is a collection of
node profiles, collecting the domain name, IP address, physical
location, \etc, for each node in the cluster.  A second example is a
list of authentication information for logging into the machines.
These two data sources have different formats, locations, and update
schedules, but system administrators want to keep a combined archive
of the administrative information present in these sources.  If
\cd{sites\_mime} is a feed description of the profile
information and \cd{sites\_keyscan\_mime} is a feed of authentication
information, then the declaration
%
\begin{code}
\kw{feed} sites = 
    Locale \kw{of} sites_mime
  | Keyscan \kw{of} sites_keyscan_mime
\end{code}
%
creates a feed with elements drawn from each of the two 
feeds.  The constructors \cd{Locale} and \cd{Keyscan} tag each item in
the compound feed to indicate its source. 

% \subsection{\vizGems{} Example}
% We now shift to an example drawn from AT\&T's \vizGems{} project.  Like
% the earlier CoMon example, the \cd{stats} feed in \figref{fig:pulse}
% monitors a collection of machines described in a configuration file.
% The \cd{hostList} description has the same form as the \cd{nodes} feed
% we saw earlier, except it draws the data from a local file and only
% once a day.  Unlike the earlier example, we use a \textit{feed
% comprehension} to clean up this feed before creating the \cd{stats} feed.
% The comprehension filters
% the list of hosts to remove comments using the \cd{is_node} function
% and we use the built-in \cd{flatten} function to convert the feed of
% lists of hosts into a simple feed of hosts.  We use the \cd{mk\_host}
% function to remove the \cd{Data} constructor around the hostnames to
% simplify down-stream processing.

% The \cd{stats} feed depends upon the \cd{hosts} feed.  For each host
% \cd{h} in \cd{hosts}, it creates a schedule consisting of a single
% time corresponding to ``now'' with a timeout of one minute.  It uses
% this schedule \cd{s} to describe a compound feed, which pairs two base
% feeds: the first uses the Unix command \cd{ping} to collect network
% statistics about the remote machine while the second performs a remote
% shell invocation using \cd{ssh} to find out statistics about how long
% the machine has been up.  Both of these feeds use the \kw{proc}
% constructor in the \kw{sources} field to compute the data on the fly,
% rather than reading it from a file.  The argument to \kw{proc} is a
% string that is executed in a freshly constructed shell.  The pairing
% constructor for feeds takes a pair of feeds and returns a feed of
% pairs, with elements sharing the same scheduled fetch time being
% paired. This semantics is very convenient in this case, as it produces
% a feed that for each host returns a pair of its ping and uptime
% statistics, grouping together the information for each host. 
% Of course, the full \vizGems{} monitoring application
% uses many more tools than just ping and uptime to probe the remote
% machine; the corresponding feed description has many more branches than this simplified version.


% \begin{figure}
% \begin{code}
% \kw{let} config_locations = 
%     [("file:///arrakis/config/machine_list")];

% \kw{feed} hostList =  
%   \kw{base} \{|
%    \kw{sources}  = \kw{all} config_location;
%    \kw{schedule} = every 1 day;
%    \kw{format}   = Nodelist.Source;
%  |\}

% \kw{let} mk_host (Nodelist.Data h) = h

% \kw{feed} hosts = 
%     \{| mk_host n | n <- (flatten hostList), 
%                    Nodelist.is_node n |\}

% \kw{feed} stats =
%   \kw{foreach} h \kw{in} hosts \kw{create}
%   \kw{let} s = once, timeout 1 min \kw{in}
%   (
%    \kw{base} \{| 
%       \kw{sources}  = \kw{proc} ("ping -c 1 " ^ h);   
%       \kw{format}   = Ping.Source;  
%       \kw{schedule} = s; |\},
%    \kw{base} \{| 
%       \kw{sources}  = \kw{proc} ("ssh " ^ h ^ " uptime");  
%       \kw{format}   = Uptime.Source;  
%       \kw{schedule} = s; |\}
%   ) 
% \end{code}
% \vskip -2ex
% \caption{\texttt{arrakis.fml} Simplified version of \vizGems{} feed.}
% \label{fig:pulse}
% \end{figure}


\subsection{\vizGems{} Example}
We now shift to an example drawn from AT\&T's \vizGems{} project.  Like
the earlier CoMon example, the \cd{stats} feed in \figref{fig:pulse2}
monitors a collection of machines described in a configuration file.
Before we discuss the \cd{stats} feed itself, we first explain some
auxiliary feeds that we use in the definition of the \cd{stats} feed.   

The \cd{raw\_hostLists} description has the same form as the \cd{nodes}
feed we saw earlier, except it draws the data from a local file once a
day.  We use a \textit{feed comprehension} to define a clean
version of the feed, \cd{host\_lists}.  In the comprehension, the
built-in predicate \cd{is\_good} verifies that no errors occurred in
fetching the current list of machines \cd{hl}, as would be expected
for a local file.  The function \cd{get\_hosts} takes \cd{hl} and uses
the built-in function \cd{get\_good} to unwrap the payload data from
the error infrastructure, an operation that is guaranteed to succeed
because of the \cd{is\_good} guard. The function \cd{get_hosts} then
selects the host name entries and unwraps them to produce a list of
unadorned host names.   

We next define a feed generator \cd{gen_stats} that yields an
integrated feed of performance statistics for each supplied host.  In
more detail, when given a host \cd{h}, \cd{gen_stats} creates a five
minute schedule with a one minute timeout. It then uses this schedule
to describe a compound feed, which pairs two base feeds: the first
uses the Unix command \cd{ping} to collect network statistics about
the route to \cd{h} while the second performs a remote shell
invocation using \cd{ssh} to gather statistics about how long the
machine has been up.  Both of these feeds use the \kw{proc}
constructor in the \kw{sources} field to compute the data on the fly,
rather than reading it from a file.  The argument to \kw{proc} is a
string that the system executes in a freshly constructed shell.  The
pairing constructor for feeds takes a pair of feeds and returns a feed
of pairs, with elements sharing the same scheduled fetch-time being
paired. This semantics conveniently produces a compound feed that for
each host returns a pair of its ping and uptime statistics, grouping
together the information for each host.  Of course, the full
\vizGems{} monitoring application uses many more tools than just ping
and uptime to probe remote machines so the full feed description would have
many more components than this simplified version. 


%% For each list of hosts in the
%% \cd{host_lists} feed, we generate a {\em feed of lists} using 
Finally, we define the feed \cd{stats}.  The most interesting piece
of this declaration is the {\em list feed comprehension}, given in
square brackets, that we use to generate a feed of lists. 
Given a host list element \cd{hl}, the right-hand side of the comprehension
uses the \cd{value} function to extract the data from the meta-data and
then considers each host \cd{h} from \cd{hl} in turn.  The left-hand side of
the comprehension uses the \cd{gen_stats} feed generator to construct
a feed of the statistics for \cd{h}.  The list feed comprehension then
takes this collection of statistics feeds and converts them into a
single feed, where each entry is a list of the statistics for
the machines in \cd{hl} at a particular scheduled fetch-time.  
We call each such entry a \textit{snapshot} of the system.
The resulting feed makes it easy for down-stream users to perform
actions over snapshots, relieving them of the burden of having to
implement their own multi-way synchronization.
Given the list feed comprehension, the \kw{foreach...create} construct
generates a feed of snapshots from the feed of host lists.  
%Whenever a
%new host list arrives, the \kw{foreach...update} construct
%terminates the snapshot feed from the old host list and starts
%generating a new snapshot feed from the new host list.  

% Note the difference between the \kw{foreach...create} construct 
% from the CoMon example and the \kw{foreach...update} construct.  The
% create form generates a collection of feeds and merges their
% contents into a single all-inclusive feed.  The update form
% generates a collection of feeds and produces a single feed by
% concatenating the collection, stopping one feed as soon as the next is
% generated. 
% We have found the create form to be useful when the actual arrival
% times of the argument feed are regular because the regularity means we
% can give a finite schedule for the dependent feed.  In contrast, the
% update form is useful when the argument feed is irregular and we must
% give an infinite schedule for the dependent feed to ensure we get the
% desired values.  We define precise semantics for the create and update
% forms in \secref{sec:semantics}.


% construct then merges this list of
% feeds into a single feed with data for all the hosts.  This merged
% feed delivers values for a given host list \cd{hl} until the next host
% list \cd{hl}' arrives from the feed \cd{host\_lists}, at which point
% it stops collecting data from the feeds generated from \cd{hl} and
% starts gathering data from the machines in \cd{hl}'.

% With this specification, we expect the system to return data for each
% machine listed in the configuration file every five minutes.  We will
% notice changes in the configuration file once a day.

\begin{figure}
\begin{code}
\kw{let} config_locations =
  [("file:///arrakis/config/machine_list")];

\kw{feed} raw_hostLists =  
  \kw{base} \{|
   \kw{sources}  = \kw{all} config_locations;
   \kw{schedule} = every 24 hours; 
   \kw{format}   = Hosts.Source;   |\}

\kw{let} get_host (Hosts.Data h) = h
\kw{let} get_hosts hl =
    List.map get_host 
     (List.filter Hosts.is_node hl)

\kw{feed} host_lists = 
  \{| get_hosts (get_good hl) | 
     hl <- raw_hostLists, is_good hl |\}

\kw{feed} gen_stats (h) = 
  \kw{let} s = every 5 mins, 
          timeout 1 min, 
          lasting 24 hours \kw{in}
  (
   \kw{base} \{| 
     \kw{sources}  = \kw{proc} ("ping -c 1 " ^ h);   
     \kw{format}   = Ping.Source;  
     \kw{schedule} = s; |\},
   \kw{base} \{| 
     \kw{sources}  = \kw{proc} ("ssh " ^ h ^ " uptime");  
     \kw{format}   = Uptime.Source;  
     \kw{schedule} = s; |\}
  )

\kw{feed} stats =
  \kw{foreach} hl \kw{in} host_lists \kw{create}
     [ gen_stats (h) | h <- value hl ]
\end{code}
\vskip -2ex
\caption{\texttt{arrakis.fml}: Simplified version of \vizGems{} feed.}
\label{fig:pulse2}
\end{figure}


\section{\padsd{} Semantics}
\label{sec:semantics}
\begin{figure}[t]
\[
\begin{array}{lll}
\multicolumn{3}{l}{\mbox{(host-language base types)}}\\ 
\multicolumn{3}{l}{\basety \ ::= \boolty \bnfalt \stringty \bnfalt
  \timety \bnfalt \locty} \\
\\
\multicolumn{3}{l}{\mbox{(host-language types)}}\\ 
\multicolumn{3}{l}{\ty \ ::=\ \basety
\bnfalt \optionty{\ty}
\bnfalt \ty_1 * \ty_2
\bnfalt \ty_1 + \ty_2
\bnfalt \listty{\ty}
\bnfalt \setty{\ty}
\bnfalt \ty_1 \arrow \ty_2
} \\
\\
\multicolumn{3}{l}{\mbox{(host-language values)}}\\ 
\multicolumn{3}{l}{\data \ ::=} \\
& \boolf \bnfalt \boolt & \mbox{booleans} \\
\bnfalt & \astring \bnfalt \atime \bnfalt \loc &
 \mbox{strings, times, locations} \\
\bnfalt & \none \bnfalt 
                           \some{\data} & \mbox{optional values}\\
\bnfalt & (\data_1,\data_2) & \mbox{pairs} \\
\bnfalt & \inl{\data} \bnfalt 
                           \inr{\data} & \mbox{sum values} \\
\bnfalt & 
%\nillist \bnfalt 
                           \conslist{\data_1}{\data_n} & \mbox{list values} \\
\bnfalt &                  \{\data_1,\ldots,\data_n\} & \mbox{set values} \\

% & \bnfalt & \nilstream \bnfalt 
%                           \consstream{\data_1}{\data_2} & \mbox{stream values} \\

\bnfalt & \lambda x{:}\ty.\expression & \mbox{function values} \\
\\
\multicolumn{3}{l}{\mbox{(host-language expressions)}}\\ 
\multicolumn{3}{l}{\expression \ ::=}\\ 
& \generalvar & \mbox{variables} \\
\bnfalt & \data & \mbox{data values} \\
\bnfalt & \none \bnfalt 
              \some{\expression} & \mbox{option expressions}\\
%\bnfalt & (\expression_1,\expression_2) \bnfalt e.1 \bnfalt e.2 
%    & \mbox{pair expressions} \\
% & \bnfalt & \inl{\expression} \bnfalt 
%             \inr{\expression} & \mbox{sum expressions} \\
% & \bnfalt & \expression_1 \; \expression_2 & \mbox{application expression} \\
\bnfalt & ... & \mbox{more typed lambda expressions} \\
%\\
%\multicolumn{4}{l}{\mbox{(feed meta-data:  a subset of host language values)}}\\ 
%\multicolumn{4}{l}{\mbox{(a special location (\generatedloc) is used when data is created artificially)}}\\ 
\end{array}
\]\caption{Host Language Syntax.}
\label{fig:host-language}
\end{figure}


\begin{figure}[t]
\[
\begin{array}{lll}
\multicolumn{3}{l}{\mbox{(feed payload types)}}\\ 
\multicolumn{3}{l}{\sigma \ ::= \tau \bnfalt \optionty{\tau} 
  \bnfalt \sigma_1 * \sigma_2
  \bnfalt \sigma_1 + \sigma_2
  \bnfalt \listty{\sigma}
}   \\  
\\
\multicolumn{3}{l}{\mbox{(core feeds)}}\\ 
\multicolumn{3}{l}{\corefeed \ ::= }\\
& \{ \ \mathtt{src=}\    e_1;    & \mbox{source specification} \\
& \ \ \ \mathtt{sched=}\  e_2;    & \mbox{schedule specification}\\
& \ \ \ \mathtt{win=}\    e_3;    & \mbox{time-out window specification} \\
& \ \ \ \mathtt{pp=}\     e_4;    & \mbox{pre-processor} \\
& \ \ \ \mathtt{format=}\ e_5; \} & \mbox{format specification}\\ 
\\
\multicolumn{3}{l}{\mbox{(feeds)}}\\ 
\multicolumn{3}{l}{\feed \ ::=}   \\  
% & x &  \mbox{feed variable} \\ %% no feed variables now
% & \bnfalt 
         & \mathtt{all}\ \corefeed & \mbox{all sources}\\ 
 \bnfalt & \mathtt{any}\ \corefeed & \mbox{one of multiple sources}\\ 
 \bnfalt & \emptyfeed & \mbox{empty feed} \\
 \bnfalt & \onefeed{e_v}{e_t} & \mbox{singleton feed} \\
 \bnfalt & \sfeed{e} & \mbox{schedule to feed} \\
% \bnfalt & \lfeed{e} & \mbox{list to feed} \\
 \bnfalt & \feed_1 \unionfeed \feed_2 & \mbox{union feed} \\
 \bnfalt & \feed_1 \sumfeed \feed_2 & \mbox{sum feed} \\
 \bnfalt & (\feed_1, \feed_2) & \mbox{pair feed} \\
 \bnfalt & [\feed \bnfalt x \leftarrow e ] & \mbox{list comprehension feed} \\
 \bnfalt & \comprehensionfeed{\feed_2}{x}{\feed_1} & \mbox{feed comprehension} \\
 \bnfalt & \filterfeed{\feed}{e} & \mbox{filter feed} \\
 \bnfalt & \letfeed{x}{e}{\feed} & \mbox{let feed} \\
% & \bnfalt & \feed_1 cartesian \feed_2 & \mbox{cartesian pair -- use a symbol different from *} \\
% & \bnfalt & \feed_1 * \feed_2 & \mbox{continuous pair} \\
% & \bnfalt & \feed_1 {*}{*} \feed_2 & \mbox{local pair} \\
% \bnfalt & x{:}\feed_1 * \feed_2 & \mbox{dependent continuous pair} \\
% \bnfalt & x{:}\feed_1\, {*}{*} \, \feed_2 & \mbox{dependent local pair} \\
% \bnfalt &     \mathtt{foreach{*}}\; x \; 
%    \mathtt{in}\; \feed_1 & \mbox{for each $x$ create continuous $\feed_2$} \\
% &   \quad \mathtt{create}\; \feed_2 \\
% \bnfalt &     \mathtt{foreach{*}{*}}\; x \; 
%    \mathtt{in}\; \feed_1 & \mbox{for each $x$ update local $\feed_2$}\\
% &   \quad \mathtt{update}\; \feed_2 \\
%\foreachcreate{x}{\feed_1}{\feed_2} & \mbox{for each $x$ create continuous $F_2$} \\
% \bnfalt & \foreachupdate{x}{\feed_1}{\feed_2} & \mbox{for each $x$ create local $F_2$} \\
% & \bnfalt & \ppfeed{\feed}{e} & \mbox{preprocess (eg, unzip) data} \\
% & \bnfalt & \remap{\feed}{e} & \mbox{direct feed to different locations/times} \\
% & \bnfalt & \refeed{\feed}{e} & \mbox{adapt feed to new schedule; 
%                                               fill missing entries with ``None''} \\
% & \bnfalt & \stutterfeed{\feed}{e} & \mbox{stutter on new schedule} \\
\end{array}
\]
\caption{Feed Language Syntax.}
\label{fig:syntax}
\end{figure}


Developing a formal semantics for \padsd{} has been an integral part
of our language design process.  We have used the semantics to
communicate our ideas precisely and to explore the nuances of design
decisions. Furthermore, the semantics provides users with a tool to
reason about the feeds resulting from \padsd{} descriptions, including
subtleties related to synchronization, timeouts and errors.

To express locations, times, schedules and constraints, the feed calculus
depends upon a {\em host language}, which we take to be the
simply-typed lambda calculus.  Figure~\ref{fig:host-language} presents
its syntax, which includes a collection of constants to simplify the
semantics: strings ($\astring$), times ($\atime$) and locations
($\loc$).  We assume times may be added and 
compared and we let $\infty$ represent a time later than all others.
We assume that the set of locations includes the constant
$\generatedloc{}$, indicating the associated data was computed rather
than fetched.
We treat schedules as sets of times and use the notation $\atime
\in \schedule$ to refer to a time $\atime$ drawn from the set
$\schedule$.  We use a similar notation to refer to elements of a
list.  The host language also includes standard structured types such as
options, pairs, sums, lists and functions.
We omit the typing annotations from lambda expressions when they can
be reconstructed from the context.


\subsection{Feed Syntax and Typing}
The abstract syntax for our feed calculus and its typing rules appear
in Figures~\ref{fig:syntax} and~\ref{fig:typing}, respectively.  
The feed typing judgment has the form 
$\Gamma \turn \feed : \feedty{\sigma}$, 
which means that in the context $\Gamma$ mapping variables to host
language types $\tau$, $\feed$ is a feed of $\sigma$ values. 
The core typing judgment, which has the form 
$\Gamma \turn \corefeed{} : \corety{\sigma}$, conveys the same
information for core feeds.


\begin{figure}

% \[
% \infer[(\textit{t-var})]
% {\Gamma \turn x : \Gamma(x)}
% {}
% \]

\[
\infer[(\textit{t-core})]
{ \begin{array}{l}
  \Gamma \turn 
   \{
      \mathtt{src=} e_1;\
      \mathtt{sched=} e_2; \
      \mathtt{win=} e_3;\\ \qquad \ \ 
      \mathtt{pp=} e_4;\
      \mathtt{ format=} e_5; 
   \} 
   : \corety{\optionty{\ty}}
 \end{array}
}
{
 \begin{array}{c}
  \Gamma \turn e_1 : \listty{\locty} \quad \
  \Gamma \turn e_2 : \schedulety \quad \
  \Gamma \turn e_3 : \timety\\
  \Gamma \turn e_4 : \optionty{\stringty} \arrow \optionty{\stringty}  \\
  \Gamma \turn e_5 : \optionty{\stringty} \arrow \optionty{\ty} \\
 \end{array}
}
\]

\[
\infer[(\textit{t-all})]
{ \begin{array}{l}
  \Gamma \turn \mathtt{all}\ \corefeed{} : \feedty{\sigma}
 \end{array}
}
{
 \begin{array}{c}
  \Gamma \turn \corefeed{} : \corety{\sigma}
 \end{array}
}
\]

\[
\infer[(\textit{t-any})]
{ \begin{array}{l}
  \Gamma \turn \mathtt{any}\ \corefeed{} : \feedty{\sigma}
 \end{array}
}
{
 \begin{array}{c}
  \Gamma \turn \corefeed{} : \corety{\sigma}
 \end{array}
}
\]

\[
\infer[(\textit{t-empty})]
{\Gamma \turn \emptyfeed : \feedty{\sigma}}
{}
\]

\[
\infer[(\textit{t-one})]
{\Gamma \turn \onefeed{e_v}{e_t} : \feedty{\tau}}
{\Gamma \turn e_v : \tau
 \qquad
 \Gamma \turn e_t : \timety
}
\]

\[
\infer[(\textit{t-schedule})]
{\Gamma \turn \sfeed{e} : \feedty{\timety}}
{\Gamma \turn e : \schedulety
}
\]


%% \[
%% \infer[(\textit{t-list})]
%% {\Gamma \turn \lfeed{e} : \feedty{\tau}}
%% {\Gamma \turn e : \listty{\tau}
%% }
%% \]

\[
\infer[(\textit{t-union})]
{\Gamma \turn \feed_1 \unionfeed \feed_2  : \feedty{\sigma}}
{
  \Gamma \turn \feed_1 : \feedty{\sigma} &
  \Gamma \turn \feed_2 : \feedty{\sigma}
}
\]

\[
\infer[(\textit{t-sum})]
{\Gamma \turn \feed_1 \sumfeed \feed_2  : \feedty{\sigma_1 + \sigma_2}}
{
  \Gamma \turn \feed_1 : \feedty{\sigma_1} &
  \Gamma \turn \feed_2 : \feedty{\sigma_2}
}
\]

\[
\infer[(\textit{t-pair})]
{\Gamma \turn (\feed_1, \feed_2)  : \feedty{\sigma_1 * \sigma_2}}
{
  \Gamma \turn \feed_1 : \feedty{\sigma_1} &
  \Gamma \turn \feed_2 : \feedty{\sigma_2}
}
\]

\[
\infer[(\textit{t-list})]
{\Gamma \turn [\feed \bnfalt x \leftarrow e ]  : \feedty{\listty{\sigma}}}
{
  \Gamma \turn e : \listty{\tau} &
  \Gamma,x{:}\tau \turn \feed : \feedty{\sigma} 
}
\]

\[
\infer[(\textit{t-comp})]
{\Gamma \turn \comprehensionfeed{\feed_2}{x}{\feed_1} : \feedty{\sigma}}
{
  \Gamma \turn \feed_1 :  \feedty{\sigma} &
  \Gamma,x{:}\metatype{\sigma} * \sigma \turn \feed_2 : \feedty{\sigma} 
}
\]

\[
\infer[(\textit{t-filter})]
{\Gamma \turn \filterfeed{\feed}{e} : \feedty{\sigma}}
{
  \Gamma \turn \feed : \feedty{\sigma} &
  \Gamma \turn e : (\metatype{\sigma} * \sigma) \arrow \boolty
}
\]

\[
\infer[(\textit{t-let})]
{\Gamma \turn \letfeed{x}{e_1}{\feed_2} : \feedty{\sigma_2}}
{
  \Gamma \turn e_1 : \ty_1 & 
  \Gamma,x{:}\ty_1 \turn \feed_2 : \feedty{\sigma_2} 
}
\]
\caption{Feed Language Typing.}
\label{fig:typing}
\end{figure}


Intuitively, a feed carrying values of type $\sigma$ is a sequence of
payload values of type $\sigma$.  However, to record provenance
information, we pair each payload value with meta-data, so a feed is
actually a sequence of (meta-data, payload) pairs.  At the top-level,
meta-data consists of a triple of the scheduled time for the payload,
a \textit{dependency set} that records the origin and scheduled time of any data
that contributed to the payload, and a nested meta-data field whose
form depends upon the type of the payload.

Formally, we let 
$\meta$ range over top-level meta-data,
$\ds$ range over dependency sets, and 
$\nested$ range over ``nested'' meta-data:
\[
\begin{array}{lcll} 
\meta & ::= & (\atime,\ds,\nested) & \mbox{top-level meta-data} \\  
\\
\ds   & ::= & \{(\atime_1,\loc_1),\ldots,(\atime_n, \loc_n) \}  & \mbox{dependency set}\\ 
\\
\nested & ::=     
          & (\atime,\loc,\mathtt{None}) & \mbox{base meta-data (timeout)} \\
& \bnfalt & (\atime,\loc,\mathtt{Some}\; \atime) & \mbox{base meta-data (success)} \\
& \bnfalt & (\nested_1,\nested_2) & \mbox{pair meta-data} \\
& \bnfalt & \inl{\nested} & \mbox{sum meta-data} \\
& \bnfalt & \inr{\nested} & \mbox{sum meta-data} \\
& \bnfalt & [\nested_1,\ldots,\nested_k] & \mbox{list meta-data} \\
\end{array}
\] 
Given meta-data $\meta$, we write $\mytime{\meta}$, $\myds{\meta}$ and
$\myval{\meta}$ for the first, second and third projections (respectively) of $\meta$.
Base meta-data is a triple of the scheduled time, the location of origin 
and an optional arrival time where {\tt None} indicates the data did not arrive
in a timely fashion.

As shown in \figref{fig:syntax}, we define the feed payload type
$\sigma$ in terms of host 
language types, stratified to facilitate the proof of
semantic soundness.  
We use the function $\metatype{\sigma}$ to define the type of
meta-data associated with payload of type $\sigma$:
\[
\begin {array} {lcl}
\nestedtype{\ty} & = & \timety * \locty * (\optionty{\timety}) \\
\nestedtype{\optionty{\ty}} & = & \timety * \locty * (\optionty{\timety}) \\
\nestedtype{\sigma_1 * \sigma_2} & = & \nestedtype{\sigma_1} * \nestedtype{\sigma_2} \\
\nestedtype{\sigma_1 + \sigma_2} & = & \nestedtype{\sigma_1} + \nestedtype{\sigma_2} \\
\nestedtype{\listty{\sigma}} & = & \listty{\nestedtype{\sigma}} \\
\\
\metatype{\sigma} & = & \timety * \dsty * \nestedtype{\sigma} \\
\end{array}
\]
Feed typing depends upon a standard judgment for
typing lambda calculus expressions: $\Gamma \turn e : \ty$.  

With these preliminaries, we can now discuss the syntax and typing for
each of the feed constructs in \figref{fig:syntax}. 
Core feeds express the structure of base feeds, describing
the data sources ($\mathtt{src}$), schedule ($\mathtt{sched}$), window
($\mathtt{win}$), preprocessing function ($\mathtt{pp}$) and file
format ($\mathtt{format}$).  The source field describes the set of
locations from which to fetch data.  It may contain
pseudo-locations that model the $\mathtt{proc}$ form found in the
implementation.  Instead of having timeouts specified as part of
schedules, as we did in the surface language, the calculus separates
these two concepts into distinct fields, which simplifies the semantics.
If an item specified to arrive at time $\atime$ by schedule $e_2$ fails
to arrive within the window $e_3$, the feed pretends it received the
value \texttt{None}.  Otherwise, it wraps the received data string in
an option. As a result, the preprocessor $e_4$ maps a $\optionty{\stringty}$
to a $\optionty{\stringty}$, where a result of \texttt{None} indicates
either a network or preprocessing error.  Finally, the formatting
function $e_5$ parses the output of the preprocessor to produce a 
value of type $\optionty{\tau}$, where a \texttt{None} result
indicates a network, preprocessing or formatting error. (For the sake
of simplicity, we do not model the variety of error codes that the
implementation supports.)  

The feed $\mathtt{all}\ \corefeed$ selects all the data from the core
feed \corefeed.  The feed $\mathtt{any}\ \corefeed$ selects the first
good value to arrive from any location for each time in the schedule
for \corefeed{}, returning \texttt{None} paired with appropriate
meta-data if no such good value exists.    

The empty feed ($\emptyset$) contains no elements and has polymorphic
type a l\`a the empty list.  The singleton feed $\onefeed{e_v}{e_t}$
constructs a feed containing a single value $e_v$ at a single time
$e_t$.  The schedule feed $\sfeed{e}$ builds a feed whose elements 
are the times in the schedule $e$.
The union feed merges two feeds with the same type
into a single feed.  In contrast, the sum feed takes two feeds
with (possibly) different types and injects the elements of each feed
into a sum before merging the results into a single feed.  
The pair feed, written $(\feed_1, \feed_2)$, combines the elements of
the two nested feeds synchronously, matching elements that have the
same {\em scheduled} time, regardless of when those elements
actually {\em arrive}.
The list feed $[\feed \bnfalt x \leftarrow e ]$, in contrast, provides
$n$-way synchronization, where $n$ is the length of the input list
$e$.  Each element $e_i$ in $e$ defines a feed $\feed_i = \feed[x \mapsto e_i]$.
For each time $\atime$ with a value $v_i$ in each feed $\feed_i$, the
list feed returns the list $[v_1, \ldots, v_n]$.  Note that if the
$\feed_i$ feeds share a schedule $s$, then each feed will have a value
for every time in the schedule $s$, even in the presence of errors, so
the synchronization will succeeed at each time in the schedule $s$.
The feed comprehension $\comprehensionfeed{\feed_2}{x}{\feed_1}$
creates a feed with elements $\feed_2[x \mapsto v]$ when $v$ is an
element of $\feed_1$. Note that the entry $v$ is a pair of meta-data
(with type $\metatype{\sigma}$) and payload data (with type $\sigma$).
The feed $\filterfeed{\feed}{e}$ eliminates elements $v$ from $\feed$ when
$e\; v$ is $\boolf$.  Let feeds $\letfeed{x}{e}{\feed}$
provide a convenient mechanism for binding intermediate values. 


\newcommand{\rb}[1]{\raisebox{6ex}[0pt]{#1}}

\begin{figure*}[t]
\[
\begin{array}{lcl}

    {\cal C}\lsem\mathtt{\{ src=} e_{src}; 
 &=& (S, \{((\atime,\loc), \esemantics{e_f\; (\universe'(\loc,\atime))}{\environment}))
          \setalt \atime \in S
          \;\mbox{and}\; \loc \in  \esemantics{e_{src}}{\environment}
     \})
\\
 \quad\ \   \mathtt{sched=} e_{sched};
&&\quad\mbox{where} \\
 \quad\ \  \mathtt{win=} e_{win};
&& \qquad S = \esemantics{e_{sched}}{\environment} \\
 \quad\ \  \mathtt{pp=} e_{pp};
&& 
\qquad \mathtt{timeout} =  
     \lambda (x_t,(x_{at},x_s)).
        \mathtt{if}\, x_{at} \leq x_t + \esemantics{e_{win}}{\environment} \,
        \mathtt{then}\,  x_s \, \mathtt{else} \, \mathtt{None} 
 \\
 \quad\ \  \mathtt{format=} e_{f}; \}\rsem_{{\environment} \, {\universe}}
&& \qquad \universe' =
     \lambda (x_{\ell}, x_t). 
           \esemantics{e_{pp}}{\environment}\, 
                 (\mathtt{timeout}\, (x_t,\universe (x_\ell,x_t))) 
\\\\

\semantics{\mathtt{all}\ \corefeed}{\universe}{\environment} 
&=& 
A\ \ \ \mbox{where}\ (S,A) = \csemantics{C}{\universe}{\environment}
\\\\


%New version: any rule
\semantics{\mathtt{any}\ \corefeed}{\universe}{\environment} 
& = & \{ i_t\ |\ \atime \in S\}\\
&&
\begin{array}{l}
 \begin{array}{ll@{\hspace{1ex}}c@{\hspace{1ex}}l}
 \mbox{where} & (S,A)   & = &\csemantics{C}{\environment}{\universe}\\
              & A_\atime & = & \{(\meta,\some{v})\ | \ (\meta, \some{v}) \in A\ \mbox{and} \ \mytime{\meta} = \atime\}\\
              & i_\atime & = & \left\{ \begin{array}{lll}
                                           \mbox{\selectOne}(A_\atime) & \mbox{if} & |A_\atime| > 0\\
                                           ((\atime,\generatedloc), \none) & \mbox{if} & |A_\atime| = 0 \\
                                           \end{array} \right.\\
 \end{array}
\end{array} 
%%End New version: any rule
\\\\

\semantics{\emptyfeed}{\environment}{\universe} 
 &=& \{\;\}
\\\\
\semantics{\computed{e_1}{x}{e_2}}{\environment}{\universe} 
 &=& \{((\atime,\generatedloc), \esemantics{(\lambda x.e_1) \; \atime}{\environment}) 
          \setalt \atime \in  \esemantics{e_2}{\environment} 
     \} 
\\\\
\semantics{\comprehensionfeed{e}{x}{\feed}}{\environment}{\universe} 
 &=& \{((\mytime{\meta},\generatedloc), \esemantics{(\lambda x.e) \; v}{\environment}) 
          \setalt (\meta,v) \in  \semantics{\feed}{\environment}{\universe}  
     \} 
\\\\
\semantics{\filterfeed{\feed}{e}}{\environment}{\universe} 
 &=&
\{(\meta,v) \setalt (\meta,v) \in \semantics{\feed}{\environment}{\universe} \; \mbox{and} \;
            \esemantics{e \; v}{\environment} = \mathtt{true}
\}
\\\\
\semantics{\letfeed{x}{e_1}{\feed_2}}{\environment}{\universe} 
 &=& \semantics{\feed_2}{(\environment,x\mapsto\esemantics{e_1}{\environment})}{\universe} 
\\\\

\semantics{\feed_1 \unionfeed \feed_2}{\environment}{\universe} 
 &=& \semantics{\feed_1}{\environment}{\universe} 
     \bigcup
     \semantics{\feed_2}{\environment}{\universe} 
\\\\
\semantics{\feed_1 \sumfeed \feed_2}{\environment}{\universe} 
 &=& \{
      ((\mytime{\meta},\inl{\meta}),\inl{v}) \setalt 
        (\meta,v) \in \semantics{\feed_1}{\environment}{\universe} 
     \}
     \bigcup
     \{
      ((\mytime{\meta},\inr{\meta}),\inr{v}) \setalt 
        (\meta,v) \in \semantics{\feed_2}{\environment}{\universe}
     \}
\\\\
\semantics{(\feed_1, \feed_2)}{\environment}{\universe} 
 &=&
 \{((\mytime{\meta_1},(\meta_1,\meta_2)),(v_1,v_2)) \setalt 
     (\meta_1,v_1) \in \semantics{\feed_1}{\environment}{\universe} 
     \; \mbox{and} \; 
     (\meta_2,v_2) \in \semantics{\feed_2}{\environment}{\universe}
     \; \mbox{and} \; 
     \mytime{\meta_1} = \mytime{\meta_2}
  \}
\\\\
% \semantics{\feed_1 * \feed_2}{\environment}{\universe} 
%  &=&
%  \{(\atime_2,(v_1,v_2)) \setalt 
%      (\atime_1,v_1) \in \semantics{\feed_1}{\environment}{\universe} 
%      \; \mbox{and} \; 
% \\&&\qquad\qquad\qquad\ \ \,
%      (\atime_2,v_2) \in \semantics{\feed_2}{\environment}{\universe}
%      \; \mbox{and} \;
% \\&&\qquad\qquad\qquad\ \ \,
%      ((\atime_1',v_1') \in \semantics{\feed_1}{\environment}{\universe} 
%       \; \mbox{implies} \; (t_1' \leq t_1 \; \mbox{or} \; t_1' > t_2)) 
%   \}
% \\\\
\semantics{x{:}\feed_1 * \feed_2}{\environment}{\universe} 
 &=&
 \{(\mytime{\meta_2},(\meta_1,\meta_2)),(v_1,v_2)) \setalt 
     (\meta_1,v_1) \in \semantics{\feed_1}{\environment}{\universe} 
     \; \mbox{and} \; 
\\&&\qquad\qquad\qquad\qquad\qquad\qquad\ \ \,
     (\meta_2,v_2) \in \semantics{\feed_2}{(\environment,x\mapsto{}v_1)}{\universe}
     \; \mbox{and} \; \mytime{\meta_2} > \mytime{\meta_1}
  \}
\\\\
\semantics{x{:}\feed_1 \, {*}{*} \, \feed_2}{\environment}{\universe} 
 &=&
 \{(\mytime{\meta_2},(\meta_1,\meta_2)),(v_1,v_2)) \setalt 
     (\meta_1,v_1) \in \semantics{\feed_1}{\environment}{\universe} 
     \; \mbox{and} \; 
\\&&\qquad\qquad\qquad\qquad\qquad\qquad\ \ \,
     (\meta_2,v_2) \in \semantics{\feed_2}{(\environment,x\mapsto{}v_1)}{\universe}
     \; \mbox{and} \; \mytime{\meta_2} > \mytime{\meta_1}
\\&&\qquad\qquad\qquad\qquad\qquad\qquad\ \ \,
     ((\meta_1',v_1') \in \semantics{\feed_1}{\environment}{\universe} 
      \; \mbox{implies} \; (\mytime{\meta_1'} \leq \mytime{\meta_1} 
            \; \mbox{or} \; \mytime{\meta_1'} > \mytime{\meta_2})) 
  \}
\\\\
%%OLD foreach update
%%{\cal F}\lsem
%%\mathtt{foreach{*}}\; x \; \mathtt{in}\; \feed_1 
%%%\semantics{\foreachupdate{x}{\feed_1}{\feed_2}}{\environment}{\universe} 
%% &=&
%% \{(\meta_2,v_2) \setalt 
%%     (\meta_1,v_1) \in \semantics{\feed_1}{\environment}{\universe} 
%%     \; \mbox{and} \; 
%%\\
%%\qquad\qquad\ \ \mathtt{create}\; \feed_2 \rsem_{{\environment} \, {\universe}}
%%&&\qquad\qquad\ \,
%%     (\meta_2,v_2) \in \semantics{\feed_2}{(\environment,x\mapsto{}v_1)}{\universe}
%%     \; \mbox{and} \;
%%%\\&&\qquad\qquad\ \,
%%     \mytime{\meta_2} > \mytime{\meta_1} 
%%  \}
%%\\\\




%% New foreach *
{\cal F}\lsem
\mathtt{foreach{*}}\; x \; \mathtt{in}\; \feed_1 
%\semantics{\foreachupdate{x}{\feed_1}{\feed_2}}{\environment}{\universe} 
 &=&
   \{(\meta_2,v_2) \setalt (\atime,(\meta_1,\meta_2)),(v_1,v_2)) \in 
       \semantics{x{:}\feed_1 * \feed_2}{\environment}{\universe} \}
%% \{(\meta_2,v_2) \setalt 
%%     (\meta_1,v_1) \in \semantics{\feed_1}{\environment}{\universe} 
%%     \; \mbox{and} \; 
\\
\qquad\qquad\ \ \mathtt{create}\; \feed_2 \rsem_{{\environment} \, {\universe}}
%% End New foreach *
\\\\

%% Old foreach **
%%{\cal F}\lsem
%%\mathtt{foreach{*}{*}}\; x \; \mathtt{in}\; \feed_1 
%%%\semantics{\foreachcreate{x}{\feed_1}{\feed_2}}{\environment}{\universe} 
%% &=&
%% \{(\meta_2,v_2) \setalt 
%%     (\meta_1,v_1) \in \semantics{\feed_1}{\environment}{\universe} 
%%     \; \mbox{and} \; 
%%\\
%%\qquad\qquad\ \ \ \mathtt{update}\; \feed_2 \rsem_{{\environment} \, {\universe}}
%%&&\qquad\qquad\ \,
%%     (\meta_2,v_2) \in \semantics{\feed_2}{(\environment,x\mapsto{}v_1)}{\universe}
%%     \; \mbox{and} \; \mytime{\meta_2} > \mytime{\meta_1} \; \mbox{and} \;
%%\\&&\qquad\qquad\ \,
%%     ((\meta_1',v_1') \in \semantics{\feed_1}{\environment}{\universe} 
%%      \; \mbox{implies} \; (\mytime{\meta_1'} \leq \mytime{\meta_1} 
%%           \; \mbox{or} \; \mytime{\meta_1'} > \mytime{\meta_2}))      
%%  \}
%%\\\\
%% End Old foreach **

%% New Foreach **
{\cal F}\lsem
\mathtt{foreach{*}{*}}\; x \; \mathtt{in}\; \feed_1 
 &=&
   \{(\meta_2,v_2) \setalt (\atime,(\meta_1,\meta_2)),(v_1,v_2)) \in 
       \semantics{x{:}\feed_1\, {**}\, \feed_2}{\environment}{\universe} \}\\
\qquad\qquad\ \ \ \mathtt{update}\; \feed_2 \rsem_{{\environment} \, {\universe}}
\\\\
%% end new foreach **

% \semantics{\ppfeed{\feed}{e}}{\environment}{\universe} 
%  &=&
% \semantics{\feed}{\environment}{
%   (\lambda x{:}\locty * \timety. \esemantics{e}{\environment} (x,\universe(x)))} 
% \\\\
% \semantics{\remapfeed{\feed}{e}}{\environment}{\universe} 
%  &=&
% \semantics{\feed}{\environment}{(\universe \circ \esemantics{e}{\environment})}
% \\\\
%% \semantics{\refeed{\feed}{e}}{\environment}{\universe} 
%%  &=&
%% \{(\atime,\some{v}) \setalt 
%%    (\atime,v) \in \semantics{\feed}{\environment}{\universe} \; \mbox{and} \;
%%    \atime \in \esemantics{e}{\environment}
%% \} \bigcup
%% \\&&
%% \{(\atime,\none) \setalt
%%    (\atime,\_) \not\in \semantics{\feed}{\environment}{\universe} \; \mbox{and} \;
%%    \atime \in \esemantics{e}{\environment}
%% \}
%% \\\\
%% \semantics{\stutterfeed{\feed}{e}}{\environment}{\universe} 
%%  &=&
%% \{(\atime,v) \setalt 
%%    (\atime,v) \in \semantics{\feed}{\environment}{\universe} \; \mbox{and} \;
%%    \atime \in \esemantics{e}{\environment}
%% \} \bigcup
%% \\&&
%% \{(\atime,v) \setalt 
%%    (\atime',v) \in \semantics{\feed}{\environment}{\universe} \; \mbox{and} \;
%%    \atime \in \esemantics{e}{\environment}  \; \mbox{and} \;
%% \\&&\qquad\qquad\qquad\ \ \,
%%     \mbox{for all $\atime''$ such that $\atime' < \atime'' \leq \atime$,} \;
%%    (\atime'',\_) \not\in \semantics{\feed}{\environment}{\universe} \; 
%% \}
\semantics{[\feed \bnfalt x \leftarrow e]}{\environment}{\universe} 
 &=&
 \{((\atime,[\meta_1,\ldots,\meta_k]),[v_1,\ldots,v_k]) \setalt 
    \exists \atime.\forall i:1\ldots k.
     (\meta_i,v_i) \in \semantics{\feed}{\environment[x\mapsto z_i]}{\universe} 
     \; \mbox{and} \; 
     \mytime{\meta_i} =\atime
  \} \\
&&\quad\mbox{where} \quad\mbox{$[z_1,\ldots,z_k] = \esemantics{e}{\environment}$}
\\
\end{array}
\]
\caption{Feed Language Semantics.}
\label{fig:semantics}
\end{figure*}


\subsection{Feed Semantics}
We give the semantics of our formal feed language in 
a denotational style in \figref{fig:semantics}.  The principal semantic functions are
$\csemantics{\corefeed}{\environment}{\universe}$ and
$\semantics{\feed}{\environment}{\universe}$, defining core feeds and
feeds, respectively.  In these definitions,
$\environment$ is an {\em environment} mapping variables to values
and $\universe$ is a {\em universe} mapping pairs of
schedule time and location to arrival time and a string option
representing the actual data.
Intuitively, the universe models the network.
When $\universe (\atime_s, \ell) = (\atime_a, \mathtt{Some} \; \astring)$,
 the interpretation is that if the run-time system requests data
from location $\ell$ at time $\atime_s$ then string data $\astring$
will be returned at time $\atime_a$.  The time $\atime_a$ must be
no earlier than $\atime_s$.
When $\universe (\atime_s, \ell) = (\infty, \mathtt{None})$,
networking errors have made location $\ell$ unreachable.

\cut{
Both semantic functions yield a set of (meta-data, payload) pairs,
as we saw in the previous section.
Every meta-data item contains a top-level time $\atime$ that
time can be used to serialize the set of items as a stream, and 
our implementation does just that.  Items scheduled at the
same time may appear in any order in the implementation's
serialized stream.  }

The semantic definitions for ${\cal C}$ and ${\cal F}$ use
conventional set-theoretic notations.  They depend upon a
semantics for the simply-typed host language, written
$\esemantics{e}{\environment}$, whose definition we omit. We assume
that given environment $\environment$ with type $\Gamma$ and
expression $e$ with type $\tau$ in $\Gamma$,
$\esemantics{e}{\environment} = v$ and $\turn v : \tau$.

The meaning of core feed \corefeed{} is the set of
(meta-data, payload) pairs for the feed.  To construct this set, the
function first computes the set of times in the schedule $S$, the
length of the window $W$, and the set of source locations $L$.  It
uses the \texttt{timeout} function to check whether the item arrival time
$x_{at}$ is within the window $W$
of the scheduled time ($x_t \in S$), returning \texttt{None} if
not. Otherwise, \texttt{timeout} returns its data argument ($x_s$),
which may be {\tt None} because of other networking errors.  
Similarly, the \texttt{arrival} function returns the arrival time
$\texttt{Some}\ x_{at}$ if the item arrived within the window and 
\texttt{None} otherwise. The function \texttt{meta} uses the 
\texttt{arrival} function to construct the meta-data for the item,
consisting of the scheduled time $\atime$, the dependency set of the scheduled
time and  source location $\{(\atime, \loc)\}$, and the nested
meta-data, which includes the scheduled 
time $\atime$, the source location $\loc$, and the actual arrival time
$\texttt{arrival(\atime,\universe(\atime,\loc)}$. (This apparent
redundancy in the meta-data goes away with non-core feeds.) 
Using the
\texttt{timeout} function, we define an alternate universe
$\universe'$ that retrieves data from the outside world using the
original universe $\universe$, checks for a
timeout, and applies the preprocessor
($\esemantics{e_{pp}}{\environment}$) before returning.  
The \texttt{val} function applies the formating function 
$\esemantics{e_{f}}{\environment}$ to the entry returned at time
$\atime$ for location $\loc$ in alternative universe $\universe'$.
Finally, the result is the set of all pairs of meta-data and payload
produced from each time $\atime$ in the schedule $S$ and location $\loc$
in the set $L$.

The semantics of the $\mathtt{all}\ \corefeed$ is simply the semantics
of the underlying core feed.
The semantics of the $\mathtt{any}\ \corefeed$ feed selects for each
time $\atime$ in the schedule $S$ of the core feed $\corefeed$ the
earliest good payload value from any location if one exists, or
\texttt{None} otherwise.  It then returns the set of all such values
$v_t$, paired with the appropriate meta-data.  
To compute this set, the function first computes
the meaning $A$ of the core feed $\corefeed{}$.  It extracts the
schedule $S$ from the meta data in $A$.  For each time $\atime$ in the
schedule, it computes the set $A_t$ of (meta-data, payload) pairs
fetched at time $\atime$.  For each such set, it computes the
dependency set $DS_t$, which collects the dependencies of all the
items fetched at time $\atime$.  The set $G_t$ collects all the good
items from $A_t$. If this set is non-empty, we use the function 
$\selectOne$ to choose the (meta-data, payload) pair $(m,v)$ with the
earliest arrival time from $G_t$.  (We assume that there is always one
such earliest item.)  In this case, we set the nested
meta-data $nest_t$ to be the nested meta-data of $m$, and the payload
value $v_t$ to be $v$.  If the set of good values is empty, then we set the nested
meta-data to indicate that at time $\atime$, we created (location =
$\generatedloc$) a payload value that had no actual
arrival time \texttt{None}.  In this case, the payload value $v_t$ is
just \texttt{None}. 

The meaning of the empty feed is the empty set.  
The meaning of the singleton feed $\onefeed{e_v}{e_t}$ is a single
pair, the payload portion of which is the meaning of $e_v$.  The
meta-data indicates $e_t$ is the scheduled time, the dependency
set is empty, the data came from $\generatedloc$ (a dummy location
indicating that the value was generated internally), and the arrival time
matched the scheduled time. 
A schedule feed $\sfeed{e}$ yields a feed with one payload value for
each $\atime$ in the meaning of the schedule $e$.  The corresponding
meta-data follows the same pattern as for the singleton feed.  
The union feed is simply the set-theorectic union of its constituent
feeds. The sum feed injects the elements of its constituent
feeds into a sum and likewise takes their union.  It also constructs
compound meta-data from the meta-data of the consituent feeds in the
obvious way.

The pair feed $(\feed_1,\feed_2)$ is formed by finding for each time
$\atime$ all elements
of $\feed_1$ at a time $\atime$ (including erroneous elements) and all
elements of $\feed_2$ at time $\atime$ (again including erroneous
elements) and generating their Cartesian product.  Notice that 
if the schedules do not intersect, the pair feed will empty.  The
meta-data is constructed by combining the meta-data for the paired
feeds.
The semantics of the list feed $[\feed \bnfalt x \leftarrow e ]$ is
similar to that of the pair feed except the synchronization is $n$-way
instead of pairwise, where $n$ is the length of the list $e$. 

The feed comprehension $\comprehensionfeed{\feed_2}{x}{\feed_1}$
contains payload values $v_2$ taken from the meaning of feed $\feed_2$
when $x$ is mapped to (meta-data, payload) pairs drawn from the meaning
of feed $\feed_1$.  The dependency set for the feed
comprehension includes the dependency sets of {\em both} $\feed_1$ and
$\feed_2$. 
The filter feed $\filterfeed{\feed}{e}$ selects those
(meta-data, payload) pairs from 
the meaning of $\feed$  that satisfy the predicate $e$.
Finally, the let feed $\letfeed{x}{e}{\feed}$ returns the meaning
of feed $\feed$ when $x$ is mapped to the meaning of $e$. 

\subsection{Soundness}
We have proven a soundness theorem for the semantics: the values
contained in the semantics of each feed are (meta-data, payload) pairs
with the appropriate type.  More specifically, if the feed typing
rules give feed $\feed$ type $\feedty{\sigma}$, 
then its data has type $\sigma$ and its meta data has type $\metatype{\sigma}$.
We state this property formally as follows.

\begin{theorem}[Semantic Soundness]
\begin{itemize}
\item If $\Gamma \turn \feed : \feedty{\sigma}$ and
for all $x$ in $\dom(\Gamma)$, $\turn \environment(x) : \Gamma(x)$
and $\turn \universe : \timety * \locty \arrow \timety * (\optionty{\stringty})$
then
for all $(\meta,v) \in \semantics{\feed}{\environment}{\universe}$,
$\turn (\meta,v) : \metatype{\sigma} * \sigma$. 
\item If $\Gamma \turn \corefeed : \corety{\sigma}$ and
for all $x$ in $\dom(\Gamma)$, $\turn \environment(x) : \Gamma(x)$
and $\turn \universe : \timety * \locty \arrow \timety * (\optionty{\stringty})$
then
for all $(\meta,v) \in \csemantics{\corefeed}{\environment}{\universe}$,
$\turn (\meta,v) : \metatype{\sigma} * \sigma$. 
\end{itemize}
\end{theorem}
The proof follows by induction on the structure of $\feed$.

\textbf{We need to add text here explaining these defintions and why they are important.}

\begin{definition}[Equal Universes Relative to a Dependency Set]
$\ueq{\universe_1}{\ds}{\universe_2}$ if and only if for all
$(\atime,\loc) \in \ds$, $\universe_1(\atime,\loc) = \universe_2(\atime,\loc)$.
\end{definition}

Let $S_1$, $S_2$ range over denotations of feeds.

\begin{definition}[Feed Subset Relative to a Dependency Set]
$\fsubset{S_1}{\ds}{S_2}$ if and only if for all
$(\meta,v) \in S_1$ such that
$\myds{\meta} \subseteq \ds$, $(\meta,v) \in S_2$.
\end{definition}

\begin{definition}[Feed Equality Relative to a Dependency Set]
$\feq{S_1}{\ds}{S_2}$ if and only if 
$\fsubset{S_1}{\ds}{S_2}$ and
$\fsubset{S_2}{\ds}{S_1}$
\end{definition}

\begin{theorem}[Dependency Correctness]
If $\ueq{\universe_1}{\ds}{\universe_2}$ then
$\feq{\semantics{\feed}{\environment}{\universe_1}}{\ds}{\semantics{\feed}{\environment}{\universe_2}}$
\end{theorem}

The proof of dependency correctness is by induction on the structure of feeds.  






\section{\padsd{}:  Programming Interfaces and Tools}
\label{sec:programming}

\section{Related Work}
\label{sec:related}
\section{Related Work}\label{sec:related}



\section{Conclusions}
\label{sec:conclusions}

\section*{Acknowledgments}

This material is based upon work 
supported by the NSF
   under grants 0612147 and 0615062.
Any opinions, findings, and conclusions or recommendations
   expressed in this material are those of the authors and do not
   necessarily reflect the views of the NSF.

\bibliographystyle{abbrv}
\bibliography{pads,vivek}

\end{document}

%%% Local Variables:
%%% mode: outline-minor
%%% End:

