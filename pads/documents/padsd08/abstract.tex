%The term {\em ad hoc data} refers to the billions of bytes of
%non-standard and continuously evolving data spread across all computer
%systems.  Such data includes server logs, distributed system
%performance and debugging data, telephone call records, financial data
%and online repositories of scientific data.
% including genetic information, cosmology data and
%global weather systems.

This paper presents the design, theory and implementation of
\padsd{}, a  domain-specific language that allows users to
specify the provenance, syntax and semantic properties of 
collections of distributed data sources.  In particular, \padsd{}
specifications indicate 
{\em where} to locate desired data, {\em how} to obtain it, {\em
when} to get it or to give up trying, and {\em what}
format it will be in on arrival.  The \padsd{} system compiles such
specification into a suite of data-processing tools including 
an archiver, a provenance tracking system, a database loading tool, 
an alert system, an RSS feed generator and a
debugging tool.  In addition, the system generates description-specific
libraries so that developers can create their
own applications. \padsd{} also provides a generic infrastructure
so that advanced users can build new tools applicable to any 
data source with a \padsd{} description.
We show how \padsd{} may be used to specify data sources from two domains:
CoMon, a monitoring system for PlanetLab's 800+ nodes, and Arrakis, a monitoring
system for an AT\&T web hosting service.  
We show experimentally that our system can scale to distributed
systems the size of CoMon.  
Finally, we provide a
denotational semantics for \padsd{} and use this semantics to prove two important
theorems.  The first shows that our denotational semantics respects the
typing rules for the language, while the second demonstrates that our 
system correctly maintains provenance meta-data.




% CoMon 
% the Safari web
% cache to SDSS star charts {\em [dpw: correct me ... cites]} to
% PlanetLab's CoMon network monitoring system to the log files for the
% Coral content distribution network.  We also provide a denotational
% semantics for the language, which is helpful in understanding the
% language's many features.

% Descriptions written in \padsd{} are compiled into O'Caml libraries
% with clean interfaces that support both polymorphic and monomorphic
% programming.  We illustrate how to use the polymorphic interfaces to
% program description-independent applications, including a selector
% tool that can extract key elements of a data source as well as an
% accumulator tool that collects simple statistical information about
% the data.  The monomorphic interfaces allow users to develop
% ``one-of'' data source-specific applications.

