{\em Ad hoc data} refers to the untold billions upon billions of bytes
worth of non-standard, unreliable, and continuously evolving data
spread across all computer systems.  This paper defines \padsd{}, a
new domain-specific programming language designed to help programmers
specify {\em where} their ad hoc data is located, {\em when} to get
it, {\em how} to obtain it, and {\em what} preprocessing you should do
when it arrives.  As its name suggests, \padsd{} relies on the \pads{}
sublanguage, developed in previous research efforts, for specification
of the {\em format} of the data, which, as we will make clear, is
just one facet of the definition any ad hoc data source.  

We illustrate the expressiveness of \padsd{} by providing descriptions
of a number of different data sources ranging from the Safari web
cache to SDSS star charts {\em [dpw: correct me ... cites]} to
PlanetLab's CoMon network monitoring system to the log files for the
Coral content distribution network.  We also provide a denotational
semantics for the language, which is helpful in understanding the
language's many features.

Descriptions written in \padsd{} are compiled into O'Caml libraries
with clean interfaces that support both polymorphic and monomorphic
programming.  We illustrate how to use the polymorphic interfaces to
program description-independent applications, including a selector
tool that can extract key elements of a data source as well as an
accumulator tool that collects simple statistical information about
the data.  The monomorphic interfaces allow users to develop
``one-of'' data source-specific applications.
