The term {\em ad hoc data} refers to the untold billions upon billions of bytes
of non-standard, unreliable, and continuously evolving data spread
across all computer systems.  This sort of data includes such varied
sources as server logs, distributed systems performance and
debugging data, automated billing systems information, telephone call
records, financial data and online repositories of scientific data
including genetic information, cosmology data and global weather
systems data.

This paper defines a system for automatic generation of monitoring,
analysis and transformation tools for such distributed ad hoc data.
Our generated tools include data archiving system, a database loading
system, a system for tracking the statistical profile of the data and
generating alerts under customizable error conditions, an RSS feed
generator, and a simple query tool.  The system also generates a
number of libraries for application developers, including those for
parsing, printing, error management, data traversal and
transformation.  Software developers can use the programming framework
to create their own new application-specific tools or more general
generic tools that can be applied to any collection of data sources.

The tool suite is generated from high-level data descriptions
written in a domain-specific programming language called \padsd{}.  To
use the language, programmers specify {\em where} their ad hoc data is
located, {\em when} to get it, {\em how} to obtain it, and {\em what}
preprocessing the system should do when it arrives.  As its name suggests,
\padsd{} is layered on top of the \pads{} sublanguage, developed in
previous research efforts, for specification of the {\em format} of
the data sources.  We illustrate the expressiveness of \padsd{} by
providing descriptions for several different distributed systems
including CoMon, PlanetLab's node and experiment monitoring system,
and AT\&T's web hosting service.

% CoMon 
% the Safari web
% cache to SDSS star charts {\em [dpw: correct me ... cites]} to
% PlanetLab's CoMon network monitoring system to the log files for the
% Coral content distribution network.  We also provide a denotational
% semantics for the language, which is helpful in understanding the
% language's many features.

% Descriptions written in \padsd{} are compiled into O'Caml libraries
% with clean interfaces that support both polymorphic and monomorphic
% programming.  We illustrate how to use the polymorphic interfaces to
% program description-independent applications, including a selector
% tool that can extract key elements of a data source as well as an
% accumulator tool that collects simple statistical information about
% the data.  The monomorphic interfaces allow users to develop
% ``one-of'' data source-specific applications.

