The \padsd{} language allows users to describe streams of data and
meta-data that we refer to as {\em feeds}.  
To introduce the central features of the language,
we work through a series of examples drawn from the CoMon monitoring system.

\begin{figure}[t]
\begin{code}
\kw{let} sites = 
  [
  "http://pl1.csl.utoronto.ca:3121";
  "http://plab1-c703.uibk.ac.at:3121";
  "http://pl1.csl.utoronto.ca:3121"
  ] 
\kw{feed} simple_comon =
  \kw{base} \{|
    \kw{sources}  = \kw{all} sites;
    \kw{schedule} = every 5 min, starting now, 
               timeout 60.0 sec; 
    \kw{format}   = Comon_format.Source;  
  |\}
\end{code}
\shrink
\caption{\texttt{simple\_comon.fml}:  Simple CoMon feed.} 
\label{fig:simplecomon}
\end{figure}

\begin{figure}[t]
\begin{code}
\kw{feed} comon_1 =
  \kw{base} \{|
    \kw{sources}  = \kw{any} sites;
    \kw{schedule} = every 1 min, 
              lasting 2 hours;
    \kw{format}   = Comon_format.Source;
  |\}
\end{code}
\shrink
\caption{\texttt{sites.fml}:  Code fragment for data from one of many sites.}
\label{fig:comon_1}
\end{figure}


%\subsection{CoMon Feeds}
\figref{fig:simplecomon} presents our first attempt to define
a simple CoMon statistics feed.  This description
specifies the \cd{simple\_comon} feed
using the \kw{base} feed constructor.  The \kw{sources} field
indicates that data for the feed comes from all of the locations
listed in \cd{sites}.  The \kw{schedule} field specifies that relevant
data is available from each source every five minutes, starting immediately.
When trying to fetch such data, the system may occasionally fail,
either because a remote machine is down or because of network
problems. To manage such errors, the schedule specifies that the
system should try to collect the data from each source for 60 seconds.
If the data does not arrive within that window, the system should give
up. Finally, the \kw{format} field indicates that the fetched data
conforms to the \padsml{}~\cite{mandelbaum+:pads-ml} description named
\cd{Source} defined in the file \cd{comon_format}.

In contrast to the \cd{simple\_comon} feed, which returns data from
all sites, the \cd{comon\_1} feed defined in \figref{fig:comon_1}
uses the \kw{any} constructor in the \kw{sources} field to 
return only a single value per time slice: that of the first site to
supply a complete set of data.
%
%
This feature is particularly useful when monitoring the
behavior of replicated systems, such as those using
state machine replication, consensus protocols, or even
loosely-coupled ones such as Distributed Hash Tables (DHTs) 
\cite{Balakrishnan+03:dht}.
%In these systems, the same data will be available from any
%of the functioning nodes, so receiving results from the first
%available node is sufficient. These kinds of monitoring systems
%are useful in the face of partial network unreachability or
%machine failure. Specifying this behavior at the language level
%provides a simpler implementation than network-centric approaches such
%as anycast \cite{anycast}. 

The schedule for \cd{comon\_1} indicates the system should fetch data
every minute for two hours, using the \cd{lasting} field to indicate
the duration of the feed.  It omits the \cd{starting} and
\cd{timeout} specifications, causing the system to use default
%settings of \cd{now} for the start time and 30 seconds for the
%timeout.  
settings for the start time and the timeout window.  



The \cd{simple\_comon} example hard-codes the set of locations from
which to gather performance data.  In reality, the CoMon system has an
Internet-addressable configuration file that contains a list of hosts
to be queried, one per non-comment line. This list is periodically
updated to reflect the set of active nodes in PlanetLab. 



\begin{figure}[t]
\begin{code}
(* Ocaml helper values and functions *)
\kw{let} config_location = 
 ["http://summer.cs.princeton.edu/status/ \\
   tabulator.cgi?table=slices/ \\
   table_princeton_comon&format=nameonly"]
\kw{let} makeURL (Nodelist.Data x) = 
     "http://" ^ x ^ ":3121"
\kw{let} old_locs = ref []
\kw{let} current list_opt =
  \kw{match} list_opt \kw{with}
    Some l ->  old_locs := l; l
  | None   -> !old_locs

(* Feed of nodes to query *)
\kw{feed} nodes =  
  \kw{base} \{|
    \kw{sources}  = \kw{all} config_location;
    \kw{schedule} = every 2 min;
    \kw{format}   = Nodelist.Source;
  |\}

(* Dependent CoMon feed of nodes feed *)
\kw{feed} comon =
  \kw{foreach} nodelist \kw{in} nodes 
  \kw{create}
    \kw{base} \{|
      \kw{sources}  = \kw{all} 
          (List.map makeURL 
          (List.filter Nodelist.is_node 
          (current nodelist)));
      \kw{schedule} = once, timeout 60.0 sec; 
      \kw{format}   = Comon_format.Source;
    |\}
\end{code}
\shrink
\caption{\texttt{comon.fml}: Uses feed of node locations to drive
  data collection.} \shrink
\label{fig:feedcomon}
\end{figure}

\begin{figure}[t]
\begin{code}
\kw{ptype} nodeitem =
  Comment of '#' * pstring_SE(peor)
| Data of pstring_SE(peor)

\kw{let} is_node item = 
  \kw{match} item \kw{with}
  Data _ -> true
  | _ -> false

\kw{ptype} source = 
  nodeitem precord plist (No_sep, No_term)
\end{code}
\shrink
\shrink
\caption{\texttt{nodelist.pml}: \padsml{} description for CoMon configuration 
  files.}\shrink
\label{fig:nodepml}
\end{figure}


\figref{fig:feedcomon} specifies a version of the \cd{comon} feed that
depends upon this configuration information.  To do so, the
description includes an auxiliary feed \cd{nodes} that describes the
configuration information: it is available from the
\cd{config_location}, it should be fetched every two minutes, and its
format is described by the \padsml{} description \cd{source} given in
the file \cd{nodelist.pml}, which appears in \figref{fig:nodepml}.
This \padsml{} description specifies that \cd{source} is a list of
new-line terminated records, each containing a \cd{nodeitem}.  In
turn, a \cd{nodeitem} is either a \cd{'#'} character followed by a
comment string, which should be tagged with the \cd{Comment}
constructor, or a host name, which should be tagged as
\cd{Data}. The description also defines a helper function \cd{is_node},
which returns true if the data item in question is a host name
rather than a comment.  Given this specification, the \cd{nodes} feed
logically yields a list of host names and comments every two minutes.
In fact, because of the possibility of errors, the feed actually
delivers a {\em list option} every two minutes: \cd{Some} if the list is
populated with data, \cd{None} if the data was unavailable at the
given time-slice.

Using the \cd{nodes} specification, we define the \cd{comon} feed as a
\textit{dependent} feed: each \cd{nodelist} in the \cd{nodes} feed defines a
collection of sources for the \cd{comon} feed.  The \cd{comon} \kw{sources}
specification processes the \cd{nodelist} to manage errors and strip
out comment fields.  The code that handles this processing illustrates
that the \padsd{} domain-specific language is embedded in \ocaml{}.
We use \ocaml{} terms where necessary to specify simple
transformations.  In particular, the \cd{current} function checks if
the \cd{nodelist} is \cd{None}, signaling a fetching error, in which
case it uses the most recently cached list of nodes instead.  The
\kw{source} specification filters out comment fields, and then
converts the host names to URLs with the required port using the
auxiliary function \cd{makeURL}.  The \kw{schedule} for this CoMon
feed is \cd{once} (with a timeout of sixty seconds) because we want to
collect the data for each host in a given \cd{hostlist} just once.
The \kw{foreach ... create} construct merges the resulting data from
each machine into a single feed.  As
before, the format of data fetched from each node matches the
description \cd{Comon\_format.Source}.  

With this specification, we
expect to get data from all the active machines listed in the
configuration file every two minutes.  We further expect the system to
notices changes in the configuration file within two minutes.

The previous examples all showcased feeds that contained a single type
of data.  \padsd{} also provides a datatype mechanism that allows us
to construct compound feeds containing data of different sorts.  
As an example where such a construct is useful, the CoMon system includes a
number of administrative data sources.  
%One example is a collection of
%node profiles, collecting the domain name, IP address, physical
%location, \etc, for each node in the cluster.  A second example is a
%list of authentication information for logging into the machines.
%These two data sources have different formats, locations, and update
%schedules, but system administrators want to keep a combined archive
%of the administrative information present in these sources.  
If \cd{sites\_mime} is a feed description of the profile
information of all the nodes on CoMon, and \cd{sites\_keyscan\_mime} is 
a feed of authentication information for each node, then the declaration
%
\begin{code}
\kw{feed} sites = 
    Locale \kw{of} sites_mime
  | Keyscan \kw{of} sites_keyscan_mime
\end{code}
%
creates a feed with elements drawn from each of the two 
feeds.  The constructors \cd{Locale} and \cd{Keyscan} tag each item in
the compound feed to indicate its source. 

% \subsection{\vizGems{} Example}
% We now shift to an example drawn from AT\&T's \vizGems{} project.  Like
% the earlier CoMon example, the \cd{stats} feed in \figref{fig:pulse}
% monitors a collection of machines described in a configuration file.
% The \cd{hostList} description has the same form as the \cd{nodes} feed
% we saw earlier, except it draws the data from a local file and only
% once a day.  Unlike the earlier example, we use a \textit{feed
% comprehension} to clean up this feed before creating the \cd{stats} feed.
% The comprehension filters
% the list of hosts to remove comments using the \cd{is_node} function
% and we use the built-in \cd{flatten} function to convert the feed of
% lists of hosts into a simple feed of hosts.  We use the \cd{mk\_host}
% function to remove the \cd{Data} constructor around the hostnames to
% simplify down-stream processing.

% The \cd{stats} feed depends upon the \cd{hosts} feed.  For each host
% \cd{h} in \cd{hosts}, it creates a schedule consisting of a single
% time corresponding to ``now'' with a timeout of one minute.  It uses
% this schedule \cd{s} to describe a compound feed, which pairs two base
% feeds: the first uses the Unix command \cd{ping} to collect network
% statistics about the remote machine while the second performs a remote
% shell invocation using \cd{ssh} to find out statistics about how long
% the machine has been up.  Both of these feeds use the \kw{proc}
% constructor in the \kw{sources} field to compute the data on the fly,
% rather than reading it from a file.  The argument to \kw{proc} is a
% string that is executed in a freshly constructed shell.  The pairing
% constructor for feeds takes a pair of feeds and returns a feed of
% pairs, with elements sharing the same scheduled fetch time being
% paired. This semantics is very convenient in this case, as it produces
% a feed that for each host returns a pair of its ping and uptime
% statistics, grouping together the information for each host. 
% Of course, the full \vizGems{} monitoring application
% uses many more tools than just ping and uptime to probe the remote
% machine; the corresponding feed description has many more branches than this simplified version.


% \begin{figure}
% \begin{code}
% \kw{let} config_location = 
%     [("file:///arrakis/config/machine_list")];

% \kw{feed} hostList =  
%   \kw{base} \{|
%    \kw{sources}  = \kw{all} config_location;
%    \kw{schedule} = every 1 day;
%    \kw{format}   = Nodelist.Source;
%  |\}

% \kw{let} mk_host (Nodelist.Data h) = h

% \kw{feed} hosts = 
%     \{| mk_host n | n <- (flatten hostList), 
%                    Nodelist.is_node n |\}

% \kw{feed} stats =
%   \kw{foreach} h \kw{in} hosts \kw{create}
%   \kw{let} s = once, timeout 1 min \kw{in}
%   (
%    \kw{base} \{| 
%       \kw{sources}  = \kw{proc} ("ping -c 1 " ^ h);   
%       \kw{format}   = Ping.Source;  
%       \kw{schedule} = s; |\},
%    \kw{base} \{| 
%       \kw{sources}  = \kw{proc} ("ssh " ^ h ^ " uptime");  
%       \kw{format}   = Uptime.Source;  
%       \kw{schedule} = s; |\}
%   ) 
% \end{code}
% \vskip -2ex
% \caption{\texttt{arrakis.fml} Simplified version of \vizGems{} feed.}
% \label{fig:pulse}
% \end{figure}


%% \subsection{\vizGems{} Example}
%% We now shift to an example drawn from AT\&T's \vizGems{} project.  Like
%% the earlier CoMon example, the \cd{stats} feed in \figref{fig:pulse2}
%% monitors a collection of machines described in a configuration file.
%% Before we discuss the \cd{stats} feed itself, we first explain some
%% auxiliary feeds that we use in the definition of the \cd{stats} feed.   

%% The \cd{raw\_hostLists} description has the same form as the \cd{nodes}
%% feed we saw earlier, except it draws the data from a local file once a
%% day.  We use a \textit{feed comprehension} to define a clean
%% version of the feed, \cd{host\_lists}.  In the comprehension, the
%% built-in predicate \cd{is\_good} verifies that no errors occurred in
%% fetching the current list of machines \cd{hl}, as would be expected
%% for a local file.  The function \cd{get\_hosts} takes \cd{hl} and uses
%% the built-in function \cd{get\_good} to unwrap the payload data from
%% the error infrastructure, an operation that is guaranteed to succeed
%% because of the \cd{is\_good} guard. The function \cd{get_hosts} then
%% selects the host name entries and unwraps them to produce a list of
%% unadorned host names.   

%% We next define a feed generator \cd{gen_stats} that yields an
%% integrated feed of performance statistics for each supplied host.  In
%% more detail, when given a host \cd{h}, \cd{gen_stats} creates a five
%% minute schedule with a one minute timeout. It then uses this schedule
%% to describe a compound feed, which pairs two base feeds: the first
%% uses the Unix command \cd{ping} to collect network statistics about
%% the route to \cd{h} while the second performs a remote shell
%% invocation using \cd{ssh} to gather statistics about how long the
%% machine has been up.  Both of these feeds use the \kw{proc}
%% constructor in the \kw{sources} field to compute the data on the fly,
%% rather than reading it from a file.  The argument to \kw{proc} is a
%% string that the system executes in a freshly constructed shell.  The
%% pairing constructor for feeds takes a pair of feeds and returns a feed
%% of pairs, with elements sharing the same scheduled fetch-time being
%% paired. This semantics conveniently produces a compound feed that for
%% each host returns a pair of its ping and uptime statistics, grouping
%% together the information for each host.  Of course, the full
%% \vizGems{} monitoring application uses many more tools than just ping
%% and uptime to probe remote machines; the full feed description has
%% many more branches than this simplified version. 


%% %% For each list of hosts in the
%% %% \cd{host_lists} feed, we generate a {\em feed of lists} using 
%% Finally, we define the feed \cd{stats}.  The most interesting piece
%% of this declaration is the {\em list feed comprehension}, given in
%% square brackets, that we use to generate a feed of lists. 
%% Given a host list \cd{hl}, the right-hand side of the comprehension
%% considers each host \cd{h} from \cd{hl} in turn.  The left-hand side of
%% the comprehension uses the \cd{gen_stats} feed generator to construct
%% a feed of the statistics for \cd{h}.  The list feed comprehension then
%% takes this collection of statistics feeds and converts them into a
%% single feed, where each entry is a list of the statistics for
%% the machines in \cd{hl} at a particular scheduled fetch-time.  
%% We call each such entry a \textit{snapshot} of the system.
%% The resulting feed makes it easy for down-stream users to perform
%% actions over snapshots, relieving them of the burden of having to
%% implement their own multi-way synchronization.

%% Given the list feed comprehension, the \kw{foreach...update} construct
%% generates a feed of snapshots from the feed of host lists.  Whenever a
%% new host list arrives, the \kw{foreach...update} construct
%% terminates the snapshot feed from the old host list and starts
%% generating a new snapshot feed from the new host list.  

%% Note the difference between the \kw{foreach...create} construct 
%% from the CoMon example and the \kw{foreach...update} construct.  The
%% create form generates a collection of feeds and merges their
%% contents into a single all-inclusive feed.  The update form
%% generates a collection of feeds and produces a single feed by
%% concatenating the collection, stopping one feed as soon as the next is
%% generated. 
%% We have found the create form to be useful when the actual arrival
%% times of the argument feed are regular because the regularity means we
%% can give a finite schedule for the dependent feed.  In contrast, the
%% update form is useful when the argument feed is irregular and we must
%% give an infinite schedule for the dependent feed to ensure we get the
%% desired values.  We define precise semantics for the create and update
%% forms in \secref{sec:semantics}.


%% % construct then merges this list of
%% % feeds into a single feed with data for all the hosts.  This merged
%% % feed delivers values for a given host list \cd{hl} until the next host
%% % list \cd{hl}' arrives from the feed \cd{host\_lists}, at which point
%% % it stops collecting data from the feeds generated from \cd{hl} and
%% % starts gathering data from the machines in \cd{hl}'.

%% % With this specification, we expect the system to return data for each
%% % machine listed in the configuration file every five minutes.  We will
%% % notice changes in the configuration file once a day.

%% \begin{figure}
%% \begin{code}
%% \kw{let} config_locations =
%%   [("file:///arrakis/config/machine_list")];

%% \kw{feed} raw_hostLists =  
%%   \kw{base} \{|
%%    \kw{sources}  = \kw{all} config_locations;
%%    \kw{schedule} = every 24 hours; 
%%    \kw{format}   = Hosts.Source;   |\}

%% \kw{let} get_host (Hosts.Data h) = h
%% \kw{let} get_hosts hl =
%%     List.map get_host 
%%      (List.filter Hosts.is_node (get_good hl))

%% \kw{feed} host_lists = 
%%   \{| get_hosts hl | hl <- raw_hostLists, 
%%                     is_good hl |\}

%% \kw{feed} gen_stats (h) = 
%%   \kw{let} s = every 5 mins, timeout 1 min \kw{in}
%%   (
%%    \kw{base} \{| 
%%      \kw{sources}  =  \kw{proc} ("ping -c 1 " ^ h);   
%%      \kw{format}   = Ping.Source;  
%%      \kw{schedule} = s; |\},
%%    \kw{base} \{| 
%%      \kw{sources}  = \kw{proc} ("ssh " ^ h ^ " uptime");  
%%      \kw{format}   = Uptime.Source;  
%%      \kw{schedule} = s; |\}
%%   )

%% \kw{feed} stats =
%%   \kw{foreach} hl \kw{in} host_lists \kw{update}
%%      [ gen_stats h | h <- hl ]
%% \end{code}
%% \vskip -2ex
%% \caption{\texttt{arrakis.fml}: Simplified version of \vizGems{} feed.}
%% \label{fig:pulse2}
%% \end{figure}
