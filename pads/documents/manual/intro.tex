\chapter{Introduction}
\label{chap:introduction}
Vast amounts of useful data are stored and processed in ad hoc formats.
Traditional databases and \xml{} systems provide rich infrastructure
for processing well-behaved data, but are of little help when dealing with ad hoc formats.
Examples that we face at AT\&T include call detail data~\cite{hancock-toplas}, 
%compiler traces~\cite{chilimbi},
web server logs~\cite{wpp}, 
netflows capturing internet traffic~\cite{netflow}, 
log files characterizing IP backbone resource utilization,
wire formats for legacy telecommunication billing systems, 
\etc{}
Such data may simply require processing before it can be loaded into a
data management system, or it may be too large or too transient to
make such loading cost effective.

Processing ad hoc data can be challenging for a variety of
reasons. First, ad hoc data typically arrives ``as is'': the analysts
who receive it can only say ``thank you,'' not request a more convenient format. 
Second, documentation for the format may not exist at all, or it may be
out of date.  A common phenomenon is for a field in a
data source to fall into disuse.  After a while, a new piece of
information becomes interesting, but compatibility issues prevent 
data suppliers from modifying the
shape of their data, so instead they hijack the unused field, often
failing to update the documentation in the process.

Third, such data frequently contain errors, for a variety of
reasons: malfunctioning equipment, race conditions on log
entry~\cite{wpp}, non-standard values to indicate ``no data
available,'' human error in entering data, unexpected data
values, \etc{} The appropriate response to such errors depends on the application. Some applications require the data to be error free: 
if an error is detected, processing needs to stop immediately and a human
must be alerted.  Other applications can repair the data, while still
others can simply discard erroneous or unexpected values.  
For some applications,
errors in the data can be the most interesting part  because
they can signal where two systems are failing to communicate.

A fourth challenge is that ad hoc data sources can be high volume:
AT\&T's call-detail stream contains roughly 300~million calls per day
requiring approximately 7GBs of storage space. Although this data is
eventually archived in a database, analysts mine it profitably before
such archiving~\cite{kdd98,kdd99}. More challenging, the \ningaui{} project at AT\&T
accumulates billing data at a rate
of 250-300GB/day, with occasional spurts of 750GBs/day. Netflow data
arrives from Cisco routers at rates over a gigabyte per
second~\cite{gigascope}! Such volumes mean it must be possible to
process the data without loading it all into memory at once.

Finally, before anything can be done with an ad hoc data source,
someone has to produce a suitable parser for it.
Today, people tend to use \C{} or \perl{} for this task.
Unfortunately, writing
parsers this way is tedious and error-prone, complicated by the lack
of documentation, convoluted encodings designed to save space, 
the need to produce efficient code,
and the need to handle errors robustly to avoid corrupting down-stream data.
Moreover, the parser writers' hard-won understanding of the data
ends up embedded in parsing code, making long-term maintenance
difficult for the original writers and sharing the knowledge with
others nearly impossible.

The \pads{} system makes life easier for data analysts by addressing
each of these concerns.\footnote{
  \pads{} is short for Processing Ad hoc Data Sources.
}
It provides a declarative data description
language that permits analysts to describe the physical layout of
their data, \textit{as it is}.  The language also permits analysts to
describe expected semantic properties of their data so that deviations can
be flagged as errors. The intent is to allow analysts to capture in a
\pads{} description all that they know about a given data source
and to provide the analysts with a library of useful routines in exchange. 


\pads{} descriptions are concise enough to
serve as documentation and flexible enough to describe most of
the data formats we have seen in practice, including ASCII, binary,
Cobol, and mixed data formats.  The fact that useful software
artifacts are generated from the descriptions provides strong
incentive for keeping the descriptions current, allowing them to serve
as living documentation.  

Given a \pads{} description, the \pads{} compiler produces
customizable \C{} libraries and tools for parsing, manipulating, and
summarizing the data. 
The core \C{} library includes functions for
reading the data, writing it back out in its original form, 
and accumulating statistical
properties.  
Future releases will include support for pretty printing the data,
writing it into a canonical \xml{} form, and implementing 
the Galax data model, which allows user to query \pads{} data using XQuery.
\cut{
pretty printing it in forms suitable for
loading into a relational database, 
writing it into a canonical \xml{} form, 

XXX: future release
An auxiliary library provides an instance of the data API
for Galax~\cite{galax,galaxmanual}, an implementation of XQuery.  This
library allows users to query data with a \pads{} description as if
the data were in \xml{} without having to convert to \xml{}.  In
addition to these libraries, the \pads{} system provides wrappers that
build tools to summarize the data, format it, or convert it to \xml{}.
}

The declarative nature of \pads{} descriptions facilitates the
insertion of error handling code.
The generated parsing code checks all possible error cases: system
errors related to the input file, buffer, or socket; syntax errors
related to deviations in the physical format; and semantic errors in
which the data violates user constraints.  Because these checks appear
only in generated code, they do not clutter the high-level declarative
description of the data source.
The result of a parse is a pair consisting of a canonical in-memory
representation of the data and a parse descriptor. The parse
descriptor precisely characterizes both the syntactic and the semantic
errors that occurred during parsing.  This structure allows analysts
to choose how to respond to errors in application-specific ways.  

With such huge datasets, performance is critical. The \pads{} system
addresses performance in a number of ways.  First, we compile the
\pads{} description rather than simply interpret it to reduce run-time
overhead.  Second, the generated parser provides multiple entry
points, so the data consumer can choose the appropriate level of
granularity for reading the data into memory to accommodate very large
data sources.  Finally, we parameterize library functions by
\textit{masks}, which allow data analysts to choose which semantic
conditions to check at run-time, permitting them to specify all known
properties in the source description without forcing all users of that
description to pay the run-time cost of checking them.

Given the importance of the problem, it is perhaps surprising that
more tools do not exist to solve it.  \xml{} and relational databases
only help with data already in well-behaved formats.  Lex and Yacc are
both over- and under- kill.  Overkill because the division into a
lexer and a context free grammar is not necessary for many ad hoc data
sources, and under-kill in that such systems require the user to build
in-memory representations manually, support only ASCII sources, and
don't provide extra tools.  ASN.1~\cite{asn} and related
systems~\cite{asdl} allow the user to specify an in-memory
representation and generate an on-disk format, but this doesn't help
when given a particular on-disk format.  Existing ad hoc description
languages~\cite{gpce02,sigcomm00,erlang} are steps in the right
direction, but they focus on binary, error-free data and they do not
provide auxiliary tools.



\section{\PADS{} language}
A \pads{} description specifies the physical layout and 
semantic properties of an ad hoc data source. 
The language provides a type-based model:
basic types describe atomic data such as integers, strings, dates, \etc{}, while
structured types describe compound data built from simpler pieces.
\suppressfloats

The \pads{} library provides a collection of broadly useful base
types.  Examples include 8-bit unsigned integers (\cd{Puint8}), 32-bit
integers (\cd{Pint32}), dates (\cd{Pdate}), strings (\cd{Pstring}),
and IP addresses (\cd{Pip}).  Semantic conditions for such base types
include checking that the resulting number fits in the indicated
space, \ie, 16-bits for \cd{Pint16}.  By themselves, these base types
do not provide sufficient information to allow parsing because they do
not specify how the data is coded, \ie{}, in ASCII, EBCDIC, or binary.
To resolve this ambiguity, \pads{} uses the \textit{ambient} coding,
which the programmer can set.  By default, \pads{} uses ASCII.  To
specify a particular coding, the description writer can select base
types which indicate the coding to use.  Examples of such types
include ASCII 32-bit integers (\cd{Pa_int32}), binary bytes
(\cd{Pb_int8}), and EBCDIC characters (\cd{Pe_char}).  In addition to
these types, users can define their own base types to specify more
specialized forms of atomic data.

To describe more complex data, \pads{} provides a collection of 
structured types loosely based on \C{}'s type structure.
In particular, \pads{} has 
\kw{Pstruct}s, \kw{Punion}s, and \kw{Parray}s to describe
record-like structures, alternatives, and sequences, respectively.
\kw{Penum}s describe a fixed collection of literals, while \kw{Popt}s 
provide convenient syntax for optional data.
Each of these
types can have an associated predicate that indicates whether a
value calculated from the physical specification is indeed a legal
value for the type.  For example, a predicate might require that two
fields of a \kw{Pstruct} are related or that the elements
of a sequence are in increasing order.  Programmers can specify such
predicates using \pads{} expressions and functions, 
written using a \C{}-like syntax.
Finally, \pads{} \kw{Ptypedef}s can be used
to define new types that add further constraints to existing types.

\pads{} types can be parameterized by values.
This mechanism
serves both to reduce the number of base types and to permit the
format and properties of later portions of the data to depend upon earlier portions.
For example, 
the base type \cd{Puint16_FW(:3:)} specifies an unsigned two byte integer
physically represented by exactly three characters, while the type
\cd{Pstring(:' ':)} 
describes a string terminated by a space.  Parameters can be 
used with compound types to specify the size of an array or which
branch of a union should be taken.


As an example, consider the common log format for Web server logs.  A
typical record looks like the following:
\begin{verbatim}
207.136.97.49 - - [15/Oct/1997:18:46:51 -0700] "GET /tk/p.txt HTTP/1.0" 200 30
\end{verbatim}

\noindent
recording the IP address of the requester; either a dash or the owner
of the TCP session; either a dash or the login of the requester; the
date; the actual request, which consists of the HTTP method, the
requested URL, the HTTP version number; a response code; and the
number of bytes returned.  A \PADSL{} type describing the request
portion is
\input{code/httpRequest}
This \pstruct{} uses (omitted) auxiliary types \cd{http_method_t} and
\cd{http_v_t} to describe
the HTTP method and version formats, respectively.
It uses character literals (\cd{'\\"'} and \cd{' '}) to consume
the quotes and 
spaces from the physical representation. 
The \cd{version} field has a constraint predicate \cd{checkVersion}
which ensures that obsolete HTTP methods \cd{LINK} and \cd{UNLINK} 
are only used with HTTP version \cd{1.0}.

\section{Generated library}
From a description, the \pads{} compiler generates a \C{} library
for parsing and manipulating the associated data source.  We chose \C{}
as the target language for pragmatic reasons: there were 
libraries that made building the compiler and run-time libraries easier,
our target users are comfortable with \C{}, and it can serve 
as a lingua franca in that essentially all languages have provisions for 
calling \C{} libraries.  Nothing about the \pads{} language mandates compiling
to \C{}, however, and we envision eventually building alternate bindings.

From each type in a \pads{} description, the compiler generates 
\begin{itemize}
\setlength{\itemsep}{0ex plus0.2ex}
\item an in-memory representation, 
\item a mask, which allows users to customize generated functions,
\item a parse descriptor, which describes syntactic and
semantic errors detected during parsing, 
\item parsing and printing functions, and 
\item a collection of utility functions.
\end{itemize}
%
\setcounter{totalnumber}{1}
\setcounter{dbltopnumber}{1}
\renewcommand{\topfraction}{0.85}
\renewcommand{\textfraction}{0.1}
\renewcommand{\floatpagefraction}{0.75}
\begin{figure*}
\begin{tiny}
\begin{code}
\kw{typedef} \kw{struct} \{
  Pbase\_m compoundLevel;   // Struct-level controls, eg., check Pwhere clause
  order\_header\_t\_m h;
  eventSeq\_t\_m events;
\} entry\_t\_m;
\mbox{}
\kw{typedef} \kw{struct} \{
  Pflags\_t pstate;         // Normal, Partial, or Panicking 
  Puint32 nerr;            // Number of detected errors.
  PerrCode\_t errCode;      // Error code of first detected error
  Ploc\_t loc;              // Location of first error
  order\_header\_t\_pd h;     // Nested header information
  eventSeq\_t\_pd events;    // Nested event sequence information
\} entry\_t\_pd;
\mbox{}
\kw{typedef} \kw{struct} \{
  order\_header\_t h;
  eventSeq\_t events;
\} entry\_t;
\end{code}

\begin{code}
/* Core parsing library */
Perror\_t entry\_t\_read (P\_t *pads,entry\_t\_m *m,entry\_t\_pd *pd,entry\_t *rep);
ssize\_t entry\_t\_write2io (P\_t *pads,Sfio\_t *io,entry\_t\_pd *pd,entry\_t *rep);
\end{code}

\begin{code}
/* Selected utility functions */
\kw{void} entry\_t\_m\_init (P\_t *pads,entry\_t\_m *mask,Pbase\_m baseMask);
\kw{int} entry\_t\_verify (entry\_t *rep);
\end{code}

\begin{code}
/* Selected accumulator functions */
Perror\_t entry\_t\_acc\_init (P\_t *pads,entry\_t\_acc *acc);
Perror\_t entry\_t\_acc\_add (P\_t *pads,entry\_t\_acc *acc,entry\_t\_pd *pd,entry\_t *rep);
Perror\_t entry\_t\_acc\_report (P\_t *pads,\kw{char} \kw{const} *prefix,\kw{char} \kw{const} *what,
                             \kw{int} nst,entry\_t\_acc *acc);
\end{code}

\begin{code}
/* Formatting */
ssize\_t entry\_t\_fmt2io (P\_t *pads,Sfio\_t *io,\kw{int} *requestedOut,
                        \kw{char} \kw{const} *delims,entry\_t\_m *m,entry\_t\_pd *pd,entry\_t *rep);
\end{code}

\begin{code}
/* Conversion to XML */
ssize\_t entry\_t\_write\_xml\_2io (P\_t *pads,Sfio\_t *io,entry\_t\_pd *pd,
                               entry\_t *rep,\kw{char} \kw{const} *tag,\kw{int} indent);
\end{code}

\begin{code}
/* Galax Data API */
PDCI\_node\_t *entry\_t\_node\_new (PDCI\_node\_t *parent,\kw{char} \kw{const} *name,\kw{void} *m,
                               \kw{void} *pd,\kw{void} *rep,\kw{char} \kw{const} *kind,\kw{char} \kw{const} *whatfn);
PDCI\_node\_t *entry\_t\_node\_kthChild (PDCI\_node\_t *self,PDCI\_childIndex\_t idx);
\end{code}


\caption{Selected portions of the library generated for the \texttt{entry\_t}
  declaration from \dibbler{} data description.}
\label{figure:library}
\end{tiny}
\end{figure*}
To give a feeling for the library that \pads{} generates, 
\figref{figure:library} includes selected portions of the generated 
library for the \dibbler{} \cd{entry_t} declaration.

The \C{} declarations for the in-memory representation, the mask, 
and the parse descriptor all share the structure of the \pads{}
type declaration.  The mapping to \C{} for each is straightforward: 
\kw{Pstruct}s map to \C{} structs with appropriately mapped fields, 
\kw{Punion}s map to tagged unions coded as \C{} structs with a tag field 
and an embedded 
union, \kw{Parray}s map to a \C{} struct with a length field and an 
embedded sequence, \kw{Penums} map to \C{} enumerations, \kw{Poptions} 
to tagged unions, and \kw{Ptypedef}s to \C{} typedefs.  Masks include
auxiliary fields to control behavior at the level of a structured
type, and parse descriptors include extra fields to record the 
state of the parse, the number of detected errors, 
the error code of the first detected error, and the location of that error.

The parser takes a mask as an argument and returns an
in-memory representation and a parse descriptor.  
The mask allows the user to specify 
which constraints the parser should check and which portions of the
in-memory representation it should fill in.  This control allows the
description-writer to specify all known constraints about the data
without worrying about the run-time cost of verifying potentially
expensive constraints for time-critical applications.

Appropriate error-handling can be as important as processing
error-free data.  The parse descriptor marks which portions of the
data contain errors and characterizes the detected errors.
Depending upon the nature of the errors and the desired application,
programmers can take the appropriate action: halting the program,
discarding parts of the data, or repairing the errors.
If the mask requests
that a data item be verified and set, and if the parse descriptor
indicates no error, then the in-memory representation satisfies the
semantic constraints on the data.

Because we generate a parsing function for each type in a \pads{} description,
we support multiple-entry point parsing, which allows us to 
accommodate larger-scale data.
For a small file, programmers can define a \pads{} type that describes
the entire file and use that type's parsing function to read the whole
file with one call.  For larger-scale data, programmers can sequence
calls to parsing functions that read manageable portions of the file,
\eg{}, reading a record at a time in a loop.  The parsing code generated
for \kw{Parrays} allows users to choose between reading the entire array
at once or reading it one element at a time, again to support parsing
and processing very large data sources.

The ratio of the size of the data description to the size of the generated code gives a rough measure of the leverage of the
declarative description.  For the 
68~line \dibbler{} data description, the compiler yields a 1432~\texttt{.h} file
and a 6471~\texttt{.c} file.  This expansion comes from the extensive error checking in the generated parser and the number of generated utility functions.

We discuss details of the generated library in the following section
as we describe its uses.

\section{Related work}
There are many tools for describing data formats. For example,
\textsc{ASN.1}~\cite{asn} and \textsc{ASDL}~\cite{asdl} are both
systems for declaratively describing data and then generating
libraries for manipulating that data.  In contrast to \PADS{},
however, both these systems specify the {\em logical\/} representation
and automatically generate a {\em physical\/} representation.
Although useful for many purposes, this technology does not help
process data that arrives in predetermined, \textit{ad hoc} formats.


More closely related work allows declarative descriptions of physical
data~\cite{sigcomm00,erlang-bit-syntax,gpce02}, motivated by parsing
\textsc{TCP/IP} packets and \java{} jar-files.  In contrast to our
work, these systems only handle binary data and assume the data is
error-free or halt parsing if an error is detected.  

\section{Terminology}
We use the term \textit{\external{}} to refer to the physical
data characterized by a \PADS{} description.  Such data might 
come from disk or over-the-wire for network applications.

\section{Notation}
In describing the syntax of various \PADSL{} expressions, we will use
a BNF grammar.  The syntax for pervasive features of the languages
appears in \chapref{chap:common-features}.  The syntax of a particular
feature appears in the chapter describing that feature.
We adopt the following conventions:
\begin{itemize}
\item \kw{Keywords}
\item \opt{Expressions} are optional.
\item \nont{Non-terminals}

\end{itemize}

\section{How do I read this manual?}
In \chapref{chap:example}, we describe a sample use of the \PADS{}
system. It provides an overview of the data description language and
illustrates the use of the generated library.
\chapref{chap:common-features} describes features of the \PADS{}
language common to all \padsl{} types.  Successive chapters assume
familiarity with the material in this chapter.
\chapref{chap:base-types} describes built-in \padsl{} base types, while
Chapters \ref{chap:structs} through \ref{chap:typedefs} describe the
structured types that \padsl{} provides. These chapters describe the
syntax and semantics of the \pads{} language and the core of the
library generated for each such type declaration.
In \chapref{chap:library-use}, we document how to use the \pads{}
library, and in \chapref{chap:library-customization} we explain the various
ways in which the \pads{} library can be customized. 
\chapref{chap:accumulators} describes the generated accumulator
library.


\section{Getting the \PADS{} system}
Source code for \PADS{} available for download with
a non-commercial use license from: 
\begin{centercode}
http://www.research.att.com/projects/pads
\end{centercode}

The \PADS{} distribution contains:
\begin{itemize}
\item this manual
\item source code 
\item README file describing installation procedure.
\end{itemize}

\section{Using the \PADS{} compiler}
This section describes how to use the \PADS{} compiler \texttt{padsc}.
The simplest use of \texttt{padsc} is to compile a \PADSL{} source file
to a C header and implementation file.  The command
\begin{centercode}
padsc my.p
\end{centercode} %
will produce my.h and my.c

Describe switches

picture linking generated library with core library and main.c to
produce executable.

\section{Practical notes}
\pads{} uses the \texttt{Sfio} library for managing I/O
streams. \texttt{Sfio} provides functionality similar to that of
\texttt{Stdio}, the ANSI C Standard I/O library, but via a distinct
interface.  Sfio provides both source and binary \texttt{Stdio}
emulation packages. \texttt{Sfio} is included in the \pads{}
distribution.  More information about \texttt{Sfio} is available from
\url{www.research.att.com/sw/tools/sfio}.


Please report any bugs in the \PADS{} implementation or problems with
this manual by electronic mail to 
\[
\texttt{pads@research.att.com}.
\]