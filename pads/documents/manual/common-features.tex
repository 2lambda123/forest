\chapter{Common features}
In this chapter, we describe \PADSL{} features shared by all types. 
Subsequent chapters describe features particular to various \PADSL{}
types. 

\section{Overall structure}
\label{sec:common-overall}
\begin{tabular}{rcll}
\nont{p\_ty} & \is{} & \nont{base\_ty} \\[1ex]
& \alt{} & \nont{struct\_ty} \\[1ex]
& \alt{} & \nont{union\_ty} \\[1ex]
& \alt{} & \nont{array\_ty} \\[1ex]
& \alt{} & \nont{typedef\_ty} \\[1ex]
& \alt{} & \nont{enum\_ty} \\[1ex]
\end{tabular}



\section{Comments}
\label{sec:common-comments}
c-style and /-- form, which are reflected into output.

\section{Predicates}
\label{sec:common-predicates}
 written in C. integer type
 assumed to be side-effect free
 \term{predicate}

\section{Literals}
\label{sec:common-literals}

\begin{tabular}{l}
\term{char\_lit} \\[1ex]
\term{str\_lit} \\[1ex]
\end{tabular}

\PADSL{} supports C-style character and string literals such as {\tt '}\cd{a}{\tt '} and
\cd{"foo"}.  These literals may use all of C character escapes such as
\begin{verbatim}\" and \'\end{verbatim}.

Character and string literals are used to read external characters
without putting any corresponding data in the resulting in-memory representation.
\Ie, they 'skip over' some expected character(s).
They are typically used in \Pstruct{}s; see \secref{sec:sec:structs-literal-fields}.

\section{Character Sets}
\label{sec:common-character-sets}

The library discipline contains a field \cd{def_charset} that
indicates the expected character set of the external representation of
character and string literals, as well as the external representation of all character and string
base types that do not explicitly name a character set.  Supported
character sets include ASCII (\cd{PDC_charset_ASCII}) and EBCDIC
(\cd{PDC_charset_EBCDIC}), where ASCII is the default character set.
\secref{sec:library-customization-character-encodings} describes how
to set \cd{def_charset}.

\section{Parameterization}
\label{sec:common-parameterization}
 mechanism to pass values from earlier portions of data to later
 portions
 
 reduce number of type declarations.

\nont{formal\_param\_list}
\nont{actual\_param\_list}

\section{\Precord{} modifier}

\section{\Pfile{} modifier}

\section{Error model}
\label{sec:common-error-model}
 panic versus non-panic errors

\appref{app:error-codes} contains a list of all error codes and
describes their meanings.

\section{In-memory representations}
\label{sec:common-rep}
Each \PADS{} type \cd{foo} has an associated in-memory representation
type of the same name.  The structure of this representation depends
upon the particular \PADS{} type.  
In general, these representations fall into two broad categories:
\textit{static} representations, whose size can be computed at
library-generation time, and \textit{dynamic} representations, whose size 
depends on the data being parsed. 
Details appear in 
\secref{sec:base-types-rep}, \secref{sec:structs-rep},
\secref{sec:unions-rep}, \secref{sec:arrays-rep}, 
\secref{sec:enums-rep}, and \secref{sec:typedefs-rep}.

\section{Masks}
\label{sec:common-masks}
Each \PADS{} type \cd{foo} has an associated mask type, \csmSuf{foo}.
Masks allow the library user to customize operations to portions of
the associated values.  The structure of the mask for a given \PADS{}
type varies depending on the structure of the \PADS{} type.  Details
appear in \secref{sec:base-types-masks}, \secref{sec:structs-masks},
\secref{sec:unions-masks}, \secref{sec:arrays-masks}, 
\secref{sec:enums-masks}, and \secref{sec:typedefs-masks}.

\section{Parse descriptors}
\label{sec:common-parse-descriptor}
Each \PADS{} type \cd{foo} has an associated parse descriptor type, 
\pdSuf{foo}, coded as a \C{} struct with at least the following four
fields:
\vskip 1ex
\begin{tabular}{lp{2.25in}}
 \cd{PDC\_uint32 pstate} & Flags that describe the state after parsing the
                           associated value.\\[1ex]
 \cd{PDC\_errCode\_t errCode} & A code indicating the nature of the 
                               first detected error.\\[1ex]
 \cd{PDC\_loc\_t loc}  & The location in the data source of the first 
                         error.\\[1ex]
 \cd{PDC\_uint32 nerr} & The number of errors detected during parsing
                        of the associated value.\\[1ex]
\end{tabular}

\noindent
Field \cd{pstate} contains the following flags:
\vskip 1ex
\begin{tabular}{lp{3.5in}}
 \cd{PDC\_Panic} & Set if the parser was in panic mode during the
                  parsing of the associated data.\\[1ex]
\end{tabular}

\noindent
The \PADS{} library provides a collection of functions (macros
actually) for manipulating the parse state field:
\vskip 1ex
\begin{code}
void PDC_PS_init(void *pd);         
void PDC_PS_setPanic(void *pd);     
void PDC_PS_unsetPanic(void *pd);   
int  PDC_PS_isPanic(void *pd);      
\end{code}

Details about how parse descriptors are customized for various \PADS{}
types appear in
\secref{sec:base-types-parse-descriptors}, \secref{sec:structs-parse-descriptors},
\secref{sec:unions-parse-descriptors}, \secref{sec:arrays-parse-descriptors}, 
\secref{sec:enums-parse-descriptors} and \secref{sec:typedefs-parse-descriptors}.

\section{Operations}
\label{common-operations}
\subsection{Initialization and cleanup functions}
\begin{code}
PDC_error_t foo_init (PDC_t *pdc, foo *rep);

PDC_error_t foo_cleanup (PDC_t *pdc, foo *rep);

PDC_error_t foo_pd_init (PDC_t *pdc, foo_pd *pd);

PDC_error_t foo_pd_cleanup (PDC_t *pdc, foo_pd *pd);

void foo_m_init (PDC_t *pdc, foo_m *mask, PDC_base_m baseMask);

\end{code}

\subsection{Utilty functions}
\begin{code}
PDC_error_t foo_copy (PDC_t *pdc, foo *rep_dst, foo *rep_src);

PDC_error_t foo_pd_copy (PDC_t *pdc, foo_pd *pd_dst, foo_pd *pd_src);
\end{code}

\subsection{Reading functions}
\begin{code}
PDC_error_t foo_read (PDC_t *pdc, foo_m *m, foo_pd *pd, foo *rep);
\end{code}

\subsection{Writing functions}
\begin{code}
ssize_t foo_write2buf (PDC_t *pdc, PDC_byte *buf, size_t buf_len, 
                       int *buf_full, foo_pd *pd, foo *rep);

ssize_t foo_write2io (PDC_t *pdc, Sfio_t *io, foo_pd *pd, foo *rep);

\end{code}

\subsection{Accumulator functions}
\begin{code}
PDC_error_t foo_acc_init (PDC_t *pdc, foo_acc *acc);

PDC_error_t foo_acc_reset (PDC_t *pdc, foo_acc *acc);

PDC_error_t foo_acc_cleanup (PDC_t *pdc, foo_acc *acc);

PDC_error_t foo_acc_add (PDC_t *pdc, foo_acc *acc, foo_pd *pd, foo *rep);

PDC_error_t foo_acc_report2io (PDC_t *pdc, Sfio_t *outstr, 
                               char const *prefix, char const *what, 
                               int nst, foo_acc *acc);

PDC_error_t foo_acc_report (PDC_t *pdc, char const *prefix, 
                            char const *what, int nst, foo_acc *acc);

\end{code}
