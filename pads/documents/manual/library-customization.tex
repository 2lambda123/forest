\chapter{Library customization}
\label{chap:library-customization}
\cutname{library_customization.html}
The \pads{} core library is parameterized by a main discpline and by
an IO discipline that allow users to customize the behavior of the
core system and the IO subsystem, respectively.  One aspect of
controlling IO is choosing what constitues a \textit{Precord} in the
data source.

\section{The \pads{} discipline}
\subsection{Skipping white space}
\label{sec:library-customization-white-space}
Describe how to set various white space skipping modes.

\subsection{Scanning extent}
\label{sec:library-customization-scanning-extent}

\subsection{Endian-ness}
\label{sec:library-customization-endian}

\subsection{Character encodings}
\label{sec:library-customization-character-encodings}
how set discipline to ascii, ibcdic.

\section{The IO Discpline}
IO discipline values, which have type \cd{Pio\_disc\_t}, control the
'raw' reading of data from a file or from some other data source.  
The \pads{} system provides a collection of functions for generating
various IO disciplines, corresponding to various kinds of record
structures: new-line terminated, fixed width, IBM-style (initial data
indicating size of record, followed by payload), \etc{}  In addition,
the discipline indicates if the data source is seekable (a file) or
not (a stream).

To use an IO discipline, the user first creates one by invoking
a creation function supplied by the \pads{} system.  The resulting IO
discipline is then installed by passing it as an argument to either
\cd{P\_open} or to \cd{P\_set\_io\_disc}.

\begin{description}
\item
[\small{\cd{Pio\_disc\_t * P\_fwrec\_make(size\_t leader\_len, size\_t data\_len, size\_t trailer\_len)}}]
 Instantiates an instance of \cd{fwrec}, a discipline for fixed-width
 records.  The parameter \cd{data\_len} specifies the number of data bytes per record,
 while \cd{leader\_len} and \cd{trailer\_len} specify the number of bytes that
 occur before and after the data bytes within each record (either or
 both can be zero).  Thus the total record size in bytes is the sum
 of the three arguments.  

\item[\small{\cd{Pio\_disc\_t * P\_fwrec\_noseek\_make(size\_t leader\_len,
       size\_t data\_len, size\_t trailer\_len)}}]
Instantiates an instance of \texttt{fwrec\_noseek}, a version of \texttt{norec}
that does not require that the SFIO stream is seekable.

\item[\small{\cd{Pio\_disc\_t * P\_ctrec\_make(Pbyte termChar, size\_t block\_size\_hint);}}]
Instantiates an instance of \cd{ctrec}, a discipline for
character-terminated variable-width records. Argument \texttt{termChar} is the
character that marks the end of a record. Argument
\texttt{block\_size\_hint} suggests a block size to use, if the
discipline chooses to do fixed block-sized reads 'under the covers'.
It may be ignored by the discipline.
For ASCII newline-terminated records use, \literal{\cd{'\\n'}} or
\cd{P\_ASCII\_NEWLINE} 
as the term character.  For \cd{EBCDIC} newline-terminated records, use
\cd{P\_EBCDIC\_NEWLINE} as the term character.


\item[\small{\cd{Pio\_disc\_t * P\_ctrec\_noseek\_make(Pbyte termChar,
      size\_t block\_size\_hint)}}] 
Instantiates an instance of \cd{ctrec\_noseek}, a version of \cd{norec}
that does not require that the SFIO stream is seekable.

\item[\small{\cd{Pio\_disc\_t * P\_nlrec\_make(size\_t block\_size\_hint)}}]
Shorthand for calling the corresponding \cd{ctrec} make function with
\literal{\cd{'\\n'}} as \cd{termChar}.

\item[\small{\cd{Pio\_disc\_t * P\_nlrec\_noseek\_make(size\_t block\_size\_hint)}}]
Shorthand for calling the corresponding \cd{ctrec} make function with
\literal{\cd{'\\n'}} as \cd{termChar}.



\item[\small{\cd{Pio\_disc\_t * P\_vlrec\_make(int blocked, size\_t avg\_rlen\_hint)}}]
 Instantiates an instance of \cd{vlrec}, a discipline for IBM-style
 variable-length records with record length specified at the start
 of each record.  If blocked is set (\cd{!= 0}) then the records are
 grouped into blocks, where each block has a length given at the
 start of each block.  Argument \cd{avg\_rlen\_hint} is a hint as to what the
 average record length is, to help the discipline allocate memory.
 It should include the four bytes at the start of each record used for
 the record length.  It may be ignored by the discipline.
 

\item[\small{\cd{Pio\_disc\_t * P\_vlrec\_noseek\_make(int blocked,
 size\_t avg\_rlen\_hint)}}] Instantiates an instance of
 \cd{vlrec\_noseek}, a version of \cd{vlrec} that does not require
 that the SFIO stream is seekable.


\item[\small{\cd{Pio\_disc\_t * P\_norec\_make(size\_t block\_size\_hint)}}]
Instantiates an instance of \cd{norec}, a raw bytes discipline that
does not use records.  Argument \cd{block\_size\_hint} is a hint as to what block size
to use, if the discipline chooses to do fixed block-sized reads
'under the covers'.  It may be ignored by the discipline.


\item[\small{\cd{Pio\_disc\_t * P\_norec\_noseek\_make(size\_t block\_size\_hint)}}]
Instantiates an instance of \cd{norec\_noseek}, a version of \cd{norec}
that does not require that the SFIO stream is seekable.


\end{description}



\subsection{Closing an IO discipline}
When an IO discipline is no longer needed, the user should unmake it.
The function \cd{P\_io\_disc\_unmake} explicitly deallocates an IO
discipline. In addition, the function \cd{P\_close}
deallocates the installed IO discipline.  
The function \cd{P\_set\_io\_disc} deallocates the previously
installed discipline.
If desired, an IO discipline can be preserved using
\cd{P\_close\_keep\_io\_disc} or \cd{P\_set\_io\_disc\_keep\_old}, in
which case it can be re-used in a future \cd{P\_open} or
\cd{P\_set\_io\_disc} call. 

\subsection{Implementations}
Implementations of the standard IO disciplines can be found in
\texttt{libpads/default\_io\_disc.c}.  Anyone planning to implement a new IO
discipline should consult \texttt{default\_io\_disc.c}.



\section{Adding new base types}
\label{sec:library-adding-new-base-types}
