\chapter{Library customization}
\label{chap:library-customization}
\cutname{library_customization.html}
The \pads{} core library is parameterized by a main discpline and by
an IO discipline that allow users to customize the behavior of the
system.


\section{The IO Discpline and Defining Records}
IO discipline values, which have type \cd{Pio\_disc\_t}, control the
'raw' reading of data from a file or from some other data source.  
The \pads{} system provides a collection of functions for generating
various IO disciplines, corresponding to various kinds of record
structures: new-line terminated, fixed width, IBM-style (initial data
indicating size of record, followed by payload), \etc{}  In addition,
the discipline indicates if the data source is seekable (a file) or
not (a stream).

To use an IO discipline, the user first creates one by invoking
a creation function supplied by the \pads{} system.  The resulting IO
discipline is then installed by passing it as an argument to either
\cd{P\_open} or to \cd{P\_set\_io\_disc}.

\subsection{Closing an IO discipline}
When an IO discipline is no longer needed, the user should unmake it.
The function \cd{P\_io\_disc\_unmake} explicitly deallocates an IO
discipline. In addition, the function \cd{P\_close}
deallocates the installed IO discipline.  
The function \cd{P\_set\_io\_disc} deallocates the previously
installed discipline.
If desired, an IO discipline can be preserved using
\cd{P\_close\_keep\_io\_disc} or \cd{P\_set\_io\_disc\_keep\_old}, in
which case it can be re-used in a future \cd{P\_open} or
\cd{P\_set\_io\_disc} call. 

\begin{verbatim}
Pio_disc_t * P_fwrec_make(size_t leader_len, size_t data_len, size_t trailer_len);
/* Instantiates an instance of fwrec, a discipline for fixed-width
 * records.  data_len specifies the number of data bytes per record,
 * while leader_len and trailer_len specifies the number of bytes that
 * occur before and after the data bytes within each record (either or
 * both can be zero).  Thus the total record size in bytes is the sum
 * of the 3 arguments.  
 */

Pio_disc_t * P_fwrec_noseek_make(size_t leader_len, size_t data_len, size_t trailer_len);
/* Instantiates an instance of fwrec_noseek, a version of norec
 * that does not require that the sfio stream is seekable.
 */

Pio_disc_t * P_ctrec_make(Pbyte termChar, size_t block_size_hint);
/* Instantiates an instance of ctrec, a discipline for
 * character-terminated variable-width records. termChar is the
 * character that marks the end of a record. block_size_hint is a
 * hint as to what block size to use, if the discipline chooses to do
 * fixed block-sized reads 'under the covers'.  It may be ignored by
 * the discipline.
 * 
 * For ASCII newline-terminated records use, '\n' or P_ASCII_NEWLINE
 * as the term character.  For EBCDIC newline-terminated records, use
 * P_EBCDIC_NEWLINE as the term character.
 */

Pio_disc_t * P_ctrec_noseek_make(Pbyte termChar, size_t block_size_hint);
/* Instantiates an instance of ctrec_noseek, a version of norec
 * that does not require that the sfio stream is seekable.
 */

Pio_disc_t * P_vlrec_make(int blocked, size_t avg_rlen_hint);
/* Instantiates an instance of vlrec, a discipline for IBM-style
 * variable-length records with record length specified at the start
 * of each record.  If blocked is set (!= 0) then the records are
 * grouped into blocks, where each block has a length given at the
 * start of each block.  avg_rlen_hint is a hint as to what the
 * average record length is, to help the discipline allocate memory.
 * It should include the 4 bytes at the start of each record used for
 * the record length.  It may be ignored by the discipline.
 */

Pio_disc_t * P_vlrec_noseek_make(int blocked, size_t avg_rlen_hint);
/* Instantiates an instance of vlrec_noseek, a version of vlrec
 * that does not require that the sfio stream is seekable.
 */

Pio_disc_t * P_norec_make(size_t block_size_hint);
/* Instantiates an instance of norec, a raw bytes discipline that
 * does not use records.  block_size_hint is a hint as to what block size
 * to use, if the discipline chooses to do fixed block-sized reads
 * 'under the covers'.  It may be ignored by the discipline.
 */

Pio_disc_t * P_norec_noseek_make(size_t block_size_hint);
/* Instantiates an instance of norec_noseek, a version of norec
 * that does not require that the sfio stream is seekable.
 */

/* Shorthands for calling corresponding ctrec make functions with '\n' as termChar: */
#define P_nlrec_make(block_size_hint)         P_ctrec_make('\n', block_size_hint)
#define P_nlrec_noseek_make(block_size_hint)  P_ctrec_noseek_make('\n', block_size_hint)

/* Pio_elt_t: used for list of input records managed by the io
 * discipline.  The io discipline maintains a doubly-linked list of
 * these records using the prev/next fields, where the head of the
 * list is always a 'dummy' record that is not used except as a
 * placeholder for managing the list.
 * 
 * XXX_TODOC: offset, begin, end, etc.
 *
 * There is an extra data fields, disc_ptr, which is optionally used by
 * the io discipline and ignored by the main library code.
 */
\end{verbatim}

Implementations of the standard IO disciplines can be found in
\texttt{libpads/default\_io\_disc.c}.  Anyone planning to implement a new IO
discipline should consult \texttt{default\_io\_disc.c}.


\section{Skipping white space}
\label{sec:library-customization-white-space}
Describe how to set various white space skipping modes.

\section{Scanning extent}
\label{sec:library-customization-scanning-extent}

\section{Endian-ness}
\label{sec:library-customization-endian}

\section{Character encodings}
\label{sec:library-customization-character-encodings}
how set discipline to ascii, ibcdic.

\section{Adding new base types}
\label{sec:library-adding-new-base-types}
