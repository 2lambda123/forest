\chapter{Library customization}
\label{chap:library-customization}
\cutname{library_customization.html}
The \pads{} core library is parameterized by a main discpline and by
an IO discipline that allow users to customize the behavior of the
core system and the IO subsystem, respectively.  One aspect of
controlling IO is choosing what constitues a \textit{Precord} in the
data source.

\section{The \pads{} discipline}
A \pads{} handle, \ie{}, a value of type \cd{P\_t}, contains a field
named \cd{my\_disc} of type \cd{Pdisc\_t}, which appears in
\figref{fig:pdisc}. The various fields of \cd{my\_disc} allow users to
customize aspects of the \pads{} system.  We describe these fields in
the following sections.

\begin{figure}
\inputCode{code/pdisc}
\caption{The \texttt{Pdisc\_t} type, which allows users to customize the
  behavior of the \pads{} system.}
\label{fig:pdisc}
\end{figure}

\subsection{Character encodings}
\label{sec:library-customization-character-encodings}
how set discipline to ascii, ibcdic.

\subsection{Copying strings} 
\label{sec:library-customization-copy-strings}
 If \cd{copy\_string} field in \pads{} discipline is non-zero, the string read functions
 copy the strings found into the supplied representation, otherwise they do not.
 Instead, the target \cd{Pstring} points to memory managed by the current
 IO discipline.  This sharing avoids copying and speeds programs that
 will never reference an old record after a new one is read in.
Field \cd{copy\_strings} should only be set to zero for record-based IO disciplines where
strings from record K are not used after \cd{P\_io\_next\_rec} has been called to move
the IO cursor to record K+1.  Note: \cd{Pstring\_preserve} can be used to
force a string that is using sharing to make a copy so that the string is 'preserved'
(remains valid) across calls to \cd{P\_io\_next\_rec}.

\subsection{Input time zone}
\label{sec:library-customization-input-time-zone}
The field \cd{in\_time\_zone} specifies the default time zone for
string-based time input, used for input date strings 
that do not have time zone information in them.     For example, the date
\texttt{15/Oct/1997:18:46:51} has no time zone information.  If
\cd{in\_time\_zone} is set  to \literal{\cd{"UTC"}}, 
then this date/time would be assumed to be a UTC time.
In contrast, regardless of the \cd{in\_time\_zone} setting, the date
\texttt{15/Oct/1997:18:46:51 -0700}
will always be interpreted as being in a time zone seven hours
earlier than UTC time.

The \cd{in\_time\_zone} is passed to the \cd{tmzone} function, so it
must be a time zone description that \cd{tmzone}
understands. Intuitively, it accepts three-letter strings such as 
\literal{\cd{"PST"}}  and \literal{\cd{"EDT"}}  as well as strings
denoting numeric offsets from UTC time, such as \literal{\cd{"-0500"}}.
Documentation for the \cd{tmzone} function appears on the web page:
\ifthenelse{\boolean{hevea}}{
\myurl{www.research.att.com/\~gsf/man/man3/tm.html}}{
\myurl{www.research.att.com/~gsf/man/man3/tm.html}}

Before calling \cd{P\_open}, the discipline field \cd{disc->in\_time}
can be initialized directly.  After calling \cd{P\_open}, however, it
must be changed by passing the pads handle and a time zone string to 
\cd{P\_set\_in\_time\_zone}, \eg{},

\begin{centercode}
    P\_set\_in\_time\_zone(pads,\literal{"PST"});
\end{centercode}

This will set \cd{pads->disc->in\_time\_zone}, and will also update
an internal representation of the time zone maintained as part of
the pads state.

\subsection{Output time zone}
\label{sec:library-customization-output-time-zone}
This field specifies the output time zone for formatted time output.
Regardless of the time zone used to read in a time,
\cd{disc->output\_time\_zone} controls which time zone is used when
formatting the time for output.  For example, a time that is read as 6am UTC time
would be formatted as 1am if the \cd{output\_time\_zone} is \cd{\literal{"-0500"}}.
Note that in the normal case you should use the same time zone
for both input and output, unless you are intentially translating
times from one time zone to another one. The format of output time
zone specification strings is the same as for input time zone.

Before calling \cd{P\_open}, the discipline field \cd{disc->in\_time}
can be initialized directly.  After calling \cd{P\_open}, however, it
must be changed by passing the pads handle and a time zone string to 
\cd{P\_set\_output\_time\_zone}, \eg{},

\begin{centercode}
    P\_set\_output\_time\_zone(pads,\literal{"CDT"});
\end{centercode}

This will set \cd{pads->disc->output\_time\_zone}, and will also update
an internal representation of the output time zone maintained as part of
the pads state.

\subsection{Input formats}
\label{sec:library-customization-input-formats}
The \cd{in\_formats} field of the discipline allows one to specify
default input formats for some special types where there is 
in no 'obvious' default. \figref{fig:input-formats} contains the type
of this field.
\begin{figure}
\inputCode{code/pinformats}
\caption{The \texttt{Pin\_formats\_t} type, which allows users to specify the
  input format of various \pads{} base types. Each of the fields must
  be a non-null string with a format understood by the \texttt{tmdate} function.}
\label{fig:input-formats}
\end{figure}
The current entries are:

\begin{description}
\item[\cd{in\_formats.timestamp}]
This field contains a format string specifying the input timestamp
format, for use with \cd{Ptimestamp} and its variants.  Alternatives
can be given using \cd{\%|}, and the special \cd{\%\&} format can be
used to indicate all formats that can be parsed by the
\cd{tmdate} fuction.  The default,

\inputCode{code/timestamp-format}
%
\noindent
allows for timestamps of these forms:
\begin{verbatim}
 091172+230202
 091172+23:02:02
 09111972+230202
 09111972+23:02:02
\end{verbatim}
 and also allows for all date/time formats parsed by \cd{tmdate}.
Documentation for the \cd{tmdate} function appears on the web page:
\ifthenelse{\boolean{hevea}}{
\myurl{www.research.att.com/\~gsf/man/man3/tm.html}}{
\myurl{www.research.att.com/~gsf/man/man3/tm.html}}
 
\item[\cd{in\_formats.date}]
 A format string specifying the input date format, for use with
 \cd{Pdate} and its variants.  The default, 

\inputCode{code/date-format}
%
\noindent
 allows for dates of these two forms:
\begin{verbatim}
   091172
   09111972
\end{verbatim}
and also allows for all date formats parsed by \cd{tmdate}.
Documentation for the \cd{tmdate} function appears on the web page:
\ifthenelse{\boolean{hevea}}{
\myurl{www.research.att.com/\~gsf/man/man3/tm.html}}{
\myurl{www.research.att.com/~gsf/man/man3/tm.html}}

\item[\cd{in\_formats.tme}]
 A format string specifying the input time format, for use with
 \cd{Ptime} and its variants.  The default, 

\inputCode{code/time-format}
%
\noindent
 allows for times of these two forms:
\begin{verbatim}
  230202
  23:02:02
\end{verbatim}
and also allows for all date formats parsed by \cd{tmdate}.
Documentation for the \cd{tmdate} function appears on the web page:
\ifthenelse{\boolean{hevea}}{
\myurl{www.research.att.com/\~gsf/man/man3/tm.html}}{
\myurl{www.research.att.com/~gsf/man/man3/tm.html}}

\end{description}

\subsection{Output formats}
\label{sec:library-customization-output-formats}
The \cd{out\_formats} field of the discipline allows one to specify
default output formats for some special types where there is 
in no 'obvious' default. \figref{fig:output-formats} contains the type
of this field.
\begin{figure}
\inputCode{code/poutformats}
\caption{The \texttt{Pout\_formats\_t} type, which allows users to specify the
  output format of various \pads{} base types. Each of the fields must
  be a non-null string with a format understood by the \texttt{fmttime} function.}
\label{fig:output-formats}
\end{figure}
The current entries are:
\begin{description}
\item[\cd{out\_formats.timestamp}]
\item[\cd{out\_formats.timestamp\_explicit}]
These two values specifying the default output formats for the \cd{Ptimestamp}
and \cd{Ptimestamp\_explicit} families of types, respectively.  The normal use is for these formats
to describe both the date and time of day.  Some examples:

\inputCode{code/timestamp-output-format}
%
\noindent
Documentation for the \cd{tmdate} function appears on the web page:
\ifthenelse{\boolean{hevea}}{
\myurl{www.research.att.com/\~gsf/man/man3/tm.html}}{
\myurl{www.research.att.com/~gsf/man/man3/tm.html}}


\item[\cd{out\_formats.date}]
\item[\cd{out\_formats.date\_explicit}]
These two values specifying the default output formats for the \cd{Pdate}
and \cd{Pdate\_explicit} families of types, respectively.  The normal use is for these formats
to describe the date but not the time of day.  Some examples:

\inputCode{code/date-output-format}
%
\noindent
Documentation for the \cd{tmdate} function appears on the web page:
\ifthenelse{\boolean{hevea}}{
\myurl{www.research.att.com/\~gsf/man/man3/tm.html}}{
\myurl{www.research.att.com/~gsf/man/man3/tm.html}}



\item[\cd{out\_formats.time}]
\item[\cd{out\_formats.time\_explicit}]
These two values specifying the default output formats for the \cd{Ptime}
and \cd{Ptime\_explicit} families of types, respectively.  The normal use is for these formats
to describe a time of day but not the date.  Some examples:

\inputCode{code/time-output-format}
%
\noindent
Documentation for the \cd{tmdate} function appears on the web page:
\ifthenelse{\boolean{hevea}}{
\myurl{www.research.att.com/\~gsf/man/man3/tm.html}}{
\myurl{www.research.att.com/~gsf/man/man3/tm.html}}
\end{description}

\subsection{Scanning extent}
\label{sec:library-customization-scanning-extent}

\subsection{Skipping white space}
\label{sec:library-customization-white-space}
Describe how to set various white space skipping modes.




\subsection{Endian-ness}
\label{sec:library-customization-endian}


\section{The IO Discpline}
\label{sec:io-discipline}
IO discipline values, which have type \cd{Pio\_disc\_t}, control the
'raw' reading of data from a file or from some other data source.  
The \pads{} system provides a collection of functions for generating
various IO disciplines, corresponding to various kinds of record
structures: new-line terminated, fixed width, IBM-style (initial data
indicating size of record, followed by payload), \etc{}  In addition,
the discipline indicates if the data source is seekable (a file) or
not (a stream).

To use an IO discipline, the user first creates one by invoking
a creation function supplied by the \pads{} system.  The resulting IO
discipline is then installed by passing it as an argument to either
\cd{P\_open} or to \cd{P\_set\_io\_disc}.

\begin{description}
\item
[\small{\cd{Pio\_disc\_t * P\_fwrec\_make(size\_t leader\_len, size\_t data\_len, size\_t trailer\_len)}}]
 Instantiates an instance of \cd{fwrec}, a discipline for fixed-width
 records.  The parameter \cd{data\_len} specifies the number of data bytes per record,
 while \cd{leader\_len} and \cd{trailer\_len} specify the number of bytes that
 occur before and after the data bytes within each record (either or
 both can be zero).  Thus the total record size in bytes is the sum
 of the three arguments.  

\item[\small{\cd{Pio\_disc\_t * P\_fwrec\_noseek\_make(size\_t leader\_len,
       size\_t data\_len, size\_t trailer\_len)}}]
Instantiates an instance of \texttt{fwrec\_noseek}, a version of \texttt{norec}
that does not require that the SFIO stream is seekable.

\item[\small{\cd{Pio\_disc\_t * P\_ctrec\_make(Pbyte termChar, size\_t block\_size\_hint);}}]
Instantiates an instance of \cd{ctrec}, a discipline for
character-terminated variable-width records. Argument \texttt{termChar} is the
character that marks the end of a record. Argument
\texttt{block\_size\_hint} suggests a block size to use, if the
discipline chooses to do fixed block-sized reads 'under the covers'.
It may be ignored by the discipline.
For ASCII newline-terminated records use, \literal{\cd{'\\n'}} or
\cd{P\_ASCII\_NEWLINE} 
as the term character.  For \cd{EBCDIC} newline-terminated records, use
\cd{P\_EBCDIC\_NEWLINE} as the term character.


\item[\small{\cd{Pio\_disc\_t * P\_ctrec\_noseek\_make(Pbyte termChar,
      size\_t block\_size\_hint)}}] 
Instantiates an instance of \cd{ctrec\_noseek}, a version of \cd{norec}
that does not require that the SFIO stream is seekable.

\item[\small{\cd{Pio\_disc\_t * P\_nlrec\_make(size\_t block\_size\_hint)}}]
Shorthand for calling the corresponding \cd{ctrec} make function with
\literal{\cd{'\\n'}} as \cd{termChar}.

\item[\small{\cd{Pio\_disc\_t * P\_nlrec\_noseek\_make(size\_t block\_size\_hint)}}]
Shorthand for calling the corresponding \cd{ctrec} make function with
\literal{\cd{'\\n'}} as \cd{termChar}.



\item[\small{\cd{Pio\_disc\_t * P\_vlrec\_make(int blocked, size\_t avg\_rlen\_hint)}}]
 Instantiates an instance of \cd{vlrec}, a discipline for IBM-style
 variable-length records with record length specified at the start
 of each record.  If blocked is set (\cd{!= 0}) then the records are
 grouped into blocks, where each block has a length given at the
 start of each block.  Argument \cd{avg\_rlen\_hint} is a hint as to what the
 average record length is, to help the discipline allocate memory.
 It should include the four bytes at the start of each record used for
 the record length.  It may be ignored by the discipline.
 

\item[\small{\cd{Pio\_disc\_t * P\_vlrec\_noseek\_make(int blocked,
 size\_t avg\_rlen\_hint)}}] Instantiates an instance of
 \cd{vlrec\_noseek}, a version of \cd{vlrec} that does not require
 that the SFIO stream is seekable.


\item[\small{\cd{Pio\_disc\_t * P\_norec\_make(size\_t block\_size\_hint)}}]
Instantiates an instance of \cd{norec}, a raw bytes discipline that
does not use records.  Argument \cd{block\_size\_hint} is a hint as to what block size
to use, if the discipline chooses to do fixed block-sized reads
'under the covers'.  It may be ignored by the discipline.


\item[\small{\cd{Pio\_disc\_t * P\_norec\_noseek\_make(size\_t block\_size\_hint)}}]
Instantiates an instance of \cd{norec\_noseek}, a version of \cd{norec}
that does not require that the SFIO stream is seekable.


\end{description}



\subsection{Closing an IO discipline}
When an IO discipline is no longer needed, the user should unmake it.
The function \cd{P\_io\_disc\_unmake} explicitly deallocates an IO
discipline. In addition, the function \cd{P\_close}
deallocates the installed IO discipline.  
The function \cd{P\_set\_io\_disc} deallocates the previously
installed discipline.
If desired, an IO discipline can be preserved using
\cd{P\_close\_keep\_io\_disc} or \cd{P\_set\_io\_disc\_keep\_old}, in
which case it can be re-used in a future \cd{P\_open} or
\cd{P\_set\_io\_disc} call. 

\subsection{Implementations}
Implementations of the standard IO disciplines can be found in
\texttt{libpads/default\_io\_disc.c}.  Anyone planning to implement a new IO
discipline should consult \texttt{default\_io\_disc.c}.



\section{Adding new base types}
\label{sec:library-adding-new-base-types}
