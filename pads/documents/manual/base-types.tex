\chapter{Base types}
\label{chap:base-types}

\PADSL{} base types describe inidivual (typically small) values: numbers, strings, dates, and so on.
This chapter describes a large number of built-in \PADSL{} base types.
It is also possible to extend \PADSL{} with any
number of new base types; see \secref{sec:library-adding-new-base-types}.

\section{In-Memory Representation}
\label{sec:base-types-rep}

Each base type as an external and an in-memory representation, where
related base types share the same in-memory representation.  For
example, while there are 18 different string base types, all of them
use \PDC{string} as their in-memory representation.

This section reviews the different in-memory representation types.

\subsection{{\tt PDC\_char}}

Type \PDC{char} is the in-memory representation of an external
character.  It is equivalent to the C type \cd{unsigned char}, or type
\PDC{uint8}: all are 8-bit unsigned values.  N.B.: Regardless of the
external character that is read, the corresponding ASCII character is stored in 
the in-memory representation.

\subsection{{\tt PDC\_string}}

Type \PDC{string} is the in-memory representation for all forms of
external strings.  A \PDC{string} \cd{s} has two fields:
\begin{itemize}
\item \cd{s.len} : the length of the string.
\item \cd{s.str} : a pointer to a sequence of \cd{s.len} characters.
\end{itemize}
In addition, \PDC{string} has fields that are manipulated
by various string functions (some of which are described below).
Most programmers should only use \cd{s.len} and \cd{s.str}.

The library discipline has a field \cd{copy\_strings} which controls
copying behavior for string read calls.  If \cd{copy\_strings} is
non-zero, the string read functions always copy strings. Otherwise, a
copy is not made and the target \PDC{string} points to memory managed by
the current IO discipline.  \cd{copy\_strings} should only be set to
zero for record-based IO disciplines where strings from record K are
not used after \PDC{IO\_next\_rec} has been called to move the IO cursor
to record K+1.  Note: \PDC{string\_preserve} can be used to force
a string that is using sharing to make a copy so that the string is
'preserved' (remains valid) across calls to \PDC{IO\_next\_rec}.

When copying is used, the string copies are stored in an internal
resizable buffer, and \cd{s.str} points into this buffer. 
To ensure correct behavior, function \PDC{string\_init} should be called
prior to using a \PDC{string}, and function \PDC{string\_cleanup} should
be called when a given \PDC{string} is no longer in use.
\\[1ex]
These are the full set of string helper functions.
\vskip 1ex
\begin{tinycodeaux}{\leftmargin=0in}
PDC_error_t PDC_string_init(PDC_t *pdc, PDC_string *s);
PDC_error_t PDC_string_cleanup(PDC_t *pdc, PDC_string *s);
PDC_error_t PDC_string_share(PDC_t *pdc, PDC_string *targ, const PDC_string *src);
PDC_error_t PDC_string_Cstr_share(PDC_t *pdc, PDC_string *targ, const char *src, size_t len);
PDC_error_t PDC_string_copy(PDC_t *pdc, PDC_string *targ, const PDC_string *src);
PDC_error_t PDC_string_Cstr_copy(PDC_t *pdc, PDC_string *targ, const char *src, size_t len);
PDC_error_t PDC_string_preserve(PDC_t *pdc, PDC_string *s);
int PDC_string_eq(const PDC_string *str1, const PDC_string *str2);
int PDC_string_eq_Cstr(const PDC_string *str, const char *Cstr);
\end{tinycodeaux}
\mbox{}\\[1ex]
\cd{PDC\_string\_init(pdc, s)}: Initialize \cd{s} to valid empty string (no dynamic memory allocated yet).\\[1ex]
\cd{PDC\_string\_cleanup(pdc, s)}: Free any dynamic memory allocated for \cd{s}.\\[1ex]
\cd{PDC\_string\_share(pdc, targ, src)}: make \cd{targ} refer to the string in \cd{src},
sharing the space with the original owner.\\[1ex]
\cd{PDC\_string\_Cstr\_share(pdc, targ, src, len)}: make \cd{targ} refer to \cd{len} characters in C-string \cd{src}.\\[1ex]
\cd{PDC\_string\_copy(pdc, targ, src)}: Copy the string in \cd{src} into \cd{targ}; sharing is not used.\\[1ex]
\cd{PDC\_string\_Cstr\_copy(pdc, targ, src, len)}: Copy \cd{len} characters from C-string \cd{src} into \cd{targ}; sharing is not used.\\[1ex]
\cd{PDC\_string\_eq(str1, str2)}: returns \cd{0} if \cd{str1} equals \cd{str2}, a negative number if \cd{str1 < str2},
and a positive number if \cd{str1 > str2}.\\[1ex]
\cd{PDC\_string\_eq(str, Cstr)}: returns \cd{0} if \cd{str} equals \cd{Cstr}, a negative number if \cd{str < Cstr},
and a positive number if \cd{str > Cstr}.\\[1ex]

Although not strictly necessary, both \PDC{string\_copy} and \PDC{string\_Cstr\_copy} null-terminate \cd{targ->str}.
Each copy function returns \PDC{ERR} on bad arguments or on failure to alloc space, otherwise it returns \PDC{OK}.

\subsection{Integer types}

There are 8 in-memory representations for integers, four types for
signed values (\PDC{int8}, \PDC{int16}, \PDC{int32}, and \PDC{int64})
and four types for unsigned values (\PDC{uint8}, \PDC{uint16}, \PDC{uint32}, and \PDC{uint64}).
The number in these type names indicates the number of bits in the in-memory representation,
thus there are signed and unsigned integers that use 1, 2, 4, or 8 bytes of memory.
The endian-ness of these in-memory representation types is the same as the
endian-ness of the processor that the code is executing on.  Thus,
the external representation, which may be in another format, is always converted
to a primitive in-memory representation supported by the processor.

For programming convenience, the header file {\tt libpadsc.h} includes definitions of the minimum
and maximum values for each signed type, and of the maximum value for each unsigned type:
\begin{tinycode}
PDC_MIN_INT8
PDC_MAX_INT8
PDC_MAX_UINT8

PDC_MIN_INT16
PDC_MAX_INT16
PDC_MAX_UINT16

PDC_MIN_INT32
PDC_MAX_INT32
PDC_MAX_UINT32

PDC_MIN_INT64
PDC_MAX_INT64
PDC_MAX_UINT64
\end{tinycode}

\subsection{Floating-point types}

\PADSL{} has only two in-memory floating point representations,
\PDC{float32} and \PDC{float64}, which correspond to ansi C types \cd{float}
and \cd{double}, respectively.

\subsection{Fixed-point types}

A fixed-point number is a number with a fixed number of decimal digits
(digits after the 'dot').  For example, 123.456 is a fixed-point
number with three decimal digits.  External formats for such numbers
occur, e.g., in COBOL data.  We have chosen a very simple in-memory
representation for such numbers: a struct with a numerator (\cd{num})
that contains all of the digits and a denominator (\cd{denom}) that
contains some power of 10.  For example, 123.456 would be represented
with a numerator containing 123456 and a denominator containing 1000 ($10^3$).

The in-memory representation always has an unsigned denominator. We
provide signed and unsigned representations that use signed and
unsigned numerators, respectively.  One can choose the number of bits
used for both numerator and denominator in the same way as for the
integer types, thus there are four four types for signed values
and four types for unsigned values:

\begin{tinycodeaux}{\leftmargin=0in}
typedef struct \{ PDC_int8   num; PDC_uint8  denom; \} PDC_fpoint8;
typedef struct \{ PDC_int16  num; PDC_uint16 denom; \} PDC_fpoint16;
typedef struct \{ PDC_int32  num; PDC_uint32 denom; \} PDC_fpoint32;
typedef struct \{ PDC_int64  num; PDC_uint64 denom; \} PDC_fpoint64;

typedef struct \{ PDC_uint8  num; PDC_uint8  denom; \} PDC_ufpoint8;
typedef struct \{ PDC_uint16 num; PDC_uint16 denom; \} PDC_ufpoint16;
typedef struct \{ PDC_uint32 num; PDC_uint32 denom; \} PDC_ufpoint32;
typedef struct \{ PDC_uint64 num; PDC_uint64 denom; \} PDC_ufpoint64;
\end{tinycodeaux}

There are two macros to help one use values of these in-memory types.
\PDC{FPOINT2FLT(fp)} calculates \cd{fp.num/fp.denom} as a float,
while \PDC{FPOINT2DBL(fp)} calculates \cd{fp.num/fp.denom} as a double.

\section{Base Type Mask}
\label{sec:base-type-mask}

XXX \PDCbasem{} was already described?

\section{Base Type Parse Descriptor}
\label{sec:base-type-parse-descriptors}

XXX already described \PDCbasepd{}.  Which involves describing \PDCloct{}, \PDCpost{},
and \PDCerrCodet{}.  

Specific error codes are discussed below
when the base type read functions are described.

\section{Character Sets}

\PADSL{} currently supports two character sets for external data, ASCII and EBCDIC.
As discussed in \secref{sec:common-character-sets}, the library
discipline contain a field
\cd{def\_charset} that selects which character set to use when
one is not specified explicitly.  As a result, for each 'kind' of data
that has an external form made up of characters (characters, strings,
character-based dates, character-based integer and floating point
numbers, \etc{}), \PADSL{} has three types: a type that indicates the external
form is always ASCII, a type that indicates the external form is
always EBCDIC, and a type that indicates that the external form uses
the character set specified in \cd{def\_charset}.

In each section below describing character-based types, we give a
three-column table indicating the type(s) that use ASCII, EBCDIC, or
DEFAULT character sets.  For example, the next section begins with a
table showing types \Pa{char} (ASCII), \Pe{char} (EBCDIC), \Pd{char}
(DEFAULT).

\section{Character Base Types}

\subsection{Fixed-width character-based encoding}

\aedBegin{}
\aedLine{char}
\aedEnd{}
\\[1ex]
The corresponding read functions are:
\begin{tinycodeaux}{\leftmargin=0in}
PDC_error_t PDC_char_read   (PDC_t *pdc, const PDC_base_m *m, PDC_base_pd *pd, PDC_char *c_out);
PDC_error_t PDC_a_char_read (PDC_t *pdc, const PDC_base_m *m, PDC_base_pd *pd, PDC_char *c_out);
PDC_error_t PDC_e_char_read (PDC_t *pdc, const PDC_base_m *m, PDC_base_pd *pd, PDC_char *c_out);
\end{tinycodeaux}

\subsection{Special character base types}

\aedBegin{}
\aedLine{countX}
\aedLine{countXtoY}
\aedEnd{}

\begin{tinycodeaux}{\leftmargin=0in}
PDC_error_t PDC_countX     (PDC_t *pdc, const PDC_base_m *m, PDC_uint8 x, int eor_required,
		            PDC_base_pd *pd, PDC_int32 *res_out);
PDC_error_t PDC_a_countX   (PDC_t *pdc, const PDC_base_m *m, PDC_uint8 x, int eor_required,
			    PDC_base_pd *pd, PDC_int32 *res_out);
PDC_error_t PDC_e_countX   (PDC_t *pdc, const PDC_base_m *m, PDC_uint8 x, int eor_required,
			    PDC_base_pd *pd, PDC_int32 *res_out);

PDC_error_t PDC_countXtoY  (PDC_t *pdc, const PDC_base_m *m, PDC_uint8 x, PDC_uint8 y,
		            PDC_base_pd *pd, PDC_int32 *res_out);
PDC_error_t PDC_a_countXtoY(PDC_t *pdc, const PDC_base_m *m, PDC_uint8 x, PDC_uint8 y,
			    PDC_base_pd *pd, PDC_int32 *res_out);
PDC_error_t PDC_e_countXtoY(PDC_t *pdc, const PDC_base_m *m, PDC_uint8 x, PDC_uint8 y,
			    PDC_base_pd *pd, PDC_int32 *res_out);
\end{tinycodeaux}

\section{String Base Types}

\subsection{Fixed-width character-based encoding}

\aedBegin{}
\aedLine{string\_FW}
\aedEnd{}

\begin{tinycodeaux}{\leftmargin=0in}
PDC_error_t PDC_string_FW_read   (PDC_t *pdc, const PDC_base_m *m, size_t width,
				  PDC_base_pd *pd, PDC_string *s_out);
PDC_error_t PDC_a_string_FW_read (PDC_t *pdc, const PDC_base_m *m, size_t width,
				  PDC_base_pd *pd, PDC_string *s_out);
PDC_error_t PDC_e_string_FW_read (PDC_t *pdc, const PDC_base_m *m, size_t width,
				  PDC_base_pd *pd, PDC_string *s_out);
\end{tinycodeaux}

\subsection{Variable-width character-based encoding}

\aedBegin{}
\aedLine{string}
\aedLine{string\_ME}
\aedLine{string\_CME}
\aedLine{string\_SE}
\aedLine{string\_CSE}
\aedEnd{}

\begin{tinycodeaux}{\leftmargin=0in}
PDC_error_t PDC_string_read      (PDC_t *pdc, const PDC_base_m *m, PDC_char stopChar,
			          PDC_base_pd *pd, PDC_string *s_out);
PDC_error_t PDC_a_string_read    (PDC_t *pdc, const PDC_base_m *m, PDC_char stopChar,
			          PDC_base_pd *pd, PDC_string *s_out);
PDC_error_t PDC_e_string_read    (PDC_t *pdc, const PDC_base_m *m, PDC_char stopChar,
			          PDC_base_pd *pd, PDC_string *s_out);

PDC_error_t PDC_string_ME_read   (PDC_t *pdc, const PDC_base_m *m, const char *matchRegexp,
				  PDC_base_pd *pd, PDC_string *s_out);
PDC_error_t PDC_a_string_ME_read (PDC_t *pdc, const PDC_base_m *m, const char *matchRegexp,
				  PDC_base_pd *pd, PDC_string *s_out);
PDC_error_t PDC_e_string_ME_read (PDC_t *pdc, const PDC_base_m *m, const char *matchRegexp,
				  PDC_base_pd *pd, PDC_string *s_out);

PDC_error_t PDC_string_CME_read  (PDC_t *pdc, const PDC_base_m *m, PDC_regexp_t *matchRegexp,
				  PDC_base_pd *pd, PDC_string *s_out);
PDC_error_t PDC_a_string_CME_read(PDC_t *pdc, const PDC_base_m *m, PDC_regexp_t *matchRegexp,
				  PDC_base_pd *pd, PDC_string *s_out);
PDC_error_t PDC_e_string_CME_read(PDC_t *pdc, const PDC_base_m *m, PDC_regexp_t *matchRegexp,
				  PDC_base_pd *pd, PDC_string *s_out);

PDC_error_t PDC_string_SE_read   (PDC_t *pdc, const PDC_base_m *m, const char *stopRegexp,
				  PDC_base_pd *pd, PDC_string *s_out);
PDC_error_t PDC_a_string_SE_read (PDC_t *pdc, const PDC_base_m *m, const char *stopRegexp,
				  PDC_base_pd *pd, PDC_string *s_out);
PDC_error_t PDC_e_string_SE_read (PDC_t *pdc, const PDC_base_m *m, const char *stopRegexp,
				  PDC_base_pd *pd, PDC_string *s_out);

PDC_error_t PDC_string_CSE_read  (PDC_t *pdc, const PDC_base_m *m, PDC_regexp_t *stopRegexp,
				  PDC_base_pd *pd, PDC_string *s_out);
PDC_error_t PDC_a_string_CSE_read(PDC_t *pdc, const PDC_base_m *m, PDC_regexp_t *stopRegexp,
				  PDC_base_pd *pd, PDC_string *s_out);
PDC_error_t PDC_e_string_CSE_read(PDC_t *pdc, const PDC_base_m *m, PDC_regexp_t *stopRegexp,
				  PDC_base_pd *pd, PDC_string *s_out);
\end{tinycodeaux}

\subsection{Special string base types}

\aedBegin{}
\Pd{date}, \Pa{date}, \Pe{date}.
\aedEnd{}

\begin{tinycodeaux}{\leftmargin=0in}
PDC_error_t PDC_date_read  (PDC_t *pdc, const PDC_base_m *m, PDC_char stopChar,
			    PDC_base_pd *pd, PDC_uint32 *res_out);
PDC_error_t PDC_a_date_read(PDC_t *pdc, const PDC_base_m *m, PDC_char stopChar,
			    PDC_base_pd *pd, PDC_uint32 *res_out);
PDC_error_t PDC_e_date_read(PDC_t *pdc, const PDC_base_m *m, PDC_char stopChar,
			    PDC_base_pd *pd, PDC_uint32 *res_out);
\end{tinycodeaux}

\section{Integer Base Types}

\subsection{Fixed-width character-based encoding}

\aedBegin{}
\aedLine{int8\_FW}
\aedLine{int16\_FW}
\aedLine{int32\_FW}
\aedLine{int64\_FW}
\aedLine{uint8\_FW}
\aedLine{uint16\_FW}
\aedLine{uint32\_FW}
\aedLine{uint64\_FW}
\aedEnd{}

\begin{tinycodeaux}{\leftmargin=0in}
PDC_error_t PDC_a_int8_FW_read (PDC_t *pdc, const PDC_base_m *m, size_t width,
				PDC_base_pd *pd, PDC_int8 *res_out);
PDC_error_t PDC_a_int16_FW_read(PDC_t *pdc, const PDC_base_m *m, size_t width,
				PDC_base_pd *pd, PDC_int16 *res_out);
PDC_error_t PDC_a_int32_FW_read(PDC_t *pdc, const PDC_base_m *m, size_t width,
				PDC_base_pd *pd, PDC_int32 *res_out);
PDC_error_t PDC_a_int64_FW_read(PDC_t *pdc, const PDC_base_m *m, size_t width,
				PDC_base_pd *pd, PDC_int64 *res_out);

PDC_error_t PDC_a_uint8_FW_read (PDC_t *pdc, const PDC_base_m *m, size_t width,
				 PDC_base_pd *pd, PDC_uint8 *res_out);
PDC_error_t PDC_a_uint16_FW_read(PDC_t *pdc, const PDC_base_m *m, size_t width,
				 PDC_base_pd *pd, PDC_uint16 *res_out);
PDC_error_t PDC_a_uint32_FW_read(PDC_t *pdc, const PDC_base_m *m, size_t width,
				 PDC_base_pd *pd, PDC_uint32 *res_out);
PDC_error_t PDC_a_uint64_FW_read(PDC_t *pdc, const PDC_base_m *m, size_t width,
				 PDC_base_pd *pd, PDC_uint64 *res_out);

PDC_error_t PDC_e_int8_FW_read (PDC_t *pdc, const PDC_base_m *m, size_t width,
				PDC_base_pd *pd, PDC_int8 *res_out);
PDC_error_t PDC_e_int16_FW_read(PDC_t *pdc, const PDC_base_m *m, size_t width,
				PDC_base_pd *pd, PDC_int16 *res_out);
PDC_error_t PDC_e_int32_FW_read(PDC_t *pdc, const PDC_base_m *m, size_t width,
				PDC_base_pd *pd, PDC_int32 *res_out);
PDC_error_t PDC_e_int64_FW_read(PDC_t *pdc, const PDC_base_m *m, size_t width,
				PDC_base_pd *pd, PDC_int64 *res_out);

PDC_error_t PDC_e_uint8_FW_read (PDC_t *pdc, const PDC_base_m *m, size_t width,
				 PDC_base_pd *pd, PDC_uint8 *res_out);
PDC_error_t PDC_e_uint16_FW_read(PDC_t *pdc, const PDC_base_m *m, size_t width,
				 PDC_base_pd *pd, PDC_uint16 *res_out);
PDC_error_t PDC_e_uint32_FW_read(PDC_t *pdc, const PDC_base_m *m, size_t width,
				 PDC_base_pd *pd, PDC_uint32 *res_out);
PDC_error_t PDC_e_uint64_FW_read(PDC_t *pdc, const PDC_base_m *m, size_t width,
				 PDC_base_pd *pd, PDC_uint64 *res_out);

PDC_error_t PDC_int8_FW_read (PDC_t *pdc, const PDC_base_m *m, size_t width,
			      PDC_base_pd *pd, PDC_int8 *res_out);
PDC_error_t PDC_int16_FW_read(PDC_t *pdc, const PDC_base_m *m, size_t width,
			      PDC_base_pd *pd, PDC_int16 *res_out);
PDC_error_t PDC_int32_FW_read(PDC_t *pdc, const PDC_base_m *m, size_t width,
			      PDC_base_pd *pd, PDC_int32 *res_out);
PDC_error_t PDC_int64_FW_read(PDC_t *pdc, const PDC_base_m *m, size_t width,
			      PDC_base_pd *pd, PDC_int64 *res_out);

PDC_error_t PDC_uint8_FW_read (PDC_t *pdc, const PDC_base_m *m, size_t width,
			       PDC_base_pd *pd, PDC_uint8 *res_out);
PDC_error_t PDC_uint16_FW_read(PDC_t *pdc, const PDC_base_m *m, size_t width,
			       PDC_base_pd *pd, PDC_uint16 *res_out);
PDC_error_t PDC_uint32_FW_read(PDC_t *pdc, const PDC_base_m *m, size_t width,
			       PDC_base_pd *pd, PDC_uint32 *res_out);
PDC_error_t PDC_uint64_FW_read(PDC_t *pdc, const PDC_base_m *m, size_t width,
			       PDC_base_pd *pd, PDC_uint64 *res_out);
\end{tinycodeaux}

\subsection{Variable-width character-based encoding}

\aedBegin{}
\aedLine{int8}
\aedLine{int16}
\aedLine{int32}
\aedLine{int64}
\aedLine{uint8}
\aedLine{uint16}
\aedLine{uint32}
\aedLine{uint64}
\aedEnd{}

\begin{tinycodeaux}{\leftmargin=0in}
\codeallowbreaks
PDC_error_t PDC_a_int8_read (PDC_t *pdc, const PDC_base_m *m,
			     PDC_base_pd *pd, PDC_int8 *res_out);
PDC_error_t PDC_a_int16_read(PDC_t *pdc, const PDC_base_m *m,
			     PDC_base_pd *pd, PDC_int16 *res_out);
PDC_error_t PDC_a_int32_read(PDC_t *pdc, const PDC_base_m *m,
			     PDC_base_pd *pd, PDC_int32 *res_out);
PDC_error_t PDC_a_int64_read(PDC_t *pdc, const PDC_base_m *m,
			     PDC_base_pd *pd, PDC_int64 *res_out);

PDC_error_t PDC_a_uint8_read (PDC_t *pdc, const PDC_base_m *m,
			      PDC_base_pd *pd, PDC_uint8 *res_out);
PDC_error_t PDC_a_uint16_read(PDC_t *pdc, const PDC_base_m *m,
			      PDC_base_pd *pd, PDC_uint16 *res_out);
PDC_error_t PDC_a_uint32_read(PDC_t *pdc, const PDC_base_m *m,
			      PDC_base_pd *pd, PDC_uint32 *res_out);
PDC_error_t PDC_a_uint64_read(PDC_t *pdc, const PDC_base_m *m,
			      PDC_base_pd *pd, PDC_uint64 *res_out);

PDC_error_t PDC_e_int8_read (PDC_t *pdc, const PDC_base_m *m,
			     PDC_base_pd *pd, PDC_int8 *res_out);
PDC_error_t PDC_e_int16_read(PDC_t *pdc, const PDC_base_m *m,
			     PDC_base_pd *pd, PDC_int16 *res_out);
PDC_error_t PDC_e_int32_read(PDC_t *pdc, const PDC_base_m *m,
			     PDC_base_pd *pd, PDC_int32 *res_out);
PDC_error_t PDC_e_int64_read(PDC_t *pdc, const PDC_base_m *m,
			     PDC_base_pd *pd, PDC_int64 *res_out);

PDC_error_t PDC_e_uint8_read (PDC_t *pdc, const PDC_base_m *m,
			      PDC_base_pd *pd, PDC_uint8 *res_out);
PDC_error_t PDC_e_uint16_read(PDC_t *pdc, const PDC_base_m *m,
			      PDC_base_pd *pd, PDC_uint16 *res_out);
PDC_error_t PDC_e_uint32_read(PDC_t *pdc, const PDC_base_m *m,
			      PDC_base_pd *pd, PDC_uint32 *res_out);
PDC_error_t PDC_e_uint64_read(PDC_t *pdc, const PDC_base_m *m,
			      PDC_base_pd *pd, PDC_uint64 *res_out);

PDC_error_t PDC_int8_read (PDC_t *pdc, const PDC_base_m *m,
			    PDC_base_pd *pd, PDC_int8 *res_out);
PDC_error_t PDC_int16_read(PDC_t *pdc, const PDC_base_m *m,
			   PDC_base_pd *pd, PDC_int16 *res_out);
PDC_error_t PDC_int32_read(PDC_t *pdc, const PDC_base_m *m,
			   PDC_base_pd *pd, PDC_int32 *res_out);
PDC_error_t PDC_int64_read(PDC_t *pdc, const PDC_base_m *m,
			   PDC_base_pd *pd, PDC_int64 *res_out);

PDC_error_t PDC_uint8_read (PDC_t *pdc, const PDC_base_m *m,
			    PDC_base_pd *pd, PDC_uint8 *res_out);
PDC_error_t PDC_uint16_read(PDC_t *pdc, const PDC_base_m *m,
			    PDC_base_pd *pd, PDC_uint16 *res_out);
PDC_error_t PDC_uint32_read(PDC_t *pdc, const PDC_base_m *m,
			    PDC_base_pd *pd, PDC_uint32 *res_out);
PDC_error_t PDC_uint64_read(PDC_t *pdc, const PDC_base_m *m,
			    PDC_base_pd *pd, PDC_uint64 *res_out);
\end{tinycodeaux}

\subsection{Raw binary encoding}

\bBegin{}
\bLine{int8}
\bLine{int16}
\bLine{int32}
\bLine{int64}
\bLine{uint8}
\bLine{uint16}
\bLine{uint32}
\bLine{uint64}
\bEnd{}

\begin{tinycodeaux}{\leftmargin=0in}
\codeallowbreaks
PDC_error_t PDC_b_int8_read (PDC_t *pdc, const PDC_base_m *m,
			     PDC_base_pd *pd, PDC_int8 *res_out);
PDC_error_t PDC_b_int16_read(PDC_t *pdc, const PDC_base_m *m,
			     PDC_base_pd *pd, PDC_int16 *res_out);
PDC_error_t PDC_b_int32_read(PDC_t *pdc, const PDC_base_m *m,
			     PDC_base_pd *pd, PDC_int32 *res_out);
PDC_error_t PDC_b_int64_read(PDC_t *pdc, const PDC_base_m *m,
			     PDC_base_pd *pd, PDC_int64 *res_out);

PDC_error_t PDC_b_uint8_read (PDC_t *pdc, const PDC_base_m *m,
			      PDC_base_pd *pd, PDC_uint8 *res_out);
PDC_error_t PDC_b_uint16_read(PDC_t *pdc, const PDC_base_m *m,
			      PDC_base_pd *pd, PDC_uint16 *res_out);
PDC_error_t PDC_b_uint32_read(PDC_t *pdc, const PDC_base_m *m,
			      PDC_base_pd *pd, PDC_uint32 *res_out);
PDC_error_t PDC_b_uint64_read(PDC_t *pdc, const PDC_base_m *m,
			      PDC_base_pd *pd, PDC_uint64 *res_out);
\end{tinycodeaux}

\subsection{Serialize binary encoding}

\sbBegin{}
\sbLine{int8}
\sbLine{int16}
\sbLine{int32}
\sbLine{int64}
\sbLine{uint8}
\sbLine{uint16}
\sbLine{uint32}
\sbLine{uint64}
\bEnd{}

\begin{tinycodeaux}{\leftmargin=0in}
\codeallowbreaks
PDC_error_t PDC_sbl_int8_read    (PDC_t *pdc, const PDC_base_m *m, PDC_uint32 num_bytes,
			         PDC_base_pd *pd, PDC_int8 *res_out);
PDC_error_t PDC_sbl_int16_read   (PDC_t *pdc, const PDC_base_m *m, PDC_uint32 num_bytes,
			         PDC_base_pd *pd, PDC_int16 *res_out);
PDC_error_t PDC_sbl_int32_read   (PDC_t *pdc, const PDC_base_m *m, PDC_uint32 num_bytes,
			         PDC_base_pd *pd, PDC_int32 *res_out);
PDC_error_t PDC_sbl_int64_read   (PDC_t *pdc, const PDC_base_m *m, PDC_uint32 num_bytes,
			         PDC_base_pd *pd, PDC_int64 *res_out);

PDC_error_t PDC_sbl_uint8_read   (PDC_t *pdc, const PDC_base_m *m, PDC_uint32 num_bytes,
			 	 PDC_base_pd *pd, PDC_uint8 *res_out);
PDC_error_t PDC_sbl_uint16_read  (PDC_t *pdc, const PDC_base_m *m, PDC_uint32 num_bytes,
			 	 PDC_base_pd *pd, PDC_uint16 *res_out);
PDC_error_t PDC_sbl_uint32_read  (PDC_t *pdc, const PDC_base_m *m, PDC_uint32 num_bytes,
			 	 PDC_base_pd *pd, PDC_uint32 *res_out);
PDC_error_t PDC_sbl_uint64_read  (PDC_t *pdc, const PDC_base_m *m, PDC_uint32 num_bytes,
			 	 PDC_base_pd *pd, PDC_uint64 *res_out);

PDC_error_t PDC_sbh_int8_read    (PDC_t *pdc, const PDC_base_m *m, PDC_uint32 num_bytes,
			         PDC_base_pd *pd, PDC_int8 *res_out);
PDC_error_t PDC_sbh_int16_read   (PDC_t *pdc, const PDC_base_m *m, PDC_uint32 num_bytes,
			         PDC_base_pd *pd, PDC_int16 *res_out);
PDC_error_t PDC_sbh_int32_read   (PDC_t *pdc, const PDC_base_m *m, PDC_uint32 num_bytes,
			         PDC_base_pd *pd, PDC_int32 *res_out);
PDC_error_t PDC_sbh_int64_read   (PDC_t *pdc, const PDC_base_m *m, PDC_uint32 num_bytes,
			         PDC_base_pd *pd, PDC_int64 *res_out);

PDC_error_t PDC_sbh_uint8_read   (PDC_t *pdc, const PDC_base_m *m, PDC_uint32 num_bytes,
			 	 PDC_base_pd *pd, PDC_uint8 *res_out);
PDC_error_t PDC_sbh_uint16_read  (PDC_t *pdc, const PDC_base_m *m, PDC_uint32 num_bytes,
			 	 PDC_base_pd *pd, PDC_uint16 *res_out);
PDC_error_t PDC_sbh_uint32_read  (PDC_t *pdc, const PDC_base_m *m, PDC_uint32 num_bytes,
			 	 PDC_base_pd *pd, PDC_uint32 *res_out);
PDC_error_t PDC_sbh_uint64_read  (PDC_t *pdc, const PDC_base_m *m, PDC_uint32 num_bytes,
			 	 PDC_base_pd *pd, PDC_uint64 *res_out);
\end{tinycodeaux}

\subsection{EBC and BCD encoding}

\cBegin{}
\cLine{int8}
\cLine{int16}
\cLine{int32}
\cLine{int64}
\cLine{uint8}
\cLine{uint16}
\cLine{uint32}
\cLine{uint64}
\cEnd{}

\begin{tinycodeaux}{\leftmargin=0in}
\codeallowbreaks
PDC_error_t PDC_ebc_int8_read   (PDC_t *pdc, const PDC_base_m *m, PDC_uint32 num_digits,
				 PDC_base_pd *pd, PDC_int8 *res_out);
PDC_error_t PDC_ebc_int16_read  (PDC_t *pdc, const PDC_base_m *m, PDC_uint32 num_digits,
				 PDC_base_pd *pd, PDC_int16 *res_out);
PDC_error_t PDC_ebc_int32_read  (PDC_t *pdc, const PDC_base_m *m, PDC_uint32 num_digits,
				 PDC_base_pd *pd, PDC_int32 *res_out);
PDC_error_t PDC_ebc_int64_read  (PDC_t *pdc, const PDC_base_m *m, PDC_uint32 num_digits,
				 PDC_base_pd *pd, PDC_int64 *res_out);

PDC_error_t PDC_ebc_uint8_read  (PDC_t *pdc, const PDC_base_m *m, PDC_uint32 num_digits,
				 PDC_base_pd *pd, PDC_uint8 *res_out);
PDC_error_t PDC_ebc_uint16_read (PDC_t *pdc, const PDC_base_m *m, PDC_uint32 num_digits,
				 PDC_base_pd *pd, PDC_uint16 *res_out);
PDC_error_t PDC_ebc_uint32_read (PDC_t *pdc, const PDC_base_m *m, PDC_uint32 num_digits,
				 PDC_base_pd *pd, PDC_uint32 *res_out);
PDC_error_t PDC_ebc_uint64_read (PDC_t *pdc, const PDC_base_m *m, PDC_uint32 num_digits,
				 PDC_base_pd *pd, PDC_uint64 *res_out);

PDC_error_t PDC_bcd_int8_read   (PDC_t *pdc, const PDC_base_m *m, PDC_uint32 num_digits,
				 PDC_base_pd *pd, PDC_int8 *res_out);
PDC_error_t PDC_bcd_int16_read  (PDC_t *pdc, const PDC_base_m *m, PDC_uint32 num_digits,
				 PDC_base_pd *pd, PDC_int16 *res_out);
PDC_error_t PDC_bcd_int32_read  (PDC_t *pdc, const PDC_base_m *m, PDC_uint32 num_digits,
				 PDC_base_pd *pd, PDC_int32 *res_out);
PDC_error_t PDC_bcd_int64_read  (PDC_t *pdc, const PDC_base_m *m, PDC_uint32 num_digits,
				 PDC_base_pd *pd, PDC_int64 *res_out);

PDC_error_t PDC_bcd_uint8_read  (PDC_t *pdc, const PDC_base_m *m, PDC_uint32 num_digits,
				 PDC_base_pd *pd, PDC_uint8 *res_out);
PDC_error_t PDC_bcd_uint16_read (PDC_t *pdc, const PDC_base_m *m, PDC_uint32 num_digits,
				 PDC_base_pd *pd, PDC_uint16 *res_out);
PDC_error_t PDC_bcd_uint32_read (PDC_t *pdc, const PDC_base_m *m, PDC_uint32 num_digits,
				 PDC_base_pd *pd, PDC_uint32 *res_out);
PDC_error_t PDC_bcd_uint64_read (PDC_t *pdc, const PDC_base_m *m, PDC_uint32 num_digits,
				 PDC_base_pd *pd, PDC_uint64 *res_out);
\end{tinycodeaux}

\section{Floating Point Base Types}

\subsection{Fixed-width character-based encoding}

\aedBegin{}
\aedLine{float32\_FW}
\aedLine{float64\_FW}
\aedEnd{}

\begin{tinycodeaux}{\leftmargin=0in}
\end{tinycodeaux}

\subsection{Variable-width character-based encoding}

\aedBegin{}
\aedLine{float32}
\aedLine{float64}
\aedEnd{}

\begin{tinycodeaux}{\leftmargin=0in}
\end{tinycodeaux}

\subsection{Raw binary encoding}

\bBegin{}
\bLine{float32}
\bLine{float64}
\bEnd{}

\begin{tinycodeaux}{\leftmargin=0in}
\end{tinycodeaux}

\subsection{Serialized binary encoding}

\sbBegin{}
\sbLine{float32}
\sbLine{float64}
\sbEnd{}

\begin{tinycodeaux}{\leftmargin=0in}
\end{tinycodeaux}

\section{Fixed Point Base Types}

\subsection{Serialized binary encoding}

\sbBegin{}
\sbLine{fpoint8}
\sbLine{fpoint16}
\sbLine{fpoint32}
\sbLine{fpoint64}
\sbLine{ufpoint8}
\sbLine{ufpoint16}
\sbLine{ufpoint32}
\sbLine{ufpoint64}
\bEnd{}

\begin{tinycodeaux}{\leftmargin=0in}
\codeallowbreaks
PDC_error_t PDC_sbl_fpoint8_read    (PDC_t *pdc, const PDC_base_m *m, PDC_uint32 num_bytes, PDC_uint32 d_exp,
				    PDC_base_pd *pd, PDC_fpoint8 *res_out);
PDC_error_t PDC_sbl_fpoint16_read   (PDC_t *pdc, const PDC_base_m *m, PDC_uint32 num_bytes, PDC_uint32 d_exp,
				    PDC_base_pd *pd, PDC_fpoint16 *res_out);
PDC_error_t PDC_sbl_fpoint32_read   (PDC_t *pdc, const PDC_base_m *m, PDC_uint32 num_bytes, PDC_uint32 d_exp,
				    PDC_base_pd *pd, PDC_fpoint32 *res_out);
PDC_error_t PDC_sbl_fpoint64_read   (PDC_t *pdc, const PDC_base_m *m, PDC_uint32 num_bytes, PDC_uint32 d_exp,
				    PDC_base_pd *pd, PDC_fpoint64 *res_out);

PDC_error_t PDC_sbl_ufpoint8_read   (PDC_t *pdc, const PDC_base_m *m, PDC_uint32 num_bytes, PDC_uint32 d_exp,
				    PDC_base_pd *pd, PDC_ufpoint8 *res_out);
PDC_error_t PDC_sbl_ufpoint16_read  (PDC_t *pdc, const PDC_base_m *m, PDC_uint32 num_bytes, PDC_uint32 d_exp,
				    PDC_base_pd *pd, PDC_ufpoint16 *res_out);
PDC_error_t PDC_sbl_ufpoint32_read  (PDC_t *pdc, const PDC_base_m *m, PDC_uint32 num_bytes, PDC_uint32 d_exp,
				    PDC_base_pd *pd, PDC_ufpoint32 *res_out);
PDC_error_t PDC_sbl_ufpoint64_read  (PDC_t *pdc, const PDC_base_m *m, PDC_uint32 num_bytes, PDC_uint32 d_exp,
				    PDC_base_pd *pd, PDC_ufpoint64 *res_out);

PDC_error_t PDC_sbh_fpoint8_read    (PDC_t *pdc, const PDC_base_m *m, PDC_uint32 num_bytes, PDC_uint32 d_exp,
				    PDC_base_pd *pd, PDC_fpoint8 *res_out);
PDC_error_t PDC_sbh_fpoint16_read   (PDC_t *pdc, const PDC_base_m *m, PDC_uint32 num_bytes, PDC_uint32 d_exp,
				    PDC_base_pd *pd, PDC_fpoint16 *res_out);
PDC_error_t PDC_sbh_fpoint32_read   (PDC_t *pdc, const PDC_base_m *m, PDC_uint32 num_bytes, PDC_uint32 d_exp,
				    PDC_base_pd *pd, PDC_fpoint32 *res_out);
PDC_error_t PDC_sbh_fpoint64_read   (PDC_t *pdc, const PDC_base_m *m, PDC_uint32 num_bytes, PDC_uint32 d_exp,
				    PDC_base_pd *pd, PDC_fpoint64 *res_out);

PDC_error_t PDC_sbh_ufpoint8_read   (PDC_t *pdc, const PDC_base_m *m, PDC_uint32 num_bytes, PDC_uint32 d_exp,
				    PDC_base_pd *pd, PDC_ufpoint8 *res_out);
PDC_error_t PDC_sbh_ufpoint16_read  (PDC_t *pdc, const PDC_base_m *m, PDC_uint32 num_bytes, PDC_uint32 d_exp,
				    PDC_base_pd *pd, PDC_ufpoint16 *res_out);
PDC_error_t PDC_sbh_ufpoint32_read  (PDC_t *pdc, const PDC_base_m *m, PDC_uint32 num_bytes, PDC_uint32 d_exp,
				    PDC_base_pd *pd, PDC_ufpoint32 *res_out);
PDC_error_t PDC_sbh_ufpoint64_read  (PDC_t *pdc, const PDC_base_m *m, PDC_uint32 num_bytes, PDC_uint32 d_exp,
				    PDC_base_pd *pd, PDC_ufpoint64 *res_out);
\end{tinycodeaux}

\subsection{EBC and BCD encoding}

\cBegin{}
\cLine{fpoint8}
\cLine{fpoint16}
\cLine{fpoint32}
\cLine{fpoint64}
\cLine{ufpoint8}
\cLine{ufpoint16}
\cLine{ufpoint32}
\cLine{ufpoint64}
\cEnd{}

\begin{tinycodeaux}{\leftmargin=0in}
\codeallowbreaks
PDC_error_t PDC_ebc_fpoint8_read   (PDC_t *pdc, const PDC_base_m *m, PDC_uint32 num_digits, PDC_uint32 d_exp,
				    PDC_base_pd *pd, PDC_fpoint8 *res_out);
PDC_error_t PDC_ebc_fpoint16_read  (PDC_t *pdc, const PDC_base_m *m, PDC_uint32 num_digits, PDC_uint32 d_exp,
				    PDC_base_pd *pd, PDC_fpoint16 *res_out);
PDC_error_t PDC_ebc_fpoint32_read  (PDC_t *pdc, const PDC_base_m *m, PDC_uint32 num_digits, PDC_uint32 d_exp,
				    PDC_base_pd *pd, PDC_fpoint32 *res_out);
PDC_error_t PDC_ebc_fpoint64_read  (PDC_t *pdc, const PDC_base_m *m, PDC_uint32 num_digits, PDC_uint32 d_exp,
				    PDC_base_pd *pd, PDC_fpoint64 *res_out);

PDC_error_t PDC_ebc_ufpoint8_read  (PDC_t *pdc, const PDC_base_m *m, PDC_uint32 num_digits, PDC_uint32 d_exp,
				    PDC_base_pd *pd, PDC_ufpoint8 *res_out);
PDC_error_t PDC_ebc_ufpoint16_read (PDC_t *pdc, const PDC_base_m *m, PDC_uint32 num_digits, PDC_uint32 d_exp,
				    PDC_base_pd *pd, PDC_ufpoint16 *res_out);
PDC_error_t PDC_ebc_ufpoint32_read (PDC_t *pdc, const PDC_base_m *m, PDC_uint32 num_digits, PDC_uint32 d_exp,
				    PDC_base_pd *pd, PDC_ufpoint32 *res_out);
PDC_error_t PDC_ebc_ufpoint64_read (PDC_t *pdc, const PDC_base_m *m, PDC_uint32 num_digits, PDC_uint32 d_exp,
				    PDC_base_pd *pd, PDC_ufpoint64 *res_out);

PDC_error_t PDC_bcd_fpoint8_read   (PDC_t *pdc, const PDC_base_m *m, PDC_uint32 num_digits, PDC_uint32 d_exp,
				    PDC_base_pd *pd, PDC_fpoint8 *res_out);
PDC_error_t PDC_bcd_fpoint16_read  (PDC_t *pdc, const PDC_base_m *m, PDC_uint32 num_digits, PDC_uint32 d_exp,
				    PDC_base_pd *pd, PDC_fpoint16 *res_out);
PDC_error_t PDC_bcd_fpoint32_read  (PDC_t *pdc, const PDC_base_m *m, PDC_uint32 num_digits, PDC_uint32 d_exp,
				    PDC_base_pd *pd, PDC_fpoint32 *res_out);
PDC_error_t PDC_bcd_fpoint64_read  (PDC_t *pdc, const PDC_base_m *m, PDC_uint32 num_digits, PDC_uint32 d_exp,
				    PDC_base_pd *pd, PDC_fpoint64 *res_out);

PDC_error_t PDC_bcd_ufpoint8_read  (PDC_t *pdc, const PDC_base_m *m, PDC_uint32 num_digits, PDC_uint32 d_exp,
				    PDC_base_pd *pd, PDC_ufpoint8 *res_out);
PDC_error_t PDC_bcd_ufpoint16_read (PDC_t *pdc, const PDC_base_m *m, PDC_uint32 num_digits, PDC_uint32 d_exp,
				    PDC_base_pd *pd, PDC_ufpoint16 *res_out);
PDC_error_t PDC_bcd_ufpoint32_read (PDC_t *pdc, const PDC_base_m *m, PDC_uint32 num_digits, PDC_uint32 d_exp,
				    PDC_base_pd *pd, PDC_ufpoint32 *res_out);
PDC_error_t PDC_bcd_ufpoint64_read (PDC_t *pdc, const PDC_base_m *m, PDC_uint32 num_digits, PDC_uint32 d_exp,
				    PDC_base_pd *pd, PDC_ufpoint64 *res_out);

\end{tinycodeaux}
