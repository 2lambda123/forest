\chapter{Base types}
\label{chap:base-types}

\PADSL{} base types describe inidivual (typically small) values: numbers, strings, dates, and so on.
This chapter describes a large number of built-in \PADSL{} base types.
It is also possible to extend \PADSL{} with any
number of new base types; see \secref{sec:library-adding-new-base-types}.

\section{In-Memory Representation}
\label{sec:base-types-rep}

Each base type as an external and an in-memory representation, where
related base types share the same in-memory representation.  For
example, while there are 18 different string base types, all of them
use \PDC{string} as their in-memory representation.

This section reviews the different in-memory representation types.

\subsection{{\tt PDC\_char}}

\subsection{{\tt PDC\_string}}

\subsection{Integer types}

\PDC{int8} ... \PDC{int64}.

\PDC{uint8} ... \PDC{uint64}.

\subsection{Floating-point types}

\PDC{float32}, \PDC{float64}

\subsection{Fixed-point types}

\PDC{fpoint8} ... \PDC{fpoint64}

\PDC{ufpoint8} ... \PDC{ufpoint64}

\section{Base Type Mask}
\label{sec:base-type-mask}

Describe \PDCbasem{} and the 'normal' meaning for each value.

\section{Base Type Parse Descriptor}
\label{sec:base-type-parse-descriptor}

Describe \PDCbasepd{}.  This involves describing \PDCloct{}, \PDCpost{},
and \PDCerrCodet{}.  

Specific error codes are discussed below
when the base type read functions are described.

\section{Helper Types}

\PDCregexpt{}.

Other?

\section{Character Base Types}
\Pd{char}, \Pa{char}, \Pe{char}.

\subsection{Special character base types}

\Pd{countX}, \Pa{countX}, \Pe{countX}.

\Pd{countXtoY}, \Pa{countXtoY}, \Pe{countXtoY}.

\section{String Base Types}

\PFW{string}, \PaFW{string}, \PeFW{string}.

\Pd{string}, \Pa{string}, \Pe{string}.

\PME{string}, \PaME{string}, \PeME{stringME}.

\PCME{string}, \PaCME{string}, \PeCME{string}.

\PSE{string}, \PaSE{string}, \PeSE{string}.

\PCSE{stringCSE}, \PaCSE{string}, \PeCSE{string}.

\subsection{Special string base types}

\Pd{date}, \Pa{date}, \Pe{date}.

\section{Integer Base Types}

\PFW{int8} ... \PFW{int64}, \PaFW{int8} ... \PaFW{int64}, \PeFW{int8} ... \PeFW{int64}.

\PFW{uint8} ... \PFW{uint64}, \PaFW{uint8} ... \PaFW{uint64}, \PeFW{uint8} ... \PeFW{uint64}.

\Pd{int8} ... \Pd{int64}, \Pa{int8} ... \Pa{int64}, \Pe{int8} ... \Pe{int64}.

\Pd{uint8} ... \Pd{uint64}, \Pa{uint8} ... \Pa{uint64}, \Pe{uint8} ... \Pe{uint64}.

\Pb{int8} ... \Pb{int64}.

\Pb{uint8} ... \Pb{uint64}.

\Pebc{int8} ... \Pebc{int64}.

\Pebc{uint8} ... \Pebc{uint64}.

\Pbcd{int8} ... \Pbcd{int64}.

\Pbcd{uint8} ... \Pbcd{uint64}.

\Psbl{int8} ... \Psbl{int64}.

\Psbl{uint8} ... \Psbl{uint64}.

\Psbh{int8} ... \Psbh{int64}.

\Psbh{uint8} ... \Psbh{uint64}.

\section{Floating Point Base Types}

\Pd{float32}, \Pa{float32}, \Pe{float32}.

\Pd{float64}, \Pa{float64}, \Pe{float64}.

\Pb{float32}, \Pb{float64}.

\Psbl{float32}, \Psbl{float64}.

\Psbh{float32}, \Psbh{float64}.

\section{Fixed Point Base Types}

\Pebc{fpoint8} .. \Pebc{fpoint64}.

\Pebc{ufpoint8} .. \Pebc{ufpoint64}.

\Pbcd{fpoint8} .. \Pbcd{fpoint64}.

\Pbcd{ufpoint8} .. \Pbcd{ufpoint64}.

\Psbl{fpoint8} .. \Psbl{fpoint64}.

\Psbl{ufpoint8} .. \Psbl{ufpoint64}.

\Psbh{fpoint8} .. \Psbh{fpoint64}.

\Psbh{ufpoint8} .. \Psbh{ufpoint64}.

 
