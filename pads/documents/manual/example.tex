\chapter{Example}
\label{chap:example}
In this chapter, we use examples to give an overview of \pads{}.

\section{CLF: Common log format}
\label{sec:example:common-log-format}
The Common Log Format (CLF) \cite{clf} for web proxy/servers is used
to describe client request/server response pairs.  Such logs consist
of a sequence of entries, each of which contains information about a
single request/response.  In particular, each entry specifies the IP
address/hostname of the client, the owner of the TCP connection if
known, the authenticated user makeing the request if known, the time
of the request, the HTTP method used, the requested URL, the HTTP
version, a three-digit response code, and the number of bytes
returned. 

\begin{verbatim}
207.136.97.49 - - [15/Oct/1997:18:46:51 -0700] "GET /tk/p.txt HTTP/1.0" 200 30
include other record formats here.
\end{verbatim}

\section{\padsl{} description}
\label{sec:example:padsl-description}
Include top-down presentation of pads description

\section{Generated library}
\label{sec:example:generated-library}
Explain selected outputs of library.