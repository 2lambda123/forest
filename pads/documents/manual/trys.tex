\chapter{Ptry}
\label{chap:trys}
\cutname{trys.html}
\Ptry{}s provide the ability to do arbitrary look-ahead.  It
checkpoints the data source, then parses the data at the
current position using the underlying type of the \ptry{}.  It sets
the in-memory representation and parse descriptor for the \ptry{} as
indicated by the parse, and then rolls back the data source to the
check point.  

\section{Syntax}
\begin{tabular}{rcl}
\nont{try\_ty}    & \is{} & \Ptry{} identifier \opt{\nont{p\_formals}} \opt{\nont{p\_actuals}} ;\\[4ex]
\end{tabular}

\subsection{Example}
The following code fragment defines the \ptry{} type \cd{ForwardInt}.
\inputCode{code/try}
%
\noindent
The type \cd{entry\_t} uses \cd{ForwardInt} to parse the first digit
of an integer. The \punion{} \cd{VarInt} switches on whether this
first digit is greater than \cd{5} to determine if the integer should
be parsed as a 32- or 64-bit integer.  

\section{Generated library}
\subsection{In-memory representation}
\label{sec:try-rep}
The in-memory representation of a \ptry{} is the same as the
representation of underlying type.

\subsection{Mask}
\label{sec:try-masks}
The mask of a \ptry{} is the same as the
mask of the underlying type.

\subsection{Parse descriptor}
\label{sec:try-parse-descriptors}
The parse descriptor for a \Ptry{} is the same as the parse
descriptor for the underlying type.

\subsection{Operations}
The operations generated by the \pads{} compiler for a \Ptry{} are
those described in \chapref{chap:common-features}.


\subsubsection{Read function}
The error codes for \Ptry{} are the same as for the underlying type.

\subsubsection{Accumulator functions}
Accumulator functions for \Ptry{} are just as the accumulator
functions for the underlying type.

\subsubsection{Histogram functions}
Histogram functions for \Ptry{} are just as the histogram functions
for the underlying type.

\subsubsection{Clustering functions}
Clustering functions for \Ptry{} are just as the clustering
functions for the underlying type.
