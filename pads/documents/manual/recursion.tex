\chapter{Precurs}
\label{chap:recursion}
\cutname{recursion.html}

\Precur{} supports the description of recursive (\ie{},
tree-structured) data.

\section{Syntax}
\label{sec:recur-syntax}
The syntax for \Precur{} is given by the following BNF grammar fragment:

\begin{tabular}{rcl}
\nont{recur\_ty} & \is{} & \Precur{} \opt{(\nont{p\_ty})} identifier  \opt{\nont{p\_formals}};\\
                 & \alt{} & \Precur{} \nont{struct\_ty} \\
                 & \alt{} & \Precur{} \nont{union\_ty}\\[4ex]
\end{tabular}

\noindent
We explain the meaning of this syntax in the remainder of this chapter.
All non-terminals not defined in this grammar fragment were
defined previously:
\padsl{} types (\nont{p\_ty}) and parameter lists (\nont{p\_formals})
in \secref{sec:common-parameterization}, \nont{struct\_ty} in
\secref{sec:structs-syntax}, and \nont{union\_ty} in \secref{sec:unions-syntax}. 

\subsection{Examples}
The \Precur{} construct has two components: a (forward) declaration
and a definition. The forward declaration allows later types to refer
to the recursive type before its definition, just as in \C{}. The
definition specifies the structure of the recursive data structure.
Currently, the definition can be either a \Pstruct{} or \Punion{}.

We begin this example with a simple data fragment encoding a sorted
binary tree containing the numbers 3,4,7,9, and 10.
\begin{center}
\begin{verbatim}
(3,((4,(7,9)),10))
\end{verbatim}
\end{center}
In this encoding (based on the Newick format~\cite{newick}), a leaf is
an integer while a full tree is a matched pair of parentheses enclosing two
child trees separated by a comma. This encoding is naturally described
recursively, as shown in the simple format below:
\inputCode{code/binarytree}

In this example, types \cd{tree} and \cd{fullTree} are mutually
recursive. As recursive types are treated differently than other
types, we must explicitly designate at least one of them as a
recursive type.  In this example, we choose \cd{tree} to be the
recursive type and so begin with a forward declaration of \cd{tree} so
that it may be referenced in \cd{fullTree}.

\section{Generated library}

\Precur{}s are different than other types in two important ways.
First, in \C{}, mutual recursion (of both functions and types) is handled
through forward declarations. Therefore, for every recursive type, the
compiler generates type and function forward declarations in addition
to the standard type and function definitions.

Second, in general, we can't know the size of a data source described
with recursive types. Therefore, the data structures corresponding to
recursive types must be dynamically allocated. To accomplish this, the
compiler generates two types for each of the generated data
structures. One is an anonymous~\footnote{For technical reasons, we do
  not actually generate an anonymous type, but rather generate for the
  type a fresh name that is not chosen by the user.} type that is
generated from the recursive type's definition. The other is a named
type (where the name corresponds to the name of the recursive type)
that is a pointer to the anonymous type. 

We will use the \cd{tree} examples to illustrate the data structures
and functions generated for recursive types.

\subsection{In-memory representation}
\label{sec:recur-rep}

The in-memory representation of a recursive type has two parts: the
representation of the underlying type and a pointer to that
representation. The name of the underlying type is derived from the
name of the type by prefixing the name with an underscore ('\_'). In
our example, type \cd{tree} is defined as follows:
\inputCode{code/binarytree.BinarytreeRep}
The definition of \cd{struct \_tree\_s} is just the representation of
the underlying \Punion{}.

\subsection{Mask}
\label{sec:recur-masks}
The mask of a recursive type has two parts: the
mask of the underlying type and a pointer to that
mask. The name of the underlying type is derived from the
name of the type by prefixing the name with an underscore ('\_'). In
our example, type \cd{tree\_m} is defined as follows:
\inputCode{code/binarytree.BinarytreeCSM}

The definition of \cd{struct \_tree\_m\_s} is just the mask of the
underlying \Punion{}.  Note the difference from \ptypedef{}s, which
provide a new mask type for the new type.

\subsection{Parse descriptor}
\label{sec:recur-parse-descriptors}
The parse descriptor of a \Precur{} with name \cd{myRecur} is a \C{}
struct with name \pdSuf{myRecur}.  This struct has the fields
described in \secref{sec:common-parse-descriptor}. In addition, it
contains a field \cd{val}, the type of which is a pointer to the parse
descriptor for the underlying type.

For example, the parse descriptor type \pdSuf{tree} has
the following structure:
\inputCode{code/binarytree.BinarytreePD}

The definition of \cd{struct \_tree\_pd\_s} is just the parse
descriptor of the underlying \Punion{}.

\subsection{Operations}
The operations generated by the \pads{} compiler for a recursive type
are those described in \chapref{chap:common-features}. However, the
compiler generates two sets of functions for every recursive type. One
is the standard set of functions for the underlying type. Note that
these are not intended for use by the user.  The other set of
functions are those for operating on the type itself. These functions
essentially wrap the operations of the underlying type and handle the
dynamic allocation and deallocation of the representation and parse
descriptor.

The operations appear in \figref{figure:recur-ops}.

Note that the mask is handled differently from the parse descriptor
and representation. As the mask is an input to the parsing process, it
must be fully allocated and initialized before parsing begins.
However, due to the fact the size of the data is not known before
parsing it, the mask must able to guide the parsing process for data
of arbitrary size. The simplest way to accomplish this is to set the
(recursive) pointers in the mask to point back to the root of the
mask. In this way, we make the mask a cyclic data structure thereby
simulating an infinitely large data structure. However, as traversal
of the mask is driven by the data itself, this structure does not
introduce a danger of infinite loops.

In the current release there is no support for automatically
initializing masks to be cyclic. Instead, users must initialize the
pointers correctly themselves. To assist, we provide a macro that
bundles the various steps in initializing the mask. It allocates the
mask, initializes it and then sets the specified pointer field to
point back the root of the mask. It has the following signature:

\begin{code}
P_DynamicMaskInit(m, maskTy, baseMaskTy, maskVal, PATH_TO_MASK)
\end{code}

The argument \cd{m} is the (pointer) variable for the root of
the mask. Argument \cd{maskTy} is the type of the mask itself, and
\cd{baseMaskTy} is the underlying mask type. Argument \cd{maskVal}
specifies the default setting for all of the mask's fields. Finally,
\cd{PATH\_TO\_MASK}, specifies the path from the root of the mask to
the pointer field in the mask. Note that this assumes that there is
only one pointer field. In the case of more, the user must initialize
those herself. Here is an example use of this macro for our \cd{tree}
format:
\begin{code}
P_DynamicMaskInit(m, tree_m, _tree_m, READ_MASK, m->nested.left);
m->nested.right = m;
\end{code}
Note the need to initialize the second pointer separately. 

Code such as this can be specified for some of the template programs
with the macro \cd{CUSTOM\_MASK\_CODE}.

\begin{figure}
\inputCode{code/binarytree.BinarytreeOps}
% \begin{code}
% Perror_t tree_init (P_t *pads,tree *rep);

% Perror_t tree_pd_init (P_t *pads,tree_pd *pd);

% Perror_t tree_cleanup (P_t *pads,tree *rep);

% Perror_t tree_pd_cleanup (P_t *pads,tree_pd *pd);

% Perror_t tree_copy (P_t *pads,tree *rep_dst,tree *rep_src);

% Perror_t tree_pd_copy (P_t *pads,tree_pd *pd_dst,tree_pd *pd_src);

% void tree_m_init (P_t *pads,tree_m *mask,Pbase_m baseMask);

% Perror_t tree_read (P_t *pads,tree_m *m,tree_pd *pd,tree *rep);

% int tree_verify (tree *rep);

% int tree_genPD (P_t *,tree *,tree_pd *);

% ssize_t tree_write2buf (P_t *pads,Pbyte *buf,size_t buf_len,
%                         int *buf_full,tree_pd *pd,tree *rep);

% ssize_t tree_write2io (P_t *pads,Sfio_t *io,tree_pd *pd,tree *rep);

% ssize_t tree_write_xml_2buf (P_t *pads,Pbyte *buf,size_t buf_len,
%                              int *buf_full,tree_pd *pd,tree *rep,
%                              char const *tag,int indent);

% ssize_t tree_write_xml_2io (P_t *pads,Sfio_t *io, tree_pd *pd,tree *rep, 
% char const *tag,int indent);

% ssize_t tree_fmt2buf_final (P_t *pads,Pbyte *buf,size_t buf_len,int *buf_full,int *requestedOut,char const *delims,tree_m *m,tree_pd *pd,tree *rep);

% ssize_t tree_fmt2buf (P_t *pads,Pbyte *buf,size_t buf_len,int *buf_full,int *requestedOut,char const *delims,tree_m *m,tree_pd *pd,tree *rep);

% ssize_t tree_fmt2io (P_t *pads,Sfio_t *io,int *requestedOut,char const *delims,tree_m *m,tree_pd *pd,tree *rep);
% \end{code}
\caption{Prototypes of operations generated for
  the \Precur{} \texttt{tree}.}
\label{figure:recur-ops}
\end{figure}


\subsubsection{Read function}
The error codes for \Precur{}s are:

\tskip{}
\begin{center}
\begin{tabular}{l|p{4in}}
Code                           & Meaning \\ \hline
 \cd{P_OK}                 & Indicates no error occurred\\[1ex]
 \cd{P_ERR}                & Indicates an unspecified error occured.\\[1ex]
\end{tabular}
\end{center}

\noindent

\subsubsection{Accumulator functions}
Accumulator functions for recursive types are not yet supported. 

\subsubsection{Histogram functions}
Histogram functions for recursive types are not yet supported.


\subsubsection{Clustering functions}
Clustering functions for recursive types are not yet supported.
