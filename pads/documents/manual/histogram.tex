\chapter{Histogram}
\label{chap:histogram}
\cutname{histogram.html}

A histogram is a piecewise-constant approximation of an observed 
data distribution. It is used as a small space, approximate
synopsis of the underlying data distribution which are ofter too
large to be stored precisely. Histograms are built for each meaningful 
piece of a \pads{} description. \figref{figure:wsl-hist-report-1}
is an example report for the length field of a web server log data.     
%
\begin{figure*}
\begin{small}
\begin{verbatim}
*** Histogram Result *** 
From 0 to 397, with height 4016. 
From 398 to 398, with height 36122. 
From 399 to 423, with height 5584. 
From 424 to 424, with height 33250. 
From 425 to 499, with height 3126. 
*** Histogram Result *** 
From 0 to 196, with height 3286. 
From 197 to 197, with height 30430. 
From 198 to 313, with height 3233. 
From 314 to 314, with height 30430. 
From 315 to 499, with height 3655. 
*** Histogram Result *** 
From 0 to 242, with height 3686. 
From 243 to 243, with height 36122. 
From 244 to 441, with height 3720. 
From 442 to 442, with height 26074. 
From 443 to 499, with height 7035. 
*** Histogram Result *** 
From 0 to 93, with height 4204. 
From 94 to 94, with height 36122. 
From 95 to 206, with height 3000. 
From 207 to 210, with height 21496. 
From 211 to 499, with height 3890. 
. . . . . . . . . . . . . . . . . . . . . . 
\end{verbatim}
\end{small}
\caption{Portion of histogram report for length field of web server
log data.}
\label{figure:wsl-hist-report-1}
\end{figure*}
%
In this particular run, optimal 5-bucket histogram is built for 
every 500 values seen in the data source.

\section{Operations}
\figref{figure:histogram} shows the histogram functions declared 
for a \pads{} type.
\begin{figure}
\inputCode{hand_code/hist-declare}
\caption{Histogram functions generated for the \texttt{entry\_t}  type.}
\label{figure:histogram}
\end{figure}
%
These functions have the following behaviors:
\begin{description}
\item[\cd{entry\_t\_hist\_init}] Initializes histogram data
  structure. This function must be called before any data can be added
  to the histogram.
\item[\cd{entry\_t\_hist\_setConv}] Sets functions in histogram data
  structure, mapping between \cd{Pfloat64} type and \cd{entry_t} type. 
\item[\cd{entry\_t\_hist\_reset}] Reinitializes histogram data
  structure. This function can be used to set any point of the data
  source as the start point of a new run. But it can't be used to 
  reset any previous defined parameters.
\item[\cd{entry\_t\_hist\_cleanup}] Deallocates all memory associated
  with histogram.
\item[\cd{entry\_t\_hist\_add}] Inserts a data value. This function 
  is called once a new record is coming. Any data type with an
  associated mapping function to \cd{Pfloat64} is considered as a 
  meaningful type. This function tracks fields with meaningful type
  and legal values only.  
\item[\cd{entry\_t\_hist\_report2io}] Writes summary report for
  histogram \cd{h} to \cd{*outstr}. 
\item[\cd{entry\_t\_hist\_report}] Writes summary report for
  histogram \cd{h} to screen. 
\end{description}
\figref{figure:wsl-hist-hand} illustrates a sample use of histogram
functions for printing a summary of CLF \cd{entry_t}s.  
\begin{figure}
\inputCode{hand_code/histogram}
\caption{Simple use of histogram functions for the
  \texttt{entry\_t} type from CLF data.}
\label{figure:wsl-hist-hand}
\end{figure}

\section{Customization}
\label{sec:accumulators-customization}
Users are allowed to customize various aspects of histogram by 
setting the appropriate field in the histogram data structure, 
which contains: 

\begin{description}
\item[\cd{INIT\_N}] is a \cd{Puint64} denoting the number of values
  for histogram to summarize. If the number of values in the data
  source exceeds \cd{INIT\_N}, histograms will be built on each
  \cd{INIT\_N} data values respectively, until the end of data
  source is reached. 

\item[\cd{INIT\_B}] is a \cd{Puint32} denoting the number of buckets
  in final histogram. As \cd{INIT\_B} increases, accuracy of final
  histogram approximation is increased, while more time and space is consumed.

\item[\cd{INIT\_M}] is a \cd{Pint64} denoting an upper bound of data
  values in data source. Time consumed increases in poly-logrithm of
  \cd{INIT\_M}, so \cd{INIT\_M} can be set very large if little about
  data values in data source is known.

\item[\cd{INIT\_ISE}] is a \cd{Pint8} denoting whether buckets in the final
  histogram are required to be of the same width. If \cd{INIT\_ISE} is
  set to be non-zero, all buckets have equal width. In this case,
  the time needed is in linear-\cd{INIT\_N}, and only constant space will
  be used. However, the result histogram will have less accuracy.

\item[\cd{INIT\_ISO}] is a \cd{Pint8} denoting whether final histogram
  is required to be optimal or not. This parameter is valid only
  when \cd{INIT\_ISE} is zero. If \cd{INIT\_ISO} is set to be
  non-zero, the result histogram will be the most accurate one among
  all \cd{INIT\_B} bucket histograms. However, the time needed is in
  cubic-\cd{INIT\_N}, which could be extremely slow, and the space
  needed is in linear-\cd{INIT\_N}, since all data values in each
  \cd{INIT\_N} section are required to be stored.

\item[\cd{INIT\_n}] is a \cd{Pint8} denoting whether norm 1 or norm 2
  is used to measure accuracy of final histogram. Currently, norm 1
  measurement is supported only when all the data values are
  stored. In other words, it is supported only when \cd{INIT\_ISE} is
  zero, and \cd{INIT\_ISO} is non-zero. 

\item[\cd{INIT\_e}] is a \cd{Pfloat64} denoting error tolerance of the
  final histogram. This parameter is valid only when non-optimal
  result is allowed, namely both \cd{INIT\_ISE} and \cd{INIT\_ISO} are
  zeroes. The final histogram will be guaranteed to be no worse than
  poly- (1+\cd{INIT\_e}) times of the optimal one, but the time and
  space needed increase as the error tolerance decreases.  

\item[\cd{INIT\_scale}] is a \cd{Pint64} denoting scale factor for
  each data value. This parameter is important for computing, but will
  not affect the final result. For example, if the data source can
  take values up to \cd{64} bits, the overall \cd{SSE} could need
  as many as \cd{128} bits, which exceeds the representation limit of
  \pads{}. In this case, \cd{INIT\_scale} is needed.        

\item[\cd{entry\_t\_toFloat}] is a function pointer, taking
  \cd{entry\_t} as input parameter, and returning corresponding
  \cd{Pfloat64}. Histograms will handle \cd{Pfloat64} type data value 
  only. Any type with a well-defined conversion function to
  \cd{Pfloat64} is considered as a meaningful type, and could be summarized
  correctly by histograms. By default, all base types in \pads{} have
  conversion functions to \cd{Pfloat64}. And users are allowed write their
  own conversion functions, and use \cd{entry_t_hist_setConv} function
  to set them.
   
\item[\cd{entry\_t\_fromFloat}] is a function pointer, taking
  \cd{Pfloat64} as input parameter, and returning corresponding
  \cd{entry\_t} type. Any type without a well-defined conversion
  function from \cd{Pfloat64} may not be printed correctly. By
  default, all base types other than \cd{Pstring} in \pads{} have
  conversion functions from \cd{Pfloat64}. And users are allowed to write their
  own conversion functions, and use \cd{entry_t_hist_setConv} function
  to set them.

\end{description}
