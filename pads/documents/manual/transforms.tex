\chapter{Ptrans}
\label{chap:trans}
\cutname{transforms.html}
\ptrans{} (short for transformations) allow users to specify both a physical and a logical
representation for data, both as \pads{} types.  Users supply
a pair of functions that map between the physical and logical
representations.    The physical type is used to parse the data, which
is then transformed according to the supplied transformation
function.  The logical type is used in tools, such as accumulators,
\xml{} convertion, \etc{}  The \cd{write2buf} and \cd{write2io}
functions first call the reverse translations to map the logical
representation to the physical representation before printing the
data.  This mechanism has been designed to give users some control
over the logical representation of their data and to make it easier to
add new base types.

\section{Syntax}
\begin{tabular}{rcl}
\nont{p\_trans\_spec}    & \is{} & \nont{expression} \opt{\nont{p\_actuals}} \\
\nont{p\_trans\_maskmap} & \is{} & \Pmaskmap\ \nont{expression} \cd{;} \\
\nont{trans\_ty}         & \is{} & \Ptrans{} identifier \opt{\nont{p\_formals}} \{  \\
                      &       & \nont{p\_trans\_spec} \cd{:} \nont{p\_ty} \cd{<=>} \nont{p\_ty} \cd{:} \nont{p\_trans\_spec} \cd{;} \\
                      &       & \opt{\nont{p\_trans\_maskmap}} \cd{\}}\\[4ex]
\end{tabular}

\noindent
We explain the meaning of this syntax in the remainder of this chapter.
All non-terminals not defined in this grammar fragment were
defined previously, as follows.
Expressions (\term{expression}) are any C-expression,
\padsl{} types (\nont{p\_ty}), parameter lists (\nont{p\_formals}), and lists of actual arguments (\nont{p\_actuals})
in \secref{sec:common-parameterization}. 

\subsection{Examples}
The following \Ptrans{} declaration defines a type of 32-bit integers
represented on disk as ASCII strings of length \cd{size}.
\inputCode{code/hexint32}
\noindent
In this case, the type \cd{Pstring_FW(:8:)} denotes the physical type
while \cd{Puint32} denotes the logical type.  The function
\cd{toInt32} converts the parsed string and the corresponding parse
descriptor into a 32-bit integer and parse descriptor, while
the function \cd{toHexString} performs the reverse translation.  
The notation \cd{toHexString(:pads,size:)} allows the user to pass
additional arguments to the translation function.  The parameter
\cd{pads} is a special variable in scope within the body of the
\ptrans{} expression.  It is bound to the \pads{} handle active 
when operations related to the \ptrans{} are invoked.
Figures~\ref{fig:trans-toInt32}and \ref{fig:trans-toHexString} show
the implementations of the two conversion functions.  Note that the
function \cd{toHexString} needs the \pads{} handle to pass as an
argument to the \cd{Pstring_cstr_copy} function that copies the \C{}
string \cd{str} into the desired \cd{Pstring}.  

\begin{figure}
\inputCode{code/trans-toInt32}
\caption{Code to convert a hex string of length \texttt{size} into a \texttt{Puint32}}
\label{fig:trans-toInt32}
\end{figure}

\begin{figure}
\inputCode{code/trans-toHexString}
\caption{Code to convert a \texttt{Puint32} into a hex string of length \texttt{size}.}
\label{fig:trans-toHexString}
\end{figure}

\subsection{Supplying a mask conversion function}
If the masks of the logical and physical types differ, or if the
desired conversion from the logical to the physical mask is not the
identity function, the \Ptrans{} declaration must specify a function
to convert the logical mask to the physical mask.  For example, the
\Ptrans{} \cd{hexint32withMask} in \figref{fig:trans-hex32withMask}
uses the function \cd{cnvMask} to map 
the mask for the logical type to one for the physical type.  In the
example, the mapping is simply the identify function, but the user is
free to supply an arbitrary mapping function as long as it has the
necessary type signature.  The generated library calls this function
to convert the logical mask to the physical mask just before invoking
the parser for the physical representation.

\begin{figure}
\inputCode{code/hexint32withMask}
\caption{\Ptrans{} declaration with a supplied mask translation function.}
\label{fig:trans-hex32withMask}
\end{figure}



\subsection{Special variables}
\label{sec:trans-special-variables}
Within the various expression contexts of \Ptrans{} declarations, 
the special variable \cd{pads} is in scope.

\myvskip{1ex}
\begin{center}
\begin{tabular}{|l|l|l|l|}
\hline
Variable & Type & Contexts & Binding \\\hline \hline
\texttt{pads}  & \texttt{P\_t}     & all & Active \pads{} handle. \\ \hline
\end{tabular}
\end{center}
\myvskip{1ex}


\section{Generated library}
\subsection{In-memory representation}
\label{sec:trans-rep}
The in-memory representation of a \ptrans{} is the same as the
representation of logical type.

\subsection{Mask}
\label{sec:trans-masks}
The mask of a \ptrans{} is the same as the
mask of the logical type.

\subsection{Parse descriptor}
\label{sec:trans-parse-descriptors}
The parse descriptor for a \Ptrans{} is the same as the parse
descriptor for the logical type.

\subsection{Operations}
The operations generated by the \pads{} compiler for a \Ptrans{} are
those described in \chapref{chap:common-features}.


\subsubsection{Read function}
The error codes for \Ptrans{} are:

\tskip{}
\begin{center}
\begin{tabular}{l|p{4in}}
Code                           & Meaning \\ \hline
 \cd{P_NO_ERR}                 & Indicates no error occurred\\[1ex]
 \cd{P_TRANSFORM_FAILED}       & Indicates a failure during transformation.\\[1ex]
\end{tabular}
\end{center}

\noindent

\subsubsection{Accumulator functions}
Accumulator functions for \Ptrans{} are just as the accumulator
functions for the logical type.

\subsubsection{Histogram functions}
Histogram functions for \Ptrans{} are just as the histogram functions
for the logical type.

\subsubsection{Clustering functions}
Clustering functions for \Ptrans{} are just as the clustering
functions for the logical type.
