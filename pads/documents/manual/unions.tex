\chapter{Punions}
\label{chap:unions}
\section{Syntax}
To express variations.  Two forms: in-place and switched.
\subsection{in-place}
\begin{verbatim}
Punion name ( param list ){
 qualifier type name constraint comment
...
}
\end{verbatim}
tries each branch in turn til one found that succeeds in finding an
element of the indicated type and satisifies the constraint.

\subsection{switched}
jumps to appropriate branch indicated by switch expression.
switch must be of integral type. 
\begin{verbatim}
Punion branches(a_uint32 a){
  Pswitch (a) {
  case 1 : a_int32  number : number % 2 == 0;
  case 2 : a_string_SE(:"EOR":) name;
  default: compute PDC_int32 def = 3; 
  }
}
\end{verbatim}

\section{Generated library}
\subsection{In-memory representation}
\label{sec:unions-rep}
\subsection{Mask}
\label{sec:unions-masks}
\subsection{Parse descriptor}
\label{sec:unions-parse-descriptors}
\subsection{Operations}
init/cleanup rep
init/cleanup ed
\subsubsection{read}
  error codes
\subsubsection{Accumulator functions}

