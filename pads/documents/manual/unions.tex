\chapter{Punions}
\label{chap:unions}
\cutname{unions.html}
\Punion{}s are used to express variations in data.  \pads{}
supports two forms of union: switched and in-place.  The first form
supports data sources where there is an indication (\ie, a switch) in
the data prior to the union indicating which alternative should be
chosen.  The second form supports data sources where no such switch is
present.  The default for this case is to try the branches in order
until one parses without any errors.  There is also a qualifier
(\Plongest{}) that indicates the parser should take the branch that
consumes the most input. 
\section{Syntax}
\label{sec:unions-syntax}
\begin{tabular}{rcl}
\nont{union\_field} & \is{} & \nont{full\_field} \alt{} \nont{comp\_field} \alt{} \nont{literal\_field}
			     \alt{} \nont{array\_field} \alt{} \nont{opt\_field} \\[1ex]
\nont{branch}     & \is{} & \Pcase{} \nont{expression} : \nont{union\_field}
                    \alt{}  \Pdefault : \nont{union\_field}\\[1ex]
\nont{branches}   & \is{} & \nont{branch} \alt{} \nont{branch} \nont{branches} \\[1ex]
\nont{switched}   & \is{} & \Pswitch{} (\nont{expression})\{ \nont{branches} \}\\[1ex]
\nont{in\_place}  & \is{} & \nont{union\_field} \alt{} \nont{union\_field} \nont{in\_place}\\[1ex]
\nont{union\_bdy} & \is{} & \nont{switched} \alt{} \nont{in\_place}\\[1ex]
\nont{union\_ty}  & \is{} & \opt{\Plongest{}} \Punion{} identifier \opt{\nont{p\_formals}} \{ \\
&& \quad \nont{union\_bdy} \\
&& \}\ \opt{ \Pwhere{} \ \{\ \nont{predicate}\ \}}; \\[4ex]

\end{tabular}

\noindent
We explain the meaning of this syntax in the remainder of this chapter.
All non-terminals not defined in this grammar fragment were
defined previously, as follows.
Full fields (\nont{full\_field}), in-line array declarations
(\nont{array\_field}), and in-line option declarations (\nont{option\_field})
appear in \secref{sec:structs-full-fields}, 
computed fields (\nont{comp\_field}) in
\secref{sec:structs-computed-fields}, 
literals (\nont{literal\_fields}) in \secref{sec:common-literals}
and
\padsl{} parameter lists (\nont{p\_formals}) in \secref{sec:common-parameterization}.
Expressions (\nont{expression}) represent any \C{} expression. 


\subsection{Example: Switched \Punion{}s}
The \pads{} declarations in \figref{fig:switched-union} describe data
which uses an integer tag to determine the format of the rest of the
data. 
The \Pstruct{} \cd{choice} specifies that the integer field \cd{which}
should be passed to the switched \Punion{} \cd{branches}.   
The \cd{branches} declaration describes three alternatives, depending
upon the value of the tag \cd{which}.  
%
\begin{figure}
\inputCode{code/union.switch}
\caption{Switched \Punion{} for describing data variations determined
  by tags earlier in the data source.}
\label{fig:switched-union}
\end{figure}
%
A tag value of \cd{1} indicates an unsigned integer will follow:
\begin{verbatim}
 1 4
\end{verbatim}
while a tag
value of \cd{2} indicates a string terminated by an end-of-record
mark:  
\begin{verbatim}
 2 hello
\end{verbatim}
Any other value for the tag will fall into the default clause of the
union, which indicates that no further data
exists:
\begin{verbatim}
 3
\end{verbatim}

\subsection{Example: In-place \Punion{}s}
The following in-place \Punion{} describes a data fragment that is
either a resolved or a symbolic IP address:

\inputCode{code/host_t}
%
\noindent
The (omitted) types \cd{nIP} and \cd{sIP} describe named and symbolic
IP addresses, respectively.
The comments embedded in the description give an example of each of the two
forms.   With in-place \Punion{}s, the parser tries each of the branches
in turn until it finds one that matches the data without any errors.

\subsection{Switched \Punion{}s}
\subsubsection{\Pswitch{}}
The expression on which a switched \Punion{} branches can be any \C{}
expression of integer type (as in a \kw{switch} statement in \C{}).
Typically, this expression is computed from a parameter to the
switched \Punion{}.

\subsubsection{Branches}
The body of a switched \Punion{} is a non-empty sequence of branches.
Each branch in a switched \Punion{} can have one of two forms: a
\Pcase{} statement, or a \Pdefault{} statement.  The \Pcase{} form
specifies an integer value.  During parsing, the first \Pcase{}
expression whose value equals the \Pswitch{} expression is selected as
the active description. If no \Pcase{} expression
matches, the \Pdefault{} expression (if present) matches instead.

\subsection{Union fields}
The body of each branch in a switched \Punion{} is a \nont{union\_field},
while the body of each in-place \Punion{} is a non-empty sequence of
such fields. 
There are five varieties of union fields: 
full fields (\secref{sec:structs-full-fields}),
computed fields (\secref{sec:structs-computed-fields}), 
in-place array declarations (\secref{sec:structs-arrays-inline}),
in-place option declarations (\secref{sec:structs-options-inline}),
and literal fields, all borrowed from \Pstruct{}s.
The only semantic change from \Pstruct{}s  is
that in\Punion{}s, earlier fields are not in scope for later fields
because only one branch of a union can be active at a time.

The name of a branch in either kind of union is the name of the
declared identifier for full and computed fields.  For literal fields,
it is the literal itself, unless the programmer specifies a different
name using the \Pfrom{} form.  The \Pfrom{} form must be used when the
literal is not a valid \C{} identifier.  For example, the following
in-place union uses the \Pfrom{} form to provide names for the
string literal \literal{\cd{"*"}} and the regular expression literal \cd{\Pre{}
  \literal{"/a+/"}}.

\inputCode{code/union.literal}


\cut{
Optional fields in data sources can be described using an in-place
\Punion{} with a computed field.  For example, suppose a data source
contains a sequence of fields separated by vertical bars, where each
field is either an unsigned 32-bit integer or is omitted.  The
description in \figref{fig:union-option} 
uses the \Punion{} \cd{intOpt} to specify such a data source.
The \cd{intOpt} \Punion{} relies on the parameter \cd{defVal} to set
a default value for the omitted field.

\begin{figure}
\inputCode{code/union.option}
\caption{A \Punion{} with a computed field describes an optional field.}
\label{fig:union-option}
\end{figure}
}

\subsection{In-line declarations}
\label{sec:unions-inline}
In-line declarations in \Punion{}s have the same form as in
\Pstruct{}s \cf{} \secref{sec:structs-inline}.

\subsection{\Plongest{}}
By default, in-place \Punion{}s commit to the first branch that parses
without any errors. Adding the \Plongest{} qualifier to the
\Punion{} declaration, however, indicates that the parser should
instead select the \textit{longest} match, \ie{}, the match that
consumes the most input. 

\subsection{Optional \texttt{Pwhere} clause}
If given, a \Pwhere{} clause expresses constraints over the entirety
of a \Punion{} value.  Special constants \cd{tag} and \cd{val} are
in scope, of the tag and value types for the union, respectively
(\cf{} \secref{sec:unions-tags}).  The
first indicates which branch of the union matched, while the second
contains the representation of the matched branch. 
Within the context of a 
\Pparsecheck{} clause, constants \cd{begin} and \cd{end}, each of type 
\Ppost{} are available.  Constant \cd{begin} is bound to the input
position of the beginning of the \punion{}; \cd{end} is bound to its end.
If the predicate given in
the \Pwhere{} clause evaluates to false (\ie{}, zero), the error code
in the associated parse descriptor will indicate a user-constraint
error has occurred.  

\section{Generated library}
\subsection{Tags}
\label{sec:unions-tags}
In addition to the types generated for every \pads{} specification,
the \pads{} compiler generates an extra type 
declaration for every \Punion{}: a enumerated type with one component
for each branch in the union, plus an extra component corresponding to
a match failure.  The names of the tags correspond to the names of the
branches in the union, unless that name
has already been defined 
elsewhere.  In this case, the name of the tag is 
\texttt{unionName\_branchName}.  
The name of the tag enumeration is the name of
the \pads{} specification with the \cd{_tag} suffix.
For example, the generated enumeration type
for the \Punion{} \cd{branches} is the following:

\inputCode{code/union-impl-branches-tag}
%


\subsection{In-memory representation}
\label{sec:unions-rep}
The in-memory representation of both forms of \Punion{} is 
a \C{} struct containing a \cd{tag} field to indicate which branch of the
union has been populated followed by a \cd{val} field storing the union
itself.  We represent unions as \C{} unions, with one component per
non-literal branch of the union.  
The representation-related type declarations for
the \Punion{} \cd{branches} appear in \figref{fig:punion-rep}.

\begin{figure}
\inputCode{code/union-impl-branches-rep}
\caption{Type declarations related to the in-memory representation of
  the \Punion{} \texttt{branches}.}
\label{fig:punion-rep}
\end{figure}

\subsection{Mask}
\label{sec:unions-masks}
The mask of a \Punion{} is a \C{} struct.  
For each full and computed field in the union,
there is a corresponding field in the mask, the type of which is the
type of the mask type for that field.   For example, the mask type
\csmSuf{branches} has the following structure:

\inputCode{code/union-impl-branches-mask}
%
\noindent
Union masks have one additional field \cd{unionLevel} that allows the
programmer to toggle behavior at the level of the union as a whole.

\subsection{Parse descriptor}
\label{sec:unions-parse-descriptors}
The parse descriptor of a \Punion{} is a \C{} struct, with all
the fields described in \secref{sec:common-parse-descriptor}. In
addition,  there is a \cd{tag} field indicating which branch of the
union was populated during parsing and a \cd{val} field which stores
the parse descriptor of the populated branch, represented as a \C{}
union.  The parse descriptor declarations corresponding to the
\pads{} type \cd{branches}
appear in \figref{fig:punion-pd}.

\begin{figure}
\inputCode{code/union-impl-branches-pd}
\caption{Type declarations related to the parse descriptor for
  the \Punion{} \texttt{branches}.}
\label{fig:punion-pd}
\end{figure}

\subsection{Operations}
The operations generated by the \pads{} compiler for a \Punion{} are
those described in \chapref{chap:common-features}.  In addition, there
is an extra function that converts a value of the tagtype for the
union to a string.  For a \Punion{} named \cd{myUnion}, this function
has the name \cd{myUnion_tag2str}.  
For the \Punion{}
\cd{branches}, the prototypes for all the generated functions appear in
\figref{fig:punion-ops}.
\begin{figure}
\inputCode{code/union-impl-branches-ops}
\caption{Prototypes of operations generated for
  the \Punion{} \texttt{branches}.}
\label{fig:punion-ops}
\end{figure}



\subsubsection{Read function}
The read function for a switched \Punion{} evaluates the \Pswitch{}
expression and uses a \C{} \cd{switch} statement to jump to the
appropriate branch.  If there is no \Pdefault{} branch and none of the
\Pcase{} branches match,  the read function will return the error code
\cd{P_UNION_MATCH_ERR} and set the \cd{tag} fields of the parse
descriptor and in-memory representation to the error tag. 

For an in-place \Punion{} without the \Plongest{} qualifier, the read
function speculatively reads each 
branch in turn until it finds one that parses without errors.  Before
reading a branch, the read function marks the 
current location in the input.  It then tries to read the data
described by the type of the branch.  If the nested read function
succeeds and any user-level constraint on the branch also succeeds,
the read function commits to the parse, sets the \cd{tag} fields to
the name of the successful branch, and returns
\cd{P_NO_ERR}. If an error occurs, the read function aborts the
read, rolling back the input to the marked location, and tries the next
branch.  If the last branch in an in-place \Punion{} fails, the read
function returns the error code
\cd{P_UNION_MATCH_ERR} and sets the \cd{tag} fields of the parse
descriptor and in-memory representation to the error tag. 
Errors that occur duing parsing branches of an in-line \Punion{} are
surpressed because of the speculative nature of the parsing.

For in-place \Punion{}s with the \Plongest{} qualifier, the read
function parses each branch, preserving in the in-memory representation the
non-erroneous branch that occupied the most space in the physical source.
Ties are resolved in favor of the earliest branch.  Non-erroneious branches
that occupy zero space in the physical representation, such as
\Pcompute{} fields or non-matching options, are considered matches;
the first such branch will be preserved in the in-memory
representation if no longer non-erroneous branch is found.   The read
function reports an error if no branch matches without an error.

The error codes for \Punion{}s are:

\tskip{}
\begin{center}
\begin{tabular}{l|p{3in}}
Code                           & Meaning \\ \hline
 \cd{P_NO_ERR}                 & Indicates no error occurred\\[1ex]
 \cd{P_UNION_MATCH_ERR}         & Indicates that no branch of the
                                    union parsed without error.\\[1ex]
\end{tabular}
\end{center}
\noindent

\subsubsection{Accumulator functions}
Accumulator functions for \Punion{}s are described in \chapref{chap:accumulators}. 

\subsubsection{Histogram functions}
Histogram functions for \Punion{}s are described in
\chapref{chap:histogram}. 

\subsubsection{Clustering functions}
Clustering functions for \Punion{}s are described in
\chapref{chap:cluster}. 
