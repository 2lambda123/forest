\chapter{Punions}
\label{chap:unions}
\Punion{}s are used to express variations in data.  \pads{}
supports two forms of union: switched and in-place.  The first form
supports data sources where there is an indication (\ie, a switch) in
the data prior to the union indicating which alternative should be
chosen.  The second form supports data sources where no such switch is
present.  In this case, the read code tries the branches in order
until it finds one in which no errors occurred during parsing.
\section{Syntax}
\begin{tabular}{rcl}
\nont{qualifier}  & \is{} & \Pomit{} \alt{} \Pendian{}\\[1ex]
\nont{qualifiers}  & \is{} & \nont{qualifier} \alt{} \nont{qualifier} \nont{qualifiers}\\[1ex]
\nont{constraint} & \is{} & : \nont{predicate}\\[1ex]
\nont{full\_field} & \is{} & \opt{\nont{qualifiers}}
     \nont{p\_ty} \opt{\nont{p\_actual\_list}} identifier 
       \opt{\nont{constraint}}; \opt{\nont{p\_comment}} \\[1ex]
\nont{literal\_field} & \is{} & \term{char\_lit}; \alt{} \term{str\_lit};\\[1ex]
\nont{comp\_field} & \is{} & \Pcompute{} \nont{c\_ty} identifier \cd{=} \nont{expression};\\[1ex]
\nont{field} & \is{} & \nont{full\_field} \alt{} \nont{literal\_field}  \alt{} \nont{comp\_field}\\[1ex]
\nont{fields} & \is{} & \nont{field} \alt{} \nont{field} \ \nont{fields}\\[1ex]

\nont{union\_field} & \is{} & \nont{full\_field} \altP{} \nont{comp\_field}\\[1ex]
\nont{branch}     & \is{} & \Pcase \nont{expression} : \nont{union\_field}\\[1ex]
                  & \alt{} & \Pdefault : \nont{full\_field}\\[1ex]
\nont{branches}   & \is{} & \nont{branch} \alt{} \nont{branch} \nont{branches} \\[1ex]
\nont{switched}   & \is{} & \Pswtich (\nont{expression})\{ branches \}\\[1ex]
\nont{union\_bdy} & \is{} & \nont{in\_place} \alt{} \nont{switched}\\[1ex]
\nont{union\_ty}  & \is{} & \Punion{} identifier \opt{\nont{formals}} \{ union\_bdy \} \\[4ex]

\end{tabular}

\subsection{switched}
jumps to appropriate branch indicated by switch expression.
switch must be of integral type. 
\begin{verbatim}
Punion branches(a_uint32 a){
  Pswitch (a) {
  case 1 : a_int32  number : number % 2 == 0;
  case 2 : a_string_SE(:"EOR":) name;
  default: compute PDC_int32 def = 3; 
  }
}
\end{verbatim}

\subsection{in-place}
\begin{verbatim}
Punion name ( param list ){
 qualifier type name constraint comment
...
}
\end{verbatim}
tries each branch in turn til one found that succeeds in finding an
element of the indicated type and satisifies the constraint.


\section{Generated library}
\subsection{In-memory representation}
\label{sec:unions-rep}
\subsection{Mask}
\label{sec:unions-masks}
\subsection{Parse descriptor}
\label{sec:unions-parse-descriptors}
\subsection{Operations}
init/cleanup rep
init/cleanup ed
\subsubsection{read}
  error codes
\subsubsection{Accumulator functions}

