\section{Introduction}
\label{sec:intro}

{\em Data description languages} are a class of domain specific
languages for specifying {\em ad hoc data formats}, from billing 
records to TCP packets to scientific data sets to server logs.  Examples 
of such languages include 
\bro~\cite{paxson:bro}, \datascript{}~\cite{gpce02}, \demeter~\cite{lieberherr+:class-dictionaries},
\packettypes{}~\cite{sigcomm00}, \padsc{}~\cite{fisher+:pads}, 
\padsml{}~\cite{mandelbaum+:padsml}  and
\xsugar~\cite{xsugar2005}, among others.  All of these languages
generate parsers from data descriptions.  In addition, and unlike
conventional parsing tools such as Lex and Yacc, many also automatically
generate auxiliary tools ranging from printers to \xml{} converters to
visitor libraries to visualization and editor tools.

In previous work, we developed the {\em Data Description Calculus}
(\ddcold{}), a calculus of simple, orthogonal type constructors,
designed to capture the core features of many existing type-based data
description languages~\cite{fisher+:next700ddl,fisher+:ddcjournal}.
This calculus had a multi-part denotational semantics that interpreted
the type constructors as (1) parsers that transform external bit
strings into internal data representations and {\em parse descriptors}
(representations of parser errors), (2) types for the data
representations and parse descriptors, and (3) types for the parsers
as a whole.  We proved that this multi-part semantics was coherent in
the sense that the generated parsers always have the expected types
and generate representations that satisfy an important {\em
canonical forms} lemma.

The \ddcold{} has been very useful already, helping us debug and
improve several aspects of \padsc{}~\cite{fisher+:pads}, and serving
as a guide for the design of \linebreak \padsml{}~\cite{mandelbaum+:padsml}.
However, this initial work on the \ddcold{} told only a fraction of the
semantic story concerning data description languages.  As mentioned
above, many of these languages not only provide parsers, but
also other tools.  Amongst the most common auxiliary tools
are printers, as reliable communication between programs, either through
the file system or over the Web, depends upon both input (parsing) 
and output (printing).

In this work, we begin to address the limitations of
\ddcold{} by specifying a printing semantics for the
various features of the calculus.  We also
prove a collection of theorems for the new semantics that serve as
duals to our theorems concerning parsing.  This new printing semantics
has many of the same practical benefits as our older parsing 
semantics: We can
use it as a check against the correctness of our printer
implementations and as a guide for the
implementation of future data description languages.  


% First, we extend \ddcold{} with
% abstractions over types, which provides a basis for specifying the
% semantics of \padsml{}. In the process, we also improve upon the
% \ddcold{} theory by making a couple of subtle changes. For example, we
% are able to eliminate the complicated ``contractiveness'' constraint
% from our earlier work. Second, .

% The main practical benefit of the calculus has been as a guide for our
% implementation. Before working through the formal semantics, we
% struggled to disentangle the invariants related to polymorphism. After
% we had defined the calculus, we were able to implement type
% abstractions as \ocaml{} functors in approximately a week.  Our new
% printing semantics was also very important for helping us define and
% check the correctness of our printer implementation.  We hope the
% calculus will serve as a guide for implementations of \pads{} in
% other host languages.  

% In summary, this work makes the following key contributions:
% \begin{itemize}
% \item We simultaneously specify both a parsing and a printing semantics
%   for the \ddc{}, a calculus of polymorphic, dependent types.
% \item We prove that \ddc{} parsers and printers are type safe
%   and well-behaved as defined by a canonical forms theorem.
% \end{itemize}

In this paper, we give an overview of the calculus, its dual semantics
and their properties.  A companion technical report contains a
complete formal specification~\cite{fisher+:popl-sub-long}.  In
comparison to our previous work on the \ddcold{} at POPL
06~\cite{fisher+:next700ddl}, the calculus we present here has been
streamlined in several subtle, but useful ways.  It has also been
improved through the addition of polymorphic types.  We call this new
polymorphic variant \ddc{}.  These improvements and extensions,
together with proofs, appear in Mandelbaum's
thesis~\cite{mandelbaum:thesis} and in a recently submitted journal
article~\cite{fisher+:ddcjournal}.  This abstract reviews the \ddc{}
and extends all the previous work with a printing semantics and
appropriate theorems.  To be more specific,
sections~\ref{sec:ddc-syntax} through \ref{sec:ddc-sem} present the
extended \ddc{} calculus, focusing on the semantics of polymorphic
types for parsing and the key elements of the printing semantics.
Then, \secref{sec:meta-theory} shows that both parsers and printers in
the \ddc{} are type correct and furthermore that parsers produce pairs
of parsed data and parse descriptors in {\em canonical form}, and that
printers, given data in canonical form, print successfully. We briefly
discuss related work in \secref{sec:related}, and conclude in
\secref{sec:conc}.

%%% Local Variables: 
%%% mode: latex
%%% TeX-master: "paper"
%%% End: 
