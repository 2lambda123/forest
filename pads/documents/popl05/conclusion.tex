\section{Conclusion} 
\label{sec:conclusion}
Ad hoc data is pervasive and valuable: in industry, in medicine, and in scientific research.  Such data tends to have poor documentation, to contain various kinds of errors, and to be voluminous.  Unlike well-behaved data in standardized relational or \xml{} formats, such data has little or no tool support, forcing data analysts and scientists to waste valuable time writing brittle, custom code, even if all they want to do is convert their data into a well-behaved format.  To improve the situation, various researchers have developed data description languages such as \pads{}, \datascript{}, and \packettypes{}.  Such languages allow analysts to write terse, declarative descriptions of ad hoc data.  A compiler then generates a parser and customized tools.  Because these languages are tailored to their domain, they can provide useful services automatically while a more general purpose tool, such as lex/yacc or \perl{}, cannot.

In this work, we have taken the first steps toward specifying a semantics for this class of languages by defining our data description calculus \ddc{}.  This calculus, which has a simple set of orthogonal primitives, is expressive enough to describe the features of \pads{}, \datascript{}, and \packettypes{}.  In keeping with the spirit of the data description languages, our semantics is transformational: instead of simply recognizing a collection of input strings, we specify how to transform those strings into canonical in-memory representations annotated with error information.  Furthermore, we prove that the error information is meaningful, allowing analysts to rely on the error summaries rather than having to re-vet the data by-hand. 

