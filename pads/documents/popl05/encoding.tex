\section{Encoding DDLs in the DDC}
\label{sec:encodings}

\subsection{IPADS Translation}
\label{sec:trans-sl}

% We now define some syntactic sugar for multi-argument versions of type
% constructors that only take one or two arguments. Some of these
% abbreviations take a list of bindings $b$, each of the form $x{:}t$.

% \begin{align*}
%   &\begin{array}{l}
%     \plamA{x{:}\ity}{\ty} = \plam{x}{\ity}{\ty} \\
%     \plamA{x{:}\ity,\vec{b}}{\ty} = \plam{x}{\ity}{\plamA{\vec{b}}{\ty}}
%   \end{array}
%   &
%   &\begin{array}{l}
%     \gS^* x:\ty,y:\ty' = \gS x:\ty.\ty' \\
%     \gS^* x:t,\vec b   = \gS x:\ty.\gS^* \vec b
%   \end{array}
%   \\ \\
%   &\begin{array}{l}
%     +^* \ty,\ty' = \ty + \ty'\\
%     +^* \ty,\vec{\ty} = \ty + (+^* \vec{\ty})
%   \end{array}
%   &
%   &\begin{array}{l}
%     \&^* \ty,\ty' =\pand \ty {\ty'}\\
%     \&^* \ty,\vec{\ty} = \pand \ty {(\&^* \vec{\ty})}
%   \end{array}
% \end{align*}

\begin{figure*}
{\small
\fbox{$\ctxt \turn {prog} \cipads \ty \; \text{prog}$}

\[
 \infer[\text{Prog-One}]{ 
    \ctxt \turn {pty} \cipads \ty \; \text{prog}
  }{
    \ctxt \turn {pty} \cipads \ty : \kty
  }
\quad
  \infer[\text{Prog-Def}]{ 
    \ctxt \turn \ga = {pty}; \; {prog} \cipads \ty \; \text{prog}
  }{
    \ctxt \turn {prog}[{pty}/\ga] \cipads \ty \; \text{prog}
  }
\quad
  \infer[\text{Prog-RecDef}]{ 
    \ctxt \turn  \Prec{}\; \ga = {pty}; \; {prog} \cipads \ty \; \text{prog}
  }{
    \ctxt \turn {prog}[\Prec{}\; \ga.{pty}/\ga] \cipads \ty \; \text{prog}
  }
\]

\fbox{$\ctxt \turn {pty}  \cipads \ty : \kind$}

\[
  \infer[\text{Base}]{ 
    \ctxt \turn \Pbase e \cipads \pbase e : \kty
  }{
    \ctxt \turn e : \ity & (\Ikind(C) = \ity \to \kty)
  }
\quad
  \infer[\text{Pfun}]{ 
    \ctxt \turn \Pfun (x:\ity) = {pty} \cipads \plam x
    \ity \ty : \ity \to \kind
  }{
    \ectxt{x{:}\ity} \turn {pty} \cipads \ty : \kind
  }
\quad
  \infer[\text{App}]{ 
    \ctxt \turn {pty} \; e \cipads \ty \; e : \kind
  }{
    \ctxt \turn {pty} \cipads \ty : \ity \to \kind & \ctxt \turn e : \ity
  }
\]

\[
  \infer[\text{Pstruct}]{
    \begin{array}{l}
    \ctxt \turn \Pstruct \{x_1{:}{pty}_1 \dots x_n{:}{pty}_n\}
    \cipads \\
    \qquad \gS \; x_1{:}\ty_1. \cdots \gS \; x_{n-1}{:}\ty_{n-1}.\ty_n : \kty      
    \end{array}
  }{ 
    \begin{array}{c}
%       \ctxt_1 = \ctxt \quad \ity_{i-1} = \iprod{\itsem[\ty_{i-1}]} {\itpdsem[\ty_{i-1}]}\\
%       \ctxt_i = \ctxt_{i-1},x_{i-1}{:} \ity_{i-1} \quad 
%       (\text{for} \; i = 2 \ldots n) \\
%       \ctxt_i \turn {pty}_i \cipads \ty_i : \kty \quad
%       (\text{for} \; i = 1 \ldots n)
      \ctxt_i = \ctxt,x_1{:}\ity_1, \dots, x_{i-1}{:} \ity_{i-1} \\
      \ity_{i} = \iprod{\itsem[\ty_{i}]} {\itpdsem[\ty_{i}]} \quad
      \ctxt_i \turn {pty}_i \cipads \ty_i : \kty
    \end{array}
  }
\quad
  \infer[\text{Punion}]{
    \begin{array}{l}
      \ctxt \turn \Punion \{x_1{:}{pty}_1 \dots x_n{:}{pty}_n\}
      \cipads \\
      \ty_1 + \dots + \ty_n + \pfalse : \kty
    \end{array}
  }{ 
    \ctxt \turn {pty}_i \cipads \ty_i : \kty
  }
\]

\[
  \infer[\text{Palt}]{
    \ctxt \turn \Palt \{x_1{:}{pty}_1 \dots x_n{:}{pty}_n\} \cipads
    \ty_1 \& \dots \& \ty_n: \kty
  }{ 
    \ctxt \turn {pty}_i \cipads \ty_i : \kty
  }
\quad
  \infer[\text{Popt}]{
    \ctxt \turn \Popt \; {pty} \cipads
     \psum \ty {} \ptrue
  }{
    \ctxt \turn {pty} \cipads \ty : \kty
  }
\quad
  \infer[\text{Pwhere}]{
    \begin{array}{l}
      \ctxt \turn {pty} \; \Pwhere \, x.e \cipads  \\
      \qquad
      \pset x \ty {\iif {\pdok {x.\codefont{pd}}} \; \ithen e \; \ielse
        \itrue}
    \end{array}
  }{ 
     \ctxt \turn {pty} \cipads \ty : \kty & 
     \stsem[e,\ectxt{x{:}\iprod{\itsem[\ty]} {\itpdsem[\ty]}},\iboolty]
  }
\]

\[
  \infer[\text{Parray}]{
    \ctxt \turn \iParray{{pty}}{{pty_{sep}}}{{pty_{term}}}{} \cipads 
    \pseq \ty {\pscan {\ty_s}} {\pterm f {\ty_t}} : \kty
  }{ 
       \begin{array}{lr}
         \ctxt \turn {pty} \cipads \ty : \kty & 
         \ctxt \turn {pty}_{sep} \cipads \ty_s : \kty \\ 
         \ctxt \turn {pty}_{term} \cipads \ty_t : \kty &
         (f = \ilam x{}\ifalse)\\
       \end{array}
  }
\quad
  \infer[\text{Pcompute}]{ 
    \ctxt \turn \Pcompute{} \; e \cipads \pcompute e \ity : \kty
  }{
    \stsem[e,\ctxt,\ity]
  }
\]

\[
  \infer[\text{Plit}]{ 
    \ctxt \turn \Plit \const \cipads 
    \pscan {\pabsorb {\pset \var \ty {\ivar = \iconst}}} : \kty
  }{
    \stsem[\const,\ctxt,\ity] & \defty \ity \ty
  }
\quad
  \infer[\text{Var}]{ 
    \ctxt \turn \ga \cipads \ga
  }{}    
\quad
  \infer[\text{Prec}]{ 
    \ctxt \turn \Prec{}\; \ga.{pty} \cipads \pmu \ga \ty
  }{
    \ctxt \turn {pty} \cipads \ty
  }
\]
}
  \caption{Encoding \ipads{} in \ddc{}}
  \label{fig:encode-ipads}
\end{figure*}

The function $\Ikind$ provides the kind of base types $C$.

\subsection{Sketches of Other Features and DDLs}

Overlays are syntactic sugar. Formalism allows  us to check them for
well-formedness.
\\
\noindent
Switched unions:
\begin{code}
  if p then T1 else T2 =
    \{y:\{x:unit|!p\} + T1 | p\} +  T2\linebreak

  if p then T1 else T2 =
    S y:(\{x:unit|!p\} + T1).(\{x:unit| p\} + T2)\linebreak
  
  switch(e)\{
    c1 => T1
    c2 => T2
    ...
    default => Tdef
  \} \(\cipads{}\)
  (\(\gl\) c. 
   if c = c1 then T1 else
   if c = c2 then T2 else
   ...
   Tdef) e
\end{code}

Features of other languages that we can encode:
\begin{itemize}
\item \datascript{}: primitive types, set types (enums and bitmask
  sets), arrays (greedy and fixed size), constraints, type parameters,
  non-overlapping lables.
\item \packettypes{}: arrays (fixed size and greedy), where clauses,
  structures, overlays, alternation.
\end{itemize}
Note, as far as I can tell, neither \packettypes{} nor \datascript{}
supports greedy arrays that include erroneous records.

The syntax of a \datascript{}-style array named ${id}$ with elements of
type ${dty}$ is ${dty}\;{id}[{length}]$, with length optional. If the
user supplies the length of the sequence then the array parser reads
exactly that number of elements. Otherwise, the parser continues until
an element constraint is violated or the input is completely consumed.

The first case be encoded in a straightforward manner with \ddc{}
sequences: 
\[
  \infer{
    \ctxt \turn {dty} \; [{length}] \cipads 
    \pseq \ty {\ptrue{}} {\pterm f {\pfalse{}}} : \kty
  }{ 
    \ctxt \turn {dty} \cipads \ty : \kty & 
    (f = \codefont{isLength} \; {length})
  }
\]
As binary data has no separators, we use $\ptrue$. The function {\tt
  isLength} is a curried function taking a length and a pair of rep
and PD arrays and returning a boolean indicating whether both arrays
have the specified length. As there is no lookahead-terminator, we use
$\pfalse$ (thereby guaranteeing that the array will not be terminated
due to lookahead).

For the second case, we need a way to terminate if the next element
will not parse correctly without consuming it. We cannot use the
termination predicate as it can only look at elements that have
already been parsed and hence consumed from the input. Therefore, we
need a way to encode the notion of a failed element parse in the
terminator type. To do so, we define $\pnot \ty$ to be:
 {\small
\[
\pset x {\psum \ty {} \ptrue} {\icaseg {x.rep} {\_}
  {\ifalse} {\_} {\itrue}}
\]}
\noindent
This type attempts to parse a $\ty$. On success, the rep will be a left
injection. The constraint in the set type will therefore fail. If a
$\ty$ cannot be parsed, the sum will default to $\ptrue$, the rep will
be a right injection, and the constraint will succeed. The use of the
sum in the underlying type is critical as it allows the set-type to be
error free even if parsing a $\ty$ fails.

With $\pnotn$, we can encode the unbounded \datascript{} array as
follows:
\[
  \infer{
    \ctxt \turn {dty} \; [{length}] \cipads 
    \pseq \ty {\ptrue{}} {\pterm {\ilam x {} \ifalse} {\pnot \ty}} : \kty
  }{ 
    \ctxt \turn {dty} \cipads \ty : \kty 
  }
\]
Note that the termination predicate is trivially false, as we use the
lookahead-terminator exclusively to terminate the array.

%%% Local Variables: 
%%% mode: latex
%%% TeX-master: "semantics"
%%% End: 
