\section{Encoding \ddl{}s in \ddc{}}
\label{sec:encodings}

We can better understand the data description languages mentioned
earlier by translating their constructs into the types of \ddc{}. We
start with the translation of \ipads{}, which captures many of the
common features of \ddl{}s. We then discuss features of \pads{},
\datascript{}, and \packettypes{} that are not found in \ipads{}.
\trversion{
Finally, we briefly discuss some limitations of \ddc{}.
}
\subsection{\ipads{} Translation}
\label{sec:trans-sl}

\trversion{
\begin{figure*}
{\small
\fbox{$\ctxt \turn \mathit{prog} \cipads \ty \; \text{prog}$}

\[
 \infer[\text{Prog-One}]{ 
    \ctxt \turn \itmv \cipads \ty \; \text{prog}
  }{
    \ctxt \turn \itmv \cipads \ty : \kty
  }
\quad
  \infer[\text{Prog-Def}]{ 
    \ctxt \turn \ga = \itmv; \; \mathit{prog} \cipads \ty \; \text{prog}
  }{
    \ctxt \turn \mathit{prog}[\itmv/\ga] \cipads \ty \; \text{prog}
  }
\quad
  \infer[\text{Prog-RecDef}]{ 
    \ctxt \turn  \Prec{}\; \ga = \itmv; \; \mathit{prog} \cipads \ty \; \text{prog}
  }{
    \ctxt \turn \mathit{prog}[\Prec{}\; \ga.\itmv/\ga] \cipads \ty \; \text{prog}
  }
\]

\fbox{$\ctxt \turn \itmv  \cipads \ty : \kind$}

\[
  \infer[\text{Base}]{ 
    \ctxt \turn \pbase e \cipads \pbase e : \kty
  }{
    \ctxt \turn e : \ity & (\Ikind(C) = \ity \to \kty)
  }
\quad
  \infer[\text{Pfun}]{ 
    \ctxt \turn \Pfun (x:\ity) = \itmv \cipads \plam x
    \ity \ty : \ity \to \kind
  }{
    \ectxt{x{:}\ity} \turn \itmv \cipads \ty : \kind
  }
\quad
  \infer[\text{App}]{ 
    \ctxt \turn \itmv \; e \cipads \ty \; e : \kind
  }{
    \ctxt \turn \itmv \cipads \ty : \ity \to \kind & \ctxt \turn e : \ity
  }
\]

\[
  \infer[\text{Pstruct}]{
    \begin{array}{l}
    \ctxt \turn \Pstruct \{x_1{:}\itmv_1 \dots x_n{:}\itmv_n\}
    \cipads \\
    \qquad \gS \; x_1{:}\ty_1. \cdots \gS \; x_{n-1}{:}\ty_{n-1}.\ty_n : \kty      
    \end{array}
  }{ 
    \begin{array}{c}
%       \ctxt_1 = \ctxt \quad \ity_{i-1} = \iprod{\itsem[\ty_{i-1}]} {\itpdsem[\ty_{i-1}]}\\
%       \ctxt_i = \ctxt_{i-1},x_{i-1}{:} \ity_{i-1} \quad 
%       (\text{for} \; i = 2 \ldots n) \\
%       \ctxt_i \turn \itmv_i \cipads \ty_i : \kty \quad
%       (\text{for} \; i = 1 \ldots n)
      \ctxt_i = \ctxt,x_1{:}\ity_1, \dots, x_{i-1}{:} \ity_{i-1} \\
      \ity_{i} = \iprod{\itsem[\ty_{i}]} {\itpdsem[\ty_{i}]} \quad
      \ctxt_i \turn \itmv_i \cipads \ty_i : \kty
    \end{array}
  }
\quad
  \infer[\text{Punion}]{
    \begin{array}{l}
      \ctxt \turn \Punion \{x_1{:}\itmv_1 \dots x_n{:}\itmv_n\}
      \cipads \\
      \ty_1 + \dots + \ty_n + \pfalse : \kty
    \end{array}
  }{ 
    \ctxt \turn \itmv_i \cipads \ty_i : \kty
  }
\]

\[
  \infer[\text{Palt}]{
    \ctxt \turn \Palt \{x_1{:}\itmv_1 \dots x_n{:}\itmv_n\} \cipads
    \ty_1 \& \dots \& \ty_n: \kty
  }{ 
    \ctxt \turn \itmv_i \cipads \ty_i : \kty
  }
\quad
  \infer[\text{Popt}]{
    \ctxt \turn \Popt \; \itmv \cipads
     \psum \ty {} \ptrue : \kty
  }{
    \ctxt \turn \itmv \cipads \ty : \kty
  }
\quad
  \infer[\text{Pwhere}]{
    \begin{array}{l}
      \ctxt \turn \itmv \; \Pwhere \, x.e \cipads  \\
      \qquad
      \pset x \ty {\iif {\pdok {x.\codefont{pd}}} \; \ithen e \; \ielse
        \itrue}
    \end{array}
  }{ 
     \ctxt \turn \itmv \cipads \ty : \kty & 
     \stsem[e,\ectxt{x{:}\iprod{\itsem[\ty]} {\itpdsem[\ty]}},\iboolty]
  }
\]

\[
  \infer[\text{Parray}]{
    \ctxt \turn \iParray{\itmv}{{\itmv_{sep}}}{{\itmv_{term}}}{} \cipads 
    \pseq \ty {\pscan {\ty_s}} {\pterm f {\ty_t}} : \kty
  }{ 
       \begin{array}{lr}
         \ctxt \turn \itmv \cipads \ty : \kty & 
         \ctxt \turn \itmv_{sep} \cipads \ty_s : \kty \\ 
         \ctxt \turn \itmv_{term} \cipads \ty_t : \kty &
         (f = \ilam x{}\ifalse)\\
       \end{array}
  }
\quad
  \infer[\text{Pcompute}]{ 
    \ctxt \turn \Pcompute{} \; e \cipads \pcompute e \ity : \kty
  }{
    \stsem[e,\ctxt,\ity]
  }
\]

\[
  \infer[\text{Plit}]{ 
    \ctxt \turn \Plit \const \cipads 
    \pscan {\pabsorb {\pset \var \ty {\ivar = \iconst}}} : \kty
  }{
    \stsem[\const,\ctxt,\ity] & \defty \ity \ty
  }
\quad
  \infer[\text{Var}]{ 
    \ctxt \turn \ga \cipads \ga
  }{}    
\quad
  \infer[\text{Prec}]{ 
    \ctxt \turn \Prec{}\; \ga.\itmv \cipads \pmu \ga \ty
  }{
    \ctxt \turn \itmv \cipads \ty
  }
\]
}
  \caption{Encoding \ipads{} in \ddc{} with type checking}
  \label{fig:encode-ipads-with-tc}
\end{figure*}

 \reminder{Add poly. context to all typing judgments that appear in
   premises. See email (subject:Prec) for details on how to fix translation}

\begin{figure*}
{\small
\fbox{$ \mathit{prog} \cipads \ty \; \text{prog}$}

\[
 \infer[\text{Prog-One}]{ 
     \itmv \cipads \ty \; \text{prog}
  }{
     \itmv \cipads \ty
  }
\qquad
  \infer[\text{Prog-Def}]{ 
     \ga = \itmv; \; \mathit{prog} \cipads \ty \; \text{prog}
  }{
     \mathit{prog}[\itmv/\ga] \cipads \ty \; \text{prog}
  }
\qquad
  \infer[\text{Prog-RecDef}]{ 
      \Prec{}\; \ga = \itmv; \; \mathit{prog} \cipads \ty \; \text{prog}
  }{
     \mathit{prog}[\Prec{}\; \ga.\itmv/\ga] \cipads \ty \; \text{prog}
  }
\]

\fbox{$ \itmv  \cipads \ty$}

\[
  \infer[\text{Base}]{ 
     \pbase e \cipads \pbase e
  }{}
\qquad
  \infer[\text{Pfun}]{ 
     \Pfun (x:\ity) = \itmv \cipads \plam x
    \ity \ty
  }{
    \itmv \cipads \ty
  }
\qquad
  \infer[\text{App}]{ 
     \itmv \; e \cipads \ty \; e
  }{
     \itmv \cipads \ty
  }
\qquad
  \infer[\text{Pstruct}]{
    \begin{array}{l}
     \Pstruct \{x_1{:}\itmv_1 \dots x_n{:}\itmv_n\}
    \cipads \\
    \qquad \gS \; x_1{:}\ty_1. \cdots \gS \; x_{n-1}{:}\ty_{n-1}.\ty_n
   \end{array}
  }{ 
    \itmv_i \cipads \ty_i
  }
\]

\[
  \infer[\text{Punion}]{
    \begin{array}{l}
       \Punion \{x_1{:}\itmv_1 \dots x_n{:}\itmv_n\}
      \cipads \\
      \qquad \ty_1 + \dots + \ty_n + \pfalse
    \end{array}
  }{ 
     \itmv_i \cipads \ty_i
  }
\qquad
  \infer[\text{Palt}]{
     \Palt \{x_1{:}\itmv_1 \dots x_n{:}\itmv_n\} \cipads
    \ty_1 \& \dots \& \ty_n
  }{ 
     \itmv_i \cipads \ty_i
  }
\qquad
  \infer[\text{Popt}]{
     \Popt \; \itmv \cipads
     \psum \ty {} \ptrue
  }{
     \itmv \cipads \ty
  }
\]

\[
  \infer[\text{Pwhere}]{
    \begin{array}{l}
       \itmv \; \Pwhere \, x.e \cipads  \\
      \qquad
      \pset x \ty {\iif {\pdok {\nrm x.\codefont{pd}}} \; \ithen e \; \ielse
        \itrue}
    \end{array}
  }{ 
    \itmv \cipads \ty
  }
\qquad\qquad
  \infer[\text{Parray}]{
     \iParray{\itmv}{{\itmv_{sep}}}{{\itmv_{term}}}{} \cipads 
    \pseq \ty {\pscan {\ty_s}} {\pterm f {\ty_t}}
  }{ 
    \itmv \cipads \ty & 
    \itmv_{sep} \cipads \ty_s &
    \itmv_{term} \cipads \ty_t &
    (f = \ilam x{}\ifalse)
  }
\]

\[
  \infer[\text{Pcompute}]{ 
     \Pcompute{} \; e{:}\ity \cipads \pcompute e \ity
  }{}
\qquad
  \infer[\text{Plit}]{ 
     \Plit \const \cipads 
    \pscan {\pabsorb {\pset \var \ty {\var = \const}}}
  }{
    \defty \const \ty
  }
\qquad
  \infer[\text{Var}]{ 
     \ga \cipads \ga
  }{}    
\qquad
  \infer[\text{Prec}]{ 
     \Prec{}\; \ga.\itmv \cipads \pmu \ga \ty
  }{
     \itmv \cipads \ty
  }
\]
}
  \caption{Encoding \ipads{} in \ddc{}}
  \label{fig:encode-ipads}
\end{figure*}
}

\poplversion{
\begin{figure}
{\small
\fbox{$ \mathit{prog} \cipads \ty \; \text{prog}$}

\[
 \infer{ 
     \itmv \cipads \ty \; \text{prog}
  }{
     \itmv \cipads \ty
  }
\qquad
  \infer{ 
     \ga = \itmv; \; p \cipads \ty \; \text{prog}
  }{
     p[\itmv/\ga] \cipads \ty \; \text{prog}
  }
\qquad
  \infer{ 
      \Prec{}\; \ga = \itmv; \; p \cipads \ty \; \text{prog}
  }{
     p[\Prec{}\; \ga.\itmv/\ga] \cipads \ty \; \text{prog}
  }
\]

\fbox{$ \itmv  \cipads \ty$}

\[
  \infer{
%    \begin{array}{l}
       \Punion \{x_1{:}\itmv_1 \dots x_n{:}\itmv_n\}
      \cipads %\\
      %\qquad 
      \ty_1 + \dots + \ty_n + \pfalse
%    \end{array}
  }{ 
     \itmv_i \cipads \ty_i
  }
\]

\[
%\quad
  \infer{
%    \begin{array}{l}
       \itmv \; \Pwhere \, x.e \cipads  %\\
      %\qquad
      \pset x \ty {\iif {\pdok {\nrm x.\codefont{pd}}} \; \ithen e \; \ielse
        \itrue}
%    \end{array}
  }{ 
    \itmv \cipads \ty
  }
\]

\[
%\qquad\qquad
  \infer{
     \iParray{\itmv}{{\itmv_{sep}}}{{\itmv_{term}}}{} \cipads 
    \pseq \ty {\pscan {\ty_s}} {\pterm f {\ty_t}}
  }{ 
    \itmv \cipads \ty & 
    \itmv_{sep} \cipads \ty_s &
    \itmv_{term} \cipads \ty_t &
    (f = \ilam x{}\ifalse)
  }
\]

\[
%\qquad\qquad
  \infer{
     \Popt \; \itmv \cipads
     \psum \ty {} \ptrue
  }{
     \itmv \cipads \ty
  }
\qquad
\infer{ 
     \Plit \const \cipads 
    \pscan {\pabsorb {\pset \var \ty {\var = \const}}}
  }{
    \defty \const \ty
  }
\]
}
\caption{Selected Rules for Encoding \ipads{} in \ddc{}. The full
  collection of rules appears in \appref{app:ipads-to-ddc}.}
  \label{fig:encode-ipads}
\end{figure}
}

\trversion{
In \secref{sec:ddc-example}, we showed how an
\ipads{} description would appear in \ddc{} and informally described
how \ipads{} is encoded in \ddc{}. In \figref{fig:encode-ipads}, we
formalize this translation. The judgment $p \cipads \ty\;
\text{prog}$ indicates that the \ipads{} program $p$ is encoded
as \ddc{} type $\ty$, while $\itmv \cipads \ty$ does the same
for \ipads{} types $\itmv$.

As much of the translation is straightforward, we mention only a few
important points.}
\poplversion{
We formalize the translation from \ipads{} to \ddc{}, described
informally in \secref{sec:ddc-example}, with two judgments:
 $p \cipads \ty\; \text{prog}$ indicates that the \ipads{} program
$p$ is encoded as \ddc{} type $\ty$, while $\itmv \cipads \ty$
does the same for \ipads{} types $\itmv$.

As much of the translation is straightforward, we present only
selected rules in \figref{fig:encode-ipads}.}  Notice we add $\pbot$ as the
last branch of the \ddc{} sum when translating \Punion{} so that the
parse will fail if none of the branches match rather than returning
the result of the last branch.  \trversion{We based this behaviour
  directly on the actual \pads{} language.}  In the translation of
\Pwhere{}, we only check the constraint if the underlying value parsed
with no errors. For \Parray{}s, we add simple error recovery by
scanning for the separator type.  This behaviour allows us to easily
skip erroneous elements. We use the $\pscann$ type in the same way for
\padskw{Plit}, as literals often appear as field separators in
\Pstruct{}s.  We also absorb the literal as its value is known
statically.  We use the function ${\rm Ty}(c)$ to determine the
correct type for the particular literal. For example, a string
literal would require a \Pstring{} type.

\subsection{Beyond \ipads{}}

We now give semantics to three features not found in \ipads{}:
\pads{} switched unions, \packettypes{} overlays, and \datascript{} arrays.

A switched union, like a \Punion, indicates variability in the data
format with a set of alternative formats (branches). However, instead
of trying each branch in turn, the switched union takes an expression that
determines which branch to use. Typically, this expression depends upon data read earlier in the parse. Each branch is preceded by a tag,
and the first branch whose tag matches the expression is selected.
If none match then the default branch $\itmv_{\text{def}}$ is chosen.
The syntax of a switched union is $\Pswitch{}\;e\;\{\overrightarrow{e
  \Rightarrow x{:}\itmv}\; \itmv_{\text{def}}\}$.

To aid in our translation of \Pswitch{}, we define a type $\iif e\;
\ithen {\itmv_1}\; \ielse {\itmv_2}$ that allows us to choose
between two types conditionally : 
\trversion{
Our first attempt to
translate the conditional relies on \ddc{} sums:
\[
\infer{
  \iif e\; \ithen {\itmv_1}\; \ielse {\itmv_2} 
  \cipads \psum
  {\pset y {\psum {\pset x \ptrue {\inotop\;e}} {} {\ty_1}} e}
  {} {\ty_2}
}{
  \itmv_1 \cipads \ty_1 &
  \itmv_2 \cipads \ty_2
}
\]
If $e$ is true, then the nested constraint will fail causing the left
branch of the nested sum to fail and $\ty_1$ will be used to parse
the data. As $e$ is true, the second constraint will succeed, leaving
us with the left branch of the root sum. Conversely, if $e$ is false,
the first constraint will be satisfied, but the second will not,
causing us to parse the data with $\ty_2$. 

Unfortunately. while this seems to work, it will fail if there is an
error in $\ty_1$ when $e$ is true. The error will cause errors in the
set type despite the satisfaction of its constraint, thereby causing
the parser to use $\ty_2$. To address this problem, we use 
intersections:
}
{\small
\[
\infer{
    \begin{array}{l}
    \iif e\; \ithen {\itmv_1}\; \ielse {\itmv_2} 
    \cipads \\
    \qquad \qquad c \iprodi \pand {({\psum {\pset x \ptrue {\inotop\;e}} {} {\ty_1}})}
    {({\psum {\pset x \ptrue {e}} {} {\ty_2}})}
    \end{array}
}{
  \itmv_1 \cipads \ty_1 &
  \itmv_2 \cipads \ty_2 &
  (c  = \pcompute {\codefont{if}
          \;e\;\codefont{then\;1\;else\;2\;}}{\codefont{Pint}})
}
\]}%
\noindent
The computed value $c$ records which branch of the conditional is
selected.  If the condition $e$ is true, $c$ will be 1, the left-hand
side of the intesection will parse $\ty_1$ and the right will parse nothing.
Otherwise, $c$ will be 2, the left-hand side will parse nothing and the
right $\ty_2$.
% The computed value $c$ indicates which branch is to be taken and the
% intersection parses exactly that branch of the conditional-type.  We
% use a combination of set and sum type to guard each branch with a
% condition such that the branch is only parsed if the condition is
% false. We guarantee that exactly one branch is parsed by guarding the
% two branches with mutually exclusive conditions. 

Now, we can encode \Pswitch{} as syntactic sugar for a series of cascading
conditional types.
{\small
\[
\begin{array}{l}
\Pswitch{}\;e\;\{\\
\quad e_1 \Rightarrow x_1{:}\itmv_1\\ 
\quad \dots \\
\quad e_n \Rightarrow x_n{:}\itmv_n\\
\quad \itmv_{\text{def}} \}
\end{array}
\quad
 \mathbf{=}
\quad
\begin{array}{l} 
    \iif {e = e_1}\; \ithen {\itmv_1}\; \ielse{}\\
%    \iif {e = e_2}\; \ithen {\itmv_2}\; \ielse{}\\
    \dots\\
    \iif {e = e_n}\; \ithen {\itmv_1}\; \ielse{}\\
    \itmv_{\text{def}}
\end{array}
\]}%
\noindent
Note that we can safely replicate $e$ as the
host language is pure.

Next, we consider the {\it overlay} construct found in \packettypes{}.
An overlay allows us ``to merge two type specifications by embedding
one within the other, as is done when one protocol is {\it
  encapsulated} within another. Overlay[s] introduce additional
substructure to an already existing field.''~\cite{sigcomm00}.  For
example, consider a network packet from a fictional protocol FP, where
the packet body is represented as a simple byte-array. 
\begin{code}
FPPacket = \Pstruct \{
  header : FPHeader;
  body   : \Pbyte \Parray{}(\Pnosep,\Peof);
\}\linebreak

IPinFP = \Poverlay FPPacket.body \Pwith IPPacket
\end{code}
Type \Pnosep{} indicates that there are no separators between elements
of the byte array. It can be encoded as \cd{\Pcompute{}(():unit)}, as
this type consumes no data and produces a unit value without errors.
The overlay creates a new type \cd{IPinFP} where the body field is an
\cd{IPPacket} rather than a simple byte array.

We have developed a translation of the overlay syntax into \ddc{} (not
shown due to space constraints).
Although overlays are conceptually intuitive, we discovered a critical
subtlety, not mentioned by the authors, when formalizing their 
semantics.  Any expressions in the
original type that refer to the overlayed field may no longer be well
typed after applying the overlay.
\trversion{
We thought to disallow such expressions in the
overlayed type.However, we found this to be a difficult, if not
impossible task. More importantly, such a restriction is unnecessary.
Instead, we designed the translation so that the new type is
checked for well formedness after the overlay process, an easy task
in the \ddc{} framework.
}
\poplversion{
It is therefore necessary for the translation to check the new type 
for well formedness after the overlay process, which is an easy task
in the \ddc{} framework.
}

Finally, we introduce \datascript{}-style arrays for binary data,
$\itmv\;[\mathit{length}]$. They are parameterized by an optional
length field, instead of a separator and terminator. If the user
supplies the length of the sequence, the array parser reads exactly
that number of elements.  Otherwise, the parser continues until an
element constraint is violated or the input is completely consumed.

Fixed-length arrays can be encoded in a straightforward manner with
\ddc{} sequences:
{\small
\[
  \infer{
    \itmv \; [\mathit{length}] \cipads 
    \pseq \ty {\ptrue{}} {\pterm {f} {\pfalse{}}}
  }{ 
    \itmv \cipads \ty & 
    (f = \ilam {((len,elts),p)} {} {\codefont{len} = \mathit{length}})
  }
\]}
As these arrays have neither separators nor terminators, we use
$\ptrue$ (always succeeds, parsing nothing)
and $\pfalse$ (always fails, parsing nothing), 
respectively, for separator and terminator. The
function $f$ takes a pair of array representation and PD and compares
the sequence length recorded in the representation to
$\mathit{length}$.

Unbounded arrays are more difficult to encode as they must check the
next element for parse errors without consuming it from the data
stream. A termination predicate cannot encode this check as they
\trversion{are limited to the in-memory representation and PD and}
cannot perform lookahead. Therefore, we must use the terminator type
to look ahead for an element parse error. For this purpose, we
construct a type (abbreviated $\pnot \ty$) which succeeds where $\ty$
fails and fails where $\ty$ succeeds: 
\trversion{
\[
\pset x {\psum \ty {} \ptrue} {\icaseg {x.rep} {\_}
  {\ifalse} {\_} {\itrue}}
\]
\noindent
Abbreviated $\pnot \ty$, this type attempts to parse a $\ty$. On
success, the representation will be a left injection. The constraint
in the set type will therefore fail. If a $\ty$ cannot be parsed, the
sum will default to $\ptrue$, the rep will be a right injection, and
the constraint will succeed. The use of the sum in the underlying type
is critical as it allows the set-type to be error free even if parsing
a $\ty$ fails.

With $\pnotn$, we can encode the unbounded \datascript{} array as
follows:
\[
  \infer{
    \itmv \; [{length}] \cipads 
    \pseq \ty {\ptrue{}} {\pterm {\ilam x {} \ifalse} {\pnot \ty}}
  }{ 
    \itmv \cipads \ty
  }
\]
Note that the termination predicate is trivially false, as we use the
lookahead-terminator exclusively to terminate the array.  }
\poplversion{ 
\[
\pset x {\psum \ty {} \ptrue} {\icaseg {x.rep} {\_}
  {\ifalse} {\_} {\itrue}}
\]
\noindent
Unbounded arrays with element type $\ty$ can now be encoded as
sequences with terminator $\pnot \ty$.  }

While there are many more features that we can encode, space prevents us from
detailing them here. To give a sense of what is possible, we
briefly list those features of \datascript{} and \packettypes{} for
which we have found encodings in \ddc{}:
\begin{itemize}
\item \packettypes{}: arrays, where clauses, structures, overlays,
  and alternation.
\item \datascript{}: set types (enumerations and bitmask sets),
  arrays, constraints, value-parameterized types (which they refer to
  as ``type parameters''), and (monotonically increasing) labels.
\end{itemize}

We know of a couple of features from data description languages
that we cannot implement in \ddc{} as it stands.  An example is a label construct that permits the user to rewind the input. However, we do not view 
such limitations as particularly troublesome. Like the lambda 
calculus or pi calculus,
\ddc{} is intended to capture many common language features, while providing
a convenient and effective basis for extension with new features.

% Indeed, the two main features of \datascript{} that cannot currently
% be encoded in \ddc{}, require only straightforward modifications or
% additions.

%%% Local Variables: 
%%% mode: latex
%%% TeX-master: "semantics"
%%% End: 
