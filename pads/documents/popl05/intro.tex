
\section{The Challenge of Ad Hoc Data Formats}
\label{sec:intro}

\cut{
\begin{itemize}
\item difficulties of ad hoc data suggest need for DDLs.
  ad hoc data is buggy. need good error handling.
  e.g packettypes~\cite{sigcomm00}, datascript~\cite{gpce02}, 
  PADS~\cite{fisher+:pads}, BLT.
\item a number of ddls exist, but none have formal specification.
\item formal specification provides:
  \begin{itemize}
  \item Clear specification of language (for understanding)
  \item Validation of existing implementation - allows us to reason
    formally about the language and finds bugs in existing
    implementation.
  \item Supports fast creation of new language bindings, new language
    features
  \end{itemize}
\item Contributions:
  \begin{itemize}
  \item Calculus itself
    \begin{itemize}
    \item our language is compositional. Easy to
      reason about, add new features.
    \item formalizes error handling
    \end{itemize}    
  \item Mapping from IPADS to calc;sketches of others.
  \item Framework for designing and studying DDLs: What types of
    properties are needed by system (i.e. what to prove, how to
    structure); what's the right style of semantics.
  \item theorems and their proofs. (i.e. above we say that we figured
    out what to prove. this point is that we went ahead and proved it).
  \item new features inspired/guided by formalism: recursion,overlays.
  \end{itemize}
\end{itemize}
}

XML. HTML. CSV. JPEG. MPEG.  These data formats
represent vast quantities of industrial, governmental,
scientific, and private data.  Because they have been standardized
and are widely used, many reliable, efficient, and
convenient tools for processing data in these formats are
readily available.  For instance, your favorite programming language
undoubtedly has libraries for parsing XML and HTML as well as
reading and transforming images in JPEG or movies in MPEG.  Query engines
are available for asking questions about XML documents.
Widely-used applications like Microsoft Word and Excel automatically
translate documents back and forth between HTML and other
standard formats.  In short, life is good when working with standard data formats. In an ideal world, all data would be in such formats. In reality, however, we are not nearly so fortunate.

An {\em ad hoc data format} is any non-standard data format.  
Typically, such formats do not have parsing,
querying, analysis, or transformation tools readily available.
Every day, network administrators, financial analysts, computer
scientists, biologists, corporate IT professionals, chemists, astronomers, and
physicists deal with ad hoc data in a myriad of complex formats.
Figure~\ref{figure:data-sources} gives a partial sense of the range
and pervasiveness of such data.  Since off-the-shelf
tools for processing these ad hoc data formats do not exist
or are not readily available, talented scientists, data analysts, and
programmers must waste their time 
on low-level chores like parsing and format translation
to extract the valuable information they need from their data.
Though the syntax of everyday programming languages
might be considered ``ad hoc,'' we explicitly exclude
programming language syntax from our domain of interest.

In addition to the inconvenience of having to build custom
processing tools from scratch, the nonstandard nature of ad hoc data
frequently leads to a number of other difficulties for its users.
First, documentation for the format may not exist, or it may be out of
date.  For example, a common phenomenon is for a field in a data source to fall
into disuse.  After a while, a new piece of information becomes
interesting, but compatibility issues prevent data suppliers from
modifying the shape of their data, so instead they hijack the unused
field, often failing to update the documentation in the process.

Second, such data frequently contain errors, for a variety of reasons:
malfunctioning equipment, programming errors, non-standard values to
indicate ``no data available,'' human error in entering data, and
unexpected data values caused by the lack of good documentation.
Detecting errors is important, because otherwise they can corrupt
``good'' data.  The appropriate response to such errors depends on the
application. Some applications require the data to be error free: if
an error is detected, processing needs to stop immediately and a human
must be alerted.  Other applications can repair the data, while still
others can simply discard erroneous or unexpected values.  For some
applications, errors in the data can be the most interesting part
because they can signal where two systems are failing to communicate.

Today, many programmers tackle the challenge of ad hoc data by writing
scripts in a language like Perl.  Unfortunately, this process is slow,
tedious, and unreliable.  Error checking and recovery in these scripts
is often minimal or nonexistent because when present, such error code
swamps the main-line computation.  The program itself is often
unreadable by anyone other than the original authors (and usually not
even them in a month or two) and consequently cannot stand as
documentation for the format.  Processing code often ends up
intertwined with parsing code, making it difficult to reuse the
parsing code for different analyses. Hence, in general, software
produced in this way is not the high-quality, reliable, efficient and
maintainable code one should demand.

\begin{figure}
\begin{center}
\begin{tabular}{|l|l|}
\hline
Name \& Use   &  Representation               \\ \hline\hline
Web server logs (CLF):  &  Fixed-column ASCII records \\ 
Measure web workloads &                             \\ \hline
AT\&T provisioning data: & Variable-width ASCII records  \\ 
Monitor service activation &                              \\ \hline
Call detail: Fraud detection  &  Fixed-width binary records \\  \hline 
AT\&T billing data: & Various Cobol data formats  \\ 
Monitor billing process   &                             \\ \hline
IP backbone data:  & ASCII   \\
Monitor network performance  &        \\ \hline
Netflow:                        & Data-dependent number of     \\ 
Monitor network performance  & fixed-width binary records  \\ \hline
Newick:   Immune                 & Fixed-width ASCII records \\ 
system response simulation & in tree-shaped hierachy\\ \hline                                
Gene Ontology:             & Variable-width ASCII records \\
Gene-gene correlations     & in DAG-shaped hierarchy \\ \hline
HL7:             & Variable-width ASCII records \\
Medical lab results     &  \\ \hline
CPT codes: Medical diagnoses & Floating point numbers \\ \hline
SnowMed: Medical clinic notes & keyword tags  \\ \hline


\end{tabular}


\caption{Selected ad hoc data sources.}
\label{figure:data-sources}
\end{center}
\end{figure}
 
\subsection{Promising Solutions}

To address these challenges, 
researchers have begun to develop high-level languages 
for describing and processing ad hoc data.  For instance,
McCann and Chandra introduced
\packettypes{}~\cite{sigcomm00}, a specification language designed to help 
systems programmers process the binary data associated
with networking protocols.  Godmar Back developed
\datascript{}~\cite{gpce02}, a scripting language with explicit
support for specifying and parsing binary data formats. \datascript{}
has been used to manipulate Java jar files and ELF object files.  The
developers of Erlang have also introduced language extensions that
they refer to as {\em binaries}~\cite{erlang-bits} to aid in packet
processing and protocol programming.  At CMU, Eger is in the process
of developing a
language of Bit-Level Types~\cite{eger:blt} for specifying file
formats such as ELF, JPEG, and MIDI as well as packet layouts.
Finally, we are part of a group developing
\pads{}~\cite{fisher+:pads}, another system for specifying ad hoc data.
\pads{} focuses on robust error handling and tool generation.
It is also unusual in that it supports a variety of data encodings:
ASCII formats used by financial analysts, medical professionals and scientists,
EBCDIC formats used in Cobol-based legacy business systems,
binary data from network applications, and mixed encodings as well.

% Consequently, \pads{} can be used in applications 
% ranging from network packet processing to financial data analysis to
% medical record processing to web server log parsing and others.

% Here are several examples;
% there are likely more we are not familiar with.

% \begin{itemize}
% \item At SIGCOM,
% McCann and Chandra~\cite{sigcomm00} introduced
% \packettypes{}, a specification language aimed at helping systems
% researchers and developers parse and process binary data 
% associated with all different kinds of networking protocols.  
% \item The developers of Erlang,
% a widely used functional programming language in the 
% telecommunications industry, introduced 
% language extensions they refer to as {\em binaries}~\cite{erlang-bits}
% to aid in packet processing and protocol programming.
% \item Godmar Back~\cite{gpce02}
% developed \datascript{}, a scripting language with explicit support for 
% specifying and parsing binary data formats. \datascript has been 
% used to process Java jar files and ELF object files.  
% \item At CMU, Eger~\cite{eger:blt} has developed a language of Bit-Level Types
% once again for specifying file formats such as ELF, JPEG and
% MIDI as well as packet layouts.
% \item We ourselves are part of a group developing
% \pads{}~\cite{fisher+:pldi05}, yet another system for specifying ad hoc data.  Unlike the other systems mentioned above,
% \pads{} allows users to write specifications of both ASCII and
% binary data.  Consequently, \pads{} can be used in applications 
% ranging from network packet processing to financial data analysis to
% medical record processing to web server log parsing and others.
% \end{itemize}


\figHeight{revised-architecture}{Architecture of \pads{}  system.}{fig:architecture}{2.25in}

While differing in many details, 
these languages derive their power and utility from 
a remarkable insight: Types can simultaneously
serve as declarative specifications for ad hoc data formats 
{\em and} as classifiers for internal data representations 
that result from parsing these formats.
\figref{fig:architecture} illustrates how systems such as
\pads{}, \datascript, \packettypes{} and others exploit this
dual interpretation of types.  In the diagram,
the data consumer constructs a type {\tt T}
that describes the syntax and semantic properties of the format 
in question.  A compiler converts this
description into parsing code, which maps raw data into a canonical
in-memory {\em representation}.  
%\pads{} uses C as the host language, but in principle,
%any host language will do.  
The canonical in-memory representation is guaranteed to be a data structure
that has itself type {\tt T}, or perhaps {\tt T'}, the closest relative of
{\tt T} available in the host programming language
being used.
In the case of \pads{}, the parser also generates a {\em parse 
descriptor} (PD), which
describes the errors detected in the data.  
A host language program can then analyze, transform or
otherwise process the data representation and PD. 

This architecture helps programmers take on the
challenges of ad hoc data in multiple ways.
First, format specifications in these languages serve as high-level
documentation that is more easily read and
maintained than the equivalent low-level \perl{} script or C parser.
Importantly, \datascript{}, \packettypes, and \pads{} all
allow programmers to describe both the physical layout of data
as well as its deeper semantic properties such as equality and range 
constraints on values, sortedness, and other forms of dependency.
The intent is to allow analysts to capture all they know about
a data source in a data description.  If a data source changes,
as they frequently do, by extending a record with an additional field or new
variant, one often only needs to make a single local change to
the declarative description to keep it up to date.  

Second, basing the description language on type theory is especially helpful as
ordinary programmers have built up strong intuitions about types.  
The designers of data description languages
have been able to exploit these intuitions to make the syntax and
semantics of descriptions
particularly easy to understand, even for beginners.  For instance,
an array type is naturally used to describe sequences of data objects.
And really, what else could an array type describe?  Similarly,
union types are used to describe alternatives.

Third, programmers can write generic, type-directed programs that 
produce tools for purposes other than just parsing.  For instance,
McCann and Chandra used \packettypes{} specifications to generate
packet filters and network monitors automatically.
Back used \datascript{} to generate infrastructure for 
visitor patterns over parsed data. \pads{} generates
a statistical data analyzer, a pretty
printer, an \xml{} translator and an auxiliary library
that enables XQueries using the Galax query engine\cite{galax}.
It is the declarative, domain-specific nature of these data description
languages that makes it possible to generate all these value-added
tools for programmers.  The suite of tools, all of which can be generated
from a single description, provides additional incentive for
programmers to keep documentation up-to-date.

% For instance, since the type-based data descriptions this documentation is ``living'' in that the various systems
% convert the descriptions into parsing libraries (and in the case of
% \pads{}, a myriad of additional tools including an \xml{} translator,
% a statistical profiler, a pretty printer, and an auxiliary library
% that enables XQueries using the Galax query engine\cite{galax}).  
% If
% the format changes, analysts have strong incentive to update the
% documentation to generate a new parser and tool suite.

Fourth, these data description languages 
facilitate insertion of error handling code. 
The generated parsing code checks all possible
error cases: system errors related to the input file, buffer, or
socket; syntax errors related to deviations in the physical format;
and semantic errors in which the data violates user
constraints. Because these checks appear only in generated code, they
do not clutter the high-level declarative description of the data
source. Moreover, since tools are generated automatically by a
compiler rather than written by hand, they are far more likely to be
robust and far less likely to have dangerous vulnerabilities such as
buffer overflows. 
% Moreover, as formats evolve, one can
% make a small change in the high-level description and the compiler
% manages the rest. Consequently, it is unlikely that new
% vulnerabilities or buffer overflows will be added to the code base.

In summary, data description languages such as \datascript{},
\packettypes{}, Erlang, \blt{}, and \pads{} meet the challenge of
processing ad hoc data by providing a concise and precise form of
``living'' data documentation and producing reliable tools that handle
errors robustly.


\subsection{The Next 700 Data Description Languages}

\begin {quote}
The languages people use to communicate with computers differ in their intended aptitudes, towards either a
particular application area, or a particular phase of computer use (high level programming, program assembly,
job scheduling, etc). They also differ in physical appearance, and more important, in logical structure. The question arises, do the idiosyncracies reflect basic logical
properties of the situations that are being catered for?
Or are they accidents of history and personal background
that may be obscuring fruitful developments? This
question is clearly important if we are trying to predict or
influence language evolution.

To answer it we must think in terms, not of languages,
but of families of languages. That is to say we must
systematize their design so that a new language is a point
chosen from a well-mapped space, rather than a laboriously
devised construction.

$\qquad$ --- J. P. Landin, {\em The Next 700 Programming Languages}, 1965.
\end{quote}


Landin asserts that principled programming language design
involves thinking in terms of ``families of languages'' and
choosing from a ``well-mapped space.''  However, so far,
when it comes to the domain of processing ad hoc data, 
there is no well-mapped space and no systematic understanding
of the family of languages one might be dealing with.

The primary goal of this paper is to begin to understand the
family of ad hoc data processing languages.  We do so,
as Landin did, by developing a semantic
framework for defining, comparing, and contrasting languages
in our domain.  This semantic framework revolves around the
definition of a data description calculus (\ddc{}).  
This calculus uses types from a dependent type theory to describe
various forms of ad hoc data:
base types describe atomic pieces of data and
type constructors to describe richer structures.
We show how to give a denotational semantics
to \ddc{} by interpreting
types as parser functions that map external representations (bits)
to data structures in a typed lambda calculus.  More precisely,
these parsers produce both 
internal representations of the external data and
parse descriptors that pinpoint errors in the original source.

For many domains, researchers have a solid understanding of
what makes a ``good'' or ``bad'' language.  For instance,
a good typed language is one in which values of a given type
have a well-defined canonical form and ``programs don't go wrong.''
On the other hand, when we began this research, it
was not at all clear
how to decide whether our data description language and
its interpretation were ``good'' or ``bad.''  A conventional sort
of canonical forms property, for instance, 
is not relevant as the input data source
is not under system control, and, as
mentioned above, is frequently buggy.  Consequently,
we have had to define and formalize a new correctness criterion for the language. 
In a nutshell, rather than requiring input data be error-free, we require
that all errors in the parsed data be accurately recorded
in the parse descriptor.  We adopted this criterion because it ensures
that data consumers can rely on the integrity of data marked as error-free.

%There are other possible correctness criteria
%as well, but this is a solid starting point.
% eg: the existence of inverse mappings?

To study and compare \pads{}, \datascript{}, and/or
some other data description language, we advocate translating
the language into \ddc{}.  The translation breaks down
the relatively complex, high-level descriptions of the language 
in question into a series of lower-level
\ddc{} descriptions, which have all been formally defined.  
We have done this decomposition for \ipads{}, an idealized version of the
\pads{} language that captures the essence of the actual 
implementation.  We have also analyzed many of the features of
\packettypes{} and \datascript{} using our model.  The process
giving semantics to these languages
highlighted features that were ambiguous or 
ill-defined in the documentation we had a available to us.
% given a semantics to one of the features
% of \packettypes{}, its {\em overlays}, not found in \pads{}.
% Our semantic investigation of overlays uncovered the fact that
% they can be viewed as an syntactically 
% unorthodcx (but very useful) form of intersection type.
% Moreover, though we have not done so yet, it should be straightforward to
% add them to the \pads{} implementation by encoding them as
% ``\Palternate{}s,'' \pads{} own form of intersection type.

To our delight, the process of giving \pads{} 
a semantics in this framework has had several additional benefits.  
In particular, since we defined the
semantics by carefully reviewing the code base for the 
implementation, we found a couple of subtle bugs that we 
have fixed.  The semantics also raised several
design questions that we are continuing to study and understand. 
In addition to using the semantic specification to help us improve the
existing implementation, we also used it to explore important extensions.
In particular, driven by examples found in biological data 
sources~\cite{geneontology,newick}, we decided to add recursion to \pads{}.
We used our semantic framework to study the ramifications of this addition.

In summary, this paper makes the following theoretical and practical
contributions:
%
\begin{itemize}
\item We define a semantic framework for understanding and comparing data description languages such as \pads{},
\packettypes{}, \datascript{}, and \blt{}.
No one had previously given a formal semantics to any of these 
languages.

\item At the center of the framework is \ddc{},
a calculus of data descriptions based on dependent type theory.
We show how to give a denotational semantics
to \ddc{} by interpreting
types both as parsers and, more conventionally, 
as classifiers for parsed data. 

\item We define an important correctness criterion for our language
stating that all errors in the parsed data are reported in the parse 
descriptor.  We prove our \ddc{} parsers maintain this and several
other meta-theoretic properties.

\item We define \ipads{}, an idealized
version of the \pads{} programming language
that captures its essential features,
and show how to give it a semantics by translating it into \ddc{}.  
The process of defining the semantics led to the
discovery of several bugs in the actual implemention.
%\ipads{} can also be used to explore extensions to the existing \pads
%infrastructure.  

\item We have given semantics to features from several other data description
languages including \packettypes{} and \datascript.  As Landin asserts, 
this helps us understand the family of languages in this domain
and the totality of their features so we may engage in principled
language design
as opposed to falling prey to ``accidents of history and personal background.''

\item We use \ipads{} and \ddc{} to experiment with 
a definition and implementation strategy for recursive data types,
a feature not found in existing ad hoc data description language that
we are aware of.  Recursive types are essential for representing 
tree-shaped hierarchical
data used by biologists~\cite{geneontology,newick}.  
We have integrated recursion into the \pads{} implementation,
using our theory as a guide. 
\end{itemize}

Section~\ref{sec:ipads} gives a gentle introduction to data description
languages by introducing \ipads.  Sections~\ref{sec:ddc}, \ref{sec:ddc-sem} and \ref{sec:meta-theory}
explain the syntax and semantics and metatheory of \ddc{}.
Section~\ref{sec:encodings}  discuss encodings of \ipads{}, \packettypes
and \datascript{} in \ddc{} and Sections \ref{sec:applications}
explains how we have already made use of our semantics in practice.  
Sections~\ref{sec:related} and 
\ref{sec:conclusion} discuss related work and conclude.
