\section{The Semantics of PADS/ML}
\label{sec:ddc}

% \emph{Do we want to show an example from pads/ml shown in DDC.
%   (e.g. name-value pairs?). It think it would be nice, but I don't
%   know that there is space.  }

\trversion{\em Modify a\_pd to a\_pd\_body, everywhere. This name is more
  appropriate.  Note that the new scheme for translating type
  variables will affect WF rules. a will be in D but a\_rep and
  a\_pd\_body could appear in a sigma

  Consider a new convention: for compute type, any references to type
  interepretations should be done with the type interpretation
  functions rather than being hard coded. Otherwise, type substitution gets
  messed up,e.g., when unfolding a recursive type. 
  For example, should be compute(e:[a]\_rep) instead of
  compute(e:a\_rep). Then, substitution on DDC types will burrow into
  the type annotation $\gs$ of compute types. To support, need new
  syntax for type annotations $\gs$ and need to explicitly translate
  annotation types into F-omega types.
  
}

In this section, we introduce \ddc{}, a calculus of simple, orthogonal
type constructors, which serves to give a semantics
to the main features of \padsml.  \ddc{} is an extension
and revision of our previous work on 
\ddcold{}~\cite{fisher+:next700ddl}.  The main new feature is 
the ability to define functions from types to types, 
which are needed to model \padsml's polymorphic data types.
In the process of adding these new functions, which we call {\em type
abstractions} (as opposed to {\em value abstractions}, which are functions 
from values to types), we simplified our overall semantics
by making a couple of subtle technical changes.  For example, we were
able to eliminate the complicated ``contractiveness'' constraint from our
earlier work.  We have also added a new interpretation of \ddc{} types
as printers.

The main practical benefit of the calculus has been as a guide for our
implementation. Before working through the formal semantics, we
struggled to disentangle the invariants related to polymorphism. After
we had defined the calculus, we were able to implement type
abstractions as \ocaml{} functors in approximately a week.  
Our new printing semantics was also very important for helping us define
and check the correctness of our printer implementation.  We hope
the calculus will serve as a guide for implementations of {\sc PADS} in
other host languages.
In the remainder of this section, we give an overview of the calculus.
\appref{app:ddc-semantics} contains a complete formal specification.

\subsection{\ddc{} Syntax}
\begin{figure}
{\small
\begin{bnf}
  \name{Kinds} \meta{\gk} \::= \kty \| \kty \-> \gk
                               \| \ity \-> \gk  \\
  \name{Types} \meta{\ty} \::= 
    %\ptrue\| \pfalse \| 
    \pbase{e} \| 
    \plam{\var}{\ity}{\ty} \| \papp{\ty}{e} \| 
    \psig x \ty \ty \| \psum \ty e \ty \nlalt
    \pset x \ty e \|
    \ptyvar \| \pmu{\ptyvar}{\gk}{\ty} \| 
    \ptylam{\ptyvar}{\kty}{\ty} \| \ptyapp{\ty}{\ty} \| ...
\end{bnf}
}
\caption{\ddc{} syntax, selected constructs}
\label{fig:ddc-syntax}
\end{figure}

\figref{fig:ddc-syntax} summarizes the syntax of the \ddc.
The interpretation of a type with kind $\kty$ is a parser that maps
data from an external form into an internal one.  
A type with kind $\kty \rightarrow \gk$ is a function mapping 
a parser to the interpretation of a type with kind $\gk$.
Finally, types with kind $\ity \rightarrow \gk$  map values
with host language type $\ity$ to the interpretation of
types with kind $\gk$.  For concreteness, we adopt \fomega{} as our
host language.

%The atomic types include $\ptrue$, which consumes no input and reports
%no errors,  and $\pfalse$, which consumes no input but reports an
%error.   

The simplest description is a base type $\pbase{e}$.
The base type's parameter $e$ is drawn from the host language. 
The \padsml{} type {\tt Pstring} is an example of such a base type.
Structured types include value abstraction $\plam{\var}{\ity}{\ty}$
and application $\papp{\ty}{e}$, which allow us to parameterize types
by host language values. 
\cut{
 Any type in the language may be parameterized by a value using
 value abstraction $\plam{\var}{\ity}{\ty}$.  
 We do not include the type $\sigma$ of the variable---the reader will
 have to reconstruct it from context. 
 If $\ty$ is such an abstraction,
 the parameter $\var$ may be instantiated
 using value application $\papp{\ty}{e}$. 
}
The dependent sum type, $\psig x \ty \ty$, describes a pair of values,
where the value of the first element of the pair can be referenced
when describing the second element.  Variation in a data source can be
described with the sum type $\psum \ty e \ty$, which deterministically
describes a data source that either matches the first type, or fails
to match the first branch but does match the second one.
%Intersections $\pand \ty \ty$
%describe data sources which can be described in two ways
%simultaneously. 
We specify semantic constraints over a data source
with type $\pset x \ty e$, which describes any value $x$ that satisfies the
description $\ty$ and the predicate $e$. Type variables $\ptyvar$ are
abstract descriptions; they are introduced by recursive types and type
abstractions. Recursive types $\pmu{\ptyvar}{\gk}{\ty}$ describe
recursive formats, like lists and trees. Type abstraction
$\ptylam{\ptyvar}{\kty}{\ty}$ and application $\ptyapp{\ty}{\ty}$
allow us to parameterize types by other types.  Type variables $\ptyvar$
always have kind $\kty$.

To specify the well-formedness of types, 
we use a kinding judgment of the form
$\ddck[\ty,\pctxt;\ctxt,\kind,\mcon]$,
where $\pctxt$ maps type variables 
to kinds and $\ctxt$ maps host language value variables to host language 
types. In our original work~\cite{fisher+:next700ddl}, these kinding rules
were somewhat unorthodox, but we have sinced simplifed them.  
Details appear in \appref{app:ddc-semantics}.

\subsection{\Implang{} Language}
\label{sec:host-lang}
The host language of \ddc{} is a straightforward 
extension of \fomega{} with recursion and a variety of useful constants
and operators.  
% We use \fomega{} 
% both to write the expressions that can appear within
% \ddc{} itself and to encode the parsing semantics of \ddc{}.
For reference, the grammar appears in \appref{app:host-lang}.
The constants include bitstrings $\data$; offsets $\off$, representing
locations in bitstrings; and error codes $\iok$, $\iecerr$, and
$\iecpc$, indicating success, success with errors, and failure,
respectively. We use the constant $\ierr$ to indicate a failed parse.
Because of its specific meaning, we forbid its use in user-specified
expressions appearing in \ddc{} types.
Our base types include the type $\invty$, the singleton type of the
constant $\ierr$, and types $\iecty$ and $\ioffty$, which classify
error codes and bit string offsets, respectively.

We extend the formal syntax with some syntactic sugar for use in the
rest of this section: anonymous functions $\ilam {\nrm x} \ity e$ for
$\ifun {\nrm f} {\nrm x} e$, with $f \not\in {\rm FV}(e)$; $\ispty$
for $\iprod \ioffty \ioffty$.  We often use pattern-matching syntax
for pairs in place of explicit projections, as in $\lampair{\codefont
  e}$ and $\ilet {\itup{\off,r,p}} e\; e'$.  Although we have no
formal records with named fields, we use a dot notation for commonly
occuring projections. For example, for a pair $\mathtt x$ of rep and
PD, we use $\codefont{x.rep}$ and $\codefont{x.pd}$ for the left and
right projections of $\codefont{x}$, respectively. Also, sums and
products are right-associative.  Finally, we only specify type
abstraction over terms and application when we feel it will clarify
the presentation. Otherwise, the polymorphism is implicit.  We also
omit the usual type and kind annotations on $\lambda$, with the
expectation the reader can construct them from context.  

The static semantics ($\stsem[e,{\pctxt;\ctxt},\ity]$), operational
semantics ($e \stepsto e'$), and type 
equality ($\ity \equiv \ity'$) are those of \fomega{} extended with
recursive functions and recursive types and are entirely standard.
See Pierce's text~\cite{pierce:tapl} for details.

\subsection{\ddc{} Semantics}
\label{sec:ddc-sem}

The primitives of \ddc{} each have four interpretations: two
types in the host language, one for the data representation
itself and one for its parse descriptor, and two functions,
one for parsing and one for printing.
We therefore specify the semantics of \ddc{} types using four semantic
functions, each of which precisely conveys a particular facet of the
meaning of a type.  The functions $\itsem[\cdot]$ and $\itpdsem[\cdot]$
describe the type of the data's in-memory representation and 
parse descriptor, respectively. The semantic
functions $\trans[\cdot,,]$ and $\transpp[\cdot,,]$ define
the parsing and printing functions generated from \ddc{} descriptions.

% In our previous work~\cite{fisher+:next700ddl}, we discuss the
% semantics of \ddcold{} in detail. Here, we will focus only on those
% aspects of the semantics that are new or modified from the original
% presentation.
\trversion{
\begin{table}
  \begin{center}
    \renewcommand{\arraystretch}{1.35}
    \begin{tabular}{l l}
      $\ddck[\ty,{\pctxt;\ctxt},\kind,\mcon]$ & {\it \ddc{}-type
        kinding}\\
      $\itsem[\ty] = \ity$ & {\it representation types of \ddc{} types}\\
      $\itpdsem[\ty] = \ity$ & {\it pd types of \ddc{} types}\\
      $\trans[\ty,\ctxt,\gk] = e$   & {\it \ddc{}-type semantics} \\
      $\kTrans[\gk,\ty] = \ity$     & {\it parser type} \\
      $\ptyc \pctxt = \ctxt$     & {\it parser-type context }\\
      $\fotyc \pctxt = \pctxt$     & {\it \fomega version of poly. context }\\
      $\fortyc \pctxt = \pctxt$     & {\it Rep. type variables in $\fotyc \pctxt$ }\\
      $\fopdtyc \pctxt = \pctxt$     & {\it PD type variables in $\fotyc \pctxt$ }\\
      $\stsem[e,\pctxt;\ctxt,\ity]$ & {\it \fomega expression typing} \\
    \end{tabular}
    \caption{Translations and Judgments}
    \label{tab:judg-list}
  \end{center}
\end{table}

For reference, we provide in
\tblref{tab:judg-list} a listing of all the functions and judgments
defined in this section and a brief description of each.  
}

\begin{figure}
\fbox{$\itsem[\ty] = \ity$}
\[
\begin{array}{lcl} 
%\itsem[\ptrue] & = & \iunitty \\
%\itsem[\pfalse] & = & \invty \\
\itsem[\pbase{e}] & = & \isum {\Irty(C)} \invty   \\
\itsem[\plam{\var}{\ity}{\ty}] & = & \itsem[\ty] \\
\itsem[\papp \ty e] & = & \itsem[\ty] \\
\itsem[\psig \var {\ty_1} {\ty_2}]  & = & \iprod {\itsem[\ty_1]} {\itsem[\ty_2]}    \\
\itsem[\psum {\ty_1} e {\ty_2}]     & = & \isum {\itsem[\ty_1]} {\itsem[\ty_2]} \\
\itsem[\pset x \ty e] & = & \isum {\itsem[\ty]}{\itsem[\ty]}\\
\itsem[\ptyvar] & = & \ptyvar_\repname \\
\itsem[\pmu{\ptyvar}{\gk}{\ty}] & = & \imu{\ptyvar_\repname}{\itsem[\ty]} \\
\itsem[\lambda \ptyvar.\ty]       & = & \lambda \ptyvar_\repname.\itsem[\ty] \\
\itsem[\ty_1 \ty_2]              & = & \itsem[\ty_1] \itsem[\ty_2] \\
\end{array}
\]
\caption{Representation type translation, selected constructs}
\label{fig:rep-tys}
\end{figure}


\paragraph*{\ddc{} representation types.}
\label{sec:intty-sem}
In Figure~\ref{fig:rep-tys}, we present the representation type
of selected \ddc{} primitives. While the primitives are
dependent types, the mapping to the \implang{} language 
erases the dependency because the \implang{} language 
does not have dependent types. This involves erasing all host language
expressions that appear in types as well as 
value abstractions and applications.
A type variable $\ptyvar$ in \ddc{} is mapped to a corresponding
type variable $\ptyvar_\repname$ in \fomega{}.
Recursive types generate recursive representation types with the type
variable named appropriately. Polymorphic types and their application 
become \fomega{} type constructors and type application, respectively.

\begin{figure}
\fbox{$\itpdsem[\ty] = \ity$}
\[ 
\begin{array}{lcl} 
%% %% example: \ua.(int * a) + None
%% %%          pd = \ua.pd_hdr  * ((pd_hdr * ([int]_pd * [a]_pd)) + [None]_pd)
%% %%             = \ua.pd_hdr  * ((pd_hdr * ([int]_pd * a)) + [None]_pd)
%\itpdsem[\ptrue] & = & \ipty \iunitty \\                                                  
%\itpdsem[\pfalse] & = & \ipty \iunitty \\                                                  
\itpdsem[\pbase{e}] & = & \ipty \iunitty\\
\itpdsem[\plam \var \ity \ty] & = & \itpdsem[\ty] \\
\itpdsem[\papp \ty e] & = & \itpdsem[\ty] \\
\itpdsem[\psig \var {\ty_1} {\ty_2}] & = & 
               \ipty {\iprod {\itpdsem[\ty_1]} {\itpdsem[\ty_2]}} \\
\itpdsem[\psum {\ty_1} e {\ty_2}] & = & 
               \ipty {(\isum {\itpdsem[\ty_1]} {\itpdsem[\ty_2]})} \\
\itpdsem[\pset x \ty e] & = & \ipty {\itpdsem[\ty]} \\
\itpdsem[\ptyvar] & = & \ipty{\ptyvar_\pdbname} \\
\itpdsem[\pmu \ptyvar \kty \ty] & = & 
  \ipty{\imu{\ptyvar_\pdbname}{\itpdsem[\ty]}} \\
\itpdsem[\lambda \ptyvar.\ty]      
     & = & \lambda \ptyvar_\pdbname.\itpdsem[\ty] \\
\itpdsem[\ty_1 \ty_2]            & = & \itpdsem[\ty_1] \itpdsemstrip[\ty_2] \\
\end{array}
\]

\fbox{$\itpdsemstrip[\ty] = \ity$}

\[
\begin{array}{lcl} 
\itpdsemstrip[\ty] & = & \ity \ \ \mbox{where}\ \itpdsem[\ty] \equiv \ipty{\ity}
\end{array}
\]
\caption{Parse-descriptor type translation, selected constructs}
\label{fig:pd-tys}
\end{figure}

\paragraph*{\ddc{} parse descriptor types.}
\figref{fig:pd-tys} gives the types of the parse descriptors
corresponding to selected \ddc{} types.  The translation reveals that
all parse descriptors share a common structure, consisting of two
components, a header and a body.  The header reports on the
corresponding representation as a whole. It stores the number of
errors encountered during parsing, an error code indicating the degree
of success of the parse---success, success with errors, or
failure---and the span of data (location in the source)  
described by the descriptor.  To be precise, the type
of the header ($\tyface{pd\_hdr}$) is $\iintty \iprodi \iecty \iprodi
\ispty$. The body contains parse descriptors for the subcomponents of
the representation. For types without subcomponents, we use $\iunitty$
as the body type.  As with the representation types, dependency is
uniformaly erased.
% For types with base kind, the corresponding parse
% descriptor will always have a header and body. 

% In \figref{fig:pd-tys}, we give the types of the parse descriptors
% corresponding to each of the selected \ddc{} types. The majority of
% the types shown are products of the type $\tyface{pd\_hdr}$ and
% another host-language type. This common form reflects the fact that all
% parse descriptors consist of two components, a {\it header} and a {\it
%   body}.  The header describes the representation as a whole. It
% provides the number of errors encountered during parsing, an error
% code indicating the degree of success of the parse -- success, success
% with errors, or failure -- and the span of data described by the
% descriptor.  Formally, the type of the header ($\tyface{pd\_hdr}$) is
% $\iintty \iprodi \iecty \iprodi \ispty$. 

% The parse-descriptor body consists of parse descriptors for the
% subcomponents of the representation. For types without subcomponents,
% we use $\iunitty$ as the body type.

Like other types, \ddc{} type variables $\ptyvar$ are translated into 
a pair of header and a body.  The body has abstract type 
$\ptyvar_\pdbname$.
This translation makes it possible for polymorphic parsing code to examine the
header of a PD, even though it does not know the \ddc{} type it is parsing.
\ddc{} abstractions are translated into \fomega\ type constructors that
abstract the body of the PD (as opposed to the entire PD)
and \ddc{} applications are translated into \fomega\ type applications
where the argument type is the PD body type.

% PD-type constructors are
% parameterized over the PD \emph{body} type, rather than over the PD
% type itself. Correspondingly, type application $\papp {\ty_1}{\ty_2}$ applies
% the PD interpretation of $\ty_1$ to the body 

% This common shape enables us to write
% polymorphic functions that treat parse descriptors in a generic
% mannner. Indeed, the parsing semantics of types rely on this
% structure, as discussed in \secref{sec:parse-sem}.

% The PD types without the common shape are those corresponding to
% abstractions and applications. Value abstractions and applications are
% translated in an identical manner for PD types as for representation
% types.  Type abstractions and applications, however, are translated in
% a subtly different manner. As with the rep-type interpretation, type
% abstractions correspond to parse decriptor-type constructors in
% \fomega. The major difference is that PD-type constructors are
% parameterized over the PD \emph{body} type, rather than over the PD
% type itself. Correspondingly, type application $\papp {\ty_1}{\ty_2}$ applies
% the PD interpretation of $\ty_1$ to the body portion of the PD
% interpretation of $\ty2$ (specified formally with the $\itbdsem[\ty]$
% function). Similarly, type variables are translated to tuples of a
% header and a type variable, rather than to a type variable alone. 

% The reason for parameterization over PD-body types (rather than
% parse-descriptor types) in the translation scheme will be explained in
% the next section, once we have discussed the parsing semantics of type
% abstractions and applications. Note, however, that an important result
% of this design is that the PD interpretation is not defined for all
% types. The problem lies with the interpretation of type application.
% It requires that $\itbdsem[\ty_2]$ be defined, which, in turn,
% requires that $\itpdsem[\ty_2] \equiv \ipty{\ity}$, for some $\ity$.
% Yet, this requirement is not mebt by some types; for example, $\lambda
% \ptyvar.\ty$.


\begin{figure}
\small
\fbox{$\kTrans[\gk,\ty] = \ity$} 
    
\begin{align*}
  &\kTrans[\kty,\ty] & = & \quad \extdom * \offdom \iarrowi \offdom * \itsem[\ty] * \itpdsem[\ty]
   \\
   &\kTrans[\ity \iarrowi \gk,\ty] & = & \quad \ity \iarrowi \kTrans[\gk,\ty\ e],
   \; \mbox{for any e}.
   \\
   &\kTrans[\kty \iarrowi \gk,\ty] & = & \quad 
      \forall\tyvar_\repname.\forall\tyvar_\pdbname.
         \kTrans[\kty,\tyvar] \iarrowi \kTrans[\gk,\ty \tyvar] \\
         & & & \quad (\ptyvar_\repname, \ptyvar_\pdbname \not \in \ftv \kind \cup
         \ftv \ty)
\end{align*}  
  \caption{\Implang{} language types for parsing functions}
  \label{fig:parser-types}
\end{figure}

\begin{figure}
\small
\fbox{$\trans[\ty,\ctxt,\gk] = e$} 

\[
\begin{array}{l}
  %% None 
%\trans[\ptrue,\ctxt,\kty] =
%  \lampair{\spair<\off,\newrep{unit}{},\newpd{unit}{\off}>}
%\\[3pt] %\\
%% False 
%\trans[\pfalse,,] =
%  \lampair{\spair<\off,\newrep {bottom}{},\newpd {bottom}{\off}>}
%\\[3pt] %\\ 
%% Const 
\trans[\pbase{e},\ctxt,\kty] =
  \lampair{\iapp {\iapp {\Iimp(C)} (e)} {\itup {\idata,\off}}}
\\[3pt] %\\
%% Abs 
\trans[\plam{\var}{\ity}{\ty},,] =
   \sfn{\nrm\var}{\ity}{\trans[\ty,\ectxt{\var{:}\ity},\kind]}
\\[3pt] %\\
%% App 
\trans[\papp{\ty}{e},\ctxt,\gk] =
  \trans[\ty,,] \sapp e  
\\[3pt]
%% Prod 
%\begin{array}{l}
\trans[\psig{x}{\ty}{\ty'},\ctxt,\kty] = \\
  \begin{array}{l}  
    \lampair{} \\
    \quad  \ilet {\spair<\off',r,p>} 
    {{\trans[\ty,,]} \sapp \spair<\idata,\off>} \\
    \quad  \ilet x {\ictup{r,p}}\\
    \quad  \ilet {\spair<\off'',r',p'>} 
    {{\trans[\ty',,]} \sapp \spair<\idata,\off'>} \\
    \quad \spair<\off'',\newrep {\gS}{r,r'},\newpd {\gS}{p,p'}>
  \end{array}  
%\end{array}
\\[3pt]
%% Sum 
%\begin{array}{l}
  \trans[\psum{\ty}{e}{\ty'},,] = \\
  \begin{array}{l}  
  \lampair{} \\
  \quad \ilet {\itup{\off',r,p}}{\trans[\ty,,] \sapp \spair<\idata,\off>} \\
  \quad \iif {\pdok p} \; \ithen {
    \def \r {\newrep {+left}{r}}
    \def \p {\newpd {+left}{p}}
    \spair<\off',\r,\p>} \\
  \quad \ielse {\ilet {\itup{\off',r,p}}{\trans[\ty',,] \sapp \spair<\idata,\off>}} \\
  \quad 
  {  % begin scope
    \def \r {\newrep {+right}{r}}
    \def \p {\newpd {+right}{p}}
    %% 
    \spair<\off',\r,\p>
  }\\ % end scope
  \end{array}
\\[3pt]
%\quad
%% Set 
  \trans[\pset{x}{\ty}{e},\ctxt,\kty] = \\
  \begin{array}{l}  
    \lampair{} \\
    \quad \ilet {\itup{\off',r,p}}{\trans[\ty,,] \sapp \spair<\idata,\off>} \\
    \quad \ilet x {\ictup{r,p}}\\
    \quad \ilet c e \\
    \quad \spair<\off',\newrep {con} {c,r},\newpd {con} {c,p}>
  \end{array}
\\[3pt]
%% Var
\trans[\ptyvar,,] = \codefont{\parsename_\ptyvar}
\\[3pt]
%% Mu
\trans[\pmu \ptyvar \gk \ty,,] = \\
  \begin{array}{l}
  \ifun {\parsename_\ptyvar} {\itup{\data{:}\ibitsty,\off{:}\ioffty}  
           : \ioffty * \itsem[\pmu \ptyvar \gk \ty] 
                    * \itpdsem[\pmu \ptyvar \gk \ty] } {}\\
  \quad \ilet {\itup{\off',r,p}} 
   {\trans[\ty,,][\itsem[\pmu \ptyvar \gk \ty]/\ptyvar_\repname]
          [\itpdsemstrip[\pmu \ptyvar \gk \ty]/\ptyvar_\pdbname] \iappi \ictup{\data,\off}} \\ 
  \qquad \ictup{\off',\iroll{r}{\itsem[\pmu \ptyvar \gk \ty]},
     (p.h,\iroll{p}{\itpdsemstrip[\pmu \ptyvar\gk \ty]})}
%}}
  \end{array}  
\\[3pt]
%% lambda \alpha
\trans[\lambda\tyvar . \ty,,] = %\\
%  \begin{array}{l}
    \Lambda \tyvar_\repname. 
    \Lambda \tyvar_\pdbname. \lambda \codefont{\parsename_\ptyvar}. \trans[\ty,,]
%  \end{array}  
\\[3pt]
%% t1 t2
\trans[\ty_1 \ty_2,,] = 
    \trans[\ty_1,,]\; [\itsem[\ty_2]]\; [\itpdsemstrip[\ty_2]]\; \trans[\ty_2,,]
\\
\end{array}
\]
%\caption{\ddc{} Semantics (cont.)}
\caption{\ddc{} parsing semantics, selected constructs}
\label{fig:ddc-sem-sum}
\end{figure}


\paragraph*{\ddc{} parsing semantics.}
\label{sec:parse-sem}
The parsing semantics of a type $\tau$ with kind $\kty$ is a function that
transforms some amount of input into a pair of a representation and a
parse descriptor, the types of which are determined by $\tau$.  The
parsing semantics for types with higher kind are functions that
construct parsers, or functions that construct functions that
construct parsers, \etc{} \figref{fig:parser-types} specifies
the host-language types of the functions generated from well-kinded
\ddc{} types.

For each (unparameterized) type, the input to the corresponding parser
is a bit string to parse and an offset at which to begin parsing.  
The output is a new offset,
a representation of the parsed data, and a parse descriptor.

For any type, there are three steps to parsing: parse the
subcomponents of the type (if any), assemble the resultant
representation, and tabulate meta-data based on subcomponent meta-data
(if any). For the sake of clarity, we have factored the latter two
steps into separate representation and PD constructor functions which
we define for each type. 
%For some types, we additionally factor the PD
%header construction into a separate function. 
For example, the
representation and PD constructors for the dependent sums are
$\newrepf {\gS}$ and $\newpdf {\gS}$, 
respectively
%, and the header constructor for
%products is ${\codefont{H_{\gS}}}$.  
We have also factored out some commonly
occuring code into auxiliary functions.  These constructors and functions
appear in \appref{app:asst-functions}.

The PD constructors determine the error code and calculate the error
count.  There are three possible error codes: $\iok$, $\iecerr$, and
$\iecpc$, corresponding to the three possible results of a parse: it
can succeed, parsing the data without errors; it can succeed, but
discover errors in the process; or, it can find an unrecoverable error
and fail.  The error count is determined by subcomponent error counts
and any errors associated directly with the type itself.

In \figref{fig:ddc-sem-sum}, we specify the parsing semantics of
selected portion of \ddc{}. We explain the interpretations of select
types, from which the interpretation of the remaining types may be
understood. The full semantics appears in \appref{app:ddc-semantics}.
A dependent sum parses the data according to the first type, binding
the resulting representation and PD to $x$ before parsing the
remaining data according to the second type. It then bundles the
results using the dependent sum constructor functions.

A type variable translates to an expression variable whose name
corresponds directly to the name of the type variable. These
expression variables are bound in the interpretations of recursive
types and type abstractions. We interpret each recursive type as a
recursive function whose name corresponds to the name of the recursive
type variable. For clarity, we annotate the recursive function with its
type.

We interpret type abstraction as a function over other parsing
functions. Because those parsing functions can have
arbitrary \ddc{} types (of kind $\kty$), the interpretation must be a
polymorphic function, parameterized by the representation and PD-body
type of the \ddc{} type parameter.  For clarity, we present
this type parameterization explicitly.  Type application $\papp
{\ty_1}{\ty_2}$ simply becomes the application of the interpretation
of $\ty_1$ to the representation-type, PD-type, and parsing
interpretations of $\ty_2$.

% Now, we can explain why the PD and parsing interpretations
% of type abstraction and application involve the PD-body interpretation
% rather than just the PD interpretation. Notice that the constructor
% functions for dependent sums (and for many other types as well) are
% polymorphic. No matter the types of the subcomponent types $\ty$ and
% $\ty'$, we use the same functions $\newrepf \Sigma$ and $\newpdf
% \Sigma$. The PD constructor is not fully polymorphic, though, in that
% it assumes that the parse descriptors given to it have the familiar PD
% shape of header and body.  Therefore, in order to judge the type of
% this function, we must be sure that the PD types of the subcomponents
% $\ty$ and $\ty'$ satisfy this constraint.

% What happens, then, when we have $\ptyvar$ as a subcomponent? We must
% ensure that its PD type has the appropriate shape. To do so, we
% specify the translation of type variables explicitly as a product of a
% header and a type variable for the body, rather than simply as a type
% variable for the whole PD type. Therefore, all parameterization
% related to PD types must be over the PD-body type, rather than the PD
% type as a whole.

\begin{figure}
\small
\fbox{$\kTransPP[\gk,\ty] = \ity$} 
    
\begin{align*}
  &\kTransPP[\kty,\ty] & = & \quad \itsem[\ty] * \itpdsem[\ty] \iarrowi \extdom 
   \\
   &\kTransPP[\ity \iarrowi \gk,\ty] & = & \quad \ity \iarrowi \kTransPP[\gk,\ty\ e],
   \; \mbox{for any e}.
   \\
   &\kTransPP[\kty \iarrowi \gk,\ty] & = & \quad 
      \forall\tyvar_\repname.\forall\tyvar_\pdbname.
         \kTransPP[\kty,\tyvar] \iarrowi \kTransPP[\gk,\ty \tyvar] \\
         & & & \quad (\ptyvar_\repname, \ptyvar_\pdbname \not \in \ftv \kind \cup
         \ftv \ty)
\end{align*}  
  \caption{\Implang{} language types for printing functions}
  \label{fig:printer-types}
\end{figure}

\begin{figure}
\small
\fbox{$\transpp[\ty,\ctxt,\gk] = e$} 

\[
\begin{array}{l}
%% Const 
\transpp[\pbase{e},\ctxt,\kty] =
  \lampppair{\iapp {\iapp {\Igen{pp}(C)} (e)} {\ictup {r,pd}}}
\\[3pt] %\\
%% Abs 
\transpp[\plam{\var}{\ity}{\ty},,] =
   \sfn{\nrm\var}{\ity}{\transpp[\ty,\ectxt{\var{:}\ity},\kind]}
\\[3pt] %\\
%% App 
\transpp[\papp{\ty}{e},\ctxt,\gk] =
  \transpp[\ty,,] \sapp e  
\\[3pt]
%% Prod 
%\begin{array}{l}
\transpp[\psig{x}{\ty_1}{\ty_2},\ctxt,\kty] = \\
  \begin{array}{l}  
    \lampppair{} \\
      \quad  \ilet x {\ictup{\ipid{1}{r},\ipid{1}{\ipid{2}{pd}}}}\\
      \quad  \ilet {bs_1} 
%%    \def \rf {foo} %% {\ipid{1}{r}}
%%   \def \pf {bar} %% {\ipid{1}{\ipid{2}{pd}}}
%%  \def \rs {baz} %% {\ipid{2}{r}}
%% \def \ps {foo} %% {\ipid{2}{\ipid{2}{pd}}}
      {{\transpp[\ty_1,,]} \sapp \codefont{x}} \\
      \quad  \ilet {bs_2} 
      {{\transpp[\ty_2,,]} \sapp \spair<\ipid{2}{r},\ipid{2}{\ipid{2}{pd}}>} \\
      \quad \iappend{bs_1}{bs_2}
  \end{array}  
%\end{array}
\\
%% Sum 
%\begin{array}{l}
  \transpp[\psum{\ty_1}{e}{\ty_2},,] = \\
  \begin{array}{l}  
  \lampppair{} \\
  \quad \icasess {\ictup{r,\ipid{2}{pd}}} \\
  \quad \ipattss{\ictup{\iinl{r_1}, \iinl{p_1}}} 
	{ {\transpp[\ty_1,,]} \sapp \codefont{\ictup{r_1,p_1}}}\\
  \quad \ipattss{\ictup{\iinr{r_2}, \iinr{p_2}}} 
	{ {\transpp[\ty_2,,]} \sapp \codefont{\ictup{r_2,p_2}}}\\
  \quad \ipattss{\_}  {\failpp}\\
  \end{array}
\\
%\quad
%% Set 
  \transpp[\pset{x}{\ty}{e},\ctxt,\kty] = \\
  \begin{array}{l}  
  \lampppair{} \\
  \quad \icasess {\ictup{r,\ipid{2}{pd}}} \\
  \quad \ipattss{\ictup{\iinl{r_1}, p_1}} 
	{ {\transpp[\ty,,]} \sapp \codefont{\ictup{r_1,p_1}}}\\
  \quad \ipattss{\ictup{\iinr{r_2}, p_2}} 
	{ {\transpp[\ty,,]} \sapp \codefont{\ictup{r_2,p_2}}}\\
  \end{array}
\\[3pt]
%% Var
\transpp[\ptyvar,,] = \codefont{\printname_\ptyvar}
\\[3pt]
%% Mu
\transpp[\pmu \ptyvar \gk \ty,,] = \\
  \begin{array}{l}
  \ifun {\printname_\ptyvar} {\itup{r:\itsem[\pmu \ptyvar \gk \ty], pd:\itpdsem[\pmu \ptyvar \kty \ty]}
                     : \ibitsty} {}\\
  \quad \transpp[\ty,,][\itsem[\pmu \ptyvar \gk \ty]/\ptyvar_\repname]
          [\itpdsemstrip[\pmu \ptyvar \gk \ty]/\ptyvar_\pdbname] \\
  \qquad\ictup{\iunroll{r}{\itsem[\pmu \ptyvar \gk \ty]},
     \iunroll{\ipid{2}{pd}}{\itpdsemstrip[\pmu \ptyvar \kty \ty]}}
  \end{array}  
\\[3pt]
%% lambda \alpha
\transpp[\lambda\tyvar . \ty,,] = %\\
%  \begin{array}{l}
    \Lambda \tyvar_\repname. 
    \Lambda \tyvar_\pdbname. \lambda \codefont{\printname_\ptyvar}. \transpp[\ty,,]
%  \end{array}  
\\
%% t1 t2
\transpp[\ty_1 \ty_2,,] = 
    \transpp[\ty_1,,]\; [\itsem[\ty_2]]\; [\itpdsemstrip[\ty_2]]\; \transpp[\ty_2,,]
\\
\end{array}
\]
%\caption{\ddc{} Printing semantics (cont.)}
\caption{\ddc{} printing semantics, selected constructs}
\label{fig:ddc-print-sem}
\end{figure}


\paragraph*{\ddc{} printing semantics}
\label{sec:print-sem}
The definition of the printing semantics for a \ddc{} description
uses a similar set of concepts as the parsing semantics.  To begin,
the semantic function $\kTransPP[\gk,\ty] = \ity$ gives the
host language type $\ity$ for the printer generated from type $\ty$
with kind $\gk$.  As shown in \figref{fig:parser-types},
the printing semantics for descriptions with higher kind are functions that
construct printers and the printing semantics for descriptions with base kind
are simple first-order functions that map a representation and a
parse descriptor into a string of bits.

\figref{fig:ddc-print-sem} presents the printing semantics of
selected \ddc{} constructs.  Base types $\pbase{e}$ are printed in
various different ways according to the definition
$\Igen{pp}$, which is a parameter to the semantics.  The base type
printer $\Igen{pp}$ accepts the parse descriptor as a parameter, and
in the case of an error, a base type printer will print nothing.
Dependent sums print one component and then the next in order.\footnote{The
notation $\iappend{bs_1}{bs_2}$ appends bit string $bs_1$ to $bs_2$.}
Printing an ordinary sum type occurs by printing the underlying
tagged value.  Notice that the structure of the parse descriptor
and the representation should be isomorphic -- both should be
left injections or both should be right injections.  Any pair
of structures generated from the parser will satisfy this invariant.
If the pair of parse descriptor and representation do not match then the
programmer is using the printer has done so incorrectly.  In this case,
the printer calls an unspecified error routine named $\failpp{}$.

The semantics of printing recursive and parameterized types follows
similar lines to the semantics for parsing these constructs.
In particular, whenever a type parameter is introduced in the syntax
of a description a corresponding value parameter with printer function type 
is introduced in the generated printer code. We give the value parameter
the name $\printname_\alpha$.  Both type abstractions and recursive functions
introduce such parameters.  Notice that whereas the parsing interpretation
of a description used a fold to build a recursive data structure
when interpreting a recursive type, the printing interpretation uses
unfold to deconstruct a recursive data structure for printing.  

\subsection{Meta-theory}
\label{sec:meta-theory}
In order to further justify our semantic definitions,
we have proven two key metatheoretic results.  First, we show that
parsers and printers are {\em type-correct}, always returning representations
and parse descriptors of the appropriate type.  Second, we give a precise
characterization of the results of parsers
(and input requirements of printers) by defining the {\em canonical forms}
of representation-parse descriptor pairs associated with a
dependent \ddc{} type.

% Our key theoretical result is that the various semantic functions we
% have defined are coherent.  In particular, we show that for any 
% well-kinded \ddc{} type $\tau$, the corresponding parser is
% well typed, returning a pair of the corresponding representation and
% parse descriptor.  

\paragraph*{Type Correctness.}
Demonstrating that generated parsers and printers are well formed
and have the expected types is nontrivial primarily because
the generated code expects parse descriptors to have a particular shape,
and it is not completely obvious they do in the presence of polymorphism.
Hence, to prove type correctness, we first need to characterize the shape of
parse descriptors for arbitrary \ddc{} types.   
Unfortunately, the most straightforward characterization is
too weak to prove directly, and hence Definition~\ref{def:pd-props}
specifies a much stronger property as a logical relation.
Lemma~\ref{lemma:pd-log-rel} establishes that the logical relation
holds of all well-formed \ddc{} types by induction on kinding
derivations, and the desired characterization follows as a corollary.

\begin{definition}
\label{def:pd-props}
\begin{itemize}
\item $\hhpred \ty \kty$ iff $\ \exists\,\ity$ s.t. $\itpdsem[\ty] \equiv
  {\ipty \ity}$.
\item $\hhpred \ty {\kty \iarrowi \kind}$ iff $\ \exists\,\ity$
  s.t. $\itpdsem[\ty] \equiv \ity$ and whenever $\hhpred
  {\ty'}{\kty}$, we have $\hhpred {\papp \ty {\ty'}}{\kind}$.
\item $\hhpred \ty {\ity \iarrowi \kind}$ iff $\ \exists\,\ity'$
  s.t. $\itpdsem[\ty] \equiv \ity'$ and $\hhpred{\papp \ty e}{\kind}$
  for any expression $e$.
\end{itemize}
\end{definition}
\cut{
We can now prove the following lemma by induction on kinding derivations:
}
\begin{lemma}
\label{lemma:pd-log-rel}
If $\ddck[\ty,{\pctxt;\ctxt},\kind,{}]$ then $\hhpred \ty \kind$.
\end{lemma}

\begin{lemma}
\label{lemma:pd-props}
  \begin{itemize}
  \item If $\ddck[\ty,\pctxt;\ctxt,\kind,{}]$ then $\exists
     \ity.\itpdsem[\ty] = \ity$.
   \item If $\ddck[\ty,\pctxt;\ctxt,\kty,{}]$ then $\exists
     \ity.\itpdsem[\ty] \equiv \ipty \ity$.
  \end{itemize}
\end{lemma}


\trversion{
\begin{definition}
$\pda \ptyvar = \ipty \ptyvar$
\end{definition}

\begin{lemma}[Types of Constructors]
\label{lem:types-of-constructors}
\begin{itemize}
\item $\newrepf {unit} : \iarrow \iunitty \iunitty$
\item $\newpdf  {unit} : \iarrow \ioffty {\ipty \iunitty}$
\item $\newrepf {bottom} : \iarrow \iunitty \invty$
\item $\newpdf  {bottom} : \iarrow \ioffty {\ipty \iunitty}$
\item $\newrepf {\gS} : \forall \ga,\gb.\iarrow {\iprod \ga \gb} {\iprod \ga \gb}$
\item $\newpdf {\gS} : \forall \ga,\gb. 
  \iarrow {\iprod {\pda \ga} {\pda \gb}}
  {\pda {(\pda \ga \iprodi \pda \gb)}}
$
\item $\newrepf {+left} : \forall \ga.\forall \gb.\iarrow \ga 
                            {\isum \ga \gb}$
\item $\newrepf {+right} : \forall \ga.\forall \gb.\iarrow \gb {\isum \ga \gb}$
\item $\newpdf {+left} : \forall \ga, \gb.\iarrow {\pda \ga} 
  {\ipty {(\isum {\pda \ga}{\pda \gb})}}$
\item $\newpdf {+right} :\forall  \ga, \gb. \iarrow {\pda \gb} 
                            {\ipty {(\isum {\pda \ga} {\pda \gb})}}$
\item $\newrepf {\&} : \forall \ga,\gb.\iarrow {\iprod \ga \gb} {\iprod \ga \gb}$
\item $\newpdf {\&} : 
\forall \ga,\gb.
  \pda \ga \iprodi
  \pda \gb \iarrowi 
         {\ipty {(\pda \ga \iprodi \pda \gb)}}
$.
\item $\newrepf {con} : \forall \ga.\iprod \iboolty \ga 
  \iarrowi {\isum \ga \ga}$
\item $\newpdf {con} :\forall  \ga. \iprod \iboolty \iarrow {\pda \ga} {\ipty {\pda \ga}}$
\item $\newrepf {seq\_init} : \forall \ga.\iarrow \iunitty {\iintty \iprodi \iseq \ga}$
\item $\newpdf {seq\_init} : \forall \ga. \iarrow \ioffty {\iapty {\pda\ga}}$
\item $\newrepf {seq} : \forall \ga.\iarrow
  {(\iintty \iprodi \iseq \ga) \iprodi \ga}
  {\iintty \iprodi \iseq \ga}$
\item $\newpdf {seq} :\forall  {\ga_{elt}},{\ga_{sep}}. 
  (\iapty {\pda {\ga_{elt}}}) \iprodi
  \pda {\ga_{sep}} \iprodi 
  \pda {\ga_{elt}} \iarrowi \\
  \iapty {\pda {\ga_{elt}}}$
\item $\newrepf {compute} : \forall \ga.\iarrow \ga \ga$
\item $\newpdf {compute} : \iarrow \ioffty {\ipty \iunitty}$
\item $\newrepf {absorb} : \forall \ga.\iarrow {\pda \ga} {\isum
    \iunitty \invty}$
\item $\newpdf {absorb} :\forall  \ga. \iarrow {\pda \ga} {\ipty
    \iunitty}$
\item $\newrepf {scan} : \forall \ga.\iarrow \ga {\isum \ga \invty}$
\item $\newpdf {scan} :\forall  \ga. \iarrow {\iprod \iintty {\pda \ga}}
  {\ipty {(\isum {\iprod \iintty {\pda \ga}} \iunitty)}}$
\item $\newrepf {scan\_err} : \forall \ga.\iarrow \iunitty {\isum \ga \invty}$
\item $\newpdf {scan\_err} :\forall  \ga. \iarrow \ioffty
  {\ipty {(\isum {\iprod \iintty \ga} \iunitty)}}$
\end{itemize}  
\end{lemma}

\begin{proof}
  By typing rules of \fomega.
\end{proof}
}
With this lemma, we can establish the type correctness of the
generated parsers. We prove the theorem using a more general induction
hypothesis that applies to open types.
This hypothesis must account for the fact
that any free type variables in a \ddc{} 
type $\ty$ will become free
function variables in $\trans[\ty,,]$. To that end, 
we define the functions $\ptyc \pctxt$ and $\pptyc \pctxt$ 
which map type-variable contexts $\pctxt$ in the \ddc{}
to value-variable contexts $\ctxt$ in \fomega.  In addition, the function
$\fotyc{\pctxt}$ generates the appropriate \fomega\ type-variable context from
the \ddc{} context $\pctxt$.
\vskip -1.5ex
{\small
\[
\begin{array}{ll}
  \fotyc \cdot = \cdot \qquad &
  \fotyc {\pctxt,\alpha{:}\kty} = 
    \fotyc {\pctxt},\tyvar_\repname{:}\kty,\tyvar_\pdbname{:}\kty \\
  \ptyc{\cdot} = \cdot \qquad &
  \ptyc{\pctxt,\ptyvar{:}\kty} = \ptyc \pctxt,\codefont{\parsename_\ptyvar}{:}\kTrans[\kty,\ptyvar] \\
  \pptyc{\cdot} = \cdot \qquad &
  \pptyc{\pctxt,\ptyvar{:}\kty} = \pptyc \pctxt,\codefont{\printname_\ptyvar}{:}\kTransPP[\kty,\ptyvar]
\end{array}
\]
}

\begin{lemma}[Type Correctness Lemma]
\label{thm:type-correctness}
\begin{itemize}
\item If $\ddck[\ty,{\pctxt;\ctxt},\gk,{}]$ then
  $\stsem[{\trans[\ty,,]},{\fotyc \pctxt, \ctxt,\ptyc \pctxt},
            {\kTrans[\kind,\ty]}]$
\item If $\ddck[\ty,{\pctxt;\ctxt},\gk,{}]$ then
  $\stsem[{\transpp[\ty,,]},{\fotyc \pctxt, \ctxt,\pptyc \pctxt},
            {\kTransPP[\kind,\ty]}]$.
\end{itemize}  
\end{lemma}

\begin{proof}
  By induction on the height of the kinding derivation.
\end{proof}

\begin{theorem}[Type Correctness of Closed Types]
  \begin{itemize}
  \item If $\ddck[\ty,,\gk,\con]$ then
    $\stsem[{\trans[\ty,,]},,\kTrans[\kind,\ty]]$.
  \item If $\ddck[\ty,,\gk,\con]$ then
    $\stsem[{\transpp[\ty,,]},,\kTransPP[\kind,\ty]]$.
  \end{itemize}  
\end{theorem}

A practical implication of this theorem is that
it is sufficient to check data descriptions (i.e. \ddc{} types) for
well-formedness to ensure that the generated types and
functions are well formed. This property is sorely lacking in many
common implementations of Lex and YACC, for which users must examine
generated code to debug compile-time errors in
specifications.

\paragraph*{Canonical Forms for Parsed Data.}
\ddc{} parsers generate pairs of representations and parse descriptors
designed to satisfy a number of invariants.  Of greatest importance is 
the fact that
when the parse descriptor says there are no errors in a particular
substructure, the programmer can count on the representation
satisfying all of the syntactic and semantic
constraints expressed by the dependent
\ddc{} type description.  When parse descriptor and representation
satisfy these invariants, we say the pair
of datastructures is in {\em canonical form}.
While generated parsers produce canonical outputs, generated printers
expect canonical inputs.  

% In addition to proving our semantic functions coherent, we identify
% the canonical forms of representation parse-descriptor pairs
% produced by parsers and consumed by printers. Our canonical forms
% lemma goes beyond standard canonical forms lemmas by specifying
% essential correlation between the representation and parse
% descriptor. For example, in the canonical pair for sum types, the rep
% and PD are both injected in the same branch of the sum. Moreover, we
% specify correlations between the number of errors reported by the
% parse descriptor and the number of errors contained in the representation.

% \edcom{Should we specify the delta from last POPL paper?}  

% \edcom{Perhaps it would be good to put norm. rules back, as they
%   abstract over parsing and printing? Currently, defined in terms of
%   parser image.}

For each \ddc{} type, its canonical forms are defined via
two (mutually recursive) relations.  The first
relation, $\corr \tyval r p$,
defines the canonical form of a representation $r$ and a parse
descriptor $p$ at normal type $\tyval$.  {\em Normal types}
are those closed types with base kind $\kty$ that are defined in 
Figure~\ref{fig:ddc-reduction-rules}.  Types with higher kind such as
abstractions are not described by this relation as they cannot directly
produce representations and PDs.  
The second definition, $\corrkl \ty r p$,
normalizes $\ty$, thereby eliminating outermost type and value
applications.  The result is a normal type $\tyval$ and the
requirements on $\tyval$ are subsequently given by $\corr {\tyval} r
p$.  For brevity in these definitions, we write $p.h.{nerr}$ as
$p.{nerr}$ and use $\mathtt{pos}$ to denote the function that returns
zero when passed zero and one when passed another natural number.  The
following definition gives the notion of \fomega\ expression
equivalence we use.

% However, as \ddc{} lacks evaluation rules, we derive the weak-head
% normal form of a type by deriving the weak-head normal form of the
% corresponding parser. As a technical detail, we note that this normal
% form will not necessarily be a parser for a weak-head normal \ddc{}
% type, but might only be equivalent to one. Therefore, we begin our
% definitions with expression equivalence.

% \begin{definition}[\fomega\ Expression Equivalence]
%   $\iexp \iexpreq \iexp'$ iff $\iexp$ is syntactically equal to $\iexp'$ 
% modulo alpha-conversion of bound variables and equivalence of
% typing annotations.
% \label{def:op-eq}
% \end{definition}

\begin{figure}
\small
\begin{bnf}
%   \name{Kinds} \meta{\gk} \::= \kty \| \ity \-> \gk 
%                                \pext{\| \gk \-> \gk} \\
  \name{Normal Types} \meta{\tyval} \::= 
%    \ptrue\| \pfalse \| 
    \pbase{e} \| \plam{\var}{\ity}{\ty} \|
    \psig x \ty \ty  \|
    \psum \ty e \ty  \nlalt
 % \pand \ty \ty \|
    \pset x \ty e \|
%    \pseq \ty \ty {\pterm e \ty} \nlalt
    \pmu{\ptyvar}{}{\ty}  \| \plam{\ptyvar}{}{\ty} 
%   \nlalt
%    \pcompute e \ity \| \pabsorb \ty \| \pscan{\ty} 
    \\
  \name{Types} \meta{\ty} \::= \tyval \| \papp{\ty}{e} \|
                               \papp{\ty}{\ty} \| \ptyvar \\
\\
  \name{Normalization:}
\end{bnf}  
\[
  \infer{
    \papp {\ty} {e} \stepsto \papp {\ty'} {e}
  }{
    \ty \stepsto \ty'
  }
\quad
  \infer{
    \papp {\tyval} {e} \stepsto \papp {\tyval} {e'}
  }{
    e \stepsto e'
  }
\quad
  \infer{
    \papp {(\plam x {} \ty)} {v} \stepsto \ty[v/x]
  }{}
\]
\[
  \infer{
    \papp {\ty_1} {\ty_2} \stepsto \papp {\ty_1'} {\ty_2}
  }{
    \ty_1 \stepsto \ty_1'
  }
\quad
  \infer{
    \papp {\tyval} {\ty} \stepsto \papp {\tyval} {\ty'}
  }{
    \ty \stepsto \ty'
  }
\quad
  \infer{
    \papp {(\plam \ptyvar {} \ty)} {\tyval} \stepsto \ty[\tyval/\ptyvar]
  }{}
\]
  \caption{\ddc{} Normal Types, selected constructs}
  \label{fig:ddc-reduction-rules}
  \label{fig:revised-ddc-syntax}
\end{figure}

\begin{definition}[Canonical Forms (selected constructs)]
(1) $\corr \tyval r p$ iff exactly one of the following is true:
  \begin{itemize}
  \item $\tyval = \pbase{e}$ and $r = \iinld \ity \const$ and $p.{nerr} = 0$.
  \item $\tyval = \pbase{e}$ and $r = \iinrd \ity \ierr$ and $p.{nerr} = 1$.
  \item $\tyval = \psig x {\ty_1} {\ty_2}$ and $r =\ipair {r_1} {r_2}$ and $p =
    \ipair h {\ipair {p_1} {p_2}}$ 
    and $h.{nerr} = \mathtt{pos}(p_1.{nerr}) + \mathtt{pos}(p_2.{nerr})$, $\corrkl
    {\ty_1} {r_1} {p_1}$ and $\corrkl {\ty_2[(r,p)/x]} {r_2} {p_2}$.
  \item $\tyval = \psum {\ty_1} e {\ty_2}$ and $r =\iinld {\ity}{r'}$
    and $p = \ipair h {\iinld {\ity}{p'}}$
    and $h.{nerr} = \mathtt{pos}(p'.{nerr})$ and $\corrkl
    {\ty_1} {r'} {p'}$.
  \item $\tyval = \psum {\ty_1} e {\ty_2}$ and $r =\iinr {r'}$
    and $p = \ipair h {\iinr {p'}}$
    and $h.{nerr} = \mathtt{pos}(p'.{nerr})$ and $\corrkl
    {\ty_2} {r'} {p'}$.
  \item $\tyval = \pset x {\ty'} e$, $r = \iinld \ity {r'}$ and $p =
    \ipair h {p'}$, 
    and $h.{nerr} = \mathtt{pos}(p'.{nerr})$, $\corrkl {\ty'}{r'}{p'}$
    and $e[(r',p')/x] \kstepsto\itrue$.
  \item $\tyval = \pset x {\ty'} e$, $r = \iinrd \ity {r'}$
    and $p = \ipair h {p'}$,
    and $h.{nerr} = 1 + \mathtt{pos}(p'.{nerr})$,
    $\corrkl {\ty'}{r'}{p'}$ and $e[(r',p')/x] \kstepsto \ifalse$.
  \item $\tyval = \pmu \ptyvar {} {\ty'}$, 
    $r = \iroll{r'}{\itsem[\pmu \ptyvar {} {\ty'}]}$, $p =
    \ipair h {\iroll{p'}{
                \itpdsem[\pmu \ptyvar {} {\ty'}]}}$, 
        $p.{nerr} = p'.{nerr}$ 
    and
    $\corrkl {\ty'[\pmu \ptyvar {} {\ty'}/\ptyvar]} {r'} {p'}$. \\
  \end{itemize}

\noindent
(2) $\corrkl \ty r p$ iff $\ty \kstepsto \tyval$ and $\corr \tyval r p$.
\end{definition}

Lemma~\ref{lem:err-corr-at-T}, part 1, states that the parsers for
well-formed types (of base kind) will produce a canonical pair of
representation and parse descriptor, if they produce anything at all.
Conversely, part 2 states that, given a canonical representation and
parse descriptor, the printer for well-formed types (of base kind)
will not ``go wrong'' by calling the $\failpp{}$ function.

\begin{theorem}[Parsing to/Printing from Canonical Forms]
\label{lem:err-corr-at-T}
\begin{itemize}
\item If $\ddck[\ty,,\kty,\con]$ and $\trans[\ty,,] \sapp \spair<B,\off> \kstepsto
  \spair<\off',r,p>$ then $\corrkl \ty r p$.
\item If $\ddck[\ty,,\kty,\con]$ , $\corrkl \ty r p$ and
  $\transpp[\ty,,] \sapp \spair<r,p> \kstepsto \iexp$ then $\iexp \neq
  \failpp{}$.
\end{itemize}
\end{theorem}

\begin{proof}
  Both items are proven by induction on the length of the respective 
  \fomega\ evaluation relations.  Within the induction
  they proceed using a case-by-case analysis of the possible structures 
  of type $\ty$.
\end{proof}


% \edcom{Add text here.}
% \begin{lemma}[Function Interpretation Corellation]
%   If $\ddck[\ty,,\kind,]$ and $\trans[\ty,,] \kstepsto \ival$ then
%   \begin{enumerate}
%   \item $\ty \kstepsto \tyval$,
%   \item $\ddck[\tyval,,\kind,]$
%   \item $\ival \iexpreq \trans[\tyval,,]$,
%   \item $\itsem[\ty] \equiv \itsem[\tyval]$, and
%   \item $\itpdsem[\ty] \equiv \itpdsem[\tyval]$.
%   \end{enumerate}
% \label{lemma:eval-corr}
% \end{lemma}
% \begin{proof}
%   By induction on evaluation derivations.
% \end{proof}



%%% Local Variables: 
%%% mode: latex
%%% TeX-master: "paper"
%%% End: 
