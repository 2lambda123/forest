\section{Describing Data in \padsmlbig{}}
\label{sec:padsml-overview}

A \padsml{} description specifies the physical layout and semantic
properties of an ad hoc data source.  These descriptions are composed
of types: base types describe atomic data, while structured types
describe compound data built from simpler pieces.  Examples of base
types include ASCII-encoded, 8-bit unsigned integers (\cd{Puint8}) and
32-bit signed integers (\cd{Pint32}), binary 32-bit integers (\cd{Pbint32}),
dates (\cd{Pdate}), strings (\cd{Pstring}), zip codes (\cd{Pzip}),
phone numbers (\cd{Pphone}), and IP addresses (\cd{Pip}).  Semantic
conditions for such base types include checking 
that the resulting number fits in the indicated space, \ie, 16-bits
for \cd{Pint16}.

Base types may be parameterized by \ml{} values.  This mechanism
reduces the number of built-in base types and permits base types to
depend on values in the parsed data.  For example, the base type
\cd{Puint16_FW(3)} specifies an unsigned two byte integer physically
represented by exactly three characters, and the base type
\cd{Pstring} takes an argument indicating the \textit{terminator
character}, \ie{}, the character in the source that
follows the string.

To describe more complex data, \padsml{} provides a collection of type
constructors derived from the type structure of functional programming
languages like Haskell and ML.  We explain these structured types in
the following subsections using examples drawn from data sources we
have encountered in practice.

% Readers eager to see the complete syntax
% of types should flip forward to Appendix~\ref{app:syntax-dd}.

\subsection{Simple Structured Types}

\cut{
\begin{figure*}
{\scriptsize
\begin{verbatim}
0|1005022800
9152|9151|1|9735551212|0||9085551212|07988|no_ii152272|EDTF_6|0|APRL1|DUO|10|1000295291
9153|9153|1|0|0|0|0||152268|LOC_6|0|FRDW1|DUO|LOC_CRTE|1001476800|LOC_OS_10|1001649601
\end{verbatim}
}
  \caption{Miniscule example of \dibbler{} data.}
  \label{figure:dibbler-records}
\end{figure*}
}

The bread and butter of a \padsml{} description are the simple
structured types: tuples and records for specifying ordered data,
lists for specifying homogeneous sequences of data, sum types for
specifying alternatives, and singletons for specifying the occurrence
of literal characters in the data.  We describe each of these
constructs as applied to the \dibbler{} data presented in
\figref{fig:sample-data}(b).

\dibbler{} data summarizes orders for phone service placed with AT\&T.
Each \dibbler{} data file starts with a timestamp followed by one
record per phone service order.  Each order consists of a header and a
sequence of events.  The header has 13 pipe separated fields: the
order number, AT\&T's internal order number, the order version, four
different telephone numbers associated with the order, the zip code of
the order, a billing identifier, the order type, a measure of the
complexity of the order, an unused field, and the source of the order
data.  Many of these fields are optional, in which case nothing
appears between the pipe characters.  The billing identifier may not
be available at the time of processing, in which case the system
generates a unique identifier, and prefixes this value with the string
``no\_ii'' to indicate that the number was generated. The event
sequence represents the various states a service order goes through;
it is represented as a new-line terminated, pipe separated list of
state, timestamp pairs.  There are over 400 distinct states that an
order may go through during provisioning.  The sequence is sorted in
order of increasing timestamps.  Clearly
English is a poor language for describing data formats!

\begin{figure}
\begin{code}\scriptsize
\kw{ptype} Semicolon = Pcharlit(';')
\kw{ptype} Vbar = Pcharlit('|')
\mbox{}
\kw{ptype} Pn\_t = Puint64
\kw{ptype} Pzip = Puint32
\mbox{}
\kw{ptype} Summary\_header = "0|" * Puint32 * Peor
\mbox{}
\kw{pdatatype} Dib\_ramp = 
  Ramp of Pint 
| GenRamp of "no\_ii" * Pint
\mbox{}
\kw{ptype} Order\_header = \{ 
    order\_num : Pint;  
'|'; att\_order\_num : [i:Pint | i < order\_num];  
'|'; ord\_version : Pint;  
'|'; service\_tn : Pn\_t Popt;
'|'; billing\_tn : Pn\_t Popt;  
'|'; nlp\_service\_tn : Pn\_t Popt;  
'|'; nlp\_billing\_tn : Pn\_t Popt;  
'|'; zip\_code : Pzip Popt;  
'|'; ramp : Dib\_ramp;  
'|'; order\_sort : Pstring('|');  
'|'; order\_details : Pint;
'|'; unused : Pstring('|');  
'|'; stream : Pstring('|'); 
'|'
\} 
\mbox{}
\kw{ptype} Event  = Pstring('|') * '|' * Puint32
\kw{ptype} Events = (Event,Vbar,Peor) Plist
\mbox{}
\kw{ptype} Order  = Order\_header * Events
\kw{ptype} Orders = (Order, Peor, Peof) Plist
\mbox{}
\kw{ptype} Source = Summary\_header * Orders\end{code}

  \caption{\padsml{} description for \dibbler{} provisioning data.}
  \label{figure:sirius_pml}
\end{figure}

\figref{figure:sirius_pml} contains the \padsml{} description for the
\dibbler{} data format.  The description is a sequence of type
definitions.  Type definitions precede uses, therefore the description
should be read bottom up.
The type \cd{Source} describes a complete \dibbler{} data
file and denotes an ordered tuple containing a
\cd{Summary\_header} value followed by an \cd{Orders} value.

The type \cd{Orders} uses the list type constructor
\cd{Plist} to describe a homogenous sequence of values in a data
source.  The \cd{Plist} constructor takes three parameters: on the
left, the type of elements in the list; on the right, a literal
\emph{separator} that separates elements in the list and a literal
\emph{terminator} that marks the end of the list.  
In this example, the type \cd{Orders} is a list of
\cd{Order} elements, separated by a newline, and terminated by
\cd{peof}, a special literal that describes the \emph{end-of-file
  marker}.  Similarly, the \cd{Events} type denotes a
sequence of \cd{Event} values separated by vertical bars and
terminated by a newline.

Literal characters in type expressions denote singleton types.  For
example, the \cd{Event} type is a string terminated by a
vertical bar, followed by a vertical bar, followed by a timestamp.  The
singleton type \cd{'|'} means that the data source must contain the
character \cd{'|'} at this point in the input stream.  String,
character, and integer literals can be embedded in a description and
are interpreted as singleton types, \eg{}, the singleton type
\cd{"0|"} in the \cd{Summary\_header} type
denotes the string literal \cd{"0|"}.

The type \cd{Order\_header} is a record type, \ie{}, a tuple type in
which each field may have an associated name.  The named field
\cd{att\_order\_num} illustrates two other features of
\padsml: dependencies and constraints.  Here, \cd{att\_order\_num}
depends on the previous field \cd{order\_num} and is constrained to be
less than that value.  In practice, constraints may be complex, have
multiple dependencies, and can specify, for example, the sorted order
of records in a sequence.  Constrained types have the form \cd{[x:T |
e]} where \cd{e} is an arbitrary pure boolean expression.  Data
satisfies this description if it satisfies \cd{T} and boolean \cd{e}
evaluates to true when the parsed representation of the data is
substituted for \cd{x}.  If the boolean expression evaluates to false,
the data contains a \textit{semantic} error.

The datatype \cd{Dib\_ramp} specifies two
alternatives for a data fragment, either one integer or the fixed
string \cd{"no\_ii"} followed by one integer.  The order of
alternatives is significant, that is, the parser attempts to parse the
first alternative and only if it fails, it attempts to parse the
second alternative.  This semantics differs from similar constructs in
regular expressions and context-free grammars, which
non-deterministically choose between alternatives.
\cut{Fortunately, we have yet to come across an ad hoc data
source where we wish we had nondeterministic choice.\footnote{\padsml{}
  can recognize string data based on regular expressions.
  Non-determinism here has been useful, but as it has been confined to
  parsing elements of the \cd{Pstring} base type, it has had no impact
  on the overall parsing algorithm.}
}
\cut{
\begin{figure}
\begin{code}\scriptsize
\kw{ptype} Semicolon = Pcharlit(';')
\kw{ptype} Rparen= Pcharlit(')')
\mbox{}
\kw{ptype} Entry = \{
  name : Pstring(':'); ':'; dist : Pfloat32
\}
\mbox{}
\kw{pdatatype} Tree =
  Interior of '(' * (Tree, Semicolon, Rparen) Plist * ')'
| Leaf of Entry\end{code}

%%% Local Variables: 
%%% mode: latex
%%% TeX-master: "paper"
%%% End: 

Tiny fragment of Newick data:

{
\begin{verbatim}
(((erHomoC:0.28006,erCaelC:0.22089):0.40998,
(erHomoA:0.32304,(erpCaelC:0.58815,((erHomoB:
0.5807,erCaelB:0.23569):0.03586,erCaelA:
0.38272):0.06516):0.03492):0.14265):0.63594,
(TRXHomo:0.65866,TRXSacch:0.38791):0.32147,
TRXEcoli:0.57336)
\end{verbatim}
}
  \caption{Simplified tree-shaped Newick data.  Newlines
     inserted to improve legibility.}
  \label{fig:newick}
\end{figure}
}


\subsection{Recursive Types}

\padsml{} can describe data sources with recursive structure.  An
example of such data is the Newick format, a flat representation
of trees used by biologists~\cite{newick}.
Example Newick data provided by Steven Kleinstein appears in
\figref{fig:sample-data}(c). 
The format uses properly nested
parentheses to specify a tree hierarchy.  A leaf node is a string
label followed by a colon and a number.  An interior node contains a
sequence of children nodes, delimited by parentheses, followed by a
colon and a number.  The numbers represent the ``distance'' that
separates a child node from its parent. 
In this example,  the string labels are gene names and the distances denotes the number of mutations that occur in the antibody receptor genes of B lymphocytes. \cut{Kleinstein uses this kind of data to study
the proliferation of B lymphocytes during an immune response.}
The following \padsml{} code 
describes this format:
\begin{code}\scriptsize
\kw{ptype} Semicolon = Pcharlit(';')
\kw{ptype} Rparen= Pcharlit(')')
\mbox{}
\kw{ptype} Entry = \{
  name : Pstring(':'); ':'; dist : Pfloat32
\}
\mbox{}
\kw{pdatatype} Tree =
  Interior of '(' * (Tree, Semicolon, Rparen) Plist * ')'
| Leaf of Entry\end{code}

%%% Local Variables: 
%%% mode: latex
%%% TeX-master: "paper"
%%% End: 

 
\cut{
\begin{figure}
  \centering
  \small
\begin{verbatim}
2:3004092508||5001|dns1=abc.com;dns2=xyz.com|
c=slow link;w=lost packets|INTERNATIONAL
3:|3004097201|5074|dns1=bob.com;dns2=alice.com|
src_addr=192.168.0.10;dst_addr=192.168.23.10;
start_time=1234567890;end_time=1234568000;
cycle_time=17412|SPECIAL
\end{verbatim}  
  \caption{Simplified network-monitoring data. Newlines
     inserted to improve legibility.}
  \label{fig:darkstar-records1}
\end{figure}
}

\subsection{Polymorphic Types and Advanced Datatypes}

Polymorphic types enable more concise descriptions and allow
programmers to define convenient libraries of reusable descriptions. The
description in \figref{fig:darkstar-ml} illustrates types
parameterized by both types and values.  It specifies
the format of alarm data recorded by a network-link monitor used in
the \darkstar{} project at AT\&T.  \figref{fig:sample-data}(a) contains corresponding example data. We describe the format in tandem with describing its \padsml{} description.
\begin{figure}
  \centering
  \begin{code}\scriptsize
(* Pstring terminated by ';' or '|'. *)
\kw{ptype} SVString = Pstring\_SE("/;|\\\\|/")
\mbox{}
(* Generic name value pair. Accepts predicate 
   to validate name as argument.� *)
\kw{ptype} (Alpha) Pnvp(p : string -> bool) =
      \{ name : [name : Pstring('=') | p name]; 
            '='; 
       value : Alpha \}
\mbox{}
(* Name value pair with name specified. *)
\kw{ptype} (Alpha) Nvp(name:string) = 
   Alpha Pnvp(fun s -> s = name)
\mbox{}
(* Name value pair with any name. *)
\kw{ptype} Nvp\_a = SVString Pnvp(fun \_ -> true)
\mbox{}
\kw{ptype} Details = \{
      source      : Pip Nvp("src\_addr");
';';  dest        : Pip Nvp("dest\_addr");
';';  start\_time  : Ptimestamp Nvp("start\_time");
';';  end\_time    : Ptimestamp Nvp("end\_time");
';';  cycle\_time  : Puint32 Nvp("cycle\_time")
\}

\kw{pdatatype} Info(alarm\_code : int) =
  match alarm\_code with
    5074 -> Details of Details
  | \_    -> Generic of Nvp\_a Plist(';','|')

\kw{pdatatype} Service = 
    DOMESTIC      of "DOMESTIC" 
  | INTERNATIONAL of "INTERNATIONAL" 
  | SPECIAL       of "SPECIAL"

\kw{ptype} Alarm = \{
       alarm    : [i : Puint32 | i = 2 or i = 3];
 ':';  start    : Ptimestamp Popt;
 '|';  clear    : Ptimestamp Popt;
 '|';  code     : Puint32;
 '|';  src\_dns  : SVString Nvp("dns1");
 ';';  dest\_dns : SVString Nvp("dns2");
 '|';  info     : Info(code);
 '|';  service  : Service
\} 

\kw{ptype} Source = Alarm Plist('\\n',\kw{peof})\end{code}
%
%\kw{let} checkCorr ra = ...
%\kw{ptype} Alarm = [x:Raw\_alarm | checkCorr x]



  \caption{Description of \darkstar{} data.}
  \label{fig:darkstar-ml}
\end{figure}

This data format has several variants of name-value pairs. The
\padsc{} description of this format (shown in~\appref{app:regulus-padsc})
must define a different type for each variant. In contrast, the
polymorphic types of \padsml{} allow us to define the type \cd{Pnvp},
which takes both type and value parameters to encode all the
variants. As is customary in \ml{},  type parameters appear to the left of the type name, while value parameters and their \ml{} types appear to
the right.  The type\cd{Pnvp} has
one type parameter named \cd{Alpha} and one value parameter named
\cd{p}.  Informally, \cd{Alpha Pnvp(p)} is a name-value pair where the
value is described by \cd{Alpha} and the name must satisfy the
predicate \cd{p}.

The \cd{Nvp} type reuses the\cd{Pnvp} type to define a name-value pair whose name must match the argument string \texttt{name} but whose value can have any type. 
The \cd{Nvp\_a} also uses the type \cd{Pnvp}. It defines a name-value pair that permits any name, but requires the value to have type \cd{SVString} (a string terminated by a semicolon or vertical bar).  Later in the description, the type parameter to \cd{Nvp} is instantiated with IP addresses, timestamps, and integers.

% The source type is an array of \cd{alarm}s, where each alarm is a
% \cd{raw\_alarm}, constrained to ensure that the alarm number is
% properly correlated with the timestamps.  We check this correlation
% with the function \cd{checkCorr}.  The type \cd{raw\_alarm} closely
% follows the description above. We highlight a few important features.
% First, we note that the type of the field \cd{info} depends on the
% alarm code, reflecting the text above. More interestingly, the type
% \cd{info} is implemented with a switched datatype, deciding how to
% parse based on the parameter \cd{alarm\_code}.  Next, we note that the
% description includes five different types of name-value pairs. We take
% advantage of both the type and value parameterization of types to
% encode all of these pair types based on one common description,
% \cd{pnvp}. This type is polymorphic in the type of the value and takes
% an arbitrary constraint \cd{c} as an argument. The type \cd{nvp} is
% polymorphic in the type of the value, but takes the expected name of
% the string as an argument. 

The \darkstar{} description also illustrates the use of
\textit{switched} datatypes.  A switched datatype selects a variant
based on the value of a user-specified \ocaml{} expression, which
typically references parsed data from earlier in the data source.  For
example, the switched datatype \cd{Info} chooses a
variant based on the value of its \cd{alarm\_code} parameter.  More
specifically, if the alarm code is \cd{5074}, the format specification
given by the \cd{Details} constructor will be used to parse the
current data.  Otherwise, the format given by the \cd{Generic}
constructor will be used.

The last construct in the \darkstar{} description is the type
qualifier \cd{omit}.  In the \cd{Service} datatype,
\cd{omit} specifies that the parsed string literal should be omitted
in the internal data representation because the literal can be
determined by the datatype constructor.

\cut{We can do this because we can discern from the
datatype constructor which string was found in the data source.}

%%% Local Variables: 
%%% mode: latex
%%% TeX-master: "../thesis.tex"
%%% End: 
