\section{Token Ambiguity Problem} \label{sec:examples}
\cut{
\begin{verbatim}
- example text
- describe where the examples come from, what the data is 
  (could be) used for
- show description written by hand (ideal case)
- show description from old learning system, which is verbose 
  and bad
  - explain why it's verbose and bad
\end{verbatim}
}
\begin{figure}[t]
\begin{center}
{\small
\begin{verbatim}
May 02 06:19:57 Updated: openssl.i686 0.9.7a-43.8
Jul 16 12:37:13 Erased: dhcp-devel
Jul 16 12:37:13 Erased: dhcpv6_client
Dec 10 04:07:51 Updated: openldap.x86_64 2.2.13-4
Dec 10 04:07:52 Updated: util-linux.x86_64 2.12a-16.EL4.12
Dec 10 04:07:58 Updated: nss_ldap.x86_64 226-10
Dec 10 04:11:17 Installed: postgresql-libs.x86_64 7.4.8-1.RHEL4.1
Dec 10 04:11:27 Installed: fonts-xorg-75dpi.noarch 6.8.1.1-1.EL.1
\end{verbatim}
}
\end{center}
\caption{Fragment of {\tt yum.txt} log file} \label{fig:yum}
\end{figure}

\begin{figure}[t]
\begin{minipage}[t]{0.5\columnwidth}
\begin{code}
\kw{Penum} action \{
  install \kw{Pfrom}("Installed");
  update \kw{Pfrom}("Updated");
  erase \kw{Pfrom}("Erased");
\};
\end{code}
\end{minipage}
\hfill
\begin{minipage}[t]{0.5\columnwidth}
\begin{code}
\kw{Precord Pstruct} entry_t \{
  \kw{Pdate} date;
  ' ';
  \kw{Ptime} time;
  ' ';
  action m;
  ": ";
  \kw{Pid} package;
  " "
  \kw{Pid} version;
\};
\end{code}
\end{minipage}
\caption{Ideal \pads{} description of {\tt yum.txt}}\label{fig:yum-gold}
\end{figure}


In this section, we introduce the token ambiguity problem through
a real-life ad hoc data source, and motivate a new probablistic approach
to solve this problem. Many other ad hoc data sources with similar
problem exist and more examples can be found on the \pads{} web site \cite{padsweb}.

Fig. \ref{fig:yum} shows a fragment from a log file, {\tt yum.txt},
which is written by a software package manager called {\tt yum}. {\tt yum} 
is a tool for installing, updating and removing packages and dependencies
on RPM-based systems. Each line in {\tt yum.txt} is a data record
which consists of several distinct fields: date, time, action taken,
package name and version. The fields are separated by single spaces.
This file is used by system administrators to keep track of the
software updates on the system.
Fig.\ref{fig:yum-gold} shows an {\em ideal} \pads{} description of
{\tt yum.txt} written by a human expert.
The {\tt entry\_t} type describes one record line in the data source.
{\tt Pdate} stands for a date type, {\tt Ptime} stands for a time type,
action type is defined to an enumeration of three different actions,
and {\tt Pid} is any tokens that constitutes a system identifier such as a file
name, a MAC address, etc.

\begin{figure}[t]
\begin{centercode}
\kw{Penum} Enum_10 \{
  Installed10 \kw{Pfrom}("Installed"),
  Updated10 \kw{Pfrom}("Updated"),
  Erased10 \kw{Pfrom}("Erased")
\};
\kw{Popt Pstring_ME}(:"/\\+\\+/":) Opt_22;
\kw{Penum} Enum_31 \{
  x86_6431 \kw{Pfrom}("x86_64"),
  noarch31 \kw{Pfrom}("noarch"),
  i68631 \kw{Pfrom}("i686"),
  i38631 \kw{Pfrom}("i386")
\};

... /* Complex definition of Struct_154 and Struct_233 
       (141 lines of code) ommitted */

\kw{Punion} Union_155 \{
  Struct_154  v_struct_154;
  Struct_223  v_struct_223;
\};
\kw{Pstruct} Struct_205 \{
  Opt_22  v_opt_22;
  '.';
  Enum_31  v_enum_31;
  ' ';
  Union_155  v_union_155;
\};
\kw{Popt} Struct_205 Opt_206;
\kw{Precord Pstruct} entry_t \{
  \kw{Pdate}  v_date_1;
  ' ';
  \kw{Ptime}  v_time_6;
  ' ';
  Enum_10  v_enum_10;
  ": ";
  \kw{Pword} v_string_17;
  Opt_206  v_opt_206;
\};
\end{centercode}
\caption{Inferred \pads{} description of {\tt yum.txt} by \learnpads}\label{fig:yum-bad}
\end{figure}

 
To infer a \pads{} description for {\tt yum.txt},
we input the data file to the \learnpads{} system 
\cite{fisher+:dirttoshovels,fisher+:sigmod08}, and we obtain
a description in Fig.\ref{fig:yum-bad}.
This description, despite being able to correctly parse the original data,
is very verbose and un-intuitive. It is whopping 176 line long compared 
to the neat 16-liner written by human expert.

The main problem with the inferred description is that it was not
able to take advantage of the {\tt Pid} token type, but instead chose
to use finer-grained tokens such as {\tt Pword}, {\tt Pint}, etc,
to represent the package names and version numbers. This introduced the
complex structure within the package names and versions into the
description. This is because the tokenization in \learnpads{} is done by
a lexer generated from a series of regular expression token definitions.
At runtime, the lexer attempts the defined tokens, one at
a time, for a given input string, and uses the first match to parse the
string.  In other words, the order of the token definitions in the lexer
is significant, and given a string such as ``{\tt 2.2.13-4}'', the parsed
token sequence is unique and fixed. While such an arrangement is simple
and effective for some data sources, it severely restricted the types of
tokens that can be included in the lexer. In particular, tokens that are
heavily overlapping in their definitions cannot be included together in the lexer.
For example, because the definition of {\tt Pid} includes that of
{\tt Pword} and  {\tt Pint}, it is excluded from the lexer. 
When such overlapping or ambiguous tokens do co-exist in the lexer, such as
{\tt Pint} and {\tt Pfloat},
the choice made by the lexer could very well be wrong.

There are numerous sources of such token ambiguity 
in {\tt yum.txt}. For instance, the string ``{\tt Dec 10 04:07:51}'' 
can be interpreted as, among other choices, any of these
token sequences:

{\small
\begin{verbatim}
Option 1: [Pword] [' '] [Pint] [' '] [Pint] [:] [Pint] [:] [Pint]
Option 2: [Pmonth] [' '] [Pint] [' '] [Ptime]
Option 3: [Pdate] [' '] [Ptime]
\end{verbatim}
}

The string ``{\tt 2.2.13-4}'' can be parsed into one of 
these possible sequences:
{\small
\begin{verbatim}
Option 1: [Pint] [.] [Pint] [.] [Pint] [-] [Pint]
Option 2: [Pfloat] [.] [Pint] [-] [Pint]
Option 3: [Pint] [.] [Pfloat] [-] [Pint]
Option 4: [Pid]
\end{verbatim}
}

Depending on the context, any of these options might be better than
the rest.  The fixed tokenization scheme used in \learnpads{} is simply
not good enough to produce all these possibilities. What is needed
is a probablistic approach that weighs the likelyhood of all the ambiguous
tokens and picks the best option.
