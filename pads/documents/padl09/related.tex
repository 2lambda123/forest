\section{Related Work}\label{sec:related}

In the last two decades, there has been extensive work on 
classic grammar induction problems
\cite{vidal:gisurvey,gold:inference,bex+:dtd-inference,angluin:revesible-language-inference,Chen95bayesiangrammar},
XML schema inference \cite{bex+:dtd-inference,garofalakis+:xtract}, 
information extraction \cite{kushmerick-phd1997,hong:thesis,arasu+:sigmod03},
and other related areas such as natural language processing 
\cite{borkar+:text-segmentation,Heeman99:speech} and bioinformatics 
\cite{kulp96generalized}. Machine learning techniques have played a
very important role in these areas. 
Our earlier paper \cite{fisher+:dirttoshovels}
contains an extensive comparison of our basic format inference
algorithm to others that have appeared in the literature.

%
%
%
%\cite{kushmerick-phd1997,arasu+:sigmod03,hong:thesis}, semi-structured data
%\cite{soderland:whisk,Pinto+:texttables}, free-text 
%\cite{borkar+:text-segmentation,Heeman99:speech,Chen95bayesiangrammar}, and
%bioinformatics \cite{kulp96generalized}. In particular, compared to classic 
%grammar induction techniques
%\cite{angluin:revesible-language-inference},
%machine learning methods have become an attractive option for handling
%heterogeneous and complex data. Our earlier paper \cite{fisher+:dirttoshovels}
%contains an extensive comparison of our basic format inference
%algorithm to others that have appeared in the literature.
%

One of the most closely related pieces of work to this paper is Soderland's WHISK
system \cite{soderland:whisk}, which extracts useful information from
semi-structured text such as stylized advertisements from an online community
service called Craig's List \cite{craigslist}.
%Information extraction (IE) is a research field that is closely related to
%our work in this paper. 
In the WHISK system, the user is presented with a few online ads as training data 
and is asked to label which bits of information to extract. Then the system learns
extraction rules from labeled data and uses them to retrieve more wanted
information from a much larger collection of data. 
The WHISK system differs from our system in several ways. 
First, WHISK, as well as other information extraction systems,
have a clear and fixed token set, defined by words, numbers, punctuations, 
HTML tags and user pre-specified semantic classes, etc. 
Second, WHISK only focuses on certain bits of
information, namely, single or multiple fields in records, 
whereas we not only identify useful fields, but also obtain the 
organization and relations of these fields by generating the complete 
description of the entire data file. 
Last, in WHISK, the extraction rules learned from a particular domain can only
be used on data from the same domain. For example, rules learned from
sample on-line rental
ads are only relevant to other rental ads, and cannot be applied to
software job postings. But the statistical token models we learned in our system can be
applied to many different types of data, as shown in the experiments we
have done in Section \ref{sec:eval}.  

%Last, there is significantly less labeling efforts to use our system than 
%the WHISK system. Our system exploits an automatic labeling 
%infrastructure using \pads{} to generate
%labels for hundreds of records from only a single hand-written description.
%This is in contrast with having users hand label all training samples
%in WHISK.

Also closely related is the work on text table extraction by
Pinto and others \cite{Pinto+:texttables}. Text tables can
be viewed as special ad hoc data with a tabular layout. There are
often clear delimiters between columns in the table, and table rows
are well defined with new line characters as their boundaries.
Because of its tabular nature, the data studied has less variation in general. 
The goal of their work  
is to identify tables embeded in free text and the types of table rows 
such as header, sub-header and data row, etc, whereas we are learning the entire
structure of the data. 
To this end, Pinto et al. use Conditional Random Fields (CRFs) \cite{LaffertyMP01:CRF},
a statistical model that is useful in learning from sequence data with overlapping features.
This paper extracts features from white space characters,
text between white spaces and punctuations. Although not explicitly
stated, words, numbers and punctuations are used as fixed set of tokens. 
 
%Anotherdifference with their work is that, the
%learning task there only deals with dozens of lines in one table while 
%we deal with thousands of tokens in one data file. 
%Therefore, computational complexity is an important factor in 
%the selection of machine learning algorithms. 

To summarize, problems studied by previous efforts 
in grammar induction and information extraction do not 
typically suffer from token ambiguities that 
we see in ad hoc data, because tags cleanly divide \xml{} and web-based data, 
while spaces and known punctuation symbols separate natural language text.
In contrast, the separators and token types found in ad hoc data sources such as
web logs and financial records are far more variable and ambiguous.  

%Format inference of ad hoc data is related to three main areas: 
%{\em classic grammar induction}, {\em XML schema inference},
%and {\em information extraction}. Classic grammar induction aims at
%discovering a grammar for a language using both positive and
%negative examples or positive examples only \cite{vidal:gisurvey}. 
%Since there is no negative examples in our application, only the latter
%is relevant to this paper. However, because learning a perfect
%grammar using positive examples only is impossible for any superfinite class 
%of languages \cite{gold:inference}
%which includes regular expressions and context-free grammar (and also
%\pads), previous research has focused on (1) limiting the class of
%languages to non-superfinite class 
%\cite{angluin:revesible-language-inference,bex+:dtd-inference}, or 
%(2) learning an approximation using heuristics or 
%probabilistic models \cite{hong:thesis,Chen95bayesiangrammar}.
%XML schema inference involves learning a schema such as DTD or XSD
%from a collection of XML documents \cite{bex+:dtd-inference,garofalakis+:xtract}.
%In information extraction, the goal is to find interesting and relevant
%bits of information from large amount of data such as 
%web pages \cite{kushmerick-phd1997,arasu+:sigmod03} and 
%free text \cite{soderland:whisk,Pinto+:texttables,borkar+:text-segmentation}.
%These areas do not typically suffer from the token ambiguity problem that 
%we see in ad hoc data, because tags cleanly divide \xml{} and web-based data, 
%while spaces and known punctuation symbols separate natural language text.
%In contrast, the separators and token types found in ad hoc data sources such as
%web logs and financial records are far more variable and ambiguous.  

%Classic grammar induction and structure discovery problems have been
%studied for decades. \xml{}-documents are a typical source of
%text-based data that possibly adheres to some structures represented by a
%schema. ~\cite{garofalakis+:xtract} and ~\cite{bex+:inferring-xml-schema}
%infers two \xml{} schema formalisms respectively -
%\em{Document Type Definitions} (DTDs) and \em{XML Schema Definitions}
%(XSDs). ~\cite{bex+:dtd-inference} is focused on inferring
%concise DTDs by considering two class of DTDs with special
%properties. In recent researches, statistical methods have been widely
%used in grammar induction. ~\cite{kushmerick-phd1997} automatically learns the procedure
%for extracting content from the web in the context of inductive
%learning and accesses its learnability by
%fitting the problem into PAC learning model. ~\cite{hong:thesis}
%resolves a similar problem by explicit domain rules and a
%hill-climbing inference algorithm. ~\cite{soderland:whisk} tackles
%both semi-structured and free-text data, and uses covering algorithms.
%It also shares a methodology with inductive learning. For free-text,
%some syntactic and semantic pre-processing analysis is conducted in
%addition to text extraction rules. ~\cite{Chen95bayesiangrammar}
%employs a greedy heuristic search algorithm within a Bayesian
%framework for probabilistic context-free grammar induction. 
%

%In addition to applying statistical models such as HMMs and Conditional
%Random Fields (CRFs) to grammar induction problems 
%\cite{borkar+:text-segmentation,Pinto+:texttables}, researchers have also
%used these models extensively in other areas such as bioinformatics 
%\cite{kulp96generalized} and natural language processing \cite{Heeman99:speech}.  
%In this paper, we contribute to the literature on statistical
%data processing by analyzing the effectiveness of statistical models
%in a new application area, that of ad hoc data, which contains
%markedly different characteristics from the most frequently studied
%data processing domains.

%The algorithm presented in this paper is a variant of the algorithm described
%in our earlier paper~\cite{fisher+:dirttoshovels}.  We have made
%substantial modifications, however, to incorporate probabilistic
%parsing information into the tokenization and structure discovery
%phases.  We have also added a new phase to the algorithm that
%simplifies overly complex descriptions. Our earlier paper contains an
%extensive comparison of our basic grammar induction
%algorithm to others that have appeared in the literature.

%
%Besides grammar induction, researchers have also witnessed successes of
%statistical and machine learning methods in a variety of other areas.
%~\cite{borkar+:text-segmentation} extracts certain fields in records
%from text using nested HMMs. ~\cite{Pinto+:texttables} utilizes
%Conditional Random Fields (CRFs) to dig rich and overlapping features
%with the purpose of identifying row positions and types in text-based
%tables. Computational biology and natural language processing are also
%two areas in which statistical methods play an important
%role. ~\cite{kulp96generalized} predicts exons and introns in DNA
%sequences by generalized HMMs. An early step of natural language
%understanding is to assign part-of-speech (POS) tags to segments of
%speech sentences and ~\cite{Heeman99:speech} employs decision tree
%algorithms to address this task. These problems typically deal with
%sequencial data, extract features from parts of the data and apply
%machine learning methods to predict certain properties of these
%parts. The data in our token ambiguity problem is different from these
%data, and the learning methods in our system must achieve high
%precision, while still only costs tolerable execution time for large
%amount of data.

