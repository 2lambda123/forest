% This file was updated slightly by Prasad Tadepalli and Ricardo Silva
% for ICML-2007, by Andrew Moore for ICML-2006, and by Kristian Kersting
% for ICML-2005 from the file made available from the ICML-2004 web site.

% This file was updated by Dale Schuurmans and Jennifer Dy for ICML-2004
% from a file modifed slightly by Terran Lane for to support/document 
% the new option to the ``usepackage'' statement that selects between the 
% review and final version of the paper.

% This file was updated slightly by Tom Fawcett for ICML-2003
% from the file made available from the ICML-2002 web site.

% This file was modified by Claude Sammut for ICML-2002 from the file
% made available by Pat Langley, Claude-Nicolas Fiechter, Mehmet Goker,
% Cynthia Thompson and Andrea Danyluck for ICML-2K and ICLML-2001.

% Use the following line _only_ if you're still using LaTeX 2.09.
%\documentstyle[icml2006,epsf,mlapa]{article}
% If you rely on Latex2e packages, replace the above line with: 
\documentclass{article}
% Use the following version of the ``usepackage'' statement for
% submitting the draft version of the paper for review.  This will set
% the note in the first column to ``Under review.  Do not distribute.''
\usepackage{icml2007} 
% Use this version of the ``usepackage'' statement after the paper has
% been accepted, when creating the final version.  This will set the
% note in the first column to ``Appearing in''
%\usepackage[accepted]{icml2007}
\usepackage{epsf}
\usepackage{mlapa}

% The \icmltitle is too long. Therefore, a short form for the running title 
% is supplied just before \begin{document} 
\icmltitlerunning{Submission and Formatting Instructions for ICML-2007}

\begin{document} 

\twocolumn[
\icmltitle{Submission and Formatting Instructions for the Twenty-fourth \\ 
           International Conference on Machine Learning (ICML-2007)}

\icmlauthor{}{}
\icmladdress{\textbf{ICML2007, format, submission instructions}}
% The following author list should only appear in the accepted versions. 
% \icmlauthor{Pat Langley}{langley@isle.org}
% \icmladdress{Institute for the Study of Learning and Expertise, 
%             2164 Staunton Court, Palo Alto, CA 94306 USA}
% \icmlauthor{Claude-Nicolas Fiechter}{fiechter@rtna.daimlerchrysler.com}
% \icmlauthor{Mehmet G\"{o}ker}{goker@rtna.daimlerchrysler.com}
% \icmladdress{DaimlerChrysler Research and Technology Center, 
%             1510 Page Mill Road, Palo Alto, CA 94304 USA}
% \icmlauthor{Cynthia Thompson}{cthomp@csli.stanford.edu}
% \icmladdress{Computational Learning Laboratory, 
%             Center for the Study of Language and Information, 
%             Stanford University, Stanford, CA 94305 USA}
% \icmlauthor{Andrea Danyluk}{andrea@cs.williams.edu}
% \icmladdress{Department of Computer Science, Williams College,
%             Williamstown, MA 01267 USA}
% \icmlauthor{Claude Sammut}{claude@cse.unsw.edu.au}
% \icmladdress{School of Computer Science and Engineering,
%             University of New South Wales,
%             Sydney NSW 2052, Australia}
% \icmlauthor{Tom Fawcett}{tom\_fawcett@hp.com}
% \icmladdress{HP Laboratories, 1501 Page Mill Road, 
%             Palo Alto, CA 94304 USA}
% \icmlauthor{Terran Lane}{terran@cs.unm.edu}
% \icmladdress{Department of Computer Science, University of New Mexico,
%             Albuquerque, NM 87131 USA}
% \icmlauthor{Jennifer Dy}{jdy@ece.neu.edu}
% \icmladdress{Department of Electrical and Computer Engineering, 
%             Northeastern University, 
%             Boston, MA 02115 USA}
% \icmlauthor{Dale Schuurmans}{dale@cs.ualberta.ca}
% \icmladdress{Department of Computing Science, University of Alberta, 
%              Edmonton, AB T6G 2E8 Canada}
% \icmlauthor{Kristian Kersting}{kersting@informatik.uni-freiburg.de}
% \icmladdress{Institute for Computer Science, University of Freiburg, 
%              Georges-Koehler-Allee, Bulding 079, 79110 Freiburg, Germany}
% \icmlauthor{Codrina Lauth}{Codrina.Lauth@ais.franhofer.de}
% \icmladdress{Fraunhofer Institute for Autonomous Intelligent Systems,
% 	     Schloss Birlinghoven, 53754 Sankt Augustin, Germany}

\vskip 0.3in
]

\begin{abstract} 
ICML-2007 reviewing will be blind to the identities of the authors,
and therefore identifying information should not 
appear in papers submitted for review.
\end{abstract} 

\section{Electronic Submission}
\label{submission}

As in the past few years,
ICML-2007 will rely heavily on electronic formats for submission and 
review. We assume that nearly all authors will have access to standard
software for word processing, electronic mail, and ftp file transfer. 
Authors who do not have such access should send email with their concerns 
to \texttt{icml07@eecs.oregonstate.edu}.

\subsection{Templates for Papers}

Electronic templates for producing papers for submission are available
for several major word processors, including \LaTeX\/ and Microsoft Word. 
Templates will be accessible on the World Wide Web at:\\
\textbf{\texttt{http://oregonstate.edu/conferences/icml2007}}

\noindent
Send questions about these electronic templates to
\texttt{icml07@eecs.oregonstate.edu}.

\subsection{Submitting Papers}

Submission to ICML-2007 will be entirely electronic, via a web site
(not email).  The URL and information about the submission process
will appear on the conference web site at:\\
\textbf{\texttt{http://oregonstate.edu/conferences/icml2007}}

{\bf Abstract Deadline:} To submit a paper to ICML-2007, authors must
first submit an electronic abstract by Wednesday, February 7, 2007,
23:59:59 Apia, Samoa time.
If your abstract does not reach us by this date, then the paper will not be
considered for publication.

{\bf Paper Deadline:} The deadline for paper submission to ICML-2007 is 
Friday, February 9, 2007, 23:59:59 Apia, Samoa time. 
If your submission does not reach us by this date, it will not be
considered for publication.

To facilitate blind review, no author information should appear on 
the paper.  
Section~\ref{author info} will explain the details of how to format this.

To ensure our ability to print submissions, authors must provide their
manuscripts in \textbf{postscript} or \textbf{pdf} format. If you are
preparing your paper in Word, please use the Apple LaserWriter
16/600~PS driver to ensure its printability in other environments.  
Authors using \textbf{Word} must convert their document to postscript or pdf. 
Most of the latest versions of Word have the facility to do this automatically.
Those who use \textbf{latex} to format their accepted papers need to pay close attention to the typefaces used. 
Specifically, when converting the dvi output of LaTeX to Postscript the default behavior is to use non-scalable Type 3 PostScript 
bitmap fonts to represent the standard LaTeX fonts. The resulting document is difficult to read in electronic form; 
the type appears fuzzy. To avoid this problem, dvips must be instructed to use an alternative font map. 
This can be achieved with the following command:
\begin{center}
\textbf{ dvips -Ppdf -tletter -G0 -o paper.ps paper.dvi}
\end{center}

Note that it is a zero following the ``-G''. This tells dvips to use the
config.pdf file (and this file refers to a better font
mapping). Another alternative is to use the \textbf{pdflatex} program instead of straight
LaTeX. This program avoids the Type 3 font problem, however you
must ensure that all of the fonts are embedded (use pdffonts). If they
are not, you need to configure pdflatex to use a font map file that
specifies that the fonts be embedded. Also you should ensure that
images are not downsampled or otherwise compressed in a lossy way.
If you cannot deliver a postscript or pdf file electronically due to
exceptional conditions, send email to \texttt{icml07@eecs.oregonstate.edu} to
discuss alternative means of delivery.

ICML will not accept any paper which, at the time of submission, is
under review for another conference or a journal; is under review
elsewhere; or has already been published. This policy also applies to
papers that overlap substantially in technical content with papers
under review or previously published. Authors are also expected not to
submit their papers elsewhere during ICML's review period.

\subsection{Reacting to Reviews}
We will repeat the successful innovation introduced by ICML2005 in
Bonn, in which the authors will be given the option of providing a
short reaction to the initial reviews. 
%Reactions will have to be input
%in CyberChair between {\bf 18 and 25 April} by the authors. 
These reactions will be taken into account in the discussion among the
reviewers and PC-members.

\subsection{Submitting Final Camera-Ready Copy}

Final versions of papers accepted for publication should follow the same
format and naming convention as initial submissions,
except of course that the normal author information (names and affiliations)
should be given. 
See Section~\ref{final author} for details of how to format this.

The footnote,  ``Preliminary work.  Under review by the International
Conference on Machine Learning (ICML).  Do not distribute.'' must be modified
to ``Appearing in \textit{Proceedings of the $\mathit{24}^{th}$ International
Conference on Machine Learning}, Corvallis, OR, 2007.  Copyright 2007 by 
the author(s)/owner(s).''
For those using the LaTeX style file, simply change $\mathtt{\backslash usepackage\{icml2007\}}$
to $\mathtt{\backslash usepackage[accepted]\{icml2007\}}$.  
Authors using Word, must edit the footnote on the first page of the document
themselves.

Camera-ready copies should have the title of the paper as 
running head on each page except the first one. 
The running title consists of a single line centered above a horizontal rule which is $1$ point thick.
The running head should be centered, bold and in $9$ point type.
The rule should be $10$ points above the main text.
For those using the LaTeX style file, the original title is automatically set as
running head using the {\tt fancyhdr} package which can be obtained at the ICML-2007 web site. 
In case that the original title
exceeds the size restrictions, a shorter form can be supplied by using
$$\mathtt{\backslash icmltitlerunning\{\ldots\}}$$
just before $\mathtt{\backslash begin\{document\}}$.
Authors using Word, must edit the header of the document themselves.

\section{Format of the Paper} 
 
All submissions should follow the same format to ensure the printer
can reproduce them without problems and to let readers more easily
find the information that they desire. 

\subsection{Length and Dimensions}

Papers must not exceed eight (8) pages, including all figures, tables, 
references, and appendices. We will return to the authors any submissions
that exceed this page limit or that diverge significantly from the format 
specified herein.

The text of the paper should be formatted in two columns, with an
overall width of 6.75 inches, length of 9.0 inches, and 0.25 inches
between the columns. The left margin should be 0.75 inches and the top
margin 1.0 inch (2.54~cm). The right and bottom margins will depend on
whether you print on US letter or A4 paper. 

The paper body should be set in 10~point type with a vertical spacing of 
11~points. Please use Times Roman typeface throughout the text. 

\subsection{Title}

The paper title should be set in 14~point bold type and centered between 
two horizontal rules that are 1~point thick, with 1.0~inch between the
top rule and the top edge of the page. Capitalize the first letter of 
content words and put the rest of the title in lower case. 

\subsection{Author Information for Submission}
\label{author info}

To facilitate blind review, author information must not appear.
In place of the author name(s) and affiliation(s)
\textbf{keywords} will be given.
If you are using \LaTeX\/ and the \texttt{icml2007.sty} file,
simply replace the author name and address by the \textbf{keywords} 
as follows:

\verb+\icmlauthor{+\verb+}{}+\\
\verb+\icmladdress{+\textbf{keyword1,...,keywordN}\verb+}+

This is already done in the current latex file. 
Submissions that inculde the author information
will not be reviewed.

\subsubsection{Camera-Ready Author Information}
\label{final author}

If a paper is accepted, a final camera-ready copy must be prepared which
includes the usual author information.

For camera-ready papers,
author information should start 0.3~inches below the bottom rule
surrounding the title. The authors' names should appear in 10~point 
bold type, electronic mail addresses in 10~point small capitals, and
physical addresses in ordinary 10~point type.
Each author's name should be flush left, whereas the email address
should be flush right on the same line. The author's physical address
should appear flush left on the ensuing line, on a single line if
possible. If successive authors have the same affiliation, then give
their physical address only once.

\subsection{Abstract}

The paper abstract should begin in the left column, 0.4~inches below the
final address. The heading `Abstract' should be centered, bold, and in
11~point type. The abstract body should use 10~point type, with a 
vertical spacing of 11~points, and should be indented 0.25~inches more
than normal on left-hand and right-hand margins. Insert 0.4~inches 
of blank space after the body. Keep your abstract brief, limiting it
to one paragraph and no more than six or seven sentences.

\subsection{Partitioning the Text} 

You should organize your paper into sections and paragraphs to help 
readers place a structure on the material and understand its contributions. 

% Use \vspace for fine control of spacing above and below headings. 
\vspace{-0.018in}
\subsubsection{Sections and Subsections}
\vspace{-0.015in}

Section headings should be numbered, flush left, and set in 11~pt bold
type with the content words capitalized. Leave 0.25~inches of space 
before the heading and 0.15~inches after the heading. 

Similarly, subsection headings should be numbered, flush left, and set
in 10~pt bold type with the content words capitalized. Leave 0.2~inches 
of space before the heading and 0.13~inches afterward.

Finally, subsubsection headings should be numbered, flush left, and set
in 10~pt small caps with the content words capitalized. Leave 0.18~inches 
of space before the heading and 0.1~inches after the heading. Please 
use no more than three levels of headings.

\subsubsection{Paragraphs and Footnotes}

Within each section or subsection, you should further partition the
paper into paragraphs. Do not indent the first line of a given
paragraph, but insert a blank line between succeeding ones.
 
You can use footnotes\footnote{For the sake of readability, footnotes
should be complete sentences.} to provide readers with additional
information about a topic without interrupting the flow of the paper.
Indicate footnotes with a number in the text where the point is most
relevant. Place the footnote in 9~point type at the bottom of the
column in which it appears. Precede the first footnote in a column 
with a horizontal rule of 0.8~inches.\footnote{Multiple footnotes can
appear in each column, in the same order as they appear in the text,
but spread them across columns and pages if possible.}

\begin{figure}[h]
\vskip 0.2in
\begin{center}
\setlength{\epsfxsize}{3.25in}
\centerline{\epsfbox{role.ps}}
% \vskip 0.1in
\caption{Steps in the computational discovery process at which the
         developer can influence system behavior.}
\label{process-flow}
\end{center}
\vskip -0.2in
\end{figure} 

\subsection{Figures}
 
You may want to include figures in the paper to help readers visualize
your approach and your results. Such artwork should be centered,
legible, and separated from the text. Lines should be dark and at
least 0.5~points thick for purposes of reproduction, and text should
not appear on a gray background.

% Use \newpage to insert a page break between paragraphs. 

Label all distinct components of each figure. If the figure takes
the form of a graph, then give a name for each axis and include a 
legend that briefly describes each curve. However, do {\it not\/} 
include a title above the figure, as the caption already serves
this function. 

Number figures sequentially, placing the figure number and caption 
{\it after\/} the graphics, with at least 0.1~inches of space before the
caption and 0.1~inches after it, as in Figure~\ref{process-flow}. 
The figure caption should be set in 9~point type and centered unless
it runs two or more lines, in which case it should be flush left. 
You may float figures to the top or bottom of a column, and you may
set wide figures across both columns, but always place two-column
figures at the top or bottom of the page.

\comment{
% Sample commands to format two figures side by side across the page. 
\begin{figure*}[t]
\hbox{\hskip -0.55in
\setlength{\epsfxsize}{3.25in}
\epsfbox{irrel.ps}
\hskip 0.2in
% \hfill
\setlength{\epsfxsize}{3.25in}
\epsfbox{rel.ps}
}
\vskip 0.1in
\caption{Theoretical and experimental learning curves for naive Bayes
         when (a) the domain involves a `2 of 2' target concept and
         varying numbers of irrelevant attributes, and (b) for a domain
         with one irrelevant attribute and a conjunctive target concept
         with varying numbers of relevant features.}
\label{bayes-curves}
\end{figure*}
}

\subsection{Algorithms}

Please use the ``algorithm'' and ``algorithmic'' 
environments to format pseudocode. These require 
the corresponding stylefiles, algorithm.sty and 
algorithmic.sty, which are supplied with this package. 
Algorithm~\ref{alg:example} shows an example. 

\begin{algorithm}[tb]
   \caption{Bubble Sort}
   \label{alg:example}
\begin{algorithmic}
   \STATE {\bfseries Input:} data $x_i$, size $m$
   \REPEAT
   \STATE Initialize $noChange = true$.
   \FOR{$i=1$ {\bfseries to} $m-1$}
   \IF{$x_i > x_{i+1}$} 
   \STATE Swap $x_i$ and $x_{i+1}$
   \STATE $noChange = false$
   \ENDIF
   \ENDFOR
   \UNTIL{$noChange$ is $true$}
\end{algorithmic}
\end{algorithm}
 
\subsection{Tables} 
 
You may also want to include tables that summarize material. Like 
figures, these should be centered, legible, and numbered consecutively. 
However, place the title {\it above\/} the table with at least 
0.1~inches of space before the title and the same after it, as in 
Table~\ref{sample-table}. The table title should be set in 9~point 
type and centered unless it runs two or more lines, in which case it
should be flush left.

% Note use of \abovespace and \belowspace to get reasonable spacing 
% above and below tabular lines. 

\begin{table}[t]
\caption{Classification accuracies for naive Bayes and flexible 
Bayes on various data sets.}
\label{sample-table}
\vskip 0.15in
\begin{center}
\begin{small}
\begin{sc}
\begin{tabular}{lcccr}
\hline
\abovespace\belowspace
Data set & Naive & Flexible & Better? \\
\hline
\abovespace
Breast    & 95.9$\pm$ 0.2& 96.7$\pm$ 0.2& $\surd$ \\
Cleveland & 83.3$\pm$ 0.6& 80.0$\pm$ 0.6& $\times$\\
Glass2    & 61.9$\pm$ 1.4& 83.8$\pm$ 0.7& $\surd$ \\
Credit    & 74.8$\pm$ 0.5& 78.3$\pm$ 0.6&         \\
Horse     & 73.3$\pm$ 0.9& 69.7$\pm$ 1.0& $\times$\\
Meta      & 67.1$\pm$ 0.6& 76.5$\pm$ 0.5& $\surd$ \\
Pima      & 75.1$\pm$ 0.6& 73.9$\pm$ 0.5&         \\
\belowspace
Vehicle   & 44.9$\pm$ 0.6& 61.5$\pm$ 0.4& $\surd$ \\
\hline
\end{tabular}
\end{sc}
\end{small}
\end{center}
\vskip -0.1in
\end{table}

Tables contain textual material that can be typeset, as contrasted 
with figures, which contain graphical material that must be drawn. 
Specify the contents of each row and column in the table's topmost
row. Again, you may float tables to a column's top or bottom, and set
wide tables across both columns, but place two-column tables at the
top or bottom of the page.
 
\subsection{Citations and References} 

Authors should cite their own work in the third person
in the initial version of their paper submitted for blind review.

Please use APA reference format regardless of your formatter
or word processor. If you rely on the \LaTeX\/ bibliographic 
facility, use {\tt mlapa.sty} and {\tt mlapa.bst} 
at the ICML-2007 web site to obtain this format.

Citations within the text should include the authors' last names and
year. If the authors' names are included in the sentence, place only
the year in parentheses, as in \singleemcite{jones:mlj94}, but otherwise 
place the entire reference in parentheses with the authors and year
separated by a comma \cite{jones:mlj94}.

List multiple references alphabetically and separate them by semicolons 
\cite{jones:mlj94,veloso:bkchapter93}. Use the `et~al.' construct only 
for citations with four or more authors or after listing all authors
to a publication in an earlier reference.

Use an unnumbered first-level section heading for the references, and 
use a hanging indent style, with the first line of the reference flush
against the left margin and subsequent lines indented by 10 points. 
The references at the end of this document give examples for journal
articles, conference publications, book chapters, books, edited volumes, 
technical reports, and dissertations. 

Alphabetize references by the surnames of the first authors, with
single author entries preceding multiple author entries. Order
references for the same authors by year of publication, with the
earliest first.


\section*{Acknowledgments} 
 
\textbf{Do not} include acknowledgements in the initial version of
the paper submitted for blind review.

If a paper is accepted, the final camera-ready version can (and
probably should) include acknowledgements. In this case, please
place such acknowledgements in an unnumbered section at the
end of the paper. Typically, this will include thanks to reviewers
who gave useful comments, to colleagues who contributed to the ideas, 
and to funding agencies and corporate sponsors that provided financial 
support.  

% The following acknowledgements should only appear in the accepted version. 
% This document was modified from the file originally made available by
% Pat Langley and Andrea Danyluk for ICML-2K. This 2007 version was created
% by Prasad Tadepalli (tadepall@eecs.orst.edu) and is a lightly changed 
% version of the previous year's version by Andrew Moore, which was in turn 
% edited from those of Kristian Kersting and Codrina Lauth. Alex Smola 
% contributed to the algorithmic style files.  

\nocite{aha:thesis90,fisher:aaai89,jones:mlj94,langley:book95,maloof:tr98,shrager:book90}

\bibliography{blindformat}
\bibliographystyle{mlapa}

\end{document} 
