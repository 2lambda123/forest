\documentclass[11pt]{article}
        % Use font 11pt Roman.
%\documentstyle{article}

%\def\references{\newpage\section*{References}}
%\newcounter{refcount}
%\def\lit#1{~[\ref{#1}]}
%\def\cit#1#2#3#4{#1\quad#2\quad{\it #3}\quad #4\vskip.15in }
%\def\citb#1#2#3{#1\quad{\it #2}.\quad#3.}
%\def\citu#1#2{#1\quad#2.}


\newcommand{\comment}[1]{}


% No header or foot; top 2.0in, left 0.6in, right 0.9in, bottom 0.5in

%\pagestyle{empty}
%\topmargin=-.5in
%\textheight=8.8in
%\leftmargin=-1.0in
%\oddsidemargin=.25in
%\oddsidemargin=.25in
%\textwidth=6.2in

\setlength{\textheight}{22.8true cm}
%\setlength{\textwidth}{15.4true cm}
\setlength{\textwidth}{15.9true cm}
\setlength{\oddsidemargin}{.55true cm}
\setlength{\evensidemargin}{.55true cm}
\setlength{\topmargin}{-1.0cm}
\pagestyle{empty}

\newlength{\oldparindent}
\setlength{\oldparindent}{\parindent}
\setlength{\parindent}{0pt}

% No indentation at the beginning of paragraphs.

% Don't hyphenate.
%\hyphenpenalty=10000


% Rsection: a resume section environment.
\newenvironment{Rsection}[1]{
  \noindent{\bf #1}
  \begin{list}{}{\topsep=-\parskip
                 \leftmargin=-0.0in}
    \item[]
}{
  \end{list}
%  \vspace \baselineskip
}
                           
% Section: a resume section environment.
\newenvironment{Section}[1]{
  \centerline{\bf #1}
  \begin{list}{}{\topsep=-\parskip
                 \leftmargin=-0.5in}
    \item[]
}{
  \end{list}
%  \vspace \baselineskip
}

% edlist: a resume education list environment.
\newenvironment{edlist}{
  \begin{list}{}{\topsep=-\parskip
                 \itemsep=0.5\baselineskip
                 \leftmargin=0.25in}
}{
  \end{list}
}


% edlist2: a resume education list environment.
\newenvironment{edlist2}{
  \begin{list}{}{\topsep=-\parskip
                 \itemsep=0.5\baselineskip
                 \leftmargin=0.25in
                 \setlength{\rightmargin}{\leftmargin}
                 \textwidth=4in}
}{
  \end{list}
}
                           

% honorlist: a resume honor list environment.
\newenvironment{honor2list}{
  \begin{list}{}{\topsep=-\parskip
                 \itemindent=-0.in
                 \itemsep=0.0\baselineskip
                 \leftmargin=.25in}
}{
  \end{list}
}  

\newenvironment{honor2enum}{
  \begin{enumerate}{}{\topsep=-\parskip
                 \itemindent=-0.in
                 \itemsep=0.0\baselineskip
                 \leftmargin=.25in}
}{
  \end{enumerate}
}  

\newenvironment{honor3enum}{
  \begin{enumerate}{}{\topsep=-\parskip
                 \itemindent=-0in
                 \itemsep=0.0\baselineskip
                 \leftmargin=.25in}
}{
  \end{enumerate}
}  



% honorlist: a resume honor list environment.
\newenvironment{honorlist}{
  \begin{list}{}{\topsep=-\parskip
                 \itemindent=-0.in
                 \itemsep=0.2\baselineskip
                 \leftmargin=.25in}
}{
  \end{list}
}  


% joblist: a resume job list environment.
\newenvironment{joblist}{
  \begin{list}{}{\topsep=-\parskip
                 \itemsep=0.5\baselineskip
                 \leftmargin=0.5in}
}{
  \end{list}
}

                                      
% job: a job entry for a joblist environment
\newcommand{\job}[3]{
  \item[]
  \begin{tabbing}
    \hspace{-0.5in}\=\hspace{1.5in}\=\kill
    \>    {\it #1:}       \> #2   \\
    \>                    \> #3   \\
  \end{tabbing}
  \vspace{-\baselineskip}
}

              
%%%%%%%%%%%%%%%%%%%%%%%%%%%%%%%%%%%%%%%%%%%%%%%%%%%%%%%%%%%%%%%%%%%%%%%%%%%%%%%




%%%%%%%%%%%%%%%%%%%%%%%%%%%%%%%%%%%%%%%%%%%%%%%%%%%%%%%%%%%%%%%%%

%\leftmargin=-1in
\begin{document}


\centerline{\Large \bf {\LARGE \bf K}ATHLEEN {\LARGE \bf F}ISHER}
\vskip .3in

\normalsize
\it
\begin{tabular}{@{\hspace{0in}}l@{\hspace{1.4in}}l}
AT\&T Labs  & Phone: (973) 360-8675  \\
180 Park Ave., E244    & Email: {\rm kfisher@research.att.com} \\
Florham Park, NJ 07932

\end{tabular}
\rm
\vskip.3in

\begin{Rsection}{\Large \bf {Professional Preparation}}
\vskip.1in
\begin{edlist}

\item {\bf Stanford University, Stanford, California }
  \begin{honor2list}
  \item {Mathematical \& Computational Science, B.Sc. with Distinction, 1991.}  

  \end{honor2list}
\item {\bf Stanford University}
\begin{honor2list}
\item {Computer Science, Ph.D. 1996.}
\end{honor2list}
\end{edlist}
\end{Rsection}


\vskip.3in
\begin{Rsection}{\Large \bf {Appointments}}
\vskip.1in
\begin{edlist}
\item {\bf AT\&T Labs Research}\\
{\bf  Software Systems Department} \\
%\vskip .05in
\vspace{-1ex}
\begin{honor2list}
\item Senior Member of the Technical Staff, April 2002--present.
\item Principal Member of the Technical Staff, September 1996--April 2002.
\end{honor2list}


%\newpage
\end{edlist}
\end{Rsection}


\vskip.3in
\begin{Rsection}{{\Large \bf {Selected Publications}} 
}
\vskip.1in
\begin{edlist}

\item {\bf Selected papers of greatest relevance (in chronological order)}
\begin{honor2enum}
\vskip .1in

\item PADS: A Domain-Specific Language for Processing Ad Hoc Data Sources.  Kathleen Fisher and Robert Gruber. To appear in the ACM SIGPLAN Conference on Programming Language Design and Implementation, June 2005.

\item Hancock: A Language for Analyzing Transactional Data Streams.  Corinna Cortes, Kathleen Fisher, Daryl Pregibon, Anne Rogers, and Frederick Smith. 
ACM Transactions on Programming Languages and Systems, 26(2):263--300, March 2004.

\item PADS:  Processing Arbitrary Data Streams.  Kathleen Fisher and Robert Gruber. Workshop on Management and Processing of Data Streams, June 2003.

\item An Application-Specific Database.  Kathleen Fisher, Colin Goodall, Karin H\"ogstedt and Anne Rogers. 
In Proceedings of the Eighth Biennial Workshop on Data Bases and Programming Languages, September 2001.

\item Hancock: A Language for Extracting Signatures from Data Streams. Corinna Cortes, Kathleen Fisher, Daryl Pregibon, Anne Rogers, and Frederick Smith. 
In Proceedings of the Sixth International Conference on Knowledge Discovery and Data Mining, pages 9--17, August 2000.

%


\end{honor2enum}
\item {\bf Selected other publications (in chronological order):}
\begin{honor2enum}

\item 
Inheritance-Based Subtyping. Kathleen Fisher and John Reppy.
Information and Computation 177(1), August 2002.

\item
A Control-Flow Analysis for a Calculus of Concurrent Objects. Paolo di Blasio, Kathleen Fisher and Carolyn Talcott. IEEE Transactions on Software Engineering 26(7), July 2000.

\item
Extending Moby with Inheritance-Based Subtyping. Kathleen Fisher and John Reppy.  
In the European Conference on Object-Oriented Programming, pages 83--107, June 2000.

\item
A Calculus for Compiling and Linking Classes. Kathleen Fisher, John Reppy, and Jon G. Riecke.  In the Proceedings of the European Symposium on Programming, pages 135--149, March 2000.

\item
The Design of a Class Mechanism for Moby. Kathleen Fisher and John Reppy.  
In the ACM SIGPLAN Conference on Programming Language Design and Implementation, June 1999.



\end{honor2enum}

\end{edlist}
\end{Rsection}



\vskip.3in
\begin{Rsection}{\Large \bf {Synergistic Activities}}
\begin{itemize}
\vskip .1in
\item ACM SIGPLAN Vice Chair. July 2003 - June 2005.
\item ACM SIGPLAN Member at Large. July 2001 - June 2003
\item Coordinator of New Jersey Programming Language and Systems Seminar, (October 2001 -- present).
\item Program Chair for ICFP 2004.
\item Member of the Steering Committees of POPL, PLDI, ICFP, and OOPSLA.
\item I have mentored four graduate students at AT\&T: two on the PADS project and two on the Hancock project.
\item We are currently collaborating with the members of the Galax team to integrate the PADS system with their implementation of the XQuery language.
\item Member of the CRA-W Board, whose charter is to improve the situation of women researchers in computer science.  In this capacity, I organized a series of panels at the Grace Hopper Conference in 2005 devoted to women in industrial or governmental research labs.  I served as a speaker at the CRA-W's grad cohort workshop in 2004 and will again at the upcoming workshop in 2005.  I am helping to organize CRA-W's 2005 Mentoring Workshop to include material relevant to women interested in pursing a career in industry.
\item Faculty Support Chair for the Federated Conference on Research in Computing 2003.  I ran a fellowship program that permitted 50 professors at institutions with high percentages of women and minorities to attend the conferences and tutorials at FCRC.  Barbara Ryder (Rutgers) and I were awarded a grant from NSF to fund this program.

\end{itemize}
\end{Rsection}
\vskip.3in
\begin{Rsection}{\Large \bf {Collaborators and Other Affiliations}}
\begin{itemize}
\vskip .1in
\item Collaborators: 
Anne Rogers (University of Chicago),
John Reppy (University of Chicago),
Mary Fern\'andez (AT\&T Labs),
Yitzhak Mandelbaum (Princeton University),
David Walker (Princeton University),
Robert Gruber(Google),
Corinna Cortes (Google),
Daryl Pregibon (Google),
Colin Goodall (AT\&T),
Karin H\"ogstedt,
Fred Smith (Math Works),
Paolo Di Blasio,
Carolyn Talcott (Stanford),
Jon G. Riecke (Alari)

\item Graduate (Ph.D.) Advisor: 
John Mitchell (Stanford)
\item Thesis Advisees (graduated): none
\item Thesis Advisees (current): none
\end{itemize}
\end{Rsection}



% \iffalse
% \vskip.3in
% \begin{Rsection}{\Large \bf {Selected Awards and Fellowships}}
% \begin{itemize}
% \vskip .1in
% \end{itemize}
% \end{Rsection}
% \fi




%\newpage

%\newpage





%%%%%$$^{\ref{fn:second}}$







%\vskip.3in




\end{document}

