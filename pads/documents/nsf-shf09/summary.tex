%%%%%%%%%%%%%%%%%%%%%%%%%%%%%%
% \centerline{
% \begin{tabular}{cc}
%           \\[-1ex]
% \multicolumn{2}{c}{Automatic Tool Generation for Ad Hoc Scientific Data} \\
%           & \\[-1ex]
% David Walker (PI)\\
% Princeton University\\[2ex]
% \end{tabular}}
%%%%%%%%%%%%%%%%%%%%%%%%%%%%%%

\paragraph*{Intellectual Merits:} 
In every scientific discipline, researchers are digitizing their knowledge and
beginning to use computational methods to categorize,
query, filter, search, diagnose, and visualize their data.  
While this effort is leading to remarkable advances,
it is also generating enormous amounts of {\em ad hoc data}.
Ad hoc data is any data for which standard data processing tools
such as query engines, statistical packages, graphing tools, parsers, printers,
transformers or programming libraries are not readily available.
This data, which is often unpredictable, poorly documented,
filled with errors, high volume and unwieldy,
poses tremendous challenges to its users and the software
that manipulates it.  We cannot maximize the productivity of top 
computational scientists or industrial engineers
unless we can maximize the efficiency and 
accuracy with which they deal with this data.  Hence, the overall goal of
our research is to alleviate the burden, risk and confusion
associated with ad hoc data.  
%Our overall strategy is (1) to develop a
%specification language capable of precisely describing any ad hoc data
%format at an easy-to-understand, high level of abstraction and (2) to
%automatically generate useful data processing tools including
%programming libraries, a query engine, format converters and others.
We will accomplish our goal by building upon past research
on on the \pads{} data description and processing system, created by the 
PI, Senior Personnel Kathleen Fisher, their collaborators and students. 
More specifically, this team will pursue the following
new research agenda:

\begin{enumerate}
\item {\bf Language support for multi-level directory systems.}
Many ad hoc data sets are physically represented as collections of 
data files, organized in a multi-level directory system.
The PI will develop powerful, new extensions to PADS that allow programmers
to specify the structure of these multi-level directory systems
and generate libraries for programming against them.
\item {\bf Semi-automatic data description generation.}
While using the PADS data description language speeds the generation
and maintainance of data processing tools, a user must invest some effort
to learn the language syntax and semantics.  The PI will simplify
and improve the process of writing descriptions by developing 
an entirely new kind of {\em text mark-up language} that 
allows users to add simple
labels to raw data sources, and from such labels, generate full descriptions.
Such descriptions can, in turn, serve as lasting documentation or be 
used %processed by the pads compiler
to generate useful programming tools.
\item {\bf Grammatical foundations and efficient implementation.}  
Many complex, modern text and binary data formats require a very rich collection
of features in order to describe them declaratively:  
{\em full context-free grammars}, the ability to bind 
parts of a data source to {\em variables} for later use in
{\em constraints}, {\em parameters} on non-terminals and the
ability to include pre-defined libraries as {\em black boxes} 
within a description.  The PI will develop a new grammatical 
theory that explains how to implement this rich collection of features
efficiently.  
\item {\bf Automatic generation of new data processing tools.} 
The PI will use the newly developed theory and implementation techniques 
to build important new tools including description-directed 
data generators for use in testing data processing infrastructure
and a multi-source, ad hoc data query engine.
\end{enumerate}

\paragraph*{Broader Impacts:}  We are collaborating with Kathleen Fisher
(Senior Personnel for this proposal and Senior Researcher at AT\&T)
who will be able to use our tools to address real problems at AT\&T such
as network log monitoring and 
telephone fraud detection.  
%We will use such key industrial problems
%to drive our research agenda throughout the course of the grant.
In addition to working on technology transfer to industry, 
our tools will be freely
available to researchers and scientists over the web.  Moreover, part
of our mission will be to work with computational biologists and
physicists at Princeton and the broader community to help them with
their data processing needs.  One way that we plan to do so is to
develop interdisciplinary research projects for Princeton's 
undergraduates.  As independent research co-ordinator for the Princeton
Computer Science Department
for the last two years, PI Walker has the experience and track record
to develop a highly effective undergraduate research program in this area.

