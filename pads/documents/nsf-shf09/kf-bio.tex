\documentclass[11pt]{article}
        % Use font 11pt Roman.
%\documentstyle{article}

%\def\references{\newpage\section*{References}}
%\newcounter{refcount}
%\def\lit#1{~[\ref{#1}]}
%\def\cit#1#2#3#4{#1\quad#2\quad{\it #3}\quad #4\vskip.15in }
%\def\citb#1#2#3{#1\quad{\it #2}.\quad#3.}
%\def\citu#1#2{#1\quad#2.}


\newcommand{\comment}[1]{}


% No header or foot; top 2.0in, left 0.6in, right 0.9in, bottom 0.5in

%\pagestyle{empty}
%\topmargin=-.5in
%\textheight=8.8in
%\leftmargin=-1.0in
%\oddsidemargin=.25in
%\oddsidemargin=.25in
%\textwidth=6.2in

\setlength{\textheight}{22.8true cm}
%\setlength{\textwidth}{15.4true cm}
\setlength{\textwidth}{15.9true cm}
\setlength{\oddsidemargin}{.55true cm}
\setlength{\evensidemargin}{.55true cm}
\setlength{\topmargin}{-1.0cm}
\pagestyle{empty}

\newlength{\oldparindent}
\setlength{\oldparindent}{\parindent}
\setlength{\parindent}{0pt}

% No indentation at the beginning of paragraphs.

% Don't hyphenate.
%\hyphenpenalty=10000


% Rsection: a resume section environment.
\newenvironment{Rsection}[1]{
  \noindent{\bf #1}
  \begin{list}{}{\topsep=-\parskip
                 \leftmargin=-0.0in}
    \item[]
}{
  \end{list}
%  \vspace \baselineskip
}
                           
% Section: a resume section environment.
\newenvironment{Section}[1]{
  \centerline{\bf #1}
  \begin{list}{}{\topsep=-\parskip
                 \leftmargin=-0.5in}
    \item[]
}{
  \end{list}
%  \vspace \baselineskip
}

% edlist: a resume education list environment.
\newenvironment{edlist}{
  \begin{list}{}{\topsep=-\parskip
                 \itemsep=0.5\baselineskip
                 \leftmargin=0.25in}
}{
  \end{list}
}


% edlist2: a resume education list environment.
\newenvironment{edlist2}{
  \begin{list}{}{\topsep=-\parskip
                 \itemsep=0.5\baselineskip
                 \leftmargin=0.25in
                 \setlength{\rightmargin}{\leftmargin}
                 \textwidth=4in}
}{
  \end{list}
}
                           

% honorlist: a resume honor list environment.
\newenvironment{honor2list}{
  \begin{list}{}{\topsep=-\parskip
                 \itemindent=-0.in
                 \itemsep=0.0\baselineskip
                 \leftmargin=.25in}
}{
  \end{list}
}  

\newenvironment{honor2enum}{
  \begin{enumerate}{}{\topsep=-\parskip
                 \itemindent=-0.in
                 \itemsep=0.0\baselineskip
                 \leftmargin=.25in}
}{
  \end{enumerate}
}  

\newenvironment{honor3enum}{
  \begin{enumerate}{}{\topsep=-\parskip
                 \itemindent=-0in
                 \itemsep=0.0\baselineskip
                 \leftmargin=.25in}
}{
  \end{enumerate}
}  



% honorlist: a resume honor list environment.
\newenvironment{honorlist}{
  \begin{list}{}{\topsep=-\parskip
                 \itemindent=-0.in
                 \itemsep=0.2\baselineskip
                 \leftmargin=.25in}
}{
  \end{list}
}  


% joblist: a resume job list environment.
\newenvironment{joblist}{
  \begin{list}{}{\topsep=-\parskip
                 \itemsep=0.5\baselineskip
                 \leftmargin=0.5in}
}{
  \end{list}
}

                                      
% job: a job entry for a joblist environment
\newcommand{\job}[3]{
  \item[]
  \begin{tabbing}
    \hspace{-0.5in}\=\hspace{1.5in}\=\kill
    \>    {\it #1:}       \> #2   \\
    \>                    \> #3   \\
  \end{tabbing}
  \vspace{-\baselineskip}
}

              
%%%%%%%%%%%%%%%%%%%%%%%%%%%%%%%%%%%%%%%%%%%%%%%%%%%%%%%%%%%%%%%%%%%%%%%%%%%%%%%




%%%%%%%%%%%%%%%%%%%%%%%%%%%%%%%%%%%%%%%%%%%%%%%%%%%%%%%%%%%%%%%%%

%\leftmargin=-1in
\begin{document}


\centerline{\Large \bf {\LARGE \bf K}ATHLEEN {\LARGE \bf F}ISHER}
\vskip .3in

\normalsize
\it
\begin{tabular}{@{\hspace{0in}}l@{\hspace{2.4in}}l}
AT\&T Labs  & Phone: (973) 610-2669  \\
1 River Oaks Place    & Email: {\rm kfisher@research.att.com} \\
San Jose, CA 95134

\end{tabular}
\rm
\vskip.3in

\begin{Rsection}{\Large \bf {Professional Preparation}}
\vskip.1in
\begin{edlist}

\item {\bf Stanford University, Stanford, California }
  \begin{honor2list}
  \item {Mathematical \& Computational Science, B.Sc. with Distinction, 1991.}  

  \end{honor2list}
\item {\bf Stanford University}
\begin{honor2list}
\item {Computer Science, Ph.D. 1996.}
\end{honor2list}
\end{edlist}
\end{Rsection}


\vskip.3in
\begin{Rsection}{\Large \bf {Appointments}}
\vskip.1in
\begin{edlist}
\item {\bf AT\&T Labs Research}\\
{\bf  Software Systems Department} \\
%\vskip .05in
\vspace{-1ex}
\begin{honor2list}
\item Principal Member of the Technical Staff, April 2002--present.
\item Senior Member of the Technical Staff, September 1996--April 2002.
\end{honor2list}


%\newpage
\end{edlist}
\end{Rsection}


\vskip.3in
\begin{Rsection}{{\Large \bf {Selected Publications}} 
}
\vskip.1in
\begin{edlist}

\item {\bf Selected papers of greatest relevance (in chronological order)}
\begin{honor2enum}
\vskip .1in

\bibitem{pads:wasl}
K.~Fisher, D.~Walker, and K.~Q. Zhu.
\newblock Incremental learning of system log formats.
\newblock In {\em Workshop on the Analysis of System Logs}, 2009.

\bibitem{fisher+:dirttoshovels}
K.~Fisher, D.~Walker, K.~Q. Zhu, and P.~White.
\newblock From dirt to shovels: Fully automatic tool generation from ad hoc
  data.
\newblock In {\em {ACM} Symposium on Principles of Programming Languages},
  pages 421--434. {ACM} Press, Jan. 2008.

\bibitem{mandelbaum+:pads-ml}
Y.~Mandelbaum, K.~Fisher, D.~Walker, M.~Fernandez, and A.~Gleyzer.
\newblock {PADS/ML}: {A} functional data description language.
\newblock In {\em {ACM} Symposium on Principles of Programming Languages}.
  {ACM} Press, Jan. 2007.

\bibitem{fisher+:popl06}
K.~Fisher, Y.~Mandelbaum, and D.~Walker.
\newblock The next 700 data description languages.
\newblock In {\em {ACM} Symposium on Principles of Programming Languages}.
  {ACM} Press, Jan. 2006.

\bibitem{fisher+:pads}
K.~Fisher and R.~Gruber.
\newblock {PADS}: {A} domain specific language for processing ad hoc data.
\newblock In {\em {ACM} Conference on Programming Language Design and
  Implementation}, pages 295--304. {ACM} Press, June 2005.

%


\end{honor2enum}
\item {\bf Selected other publications (in chronological order):}
\begin{honor2enum}
\bibitem{signatures}
N.~Ramsey, K.~Fisher, and P.~Govereau.
\newblock An expressive language of signatures.
\newblock In {\em {ACM} International Conference on Functional Programming},
  pages 27--40. {ACM} Press, 2005.

\bibitem{hancock:toplas}
C.~Cortes, K.~Fisher, D.~Pregibon, A.~Rogers, and F.~Smith.
\newblock Hancock: {A} language for analyzing transactional data streams.
\newblock {\em {ACM} Transactions on Progamming Languages and Systems},
  26(2):301--338, March 2004.

\bibitem{class-types}
K.~Fisher and J.~Reppy.
\newblock Inheritance-based subtyping.
\newblock {\em Information and Computation}, 177(1):28--55, Aug. 2002.

\bibitem{kdd00}
C.~Cortes, K.~Fisher, D.~Pregibon, A.~Rogers, and F.~Smith.
\newblock Hancock: A language for extracting signatures from data streams.
\newblock In {\em {ACM} International Conference on Knowledge Discovery and
  Data Mining}, pages 9--17, August 2000.


\bibitem{fisher-reppy-pldi99}
K.~Fisher and J.~Reppy.
\newblock The design of a class mechanism for {\sc moby}.
\newblock In {\em {ACM} Conference on Programming Language Design and
  Implementation}, pages 37--49. {ACM} Press, 1999.



\end{honor2enum}

\end{edlist}
\end{Rsection}



\vskip.3in
\begin{Rsection}{\Large \bf {Synergistic Activities}}
\begin{itemize}
\vskip .1in
\item ACM SIGPLAN Chair (2007 - 2009).  Vice Chair (2003 - 2007).  Member at large (2001 - 2003).
\item Elected member of CRA Board.  (2009 - present).
\item Editor, Journal of Functional Programming (2005 - present).
\item Program Chair for ICFP 2004, CUFP (2006,2007), FOOL (2001).
\item Member of the Steering Committees of POPL, PLDI, and OOPSLA.
\item I have mentored six graduate students at AT\&T: four on the PADS project and two on the Hancock project.
\item I co-organized and led the 2008 SIGPLAN Workshop on Undergraduate Programming Language Curriculum, which led to a published report describing recommended practices.  
\item Co-Chair of CRA-W, CRA's standing committee charged with increasing the representation of women in research roles in computer science.  As part of my work on this committee, I serve as a speaker at the annual Grad Cohort Conference, which brings together first, second, and third-year female graduate students to provide information about how to succeed in graduate school and beyond.
\item Faculty Support Chair for the Federated Conference on Research in Computing 2003.  I ran a fellowship program that permitted 50 professors at institutions with high percentages of women and minorities to attend the conferences and tutorials at FCRC.  Barbara Ryder (Rutgers) and I were awarded a grant from NSF to fund this program.

\end{itemize}
\end{Rsection}
\vskip.3in
\begin{Rsection}{\Large \bf {Collaborators and Other Affiliations}}
\begin{itemize}
\vskip .1in
\item Collaborators: 
Mary Fern\'andez (AT\&T Labs),
Yitzhak Mandelbaum (AT\&T Labs),
David Walker (Princeton University),
Robert Gruber(Google),
Alex Aiken (Stanford University),
Peter Hawkins (Stanford University),
Norman Ramsey (Harvard University),
Paul Govereau (Harvard University),
Mooly Sagiv (Tel Aviv University),
Kenny Zhu (Shanghai Jiao Tong University),
Peter White (Galois),
Michael Burke (Galois),
Artem Glyzer (Princeton University)


\item Graduate (Ph.D.) Advisor: 
John Mitchell (Stanford)
\item Thesis Advisees (graduated): Pascal Peres (Stanford MSC, 2007)
\item Thesis Advisees (current): none
\end{itemize}
\end{Rsection}

%wasl: \cite{pads:wasl}
%dirt to shovels: \cite{fisher+:dirttoshovels},
%pads-ml: \cite{mandelbaum+:pads-ml}
%pads-theory: \cite{fisher+:popl06}
%pads-orig: \cite{fisher+:pads},

%signatures: \cite{signatures}
%moby: \cite{fisher-reppy-pldi99}
%hancock-kdd: \cite{kdd00}
%moby-classtypes : \cite{class-types}
%hancock-toplas : \cite{hancock:toplas}


%\bibliographystyle{abbrv}
%\bibliography{kfisher}

% \iffalse
% \vskip.3in
% \begin{Rsection}{\Large \bf {Selected Awards and Fellowships}}
% \begin{itemize}
% \vskip .1in
% \end{itemize}
% \end{Rsection}
% \fi




%\newpage

%\newpage





%%%%%$$^{\ref{fn:second}}$







%\vskip.3in




\end{document}

