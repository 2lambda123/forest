\section{Related Work}
\label{sec:related}

The primary purpose of this article is to develop a semantic theory
for type-based data description languages.  To the best of our
knowledge, there is no other comparable semantic theory for this
family of languages.  Existing theories of regular expressions,
context-free grammars, parsing expression
grammars~\cite{birman+:parsing,ford:pegs,ford:packrat,grimm:packrat}
or even context-sensitive grammars specify what strings can be
recognized by a grammar, but such a specification only captures half
of the semantics of languages like \pads{} or \packettypes{}.  Our new
theory gives a complete explanation of these languages both in terms
of the strings that are recognized and the properties of internal
datastructures that are generated.  That said, in the following
paragraphs, we compare and contrast our semantics and the design
of data description languages like \pads{} to more
traditional
grammar-based parser generators as well as other related technologies
such as parser combinator libraries, 
type-directed programming techniques, and \xml{}-based tools.
% we com
%the formal
%specification of the parsing dimension of the data description
%languages allows us to more easily compare data description languages
%to related programming technologies ranging from traditional
%grammar-based parser generators to parser combinator libraries to
%type-directed programming techniques.  
%In the following paragraphs, we
%compare and contrast these technologies with languages like \pads{}
%when it comes to the goal of managing ad hoc data.

% This is the first work to show how to interpret a 
% language of dependent types, as opposed to context free grammars or
% regular expressions, as type-safe parsers.  A critical component
% of our work is the fact that the primary 
% semantics of types is transformational:
% A type describes more than a set of strings on disk; it describes
% a function from strings to well-typed
% datastructures satisfying a strong canonical 
% forms property.  Notwithstanding our focus on semantic theory, 
% we give an overview of research in both theory and implementation
% in the following paragraphs.

\paragraph*{Grammar-based Parser Generators}

Some of the oldest tools for describing data formats are parser
generators for compiler construction such as \lex{} and \yacc{}. While
excellent for parsing programming languages, \lex{} and \yacc{} are
too heavyweight for parsing many of the simpler ad hoc data formats
that arise in areas like networking, the computational sciences and
finance. The user must learn both the lexer generator and the parser
generator, and then specify the lexer and the parser separately, in
addition to the glue code to use them together.  Moreover, \lex{}
and \yacc{} do not support data-dependent parsing, do not generate
internal representations automatically, and do not supply a collection
of value-added tools.  Consequently, in our experience,
programmers simply do not use tools such as \lex{} and \yacc{}
for managing ad hoc data.

More modern parser generators alleviate several of the
problems of \lex{} and \yacc{} by providing more built-in programming
support.  For instance, the \antlr{} parser generator~\cite{antlr} allows
the user to add annotations to a grammar to direct construction of a
parse tree. However, all nodes in the abstract syntax tree have a 
single type, hence the guidance is rather crude when compared with
the richly-typed structures that can be constructed using
typed languages such as \padsc{}, \padsml{}, \datascript{} or \ddc. 
The \sablecc\ compiler construction
tool~\cite{sablecc} goes beyond
\antlr{} by producing LALR(1) parsers along with richly-typed ASTs
quite similar to those of \padsc{}. Also like \padsc{} or \padsml{}, 
descriptions
do not contain actions. Instead, actions are only performed on the
generated ASTs. \demeter{}~\cite{lieberherr+:class-dictionaries} is another parser
generator in the same general tradition as Lex, Yacc, \antlr\ and
\sablecc{} in that it is based on context-free grammars.  However,
\demeter{}'s class dictionaries are even more powerful than
previous systems as they
automatically generate visitor functions that traverse the internal
representation of parsed data.

Despite their many benefits, 
all of the context-free grammar-based tools --- \lex{}, \yacc, 
\antlr, \sablecc, and \demeter{} --- have some deficiencies when compared with
tools built on the type theory described by \ddc. 
In particular, none of them include dependent or 
polymorphic data descriptions directly in their specification language
(though some forms of dependency can be ``hacked,'' at least in
\lex\ and \yacc, by programming arbitrary host language 
code in the semantic actions).  Moreover, while the semantics of
context-free grammars are obviously well understood, the semantics of the
tools themselves, including the semantic actions that generate 
internal data structures, have not been as thoroughly studied.
For instance, we know of no proof that \antlr{}- or 
\sablecc{}-generated parsers are type safe.
Finally, the error handling strategies for conventional
parser generators are different than
those of the \pads{} languages.  Traditional parsers 
do not provide the programmer with
programmatic access to errors, as \padsml{} or \padsc{} do through the
use of their parse
descriptors.  That said, such a laundry list of technical differences risks
obscuring the essential points -- that these tools are based on a 
completely different semantic foundation and have a far different
overall ``look and feel.'' 
%The type-based languages
%make it  generate tools
%specificly suited to processing ad hoc data (both binary and ASCII) 
%whereas the others generate tools suited to the processing and
%analysis of programs.


\paragraph*{Parsing Theory}
To the best of our knowledge, our work on \ddc{} is the first to
provide a formal interpretation of dependent types as parsers and to
study the properties of these parsers including error correctness and
type safety. Of course, there are other formalisms for defining
parsers, most famously, regular expressions and contex-free grammars.
In terms of recognition power, these formalisms differ from our type
theory in that they have nondeterministic choice, but do not have
dependency or constraints.  We have found that dependency and
constraints are absolutely essential for describing many of the ad hoc
data sources we have studied, particularly binary formats in which
length fields are used pervasively.  Perhaps more importantly though, unlike
standard theories of context-free grammars, we do not treat our type
theory merely as a recognizer for a collection of strings.  Our
type-based descriptions define {\em both} external data formats {\em
  and} rich invariants on the internal parsed data structures.  This
dual interpretation of types lies at the heart of tools such as \pads,
\datascript{} and \packettypes{}.
% %\pads{} programmers, for instance, demand that
% %representations produced by their \pads{} parsers have the expected type and 
% %count on the fact that the associated PD is accurately correlated
% %with the representation.  
% %Existing formalisms simply do not address
% %this elements of data description languages.

Parsing Expression Grammars (PEGs) form the basis for yet another
class of parsers. This formalism was studied in the early
70s~\cite{birman+:parsing} and was revitalized more recently by
Ford~\cite{ford:pegs}.  Like the \ddc{}, PEGs are notable for
having greedy, prioritized 
choice as opposed to the nondeterministic choice found in regular
expressions or context-free grammars.  Greedy, prioritized choice
resolves ambiguities that would otherwise arise essentially
by defining them away.  PEGs also have syntactic lookahead
operators and may be parsed in linear time through the use of
``packrat parsing'' techniques~\cite{ford:packrat,grimm:packrat}.
Once again, however, the multiple interpretations of types in \ddc{} makes our
theory substantially different from the theory of PEGs.


\paragraph*{Parser Combinators}
% There are many parallels between \ddc{} and {\it parser
% combinators}~\cite{burge:parser-combinators,hutton+:parser-combinators}. 
% In particular, \ddc{}'s dependent sum construct is 
% reminiscent of the bind operator in the monadic formulation of parser
% combinators.  Indeed, we can model \ddc{}'s dependent sums in Haskell as 
% follows.
% \begin{code}
% \mbox{}
% sigma :: P s -> (s->P t) -> P (s,t)
% sigma m q = do \{x <- m; y <- q x; return (x,y)\}
% \mbox{}
% \end{code}%
% \noindent
% However, there are a number of deeper differences between parser
% combinators and \ddc{} descriptions:  

% \begin{itemize}
% \item As a language of types,
% \ddc{}, and related systems such as \pads{} and \packettypes, exploit
% programmer intuitions concerning the meaning of types directly.  
% This makes \pads{} a good fit for users that have not been exposed to
% combinator libraries before.
% \item \ddc{}, with its parse descriptors, 
% has quite a different error reporting mechanism from
% parser combinator libraries.  When processing ad hoc data,
% the errors are often the most interesting part.  \ddc{} and \pads{}
% give direct, programmatic access to error representations.
% \item The multi-faceted, nonstandard 
% semantics of dependent \ddc{} types is structured
% entirely differently from the semantics of parser generators given
% in the literature.
% %\item A parser combinator library is specifying a \textit{parser},
% %  while a term in \ddc{} is \textit{describing a data format}, which
% %  means that the \ddc{} term can be used to generate a printer and
% %  other analysis tools in addition to a parser.
% \end{itemize}

Of all parsing technologies, the \ddc{} most closely resembles
libraries of functional parser combinators, which have been extensively 
studied in the literature, dating back at least as early as
1975~\cite{burge:parser-combinators}. In particular, the parsing
semantics of the \ddc{} could rather easily be redefined in terms of 
monadic parser combinators, like those of the popular Parsec
library~\cite{leijen+:parsec}. Indeed, Oury
\etal{}~\cite{oury+:power-of-pi} have presented a reformulation of our
theory along
these lines by embedding \ddc{} into the dependently-typed
programming language Agda.
However, 
%this similarity between \ddc{} and parser combinators in no
%ways detracts from the significance of \ddc{}. 
an essential feature of
\ddc{} that distinguishes it from parser combinator libraries is its
simultaneous interperetation of type declarations 
as parsers and as internal
representation types.  Moreover, this dual semantics has quite an
impact on the user experience ---
the ``look and feel'' of \ddc{}, and
related systems such as \pads{} and \packettypes{},
is quite different from Parsec, for instance, 
because these systems exploit programmer
intuitions concerning the meaning of types directly.  This makes such
languages a good fit for users that have not been exposed to
combinator libraries before.

%% Moreover, the purpose of the semantics presented in this paper is not
%% to propose a new parsing technique, but to understand and explain the
%% workings of existing data description languages. Indeed, the
%% similarity between \ddc{} parsers and parser combinators provides a
%% convenient basis on which to compare and contrast the parsing
%% techniques in data description languages with other approaches. We
%% briefly undertake such a comparison now.

Despite this principal difference, we believe it is important to compare and
contrast \ddc{} with the literature on parser combinators in some depth.
To do so, we
begin by noting the salient distinguishing characteristics of many
parser combinator libraries and then note where \ddc{} fits with
regard to these characteristics.
\begin{enumerate}
\item Are alternatives explored depth-first or breadth-first?
\item How much lookahead is supported?
\item What are the semantics of choice?
\item Does the algorithm support ambiguity?
\item Does the algorithm support left recursion?
\item How does the parser handle errors in the input?
\item Does the library support context-sensitive parsing?
\end{enumerate}

The first two questions are closely related, because lookahead is often integrated with the parsing process by speculatively continuing parsing at each branch point. Therefore, the question becomes: is lookahead performed breadth-first or depth-first? \ddc{} uses a depth first approach to parsing alternatives -- it tries the branches of each alternative in order, choosing the first branch to parse succesfully. Therefore, it supports unlimited lookahead, because each branch can consume arbitrary quantities of input. The depth-first approach to alternatives is quite common in parser combinators~\cite{wadler:failure-successes,hutton:higher-order-parsing,hutton+:monadic-parsing,fokker:functional-parsers}. Some libraries support a combination of these approaches. Parsec, for example, employs a breadth-first, single-token lookahead as standard, but also allows explicit invocation of depth-first, arbitrary-length lookahead through the \cd{try} combinator~\cite{leijen+:parsec}. Swierstra \etal{} explore more sophisticated breadth-first parsing, based on continuations, in a series of papers~\cite{swierstra+:fast-error-correcting,swierstra:toys-parsing,hughes+:polish-parsing}.

The third question to consider is the semantics of the choice operator. In \ddc{}, the choice operator is deterministic and greedy: it accepts the first branch that succeeds, even if accepting a later branch might ultimately lead to a longer total parse. While this form of choice is limiting, it reflects the reality of what is supported by existing data description languages. 
This semantics is a common choice in parser combinator libraries.  
For example \cd{try p <|> q} in Parsec and \cd{p +++ q} in Hutton and 
Meijer's combinators~\cite{hutton+:monadic-parsing}, both behave like 
\ddc{}'s choice operator.
%Other approaches support both deterministic and nondeterministic choice. 

Regarding the question of support for ambiguity, \ddc{}'s support for only deterministic choice removes the possibility of ambiguous grammars. In contrast, many of the basic parser combinator formulations since Wadler~\cite{wadler:failure-successes} support ambiguity and return all possible parses. However, given the efficiency impacts of such an approach, later work tries to limit the amount of ambiguity supported~\cite{swierstra+:deterministic-error-correcting,leijen+:parsec}, provide the user with more fine grained control over its use~\cite{hughes+:polish-parsing}, or generally improve the efficiency of ambiguous parsers~\cite{peake+:earley-cps,frost+:padl-pcombs}.

Left-recursion is not supported by most parser combinators, and the \ddc{} is no exception. This shortcoming is mitigated by the fact that most instances of left recursion can be elegantly rewritten using some form of repetition operator, like \ddc{}'s $\kwd{seq}$ type or
Fokker's $\kwd{listOf}$ and $\kwd{chainl}$ combinators~\cite{fokker:functional-parsers}. However, there are new techniques for directly supporting left recursion in parser combinators~\cite{frost+:padl-pcombs}.

Two of the essential features of \ddc{} parsers are the detailed
error reporting through parse descriptors and the robustness to
errors. The various parser combinator libraries take a variety of
approaches to error handling and reporting, none of them quite like
\ddc{}'s. The Parsec library uses predictive parsing to ensure that
when an error is encountered, it is clear exactly where in the input
the error occurred. However, this advantage comes at the cost of
requiring the grammar to be in (almost) LL(1) form. Moreover, when an
error occurs, parsing stops. Swierstra,
\etal{}~\cite{swierstra+:deterministic-error-correcting}, similarly
rely on predictive parsing to pinpoint errors, but add error
correction -- through token insertion and deletion -- to increase
parser robustness. All corrective actions are reported to the user via
error messages. Later variations of this
approach~\cite{swierstra+:fast-error-correcting,swierstra:toys-parsing}
eliminate the requirement of predictive parsing by performing a
breadth first search of all possible parses (including error
correcting parses). These combinators also go beyond earlier ones by
reporting all errors, with corresponding corrections, in a
special-purpose data structure, rather than simple strings.

The approach to error correction taken by Swierstra \etal{} is, in essence, closely related to that of \ddc{}. Both attempt to recover from errors via a combination of terminal and nonterminal insertion and terminal deletion; and both provide detailed reports as to the nature and location of errors and the corresponding corrective actions. Nevertheless, the differences are significant. First, they differ in their approach to choosing particular insertions and deletions. In \ddc{}, insertions happen implicitly and for any type -- when a parser returns data with errors and parsing continues, an insertion has implicitly occured. Deletion points, however, must be marked explicitly, through the $\kwd{scan}$ type. In contrast, Swierstra \etal{} completely automate the choice of insertion and deletion, relying, in part, on an analysis of the input grammar. The second difference lies in the nature of the error-reporting data structure. In \ddc{}, that data structure is the parse descriptor, and is specialized to the input grammar. As a result, the error data structure reflects the shape of the output data. In contrast, Swierstra \etal{} employ a single data structure for all error reporting, and relate errors to the raw input data, rather than the structured output data. This difference in error reporting is necessary, in part, because parse combinators abstract over the structure of the output data. 

The final distinguishing characteristic of parser-combinator libraries is support for context sensitive parsing. Leijen and Meijer~\cite{leijen+:parsec} distinguish between \textit{monadic-style} combinators, like Parsec and those of Hutton and Meijer~\cite{hutton+:monadic-parsing}, which support context sensitive parsing, and \textit{arrow-style} combinators, like those of Swierstra \etal{}, which do not. The conventional wisdom is that monadic-style combinators are not amenable to the analyses employed for arrow-style combinators~\cite{leijen+:parsec,swierstra+:deterministic-error-correcting}. \ddc{} is most similiar to monadic-style combinators.

%In some sense, the similarities between \ddc{} parsers and parser
%combinators is another confirmation of the elegance of parser
%combinators. They provide an effective vehicle for expressing many
%elements of the parsing semantics of data description languages.

\paragraph*{Marshalling and Unmarshalling}

Marshalling libraries such as Java's JXM library~\cite{jxm} 
allow programmers to
 serialize objects on disk in a fixed format.
Unmarshalling libraries read this fixed format back 
into memory.  Although useful for saving or otherwise communicating 
the state of a program, 
this technology does not help solve the problem of how to interpret
data that arrives in a non-standard, ad hoc format.

Languages such as
\textsc{ASDL}~\cite{asdl} and \textsc{ASN.1}~\cite{asn}
are somewhat related to marshalers.  Both of
these languages specify the {\em logical\/} in-memory representation of
data and then automatically generate a {\em physical\/} on-disk
representation.  Another language in this category is the Hierarchical Data
Format 5 (HDF5)~\cite{hdf5}.  This file format allows users to store
scientific data, but it does not help users deal with legacy ad hoc
formats.  Like marshalling tools, ASDL, ASN.1 and related technologies
do not help users who need to parse and process non-standard, ad hoc data.


\paragraph*{Type-Directed Programming}
{\em Type-directed} or
{\em generic} programming techniques~\cite{jansson+:97,jansson+:99,jansson:phdthesis,hinze+:generic-programming,jansson+:02,lammel+:syb} allow users to
define algorithms by induction over the structure of a type rather 
than by induction (or recursion) over the structure of a value.  
Of particular interest is the
elegant work by Jansson and Jeuring on polytypic data 
conversions~\cite{jansson+:97,jansson+:99,jansson:phdthesis,jansson+:02}.
These authors demonstrate how to program a variety of data 
transformation functions together with their inverses in PolyP, a 
type-directed extension of Haskell. For instance, they describe a 
generic compressing printing/parsing algorithm, a generic 
noncompressing printing/parsing algorithm, and a data
extraction algorithm that separates primitive data from its
containing structure.  

Also of interest is the work of 
van Weelden \textit{et al}~\cite{weelden+:polytypic-ast}.
They investigated the use of
type-directed programming to produce a parser for a language based only on
the specification of its AST type(s). In this way, the AST types
themselves serve as the grammar for the language. They also
investigate applying this approach to other compiler-related analyses,
like scope checking and type inference.

The parsers defined by \ddc{} are defined by induction over the structure
of types and hence may be thought of as type-directed programs.  However,
there are a number of reasons why one might prefer a domain-specific
language like \pads{} or \ddc{} over a full-blown
generic programming framework.  
From a programmer's perspective, the specialized syntax makes writing
descriptions, particularly descriptions with nested literals, 
regular expressions, functions and dependencies, relatively easy.
From an implementer's perspective, \pads{} is a
relatively simple, light-weight language extension:  Implementing a 
\pads{}-style language for any standard imperative, functional or 
object-oriented requires no changes to the underlying host language
type system or run-time.  In contrast, type-directed
programming languages normally need sophisticated, non-standard type systems
or modifications to the run-time to function correctly.  In the
paper ``Generics for the Masses''~\cite{hinze:masses}, Hinze cites these
complications as his motivation for the design of a new generic programming
environment for Haskell, but unfortunately, the new design is still
Haskell-specific, as it makes essential use of polymorphic data
structures and type classes.  


%Nevertheless,
%there are a number of important differences between \ddc{} or a
%\pads{}-style language and 

%% Likewise, the semantics of generic programming languages clearly does
%% not directly serve as a semantics for \padsc{} or \padsml{}.

%% The closest connection between \ddc{} and research in
%% type-directed programming can likely be found in the work of 
%% van Weelden \textit{et al}~\cite{weelden+:polytypic-ast}.
%% These authors investigated the use of
%% polytypic programming to produce a parser for a language based only on
%% the specification of its AST type(s). In this way, the AST types
%% themselves serve as the grammar for the language. They also
%% investigate applying this approach to other compiler-related analyses,
%% like scope checking and type inference.  However, while their
%% ``types-as-grammar'' approach is clearly related to \padsml{}, they
%% use standard (non-dependent) types as parser specifications, and they
%% study parsing techniques for programming languages, not ad hoc data.
%% Dependent types are very important in the domain of ad hoc data,
%% where it is very common for a tag early in data to determine later parsing 
%% behavior or an integer to determine the length of some future array.


\paragraph*{XML-based tools}
Rather than programming directly with data in its ad hoc format,
it may be useful to convert it first to \xml.  Once in \xml,
any one of hundreds of \xml-based tools may be used to manipulate the data.
XSugar~\cite{brabrand+:xsugar2005} is one tool that
allows users to specify an
alternative non-\xml{} syntax for \xml{} languages using a
context-free grammar.  This tool automatically generates conversion
tools between \xml{} and non-\xml{} syntax.  Another such tool is
the Binary Format
Description language (BFD)~\cite{bfd}.  BFD is able
to convert the raw binary or ASCII 
data into \xml{}-tagged data where it can then be
processed using \xml{}-processing tools.  While both these tools are
useful for many tasks, conversion to \xml{} is not always the answer.
Such conversion often results in an 8-10 times blowup in data size
over the native form.  Moreover, for programmers not familiar with
\xml, there is a high barrier to entry --- not 
only do they have to learn the ad hoc format, 
but they must also learn \xml{} and
the \xml{} conversion tool.  Altogether, this overhead is too heavy
for many simple data processing tasks.

Two other related \xml{}-based specification languages are
\dfdl~\cite{dfdl-proposal,dfdl-primer} and XDTM~\cite{xdtm,zhao+:sigmod05}.
Like \pads{} or \packettypes, \dfdl\ is a language for specifying data 
formats.  It has a rich collection of base types and supports a 
variety of ambient codings.  Early versions of \dfdl\ 
did not allow dependent constraints, but they were later added,
perhaps because \pads{} had demonstrated how effective they can be.
%% \dfdl\ is a data format specification language with an \xml-based
%% syntax and type structure~\cite{dfdl-proposal,dfdl-primer}. \dfdl\ is
%% a language \textit{specification}, not an entire system or an
%% implementation -- \padsml{} could, perhaps, serve as the basis of a
%% robust \dfdl{} implementation. Like the \padsml{} language, 
XDTM~\cite{xdtm,zhao+:sigmod05} uses \xml{} Schema to describe the
locations of a collection of sources spread across a local file system
or distributed across a network of computers.  However, XDTM has no
means of specifying the contents of files, so XDTM and \pads{} solve
complementary problems.  The METS schema~\cite{mets} is similar to XDTM as
it describes metadata for objects in a digital library, including a
hierarchy such objects.


\cut{
\paragraph*{Ad Hoc Data Description Languages}
Clearly, the most closely related language designs 
are \padsc{}~\cite{fisher+:pads},
which has data descriptions based on the 
type structure and syntax of the
C programming language,
and \padsml{}~\cite{mandelbaum+:padsml}, which has data descriptions based on
the type structure and syntax of O'Caml.  As discussed in
previous sections, \padsc{} was first developed prior to the theory
described in this paper, but then vetted and improved using the theory
as a guide.  On the other hand, \padsml{} was developed later, and
the implementation built relatively directly by transcription from the
formal inference rules.  Both languages are capable of generating
parser and printer libraries as well as a number of useful stand-alone
tools for query support, format translation, and analysis of
statistical properties.

The networking community has developed a number of domain-specific
languages that use a type-based model 
for describing data much like \padsc, \padsml, and \ddc. 
These include
\packettypes{}~\cite{sigcomm00}, \datascript{}~\cite{gpce02} and
Bro's network packet specifiers~\cite{paxson:bro}.  
These languages only handle binary
data as they are primarily aimed at packet processing applications.
As we suggested earlier in this article, \ddc{} will serve as
a useful platform for investigating and clarifying the
semantics of these languages.

% Not only are ASCII formats a
% common part of many software monitoring systems, parsing nonbinary
% data poses additional challenges because of the need to handle
% delimiter values and to express richer termination conditions on
% sequences of data.  PacketTypes and DataScript also focus exclusively
% on the parsing/printing problem, whereas our work exploits the
% declarative nature of our data descriptions to automatically generate
% other useful tools and programming libraries.


% \padsml{} -- a real, implemented data description
% language, and \ddc{} -- a formalism for understanding data description
% languages. We will begin our discussion with an overview of work
% related to the \padsml{} language, followed by an overview of the work
% related to \ddc{}.

% There is a lot of work that overlaps both directly and
% indirectly with both the goals and features of \padsml{} and \ddc{}. 
% For reference, we will begin with a review of the salient features of
% \padsml{}:
% \begin{itemize}
% \item \padsml{} is a language of polymorphic, dependent and recursive datatypes.
% \item \padsml{} is a description language, from which tools are
%   generated through the use of a compiler.
% \item \padsml{} generated multiple tools from every description.
% \item \padsml{} supports external development of new,
%   description-independent tools.
% \item \padsml{} is targeted at the functional programming community.
% \end{itemize}

%\subsection{\padsml{}}

% As we discussed in a number of places in this thesis, \padsml{}
% evolved from prior work by Fisher and Gruber on
% \padsc{}~\cite{fisher+:pads}.  For the reader's convenience, we review
% the differences between the two languages here.  First, \padsc{} is
% targeted at the \C{} language, while \padsml{} is
% targeted at the \ml{} family of languages.  Using \ml{} as the host
% language simplifies the implementation of many data processing tasks,
% like data transformation, which benefit from \ml{}'s pattern matching
% and high level of abstraction.  Second, unlike \padsc{} types,
% \padsml{} types may be parameterized by other types, resulting in more
% concise descriptions through code reuse.  ML-style datatypes and
% anonymous nested tuples also help improve the readability and
% compactness of descriptions.  Third, the generic tool architecture of
% \padsml{} delivers a number of benefits over the fixed architecture of
% \padsc{}.  In \padsc{}, all tools are generated from within the
% compiler.  Therefore, developing a new tool generator requires
% understanding and modifying the compiler.  Furthermore, the user
% selects the set of tools to generate when compiling the description.
% In \padsml{}, tool generators can be developed independent of the
% compiler and they can be developed more rapidly because the
% boilerplate code to traverse data need not be replicated for each tool
% generator. In addition, the user can choose which tools to generate
% for a given data format on a program-by-program basis. This
% flexibility is possible because tool generation is simply the
% composition of the desired generic tool modules with the traversal
% functor. A final difference between \padsc{} and \padsml is that
% \padsc{} is more mature than \padsml{}. However, we are actively developing
% \padsml{} and expect that this will only be a temporary difference.

% \padsc{} has a stable grammar, a (nearly)
% complete manual, extensive base type support, a public release, and a
% small, but growing, user base. \padsml{} has none of these. \padsc{}
% also boasts a larger set of compiler-generated tools and a more
% efficient error-recovery mechanism. In other words, \padsml{} is a
% research prototype, whereas \padsc{} is a successful tool slowly
% making its way into the mainstream. A significant challenge for future
% work will be to mature \padsml{} to be on par with \padsc, or even
% better.

}

% Related to attribute grammars? Focus there is on specifying
%   tools, not on specifying data. They seem to be a high-level language
% for specifying data transformations, specific to given description. We
% have support for specifying data formats and for developing generic
% tools. Quite orthogonal.


% Parser combinators, however, are a general approach to specifying
% recursive descent parsing, whereas we have targeted \ddc{}
% to the domain of parsing ad hoc data. This focus leads to 
% many features not found in parser combinators, including the implicit
% type/value correspondence, the error response mechanism, and 
% arrays. 
% Each of these features is as fundamental to \ddc{} as 
% dependent sums. These two approaches
% demonstrate the idea of a spectrum of domain-specificity in
% languages. The relationship between parser combinators and \ddc{} is
% like the relationship between a general purpose language and parser
% combinators themselves. That is, while parser combinators form an
% (embedded) domain-specific language, \ddc{} constructs form a language 
% that is even more domain-specific. 

% There are parallels between \padsml{} types and some of the elements
% of parser combinator libraries found in languages like
% Haskell~\cite{burge:parser-combinators,hutton+:parser-combinators}.
% Likewise, there are libraries to help programmers generate printers.
% Each of these technologies is very useful in its own domain, but
% \padsml{} is broader in its scope than each of them: a single
% \padsml{} description is sufficient to generate \textit{both} a parser
% and a printer.  And a statistical error analysis, a format debugger,
% an \xml{} translator, and in the future, a query
% engine~\cite{fernandez+:padx}, a content-based search
% engine~\cite{lv+:cbs,oh:siw}, more statistical analyses, \etc\
% Combinator libraries are not designed to generate such a range of
% artifacts from a single specification.  Indeed, the proper way to
% think about combinator libraries in relation to \padsml{} is that they
% might serve as an alternative implementation strategy for some of the
% generated tools.



% Since
% \ddc{} types are dependent, however, the 
% type-directed programming techniques currently available in languages
% like Haskell, do not apply directly.  More importantly,
% \ddc{} and related implementations like \pads{} and \datascript{},
% were designed with a specific task in mind -- parsing and 
% processing ad hoc data.  So, while type-directed programming 
% techniques might be useful for implementing
% \ddc{}, \pads{} or \datascript{}, they do not provide the same functionality
% ``off the shelf,'' nor do their semantics inform users about the intricacies
% of parsing.

% Lammel and Peyton Jones' ``scrap your
% boilerplate'' article~\cite{lammel+:syb} provides a detailed summary
% of the trade-offs between different techniques.   In addition, these
% techniques support only standard functional-programming types, whereas
% \padsml{} consists of dependent types (specialized to the domain of ad
% hoc data processing).  However, at this point, this distinction is
% more in priniciple than in practice, as we currently provide only
% minimal support for \padsml{}'s dependent type constructors in the generic tool
% interface.

% To summarize the central difference between generic programming and
% \padsml{}, generic programming is a powerful {\em implementation
%   technique} which can be used to implement many programs, including
% some of the programs generated from a \padsml{} description.
% \padsml{}, however, is a higher-level system that directly provides a
% simple and powerful user experience for analysts who need to parse,
% print, process or transform ad hoc data.

\paragraph*{Databases}
Commercial database products provide support for
parsing data in external formats so the data can be imported into
database systems, but these products typically support only a limited number of
formats.  Also, they do not expose a declarative description of the
original format for use apart from the database, and they
provide only fixed methods for coping with erroneous data.  For these reasons,
type-based data description languages are complementary to database systems.  We strongly believe that
in the future, commercial database systems could and should support a 
\pads{}-like description language that allows users to import information from
almost any format.

% As with \padsml{}, there are many parallels between \ddc{} and {\it
%   parser
%   combinators}~\cite{burge:parser-combinators,hutton+:parser-combinators}.
% In particular, \ddc{}'s dependent sum construct is reminiscent of the
% bind operator in the monadic formulation of parser combinators.
% Indeed, we can model dependent sums in Haskell as:
% \begin{code}
% sigma :: P s -> (s->P t) -> P (s,t)
% sigma m q = do \{x <- m; y <- q x; return (x,y)\}\end{code}%%\noindent
% Parser combinators, however, are a general approach to specifying
% recursive descent parsing, whereas we have targeted \ddc{} to the
% more-specific domain of parsing ad hoc data. This focus leads to many
% features not found in parser combinators, including the implicit
% type/value correspondence, the error response mechanism, and arrays.
% Each of these features is as fundamental to \ddc{} as dependent sums.
% These two approaches demonstrate the idea of a spectrum of
% domain-specificity in languages. The relationship between parser
% combinators and \ddc{} is like the relationship between a general
% purpose language and parser combinators themselves. That is, while
% parser combinators form an (embedded) domain-specific language, \ddc{}
% constructs form a language that is even more domain-specific.

% \section{Concluding Remarks}

% Ad hoc data presents its users with a great number of challenges and
% can be found in a wide variety of disciplines. The general rule seems
% to be that if an area involves some form of data processing, then
% there are ad hoc data formats to be found. The problems of ad hoc data
% processing, therefore, are not a niche interest, but an essential
% problem in computer science. Moreover, they are not likely to go away
% anytime soon. The existence of ad hoc data formats is not caused by
% the shortsightedness or inexperience of data format designers. Rather,
% new discoveries and new applications often legitimately demand new
% data formats, yet format standardization is a slow and difficult
% process.  While \xml\ is an extremely flexible and standardized
% format, it is not appropriate for all data sources, particularly very
% large ones.  For these data sources, the blow-up in data size and the
% performance hit of processing the \xml\ can make its use untenable.

% We hope that our work on data description languages, as described in
% this thesis, will make a significant contribution both to data
% analysts in need of tools like \padsml{} and to computer scientists
% eager to tackle the many challenges of ad hoc data. Our goal in
% presenting \padsml{} was not only to describe what we have accomplished,
% but to inspire and guide other researchers in building versions of
% \pads\ for their favorite programming languages. Similarly, our aim in
% presenting \ddc\ was not only to provide a semantics to a number of
% existing data description languages, but to pave the way for a clear
% understanding of the semantics of future data description
% languages. We hope that there will be many.

% However, our vision for the \padsml{} and \padsc{} languages does not
% stop there. Ultimately, we think that every data source should carry
% with it its own description. That description would be written in a
% low-level language (perhaps like \ddc), into which descriptions from
% many other, higher-level descriptions could be compiled.  Furthermore,
% going beyond the \padsc{} and \padsml{} languages themselves, we want to
% allow data consumers to access their data with high-level, intuitive
% tools that require no programming and free them to focus on their
% goals. If we can enable 1000 cancer researchers to become just 1\%
% more effective in their work, then we will have ``created'' (in terms
% of time) the equivalent of 10 new researchers. Of course, we don't
% intend to be satisfied with helping just 1000 cancer researchers.
% Given the large quantity and near ubiquity of existing ad hoc data, we
% strive to improve the data access of millions of people and for many
% years to come.

%%% Local Variables: 
%%% mode: latex
%%% TeX-master: "thesis"
%%% End: 
