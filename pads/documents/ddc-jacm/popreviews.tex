
% Reviewer's Scores
% -----------------------------------

% Evaluation: A
% Confidence: X
% Innovation: Q


% ------------------------------------------------------------------------ 
% ---
% Detailed Comments
% ------------------------------------------------------------------------ 
% ---

% A FIRST REVIEWER WROTE (BXQ)

...

% Detailed comments:

% Section 1:
%  Further, since
% there doesn't seem to be much (if any) core technical content in the 
% paper
% that is really specific to binary-oriented data description languages, 
% the
% paper really needs to be framed and presented in terms of the overall
% existing body of literature on parsing and parsing theory, rather than 
% just
% primarily with respect to the recent "ad hoc data"-oriented parsers
% (DATASCRIPT, PACKETTYPES, etc.) that the authors refer to throughout.

-- did not really address this point as it requires rewriting the whole 
paper.

% Section 1.1 - the "remarkable insight" the second paragraph talks about 
% is
% indeed important to this paper and should be pointed out, but it is by 
% no
% means as new as the wording seems to imply. Besides classic parser
% generators in the Yacc vein, whose parser definitions at least loosely
% amount to data type definitions, IDLs such as Sun RPC and CORBA, for
% example, have for many years been using specialized type declarations to
% produce both external and internal forms simultaneously (and there are
% probably much earlier examples). Type-based declarative languages 
% flexible
% enough to describe binary data representations that weren't designed for
% automatic parsing in the first place might be relatively new, however.

-- did not address

% Section 1.2, paragraph 4 - "...we require that all errors in the parsed
% data be accurately recorded..." The authors should probably say "all
% _detected_ errors" or something like that instead of just "all errors", 
% in
% this and other places the phrase "all errors" appears in the paper. The
% current wording seems to suggests some idealized definition of an
% individual "(human) error" along with some perfect error recovery 
% process
% that can detect "each individual (human) error" in an input stream and
% enumerate them all correctly and individually in the parser's output. 
% It
% is well-known that error detection and recovery is never perfect, in 
% that
% parsers often inevitably mis-place the location of an error, or generate
% multiple error reports from only one actual (human) error, etc. 
% Slightly
% re-wording the text that talks about error detection would avoid giving 
% the
% reader the (incorrect) first impression that the authors are 
% overambitious
% or naive to this fact.

the reviewer slightly misunderstands, though I admit the point is quite 
subtle and not made totally clear by us.
In the internal data structures, there are definitely no error-filled fields
unless the parse descriptor says so.
I rewrote (could not think of a more elegant way to say it):

``In a nutshell, rather than requiring input data be error-free, we require
that the internal data structures produced by parsing 
satisfy their specification whereever the parse descriptor says they
will.  Our invariant allows
data consumers to rely on the integrity of the internal data structures
marked as error-free.''

% Section 2:

% The IPADS examples presented here seem reasonable, but a few of
% the details raise red flags. For example, dates (Pdate) and IP 
% addresses
% (Pip) are given as "base types", although these types seem anything but
% fundamental or necessary "primitives" for a general-purpose parsing
% language. There may be good pragmatic reasons for including these as 
% "base
% types" in PADS, but their seemingly non-essential inclusion here in the
% description of IPADS gives the reader the false impression that there 
% might
% be something so horribly wrong with the data description language that 
% it
% is unable to describe something as structurally trivial as an IP 
% address in
% dotted-decimal ASCII notation.

-- we need these base types for the examples.  If the reader keeps reading,
they will realize the data description language is not ``horribly wrong''...
did not change.

% The examples and primitive type names used in this section also add 
% further
% confusion to the whole "text/binary" issue, which as discussed already 
% is
% much more than an issue than it should be in this paper. The name 
% Puint32,
% for example, suggests an integer having a 32-bit _binary_ 
% representation in
% its externalized form, although more careful reading of the examples 
% reveals
% that Puint32 is actually just a ASCII-based (decimal? C-style 
% multi-base?)
% representation for an integer, the "32" presumably only having 
% significance
% for the internalized C-language type the resulting parser will generate.

-- clarified

% Similarly, the name "Peor", which obviously stands for "end of record",
% seems to suggest either that the data description language depends on 
% some
% "primitive" lower-level mechanism to determine record boundaries in a 
% data
% file (e.g., record-based files in VMS), or upon noticing that "Peor" 
% really
% means "newline", that this data description language somehow only 
% supports
% data whose "records" are delimited by newlines. If "Peor" means 
% "newline",
% then just name it "Pnl" or something like that to avoid such confusion.

-- changed \Peor macro to Pnl -- no need to explain Peor and its real
semantics...

% The Pwhere clause seems to correspond exactly to the semantic predicates
% that ANTLR, JavaCC, and other existing parser generators already 
% support;
% this relationship should be acknowledged.

-- cited antlr when introduced.

% IPADS' recursive declaration notation appears at least on the surface to
% support only singly-recursive definitions, which seems like a serious 
% step
% backward from the mutual recursion that practically all other practical
% and formal grammar/parser specification languages support. I assume it 
% is
% probably the case that mutually recursive constructs can be represented 
% in
% terms of the singly-recursive Prec construct, but (a) this seems rather
% clumsy in a data description language that is supposed to be for 
% practical
% use (IPADS), and (b) the reducibility of mutual recursion to single
% recursion in the context of parser description languages is at least 
% not a
% well-known fact, as it is in the context of general programming 
% languages.
% The authors at least need to discuss this issue. The ideal would seem 
% to
% be for IPADS to have mutual recursion, for DDC _not_ to have it, and to
% describe an explicit multual recursion elimination procedure as part of 
% the
% reduction from IPADS to DDC.

-- if this was a journal article, I would respond to this seriously.  
In this conference paper, there is no time or space to respond.  
it seems useless to say ``we have not done it'' and leave it at that.
We should think about this for a Journal article though.

% Section 3:

% This is where the paper's really useful and interesting content begins.
% This is good stuff; it seems quite well thought-out for the most part, 
% and
% it takes an excellent first cut at integrating formal parsing theory 
% with
% modern formal programming language type theory. Unlike the previous
% sections, I don't have major criticisms here, only a couple minor nits.

-- good!

% In the second paragraph, the sentence "Sum-type parsers are 
% deterministic..."
% makes it clear, though only implicitly, that DDC adopts a
% "recognition-oriented" grammar paradigm similar to that of TDPL or PEGs
% (referenced later in the Related Work section) rather than the 
% conventional
% "generative" grammar paradigm embodied by CFGs and the like. It would 
% help
% avoid confusion if the paper mentioned this fact explicitly, either at 
% this
% point or even earlier, since most people are likely to assume that a
% parsing system is CFG-based unless they are told otherwise.

-- I said:

Unlike regular expressions or
context-free grammars, which allow nondeterministic choice,
sum-type parsers
are deterministic, transforming the data according to $\ty_1$
when possible and {\it only} attempting to
use $\ty_2$ if there is an error in $\ty_1$.

% The "seq" type, first described in the third paragraph of section 3.1,
% seems far too hairy and complex to be really suitable as a "primitive" 
% in
% DDC. I suspect that this construct could easily be left out entirely 
% and
% expressed in terms of the other facilities for basic pairing (dependent
% products), recursion, and syntactic and/or semantic predicates. Pure
% "syntax-only" parsing formalisms such as PEGs already have no trouble
% expressing sequences involving separators or terminators in terms of
% simpler primitive constructs; I see no reason why that should be a 
% problem
% for DDC, in which semantic predicates are available as well. There may 
% be
% good reasons to have such a primitive in a pragmatic parser generator 
% for
% performance or convenience reasons, but it seems out of place in a 
% formal
% work.

-- this is a good point.  I think/hope that the reviewer is correct,
but I do not know how to fix this right now.  Leave it for the
journal version.

% Section 4:

% The second paragraph, on "contractiveness", does not seem to get the
% attention it needs in the paper. This property seems to correspond
% more-or-less exactly to the "illegal left recursion check" that a 
% PEG-based
% parser generator would typically perform before accepting a grammar, 
% which
% in turn corresponds a bit more loosely to the left recursion elimination
% that a CFG-based parser generator might perform. The "contractiveness"
% property and its relationship to left recursion in traditional parsing
% theory should be discussed in a bit more detail.

-- it actually does not correspond to the "illegal left recursion check",
though this is something I think we should add at some point, at
least in the implementation...
-- to clarify the purpose of contractiveness, I wrote:

``To ensure that recursive types have properly-shaped parse descriptors
with a valid PD header (a condition necessary for the type safety of
generated parsers), we disallow types such as $\pmu \ptyvar
\ptyvar$. More generally, we ensure that
recursive type variables are separated from their binder by at least
one basic primitive, such as a product or sum, a condition called {\it contractiveness}.''

% Second-to-last paragraph: "Sequences have the most complicated
% semantics..." A good sign that they shouldn't really be formal 
% primitives.

-- agreed but not dealt with.

% Section 5:

% One thing that would be very nice to see in this section, or in future
% work, is a concrete condition under which a generated parser is 
% guaranteed
% to terminate. This condition would presumably depend on the
% contractiveness property discussed earlier, since the basic purpose of
% contractiveness seems to be to rule out parsers that trivially loop 
% forever
% without making progress. Since DDC grammars can contain arbitrary 
% semantic
% predicates and such, the condition would presumably also have to depend 
% on
% some assumption to the effect that each individual semantic expression
% appearing in the grammar is guaranteed to terminate. A parser generator
% obviously cannot be expected to check this second assumption 
% automatically
% - it's the ol' Halting Problem - but even a conditional termination 
% proof
% would at least nicely guarantee that if a parser fails to terminate, 
% it's
% not the fault of the basic grammar structure or parser generator.

-- I agree.

% Taking this idea further, since the pure "syntax-only" part of DDC's
% parsing model appears basically equivalent to that of TDPL and PEGs, 
% which
% can be parsed in linear time, it seems that it should similarly be 
% possible
% to establish formal performance guarantees for DDC-based grammars, given
% some suitable assumption on the runtimes of individual semantic
% expressions. A likely rule that comes to mind is that if all of the
% semantic actions in a DDC grammar are O(n^k), then it should be 
% possible to
% construct a parser that runs in O(n^(k+1)).

-- this is a great idea but I see no smooth and sufficiently
concise (!) way to extend the paper with comments on future work
such as this.

% >
% --------------------------------------------
% >

% A SECOND REVIEWER WROTE (AXQ)

% I really liked this paper, and I heartily recommend it for acceptance. 
% I hope
% that
% this work will inspire designers of other domain-specific 
% languages---and not
% just in
% the domain of data description---to consider analyzing their proposals 
% from the

% perspective of a "family" of related languages, rather than a one off 
% design.

% Here are my comments for the authors:

% 1. I am not convinced that the primary criterion for the "goodness" of 
% the
% language
% should be in terms of exactness of error correlation (Section 1.2, third
% paragraph).
% Sure, capturing a description of errors in data is an important and 
% desirable
% property
% of a language's interpretation. But how about criteria on the language 
% itself?
% For
% example, can you claim that *any* data description that can be parsed 
% without
% requiring a backtracking search can indeed be described in the language?

-- I changed it from ``good'' and ``bad'' to ``reasonable'' and
``unreasonable''...

-- our condition is quite analogous to type safety in the sense that it is a
baseline requirement that any language should have before even considering
its other merits.

% 2. The construct Popt in Figure 5 is not described in the text.

-- fixed

% 3. Section 3.1. Why use set types? It gives an impression of a 
% collection,
% whereas
% all you can ever have is either a null or a singleton (if condition is
% satisfied).

-- I think we should use the language ``constrained type.''  I attempted
to change all occurrences anywhere in the paper.  In the formal rules, 
I changed names of rules from ``Set'' to ``Con''.
-- we wind up saying ``the constraint in the constrained 
type'' a couple of places but I could not do better...

% 4. Section 3.1. "For sequences without terminators, we use bottom ...". 
% This is
% confusing, as bottom seems to carry a meaning of always failing. Why 
% would you
% not
% use unit instead, because for termination you do not rely on parse 
% failure of
% anything, but instead on the expression "e"? Or am I missing something?

-- reworded:

 The type 
$\ty_t$ is used when characters following the array will signal termination.
For example, if a semi-colon signals the end of the array, then $\ty_t$ 
should be $\pset{x}{\Pchar}{x = ';'}$. If no particular character or 
string of characters
signals the end of the array, $\ty_t$ can always be $\pfalse$.

% 5. Section 3.1. Some discussion on why these primitives are sufficient 
% (under
% some
% definition of 'sufficient') would be nice.

-- I argue that all of sections 6 addresses this point.
I do not think we need to make a further argument here.

% 6. Section 4.3. I think that your algorithm for parsing sequences is 
% certainly
% one
% reasonable way to do it, but it may not be sufficiently general. For 
% instance,
% you
% could have an array for which the total number of bits in the array is 
% known,
% but not
% the number of elements beforehand. Does the expression "e" take care 
% of this?
% At
% least it is not clear from the paper. Another commonly-used idiom is 
% that you
% union
% an untyped byte array of certain size with an array of some type but of
% unspecified
% number of elements. How would your parsing strategy for unions and 
% sequences
% deal with
% this? Similarly (and as you note as well), the strategy for parsing 
% unions may
% not be
% sufficiently general, because of the difficulty in telling apart an 
% accidental
% exact
% parse of an unintended branch, versus an erroneous match of an intended 
% branch
% because
% of error in the data.

-- no response as my previous comment above.  Really, we are setting
   up a framework -- a Landin ``family of languages'' in which one can
   understand slight variations and study such questions in depth.

% 7. Would it be possible to not "hard code" the traversal strategies for 
% unions
% and
% sequences in the DDC semantics?

-- no response.  I do not know what this means...

% 8. It would be great if somehow you could extend the calculus to allow 
% stateful
% parsing. Several protocols, e.g. ASN.1 encodings, subtantially alter the
% "grammar" of
% later data based on seeing certain values earlier on. Essentially, the 
% parser
% needs
% to have a notion of modes, and certain data values switch parser modes.

-- simple modes can be implemented simply but parameterizing all types.
IDEA:  Use monadic types as descriptions to hide the parameter passing??

% 9. For such a well-written paper, it is needlessly sloppy at places. 

-- I hate it when referees make comments like this.  We did a lot of
proofreading.  We tried our best.  So we're not perfect!  Just tell us
where the errors are. We will fix them.

% The
% authors
% should take care to fix all typos and nits. Is it J.P.Landin (Section 
% 1.2) or
% P.J.
% Landin (Section 7.3)? 

P.J. -- fixed where I noticed.

% You may want to add a reference to his article in 
% the
% biblio-
% graphy. There are several spelling mistakes, both in the main text and 
% in the
% bibliography, e.g. "Packrat" parsing.


% PC MEMBER COMMENTS:

% I have just a couple of comments to add to the reviewers' ones.
% The use of dependent types, while justified in the text does not
% look justified in the example. In section 3.3 the dependent 
% product types
% are not "dependent". Could you give a practical example in of \Sigma 
% x:s.t
% with x occurring in t?

-- weblog already does this...but I made it *even clearer* in the text

% The other point is that you often use sigleton types, i.e. { x:t | x=v 
% }.
% Wouldn't it be convinient to add singleton types (e.g. v) to your types?
% It is not just a matter of syntactic sugar but the fact that usually the
% match of singleton types can be much more effciently compiled and easily
% optimized w.r.t. your set definition (which BTW recall me a lot the list
% comprehensions.

-- efficiency is irrelevant here

-- singleton types are clearly a trivial extension. 

-- no comment

% ======================================================================== 
% ====
% REVIEWER #2
% ======================================================================== 
% ====


% Reviewer's Scores
% -----------------------------------

% Evaluation: A
% Confidence: X
% Innovation: Q


% ------------------------------------------------------------------------ 
% ---
% Detailed Comments
% ------------------------------------------------------------------------ 
% ---

% If I were allowed one complaint, albeit a small one, I would say that
% the use of dependent types does not come out so clear as the other
% parts of the paper. In particular, it is true that some dependent
% types are needed, but it is not clear how _much_ dependent type power
% you actually need in practice.

-- ... these people do not realize that every expression 
that appears in any of our examples gives rise to
a dependent type... Dell with a four

% ANOTHER REVIEWER WRITES:

% I find this work quite exciting as it provides a theoretical
% foundation not only for the PADS language developed by the authors,
% but a foundation that is reusable for anyone who is interested in
% creating a language for data description based on types; in particular
% if one wants to embed such a language in a typed functional language.

% The types described in the paper are such that they describe three
% different things: the set of valid bit strings in the external data
% format and two data structures in the host language. One that
% represents the data and one, a parse descriptor which keeps track of
% the errors found in the external data.

% I think that having all these three aspects be a part of the type
% complicates the description. It makes Section 4 harder to understand
% than it needs to be. Particularly Figure 13 would be more easily
% understandable if the parse descriptor were separated from the type.

% The upside is that this allows the authors to show that there will be
% no syntactic or semantic errors in the host type representing the
% data when the parse descriptor does not contain any errors.

% The authors note in Section 1 that this means that the data consumers
% can rely on the integrity of data marked as error free. And indeed
% this data will be type correct, but false positives can still happen.
% Suppose for example that a type contains a zero terminated string and
% that in the buggy input data the zero termination for a string is
% missing. If zeros only appear as string terminators and there is only
% one string in the type then the data will probably be considered
% correct.

% example type:


% person_t = Pstruct {
% number = Puint32; " ";
% name = PString("0");"0";
% } ;

% person_t = Parray(Peor, Peof)


% example file:


% 123456 Joe Smith0
% 234567 Eric Smith
% 345678 John Smith0EOF


% Two more points in the paper need further clarification:

% 1. What is the unit of offsets? In figure 13 the scan(t) updates the
% offset with 1. What unit is that 1? Is it bits, bytes, words or
% something else? You should mention this.

-- fixed

% 2. Is it necessary to convert the data into a canonical in memory
% representation and instead use the type to operate on the data as
% it is represented as raw data? This could be useful when one is
% only interested in a small fragment of the information contained in
% the file.

-- beyond the scope of this paper... not dealt with.

% SOME DETAILS NOTED BY THE PC MEMBER:

% - Page 2, middle and Page 3, first bullet: BLT probably refers to
% "Bit-Level Types" but this has not been introduced.

-- done

% - Sect 1.2: It is "P.J." not "J.P." !!
% (Also, you probably owe the guy a citation...)

-- done

% - Sect 2: I find the references to [11] and [12] a bit unnecessary.

-- I disagree. they are retained (harper's minml and Pierce et al.'s featherweight java are the cites)

% - Fig 5: Couldn't find the "Popt" explanation anywhere...

-- done.

% COMBINED COMMENTS FROM THE PC MEMBER AND THE SUB-REVIEWER:

% We think the paper is a clear accept. Besides being well-written, the
% paper makes an important contribution: It offers a solid theoretical
% basis on which to formalize a set of data description languages that
% so far have been developed in a rather ad hoc way. In comparison with
% the PLDI'05 paper on the PADS system, we think this paper describes
% these types of languages far more consistently. In fact, after reading
% the IPADS language given in this paper, we now understand PADS and the
% PLDI paper much better! (We hope that the first author will not read
% this as a criticism, but as a sincere compliment for the current
% paper.) Showing how the data description calculus can be used to
% encode features not found in PADS, but in other data description
% languages was also very helpful.

% To sum up by paraphrasing the paper's title, we think this paper is
% indeed useful not only for the PADS language of the authors, but for
% anyone wishing to construct any of the other 699 data description
% languages.

% ======================================================================== 
% ====
% REVIEWER #3
% ======================================================================== 
% ====


% Reviewer's Scores
% -----------------------------------

% Evaluation: A
% Confidence: X
% Innovation: Q


% ------------------------------------------------------------------------ 
% ---
% Detailed Comments
% ------------------------------------------------------------------------ 
% ---

% This paper is outstanding in a number of ways. Firstly, it is
% practical. It addresses the problem of ad hoc data by proposing a new
% class of languages, data description languages (DDLs), that can be
% used to give a high level specification of the structure of a given
% format, and simultaneously, a parser for that format. Given the
% multitude of ad hoc data sources this work fulfils a real need.

% The paper is also novel in that it comprehensively tackles the problem
% of erroneous data. (Unlike more formal data such as
% well-defined file formats and protocols, errors do not necessarily mean
% that a data source is unusable.) Another novel feature of the paper is
% the identification of the dual role types play in DDLs: they describe
% both the external (on-disk) and internal (programmatic) format of ad
% hoc data.

% The paper is admirably complete. First we are presented with an
% idealised DDL called IPADS and an informal description. A calculus for
% describing DDLs is then described: the Data Description Calculus
% (DDC). We are then shown how each of IPADS' features is 
% described in
% DDC. Owing to the nature of DDLs standard notions of correctness
% (such as "well typed programs do not go wrong") do not apply. A new
% correctness criterion is defined: all errors in an ad hoc data source
% must be accurately recorded in the parse descriptor. We are then
% presented with a denotational semantics for DDC via a translation to
% an modest extension to the simply typed polymorphic lambda calculus.
% Finally, the new correctness criterion is proven to be true for parse
% descriptors generated from DDC specifications. The techniques have
% also been applied to DDLs not created by the authors. They show how
% key features of PACKETTYPES and DATASCRIPT can be adequately described
% in DDC.

% The overall goal of the paper is to follow Landin's prescription for
% language design: to map out a formal space of language features so
% that language design becomes a principled exercise of choosing a
% well-defined point in that space. It has succeeded in this goal. In
% the process valuable insights have been gained, perhaps the most
% notable being the orthogonalisation of the "absorb" and "compute"
% features of PADS. Originally these were restricted to structure and
% union declarations but become first class citizens in DDC. The
% creation of DDC also highlighted language features worthy of deeper
% thought, such as the behaviour of the IPADS' Punion construct. (It
% represents a parser that applies one of a number of other parsers by
% searching for the first parser which succeeds on a given input.) At
% present, if a parser fails, the next one is applied. It may however
% make more sense to record the error and continue parsing with the
% first parser. The design of a new semantics is left as future work.

% The paper is also highly polished. The paper flowed well and the
% topics were presented in an order which facilitated the reader's
% understanding. Also, the use of the available space was
% economic. This reviewer was surprised at how much the authors managed
% to fit into a single paper.

% This reviewer was a little surprised that no mention of the similarity
% between the semantics of DDC and parser combinators was made. Graham
% Hutton's paper "Higher-Order Functions for Parsing" defines primitive
% parsers such as "fail" and "succeed" and parser combinators such as
% "alt" and "then" which struck me immediately as being analogous to the
% unit, bottom, sum and dependent product types of DDC.

-- still needs to be addressed.