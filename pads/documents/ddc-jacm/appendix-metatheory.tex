\appendix
\section{Extended Metatheory}
\label{app:ddc-meta-theory}
In this appendix, we formally state and prove the theorems
of \secref{sec:ddc-meta-theory}. To start, we state some basic
assumptions.  First, we assume that all variable names
introduced by the parsing semantics function come from a separate
syntactic domain from the variables that appear in ordinary
expressions. These names are therefore by definition ``fresh'' with respect
to any expressions that can be written by the user.  Second, for those
types with bound variables, the potential alpha-conversion when
performing a substitution on the type exactly parallels any
alpha-conversion of the same variable where it appears in the
translation of the type. Last, all constructors, support functions and
base-type parsers are closed with respect to user-defined variable
names.

Next, we require that \ddc{} base types satisfy the properties that we
desire to hold of the rest of the calculus.  Note that the interface $\Iopty$ specifies the types of base-type parsers.

\begin{condition}[Conditions on Base-types]
\label{cond:base-types}
  \begin{enumerate}
  \item $\dom {\Ikind} = \dom {\Iimp}$.
  \item If $\Ikind(C) = {\ity \iarrowi \kty}$ then $\Iopty(C) =
    \iarrow \ity {\kTrans[\kty,\pbase e]}$ (for any $e$ of type $\ity$).
  \item $\stsem[\Iimp(C),,\Iopty(C)]$.
    \label{cond:closed-op}
  \end{enumerate}
\end{condition}
%\edcom{M: canon is not defined yet. Either define it informally now or move it to
%  later where it is needed.}

\cut{
The evaluation of \fomega{} terms and the normalization of \ddc{}
types are both defined with a small-step semantics. However, it is
useful to be able to reason about terms and types that are related by
arbitrarily many ($k$) steps in these semantics, rather than just one.
To this end,  we define two
judgments that respectively generalize evaluation 
($\iexpn \stepstok k \iexpn'$) and normalization 
($\ty \stepstok k \ty'$) to $k$ steps: 

\[ 
\begin{array}{ll}
  \infer{\iexpn \stepstok 0 \iexpn}{} 
&
 \infer{\iexpn \stepstok {k+1} \iexpn''}
  {\iexpn \stepsto \iexpn' \quad \iexpn' \stepstok k \iexpn''}
\\[2ex]
  \infer{\ty \stepstok 0 \ty}{} \qquad
  &
  \infer{\ty \stepstok {k+1} \ty''}
  {\ty \stepsto \ty' \quad \ty' \stepstok k \ty''}
\end{array}
\]

Some useful properties of these judgments follow.

\begin{lemma}[Properties of K-step Evaluation]
  \begin{enumerate}
  \item If $\,\iexpn_1 \stepstok k \iexpn_1'$ then $\iexpn_1 \, \iexpn_2
    \stepstok k \iexpn_1' \, \iexpn_2$.
  \item If $\,\iexpn_2 \stepstok k \iexpn_2'$ then $\ivaln \, \iexpn_2
    \stepstok k \ivaln \, \iexpn_2'$.
  \item If $\,\iexpn \stepstok k \iexpn'$ then $\iexpn \, [\ity]
    \stepstok k \iexpn' \, [\ity]$.
  \item If $\, e_1 \stepstok i e_2$ and $e_2 \stepstok j e_3$
    then $e_1 \stepstok {(i+j)} e_3$.
  \end{enumerate}
\label{lemma:kleene-eval}
\end{lemma}

\begin{proof}
  By induction on the number of steps in evaluation relation.
\end{proof}

\begin{lemma}[Properties of K-step Normalization]
  \begin{enumerate}
  \item If $\,\ty_1 \stepstok k \ty_1'$ then $\ty_1 \, \ty_2
    \stepstok k \ty_1' \, \ty_2$.
  \item If $\,\ty_2 \stepstok k \ty_2'$ then $\tyval \, \ty_2
    \stepstok k \tyval \, \ty_2'$.
  \item If $\,\ty_1 \stepstok k \ty_1'$ then $\ty_1 \, \iexpn
    \stepstok k \ty_1' \, \iexpn$.
  \item If $\,\iexpn \stepstok k \iexpn'$ then $\tyval \, \iexpn
    \stepstok k \tyval \, \iexpn'$.
  \item If $\,\ty_1 \stepstok i \ty_2$ and $\ty_2 \stepstok j \ty_3$
    then $\ty_1 \stepstok {(i+j)} \ty_3$.
  \end{enumerate}
\label{lemma:kleene-norm}
\end{lemma}

\begin{proof}
  By induction on the number of steps in evaluation relation.
\end{proof}

\begin{lemma}[K-step Evaluation Inversion]
  \begin{enumerate}
  \item If $\, \iexpn_1 \, \iexpn_2 \stepstok k \ivaln$ then $k > 0$ and
    $\exists\;i,j,\ivaln_1,\ivaln_2$ \suchthat{} $\iexpn_1 \stepstok i \ivaln_1$ and $\iexpn_2 \stepstok j
      \ivaln_2$, with $i+j < k$.
  \item If $\, \iexpn\,[\ity] \stepstok k \ivaln$ then
    $\exists\;i,\ivaln'$ \suchthat{} $\iexpn \stepstok i
    \ivaln'$, with $i < k$.
  \item If $\, (\ifun {\nrm f} {\nrm x} \iexpn) \iappi \ivaln \stepstok k \ivaln'$ then
    $\iexpn[(\ifun {\nrm f} {\nrm x} \iexpn)/f][\ivaln/\ivarn] \stepstok {k-1} \ivaln'$.
  \item If $\, \ilet {\nrm \ivarn} \iexpn \, \iexpn' \stepstok k
    \ivaln$ then $\exists\;i,\ivaln'$ \suchthat{} $\iexpn
    \stepstok i \ivaln'$ with $i < k$. 
  \item If $\, \iif{\iexpn}\;\ithen {\iexpn_1}\; \ielse{\iexpn_2} \stepstok
    k \ivaln$ and $\iexpn \kstepsto \itrue$ then $\exists\;i$ \suchthat{} $\iexpn_1 \stepstok i
    \ivaln$ with $i < k$.
  \item If $\, \iif{\iexpn}\;\ithen {\iexpn_1}\; \ielse{\iexpn_2} \stepstok
    k \ivaln$ and $\iexpn \kstepsto \ifalse$ then $\exists\;i$ \suchthat{} $\iexpn_2 \stepstok i
    \ivaln$ with $i < k$.
  \end{enumerate}
\label{lemma:kleene-eval-inv}
\end{lemma}

\begin{proof}
  By induction on the number of steps in the evaluation relation.
\end{proof}

\begin{lemma}[Confluence of Evaluation]
  If $\iexpn \stepstok k \ivaln$ and $\iexpn \stepstok i \iexpn'$ then
  $\iexpn' \stepstok {k-i} \ivaln$.
  \label{lemma:eval-unique}
\end{lemma}

\begin{proof}
  By induction on the height of the first derivation, using
  determinacy of single-step evaluation as needed.
\end{proof}
} % end cut.

A number of \ddc\ properties
involve reasoning about terms that are equivalent up-to equivalent
typing annotations. We formally define this equivalence below.

\begin{definition}[Expression Equivalence]
  $\iexpn \iexpreq \iexpn'$ iff $\iexpn$ is syntactically equal to
  $\iexpn'$ modulo alpha-conversion of bound variables and equivalence
  of typing annotations.
\label{def:op-eq}  
\end{definition}

\cut{
\begin{lemma}[Properties of Expression Equivalence]
  \begin{enumerate}
  \item If $\iexpn \iexpreq \iexpn'$ and $\iexpn \stepstok k \iexpn_1$ then
    $\exists\;\iexpn_1'$ \suchthat{}
    $\iexpn' \stepstok k \iexpn_1'$ and $\iexpn_1 \iexpreq \iexpn_1'$.
  \item If $\iexpn \iexpreq \iexpn'$ then 
    $\iexpn_1[\iexpn/\ivarn] \iexpreq \iexpn_1[\iexpn'/\ivarn]$.
  \item If $\ity \equiv \ity'$ then 
    $\iexpn[\ity/\tyvar] \iexpreq \iexpn[\ity'/\tyvar]$.
  \item $\iexpn \iexpreq \iexpn$.
  \item If $\iexpn \iexpreq \iexpn'$ then $\iexpn' \iexpreq \iexpn$. 
  \item If $\iexpn \iexpreq \iexpn'$ and $\iexpn' \iexpreq \iexpn''$  then
    $\iexpn \iexpreq \iexpn''$ .
  \end{enumerate}
\label{lemma:misc-synt-eq}
\end{lemma}

\begin{proof}
  Part 1. By induction on the number of steps in the evaluation relation.
  Note that evaluation in \fomega{} is not influenced by typing
  annotations. Part 2: By induction on size of $\iexpn_1$. Part 3: By
  induction on size of $\iexpn$ and definition of expression
  equivalence. Parts 4, 5, 6: By reflexivity, symmetry and transitivity
  of expression equality and type equivalence.
\end{proof}

We will also require the following two standard
properties of \fomega\ type equivalence.

\begin{lemma}[Properties of \fomega\ Type Equivalence]
  \begin{enumerate}
  \item If $\;\wfty{\ctxt} {\ity} {\kind}$ and $\ity \equiv \ity'$ then
    $\wfty{\ctxt} {\ity'} {\kind}$. \label{lemma:fomega-eq-ty-kinding}
  \item If $\;\stsem[e,{\ctxt,x{:}\ity,\ctxt'},\ity_1]$ and $\ity \equiv
    \ity'$ then $\stsem[e,{\ctxt,x{:}\ity',\ctxt'},\ity_1]$. \label{lemma:fomega-eq-hyp-typing}
  \end{enumerate}
\label{lemma:fomega-ty-eq-props}
\end{lemma}
}

Next, we show that substitution commutes with all of the
semantic interpretations of \ddc{}. For clarity, we first introduce
two substitution-related abbreviations:
\[
\begin{array}{lll}
\xsubabbrev \ty \ptyvar & = & \xsub \ty \ptyvar\\
\zsubabbrev \ty \ptyvar & = & \zsub \ty \ptyvar
\end{array}
\]  

\begin{lemma}[Commutativity of Substitution and Semantic Interpretation]
  \begin{enumerate}
  \item $\itsem[{\ty[\ty'/\ptyvar]}] = \itsem[\ty]\xsubabbrev {\ty'} \ptyvar$.
  \item If $\ddck[\ty,\pctxt;\ctxt,\gk,]$ then $\itsem[{\ty[\ty'/\ptyvar]}] = 
    \itsem[\ty][\itsem[\ty']/\ptyvar_\repname]$.
  \item If $\ \exists\,\ity$ \suchthat\ $\itpdsem[\ty] = \ity$ and
    $\ \exists\,\ity$ \suchthat\ $\itpdsem[\ty'] \equiv {\ipty \ity}$
then $\itpdsem[{\ty[\ty'/\ptyvar]}] \equiv
    \itpdsem[\ty]\xsubabbrev {\ty'} \ptyvar = 
    \itpdsem[\ty][\itpdsemstrip[\ty]/\ptyvar_\pdbname]$.
  \item If $\ \exists\,\ity$ \suchthat\ $\itpdsem[\ty] = \ity$ and
    $\ \exists\,\ity$ \suchthat\ $\itpdsem[\ty'] \equiv {\ipty \ity}$
then $\trans[{\ty[\ty'/\ptyvar]},,] \iexpreq
    \trans[\ty,,]\zsubabbrev {\ty'} \ptyvar$.
  \item $\itsem[{\ty[\ivaln/\ivarn]}] = \itsem[\ty]$.
  \item $\itpdsem[{\ty[\ivaln/\ivarn]}] = \itpdsem[\ty]$.
  \item $\trans[{\ty[\ivaln/\ivarn]},,] = \trans[\ty,,][\ivaln/\ivarn]$.
  \end{enumerate}
\label{lemma:subst-comm}
\end{lemma}

\begin{proof}
  Parts 1,3-7: By induction on structure of types. 
% For part 3, the
%  most interesting case is for the type $\ptyvar$, which is shown in
%  detail in \appref{app:extended-proofs}.
  Part 2 is proven by induction
  on the height of the kinding derivation.  The most interesting case
  is $\pcomputen$, as it is the only construct in which a variable of
  the form $\ptyvar_\pdbname$ might appear. However, as the type is
  well-formed, we know from the kinding rules that the only type
  variables allowed in $\ity$ are of the form $\tyvar_\repname$.
  For part 4, note that variables of the form $\parsenamecd_\ptyvar$
  cannot appear in any $\ty$ -- they can only be introduced by the
  parsing semantics function. 
%A number of the more challenging cases
%  are shown in detail in \appref{app:extended-proofs}. 
  For part 7, note that the open variables in $\trans[\ty,,]$ are
  exactly those that are open in $\ty$ itself, as none are introduced
  in the translation.
\end{proof}

We also require a similar commutativity result for the 
$\kTrans[\cdot,\cdot]$ function.
%\edcom{M: Use new subst. abbreviation.}
\begin{lemma}
 If $\;\itpdsem[\ty] = \ity_1$ and
    $\itpdsem[\ty'] \equiv {\ipty {\ity_2}}$
then  $\kTrans[{\kind\xsubabbrev {\ty'} \ptyvar},{\ty[\ty'/\ptyvar]}] = 
  \kTrans[\kind,\ty]\xsubabbrev {\ty'} \ptyvar$.
\label{lemma:pt-subst-comm}
\end{lemma}

\begin{proof}
  By induction on the size of the kind, using
  \lemref{lemma:subst-comm} for $\kty$ case.
\end{proof}

\begin{lemma}
  The function $\itsem[\cdot]$ is total.
\end{lemma}
\begin{proof}
  By induction on the structure of types.
\end{proof}

We are now in a position to 
present some standard type-theoretic results for \ddc{}
kinding and normalization, as well as key substitution lemmas.

\begin{lemma}[\ddc{} Preservation]
  If $\; \ddck[\ty,,\kind,]$ and $\ty \kstepsto \tyval$ then $\ddck[\tyval,,\kind,]$.
\label{lemma:ddc-preservation}
\end{lemma}
\begin{proof}
  By induction on the kinding derivation.
\end{proof}

\begin{lemma}[\ddc{} Inversion]
  All kinding rules are invertible. That is, given a proof of any rule's conclusion we have a proof of the rule's premises.
  \label{lemma:inversion}
\end{lemma}
\begin{proof}
  By inspection of the kinding rules; in particular, the fact that they are syntax directed.
\end{proof}

\begin{lemma}[\ddc{} Canonical Forms]
  If $\; \ddck[\tyval,,\kind,]$ then either
  \begin{itemize}
  \item $\kind = \kty$, or
  \item $\kind = \ity \iarrowi \gk$ and $\tyval = \plam \ivarn {} {\ty'}$, or
  \item $\kind = \kty \iarrowi \gk$ and $\tyval = \plam \ptyvar {} {\ty'}$.
  \end{itemize}
\label{lemma:norm-canon-form}
\end{lemma}
\begin{proof}
  By kinding rules and grammar of normalized types $\tyval$.
\end{proof}

%Finally, we state the substitution lemmas that we assume to hold of
%the various underlying \fomega\ judgments followed by a substitution
%lemma for \ddc{}.
\cut{
\begin{lemma}[\fomega{} Substitution]
  \begin{enumerate}
  \item If $\;\wfd{}{\ctxt,\tyvar{::}\kty,\ctxt'}$ and $\wfty \ctxt \ity
    \kty$ then $\wfd{}{\ctxt,\ctxt'[\ity/\tyvar]}$.
  \item If $\;\wfty {\ctxt,\tyvar{::}\kty} \ity \kind$ and $\wfty \ctxt {\ity_1} \kty$ then
    $\wfty \ctxt {\ity[\ity_1/\tyvar]} \kty$.
  \item If $\;\stsem[\iexpn,{\ctxt,\tyvar{::}\kty,\ctxt'},\ity]$ and $\wfty \ctxt {\ity_1}
    \kty$ then
    $\stsem[{\iexpn[\ity_1/\tyvar]},{\ctxt,\ctxt'[\ity_1/\tyvar]}, {\ity[\ity_1/\tyvar]}]$.
  \item If $\;\stsem[\iexpn,{\ctxt,\ivarn{:}\ity'},\ity]$ and 
    $\stsem[\ivaln,\ctxt,\ity']$ then 
    $\stsem[{\iexpn[\ivaln/\ivarn]},\ctxt,\ity]$
  \end{enumerate}
  \label{lemma:fomega-subst}
\end{lemma}

\begin{proof}
  These are standard properties of \fomega{}. They are all proven by
  induction on the height of the first derivation.
\end{proof}
} % end cut

\begin{lemma}[\ddc{} Substitution]
  \begin{enumerate}
  \item If $\ddck[\ty,{\pctxt;\ctxt,\ivarn{:}\ity},\kind,]$ and 
    $\stsem[\ivaln,{\fotyc \pctxt;\ctxt},\ity]$
    then $\ddck[\ty[\ivaln/\ivarn],\pctxt;\ctxt,\kind,]$.
  \item If $\ddck[\ty,{\pctxt,\ptyvar{:}\kty;\ctxt,\ctxt'},\kind,]$ and 
    $\ddck[\ty',\pctxt;\ctxt,\kty,]$
    then $\ddck[\ty[\ty'/\ptyvar],{\pctxt;\ctxt,\ctxt'[\ty'/\ptyvar]},{\kind[\ty'/\ptyvar]},]$.
  \end{enumerate}
  \label{lemma:ty-val-subst} % deprecated
  \label{lemma:ty-ty-subst}  % deprecated
  \label{lemma:ddc-subst}
\end{lemma}
\begin{proof}
  For both parts, by induction on the first derivation, using
  standard \fomega{} substitution properties as needed.
\end{proof}

Finally, we state another commutativity property for the semantic
functions. In essence, it says that evaluation
commutes with semantic interpretation. This result
has inherent value for reasoning about \ddc{}, as it allows one to
reason about the semantics of \ddc{} functions directly in terms of
the stated normalization rules, rather than indirectly through
semantic interpretation and the evaluation/equivalence rules of the
semantic domain. Note that the premise of the lemma involves parser
evaluation because that is what is needed for later use.

\begin{lemma}[Commutativity of Evaluation and Semantic Interpretation]
  If $\; \ddck[\ty,,\kind,]$ and $\trans[\ty,,] \kstepsto \ivaln$ then
  $\exists\, \tyval$ such that
  \begin{enumerate}
  \item $\ty \kstepsto \tyval$,
  \item $\ivaln \iexpreq \trans[\tyval,,]$,
  \item $\itsem[\ty] \equiv \itsem[\tyval]$, and
  \item $\itpdsem[\ty] \equiv \itpdsem[\tyval]$.
  \end{enumerate}
\label{lemma:eval-corr}
\end{lemma}
\begin{proof}
  By induction on the number of steps in the evaluation.  Within the
  induction, we proceed using a case-by-case analysis of the possible
  structures of type $\ty$. 
% The complete proof is shown in
%  \appref{app:extended-proofs}.
\end{proof}


\subsection{Type Correctness}
Our first key theoretical result is that the various semantic
functions we have defined are coherent.  In particular, we show that
for any well-kinded \ddca{} type $\tau$, the corresponding parser is
well typed, returning a pair of the corresponding representation and
parse descriptor.

Demonstrating that generated parsers are well formed
and have the expected types is nontrivial primarily because
the generated code expects parse descriptors to have a particular shape,
and it is not completely obvious they do in the presence of polymorphism.
Hence, to prove type correctness, we first need to characterize the shape of
parse descriptors for arbitrary \ddc{} types.   
The particular shape required is that every parse descriptor be a pair
of a header and an (arbitrary) body. The most straightforward
characterization of this property is too weak to prove directly, so we
instead characterize it as a logical relation in
Definition~\ref{def:pd-props}.  Lemma~\ref{lemma:pd-log-rel}
establishes that the logical relation holds of all well-formed \ddca{}
types by induction on kinding derivations, and the desired
characterization follows as a corollary.

\begin{definition}
\label{def:pd-props}
\begin{itemize}
\item $\hhpred \ty \kty$ iff $\ \exists\,\ity$ s.t. $\itpdsem[\ty] \equiv
  {\ipty \ity}$.
\item $\hhpred \ty {\kty \iarrowi \kind}$ iff $\ \exists\,\ity$
  s.t. $\itpdsem[\ty] \equiv \ity$ and $\forall\,\ty'.\,\hhpred
  {\ty'}{\kty}$ implies $\hhpred {\papp \ty {\ty'}}{\kind}$.
\item $\hhpred \ty {\ity \iarrowi \kind}$ iff $\ \exists\,\ity'$
  s.t. $\itpdsem[\ty] \equiv \ity'$ and $\hhpred{\papp \ty e}{\kind}$
  for any expression $e$.
\end{itemize}
\end{definition}

\begin{lemma}
  If $\hhpred \ty \kty$ then $\exists \, \ity \, s.t. \itpdsem[\ty] =
  \ity$.
\label{lemma:H-prop}
\end{lemma}
\begin{proof}
  Follows immediately from definition of $\hhpred \ty \kty$.
\end{proof}

Note that we implicitly demand that $\itpdsem[\ty]$ is well defined in
the hypothesis of the lemma. We cannot assume that it is well-defined,
even for well-formed $\ty$, as that is part of what we are trying to
prove.

\begin{lemma}
  If $\itpdsem[\ty] \equiv \itpdsem[\ty']$ then $\hhpred \ty \kty$ iff $\;\hhpred {\ty'} \kty$.
\label{lemma:eq-preserve-H}
\end{lemma}
\begin{proof}
  By induction on the structure of the kind.
\end{proof}

\begin{lemma}
  If $\hhpred {\ty} \kind$ and $\hhpred {\ty'} \kty$ then $\hhpred {\ty[\ty'/\ptyvar]} \kind$.
\label{lemma:hh-subst}
\label{lemma:sub-preserve-H}
\end{lemma}
\begin{proof}
  By induction on the structure of the kind. 
% The proof is detailed in
%  \appref{app:extended-proofs}.
\end{proof}

\begin{lemma}
\label{lemma:pd-log-rel}
If $\ddck[\ty,{\pctxt;\ctxt},\kind,{}]$ then $\hhpred \ty \kind$.
\end{lemma}

\begin{proof}
  By induction on the height of the kinding derivation. 
% A number of
%  the more challenging cases are shown in \appref{app:extended-proofs}.
\end{proof}

\begin{corollary}
\label{cor:pd-props}
  \begin{itemize}
  \item If $\ddck[\ty,\pctxt;\ctxt,\kind,{}]$ then $\exists
     \ity.\itpdsem[\ty] = \ity$.
   \item If $\ddck[\ty,\pctxt;\ctxt,\kty,{}]$ then $\exists
     \ity.\itpdsem[\ty] \equiv \ipty \ity$.
  \end{itemize}
\end{corollary}

\begin{proof}
  Immediate from definition of $\hhpred \ty \kind$ and
  Lemma~\ref{lemma:pd-log-rel}.
\end{proof}

We can now prove a general result stating that if a type is well
formed, then its type interpretations will be well formed, and that
the kind of the type will correspond to the kinds of its
interpretations. We first state this correspondence formally and then
state and prove the lemma.

\begin{definition}[\ddc{} Kind Interpretation in \fomega{}]
\begin{itemize}
\item $K(\kty)        = \kty$
\item $K(\ity \iarrowi {\kind}) = K(\kind)$
\item $K(\kty \iarrowi {\kind}) = \kty \iarrowi K(\kind)$
\end{itemize}
\end{definition}

\begin{lemma}[Representation-Type Well Formedness]
If $\ddck[\ty,\pctxt;\ctxt,\kind,]$ then 
\begin{itemize}
\item $\wfty{\fortyc \pctxt}{\itsem[\ty]}{K(\kind)}$
\item $\wfty{\fopdtyc \pctxt}{\itpdsem[\ty]}{K(\kind)}$
\item If $\kind = \kty$ then $\wfty {\fopdtyc \pctxt}{\itpdsemstrip[\ty]}{\kty}$.
\end{itemize}
\label{lemma:rep-ty-well-form}
\end{lemma}

\begin{proof}
  By induction using Lemma~\ref{lemma:pd-log-rel} and properties of   \fomega{} type equality.
\end{proof}

\cut{
\begin{proof}
  By induction using Lemma~\ref{lemma:pd-log-rel} and
  \lemref{lemma:fomega-ty-eq-props}, part~\ref{lemma:fomega-eq-ty-kinding}.
\end{proof}
}
We continue by stating and proving that parsers are type correct.
However, to do so, we must first establish some typing properties of
the representation and parse-descriptor constructors, as at least one of them appears in most
parsing functions. In particular, we prove that each constructor
produces a value whose type corresponds to its namesake \ddc{}
type. For clarity, we occasionally abbreviate $\ipty \ity$ as $\pda \ity$.

\begin{lemma}[Types of Constructors]
\label{lem:types-of-constructors}
\begin{itemize}
\item $\newrepf {unit} : \iarrow \iunitty \iunitty$
\item $\newpdf  {unit} : \iarrow \ioffty {\ipty \iunitty}$
\item $\newrepf {bottom} : \iarrow \iunitty \invty$
\item $\newpdf  {bottom} : \iarrow \ioffty {\ipty \iunitty}$
\item $\newrepf {\gS} : \forall \ga,\gb.\iarrow {\iprod \ga \gb} {\iprod \ga \gb}$
\item $\newpdf {\gS} : \forall \ga,\gb. 
  \iarrow {\iprod {\pda \ga} {\pda \gb}}
  {\pda {(\pda \ga \iprodi \pda \gb)}}
$
\item $\newrepf {+left} : \forall \ga, \gb.\iarrow \ga 
                            {\isum \ga \gb}$
\item $\newrepf {+right} : \forall \ga, \gb.\iarrow \gb {\isum \ga \gb}$
\item $\newpdf {+left} : \forall \ga, \gb.\iarrow {\pda \ga} 
  {\ipty {(\isum {\pda \ga}{\pda \gb})}}$
\item $\newpdf {+right} :\forall  \ga, \gb. \iarrow {\pda \gb} 
                            {\ipty {(\isum {\pda \ga} {\pda \gb})}}$
\item $\newrepf {\&} : \forall \ga,\gb.\iarrow {\iprod \ga \gb} {\iprod \ga \gb}$
\item $\newpdf {\&} : 
\forall \ga,\gb.
  \pda \ga \iprodi
  \pda \gb \iarrowi 
         {\ipty {(\pda \ga \iprodi \pda \gb)}}
$.
\item $\newrepf {con} : \forall \ga.\iprod \iboolty \ga 
  \iarrowi {\isum \ga \ga}$
\item $\newpdf {con} :\forall  \ga. \iprod \iboolty \iarrow {\pda \ga} {\ipty {\pda \ga}}$
\item $\newrepf {seq\_init} : \forall \ga.\iarrow \iunitty {\iintty \iprodi \iseq \ga}$
\item $\newpdf {seq\_init} : \forall \ga. \iarrow \ioffty {\iapty {(\pda\ga)}}$
\item $\newrepf {seq} : \forall \ga.\iarrow
  {(\iintty \iprodi \iseq \ga) \iprodi \ga}
  {\iintty \iprodi \iseq \ga}$
\item $\newpdf {seq} :\forall  {\ga_{elt}},{\ga_{sep}}. 
  (\iapty {(\pda {\ga_{elt}})}) \iprodi
  \pda {\ga_{sep}} \iprodi 
  \pda {\ga_{elt}} \iarrowi \\
  \iapty {\pda {\ga_{elt}}}$
\item $\newrepf {compute} : \forall \ga.\iarrow \ga \ga$
\item $\newpdf {compute} : \iarrow \ioffty {\ipty \iunitty}$
\item $\newrepf {absorb} : \forall \ga.\iarrow {\pda \ga} {\isum
    \iunitty \invty}$
\item $\newpdf {absorb} :\forall  \ga. \iarrow {\pda \ga} {\ipty
    \iunitty}$
\item $\newrepf {scan} : \forall \ga.\iarrow \ga {\isum \ga \invty}$
\item $\newpdf {scan} :\forall  \ga. \iarrow {\iprod \iintty {\pda \ga}}
  {\ipty {(\isum {(\iprod \iintty {\pda \ga})} \iunitty)}}$
\item $\newrepf {scan\_err} : \forall \ga.\iarrow \iunitty {\isum \ga \invty}$
\item $\newpdf {scan\_err} :\forall  \ga. \iarrow \ioffty
  {\ipty {(\isum {(\iprod \iintty \ga)} \iunitty)}}$
\end{itemize}  
\end{lemma}

\begin{proof}
  By typing rules of \fomega.
\end{proof}

With our next lemma, we establish the type correctness of the
generated parsers. We prove the lemma using a general induction
hypothesis that applies to open types.
This hypothesis must account for the fact
that any free type variables in a \ddc{} 
type $\ty$ will become free
function variables in $\trans[\ty,,]$. To that end, 
we define the function $\ptyc \pctxt$ which maps the set of
type-variable bindings in a \ddc{} context $\pctxt$
to a corresponding set of function-variable bindings in an \fomega{} context $\ctxt$.  
\vskip -1.5ex
\[
  \ptyc{\cdot} = \cdot \qquad 
  \ptyc{\pctxt,\ptyvar{:}\kty} = \ptyc \pctxt,\codefont{\parsename_\ptyvar}{:}\kTrans[\kty,\ptyvar]
\]

\begin{lemma}[Type Correctness Lemma]
\label{lem:type-correctness}
If $\ddck[\ty,{\pctxt;\ctxt},\gk,{}]$ then
$\stsem[{\trans[\ty,,]},{\fotyc \pctxt, \ctxt,\ptyc \pctxt},
{\kTrans[\kind,\ty]}]$
\end{lemma}

\begin{proof}
  By induction on the height of the kinding derivation. 
% A number of
%  the more challenging cases are shown in \appref{app:extended-proofs}.
\end{proof}

\begin{theorem}[Type Correctness of Closed Types]
\label{thm:type-correctness}
  If $\ddck[\ty,,\gk,\con]$ then
  $\stsem[{\trans[\ty,,]},,\kTrans[\kind,\ty]]$.
\end{theorem}

\subsection{Canonical Forms}

\ddc{} parsers generate pairs of representations and parse descriptors
designed to satisfy a number of invariants.  Of greatest importance is
the fact that when the parse descriptor reports that there are no errors in a
particular substructure, the programmer can count on the
representation satisfying all of the syntactic and semantic
constraints expressed by the dependent \ddc{} type description.  When
a parse descriptor and representation satisfy these invariants and
correspond properly, we say the pair of data structures is {\em
  canonical} or in {\em canonical form}.

The canonical form for each \ddc{} type is defined via the relation $\corr \ty r p$, which
defines the canonical form of a representation $r$ and a parse
descriptor $p$ at type $\ty$.  This relation is defined for all closed types $\ty$ with base kind $\kty$.  The definition excludes
types with higher kind, such as abstractions, because such types
cannot directly produce representations and PDs.
All but one case of the relation apply to normalized types, $\tyval$. The final case is the case of an arbitrary type, $\ty$. It normalizes $\ty$ to a $\tyval$,
thereby eliminating outermost type and value applications. Then, the
requirements on $\tyval$ are given by $\corr \tyval r p$.  
\cut{
% The second definition, $\corrkl \ty r p$, defines the canonical form
% at an arbitrary type $\ty$ by first normalizing $\ty$ to eliminate the
% outermost type and value applications and then
% applying the relation $\corr \tyval r p$ at the resulting normal type
% $\tyval$.
}

For brevity in the definitions, we write $p.h.{nerr}$ as $p.{nerr}$ and use $\mathtt{pos}$ to
denote the function that returns zero when passed zero and one when
passed another natural number.


\begin{definition}[Canonical Forms]
$\corr \ty r p$ holds if and only if exactly one of the following is true:
  \begin{itemize}
  \item $\,\ty = \ptrue$ and $r = \iuval$ and $p.{nerr} = 0$.
  \item $\,\ty = \pfalse$ and $r = \ierr$ and $p.{nerr} = 1$.
  \item $\,\ty = \pbase{e}$ and $r = \iinld \ity \const$ and $p.{nerr} = 0$.
  \item $\,\ty = \pbase{e}$ and $r = \iinrd \ity \ierr$ and $p.{nerr} = 1$.
  \item $\,\ty = \psig x {\ty_1} {\ty_2}$ and $r =\ipair {r_1} {r_2}$ and $p =
    \ipair h {\ipair {p_1} {p_2}}$ 
    and $h.{nerr} = \mathtt{pos}(p_1.{nerr}) + \mathtt{pos}(p_2.{nerr})$, $\corrkl
    {\ty_1} {r_1} {p_1}$ and $\corrkl {\ty_2[(r_1,p_1)/x]} {r_2} {p_2}$.
  \item $\,\ty = \psum {\ty_1} e {\ty_2}$ and $r =\iinld {\ity}{r'}$
    and $p = \ipair h {\iinld {\ity}{p'}}$
    and $h.{nerr} = \mathtt{pos}(p'.{nerr})$ and $\corrkl
    {\ty_1} {r'} {p'}$.
  \item $\,\ty = \psum {\ty_1} e {\ty_2}$ and $r =\iinr {r'}$
    and $p = \ipair h {\iinr {p'}}$
    and $h.{nerr} = \mathtt{pos}(p'.{nerr})$ and $\corrkl
    {\ty_2} {r'} {p'}$.
  \item $\,\ty = \pand {\ty_1} {\ty_2}$, $r = \ipair {r_1} {r_2}$ and $p =
    \ipair h {\ipair {p_1}{p_2}}$, 
    and $h.{nerr} = \mathtt{pos}(p_1.{nerr}) + \mathtt{pos}(p_2.{nerr})$, 
    $\corrkl {\ty_1} {r_1} {p_1}$ and $\corrkl {\ty_2} {r_2} {p_2}$.
  \item $\,\ty = \pset x {\ty'} e$, $r = \iinld \ity {r'}$ and $p =
    \ipair h {p'}$, 
    and $h.{nerr} = \mathtt{pos}(p'.{nerr})$, $\corrkl {\ty'}{r'}{p'}$
    and $e[(r',p')/x] \kstepsto\itrue$.
  \item $\,\ty = \pset x {\ty'} e$, $r = \iinrd \ity {r'}$
    and $p = \ipair h {p'}$,
    and $h.{nerr} = 1 + \mathtt{pos}(p'.{nerr})$,
    $\corrkl {\ty'}{r'}{p'}$ and $e[(r',p')/x] \kstepsto \ifalse$.
  \item $\,\ty = \pseq {\ty_e}{\ty_s}{\pterm e {\ty_t}}$, \
    $r = \ipair {len} {\iarr{\vec {r_i}}}$, \
    $p = \itup{h,\itup{{neerr},{len},\iarr {\vec {p_i}}}}$,\\
    ${neerr} = \sum_{i=1}^{len} \mathtt{pos}(p_i.{nerr})$, 
    $\corrkl {\ty_e}
    {r_i} {p_i}$ (for $i=1 \ldots {len}$), and
    $h.{nerr} \geq \mathtt{pos}({neerr})$.
  \item $\,\ty = \pmu \ptyvar {} {\ty'}$, 
    $r = \iroll{r'}{\itsem[\pmu \ptyvar {} {\ty'}]}$, $p =
    \ipair h {\iroll{p'}{
                \itpdsem[\pmu \ptyvar {} {\ty'}]}}$, 
        $p.{nerr} = p'.{nerr}$ 
    and 
    $\corrkl {\ty'[\pmu \ptyvar {} {\ty'}/\ptyvar]} {r'} {p'}$.
  \item $\,\ty = \pcompute e \ity$ and $p.{nerr} = 0$.
  \item $\,\ty = \pabsorb {\ty'}$, $r = \iinl \iuval$, and $p.nerr = 0$.
  \item $\,\ty = \pabsorb {\ty'}$, $r = \iinr \ierr$, and $p.nerr > 0$.
  \item $\,\ty = \pscan {\ty'}$, $r =\iinl {r'}$,
      $p = \ipair h {\iinl {\ipair i {p'}}}$,
      $h.nerr = \mathtt{pos}(i) + \mathtt{pos}(p'.nerr)$, and \linebreak
      $\, \corrkl {\ty'}{r'}{p'}$.
  \item $\,\ty = \pscan {\ty'}$,
      $r =\iinr \ierr$,
      $p = \ipair h {\iinr \iuval}$, and
      $h.{nerr} = 1$.
  \item $\,\ty \not = \tyval$, $\ty \kstepsto \tyval$, and $\corr \tyval r p$.
  \end{itemize}
\end{definition}

We first prove that the representation and parse-descriptor constructors, under the appropriate
conditions, produce values in canonical form.

\begin{lemma}[Constructors Produce Values in Canonical Form]
  \label{lem:cons-props}
  \begin{itemize}
  \item $\corr \ptrue {\newrep {true} {}} {\newpd {true} \off}$.
  \item $\corr \pfalse {\newrep {false} {}} {\newpd {false} \off}$.
  \item If $\corrkl {\ty_1} {r_1} {p_1}$ and $\corrkl {\ty_2[(r_1,p_1)/x]} {r_2} {p_2}$
    then\\ $\corr {\psig x {\ty_1} {\ty_2}}
    {\newrep {\gS} {r_1,r_2}}{\newpd {\gS} {p_1,p_2}}$.
  \item If $\corrkl \ty r p$ then $\corr {\psum \ty e {\ty'}} 
      {\newrep {+left} r} {\newpd {+left} p}$.
  \item If $\corrkl \ty r p$ then $\corr {\psum {\ty'} e \ty} 
      {\newrep {+right} r} {\newpd {+right} p}$.
  \item If $\corrkl {\ty_1} {r_1} {p_1}$ and $\corrkl {\ty_2} {r_2} {p_2}$
    then\\ $\corr {\pand {\ty_1} {\ty_2}}
    {\newrep {\&} {r_1,r_2}}{\newpd {\&} {p_1,p_2}}$.
  \item If $\corrkl \ty r p$ and $e[(r,p)/x] \kstepsto c$ then\\ $\corr {\pset x \ty e} 
    {\newrep {set} {c,r}} {\newpd {set} {c,p}}$
  \item $\corr {\pseq \ty {\ty_s}{\pterm e {\ty_t}}} 
    {\newrep {seq\_init} {}} {\newpd {seq\_init} \off}$.
  \item If $\corr {\pseq \ty {\ty_s} {\pterm e {\ty_t}}} r p$ and
    $\corrkl \ty {r'} {p'}$ then, for any $p''$,\\
    $\corr {\pseq \ty {\ty_s}{\pterm e {\ty_t}}}
    {\newrep {seq} {r,r'}} {\newpd {seq} {p,p'',p'}}$.    
  \item $\corr {\pcompute e \ity} {\newrep {compute} {e}} {\newpd {compute} \off}$.
  \item $\corr {\pabsorb \ty} {\newrep {absorb} p} {\newpd {absorb} p}$.
  \item If $\corrkl {\ty} r p$ then $\corr {\pscan \ty} 
      {\newrep {scan} r} {\newpd {scan} {i,p}}$.
  \item $\corr {\pscan \ty} 
      {\newrep {scan\_err} {}} {\newpd {scan\_err} \off}$.
  \end{itemize}
\end{lemma}

\begin{proof}
  By inspection of the constructor functions. 
% \trversion{
%   Array case is most complicated, in particular proving the clause
%   $h.{nerr} \geq \mathtt{pos}({neerr})$. To do so, you must prove that
%   $H_{seq}$ maintains this invariant. The first case of the match is the
%   hard one, as ${nerr}$ is 0 (if its $1$, then it must be greater than
%   or equal to $\mathtt{pos}(n)$, for any $n$).  First, as $h.{nerr} =
%   0$, so too must $neerr$. Next, note that in this first case,
%   $h1.{nerr} = 0$. Now, the new value of ${neerr}$ is just the sum of
%   the original ${neerr}$ and $\mathtt{pos}(h1.{nerr})$, that is, $0 +
%   0$.}
\end{proof}

In addition, we require that base-type parsers produce values in canonical form:
\begin{condition}[Base Type Parsers Produce Values in Canonical Form]
  \label{cond:base-types-cf}
  If $\;\stsem[v,,\ity]$, $\Ikind(C) = \iarrow \ity \kty$ and $\Iimp(C)
  \sapp v \sapp \spair<\data,\off> \kstepsto \spair<\off',r,p>$ then
  $\corr {\pbase v} r p $.
\end{condition}

Theorem~\ref{thm:app-err-corr-at-T} is our final result. It 
states that the parsers for
well-formed types (of base kind) will produce a canonical pair of
representation and parse descriptor, if they produce anything at all.

\begin{theorem}[Parsing to Canonical Forms]
\label{thm:app-err-corr-at-T}
If $\; \ddck[\ty,,\kty,]$ and $\trans[\ty,,] \sapp \spair<B,\off> \kstepsto
  \spair<\off',r,p>$ then $\corrkl \ty r p$.
\end{theorem}

\begin{proof}
  By induction on the height of the second derivation -- that is, the
  number of steps taken to evaluate. Within the induction, we proceed
  using a case-by-case analysis of the possible structures of type
  $\ty$.  
% A number of the more challenging cases are shown in
%  \appref{app:extended-proofs}.
\end{proof}

% We conclude this section with a useful property of canonical
% representations and PDs. If the PD reports no errors, then there are
% no syntactic errors in the representation data structure {\it at any
%   level}.  We formally define {\it valid} (error-free) values next,
% followed by the statement of the property itself.

% \begin{definition}[Valid Value]
% $\noerr v$ iff exactly one of the following is true:
% \begin{itemize}
% \item $v = c$ and $c \neq \ierr$.
% \item $v = \ifun f x e$.
% \item $v = \ipair {v_1} {v_2}$ and $\noerr {v_1}$ and $\noerr {v_2}$.
% \item $v = \iinl {v'}$ and $\noerr {v'}$.
% \item $v = \iinr {v'}$ and $\noerr {v'}$.
% \item $v = \iarr{v_1 \cdots v_n}$ and $\noerr {v_i}$ for $i=1\ldots n$.
% \item $v = \iroll{v'}{\ity}$ and $\noerr {v'}$.
% \end{itemize}
% \end{definition}

% \begin{lemma}
%   If $\corrkl \ty r p$ and $p.h.nerr = 0$ then $\noerr r$.
% \end{lemma}

% \begin{proof}
%   By induction on the structure of r.
% \end{proof}


%%% Local Variables: 
%%% mode: latex
%%% TeX-master: "../thesis.tex"
%%% End: 
