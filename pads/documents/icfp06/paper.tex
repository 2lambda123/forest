%\documentclass[fleqn]{article}
\documentclass[nocopyrightspace]{sigplanconf}

\usepackage{xspace,pads,amsmath,math-cmds,
            math-envs,inference-rules,times,
            verbatim,alltt,multicol,proof,url}

\usepackage{code} 
\usepackage{epsfig}


%\setlength{\oddsidemargin}{0in}
%\setlength{\evensidemargin}{0in}
%\setlength{\textwidth}{6.5in}
%\setlength{\textheight}{8.5in}

\begin{document}
\title{PADS/ML: A Functional Data Description Language}
\authorinfo{Mary Fernandez \ \ \ \  Kathleen Fisher}{
	   AT\&T Labs Research}
       {\mono{kfisher,mff@research.att.com}}
\authorinfo{Yitzhak Mandelbaum  \ \ \ \ David Walker}{
	   Princeton University}
       {\mono{yitzhakm,dpw@CS.Princeton.EDU}}

\newcommand{\cut}[1]{}
\newcommand{\reminder}[1]{{\it #1 }}
\newcommand{\edcom}[1]{\textbf{{#1}}}
\newcommand{\poplversion}[1]{#1}
\newcommand{\trversion}[1]{}

\newcommand{\appref}[1]{Appendix~\ref{#1}}
\newcommand{\secref}[1]{Section~\ref{#1}}
\newcommand{\tblref}[1]{Table~\ref{#1}}
\newcommand{\figref}[1]{Figure~\ref{#1}}
\newcommand{\listingref}[1]{Listing~\ref{#1}}
%\newcommand{\pref}[1]{{page~\pageref{#1}}}

\newcommand{\eg}{{\em e.g.}}
\newcommand{\cf}{{\em cf.}}
\newcommand{\ie}{{\em i.e.}}
\newcommand{\etal}{{\em et al}}
\newcommand{\etc}{{\em etc.\/}}
\newcommand{\naive}{na\"{\i}ve}
\newcommand{\role}{r\^{o}le}
\newcommand{\forte}{{fort\'{e}\/}}
\newcommand{\appr}{\~{}}

\newcommand{\bftt}[1]{{\ttfamily\bfseries{}#1}}
\newcommand{\kw}[1]{\bftt{#1}}
\newcommand{\pads}{\textsc{pads}}
\newcommand{\padsc}{\textsc{pads/c}}
\newcommand{\padx}{\textsc{padx}}
\newcommand{\ipads}{\textsc{ipads}}
\newcommand{\ir}{\textsc{IR}}
\newcommand{\padsl}{\textsc{padsl}}
\newcommand{\padsml}{\textsc{pads/ml}}
%\newcommand{\padsd}{\textsc{pads/d}}
\newcommand{\learnpads}{{\textsc{learnpads}}}
\newcommand{\padsd}{\textsc{Gloves}}
\newcommand{\blt}{\textsc{blt}}
\newcommand{\ddc}{\textsc{ddc}}
\newcommand{\ddl}{\textsc{ddl}}
\newcommand{\C}{\textsc{C}}
\newcommand{\perl}{\textsc{Perl}}
\newcommand{\ml}{\textsc{ml}}
\newcommand{\smlnj}{\textsc{sml/nj}}
\newcommand{\ocaml}{\textsc{OCaml}\xspace}
\newcommand{\haskell}{\textsc{haskell}\xspace}
\newcommand{\ocamlbig}{\textsc{OCAML}\xspace}
\newcommand{\java}{\textsc{java}}
\newcommand{\xml}{\textsc{xml}}
\newcommand{\html}{\textsc{html}}
\newcommand{\xpath}{\textsc{xpath}}
\newcommand{\xquery}{\textsc{xquery}}
\newcommand{\datascript}{\textsc{datascript}}
\newcommand{\packettypes}{\textsc{packettypes}}
\newcommand{\erlang}{\textsc{Erlang}}
\newcommand{\camlp}{\cd{Camlp4}}
\newcommand{\ocamlnet}{\cd{Ocamlnet} \cd{2}}

\newcommand{\totalcost}[2]{\textsc{Cost}(#1,#2)}
\newcommand{\costdescription}[1]{\textsc{CT}(#1)}
\newcommand{\normcostdescription}{\textsc{NCT}}
\newcommand{\costdata}[2]{\textsc{CD}(#2 \; | \; #1)}
\newcommand{\acostdata}[2]{\textsc{ACD}(#2 \; | \; #1)}
\newcommand{\adc}[2]{\textsc{CD'}(#2 \; | \; #1)}
\newcommand{\cardt}{\textsc{Card}}
\newcommand{\costvar}[1]{\textsc{CV}(#1)}
\newcommand{\costchar}[1]{\textsc{CA}(#1)}
\newcommand{\coststring}[1]{\textsc{CS}(#1)}
\newcommand{\costint}[1]{\textsc{CI}(#1)}
\newcommand{\costparam}[1]{\textsc{CP}(#1)}
\newcommand{\costconst}[1]{\textsc{CC}(#1)}

\newcommand{\dibbler}{Sirius}
\newcommand{\ningaui}{Altair}
\newcommand{\darkstar}{Regulus}

\newcommand{\vizGems}{Arrakis}

\newcommand{\comon}{CoMon\xspace}
\newcommand{\planetlab}{PlanetLab\xspace}
\newcommand{\monall}{Monall\xspace}
%% \newcommand{}{}


%% \newcommand{\IParray}[4]{{\tt Parray} \; #1 \; \[#2, #3, #4\]}

\newcommand{\figHeight}[4]{\begin{figure}[tb]
	\centerline{
	            \epsfig{file=#1,height=#4}}
	\caption{#2}
	\label{#3}
	\end{figure}}

\newcommand{\myalt}{\ensuremath{\; | \;}}
\newcommand{\normal}[1]{\ensuremath{\bar{#1}}}
\newcommand{\relativee}[2]{\ensuremath{{\cal R}(#1 \; || \; #2)}}
\newcommand{\srelativee}[2]{\ensuremath{{\cal S}(#1 \; || \; #2)}}
\newcommand{\addh}[2]{\ensuremath{#1 \oplus #2}}

\newcommand{\irstruct}[1]{{\tt struct}\{#1\}}
\newcommand{\irunion}[1]{{\tt union}\{#1\}}
\newcommand{\irenum}[1]{{\tt enum}\{#1\}}
\newcommand{\irarray}[1]{{\tt array}\{#1\}}
\newcommand{\irarrayFW}[2]{{\tt arrayFW}\{#1\}[#2]}
\newcommand{\irswitch}[2]{{\tt switch}(#1)\{#2\}}
\newcommand{\iroption}[1]{{\tt option}\{#1\}}
\newcommand{\setof}[1]{\lsem #1 \rsem}
\newcommand{\goto}{\Rightarrow}
\newcommand{\Pvoid}{{\tt Pvoid}}
\newcommand{\Pempty}{{\tt Pempty}}
\newcommand{\sskip}{\hspace*{5mm}}
\newcommand{\shrink}{\vspace*{-4mm}}

% Semantics
\newcommand{\setalt}{{\; | \;}}
\newcommand{\denote}[1]{\lsem #1 \rsem}
\newcommand{\lsem}{{[\![}}
\newcommand{\rsem}{{]\!]}}
\newcommand{\turn}{\vdash}
\newcommand{\meta}{m}
\newcommand{\nested}{n}
\newcommand{\mytime}[1]{#1.t}
\newcommand{\myds}[1]{#1.ds}
\newcommand{\myval}[1]{#1.nest}
\newcommand{\generatedloc}{\ensuremath{\mathtt{nowhere}}}
\newcommand{\environment}{E}
\newcommand{\universe}{U}
\newcommand{\selectOne}{\ensuremath{\mathsf{earliest}}}
% core feed semantics
\newcommand{\csemantics}[3]{{\cal C}\lsem #1 \rsem_{{#2} \, {#3}}}
% feed semantics
\newcommand{\semantics}[3]{{\cal F}\lsem #1 \rsem_{{#2} \, {#3}}}
% expression semantics
\newcommand{\esemantics}[2]{{\cal E}\lsem #1 \rsem_{{#2}}}
%\newcommand{\esemantics}[2]{#2(#1)}

% Host language types
\newcommand{\ty}{\ensuremath{\tau}}
\newcommand{\basety}{\ensuremath{b}}
\newcommand{\arrow}{\rightarrow}
\newcommand{\optionty}[1]{\ensuremath{#1 \; \mathsf{option}}}
\newcommand{\listty}[1]{\ensuremath{#1 \; \mathsf{list}}}
\newcommand{\setty}[1]{\ensuremath{#1 \; \mathsf{set}}}
\newcommand{\feedty}[1]{\ensuremath{#1 \; \mathsf{feed}}}
\newcommand{\corety}[1]{\ensuremath{#1 \; \mathsf{core}}}
\newcommand{\schedulety}{\ensuremath{\mathsf{sched}}}
\newcommand{\timety}{\ensuremath{\mathsf{time}}}
\newcommand{\locty}{\ensuremath{\mathsf{loc}}}
\newcommand{\boolty}{\ensuremath{\mathsf{bool}}}
\newcommand{\unitty}{\ensuremath{\mathsf{unit}}}
\newcommand{\stringty}{\ensuremath{\mathsf{string}}}
\newcommand{\metatype}[1]{\ensuremath{\mathsf{meta}(#1)}}
\newcommand{\nestedtype}[1]{\ensuremath{\mathsf{nest}(#1)}}
\newcommand{\dsty}{\ensuremath{\mathsf{ds}}}

\newcommand{\dom}{\ensuremath{\mathsf{dom}}}
\newcommand{\ueq}[3]{\ensuremath{#1 =_{#2} #3}}
\newcommand{\fsubset}[3]{\ensuremath{#1 \subseteq_{#2} #3}}
\newcommand{\feq}[3]{\ensuremath{#1 =_{#2} #3}}

% Expressions
\newcommand{\expression}{e}
\newcommand{\constant}{c}
\newcommand{\ds}{\ensuremath{ds}}
\newcommand{\boolf}{\ensuremath{\mathtt{false}}}
\newcommand{\boolt}{\ensuremath{\mathtt{true}}}
\newcommand{\loc}{\ensuremath{\ell}}
\newcommand{\feed}{\ensuremath{F}}
\newcommand{\corefeed}{\ensuremath{C}}
\newcommand{\generalvar}{\ensuremath{x}}
\newcommand{\feedvar}{\ensuremath{x}}
\newcommand{\itemvar}{\ensuremath{x}}
\newcommand{\data}{\ensuremath{v}}
\newcommand{\atime}{\ensuremath{t}}
\newcommand{\astring}{\ensuremath{w}}
\newcommand{\unit}{\ensuremath{()}}
\newcommand{\schedule}{\ensuremath{s}}
\newcommand{\parser}{\ensuremath{p}}
\newcommand{\none}{\ensuremath{\mathtt{None}}}
\newcommand{\some}[1]{\ensuremath{\mathtt{Some}\; #1}}
\newcommand{\inl}[1]{\ensuremath{\mathtt{inl}\; #1}}
\newcommand{\inr}[1]{\ensuremath{\mathtt{inr}\; #1}}
\newcommand{\casedata}[2]{{\tt switch}(#1)\{#2\}}
%\newcommand{\nillist}{\ensuremath{\mathtt{nil}}}
\newcommand{\nillist}{\ensuremath{[\,]}}
%\newcommand{\conslist}[2]{\ensuremath{\mathtt{cons} (#1,#2)}}
\newcommand{\conslist}[2]{\ensuremath{[#1,\ldots,#2]}}
\newcommand{\nilstream}{\ensuremath{\mathtt{done}}}
\newcommand{\consstream}[2]{\ensuremath{\mathtt{next} (#1,#2)}}


% Feeds
\newcommand{\comprehensionfeed}[3]{\ensuremath{\mathtt{\{|} #1 \; \mathtt{|}\; #2 \leftarrow #3 \mathtt{|\}}}}
\newcommand{\computed}[3]{\ensuremath{\mathtt{[} #1 \; \mathtt{|}\; #2 \in #3 \mathtt{]}}}
\newcommand{\letfeed}[3]{\ensuremath{\mathtt{let}\; #1 \; \mathtt{=}\; #2 \; \mathtt{in} \; #3}}
\newcommand{\allfeed}[5]{
  \ensuremath{
    \mathtt{all \{ format=} #1; 
    \mathtt{src=} #2;
    \mathtt{sched=} #3;
    \mathtt{pp=} #4;
    \mathtt{win=} #5;
  \mathtt{\}}}}
\newcommand{\existsfeed}[5]{
  \ensuremath{
    \mathtt{any \{ format=} #1; 
    \mathtt{src=} #2;
    \mathtt{sched=} #3;
    \mathtt{pp=} #4;
    \mathtt{win=} #5;
  \mathtt{\}}}}
\newcommand{\filterfeed}[2]{
  \ensuremath{
    \mathtt{filter} \; #1 \; \mathtt{with}\; #2}}
\newcommand{\remapfeed}[2]{
  \ensuremath{
    \mathtt{redirect} \; #1 \; \mathtt{with}\; #2}}
\newcommand{\ppfeed}[2]{
  \ensuremath{
    \mathtt{pp} \; #1 \; \mathtt{with}\; #2}}
\newcommand{\foreachupdate}[3]{
  \ensuremath{
    \mathtt{foreach{*}{*}}\; #1 \;
    \mathtt{in}\; #2 \;
    \mathtt{update}\; #3}}
\newcommand{\foreachcreate}[3]{
  \ensuremath{
    \mathtt{foreach*}\; #1 \;
    \mathtt{in}\; #2 \;
    \mathtt{create}\; #3}}
\newcommand{\remap}[2]{\ensuremath{\mathtt{redirect}\; #1 \; \mathtt{with} \; #2}}
\newcommand{\stutterfeed}[2]{\ensuremath{\mathtt{stutter}\; #1 \; \mathtt{on} \; #2}}
\newcommand{\refeed}[2]{\ensuremath{\mathtt{reschedule}\; #1 \; \mathtt{to} \; #2}}
\newcommand{\emptyfeed}{\ensuremath{\emptyset}}
\newcommand{\onefeed}[2]{\ensuremath{\mathtt{One}}(#1,#2)}
\newcommand{\sfeed}[1]{\ensuremath{\mathtt{SchedF}}(#1)}
\newcommand{\lfeed}[1]{\ensuremath{\mathtt{ListF}}(#1)}
\newcommand{\unionfeed}{\ensuremath{\cup}}
\newcommand{\sumfeed}{\ensuremath{+}}
\newcommand{\spairfeed}{\; \ensuremath{\mathtt{\&} \; }}
\newcommand{\allpairfeed}{\; \ensuremath{{*}{*}} \; }

\newcommand{\Time}{\ensuremath{\mathtt{Time}}}
\newcommand{\Set}{\ensuremath{\mathtt{Set}}}

% this is used for the translations equal
\newcommand{\transeq}{\stackrel{def}{=} }
\newcommand{\ai}{{\tt wl}}



% BNF
%\newcommand{\bnfalt}{\ |\ }


\maketitle{}

\begin{abstract}  

  Massive amounts of useful data are stored and processed in
  non-standard or ad hoc formats, for which critical tools like
  parsers and formatters do not exist. Traditional databases and XML
  systems provide rich infrastructure for processing well-behaved
  data, but are of little help when dealing with data in ad hoc
  formats.
 
  To address the challenges of ad hoc data, we have designed PADS/ML -
  a declarative data description language for the ML family of
  languages. PADS/ML is based on the ML type structure and features
  polymorphic, recursive datatypes as central elements of data
  descriptions. In addition, PADS/ML exploits the ML module system to
  support an entirely new tool generation architecture designed to
  simplify the task of developing tool generators that work for any
  PADS/ML description.

  We have formalized the semantics of PADS/ML's recursive, polymorphic
  datatypes by extending our previous work on the Data Description
  Calculus with type-paramterized types and we have proven the
  resulting system ``type-correct.'' That is, generated parsers return
  data of the expected type.

  Finally, we have implemented a version of PADS/ML for the O'Caml
  language.  From a description, the O'Caml PADS/ML compiler generates
  an O'Caml module containing the relevant type declarations, a parser
  for the description, and other tools.

\end{abstract}

\section{Introduction}
\label{sec:intro}

{\em Data description languages} are a class of domain specific
languages for specifying {\em ad hoc data formats}, from billing 
records to TCP packets to scientific data sets to server logs.  Examples 
of such languages include 
\bro~\cite{paxson:bro}, \datascript{}~\cite{gpce02}, \demeter~\cite{lieberherr+:class-dictionaries},
\packettypes{}~\cite{sigcomm00}, \padsc{}~\cite{fisher+:pads}, 
\padsml{}~\cite{mandelbaum+:padsml}  and
\xsugar~\cite{xsugar2005}, among others.  All of these languages
generate parsers from data descriptions.  In addition, and unlike
conventional parsing tools such as Lex and Yacc, many also automatically
generate auxiliary tools ranging from printers to \xml{} converters to
visitor libraries to visualization and editor tools.

In previous work, we developed the {\em Data Description Calculus}
(\ddcold{}), a calculus of simple, orthogonal type constructors,
designed to capture the core features of many existing type-based data
description languages~\cite{fisher+:next700ddl,fisher+:ddcjournal}.
This calculus had a multi-part denotational semantics that interpreted
the type constructors as (1) parsers the transform external bit
strings into internal data representations and {\em parse descriptors}
(representations of parser errors), (2) types for the data
representations and parse descriptors, and (3) types for the parsers
as a whole.  We proved that this multi-part semantics was coherent in
the sense that the generated parsers always have the expected types
and generate representations that satisfy an important {\em
canonical forms} lemma.

The \ddcold{} has been very useful already, helping us debug and
improve several aspects of \padsc{}~\cite{fisher+:pads}, and serving
as a guide for the design of \padsml{}~\cite{mandelbaum+:padsml}.
However, this initial work on the \ddcold{} told only a fraction of the
semantic story concerning data description languages.  As mentioned
above, many of these languages not only provide parsers, but
also other tools.  Amongst the most common auxiliary tools
are printers, as reliable communication between programs, either through
the file system or over the Web, depends upon both input (parsing) 
and output (printing).

In this work, we begin to address the limitations of
\ddcold{} by specifying a printing semantics for the
various features of the calculus.  We also
prove a collection of theorems for the new semantics that serve as
duals to our theorems concerning parsing.  This new printing semantics
has many of the same practical benefits as our older parsing 
semantics: We can
use it as a check against the correctness of our printer
implementations and as a guide for the
implementation of future data description languages.  


% First, we extend \ddcold{} with
% abstractions over types, which provides a basis for specifying the
% semantics of \padsml{}. In the process, we also improve upon the
% \ddcold{} theory by making a couple of subtle changes. For example, we
% are able to eliminate the complicated ``contractiveness'' constraint
% from our earlier work. Second, .

% The main practical benefit of the calculus has been as a guide for our
% implementation. Before working through the formal semantics, we
% struggled to disentangle the invariants related to polymorphism. After
% we had defined the calculus, we were able to implement type
% abstractions as \ocaml{} functors in approximately a week.  Our new
% printing semantics was also very important for helping us define and
% check the correctness of our printer implementation.  We hope the
% calculus will serve as a guide for implementations of \pads{} in
% other host languages.  

% In summary, this work makes the following key contributions:
% \begin{itemize}
% \item We simultaneously specify both a parsing and a printing semantics
%   for the \ddc{}, a calculus of polymorphic, dependent types.
% \item We prove that \ddc{} parsers and printers are type safe
%   and well-behaved as defined by a canonical forms theorem.
% \end{itemize}

In this extended abstract, we give an brief overview of the calculus,
it's dual semantics and their properties.  A companion technical
report contains a complete formal
specification~\cite{fisher+:popl-sub-long}.  In comparison to our
previous work on the \ddcold{} at POPL 06~\cite{mandelbaum+:padsml},
the calculus we present here has been streamlined in several subtle,
but useful ways.  It has also been improved through the addition of
polymorphic types.  We call this new polymorphic variant
\ddc{}.  These improvements and extensions, together with
proofs, appear in Mandelbaum's thesis~\cite{mandelbaum:thesis} and in
a recently submitted journal article~\cite{fisher+:ddcjournal}.
This abstract reviews the \ddc{} and extends all the previous 
work with a printing semantics and appropriate theorems.
To be more specific,
sections~\ref{sec:ddc-syntax} through \ref{sec:ddc-sem} present the
extended \ddc{} calculus, focusing on the semantics of polymorphic
types for parsing and the key elements of the printing semantics.
Then, \secref{sec:meta-theory} shows that both parsers and
printers in the \ddc{} are type correct and furthermore that parsers
produce pairs of parsed data and parse descriptors in {\em canonical
  form}, and that printers, given data in canonical form, print
successfully. We briefly discuss related work in \secref{sec:related}, and
conclude in \secref{sec:conc}.

%%% Local Variables: 
%%% mode: latex
%%% TeX-master: "paper"
%%% End: 


\section{Describing Data in \padsmlbig{}}
\label{sec:padsml-overview}
{\em
To do:
\begin{itemize}
\item Check the details of the Kleinstein reference.
\end{itemize}
}
A \padsml{} description specifies the physical layout and semantic
properties of an ad hoc data source.  These descriptions are composed
of types: base types describe atomic data, while structured types
describe compound data built from simpler pieces.  Examples of base
types include ASCII-encoded, 8-bit unsigned integers (\cd{Puint8}) and
32-bit signed integers (\cd{Pint32}), binary 32-bit integers (\cd{Pbint32}),
dates (\cd{Pdate}), strings (\cd{Pstring}), zip codes (\cd{Pzip}),
phone numbers (\cd{Pphone}), and IP addresses (\cd{Pip}).  Semantic
conditions for such base types include checking 
that the resulting number fits in the indicated space, \ie, 16-bits
for \cd{Pint16}.

Base types may be parameterized by \ml{} values.  This mechanism
reduces the number of built-in base types and permits base types to
depend on values in the parsed data.  For example, the base type
\cd{Puint16_FW(3)} specifies an unsigned two byte integer physically
represented by exactly three characters, and the base type
\cd{Pstring} takes an argument indicating the \textit{terminator
character}, \ie{}, the character in the source that immediately
follows the string.

To describe more complex data, \padsml{} provides a collection of type
constructors derived from the type structure of functional programming
languages like Haskell and ML.  We explain these structured types in
the following subsections using examples drawn from data sources we
have encountered in practice.

% Readers eager to see the complete syntax
% of types should flip forward to Appendix~\ref{app:syntax-dd}.

\subsection{Simple Structured Types}


\begin{figure*}
{\scriptsize
\begin{verbatim}
0|1005022800
9152|9151|1|9735551212|0||9085551212|07988|no_ii152272|EDTF_6|0|APRL1|DUO|10|1000295291
9153|9153|1|0|0|0|0||152268|LOC_6|0|FRDW1|DUO|LOC_CRTE|1001476800|LOC_OS_10|1001649601
\end{verbatim}
}
  \caption{Miniscule example of \dibbler{} data.}
  \label{figure:dibbler-records}
\end{figure*}

The bread and butter of a \padsml{} description are the simple
structured types: tuples and records for specifying ordered data;
lists for specifying homogeneous sequences of data; sum types for
specifying alternatives; and singletons for specifying the occurrence
of literal characters in the data.  We describe each of these
constructs as applied to the \dibbler{} data presented in
\figref{figure:dibbler-records}.

\dibbler{} data summarizes orders for phone service placed with AT\&T.
Each \dibbler{} data file starts with a timestamp followed by one
record per phone service order.  Each order consists of a header and a
sequence of events.  The header has 13 pipe separated fields: the
order number, AT\&T's internal order number, the order version, four
different telephone numbers associated with the order, the zip code of
the order, a billing identifier, the order type, a measure of the
complexity of the order, an unused field, and the source of the order
data.  Many of these fields are optional, in which case nothing
appears between the pipe characters.  The billing identifier may not
be available at the time of processing, in which case the system
generates a unique identifier, and prefixes this value with the string
``no\_ii'' to indicate that the number was generated. The event
sequence represents the various states a service order goes through;
it is represented as a new-line terminated, pipe separated list of
state, timestamp pairs.  There are over 400 distinct states that an
order may go through during provisioning.  The sequence is sorted in
order of increasing timestamps.  From this description, it is apparent
that English is a poor language for describing data formats!

\begin{figure}
\begin{code}\scriptsize
\kw{ptype} Semicolon = Pcharlit(';')
\kw{ptype} Vbar = Pcharlit('|')
\mbox{}
\kw{ptype} Pn\_t = Puint64
\kw{ptype} Pzip = Puint32
\mbox{}
\kw{ptype} Summary\_header = "0|" * Puint32 * Peor
\mbox{}
\kw{pdatatype} Dib\_ramp = 
  Ramp of Pint 
| GenRamp of "no\_ii" * Pint
\mbox{}
\kw{ptype} Order\_header = \{ 
    order\_num : Pint;  
'|'; att\_order\_num : [i:Pint | i < order\_num];  
'|'; ord\_version : Pint;  
'|'; service\_tn : Pn\_t Popt;
'|'; billing\_tn : Pn\_t Popt;  
'|'; nlp\_service\_tn : Pn\_t Popt;  
'|'; nlp\_billing\_tn : Pn\_t Popt;  
'|'; zip\_code : Pzip Popt;  
'|'; ramp : Dib\_ramp;  
'|'; order\_sort : Pstring('|');  
'|'; order\_details : Pint;
'|'; unused : Pstring('|');  
'|'; stream : Pstring('|'); 
'|'
\} 
\mbox{}
\kw{ptype} Event  = Pstring('|') * '|' * Puint32
\kw{ptype} Events = (Event,Vbar,Peor) Plist
\mbox{}
\kw{ptype} Order  = Order\_header * Events
\kw{ptype} Orders = (Order, Peor, Peof) Plist
\mbox{}
\kw{ptype} Source = Summary\_header * Orders\end{code}

  \caption{\padsml{} description for \dibbler{} provisioning data.}
  \label{figure:sirius_pml}
\end{figure}

\figref{figure:sirius_pml} contains the \padsml{} description for the
\dibbler{} data format.  The description is a sequence of type
definitions.  Type definitions precede uses, therefore the description
should be read bottom up.
The type \cd{Source} (Line~27) describes a complete \dibbler{} data
file and denotes an ordered tuple containing a
\cd{Summary\_header} value followed by an \cd{Orders} value.
Other tuple types are defined on Lines~1, 23, and~25.

The type \cd{Orders} (Line~26) uses the list type constructor
\cd{Plist} to describe a homogenous sequence of values in a data
source.  The \cd{Plist} constructor takes three parameters: on the
left, the type of elements in the list; on the right, a literal
\emph{separator} that delimits elements in the list, and a literal
\emph{terminator}.  In this example, the type \cd{Orders} is a list of
\cd{Order} elements, separated by a newline, and terminated by
\cd{peof}, a special literal that describes the \emph{end-of-file
  marker}.  Similarly, the \cd{Events} type (Line~24) denotes a
sequence of \cd{Event} values separated by vertical bars and
terminated by a newline.

Literal characters in type expressions denote singleton types.  For
example, the \cd{Event} type (Line~23) is a string terminated by a
vertical bar, followed by a vertical bar, followed by a timestamp.  The
singleton type \cd{'|'} means that the data source must contain the
character \cd{'|'} at this point in the input stream.  String,
character, and integer literals can be embedded in a description and
are interpreted as singleton types, \eg{}, the singleton type
\cd{"0|"} in the \cd{Summary\_header} type (Line~1) 
denotes the string literal \cd{"0|"}.

The type \cd{Order\_header} is a record type, \ie{}, a tuple type in
which each field may have an associated name.  The named field
\cd{att\_order\_num} (Line~9) illustrates two other features of
\padsml: dependencies and constraints.  Here, \cd{att\_order\_num}
depends on the previous field \cd{order\_num} and is constrained to be
less than that value.  In practice, constraints may be complex, have
multiple dependencies, and can specify, for example, the sorted order
of records in a sequence.  Constrained types have the form \cd{[x:T |
e]} where \cd{e} is an arbitrary pure boolean expression.  Data
satisfies this description if it satisfies \cd{T} and boolean \cd{e}
evaluates to true when the parsed representation of the data is
substituted for \cd{x}.  If the boolean expression evaluates to false,
the data contains a \textit{semantic} error.

The datatype definition of \cd{Dib\_ramp} (Line~3) specifies two
alternatives for a data fragment, either one integer or the fixed
string \cd{"no\_ii"} followed by one integer.  The order of
alternatives is significant, that is, the parser attempts to parse the
first alternative and only if it fails, it attempts to parse the
second alternative.  This semantics differs from similar constructs in
regular expressions and context-free grammars, which
non-deterministically choose between alternatives.
\cut{Fortunately, we have yet to come across an ad hoc data
source where we wish we had nondeterministic choice.\footnote{\padsml{}
  can recognize string data based on regular expressions.
  Non-determinism here has been useful, but as it has been confined to
  parsing elements of the \cd{Pstring} base type, it has had no impact
  on the overall parsing algorithm.}
}
\begin{figure}
\begin{code}\scriptsize
\kw{ptype} Semicolon = Pcharlit(';')
\kw{ptype} Rparen= Pcharlit(')')
\mbox{}
\kw{ptype} Entry = \{
  name : Pstring(':'); ':'; dist : Pfloat32
\}
\mbox{}
\kw{pdatatype} Tree =
  Interior of '(' * (Tree, Semicolon, Rparen) Plist * ')'
| Leaf of Entry\end{code}

%%% Local Variables: 
%%% mode: latex
%%% TeX-master: "paper"
%%% End: 

Tiny fragment of Newick data:

{
\begin{verbatim}
(((erHomoC:0.28006,erCaelC:0.22089):0.40998,
(erHomoA:0.32304,(erpCaelC:0.58815,((erHomoB:0.5807,
erCaelB:0.23569):0.03586,erCaelA:0.38272):0.06516):
0.03492):0.14265):0.63594,(TRXHomo:0.65866,TRXSacch:
0.38791):0.32147,TRXEcoli:0.57336)
\end{verbatim}
}
  \caption{Simplified tree-shaped Newick data}
  \label{fig:newick}
\end{figure}

\subsection{Recursive Types}

\padsml{} can describe data sources with recursive structure.  One
example of recursive data is the Newick format, a flat representation
of trees used by biologists~\cite{newick} that uses properly nested
parentheses to specify a tree's hierarchy.  A leaf node is a string
label followed by a colon and a number.  An interior node contains a
sequence of children nodes, delimited by parentheses, followed by a
colon and a number.  The numbers represent the ``distance'' that
separates a child node from its parent.

\figref{fig:newick} contains a concise and elegant description of Newick
and a small fragment of example data.  In this example, the string
labels are gene names and the distance denotes the number of mutations
that occur in the antibody receptor genes of B lymphocytes.  This
data is provided by Steven Kleinstein, program coordinator of
Princeton's Picasso project for interdisciplinary research in
computational sciences.  \edcom{Is this true anymore? Should it be in
  past tense? Should it be here at all?}. Kleinstein is building a simulator to study
the proliferation of B lymphocytes during an immune response.

\begin{figure}
  \centering
  \small
\begin{verbatim}
 2:3004092508||5001|dns1=abc.com;dns2=xyz.com|c=slow link;w=lost packets|INTERNATIONAL
 3:|3004097201|5074|dns1=bob.com;dns2=alice.com|src_addr=192.168.0.10;
 dst_addr=192.168.23.10;start_time=1234567890;end_time=1234568000;cycle_time=17412|SPECIAL
\end{verbatim}  
  \caption{Simplified network-monitoring data. We inserted the newline
    after the ';' to improve legibility.}
  \label{fig:darkstar-records1}
\end{figure}

\subsection{Polymorphic Types}

The last key feature of \padsml{} is parameterized types.  Type
parameterization makes descriptions more concise and allows
programmers to define convenient libraries of reusable components. The
description in Figure~\ref{fig:darkstar-ml} illustrates types
parameterized by both types and values.  This description specifies
the format of alarm data recorded by a network-link monitor used by
the \darkstar{} project at AT\&T.  Sample \darkstar{} data is in
\figref{fig:darkstar-records1}.

Each alarm signals some problem with a network link; At this point, we
could write an English description of the \darkstar{} data. However,
English is a terrible specification language for ad hoc data, both
verbose and imprecise.  Hence, we dispense with the informal English
and head straight for the \padsml{} description, shown in
\figref{fig:darkstar-ml}.

\begin{figure}
  \centering
  \begin{code}\scriptsize
(* Pstring terminated by ';' or '|'. *)
\kw{ptype} SVString = Pstring\_SE("/;|\\\\|/")
\mbox{}
(* Generic name value pair. Accepts predicate 
   to validate name as argument.� *)
\kw{ptype} (Alpha) Pnvp(p : string -> bool) =
      \{ name : [name : Pstring('=') | p name]; 
            '='; 
       value : Alpha \}
\mbox{}
(* Name value pair with name specified. *)
\kw{ptype} (Alpha) Nvp(name:string) = 
   Alpha Pnvp(fun s -> s = name)
\mbox{}
(* Name value pair with any name. *)
\kw{ptype} Nvp\_a = SVString Pnvp(fun \_ -> true)
\mbox{}
\kw{ptype} Details = \{
      source      : Pip Nvp("src\_addr");
';';  dest        : Pip Nvp("dest\_addr");
';';  start\_time  : Ptimestamp Nvp("start\_time");
';';  end\_time    : Ptimestamp Nvp("end\_time");
';';  cycle\_time  : Puint32 Nvp("cycle\_time")
\}

\kw{pdatatype} Info(alarm\_code : int) =
  match alarm\_code with
    5074 -> Details of Details
  | \_    -> Generic of Nvp\_a Plist(';','|')

\kw{pdatatype} Service = 
    DOMESTIC      of "DOMESTIC" 
  | INTERNATIONAL of "INTERNATIONAL" 
  | SPECIAL       of "SPECIAL"

\kw{ptype} Alarm = \{
       alarm    : [i : Puint32 | i = 2 or i = 3];
 ':';  start    : Ptimestamp Popt;
 '|';  clear    : Ptimestamp Popt;
 '|';  code     : Puint32;
 '|';  src\_dns  : SVString Nvp("dns1");
 ';';  dest\_dns : SVString Nvp("dns2");
 '|';  info     : Info(code);
 '|';  service  : Service
\} 

\kw{ptype} Source = Alarm Plist('\\n',\kw{peof})\end{code}
%
%\kw{let} checkCorr ra = ...
%\kw{ptype} Alarm = [x:Raw\_alarm | checkCorr x]



  \caption{Description of \darkstar{} data.}
  \label{fig:darkstar-ml}
\end{figure}

This data format has several variants of name-value pairs. The
\padsc{} description of this format (shown in~\appref{app:regulus-padsc})
must define a different type for each variant. In contrast, the
polymorphic types of \padsml{} allow us to define the type \cd{Pnvp}
(Line~2), which takes both type and value parameters to encode all the
variants. Type parameters appear to the left of the type name, as is
customary in \ml{}.  Value parameters and their \ml{} types appear to
the right of the type name.  In the definition of \cd{Pnvp}, there is
one type parameter named \cd{Alpha} and one value parameter named
\cd{p}.  Informally, \cd{Alpha Pnvp(p)} is a name-value pair where the
value is described by \cd{Alpha} and the name must satisfy the
predicate \cd{p}.

The \cd{Nvp} type (Lines~6--7) is defined in terms of \cd{Pnvp}, in
which the name argument must match a particular string, but the type
parameter remains abstract.  The \cd{Nvp\_a} (Line~8) is also defined
in terms of \cd{Pnvp}.  In this case, any name is permitted, but the
value must be a \cd{SVString}, \ie{}, string terminated by a semicolon
or vertical bar.  Elsewhere in the description, the type parameter to
\cd{Nvp} is instantiated with IP addresses, timestamps, and integers.

% The source type is an array of \cd{alarm}s, where each alarm is a
% \cd{raw\_alarm}, constrained to ensure that the alarm number is
% properly correlated with the timestamps.  We check this correlation
% with the function \cd{checkCorr}.  The type \cd{raw\_alarm} closely
% follows the description above. We highlight a few important features.
% First, we note that the type of the field \cd{info} depends on the
% alarm code, reflecting the text above. More interestingly, the type
% \cd{info} is implemented with a switched datatype, deciding how to
% parse based on the parameter \cd{alarm\_code}.  Next, we note that the
% description includes five different types of name-value pairs. We take
% advantage of both the type and value parameterization of types to
% encode all of these pair types based on one common description,
% \cd{pnvp}. This type is polymorphic in the type of the value and takes
% an arbitrary constraint \cd{c} as an argument. The type \cd{nvp} is
% polymorphic in the type of the value, but takes the expected name of
% the string as an argument. 

The \darkstar{} description also illustrates the use of
\textit{switched} datatypes.  A switched datatype selects a variant
based on the value of a user-specified \ocaml{} expression, which
typically references parsed data from earlier in the data source.  For
example, the switched datatype \cd{Info} (Lines~16--19) chooses a
variant based on the value of its \cd{alarm\_code} parameter.  More
specifically, if the alarm code is \cd{5074}, the format specification
given by the \cd{Details} constructor will be used to parse the
current data.  Otherwise, the format given by the \cd{Generic}
constructor will be used to parse the current data.

The last construct in the \darkstar{} description is the type
qualifier \cd{omit}.  In the \cd{Service} datatype (Lines~20--23),
\cd{omit} specifies that the parsed string literal should be omitted
in the internal data representation, because the literal can be
determined by the datatype constructor.

\cut{We can do this because we can discern from the
datatype constructor which string was found in the data source.}

%%% Local Variables: 
%%% mode: latex
%%% TeX-master: "../thesis.tex"
%%% End: 


\section{From \padsmlbig{} to \ocamlbig{}}
\label{sec:padsml-impl}

{\em
ToDo: Move detail about Traverse functor to Generic tools section.
}

We have implemented \padsml{} for use with \ocaml{}. The \padsml{}
compiler generates libraries in \ocaml{} source code that can then be
used by any \ocaml{} program. In this section, we describe the
contents of the generated libraries followed by some examples
demonstrating their use.


\subsection{Generated Libraries}
\label{sec:gen-code}

From each \padsml{} description, we generate a collection of types and
functions in \ocaml{}, including:
\begin{itemize}
\item The types of two data structures: one to contain parsed data in
  memory and the other to hold meta-data about the parsing process.
  These data structures are respectively called the
  \emph{representation} and the \emph{parse descriptor}.
\item A parsing function, which parses a data source to produce a
  representation and parse descriptor for the data.
\item A generic tool generator, based on the new tool development
  framework for \padsml{}. This framework is discussed in
  \secref{sec:gen-tool}.
\end{itemize} 

In general, the representation and parse-descriptor type definitions
are designed to closely resemble the original description.
The aim is to minimize the amount of effort a user must invest in
order to understand and use the data structures returned by the
parser.

Furthermore, the type of parse descriptor mimics the type of the
representation so that the parse descriptor can provide a parsing
report for every element of a corresponding representation. Parse
descriptors have two components: a header and a body. The header
reports on the parsing process that produced the representation. It
includes an error count that indicates the number of subcomponents
with errors; an error code that indicates the type of error, if any;
and the location of the data within the original data source. The body
of the parse descriptor contains the parse descriptors (if any) for
subcomponents of corresponding representations. The body for a
base-type parse descriptor is always of type \cd{unit}.

Below is a simple \padsml{} description of a character
and integer separated by a vertical bar.
\begin{code}\scriptsize
  \kw{ptype} Pair = Pchar * '|' * Pint\end{code} 
Here is a partial listing of the elements generated from that description.
\begin{code}\scriptsize
\kw{type} rep = Pchar.rep * Pint.rep
\kw{type} pd_body = Pchar.pd  * Pint.pd
\kw{type} pd = Pads.pd_header * pd_body

\kw{val} parse : Pads.handle -> rep * pd\end{code} 
This sample code and others that follow make use of a module
\cd{Pads} that contains types and functions that commonly occur in
generated and base-type modules. In particular, the above declarations use
\cd{Pads.pd_header}, which is the type of all parse-descriptor
headers, and \cd{Pads.handle}, which is the type of the (abstract)
handles used for data sources.
Note the close correspondence between the structure of the description
and that of the \cd{rep} and \cd{pd_body} types. In addition, we see
that the type of the parse function is defined in terms of the
\cd{rep} and \cd{pd} types.

Given the close relationship between the elements generated from a
description, it is natural to collect them together in a module. For
each named type, therefore, we generate a module with definitions like
those shown in the above example.
% For all generated modules, \cd{rep}, \cd{pd_body},\cd{pd} define the
% types of the data's representation, parse-descriptor body, and parse
% descriptor, respectively. The parsing function is named \cd{parse}.
In general, all types with base kind (i.e. those that are not
parameterized) match the following signature,
\cd{Type.S}:
\begin{code}\scriptsize
\kw{type} rep
\kw{type} pd\_body
\kw{type} pd = Pads.pd_header * pd_body

\kw{val} parse : Pads.handle -> rep * pd\end{code}

Modules, then, become the building blocks of the \padsml{} system.
Base types, too, are implementated with modules. Polymorphic types,
which map types to types, are implemented as functors from (type)
modules to (type) modules. It would even be appropriate to map
recursive types into recursive modules. Unfortunately, this approach
fails due to the limitations of the \ocaml{} implementation of
recursive modules. We would need support for inclusion of functors in
recursive modules in order to take this approach.

Given the signature \cd{Type.S} for types of base kind, we can now
show an example signature for a polymorphic type.
\begin{code}\scriptsize
\kw{ptype} (Alpha,Beta) ABPair = Alpha * '|' * Beta\end{code}
becomes
\begin{code}\scriptsize
\kw{module} ABPair (Alpha : Type.S) (Beta : Type.S) :
\kw{sig}
  \kw{type} rep = Alpha.rep * Beta.rep
  \kw{type} pd\_body = (Pads.pd_header * Alpha.pd\_body) * 
                 (Pads.pd_header * Beta.pd\_body)
  \kw{type} pd = Pads.pd_header * pd\_body

  \kw{val} parse : Pads.handle -> rep * pd
\kw{end}\end{code}

Once a description has been compiled into an \ocaml{} module, that
module can be used like any other.  More specifically, each named type
in a description file is mapped into an \ocaml{} module of the same
name.  The collection of modules is grouped together into a
single file (compilation unit) with a name corresponding to the name
of the original description file. For example, a description file
named ``foo.pml'' with three types inside results in a file ``foo.ml''
with three submodules, each corresponding to one named type.  In the
remainder of this section, we will demonstrate a number of uses of
generated modules, highlighting data processing, transformation, and
filtering.

\subsection{Example: Data Processing}
\label{sec:ex-process}

We begin with a simple example in which we process a triple of
integers. Below is their description:
\begin{code}\scriptsize
\kw{ptype} Source = Pint * '|' * Pint * '|' * Pint\end{code} Next, we
show a complete \ocaml{} program that finds the average of the three
integers. (Note that we assume that the name of the description file
is ``intTriple.pml,'' resulting in an \ocaml{} module \cd{IntTriple}.)
\begin{code}\scriptsize
\kw{open} Pads
\kw{let} ((i1,i2,i3),pd) = 
    parse_source IntTriple.Source.parse "input.txt"
\kw{let} avg = match get_pd_hdr pd with
    \{error_code = Good\} -> (i1 + i2 + i3)/3
  | _ -> 0\end{code}

In this program, we parse the triple, check that it is valid and then
average its elements. The function \cd{parse_source} takes a parsing
function for a data source and a file name in which the data is
stored, and parses the source. In order to ensure that the data is
valid, the program projects the parse descriptor header from the parse
descriptor \cd{pd} and checks that the error code is set to \cd{Good}.
This error code is defined in the \cd{Pads} module, and indicates that
the data is syntactically and semantically valid.

Notice that checking the parse descriptor of the triple is enough to
guarantee that there are no errors in any of the triple's
subcomponents. This property is generally true of all representations
and corresponding parse descriptors. That is, if the header of a parse
descriptor reports no errors, then none of its subcomponents will
report errors. In this way, we support a ``pay-as-you-go'' approach to
application error handling, as the parse descriptor for valid data
need only be consulted once, no matter the size of the corresponding
data. Only if there are errors within the structure does the user then
need to continue consulting the parse descriptor until the error is
located.

\subsection{Example: Filtering}
\label{sec:ex-filter}

An important set of tasks relating to ad hoc data are those
related to errors, including error analysis, repair, and removal.
Programmers might want to clean their data, \ie{}, filter out data
containing errors. In this case, they need only access parse
descriptors to facilitate this task.

\begin{figure}
\begin{code}\scriptsize
\kw{open} Pads
   ...
\kw{let} split_entry (entry,pd) =
   match get\_pd\_hdr pd with
     \{error_code = Good\} -> write_valid entry
   | _ => write_invalid entry\end{code}
\caption{Error filter for \dibbler{} data}
\label{fig:ex-data-clean}
\end{figure}

\figref{fig:ex-data-clean} provides a partial demonstration of
splitting a standard data source into two separate sources, one with
valid records and the other (potentially) invalid records.  The valid
entries may then be further processed or loaded into a database
without corrupting the valuable data therein.  A human might examine
the bad entries off-line to determine the cause of errors or to figure
out how to fix the corrupted entries.

We assume that functions \cd{write_valid} and \cd{write_invalid} are
defined elsewhere to write an entry to a stream of valid and invalid
entries, respectively. The \cd{split_entry} function, then, receives
an entry and its parse descriptor, and, based on the parse descriptor,
writes the entry to the appropriate stream.

\subsection{Example: Transformation}
\label{sec:ex-trans}

\begin{figure}
  \centering
  \begin{code}\scriptsize
...
\kw{ptype} Header = \{
       alarm : [ a : Puint32 | a = 2 or a = 3];
 ':';  start :  Timestamp Popt;
 '|';  clear :  Timestamp Popt;
 '|';  code: Puint32;
 '|';  src\_dns  :  Nvp("dns1");
 ';';  dest\_dns :  Nvp("dns2");
 '|';  service  : service
\}
\mbox{}
\kw{ptype} D\_alarm = \{
       header   : header;
 '|';  details  : details
 \}
\mbox{}
\kw{ptype} G\_alarm = \{
       header   : header;
 '|';  generic  : (Nvp\_a,Semicolon,Vbar) Plist
\}\end{code}
\caption{Normalized format for \darkstar{} data. All named types not
  explicitly included in this figure are unchanged from the original
  \darkstar{} description.}
\label{fig:normal-darkstar}
\end{figure}

\begin{figure}
\begin{code}\scriptsize
\kw{open} Regulus
\kw{open} RegulusNormal
\kw{module} RA = Raw\_alarm
\kw{module} DA = D\_alarm
\kw{module} GA = G\_alarm
\kw{module} Header = H

\kw{let} splitAlarm ra =
    let h = 
       \{H.alarm=ra.RA.alarm; H.start=ra.RA.start; 
         H.clear=ra.RA.clear; H.code=ra.RA.code;
         H.src\_dns=ra.RA.src\_dns; H.dest\_dns=ra.RA.dest\_dns;
         H.service=ra.RA.service\};
    in match ra with
        \{info=Details(d)\} -> 
        (Some \{DA.header = h; DA.details = d\}, None)
      | \{info=Generic(g)\} ->
        (None, Some \{GA.header = h; GA.generic = g\})    
  \end{code}
  \caption{Shredding \darkstar{} data based on the {\tt info} field.}
  \label{fig:ex-no-err-check}
\end{figure}

Once a data source has been parsed and cleaned, a natural desire is to
transform such data to make it more amenable to further analysis.  For
example, analysts often need to convert ad hoc data into a form
suitable for loading into an existing system, such as a relational
database or statistical analysis package. Desired transformations
include removing extraneous literals, inserting delimiters, dropping
or reordering fields, and normalizing the values of fields (\eg{}
converting all times into a specified time zone).

Because relational databases typically cannot store unions directly,
another important transformation is to convert data with variation
(\ie{}, datatypes) into a form that such systems can handle.
Typically, there are two choices for such a transformation.  The first
is to chop the data into a number of relational tables: one table for
each variation.  This approach is called \textit{shredding}. The
second is to create an ``uber'' table, with one ``column'' for each
field in any variation.  If a given field is not in a particular
variation, it is marked as missing. 

The description fragment in \figref{fig:normal-darkstar} and code
fragment in \figref{fig:ex-no-err-check} demonstrate shredding
\darkstar{} data with \padsml{} and \ocaml{}. We shred the data into
two different tables based on the \cd{info} field of \cd{Alarm}
records. In the process, we also reorder the fields, putting the
\texttt{service} field into the common \texttt{header}. Notice that, in
the normalized format, \cd{Alarm} has been replaced with \cd{D\_alarm}
and \cd{G_alarm}, neither of which contain any fields with variable
type.

\begin{figure}
  \centering
  \begin{code}\scriptsize
\kw{let} normalizeTimeToGMT t = 
    match t with
      \{time=t;timezone="GMT"\} => t
    | \{time=t;timezone="EST"\} => t + (5 * 60 * 60)
    | \{time=t;timezone="PST"\} => t + (8 * 60 * 60)
    | ... \end{code}
  \caption{Normalizing timestamps}
  \label{fig:ex-normalize}
\end{figure}

In \figref{fig:ex-normalize}, we show an additional example of data
transformation, where we normalize timestamp-timezone pairs into
simple timestamps in GMT time.

%%% Local Variables: 
%%% mode: latex
%%% TeX-master: "paper"
%%% End: 


\section{The Generic Tool Framework}
\label{sec:gen-tool}

\begin{itemize}
\item show signature of traversal functor.
\item show interface - explain that one module/constructor and base types.
\item constrast records and one other module.
\item x Make it 3 tools: accum, XML, debuger.
\item x Focus on accumulator. describe in more detail.
\item show implementations,  including base types to show how they differ.
\item x then, describe XML, debugger as pretty printers.
\item use ``generic tool'' in consistent way. I like it better than
  tool generator.
\end{itemize}

An essential benefit of \padsml{} is that it can provide the users
with a high return-on-investment for describing their data. While the
generated parser alone is often enough to justify the user's effort,
we aim to increase the return by producing a large suite of data
analysis tools for every data source described in \padsml{}. However,
there is a limit, both in resources and expertise, to the range of
tool generators that we can develop. Indeed, new and interesting data
analysis tools are constantly being developed, and we have no hope of
integrating even a fraction of them into the \padsml{} system
ourselves.

Fortunately, we don't have to. A large class of data analysis tools
share a common data processing method while differing in the details
of how they transform data. These tools traverse the data
representation and parse descriptor in a depth-first, left-to-right
manner, often carrying some auxiliary state.  For each element
visited, they perform some action involving the auxiliary state,
either before, after or between visiting the element's subcomponents.
Often, the most interesting computations occur at the leaves, where
the computation is based on the leaf's type.  

To the experienced functional programmer, these tools essentially
perform a generalized fold over the representation and parse
descriptor, viewed as trees of components. Therefore, we have created
a generic tool framework modeled after \ml{}'s fold: we decouple the
mechanism of data traversal from the operations performed on
individual data items. A format-dependent traversal mechanism is
generated by the \padsml{} compiler. \emph{Generic tools}, which
specify how to process individual data elements, is developed
separately. The two interact through a signature that every generic
tool must match.

Unlike lists, the elements of data representations and PD's do not
have a uniform type. Therefore, the generic tool must provide
different processing functions for different types. The generated
traversal mechanism, which is format specific, can then choose the
correct function to apply to each element of the data. Therefore, the
generic-tool signature specifies a particular collection of types and
functions for every construct in \padsml{}.

The new, generic tool architecture of \padsml{} delivers a number of
benefits over the fixed architecture of \padsc{}. In \padsc{}, all
tools are generated from within the compiler. Therefore, developing a
new tool generator requires understanding and modifying the compiler.
Furthermore, the set of tools to be generated is chosen by the user
when compiling the description.  In \padsml{}, tool generators can be
developed independent of the compiler and they can be developed more
rapidly, as the ``boilerplate'' code to traverse data need not be
replicated for each tool generator. In addition, the user controls
which tools to ``generate'' for a given data format, and the choice
can differ on a program-by-program basis.

\subsection{The Generic-Tool Interface}
\label{sec:gentool-interface}

\begin{figure}
\begin{code}\scriptsize
\kw{module} \kw{type} S = \kw{sig}
  \kw{type} state
  ...
  \kw{module} Record : \kw{sig}
    \kw{type} partial_state
    \kw{val}  init          : (string * state) list -> state
    \kw{val}  start         : state -> Pads.pd_header 
                         -> partial_state
    \kw{val}  project       : state -> string -> state
    \kw{val}  process_field : partial_state -> string
                         -> state -> partial_state
    \kw{val}  finish        : partial_state -> state
  \kw{end}

  \kw{module} Datatype : \kw{sig}
    \kw{type} partial_state
    \kw{val}  init            : unit -> state
    \kw{val}  start           : state -> Pads.pd_header 
                           -> partial_state
    \kw{val}  project         : state -> string -> state option
    \kw{val}  process_variant : partial_state -> string 
                           -> state -> state
  \kw{end}
   ...
\kw{end}
\end{code}
\caption{An excerpt of the generic-tool interface.}
\label{fig:gentool-interface}
\end{figure}

The generic tool interface specifies a type for abstract state that is
threaded through the traversal, and, for every type constructor in
\padsml{}, a set of types and functions -- grouped together in a
module -- that a generic tool must implement. Every module has an
\cd{init} function to create an initial state object for data
processed by that module. In addition, a \cd{project} function
retrieves the state of a subcomponent from the state of an element. As
processing an element can occur before, after, or between processing
an element's children, the signature includes functions corresponding
to each of these events. The function \cd{start} begins processing the
element, \cd{process_...} to process a subcomponent, and \cd{finish}
to complete processing the element. For type constructs with only one
subcomponent, the \cd{process} and \cd{finish} functions are combined.

% The generated traversal function processes the data in a
% depth-first, left-to-right manner. In general, given an abstract
% state object for a a data element, it is traversed as follows.
% First, it provides the generic tool with the state and the PD header
% and receive a new, ``in-progress'' state object in return.  Next,
% for each child of the node, it will ask the tool for the child's
% state based on the node's state, recursively process the child node
% with the given state, and then give the child's new state to the
% tool.  For nodes with more than one child, such as records and
% tuples, it will tell the tool that it has finished, at which point
% the tool converts the ``in-progress'' state into a new state object.

An excerpt of the generic tool interface is shown in
\figref{fig:gentool-interface}. It includes the signatures of the
\cd{Record} and \cd{Datatype} modules. Both modules include a type
\cd{partial_state} that allows tools to represent intermediate state
in a different form than the general state while processing
subcomponents. In the \cd{Record} module, \textbf{I am here.}

\subsection{Example Tools}
\label{sec:gentool-motivation-ex}

% \begin{figure}
%   \centering
%   \small
% \begin{verbatim}
% 122Joe|Wright|450|95|790
% n/aEd|Wood|10|47|31
% 124Chris|Nolan|80|93|85
% 125Tim|Burton|30|82|71
% 126George|Lucas|32|62|40
% \end{verbatim}  
%   \caption{A fictitious data fragment in the Movie-director Bowling
%     Score (MBS) format. Note that the first record contains semantic
%     errors (in that the minimum is larger than the maximum and the
%     average is larger than both the minimum and the maximum).}
%   \label{fig:gentool-mbs-sample}
% \end{figure}

\begin{figure}
  \centering
  \scriptsize
\begin{verbatim}
<Order>
   <summary>
      <errors>1</errors> <total>2</total>        
   </summary>
   <Order_header>
      <summary>
         <errors>1</errors> <total>2</total>        
      </summary>
      <order_num>
         <errors>0</errors> <total>2</total>        
      </order_num>
      <att_order_num>
         <summary>
            <errors>1</errors> <total>2</total>        
         </summary>
         <val>
            <errors>0</errors> <total>2</total>                
         </val>
      </att_order_num>
      <ord_version>
         <errors>0</errors> <total>2</total>                
      </ord_version>
      ...
   </Order_header>
</Order>
\end{verbatim}  
  \caption{A fragment of the accumulator output for \dibbler{}. The
    output is encoded in XML.}
  \label{fig:gentool-acc-output}
\end{figure}

We have implemented three generic tools that illustrate important
features of the framework: a tool for generating statistical
overviews of the data, a data printer for debugging, and an XML
formatter.

A common desire of a data analyst upon receiving a new data source is
to get a sense of the quality of the data. In particular, they might
be interested in knowing what percentage of the source has errors, or
which fields are the most problematic. For this purpose, \padsc{}
provides an \emph{accumulator} tool for \padsc{} that statistically
summarizes data sources. The accumulator is designed for data sources
whose basic structure is series of records of the same type.
\emph{another sentence or two here. explain notion of one
  acccumulator, many records}

We have implemented a generic tool that provides some of the basic
features of \padsc{} accumulators. Our accummulator counts the number
of errors and the total number of values for every element of a
description. In \figref{fig:gentool-acc-output}, we show a sample
portion of accumulator output for the \dibbler{} data
from~\figref{figure:dibbler-records}.

The accumulator data shows that 1 out of the 2 \cd{Order}s has an
error. Investigating further, we notice that the problem lies in the
\cd{Order_header}, in particular within the \cd{att_order_num} field.
This field has a constraint on it, and one of the values violates the
constraint. A quick glance at the data fragment reveals that the
second order is contains the offending field. In general, specific
information like this could be found in the parse descriptor.

\begin{figure}
\begin{code}\scriptsize
\kw{type} baseAcc = int * int
\kw{type} acc = baseAcc G.metadata
\kw{type} state = acc

\kw{module} Record = \kw{struct}
  \kw{type} partial_state = baseAcc * acc Table.t
  ... 
  \kw{let} start state header =
    \kw{match} state \kw{with}
      G.RecordData ((errs, total), accs) ->
	\kw{let} errs' = if header.nerr > 0
                    then errs + 1 else errs
	\kw{in} ((errs', total + 1), accs))
    | _ -> \kw{raise} ...
	  
  \kw{let} project state label = \kw{match} state \kw{with}
      G.RecordData (_, accs) -> (\kw{try} Table.find accs label
                                 \kw{with} _ -> \kw{raise} ...)
    | _ -> \kw{raise} ...

  \kw{let} process_field (bAcc, accs) label acc =
    (bAcc, Table.update accs label acc)
      
  \kw{let} finish (bAcc, accs) = G.RecordData (bAcc, accs)
\kw{end}
    
\kw{module} Datatype = \kw{struct}
  \kw{type} partial_state = baseAcc * acc Table.t
	
  \kw{let} init () = G.DatatypeData (initBaseAcc, Table.empty)

  \kw{let} start state header = 
    \kw{match} state \kw{with}
      G.DatatypeData ((errs, total), accs) ->
	\kw{let} errs' = if header.nerr > 0
                    then errs + 1 else errs
	\kw{in} ((errs', total + 1), accs))
    | _ -> \kw{raise} ...
	  
  \kw{let} project state label = \kw{match} state \kw{with}
    G.DatatypeData (_, accs) -> (\kw{try} Some (Table.find accs label)
                                 \kw{with} _ -> None)
  | _ -> \kw{raise} ...
	  
  \kw{let} process_variant (bAcc, accs) label acc =
    G.DatatypeData (bAcc, Table.update accs label acc)
\kw{end}
...
\end{code}
\caption{Excerpts from the implementation of the accumulator.}
\label{fig:gentool-accum-code}
\end{figure}

\emph{Discuss implementation of accum briefly here.}

In addition to the accumulator, we have implemented two different
kinds of pretty printers for parsed data.  One formats the data in
XML. The other prints the data in a simple text format that is helpful
for debugging descriptions. Importantly, both tool's output corresponds to the in-memory
representation of the data rather than its original format (which may,
for example, have delimiters that are not present in the
representation).

% \begin{figure}
% \begin{code}\scriptsize
%   \kw{type} state = Xml.xml \kw{exception} Tool_error \kw{of} state
%   * string \kw{let} init () = () ...  \kw{module} Record =
%   \kw{struct} (* Partial state is a list of XML objects *
%   representing the fields processed so far * tagged with the
%   record's PD header *) \kw{type} partial_state = Pads.pd_header *
%   (Xml.xml list)

%   \kw{let} init named_states =
%     ... (* Create initial "blank" state with no data *)

%   \kw{let} start state header =
%     ... (* Initialize processing of a new record *)

%   \kw{let} project state field_name =
%     ... (* Extract named field's state *)

%   \kw{let} process_field (header, fields) field_name state =
%     (header, fields@[Xml.Element 
%     	(field_name, [], [state])])

%   \kw{let} finish pstate =
%     ... (* Convert a partial state into a final state *)
% \kw{end}
% ...
% \end{code}
% \caption{Excerpts from the implementation of the generic XML formatter
%   tool in {\tt xml\_formatter.ml}.}
% \label{fig:gentool-xmlf-code}
% \end{figure}

%%% Local Variables: 
%%% mode: latex
%%% TeX-master: "paper"
%%% End: 


\section{The Semantics of PADS/ML}
\label{sec:ddc}

% \emph{Do we want to show an example from pads/ml shown in DDC.
%   (e.g. name-value pairs?). It think it would be nice, but I don't
%   know that there is space.  }

\trversion{\em Modify a\_pd to a\_pd\_body, everywhere. This name is more
  appropriate.  Note that the new scheme for translating type
  variables will affect WF rules. a will be in D but a\_rep and
  a\_pd\_body could appear in a sigma

  Consider a new convention: for compute type, any references to type
  interepretations should be done with the type interpretation
  functions rather than being hard coded. Otherwise, type substitution gets
  messed up,e.g., when unfolding a recursive type. 
  For example, should be compute(e:[a]\_rep) instead of
  compute(e:a\_rep). Then, substitution on DDC types will burrow into
  the type annotation $\gs$ of compute types. To support, need new
  syntax for type annotations $\gs$ and need to explicitly translate
  annotation types into F-omega types.
  
}

In this section, we introduce \ddc{}, a calculus of simple, orthogonal
type constructors, which serves to give a semantics
to the main features of \padsml.  \ddc{} is an extension
and revision of our previous work on 
\ddcold{}~\cite{fisher+:next700ddl}.  The main new feature is 
the ability to define functions from types to types, 
which are needed to model \padsml's polymorphic data types.
In the process of adding these new functions, which we call {\em type
abstractions} (as opposed to {\em value abstractions}, which are functions 
from values to types), we simplified our overall semantics
by making a couple of subtle technical changes.  For example, we were
able to eliminate the complicated ``contractiveness'' constraint from our
earlier work.  We have also added a new interpretation of \ddc{} types
as printers.

The main practical benefit of the calculus has been as a guide for our
implementation. Before working through the formal semantics, we
struggled to disentangle the invariants related to polymorphism. After
we had defined the calculus, we were able to implement type
abstractions as \ocaml{} functors in approximately a week.  
Our new printing semantics was also very important for helping us define
and check the correctness of our printer implementation.  We hope
the calculus will serve as a guide for implementations of {\sc PADS} in
other host languages.
In the remainder of this section, we give an overview of the calculus.
\appref{app:ddc-semantics} contains a complete formal specification.

\subsection{\ddc{} Syntax}
\begin{figure}
{\small
\begin{bnf}
  \name{Kinds} \meta{\gk} \::= \kty \| \kty \-> \gk
                               \| \ity \-> \gk  \\
  \name{Types} \meta{\ty} \::= 
    %\ptrue\| \pfalse \| 
    \pbase{e} \| 
    \plam{\var}{\ity}{\ty} \| \papp{\ty}{e} \| 
    \psig x \ty \ty \| \psum \ty e \ty \nlalt
    \pset x \ty e \|
    \ptyvar \| \pmu{\ptyvar}{\gk}{\ty} \| 
    \ptylam{\ptyvar}{\kty}{\ty} \| \ptyapp{\ty}{\ty} \| ...
\end{bnf}
}
\caption{\ddc{} syntax, selected constructs}
\label{fig:ddc-syntax}
\end{figure}

\figref{fig:ddc-syntax} summarizes the syntax of the \ddc.
The interpretation of a type with kind $\kty$ is a parser that maps
data from an external form into an internal one.  
A type with kind $\kty \rightarrow \gk$ is a function mapping 
a parser to the interpretation of a type with kind $\gk$.
Finally, types with kind $\ity \rightarrow \gk$  map values
with host language type $\ity$ to the interpretation of
types with kind $\gk$.  For concreteness, we adopt \fomega{} as our
host language.

%The atomic types include $\ptrue$, which consumes no input and reports
%no errors,  and $\pfalse$, which consumes no input but reports an
%error.   

The simplest description is a base type $\pbase{e}$.
The base type's parameter $e$ is drawn from the host language. 
The \padsml{} type {\tt Pstring} is an example of such a base type.
Structured types include value abstraction $\plam{\var}{\ity}{\ty}$
and application $\papp{\ty}{e}$, which allow us to parameterize types
by host language values. 
\cut{
 Any type in the language may be parameterized by a value using
 value abstraction $\plam{\var}{\ity}{\ty}$.  
 We do not include the type $\sigma$ of the variable---the reader will
 have to reconstruct it from context. 
 If $\ty$ is such an abstraction,
 the parameter $\var$ may be instantiated
 using value application $\papp{\ty}{e}$. 
}
The dependent sum type, $\psig x \ty \ty$, describes a pair of values,
where the value of the first element of the pair can be referenced
when describing the second element.  Variation in a data source can be
described with the sum type $\psum \ty e \ty$, which deterministically
describes a data source that either matches the first type, or fails
to match the first branch but does match the second one.
%Intersections $\pand \ty \ty$
%describe data sources which can be described in two ways
%simultaneously. 
We specify semantic constraints over a data source
with type $\pset x \ty e$, which describes any value $x$ that satisfies the
description $\ty$ and the predicate $e$. Type variables $\ptyvar$ are
abstract descriptions; they are introduced by recursive types and type
abstractions. Recursive types $\pmu{\ptyvar}{\gk}{\ty}$ describe
recursive formats, like lists and trees. Type abstraction
$\ptylam{\ptyvar}{\kty}{\ty}$ and application $\ptyapp{\ty}{\ty}$
allow us to parameterize types by other types.  Type variables $\ptyvar$
always have kind $\kty$.

To specify the well-formedness of types, 
we use a kinding judgment of the form
$\ddck[\ty,\pctxt;\ctxt,\kind,\mcon]$,
where $\pctxt$ maps type variables 
to kinds and $\ctxt$ maps host language value variables to host language 
types. In our original work~\cite{fisher+:next700ddl}, these kinding rules
were somewhat unorthodox, but we have sinced simplifed them.  
Details appear in \appref{app:ddc-semantics}.

\subsection{\Implang{} Language}
\label{sec:host-lang}
The host language of \ddc{} is a straightforward 
extension of \fomega{} with recursion and a variety of useful constants
and operators.  
% We use \fomega{} 
% both to write the expressions that can appear within
% \ddc{} itself and to encode the parsing semantics of \ddc{}.
For reference, the grammar appears in \appref{app:host-lang}.
The constants include bitstrings $\data$; offsets $\off$, representing
locations in bitstrings; and error codes $\iok$, $\iecerr$, and
$\iecpc$, indicating success, success with errors, and failure,
respectively. We use the constant $\ierr$ to indicate a failed parse.
Because of its specific meaning, we forbid its use in user-specified
expressions appearing in \ddc{} types.
Our base types include the type $\invty$, the singleton type of the
constant $\ierr$, and types $\iecty$ and $\ioffty$, which classify
error codes and bit string offsets, respectively.

We extend the formal syntax with some syntactic sugar for use in the
rest of this section: anonymous functions $\ilam {\nrm x} \ity e$ for
$\ifun {\nrm f} {\nrm x} e$, with $f \not\in {\rm FV}(e)$; $\ispty$
for $\iprod \ioffty \ioffty$.  We often use pattern-matching syntax
for pairs in place of explicit projections, as in $\lampair{\codefont
  e}$ and $\ilet {\itup{\off,r,p}} e\; e'$.  Although we have no
formal records with named fields, we use a dot notation for commonly
occuring projections. For example, for a pair $\mathtt x$ of rep and
PD, we use $\codefont{x.rep}$ and $\codefont{x.pd}$ for the left and
right projections of $\codefont{x}$, respectively. Also, sums and
products are right-associative.  Finally, we only specify type
abstraction over terms and application when we feel it will clarify
the presentation. Otherwise, the polymorphism is implicit.  We also
omit the usual type and kind annotations on $\lambda$, with the
expectation the reader can construct them from context.  

The static semantics ($\stsem[e,{\pctxt;\ctxt},\ity]$), operational
semantics ($e \stepsto e'$), and type 
equality ($\ity \equiv \ity'$) are those of \fomega{} extended with
recursive functions and recursive types and are entirely standard.
See Pierce's text~\cite{pierce:tapl} for details.

\subsection{\ddc{} Semantics}
\label{sec:ddc-sem}

The primitives of \ddc{} each have four interpretations: two
types in the host language, one for the data representation
itself and one for its parse descriptor, and two functions,
one for parsing and one for printing.
We therefore specify the semantics of \ddc{} types using four semantic
functions, each of which precisely conveys a particular facet of the
meaning of a type.  The functions $\itsem[\cdot]$ and $\itpdsem[\cdot]$
describe the type of the data's in-memory representation and 
parse descriptor, respectively. The semantic
functions $\trans[\cdot,,]$ and $\transpp[\cdot,,]$ define
the parsing and printing functions generated from \ddc{} descriptions.

% In our previous work~\cite{fisher+:next700ddl}, we discuss the
% semantics of \ddcold{} in detail. Here, we will focus only on those
% aspects of the semantics that are new or modified from the original
% presentation.
\trversion{
\begin{table}
  \begin{center}
    \renewcommand{\arraystretch}{1.35}
    \begin{tabular}{l l}
      $\ddck[\ty,{\pctxt;\ctxt},\kind,\mcon]$ & {\it \ddc{}-type
        kinding}\\
      $\itsem[\ty] = \ity$ & {\it representation types of \ddc{} types}\\
      $\itpdsem[\ty] = \ity$ & {\it pd types of \ddc{} types}\\
      $\trans[\ty,\ctxt,\gk] = e$   & {\it \ddc{}-type semantics} \\
      $\kTrans[\gk,\ty] = \ity$     & {\it parser type} \\
      $\ptyc \pctxt = \ctxt$     & {\it parser-type context }\\
      $\fotyc \pctxt = \pctxt$     & {\it \fomega version of poly. context }\\
      $\fortyc \pctxt = \pctxt$     & {\it Rep. type variables in $\fotyc \pctxt$ }\\
      $\fopdtyc \pctxt = \pctxt$     & {\it PD type variables in $\fotyc \pctxt$ }\\
      $\stsem[e,\pctxt;\ctxt,\ity]$ & {\it \fomega expression typing} \\
    \end{tabular}
    \caption{Translations and Judgments}
    \label{tab:judg-list}
  \end{center}
\end{table}

For reference, we provide in
\tblref{tab:judg-list} a listing of all the functions and judgments
defined in this section and a brief description of each.  
}

\begin{figure}
\fbox{$\itsem[\ty] = \ity$}
\[
\begin{array}{lcl} 
%\itsem[\ptrue] & = & \iunitty \\
%\itsem[\pfalse] & = & \invty \\
\itsem[\pbase{e}] & = & \isum {\Irty(C)} \invty   \\
\itsem[\plam{\var}{\ity}{\ty}] & = & \itsem[\ty] \\
\itsem[\papp \ty e] & = & \itsem[\ty] \\
\itsem[\psig \var {\ty_1} {\ty_2}]  & = & \iprod {\itsem[\ty_1]} {\itsem[\ty_2]}    \\
\itsem[\psum {\ty_1} e {\ty_2}]     & = & \isum {\itsem[\ty_1]} {\itsem[\ty_2]} \\
\itsem[\pset x \ty e] & = & \isum {\itsem[\ty]}{\itsem[\ty]}\\
\itsem[\ptyvar] & = & \ptyvar_\repname \\
\itsem[\pmu{\ptyvar}{\gk}{\ty}] & = & \imu{\ptyvar_\repname}{\itsem[\ty]} \\
\itsem[\lambda \ptyvar.\ty]       & = & \lambda \ptyvar_\repname.\itsem[\ty] \\
\itsem[\ty_1 \ty_2]              & = & \itsem[\ty_1] \itsem[\ty_2] \\
\end{array}
\]
\caption{Representation type translation, selected constructs}
\label{fig:rep-tys}
\end{figure}


\paragraph*{\ddc{} representation types.}
\label{sec:intty-sem}
In Figure~\ref{fig:rep-tys}, we present the representation type
of selected \ddc{} primitives. While the primitives are
dependent types, the mapping to the \implang{} language 
erases the dependency because the \implang{} language 
does not have dependent types. This involves erasing all host language
expressions that appear in types as well as 
value abstractions and applications.
A type variable $\ptyvar$ in \ddc{} is mapped to a corresponding
type variable $\ptyvar_\repname$ in \fomega{}.
Recursive types generate recursive representation types with the type
variable named appropriately. Polymorphic types and their application 
become \fomega{} type constructors and type application, respectively.

\begin{figure}
\fbox{$\itpdsem[\ty] = \ity$}
\[ 
\begin{array}{lcl} 
%% %% example: \ua.(int * a) + None
%% %%          pd = \ua.pd_hdr  * ((pd_hdr * ([int]_pd * [a]_pd)) + [None]_pd)
%% %%             = \ua.pd_hdr  * ((pd_hdr * ([int]_pd * a)) + [None]_pd)
%\itpdsem[\ptrue] & = & \ipty \iunitty \\                                                  
%\itpdsem[\pfalse] & = & \ipty \iunitty \\                                                  
\itpdsem[\pbase{e}] & = & \ipty \iunitty\\
\itpdsem[\plam \var \ity \ty] & = & \itpdsem[\ty] \\
\itpdsem[\papp \ty e] & = & \itpdsem[\ty] \\
\itpdsem[\psig \var {\ty_1} {\ty_2}] & = & 
               \ipty {\iprod {\itpdsem[\ty_1]} {\itpdsem[\ty_2]}} \\
\itpdsem[\psum {\ty_1} e {\ty_2}] & = & 
               \ipty {(\isum {\itpdsem[\ty_1]} {\itpdsem[\ty_2]})} \\
\itpdsem[\pset x \ty e] & = & \ipty {\itpdsem[\ty]} \\
\itpdsem[\ptyvar] & = & \ipty{\ptyvar_\pdbname} \\
\itpdsem[\pmu \ptyvar \kty \ty] & = & 
  \ipty{\imu{\ptyvar_\pdbname}{\itpdsem[\ty]}} \\
\itpdsem[\lambda \ptyvar.\ty]      
     & = & \lambda \ptyvar_\pdbname.\itpdsem[\ty] \\
\itpdsem[\ty_1 \ty_2]            & = & \itpdsem[\ty_1] \itpdsemstrip[\ty_2] \\
\end{array}
\]

\fbox{$\itpdsemstrip[\ty] = \ity$}

\[
\begin{array}{lcl} 
\itpdsemstrip[\ty] & = & \ity \ \ \mbox{where}\ \itpdsem[\ty] \equiv \ipty{\ity}
\end{array}
\]
\caption{Parse-descriptor type translation, selected constructs}
\label{fig:pd-tys}
\end{figure}

\paragraph*{\ddc{} parse descriptor types.}
\figref{fig:pd-tys} gives the types of the parse descriptors
corresponding to selected \ddc{} types.  The translation reveals that
all parse descriptors share a common structure, consisting of two
components, a header and a body.  The header reports on the
corresponding representation as a whole. It stores the number of
errors encountered during parsing, an error code indicating the degree
of success of the parse---success, success with errors, or
failure---and the span of data (location in the source)  
described by the descriptor.  To be precise, the type
of the header ($\tyface{pd\_hdr}$) is $\iintty \iprodi \iecty \iprodi
\ispty$. The body contains parse descriptors for the subcomponents of
the representation. For types without subcomponents, we use $\iunitty$
as the body type.  As with the representation types, dependency is
uniformaly erased.
% For types with base kind, the corresponding parse
% descriptor will always have a header and body. 

% In \figref{fig:pd-tys}, we give the types of the parse descriptors
% corresponding to each of the selected \ddc{} types. The majority of
% the types shown are products of the type $\tyface{pd\_hdr}$ and
% another host-language type. This common form reflects the fact that all
% parse descriptors consist of two components, a {\it header} and a {\it
%   body}.  The header describes the representation as a whole. It
% provides the number of errors encountered during parsing, an error
% code indicating the degree of success of the parse -- success, success
% with errors, or failure -- and the span of data described by the
% descriptor.  Formally, the type of the header ($\tyface{pd\_hdr}$) is
% $\iintty \iprodi \iecty \iprodi \ispty$. 

% The parse-descriptor body consists of parse descriptors for the
% subcomponents of the representation. For types without subcomponents,
% we use $\iunitty$ as the body type.

Like other types, \ddc{} type variables $\ptyvar$ are translated into 
a pair of header and a body.  The body has abstract type 
$\ptyvar_\pdbname$.
This translation makes it possible for polymorphic parsing code to examine the
header of a PD, even though it does not know the \ddc{} type it is parsing.
\ddc{} abstractions are translated into \fomega\ type constructors that
abstract the body of the PD (as opposed to the entire PD)
and \ddc{} applications are translated into \fomega\ type applications
where the argument type is the PD body type.

% PD-type constructors are
% parameterized over the PD \emph{body} type, rather than over the PD
% type itself. Correspondingly, type application $\papp {\ty_1}{\ty_2}$ applies
% the PD interpretation of $\ty_1$ to the body 

% This common shape enables us to write
% polymorphic functions that treat parse descriptors in a generic
% mannner. Indeed, the parsing semantics of types rely on this
% structure, as discussed in \secref{sec:parse-sem}.

% The PD types without the common shape are those corresponding to
% abstractions and applications. Value abstractions and applications are
% translated in an identical manner for PD types as for representation
% types.  Type abstractions and applications, however, are translated in
% a subtly different manner. As with the rep-type interpretation, type
% abstractions correspond to parse decriptor-type constructors in
% \fomega. The major difference is that PD-type constructors are
% parameterized over the PD \emph{body} type, rather than over the PD
% type itself. Correspondingly, type application $\papp {\ty_1}{\ty_2}$ applies
% the PD interpretation of $\ty_1$ to the body portion of the PD
% interpretation of $\ty2$ (specified formally with the $\itbdsem[\ty]$
% function). Similarly, type variables are translated to tuples of a
% header and a type variable, rather than to a type variable alone. 

% The reason for parameterization over PD-body types (rather than
% parse-descriptor types) in the translation scheme will be explained in
% the next section, once we have discussed the parsing semantics of type
% abstractions and applications. Note, however, that an important result
% of this design is that the PD interpretation is not defined for all
% types. The problem lies with the interpretation of type application.
% It requires that $\itbdsem[\ty_2]$ be defined, which, in turn,
% requires that $\itpdsem[\ty_2] \equiv \ipty{\ity}$, for some $\ity$.
% Yet, this requirement is not mebt by some types; for example, $\lambda
% \ptyvar.\ty$.


\begin{figure}
\small
\fbox{$\kTrans[\gk,\ty] = \ity$} 
    
\begin{align*}
  &\kTrans[\kty,\ty] & = & \quad \extdom * \offdom \iarrowi \offdom * \itsem[\ty] * \itpdsem[\ty]
   \\
   &\kTrans[\ity \iarrowi \gk,\ty] & = & \quad \ity \iarrowi \kTrans[\gk,\ty\ e],
   \; \mbox{for any e}.
   \\
   &\kTrans[\kty \iarrowi \gk,\ty] & = & \quad 
      \forall\tyvar_\repname.\forall\tyvar_\pdbname.
         \kTrans[\kty,\tyvar] \iarrowi \kTrans[\gk,\ty \tyvar] \\
         & & & \quad (\ptyvar_\repname, \ptyvar_\pdbname \not \in \ftv \kind \cup
         \ftv \ty)
\end{align*}  
  \caption{\Implang{} language types for parsing functions}
  \label{fig:parser-types}
\end{figure}

\begin{figure}
\small
\fbox{$\trans[\ty,\ctxt,\gk] = e$} 

\[
\begin{array}{l}
  %% None 
%\trans[\ptrue,\ctxt,\kty] =
%  \lampair{\spair<\off,\newrep{unit}{},\newpd{unit}{\off}>}
%\\[3pt] %\\
%% False 
%\trans[\pfalse,,] =
%  \lampair{\spair<\off,\newrep {bottom}{},\newpd {bottom}{\off}>}
%\\[3pt] %\\ 
%% Const 
\trans[\pbase{e},\ctxt,\kty] =
  \lampair{\iapp {\iapp {\Iimp(C)} (e)} {\itup {\idata,\off}}}
\\[3pt] %\\
%% Abs 
\trans[\plam{\var}{\ity}{\ty},,] =
   \sfn{\nrm\var}{\ity}{\trans[\ty,\ectxt{\var{:}\ity},\kind]}
\\[3pt] %\\
%% App 
\trans[\papp{\ty}{e},\ctxt,\gk] =
  \trans[\ty,,] \sapp e  
\\[3pt]
%% Prod 
%\begin{array}{l}
\trans[\psig{x}{\ty}{\ty'},\ctxt,\kty] = \\
  \begin{array}{l}  
    \lampair{} \\
    \quad  \ilet {\spair<\off',r,p>} 
    {{\trans[\ty,,]} \sapp \spair<\idata,\off>} \\
    \quad  \ilet x {\ictup{r,p}}\\
    \quad  \ilet {\spair<\off'',r',p'>} 
    {{\trans[\ty',,]} \sapp \spair<\idata,\off'>} \\
    \quad \spair<\off'',\newrep {\gS}{r,r'},\newpd {\gS}{p,p'}>
  \end{array}  
%\end{array}
\\[3pt]
%% Sum 
%\begin{array}{l}
  \trans[\psum{\ty}{e}{\ty'},,] = \\
  \begin{array}{l}  
  \lampair{} \\
  \quad \ilet {\itup{\off',r,p}}{\trans[\ty,,] \sapp \spair<\idata,\off>} \\
  \quad \iif {\pdok p} \; \ithen {
    \def \r {\newrep {+left}{r}}
    \def \p {\newpd {+left}{p}}
    \spair<\off',\r,\p>} \\
  \quad \ielse {\ilet {\itup{\off',r,p}}{\trans[\ty',,] \sapp \spair<\idata,\off>}} \\
  \quad 
  {  % begin scope
    \def \r {\newrep {+right}{r}}
    \def \p {\newpd {+right}{p}}
    %% 
    \spair<\off',\r,\p>
  }\\ % end scope
  \end{array}
\\[3pt]
%\quad
%% Set 
  \trans[\pset{x}{\ty}{e},\ctxt,\kty] = \\
  \begin{array}{l}  
    \lampair{} \\
    \quad \ilet {\itup{\off',r,p}}{\trans[\ty,,] \sapp \spair<\idata,\off>} \\
    \quad \ilet x {\ictup{r,p}}\\
    \quad \ilet c e \\
    \quad \spair<\off',\newrep {con} {c,r},\newpd {con} {c,p}>
  \end{array}
\\[3pt]
%% Var
\trans[\ptyvar,,] = \codefont{\parsename_\ptyvar}
\\[3pt]
%% Mu
\trans[\pmu \ptyvar \gk \ty,,] = \\
  \begin{array}{l}
  \ifun {\parsename_\ptyvar} {\itup{\data{:}\ibitsty,\off{:}\ioffty}  
           : \ioffty * \itsem[\pmu \ptyvar \gk \ty] 
                    * \itpdsem[\pmu \ptyvar \gk \ty] } {}\\
  \quad \ilet {\itup{\off',r,p}} 
   {\trans[\ty,,][\itsem[\pmu \ptyvar \gk \ty]/\ptyvar_\repname]
          [\itpdsemstrip[\pmu \ptyvar \gk \ty]/\ptyvar_\pdbname] \iappi \ictup{\data,\off}} \\ 
  \qquad \ictup{\off',\iroll{r}{\itsem[\pmu \ptyvar \gk \ty]},
     (p.h,\iroll{p}{\itpdsemstrip[\pmu \ptyvar\gk \ty]})}
%}}
  \end{array}  
\\[3pt]
%% lambda \alpha
\trans[\lambda\tyvar . \ty,,] = %\\
%  \begin{array}{l}
    \Lambda \tyvar_\repname. 
    \Lambda \tyvar_\pdbname. \lambda \codefont{\parsename_\ptyvar}. \trans[\ty,,]
%  \end{array}  
\\[3pt]
%% t1 t2
\trans[\ty_1 \ty_2,,] = 
    \trans[\ty_1,,]\; [\itsem[\ty_2]]\; [\itpdsemstrip[\ty_2]]\; \trans[\ty_2,,]
\\
\end{array}
\]
%\caption{\ddc{} Semantics (cont.)}
\caption{\ddc{} parsing semantics, selected constructs}
\label{fig:ddc-sem-sum}
\end{figure}


\paragraph*{\ddc{} parsing semantics.}
\label{sec:parse-sem}
The parsing semantics of a type $\tau$ with kind $\kty$ is a function that
transforms some amount of input into a pair of a representation and a
parse descriptor, the types of which are determined by $\tau$.  The
parsing semantics for types with higher kind are functions that
construct parsers, or functions that construct functions that
construct parsers, \etc{} \figref{fig:parser-types} specifies
the host-language types of the functions generated from well-kinded
\ddc{} types.

For each (unparameterized) type, the input to the corresponding parser
is a bit string to parse and an offset at which to begin parsing.  
The output is a new offset,
a representation of the parsed data, and a parse descriptor.

For any type, there are three steps to parsing: parse the
subcomponents of the type (if any), assemble the resultant
representation, and tabulate meta-data based on subcomponent meta-data
(if any). For the sake of clarity, we have factored the latter two
steps into separate representation and PD constructor functions which
we define for each type. 
%For some types, we additionally factor the PD
%header construction into a separate function. 
For example, the
representation and PD constructors for the dependent sums are
$\newrepf {\gS}$ and $\newpdf {\gS}$, 
respectively
%, and the header constructor for
%products is ${\codefont{H_{\gS}}}$.  
We have also factored out some commonly
occuring code into auxiliary functions.  These constructors and functions
appear in \appref{app:asst-functions}.

The PD constructors determine the error code and calculate the error
count.  There are three possible error codes: $\iok$, $\iecerr$, and
$\iecpc$, corresponding to the three possible results of a parse: it
can succeed, parsing the data without errors; it can succeed, but
discover errors in the process; or, it can find an unrecoverable error
and fail.  The error count is determined by subcomponent error counts
and any errors associated directly with the type itself.

In \figref{fig:ddc-sem-sum}, we specify the parsing semantics of
selected portion of \ddc{}. We explain the interpretations of select
types, from which the interpretation of the remaining types may be
understood. The full semantics appears in \appref{app:ddc-semantics}.
A dependent sum parses the data according to the first type, binding
the resulting representation and PD to $x$ before parsing the
remaining data according to the second type. It then bundles the
results using the dependent sum constructor functions.

A type variable translates to an expression variable whose name
corresponds directly to the name of the type variable. These
expression variables are bound in the interpretations of recursive
types and type abstractions. We interpret each recursive type as a
recursive function whose name corresponds to the name of the recursive
type variable. For clarity, we annotate the recursive function with its
type.

We interpret type abstraction as a function over other parsing
functions. Because those parsing functions can have
arbitrary \ddc{} types (of kind $\kty$), the interpretation must be a
polymorphic function, parameterized by the representation and PD-body
type of the \ddc{} type parameter.  For clarity, we present
this type parameterization explicitly.  Type application $\papp
{\ty_1}{\ty_2}$ simply becomes the application of the interpretation
of $\ty_1$ to the representation-type, PD-type, and parsing
interpretations of $\ty_2$.

% Now, we can explain why the PD and parsing interpretations
% of type abstraction and application involve the PD-body interpretation
% rather than just the PD interpretation. Notice that the constructor
% functions for dependent sums (and for many other types as well) are
% polymorphic. No matter the types of the subcomponent types $\ty$ and
% $\ty'$, we use the same functions $\newrepf \Sigma$ and $\newpdf
% \Sigma$. The PD constructor is not fully polymorphic, though, in that
% it assumes that the parse descriptors given to it have the familiar PD
% shape of header and body.  Therefore, in order to judge the type of
% this function, we must be sure that the PD types of the subcomponents
% $\ty$ and $\ty'$ satisfy this constraint.

% What happens, then, when we have $\ptyvar$ as a subcomponent? We must
% ensure that its PD type has the appropriate shape. To do so, we
% specify the translation of type variables explicitly as a product of a
% header and a type variable for the body, rather than simply as a type
% variable for the whole PD type. Therefore, all parameterization
% related to PD types must be over the PD-body type, rather than the PD
% type as a whole.

\begin{figure}
\small
\fbox{$\kTransPP[\gk,\ty] = \ity$} 
    
\begin{align*}
  &\kTransPP[\kty,\ty] & = & \quad \itsem[\ty] * \itpdsem[\ty] \iarrowi \extdom 
   \\
   &\kTransPP[\ity \iarrowi \gk,\ty] & = & \quad \ity \iarrowi \kTransPP[\gk,\ty\ e],
   \; \mbox{for any e}.
   \\
   &\kTransPP[\kty \iarrowi \gk,\ty] & = & \quad 
      \forall\tyvar_\repname.\forall\tyvar_\pdbname.
         \kTransPP[\kty,\tyvar] \iarrowi \kTransPP[\gk,\ty \tyvar] \\
         & & & \quad (\ptyvar_\repname, \ptyvar_\pdbname \not \in \ftv \kind \cup
         \ftv \ty)
\end{align*}  
  \caption{\Implang{} language types for printing functions}
  \label{fig:printer-types}
\end{figure}

\begin{figure}
\small
\fbox{$\transpp[\ty,\ctxt,\gk] = e$} 

\[
\begin{array}{l}
%% Const 
\transpp[\pbase{e},\ctxt,\kty] =
  \lampppair{\iapp {\iapp {\Igen{pp}(C)} (e)} {\ictup {r,pd}}}
\\[3pt] %\\
%% Abs 
\transpp[\plam{\var}{\ity}{\ty},,] =
   \sfn{\nrm\var}{\ity}{\transpp[\ty,\ectxt{\var{:}\ity},\kind]}
\\[3pt] %\\
%% App 
\transpp[\papp{\ty}{e},\ctxt,\gk] =
  \transpp[\ty,,] \sapp e  
\\[3pt]
%% Prod 
%\begin{array}{l}
\transpp[\psig{x}{\ty_1}{\ty_2},\ctxt,\kty] = \\
  \begin{array}{l}  
    \lampppair{} \\
      \quad  \ilet x {\ictup{\ipid{1}{r},\ipid{1}{\ipid{2}{pd}}}}\\
      \quad  \ilet {bs_1} 
%%    \def \rf {foo} %% {\ipid{1}{r}}
%%   \def \pf {bar} %% {\ipid{1}{\ipid{2}{pd}}}
%%  \def \rs {baz} %% {\ipid{2}{r}}
%% \def \ps {foo} %% {\ipid{2}{\ipid{2}{pd}}}
      {{\transpp[\ty_1,,]} \sapp \codefont{x}} \\
      \quad  \ilet {bs_2} 
      {{\transpp[\ty_2,,]} \sapp \spair<\ipid{2}{r},\ipid{2}{\ipid{2}{pd}}>} \\
      \quad \iappend{bs_1}{bs_2}
  \end{array}  
%\end{array}
\\
%% Sum 
%\begin{array}{l}
  \transpp[\psum{\ty_1}{e}{\ty_2},,] = \\
  \begin{array}{l}  
  \lampppair{} \\
  \quad \icasess {\ictup{r,\ipid{2}{pd}}} \\
  \quad \ipattss{\ictup{\iinl{r_1}, \iinl{p_1}}} 
	{ {\transpp[\ty_1,,]} \sapp \codefont{\ictup{r_1,p_1}}}\\
  \quad \ipattss{\ictup{\iinr{r_2}, \iinr{p_2}}} 
	{ {\transpp[\ty_2,,]} \sapp \codefont{\ictup{r_2,p_2}}}\\
  \quad \ipattss{\_}  {\failpp}\\
  \end{array}
\\
%\quad
%% Set 
  \transpp[\pset{x}{\ty}{e},\ctxt,\kty] = \\
  \begin{array}{l}  
  \lampppair{} \\
  \quad \icasess {\ictup{r,\ipid{2}{pd}}} \\
  \quad \ipattss{\ictup{\iinl{r_1}, p_1}} 
	{ {\transpp[\ty,,]} \sapp \codefont{\ictup{r_1,p_1}}}\\
  \quad \ipattss{\ictup{\iinr{r_2}, p_2}} 
	{ {\transpp[\ty,,]} \sapp \codefont{\ictup{r_2,p_2}}}\\
  \end{array}
\\[3pt]
%% Var
\transpp[\ptyvar,,] = \codefont{\printname_\ptyvar}
\\[3pt]
%% Mu
\transpp[\pmu \ptyvar \gk \ty,,] = \\
  \begin{array}{l}
  \ifun {\printname_\ptyvar} {\itup{r:\itsem[\pmu \ptyvar \gk \ty], pd:\itpdsem[\pmu \ptyvar \kty \ty]}
                     : \ibitsty} {}\\
  \quad \transpp[\ty,,][\itsem[\pmu \ptyvar \gk \ty]/\ptyvar_\repname]
          [\itpdsemstrip[\pmu \ptyvar \gk \ty]/\ptyvar_\pdbname] \\
  \qquad\ictup{\iunroll{r}{\itsem[\pmu \ptyvar \gk \ty]},
     \iunroll{\ipid{2}{pd}}{\itpdsemstrip[\pmu \ptyvar \kty \ty]}}
  \end{array}  
\\[3pt]
%% lambda \alpha
\transpp[\lambda\tyvar . \ty,,] = %\\
%  \begin{array}{l}
    \Lambda \tyvar_\repname. 
    \Lambda \tyvar_\pdbname. \lambda \codefont{\printname_\ptyvar}. \transpp[\ty,,]
%  \end{array}  
\\
%% t1 t2
\transpp[\ty_1 \ty_2,,] = 
    \transpp[\ty_1,,]\; [\itsem[\ty_2]]\; [\itpdsemstrip[\ty_2]]\; \transpp[\ty_2,,]
\\
\end{array}
\]
%\caption{\ddc{} Printing semantics (cont.)}
\caption{\ddc{} printing semantics, selected constructs}
\label{fig:ddc-print-sem}
\end{figure}


\paragraph*{\ddc{} printing semantics}
\label{sec:print-sem}
The definition of the printing semantics for a \ddc{} description
uses a similar set of concepts as the parsing semantics.  To begin,
the semantic function $\kTransPP[\gk,\ty] = \ity$ gives the
host language type $\ity$ for the printer generated from type $\ty$
with kind $\gk$.  As shown in \figref{fig:parser-types},
the printing semantics for descriptions with higher kind are functions that
construct printers and the printing semantics for descriptions with base kind
are simple first-order functions that map a representation and a
parse descriptor into a string of bits.

\figref{fig:ddc-print-sem} presents the printing semantics of
selected \ddc{} constructs.  Base types $\pbase{e}$ are printed in
various different ways according to the definition
$\Igen{pp}$, which is a parameter to the semantics.  The base type
printer $\Igen{pp}$ accepts the parse descriptor as a parameter, and
in the case of an error, a base type printer will print nothing.
Dependent sums print one component and then the next in order.\footnote{The
notation $\iappend{bs_1}{bs_2}$ appends bit string $bs_1$ to $bs_2$.}
Printing an ordinary sum type occurs by printing the underlying
tagged value.  Notice that the structure of the parse descriptor
and the representation should be isomorphic -- both should be
left injections or both should be right injections.  Any pair
of structures generated from the parser will satisfy this invariant.
If the pair of parse descriptor and representation do not match then the
programmer is using the printer has done so incorrectly.  In this case,
the printer calls an unspecified error routine named $\failpp{}$.

The semantics of printing recursive and parameterized types follows
similar lines to the semantics for parsing these constructs.
In particular, whenever a type parameter is introduced in the syntax
of a description a corresponding value parameter with printer function type 
is introduced in the generated printer code. We give the value parameter
the name $\printname_\alpha$.  Both type abstractions and recursive functions
introduce such parameters.  Notice that whereas the parsing interpretation
of a description used a fold to build a recursive data structure
when interpreting a recursive type, the printing interpretation uses
unfold to deconstruct a recursive data structure for printing.  

\subsection{Meta-theory}
\label{sec:meta-theory}
In order to further justify our semantic definitions,
we have proven two key metatheoretic results.  First, we show that
parsers and printers are {\em type-correct}, always returning representations
and parse descriptors of the appropriate type.  Second, we give a precise
characterization of the results of parsers
(and input requirements of printers) by defining the {\em canonical forms}
of representation-parse descriptor pairs associated with a
dependent \ddc{} type.

% Our key theoretical result is that the various semantic functions we
% have defined are coherent.  In particular, we show that for any 
% well-kinded \ddc{} type $\tau$, the corresponding parser is
% well typed, returning a pair of the corresponding representation and
% parse descriptor.  

\paragraph*{Type Correctness.}
Demonstrating that generated parsers and printers are well formed
and have the expected types is nontrivial primarily because
the generated code expects parse descriptors to have a particular shape,
and it is not completely obvious they do in the presence of polymorphism.
Hence, to prove type correctness, we first need to characterize the shape of
parse descriptors for arbitrary \ddc{} types.   
Unfortunately, the most straightforward characterization is
too weak to prove directly, and hence Definition~\ref{def:pd-props}
specifies a much stronger property as a logical relation.
Lemma~\ref{lemma:pd-log-rel} establishes that the logical relation
holds of all well-formed \ddc{} types by induction on kinding
derivations, and the desired characterization follows as a corollary.

\begin{definition}
\label{def:pd-props}
\begin{itemize}
\item $\hhpred \ty \kty$ iff $\ \exists\,\ity$ s.t. $\itpdsem[\ty] \equiv
  {\ipty \ity}$.
\item $\hhpred \ty {\kty \iarrowi \kind}$ iff $\ \exists\,\ity$
  s.t. $\itpdsem[\ty] \equiv \ity$ and whenever $\hhpred
  {\ty'}{\kty}$, we have $\hhpred {\papp \ty {\ty'}}{\kind}$.
\item $\hhpred \ty {\ity \iarrowi \kind}$ iff $\ \exists\,\ity'$
  s.t. $\itpdsem[\ty] \equiv \ity'$ and $\hhpred{\papp \ty e}{\kind}$
  for any expression $e$.
\end{itemize}
\end{definition}
\cut{
We can now prove the following lemma by induction on kinding derivations:
}
\begin{lemma}
\label{lemma:pd-log-rel}
If $\ddck[\ty,{\pctxt;\ctxt},\kind,{}]$ then $\hhpred \ty \kind$.
\end{lemma}

\begin{lemma}
\label{lemma:pd-props}
  \begin{itemize}
  \item If $\ddck[\ty,\pctxt;\ctxt,\kind,{}]$ then $\exists
     \ity.\itpdsem[\ty] = \ity$.
   \item If $\ddck[\ty,\pctxt;\ctxt,\kty,{}]$ then $\exists
     \ity.\itpdsem[\ty] \equiv \ipty \ity$.
  \end{itemize}
\end{lemma}


\trversion{
\begin{definition}
$\pda \ptyvar = \ipty \ptyvar$
\end{definition}

\begin{lemma}[Types of Constructors]
\label{lem:types-of-constructors}
\begin{itemize}
\item $\newrepf {unit} : \iarrow \iunitty \iunitty$
\item $\newpdf  {unit} : \iarrow \ioffty {\ipty \iunitty}$
\item $\newrepf {bottom} : \iarrow \iunitty \invty$
\item $\newpdf  {bottom} : \iarrow \ioffty {\ipty \iunitty}$
\item $\newrepf {\gS} : \forall \ga,\gb.\iarrow {\iprod \ga \gb} {\iprod \ga \gb}$
\item $\newpdf {\gS} : \forall \ga,\gb. 
  \iarrow {\iprod {\pda \ga} {\pda \gb}}
  {\pda {(\pda \ga \iprodi \pda \gb)}}
$
\item $\newrepf {+left} : \forall \ga.\forall \gb.\iarrow \ga 
                            {\isum \ga \gb}$
\item $\newrepf {+right} : \forall \ga.\forall \gb.\iarrow \gb {\isum \ga \gb}$
\item $\newpdf {+left} : \forall \ga, \gb.\iarrow {\pda \ga} 
  {\ipty {(\isum {\pda \ga}{\pda \gb})}}$
\item $\newpdf {+right} :\forall  \ga, \gb. \iarrow {\pda \gb} 
                            {\ipty {(\isum {\pda \ga} {\pda \gb})}}$
\item $\newrepf {\&} : \forall \ga,\gb.\iarrow {\iprod \ga \gb} {\iprod \ga \gb}$
\item $\newpdf {\&} : 
\forall \ga,\gb.
  \pda \ga \iprodi
  \pda \gb \iarrowi 
         {\ipty {(\pda \ga \iprodi \pda \gb)}}
$.
\item $\newrepf {con} : \forall \ga.\iprod \iboolty \ga 
  \iarrowi {\isum \ga \ga}$
\item $\newpdf {con} :\forall  \ga. \iprod \iboolty \iarrow {\pda \ga} {\ipty {\pda \ga}}$
\item $\newrepf {seq\_init} : \forall \ga.\iarrow \iunitty {\iintty \iprodi \iseq \ga}$
\item $\newpdf {seq\_init} : \forall \ga. \iarrow \ioffty {\iapty {\pda\ga}}$
\item $\newrepf {seq} : \forall \ga.\iarrow
  {(\iintty \iprodi \iseq \ga) \iprodi \ga}
  {\iintty \iprodi \iseq \ga}$
\item $\newpdf {seq} :\forall  {\ga_{elt}},{\ga_{sep}}. 
  (\iapty {\pda {\ga_{elt}}}) \iprodi
  \pda {\ga_{sep}} \iprodi 
  \pda {\ga_{elt}} \iarrowi \\
  \iapty {\pda {\ga_{elt}}}$
\item $\newrepf {compute} : \forall \ga.\iarrow \ga \ga$
\item $\newpdf {compute} : \iarrow \ioffty {\ipty \iunitty}$
\item $\newrepf {absorb} : \forall \ga.\iarrow {\pda \ga} {\isum
    \iunitty \invty}$
\item $\newpdf {absorb} :\forall  \ga. \iarrow {\pda \ga} {\ipty
    \iunitty}$
\item $\newrepf {scan} : \forall \ga.\iarrow \ga {\isum \ga \invty}$
\item $\newpdf {scan} :\forall  \ga. \iarrow {\iprod \iintty {\pda \ga}}
  {\ipty {(\isum {\iprod \iintty {\pda \ga}} \iunitty)}}$
\item $\newrepf {scan\_err} : \forall \ga.\iarrow \iunitty {\isum \ga \invty}$
\item $\newpdf {scan\_err} :\forall  \ga. \iarrow \ioffty
  {\ipty {(\isum {\iprod \iintty \ga} \iunitty)}}$
\end{itemize}  
\end{lemma}

\begin{proof}
  By typing rules of \fomega.
\end{proof}
}
With this lemma, we can establish the type correctness of the
generated parsers. We prove the theorem using a more general induction
hypothesis that applies to open types.
This hypothesis must account for the fact
that any free type variables in a \ddc{} 
type $\ty$ will become free
function variables in $\trans[\ty,,]$. To that end, 
we define the functions $\ptyc \pctxt$ and $\pptyc \pctxt$ 
which map type-variable contexts $\pctxt$ in the \ddc{}
to value-variable contexts $\ctxt$ in \fomega.  In addition, the function
$\fotyc{\pctxt}$ generates the appropriate \fomega\ type-variable context from
the \ddc{} context $\pctxt$.
\vskip -1.5ex
{\small
\[
\begin{array}{ll}
  \fotyc \cdot = \cdot \qquad &
  \fotyc {\pctxt,\alpha{:}\kty} = 
    \fotyc {\pctxt},\tyvar_\repname{:}\kty,\tyvar_\pdbname{:}\kty \\
  \ptyc{\cdot} = \cdot \qquad &
  \ptyc{\pctxt,\ptyvar{:}\kty} = \ptyc \pctxt,\codefont{\parsename_\ptyvar}{:}\kTrans[\kty,\ptyvar] \\
  \pptyc{\cdot} = \cdot \qquad &
  \pptyc{\pctxt,\ptyvar{:}\kty} = \pptyc \pctxt,\codefont{\printname_\ptyvar}{:}\kTransPP[\kty,\ptyvar]
\end{array}
\]
}

\begin{lemma}[Type Correctness Lemma]
\label{thm:type-correctness}
\begin{itemize}
\item If $\ddck[\ty,{\pctxt;\ctxt},\gk,{}]$ then
  $\stsem[{\trans[\ty,,]},{\fotyc \pctxt, \ctxt,\ptyc \pctxt},
            {\kTrans[\kind,\ty]}]$
\item If $\ddck[\ty,{\pctxt;\ctxt},\gk,{}]$ then
  $\stsem[{\transpp[\ty,,]},{\fotyc \pctxt, \ctxt,\pptyc \pctxt},
            {\kTransPP[\kind,\ty]}]$.
\end{itemize}  
\end{lemma}

\begin{proof}
  By induction on the height of the kinding derivation.
\end{proof}

\begin{theorem}[Type Correctness of Closed Types]
  \begin{itemize}
  \item If $\ddck[\ty,,\gk,\con]$ then
    $\stsem[{\trans[\ty,,]},,\kTrans[\kind,\ty]]$.
  \item If $\ddck[\ty,,\gk,\con]$ then
    $\stsem[{\transpp[\ty,,]},,\kTransPP[\kind,\ty]]$.
  \end{itemize}  
\end{theorem}

A practical implication of this theorem is that
it is sufficient to check data descriptions (i.e. \ddc{} types) for
well-formedness to ensure that the generated types and
functions are well formed. This property is sorely lacking in many
common implementations of Lex and YACC, for which users must examine
generated code to debug compile-time errors in
specifications.

\paragraph*{Canonical Forms for Parsed Data.}
\ddc{} parsers generate pairs of representations and parse descriptors
designed to satisfy a number of invariants.  Of greatest importance is 
the fact that
when the parse descriptor says there are no errors in a particular
substructure, the programmer can count on the representation
satisfying all of the syntactic and semantic
constraints expressed by the dependent
\ddc{} type description.  When parse descriptor and representation
satisfy these invariants, we say the pair
of datastructures is in {\em canonical form}.
While generated parsers produce canonical outputs, generated printers
expect canonical inputs.  

% In addition to proving our semantic functions coherent, we identify
% the canonical forms of representation parse-descriptor pairs
% produced by parsers and consumed by printers. Our canonical forms
% lemma goes beyond standard canonical forms lemmas by specifying
% essential correlation between the representation and parse
% descriptor. For example, in the canonical pair for sum types, the rep
% and PD are both injected in the same branch of the sum. Moreover, we
% specify correlations between the number of errors reported by the
% parse descriptor and the number of errors contained in the representation.

% \edcom{Should we specify the delta from last POPL paper?}  

% \edcom{Perhaps it would be good to put norm. rules back, as they
%   abstract over parsing and printing? Currently, defined in terms of
%   parser image.}

For each \ddc{} type, its canonical forms are defined via
two (mutually recursive) relations.  The first
relation, $\corr \tyval r p$,
defines the canonical form of a representation $r$ and a parse
descriptor $p$ at normal type $\tyval$.  {\em Normal types}
are those closed types with base kind $\kty$ that are defined in 
Figure~\ref{fig:ddc-reduction-rules}.  Types with higher kind such as
abstractions are not described by this relation as they cannot directly
produce representations and PDs.  
The second definition, $\corrkl \ty r p$,
normalizes $\ty$, thereby eliminating outermost type and value
applications.  The result is a normal type $\tyval$ and the
requirements on $\tyval$ are subsequently given by $\corr {\tyval} r
p$.  For brevity in these definitions, we write $p.h.{nerr}$ as
$p.{nerr}$ and use $\mathtt{pos}$ to denote the function that returns
zero when passed zero and one when passed another natural number.  The
following definition gives the notion of \fomega\ expression
equivalence we use.

% However, as \ddc{} lacks evaluation rules, we derive the weak-head
% normal form of a type by deriving the weak-head normal form of the
% corresponding parser. As a technical detail, we note that this normal
% form will not necessarily be a parser for a weak-head normal \ddc{}
% type, but might only be equivalent to one. Therefore, we begin our
% definitions with expression equivalence.

% \begin{definition}[\fomega\ Expression Equivalence]
%   $\iexp \iexpreq \iexp'$ iff $\iexp$ is syntactically equal to $\iexp'$ 
% modulo alpha-conversion of bound variables and equivalence of
% typing annotations.
% \label{def:op-eq}
% \end{definition}

\begin{figure}
\small
\begin{bnf}
%   \name{Kinds} \meta{\gk} \::= \kty \| \ity \-> \gk 
%                                \pext{\| \gk \-> \gk} \\
  \name{Normal Types} \meta{\tyval} \::= 
%    \ptrue\| \pfalse \| 
    \pbase{e} \| \plam{\var}{\ity}{\ty} \|
    \psig x \ty \ty  \|
    \psum \ty e \ty  \nlalt
 % \pand \ty \ty \|
    \pset x \ty e \|
%    \pseq \ty \ty {\pterm e \ty} \nlalt
    \pmu{\ptyvar}{}{\ty}  \| \plam{\ptyvar}{}{\ty} 
%   \nlalt
%    \pcompute e \ity \| \pabsorb \ty \| \pscan{\ty} 
    \\
  \name{Types} \meta{\ty} \::= \tyval \| \papp{\ty}{e} \|
                               \papp{\ty}{\ty} \| \ptyvar \\
\\
  \name{Normalization:}
\end{bnf}  
\[
  \infer{
    \papp {\ty} {e} \stepsto \papp {\ty'} {e}
  }{
    \ty \stepsto \ty'
  }
\quad
  \infer{
    \papp {\tyval} {e} \stepsto \papp {\tyval} {e'}
  }{
    e \stepsto e'
  }
\quad
  \infer{
    \papp {(\plam x {} \ty)} {v} \stepsto \ty[v/x]
  }{}
\]
\[
  \infer{
    \papp {\ty_1} {\ty_2} \stepsto \papp {\ty_1'} {\ty_2}
  }{
    \ty_1 \stepsto \ty_1'
  }
\quad
  \infer{
    \papp {\tyval} {\ty} \stepsto \papp {\tyval} {\ty'}
  }{
    \ty \stepsto \ty'
  }
\quad
  \infer{
    \papp {(\plam \ptyvar {} \ty)} {\tyval} \stepsto \ty[\tyval/\ptyvar]
  }{}
\]
  \caption{\ddc{} Normal Types, selected constructs}
  \label{fig:ddc-reduction-rules}
  \label{fig:revised-ddc-syntax}
\end{figure}

\begin{definition}[Canonical Forms (selected constructs)]
(1) $\corr \tyval r p$ iff exactly one of the following is true:
  \begin{itemize}
  \item $\tyval = \pbase{e}$ and $r = \iinld \ity \const$ and $p.{nerr} = 0$.
  \item $\tyval = \pbase{e}$ and $r = \iinrd \ity \ierr$ and $p.{nerr} = 1$.
  \item $\tyval = \psig x {\ty_1} {\ty_2}$ and $r =\ipair {r_1} {r_2}$ and $p =
    \ipair h {\ipair {p_1} {p_2}}$ 
    and $h.{nerr} = \mathtt{pos}(p_1.{nerr}) + \mathtt{pos}(p_2.{nerr})$, $\corrkl
    {\ty_1} {r_1} {p_1}$ and $\corrkl {\ty_2[(r,p)/x]} {r_2} {p_2}$.
  \item $\tyval = \psum {\ty_1} e {\ty_2}$ and $r =\iinld {\ity}{r'}$
    and $p = \ipair h {\iinld {\ity}{p'}}$
    and $h.{nerr} = \mathtt{pos}(p'.{nerr})$ and $\corrkl
    {\ty_1} {r'} {p'}$.
  \item $\tyval = \psum {\ty_1} e {\ty_2}$ and $r =\iinr {r'}$
    and $p = \ipair h {\iinr {p'}}$
    and $h.{nerr} = \mathtt{pos}(p'.{nerr})$ and $\corrkl
    {\ty_2} {r'} {p'}$.
  \item $\tyval = \pset x {\ty'} e$, $r = \iinld \ity {r'}$ and $p =
    \ipair h {p'}$, 
    and $h.{nerr} = \mathtt{pos}(p'.{nerr})$, $\corrkl {\ty'}{r'}{p'}$
    and $e[(r',p')/x] \kstepsto\itrue$.
  \item $\tyval = \pset x {\ty'} e$, $r = \iinrd \ity {r'}$
    and $p = \ipair h {p'}$,
    and $h.{nerr} = 1 + \mathtt{pos}(p'.{nerr})$,
    $\corrkl {\ty'}{r'}{p'}$ and $e[(r',p')/x] \kstepsto \ifalse$.
  \item $\tyval = \pmu \ptyvar {} {\ty'}$, 
    $r = \iroll{r'}{\itsem[\pmu \ptyvar {} {\ty'}]}$, $p =
    \ipair h {\iroll{p'}{
                \itpdsem[\pmu \ptyvar {} {\ty'}]}}$, 
        $p.{nerr} = p'.{nerr}$ 
    and
    $\corrkl {\ty'[\pmu \ptyvar {} {\ty'}/\ptyvar]} {r'} {p'}$. \\
  \end{itemize}

\noindent
(2) $\corrkl \ty r p$ iff $\ty \kstepsto \tyval$ and $\corr \tyval r p$.
\end{definition}

Lemma~\ref{lem:err-corr-at-T}, part 1, states that the parsers for
well-formed types (of base kind) will produce a canonical pair of
representation and parse descriptor, if they produce anything at all.
Conversely, part 2 states that, given a canonical representation and
parse descriptor, the printer for well-formed types (of base kind)
will not ``go wrong'' by calling the $\failpp{}$ function.

\begin{theorem}[Parsing to/Printing from Canonical Forms]
\label{lem:err-corr-at-T}
\begin{itemize}
\item If $\ddck[\ty,,\kty,\con]$ and $\trans[\ty,,] \sapp \spair<B,\off> \kstepsto
  \spair<\off',r,p>$ then $\corrkl \ty r p$.
\item If $\ddck[\ty,,\kty,\con]$ , $\corrkl \ty r p$ and
  $\transpp[\ty,,] \sapp \spair<r,p> \kstepsto \iexp$ then $\iexp \neq
  \failpp{}$.
\end{itemize}
\end{theorem}

\begin{proof}
  Both items are proven by induction on the length of the respective 
  \fomega\ evaluation relations.  Within the induction
  they proceed using a case-by-case analysis of the possible structures 
  of type $\ty$.
\end{proof}


% \edcom{Add text here.}
% \begin{lemma}[Function Interpretation Corellation]
%   If $\ddck[\ty,,\kind,]$ and $\trans[\ty,,] \kstepsto \ival$ then
%   \begin{enumerate}
%   \item $\ty \kstepsto \tyval$,
%   \item $\ddck[\tyval,,\kind,]$
%   \item $\ival \iexpreq \trans[\tyval,,]$,
%   \item $\itsem[\ty] \equiv \itsem[\tyval]$, and
%   \item $\itpdsem[\ty] \equiv \itpdsem[\tyval]$.
%   \end{enumerate}
% \label{lemma:eval-corr}
% \end{lemma}
% \begin{proof}
%   By induction on evaluation derivations.
% \end{proof}



%%% Local Variables: 
%%% mode: latex
%%% TeX-master: "paper"
%%% End: 


\section{Related Work}
\label{sec:related}

\edcom{Small summary followed by reference to thesis.}




\section*{Acknowledgments}


\bibliographystyle{abbrv}
\bibliography{../common/pads-long,../common/pads}

%\newpage
\appendix

\section{Example Data and Descriptions}
\label{app:examples}


%%% Local Variables: 
%%% mode: latex
%%% TeX-master: "semantics"
%%% End: 


\end{document}

%%% Local Variables:
%%% mode: outline-minor
%%% End:

