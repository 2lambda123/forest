\section{The Semantics of Polymorphic Types}
\label{sec:ddc}

\begin{itemize}
\item Review of DDC.
\item Discussions of extensions.
\item Relationship between host language and F-omega(i.e. \fomega
  with following changes).
\item Theory overview and Description of how judgements change from POPL.
\item Describe simplification to theory/meta-theory of recursive types
  and relationship to poly types.
\item Example from pads/ml
  shown in DDC.
\item Meta-theory.
\end{itemize}

In previous work, Fisher et. al specified a low-level data description
language intended to capture the core features of many existing and
future data description langauges~\cite{fisher+:next700ddl}. However,
one of the essential novelties of the \padsml{} language -- support
for \textit{polymorphic types}, or user-defined type constructors
parameterized by types, cannot be expressed in \ddcold{}. Therefore,
we extend \ddcold{} support for polymorphic types to enable
reasoning about the semantics of \padsml{} and any future description
language with similar features.

\subsection{\ddc{} Syntax}
\begin{figure}
{\small
\begin{bnf}
  \name{Kinds} \meta{\gk} \::= \kty \| \ity \-> \gk 
                               \| \kty \-> \gk \\
  \name{Types} \meta{\ty} \::= 
    \ptrue\| \pfalse \| \pbase{e} \| 
    \plam{\var}{\ity}{\ty} \| \papp{\ty}{e} \nlalt
    \psig x \ty \ty \| \psum \ty e \ty \| \pand \ty \ty \|
    \pset x \ty e \| \pseq \ty \ty {\pterm e \ty} \nlalt
    \ptyvar       \| \pmu{\ptyvar}{\gk}{\ty} \| \ptylam{\ptyvar}{\kty}{\ty} \| \ptyapp{\ty}{\ty}
    \nlalt \pcompute e \ity \| \pabsorb \ty \| \pscan{\ty}  
\end{bnf}
}
\caption{\ddc{} Syntax}
\label{fig:ddc-syntax}
\end{figure}


\ddcold{} is a calculus of dependent types. 

\textit{describe calculus informally here}.

It is therefore natural to
explain polymorphic types as type constructors (with type parameters) in this calculus. We
extend \ddcold{} with type-parameterized types
$\ptylam{\ptyvar}{\kty}{\ty}$ and type-type application
$\ptyapp{\ty}{\ty}$ to capture the semantics of polymorphic types. We
refer to the extended calculus as \ddc{}. Its syntax is shown in
\figref{fig:ddc-syntax}.

\subsection{\Implang{} Language}
\label{sec:host-lang}
\begin{figure}[tp]
\small
\begin{bnf}
%   \name{Variables} \meta{f,x,y} \\
%   \name{Bit}   \meta{b}   \::= 0 \| 1 \\ 
  \name{Bits}  \meta{B}   \::= \cdot \| 0\,B \| 1\,B \\ 
  \name{Constants} \meta{c} \::=
      () \| \itrue \| \ifalse \| 0 \| 1 \| -1 \| \dots \nlalt
      \ierr \| \data \| \off \| \iok \| \iecerr \| \iecpc \| \ldots \\

  \name{Values} \meta{v} \::= 
      \const \| % \ilam{\nrm \var}{\ity}{e} \| 
      \ifun {\nrm f} {\nrm x} e \| \ipair v v \nlalt
      \iinld{\ity}{v} \| \iinrd{\ity}{v} \|
      \iarr{\vec{v}} \\

  \name{Operators} \meta{op} \::= 
      = \; \| \; < \; \| \inotop % \| \isizeofop
      \| \ldots \\

  \name{Expressions} \meta{e} \::= 
      \const \| \var \| \iop{e} \|
%      \ilam {\nrm \var} \ity e \| 
      \ifun {\nrm f} {\nrm x} e \| 
      \iapp e e \nlalt
    \Lambda \alpha.e \| e \; [\ty] \nlalt
%      \iletfun {\nrm f} {\nrm x} e \; \iin \; e' \| 
      \ilet {\nrm x} e \; e \|
      \iif e \; \ithen e \; \ielse e \nlalt
      \ipair{e}{e} \| \ipi {\nrm i}{e} \|
      \iinld{\ity}{e} \| \iinrd{\ity}{e} \nlalt
      \icaseg{e}{\nrm x}{e}{\nrm x}{e} \nlalt
      \iarr{\vec e} \| \iappend e e \| \isub e {\nrm e} \nlalt
      \iroll{e}{\mu\alpha.\tau} \| \iunroll{e}
      \\
      
  \name{Base Types} \meta{a} \::= 
      \iunitty \| \iboolty \| \iintty  \| 
      \invty \nlalt  \ibitsty \| \ioffty \| \iecty
  \\
  \name{Types} \meta{\ity} \::= 
      \ibasety \| \ityvar \| \iarrow \ity \ity \| \iprod \ity \ity \|
      \isum \ity \ity \nlalt
      \iseq \ity \| \forall \ityvar.\ity  \|
      \imu \ityvar \ity   
      \| \lambda \alpha.\ity 
      \| \ity \; \ity
  \\
  \name{Kinds} \meta{\kappa} \::= \kty \| \kappa \rightarrow \kappa
  
\end{bnf}
\caption{\Implang{} Language: F$^\omega$}
\label{fig:implang-syntax}
\end{figure}

In \figref{fig:implang-syntax}, we present the host language of \ddc{}, an
extension of \fomega. 
We use this host language both to encode the parsing semantics of \ddc{} 
and to write the expressions that can appear within \ddc{} itself. We
will discuss in~\secref{sec:ddc-sem} our choice of \fomega in place
of the polymorphic lamda calculus used in \ddcold{}.

As the host language is largely standard, we highlight only its
unusual features. The constants include bitstrings $\data$; offsets
$\off$, representing locations in bitstrings; and error codes $\iok$,
$\iecerr$, and $\iecpc$, indicating success, success with errors and
failure, respectively. We use the constant $\ierr$ to indicate a
failed parse.  Because of its specific meaning, we forbid its use in
user-supplied expressions appearing in \ddc{} types.
Our expressions include first-class polymorphic functions $\Lambda \alpha.e$ and their
instantiation $e \; [\ty]$. Unlike \fomega, we do not include kind
annotations on type parameters, as we only use parameterization over
types at kind $\kty$. Expressions also include arbitrary length
sequences $\iarr{\vec e}$, sequence append $\iappend e
{e'}$, and sequence indexing $\isub e {\nrm i}$.

The type $\invty$ is the singleton type of the constant $\ierr$.
Types $\iecty$ and $\ioffty$ classify error codes and bit string
offsets, respectively. The remaining types have standard
meanings: function types, product types, sum types, sequence types
$\iseqty \ty$; universal types $\forall \ityvar.\ity$ and type
variables $\ityvar$; recursive types $\imu \ityvar \ity$; and type
constructors $\lambda \alpha.\ity$ and their application $\ity \;
\ity$. Once again, note that universal types and type constructors do
not include kind annotations, for the reason cited above.

We extend the formal syntax with some syntactic sugar 
for use in the rest of the paper: anonymous functions
$\ilam {\nrm x} \ity e$ for $\ifun {\nrm f} {\nrm x} e$, with $f
\not\in {\rm FV}(e)$; function bindings $\iletfun {\nrm f} {\nrm x} e
\; \iin \; e'$ for $\ilet {\nrm f} {\ifun {\nrm f} {\nrm x} e} \; e'$;
$\ispty$ for $\iprod \ioffty \ioffty$.  We often use
pattern-matching syntax for pairs in place of explicit projections, as
in $\lampair{\codefont e}$ and $\ilet {\itup{\off,r,p}} e\; e'$.  Although
we have no formal records with named fields, we use a dot notation for
commonly occuring projections. For example, for a pair $\mathtt x$ of
rep and PD, we use $\codefont{x.rep}$ and $\codefont{x.pd}$ for the
left and right projections of $\codefont{x}$, respectively. Also, sums and products are
right-associative. 

The static semantics ($\stsem[e,{\ctxt},\ity]$) and operational
semantics ($e \stepsto e'$) are those of \fomega extended with
recursion functions and recursive types and are entirely standard.
See, for example, Pierce~\cite{pierce:tapl} for details.

\subsection{\ddc{} Semantics}
\label{sec:ddc-sem}

The primitives of \ddc{} are deceptively simple.  Each captures a
simple concept, often familiar from type theory. However, in reality,
each primitive is multi-faceted. Each simultaneously describes a
collection of valid bit strings, two datatypes in the host language --
one for the data representation itself and one for its parse
descriptor -- and a transformation from bit strings, including
invalid ones, into data and corresponding meta-data.
We give semantics to \ddc{} types using three semantic functions, each
of which precisely conveys a particular facet of a type's meaning.
The functions $\itsem[\cdot]$ and $\itpdsem[\cdot]$ describe the {\it
  representation semantics} of \ddc{}, detailing the types of the
data's in-memory representation and parse descriptor. The function
$\trans[\cdot,,]$ describes the {\it parsing semantics} of \ddc{},
defining a \implang{} language function for each type that parses bit
strings to produce a representation and parse descriptor. We define
the set of valid bit strings for each type to be those strings for
which the PD indicates no errors when parsed.

We begin with a kinding judgment that checks if
a type is well formed. We then formalize
the three-fold semantics of \ddc{} types.

\begin{table}
  \begin{center}
    \renewcommand{\arraystretch}{1.35}
    \begin{tabular}{l l}
      $\ddck[\ty,{\pctxt;\ctxt},\kind,\mcon]$ & {\it \ddc{}-type
        kinding}\\
      $\itsem[\ty] = \ity$ & {\it representation types of \ddc{} types}\\
      $\itpdsem[\ty] = \ity$ & {\it pd types of \ddc{} types}\\
      $\trans[\ty,\ctxt,\gk] = e$   & {\it \ddc{}-type semantics} \\
      $\kTrans[\gk,\ty] = \ity$     & {\it parser type} \\
      $\ptyc \pctxt = \ctxt$     & {\it parser-type context }\\
      $\fotyc \pctxt = \ctxt$     & {\it \fomega version of poly. context }\\
      $\fortyc \pctxt = \ctxt$     & {\it Rep. type variables in $\fotyc \pctxt$ }\\
      $\fopdtyc \pctxt = \ctxt$     & {\it PD type variables in $\fotyc \pctxt$ }\\
      $\stsem[e,\ctxt,\ity]$ & {\it \fomega expression typing} \\
    \end{tabular}
    \caption{Translations and Judgments}
    \label{tab:judg-list}
  \end{center}
\end{table}
For reference, we provide in
\tblref{tab:judg-list} a listing of all the functions and judgments
defined in this section and a brief description of each.  


\subsubsection{\ddc{} Kinding}
\label{sec:ddc-kinding}

\begin{figure*}[t]
\small
\fbox{$\ddck[\ty,\pctxt;\ctxt,\kind,\mcon]$}\\[-2ex]
\[
\infer[\text{Unit}]{
    \ddck[\ptrue,\pctxt;\ctxt,\kty,\con]
  }{\wfd {} {\fotyc \pctxt,\ctxt}}
\quad 
\infer[\text{Bottom}]{
    \ddck[\pfalse,\pctxt;\ctxt,\kty,\con]
  }{\wfd {} {\fotyc \pctxt, \ctxt}}
\quad 
\infer[\text{Const}]{
    \ddck[\pbase{e},\pctxt;\ctxt,\kty,\con]
  }{
    \begin{semcond}
      \wfd {} {\fotyc \pctxt,\ctxt} &
      \stsem[e,{\fotyc \pctxt,\ctxt},\ity] \\
      \vlet {\ity \iarrowi \kty} {\Ikind(C)}
    \end{semcond}
  }
\]

\[
\infer[\text{Abs}]{
    \ddck[\plam{\var}{\ity}{\ty},
         \pctxt;\ctxt,\ity \iarrowi \kind,\mcon]
  }{
    \ddck[\ty,\pctxt;\ectxt{\var{:}\ity},\kind,\mcon]
  }
\quad
\infer[\text{App}]{
  \ddck[\papp{\ty}{e},\pctxt;\ctxt,\gk,\mcon]
}{
  \ddck[\ty,\pctxt;\ctxt,\ity \iarrowi \gk,\mcon] &
  \stsem[e,{\fotyc \pctxt,\ctxt},\ity]
}
\]

\[
\infer[\text{Prod}]{
    \ddck[\psig{x}{\ty}{\ty'},\pctxt;\ctxt,\kty,\con]
  }{       
    \ddck[\ty,\pctxt;\ctxt,\kty,\mcon] &
    \ddck[\ty',\pctxt;
          \ectxt {x{:}\iprod {\itsem[\ty]} 
              {\itpdsem[\ty]}},
          \kty,\mcon']
  }
\]

\[
\infer[\text{Sum}]{
    \ddck[\psum{\ty}{e}{\ty'},\pctxt;\ctxt,\kty,\con]
  }{
    \ddck[\ty,\pctxt;\ctxt,\kty,\mcon] & \ddck[\ty',\pctxt;\ctxt,\kty,\mcon'] 
  }
\quad
  \infer[\text{Intersection}]{
    \ddck[\pand \ty {\ty'},\pctxt;\ctxt,\kty,\con]
  }{
    \ddck[\ty,\pctxt;\ctxt,\kty,\mcon] & \ddck[\ty',\pctxt;\ctxt,\kty,\mcon'] 
  }
\]

\[
  \infer[\text{Con}]{
    \ddck[\pset x \ty e,\pctxt;\ctxt,\kty,\con]
  }{ 
    \ddck[\ty,\pctxt;\ctxt,\kty,\mcon] & 
    \stsem[e,
     {\fotyc \pctxt,
    \ectxt{x{:}\iprod{\itsem[\ty]} 
      {\itpdsem[\ty]}}},\iboolty]
  }
\]

\[\infer[\text{Seq}]{
    \ddck[\pseq \ty {\ty_s} {\pterm e {\ty_t}},\pctxt;\ctxt,\kty,\con]
  }{
    \begin{array}{c}
    \ddck[\ty,\pctxt;\ctxt,\kty,\mcon] \qquad
    \ddck[{\ty_s},\pctxt;\ctxt,\kty,\mcon_s] \qquad
    \ddck[{\ty_t},\pctxt;\ctxt,\kty,\mcon_t] \\
    \stsem[e,{\fotyc \pctxt,\ctxt},
    \iprod {\itsem[{\ty_m}]}      
    {\itpdsem[{\ty_m}]}
    \iarrowi \iboolty]
    \quad (\ty_m = \pseq \ty {\ty_s} {\pterm e {\ty_t}})
    \end{array}
  }
\]

\[
  \infer[\text{TyVar}]{
    \ddck[\ptyvar,{\pctxt;\ctxt},\kty,\ncon]
  }{\wfd {}{\fotyc \pctxt, \ctxt} \quad \tyvar \in \dom \pctxt}
\quad
  \infer[\text{Rec}]{
    \ddck[\pmu \ptyvar \kty \ty,\pctxt;\ctxt,\kty,\con]
  }{
    \ddck[\ty,{\pctxt,\ptyvar{:}\kty;\ctxt},\kty,\con]
  }
\quad
\infer[\text{TyAbs}]{
    \ddck[\ptylam{\tyvar}{\kty}{\ty},
         \pctxt;\ctxt,\kty \iarrowi \kind,\mcon]
  }{
    \ddck[\ty,{\pctxt,\tyvar{:}\kty;\ctxt},\kind,\mcon]
  }
\quad
\infer[\text{TyApp}]{
  \ddck[\ptyapp{\ty_1}{\ty_2},\pctxt;\ctxt,\gk,\mcon]
}{
  \ddck[\ty_1,\pctxt;\ctxt,\kty \iarrowi \gk,\mcon] &
  \ddck[\ty_2,\pctxt;\ctxt,\kty,\mcon]
}
\]



\[
  \infer[\text{Compute}]{       
    \ddck[\pcompute{e}{\ity},\pctxt;\ctxt,\kty,\con]
  }{
    \wfd {}{\fotyc \pctxt, \ctxt} &
    \stsem[e,{\fotyc \pctxt,\ctxt},\ity] & 
    \fomegak{\fortyc \pctxt}{\ity}{\kty}
  }      
\quad
\infer[\text{Absorb}]{
    \ddck[\pabsorb{\ty},\pctxt;\ctxt,\kty,\con]
  }{
    \ddck[\ty,\pctxt;\ctxt,\kty,\mcon]
  }
\quad
  \infer[\text{Scan}]{
    \ddck[\pscan{\ty},\pctxt;\ctxt,\kty,\con]
  }{
    \ddck[\ty,\pctxt;\ctxt,\kty,\mcon]
  }
\]
\caption{\ddc{} Kinding Rules}
\label{fig:ddc-kinding}
\end{figure*}

The kinding judgment defined in \figref{fig:ddc-kinding} determines
well-formed \ddc{} types, assigning kind $\kty$ to basic types and
kind $\ity \iarrowi \kind$ to type abstractions.  We use two contexts to express our kinding judgment:
\[
\begin{array}{ll}
\ctxt  & \mathrel{::=} \cdot \bnfalt \ctxt,{\var{:}\ity}\\
\pctxt  & \mathrel{::=} \cdot \bnfalt \pctxt,\tyvar{:}\kty
\end{array}
\]

Context $\Gamma$ is a finite partial map that binds expression
variables to their types.
Context $\pctxt$ is a finite partial map that binds type
variables to their kinds. We provide the following translations from
$\pctxt$ to \fomega.

\[
\begin{array}{ll}
\fortyc \cdot &= \cdot \\
\fortyc {(\pctxt,\ptyvar{:}\kty)} &= \fortyc \pctxt, \ptyvar_\repname
{\mathrel{::}} \kty \\
\fopdtyc \cdot &= \cdot \\
\fopdtyc {(\pctxt,\ptyvar{:}\kty)} &= \fopdtyc \pctxt, \ptyvar_\pdbname
{\mathrel{::}} \kty \\
\fotyc \pctxt &= \fortyc \pctxt, \fopdtyc \pctxt
\end{array}
\]

As the rules are otherwise mostly straightforward, we highlight just
two of them. We use the function $\Ikind$ to assign kinds to base
types.  While their kind does not differentiate them from type
abstractions, base types are not well formed when not applied.  Once
applied, all base types have kind $\kty$. The product rule shows that
the name of the first component is bound to a pair of a representation
and corresponding PD.  The semantic functions defined in the next
section determine the type of this pair.

\subsubsection{Representation Semantics}
\label{sec:intty-sem}

\begin{figure}
\fbox{$\itsem[\ty] = \ity$}
\[
\begin{array}{lcl} 
\itsem[\ptrue] & = & \iunitty \\
\itsem[\pfalse] & = & \invty \\
\itsem[\pbase{e}] & = & \isum {\Irty(C)} \invty   \\
\itsem[\plam{\var}{\ity}{\ty}] & = & \itsem[\ty] \\
\itsem[\papp \ty e] & = & \itsem[\ty] \\
\itsem[\psig \var {\ty_1} {\ty_2}]  & = & \iprod {\itsem[\ty_1]} {\itsem[\ty_2]}    \\
\itsem[\psum {\ty_1} e {\ty_2}]     & = & \isum {\itsem[\ty_1]} {\itsem[\ty_2]} \\
\itsem[\pand {\ty_1} {\ty_2}]  & = & \iprod {\itsem[\ty_1]}{\itsem[\ty_2]}\\
\itsem[\pset x \ty e] & = & \isum {\itsem[\ty]}{\itsem[\ty]}\\
% field names: length, elts
\itsem[\pseq \ty {\ty_{\text{sep}}} {\pterm e {\ty_{\text{term}}}}] & = & 
    \iprod \iintty {(\iseq{\itsem[\ty]})}             \\
%% \itsem[\pcase e c {\ty_1} {\ty_2}]       & = & \isum {\itsem[\ty_1]} {\itsem[\ty_2]}\\
\itsem[\ptyvar] & = & \ptyvar_\repname \\
\itsem[\pmu{\ptyvar}{\gk}{\ty}] & = & \imu{\ptyvar_\repname}{\itsem[\ty]} \\
\itsem[\lambda \ptyvar.\ty]       & = & \lambda \ptyvar_\repname.\itsem[\ty] \\
\itsem[\ty_1 \ty_2]              & = & \itsem[\ty_1] \itsem[\ty_2] \\
\itsem[\pcompute e \ity]                 & = & \ity \\
\itsem[\pabsorb \ty]                     & = & \isum \iunitty \invty \\
\itsem[\pscan \ty] & = & \isum {\itsem[\ty]} \invty
%% \pext{
%% \itsem[\ptransform e e \ty]              & = & \itsem[\ty]\\
%% }
\end{array}
\]
\caption{Representation Types}
\label{fig:rep-tys}
\end{figure}

In Figure~\ref{fig:rep-tys}, we present the representation type
of each \ddc{} primitive. While the primitives are
dependent types, the mapping to the \implang{} language erases the dependency because the \implang{} language does not have dependent types. For \ddc{} types in which expressions appear,
the translation drops the expressions to remove the dependency.
With these expressions gone, variables become useless, so we drop 
variable bindings as well, as in product and constrained types.
Similarly, as type abstraction and application are only relevant for
dependency, we translate them according to their underlying types.

In more detail,
the \ddc{} type $\ptrue$ consumes no input and produces only
the $\iunitty$ value.  Correspondingly, $\pfalse$ consumes no input,
but uniformly fails, producing the value $\invty$. The
function $\Irty$ maps each base type to a representation for
successfully parsed data. Note that this representation does not depend
on the argument expression. As base type parsers can fail, we sum this type
with $\invty$ to produce the actual representation type.
Intersection types produce a pair of values, one for each sub-type,
because the representations of the subtypes need not be identical nor
even compatible. 
Constrained types produce sums, where a left branch indicates the data
satisfies the constraint and the right indicates it does not. In
the latter case, the parser returns the offending data rather than
$\ierr$ because the error is semantic rather than syntactic.
Sequences produce a \implang{} language sequence paired with its
length.  Recursive types generate recursive representations. Note that the \implang{} type uses the same variable name
as the \ddc{} type, and so the type corresponding to the type variable
$\ptyvar$ is exactly $\ityvar$.
The output of a $\pcomputen$ is exactly the computed value, and
therefore shares its type.  The output of $\pabsorbn$ is a sum
indicating whether parsing the underlying type succeeded or failed.
The type of $\pscann$ is similar, but also returns an element of the
underlying type in case of success.

{\em Modify a\_pd to a\_pd\_body, everywhere. This name is more
  appropriate.  Note that the new scheme for translating type
  variables will affect WF rules. a will be in D but a\_rep and
  a\_pd\_body could appear in a sigma

  Consider a new convention: for compute type, any references to type
  interepretations should be done with the type interpretation
  functions rather than being hard coded. Otherwise, type substitution gets
  messed up,e.g., when unfolding a recursive type. 
  For example, should be compute(e:[a]\_rep) instead of
  compute(e:a\_rep). Then, substitution on DDC types will burrow into
  the type annotation $\gs$ of compute types. To support, need new
  syntax for type annotations $\gs$ and need to explicitly translate
  annotation types into F-omega types.
  
}

\begin{figure}
\fbox{$\itpdsem[\ty] = \ity$}
\[ 
\begin{array}{lcl} 
%% %% example: \ua.(int * a) + None
%% %%          pd = \ua.pd_hdr  * ((pd_hdr * ([int]_pd * [a]_pd)) + [None]_pd)
%% %%             = \ua.pd_hdr  * ((pd_hdr * ([int]_pd * a)) + [None]_pd)
\itpdsem[\ptrue] & = & \ipty \iunitty \\                                                  
\itpdsem[\pfalse] & = & \ipty \iunitty \\                                                  
\itpdsem[\pbase{e}] & = & \ipty \iunitty\\
\itpdsem[\plam \var \ity \ty] & = & \itpdsem[\ty] \\
\itpdsem[\papp \ty e] & = & \itpdsem[\ty] \\
\itpdsem[\psig \var {\ty_1} {\ty_2}] & = & 
               \ipty {\iprod {\itpdsem[\ty_1]} {\itpdsem[\ty_2]}} \\
\itpdsem[\psum {\ty_1} e {\ty_2}] & = & 
               \ipty {(\isum {\itpdsem[\ty_1]} {\itpdsem[\ty_2]})} \\
\itpdsem[\pand {\ty_1} {\ty_2}] & = & \ipty {\iprod {\itpdsem[\ty_1]} {\itpdsem[\ty_2]}}    \\
\itpdsem[\pset x \ty e] & = & \ipty {\itpdsem[\ty]} \\
\itpdsem[\pseq \ty {\ty_{\text{sep}}} {\pterm e {\ty_{\text{term}}}}] & = & 
  \iapty {\itpdsem[\ty]} \\
\itpdsem[\ptyvar] & = & \ipty{\ptyvar_\pdbname} \\
\itpdsem[\pmu \ptyvar \kty \ty] & = & 
  \ipty{\imu{\ptyvar_\pdbname}{\itpdsem[\ty]}} \\
\itpdsem[\lambda \ptyvar.\ty]      
     & = & \lambda \ptyvar_\pdbname.\itpdsem[\ty] \\
\itpdsem[\ty_1 \ty_2]            & = & \itpdsem[\ty_1] \itpdsemstrip[\ty_2] \\
\itpdsem[\pcompute e \ity]            & = & \ipty \iunitty \\
\itpdsem[\pabsorb \ty]                & = & \ipty \iunitty \\
\itpdsem[\pscan{\ty}] & = & \ipty {(\isum {(\iprod \iintty
    {\itpdsem[\ty]})} \iunitty)}
\end{array}
\]

\fbox{$\itpdsemstrip[\ty] = \ity$}

\[
\begin{array}{lcl} 
\itpdsemstrip[\ty] & = & \ity \ \ \mbox{where}\ \itpdsem[\ty] = \ipty{\ity}
\end{array}
\]
\caption{Parse Descriptor Types}
\label{fig:pd-tys}
\end{figure}

In \figref{fig:pd-tys}, we give the parse descriptor
type for each \ddc{} type. Each PD type has a header and body.
This common shape allows us to define functions that polymorphically
process PDs based on their headers. Each header stores the number of
errors encountered during parsing, an error code indicating the degree
of success of the parse -- success, success with errors, or failure --
and the span of data described by the descriptor.  Formally, the type
of the header  ($\tyface{pd\_hdr}$) is $\iintty \iprodi \iecty \iprodi
\ispty$.  Each body consists of subdescriptors corresponding to the
subcomponents of the representation and any type-specific meta-data. For types with neither subcomponents nor special meta-data, we
use $\iunitty$ as the body type.

We discuss a few of the more complicated parse descriptors in detail.
The parse descriptor body for sequences contains the parse descriptors of its elements,
the number of element errors, and the sequence length. Note that the
number of element errors is distinct from the number of sequence
errors, as sequences can have errors that are not related to their
elements (such as errors reading separators).  We introduce an
abbreviation for array PD body types, $\iaptyname \; \ity =
\iintty \iprodi \iintty \iprodi (\iseq \ity)$.
\trversion{The $\pcomputen$ parse descriptors have no subelements because the
data they describe is not parsed from the data source.}
The $\pabsorbn$ PD
type is $\iunitty$ as with its representation. We assume that just as
the user does not want the represenation to be kept, so too the parse
descriptor.  The $\pscann{}$ parse descriptor is either $\iunitty$, in case
no match was found, or records the number of bits skipped before the
type was matched along with the type's corresponding parse descriptor.


\begin{figure}
\small
\fbox{$\kTrans[\gk,\ty] = \ity$} 
    
\begin{align*}
  &\kTrans[\kty,\ty] = \extdom * \offdom \iarrowi \offdom * \itsem[\ty] * \itpdsem[\ty]
   \\
   &\kTrans[\ity \iarrowi \gk,\ty] = \ity \iarrowi \kTrans[\gk,\ty e],
   \; \mbox{for any e}.
   \\
   &\kTrans[\kty \iarrowi \gk,\ty] = 
      \forall\tyvar_\repname.\forall\tyvar_\pdbname.
         \kTrans[\kty,\tyvar] \iarrowi \kTrans[\gk,\ty \tyvar]\; 
         (\ptyvar_\repname.\ptyvar_\pdbname \not \in \ftv \kind \cup
         \ftv \ty)
\end{align*}  
  \caption{\Implang{} Language Types for Parsing Functions}
  \label{fig:parser-types}
\end{figure}

\subsubsection{Parsing Semantics of the \ddc{}}
\label{sec:parse-sem}

\begin{figure*}
\small
\fbox{$\trans[\ty,\ctxt,\gk] = e$} 

\[
\begin{array}{l}
  %% None 
\trans[\ptrue,\ctxt,\kty] =
  \lampair{\spair<\off,\newrep{unit}{},\newpd{unit}{\off}>}
\\[3pt] %\\
%% False 
\trans[\pfalse,,] =
  \lampair{\spair<\off,\newrep {bottom}{},\newpd {bottom}{\off}>}
\\[3pt] %\\ 
%% Const 
\trans[\pbase{e},\ctxt,\kty] =
  \lampair{\iapp {\iapp {\Iimp(C)} (e)} {\itup {\idata,\off}}}
\\[3pt] %\\
%% Abs 
\trans[\plam{\var}{\ity}{\ty},,] =
   \sfn{\nrm\var}{\ity}{\trans[\ty,\ectxt{\var{:}\ity},\kind]}
\\[3pt] %\\
%% App 
\trans[\papp{\ty}{e},\ctxt,\gk] =
  \trans[\ty,,] \sapp e  
\\[3pt]
%% Prod 
%\begin{array}{l}
\trans[\psig{x}{\ty}{\ty'},\ctxt,\kty] = \\
  \begin{array}{l}  
    \lampair{} \\
    \quad  \ilet {\spair<\off',r,p>} 
    {{\trans[\ty,,]} \sapp \spair<\idata,\off>} \\
    \quad  \ilet x {\ictup{r,p}}\\
    \quad  \ilet {\spair<\off'',r',p'>} 
    {{\trans[\ty',,]} \sapp \spair<\idata,\off'>} \\
    \quad \spair<\off'',\newrep {\gS}{r,r'},\newpd {\gS}{p,p'}>
  \end{array}  
%\end{array}
\\
%% Sum 
%\begin{array}{l}
  \trans[\psum{\ty}{e}{\ty'},,] = \\
  \begin{array}{l}  
  \lampair{} \\
  \quad \ilet {\itup{\off',r,p}}{\trans[\ty,,] \sapp \spair<\idata,\off>} \\
  \quad \iif {\pdok p} \; \ithen {
    \def \r {\newrep {+left}{r}}
    \def \p {\newpd {+left}{p}}
    \spair<\off',\r,\p>} \\
  \quad \ielse {\ilet {\itup{\off',r,p}}{\trans[\ty',,] \sapp \spair<\idata,\off>}} \\
  \quad 
  {  % begin scope
    \def \r {\newrep {+right}{r}}
    \def \p {\newpd {+right}{p}}
    %% 
    \spair<\off',\r,\p>
  }\\ % end scope
  \end{array}
\\
%% Intersection 
  \trans[\pand{\ty}{\ty'},,] = \\
  \begin{array}{l}  
     \lampair{} \\
     \quad \ilet {\itup{\off',r,p}} {\trans[\ty,,] \sapp \spair<\idata,\off>} \\
     \quad \ilet {\itup{\off'',r',p'}} {\trans[\ty',,] \sapp \spair<\idata,\off>} \\
     \quad {\spair<\codefont{max}(\off',\off''),\newrep {\&}{r,r'},\newpd {\&}{p,p'}>}
   \end{array}
\\
%\quad
%% Set 
  \trans[\pset{x}{\ty}{e},\ctxt,\kty] = \\
  \begin{array}{l}  
    \lampair{} \\
    \quad \ilet {\itup{\off',r,p}}{\trans[\ty,,] \sapp \spair<\idata,\off>} \\
    \quad \ilet x {\ictup{r,p}}\\
    \quad \ilet c e \\
    \quad \spair<\off',\newrep {con} {c,r},\newpd {con} {c,p}>
  \end{array}
\\
\end{array}
\begin{array}{l}
%% Array 
\trans[\pseq{\ty}{\ty_s}{\pterm e {\ty_t}},,] = \\
  \begin{array}{l}  
    \lampair{}\\
      \quad \iletfun {isDone}{\itup{\off,r,p}}{\\
        \qquad \ior {\eofpred {\idata,\off}} {e\codefont {\sapp
          \spair<r,p>}} \iori \\
        \qquad \ilet {\itup{\off',r',p'}}{\trans[\ty_t,,] \spair<\idata,\off>}\\
        \qquad \pdok{p'}
      }\\
      \quad \iin \\
      \quad \iletfun {continue} {\itup{\off,\off',r,p}} {\\
        \qquad \iif  {\off = \off' \iori \isdone {\off',r,p}} \; \ithen {\itup{\off',\codefont{r,p}}} \\
        \qquad \ielse {
          \ilet {\itup{\off_s,r_s,p_s}}{\trans[\ty_s,,] \sapp \spair<\idata,\off'>}}\\
        \qquad \ilet {\itup{\off_e,r_e,p_e}}{\trans[\ty,,] \sapp \ictup{\idata,\off_s}}\\
        \qquad \mathtt{continue} \sapp \ictup{
            \off,\off_e,\newrep {seq} {r,r_e}, \newpd {seq} {p, p_s, p_e}
        }}\\
      \quad \iin
   \end{array}\\
  \begin{array}{l}  
      \quad \ilet {\mathtt{r}} {\newrep {seq\_init}{}}\\
      \quad \ilet {\mathtt{p}} {\newpd {seq\_init}{\off}}\\
      \quad \iif {\isdone{\off,r,p}} \; \ithen {\itup{\off,\codefont{r,p}}}\\
      \quad \ielse {\ilet {\itup{\off_e,r_e,p_e}}{\trans[\ty,,] \sapp
          \spair<\idata,\off>}} \\
      \quad \mathtt{continue} \sapp \ictup{\off,\off_e,
        \newrep {seq} {r,r_e}, \newpd {seq} {p, \newpd {unit} \off, p_e}}      
  \end{array}  
\\
%\end{array}
%\quad
%\begin{array}{l}
%% Var
\trans[\ptyvar,,] = \codefont{f_\ptyvar}
\\[3pt]
%% Mu
\trans[\pmu \ptyvar \gk \ty,,] = \\
  \begin{array}{l}
  \ifun {f_\ptyvar} {\itup{\data{:}\ibitsty,\off{:}\ioffty}  
           : \ioffty * \pmu \ptyvar \gk \itsem[\ty] 
                    * (\ipty{\pmu \ptyvar \kty \itpdsem[\ty]}) } {}\\
  \quad \ilet {\itup{\off',r,p}} 
   {\trans[\ty,,] \iappi \ictup{\data,\off}} \\ 
  \qquad \ictup{\off',\iroll{r}{\pmu \ptyvar \gk \itsem[\ty]},
     (p.h,\iroll{p}{\pmu \ptyvar \kty \itpdsem[\ty]})}
%}}
  \end{array}  
\\[3pt]
%% lambda \alpha
\trans[\lambda\tyvar . \ty,,] = %\\
%  \begin{array}{l}
    \Lambda \tyvar_\repname. 
    \Lambda \tyvar_\pdbname. \lambda \codefont{f_\ptyvar}. \trans[\ty,,]
%  \end{array}  
\\
%% t1 t2
\trans[\ty_1 \ty_2,,] = 
    \trans[\ty_1,,]\; [\itsem[\ty_2]]\; [\itpdsemstrip[\ty_2]]\; \trans[\ty_2,,]
\\
%% Compute
\trans[\pcompute e \ity,,] = \\
  \quad \lampair{\itup{\off,\newrep {compute} {\nrm e},\newpd {compute} \off}}
\\[3pt]
%% Absorb
\trans[\pabsorb \ty,,] = \\
  \begin{array}{l}  
    \lampair{}\\
    \quad \ilet {\itup {\off',r,p}} {\trans[\ty,,] \sapp \spair<\idata,\off>}\\
    \quad \itup{\off',\newrep {absorb} p,\newpd {absorb} p}   
  \end{array}  
\\
%% Scan
\trans[\pscan \ty,,] = \\
  \begin{array}{l}  
    \lampair{}\\
    \quad \iletfun {try} {i} {\\
      \qquad \ilet {\itup{\off',r,p}} {\trans[\ty,,] \sapp
        \codefont{\spair<\data,\off + i>}} \\
      \qquad \iif {\pdok p}\; \ithen \\
      \qquad {\ictup{\off',\newrep {scan} r,
        \newpd {scan} {i,p}}}\; \ielse {}\\
      \qquad \iif {\codefont{i = scanMax}}\; \ithen \\
      \qquad {\ictup{\off,\newrep {scan\_err} {},
        \newpd {scan\_err} {\off}}}\; \ielse {}\\
      \qquad \codefont {try \sapp (i+1)}
   }\\
   \quad \iin \sapp \codefont{try \sapp 0} \\
  \end{array}  
\\
\end{array}
\]
%\caption{\ddc{} Semantics (cont.)}
\caption{\ddc{} Semantics}
\label{fig:ddc-sem}
\end{figure*}

\begin{figure}
\small
\begin{itemize}
\renewcommand{\labelitemi}{}

%\item %[Unit:]
\item $\ifun {R_{unit}} \iuval \iuval$
\item $\ifun {P_{unit}} \off {\itup{\itup{0,\iok,\ipair \off \off},\iuval}}$

%\item %[Bottom:]
\item $\ifun {R_{bottom}} \iuval \ierr$
\item $\ifun {P_{bottom}} \off ((1,\iecpc,\ipair \off \off),())$

\item %[Pair:]
\item $\ifun {R_{\gS}} {\ipair {r_1} {r_2}} {\itup {\codefont{r_1,r_2}}}$
\item $\ifun{H_{\gS}} {\ictup{h_1,h_2}}{}$ \\
  $\begin{array}{l}
    \ilet {nerr} {\codefont{pos \itup{{h_1}.{nerr}} + pos \itup{{h_2}.{nerr}}}}\\
    \ilet {ec} {\iif {\codefont{h_2.ec} = \iecpc}\; \ithen {\iecpc}\\
    \quad \ielse {\codefont{max\_ec} \iappi \codefont{h_1.ec} \iappi \codefont{h_2.ec}}} \\
    \ilet {sp} {\ictup{h_1.sp.begin, h_2.sp.end}} \\
    \quad \ictup {nerr,ec,sp}
  \end{array}$

\item $\ifun {P_{\gS}} {\ictup{p_1, p_2}} {\ictup {H_{\gS} \itup{p_1.h,p_2.h},\itup{p_1,p_2}}}$

\end{itemize}
\caption{Selected Constructor Functions. 
The type of PD headers is $\iintty
  \iprodi \iecty \iprodi \ispty$. 
  We refer to the projections using
  dot notation as $\codefont{nerr}$, $\codefont{ec}$ and
  $\codefont{sp}$, respectively. A span is a pair of offsets, referred
  to as $\codefont{begin}$ and $\codefont{end}$, respectively.  The full collection of such constructor functions appears in \appref{app:asst-functions}.}
\label{fig:cons-funs}
%\caption{Constructor Functions (cont.)}
\end{figure}


The parsing semantics of a type $\tau$ is a function that transforms some amount of input into a pair of a representation and a parse descriptor, the types of which are determined by $\tau$.
\figref{fig:parser-types} specifies the \implang{} language types of the parsers generated from well-kinded \ddc{} types.  Note that parameterized \ddc{} types require their arguments before they can parse any input.

\figref{fig:ddc-sem} shows the parsing semantics function.  For each
type, the input to the corresponding parser is a bit string and an
offset which indicates the point in the bit string at which parsing
should commence.  The output is a new offset, a representation of the
parsed data, and a parse descriptor. As the bit string input is
never modified, it is not returned as an output.  In addition
to specifying how to handle correct data, each function describes how
to transform corrupted bit strings, marking detected errors in
a parse descriptor. The semantics function is partial, applying only
to well-formed \ddc{} types.

For any type, there are three steps to parsing: parse the
subcomponents of the type (if any), assemble the resultant representation, and
tabulate meta-data based on subcomponent meta-data
(if any). For the sake of clarity, we have factored the latter two
steps into separate representation and PD constructor functions which we define for
each type. For some types, we additionally factor the PD header
construction into a separate function. For example, the representation 
and PD constructors for $\ptrue$ are $\newrepf {unit}$ and $\newpdf
{unit}$, respectively, and the header constructor for products is
${\codefont{H_{\gS}}}$. Selected constructors are shown in
\figref{fig:cons-funs}. We have also factored out some commonly
occuring code into ``built-in'' functions, explained as needed and
defined formally in \appref{app:asst-functions}.

The PD constructors determine the error code and
calculate the error count.  There are three possible error codes:
$\iok$, $\iecerr$, and $\iecpc$, corresponding to the three possible results of a parse: 
it can succeed, parsing the data without errors; it can succeed,
but discover errors in the process; or, it can find an
unrecoverable error and fail.
\trversion{
Note that the the purpose of the $\iecpc$ code is to indicate to any
higher level elements that some form of error recovery is required.
Hence, the whole parse is marked as failed exactly when the parse ends
in failure.}
The error count is determined by subcomponent error counts and any errors associated directly with the type
itself.  
\trversion{
If a subcomponent has errors then the error count is
increased by one; otherwise its not increased at all. We use the
function $\codefont {pos}$, which maps all positive numbers to 1
(leaving zero as is), to assist in calculating the contribution of
subcomponents to the total error count.  Errors at the level of the
element itself - such as constraint violation in constrained types - are
generally counted individually.}

With this background, we can now discuss selected portions of the semantics.
%With this background, we can now understand the semantics. 
The semantics of $\ptrue$ and $\pfalse$ show that they do not consume any input, \ie{}, they do not change the offset. 
A look at their constructors shows that the parse
descriptor for $\ptrue$ always indicates no errors and a corresponding
$\iok$ code, while that of $\pfalse$ always indicates failure with an
error count of one and the $\iecpc$ error code. The semantics of base
types applies the implementation of the base type's parser, provided
by the function $\Iimp$, to the appropriate arguments.  Abstraction
and application are defined directly in terms of \implang language
abstraction and application.  Dependent pairs read the first element
at $\off$ and then the second at $\off'$, the offset returned from
parsing the first element.  Notice that we bind the pair of the
returned representation and parse descriptor to the variable $\codefont{x}$
before parsing the second element, implicitly mapping the 
\ddc{} variable $x$ to the \implang{} language variable $\codefont{x}$ in the process.
Finally, we combine the results
using the constructor functions, returning $\off''$ as the final
offset of the parse.

Sequences have the most complicated semantics because the number of subcomponents depends upon a combination of the data, the termination predicate, and the terminator type. Consequently, the sequence parser uses mutually
recursive functions $\codefont{isDone}$ and $\codefont{continue}$ to implement this open-ended semantics. 
Function $\codefont{isDone}$ determines if the parser
should terminate by checking whether the end of the source has been
reached, the termination condition $e$ has been satisfied, or the
terminator type can be read from the stream without errors at
$\off$.
Function $\codefont{continue}$ takes four
arguments: two offsets, a sequence representation, and a sequence PD.  The two
offsets are the starting and ending offset of the previous round of
parsing. They are compared to determine whether the parser is
progressing in the source, a check that is critical to ensuring that
the parser terminates. Next, the parser checks whether the sequence is
finished, and if so, terminates. Otherwise, it attempts to read a
separator followed by an element and then continues parsing the
sequence with a call to $\codefont{continue}$.

We translate recursive types into
recursive functions with a special
function name corresponding to the name of the 
bound type   variable.
Recursive type variables translate to these special names.

The $\pscann$ type attempts to parse the underlying type from the
stream at an increasing scan-offset, $i$, from the original offset
$\off$, until success is achieved or a predefined maximum scan-offset
(\cd{scanMax}) is reached.  In the semantics we give here, offsets are
incremented one bit at a time --- a practical implementation would choose
some larger increment ({\it e.g.,} 32 bits at a time).

\subsection{Meta-theory}
\label{sec:meta-theory}


{\em I think that condition one should be elsewhere.It relates to the
contents of the base type implementation, rather than any deep
properties of the contents.}

\begin{condition}[Conditions on Base-type Interfaces]
  \begin{enumerate}
  \item $\dom {\Ikind} = \dom {\Iimp}$.
  \item If $\Ikind(C) = {\ity \iarrowi \kty}$ then $\Iopty(C) =
    \ioparrow \ity {\kTrans[\kty,\pbase e]}$ (for any $e$).
  \item $\fomegak{}{\Irty(C)}{\kty}$.
    \label{cond:closed-op}
  \end{enumerate}
\end{condition}

\begin{definition}
$\pda \ptyvar = \ipty \ptyvar$
\end{definition}

\begin{lemma}[Types of Constructors]
\label{lem:types-of-constructors}
\begin{itemize}
\item $\newrepf {unit} : \iarrow \iunitty \iunitty$
\item $\newpdf  {unit} : \iarrow \ioffty {\ipty \iunitty}$
\item $\newrepf {bottom} : \iarrow \iunitty \invty$
\item $\newpdf  {bottom} : \iarrow \ioffty {\ipty \iunitty}$
\item $\newrepf {\gS} : \forall \ga,\gb.\iarrow {\iprod \ga \gb} {\iprod \ga \gb}$
\item $\newpdf {\gS} : \forall \ga,\gb. 
  \iarrow {\iprod {\pda \ga} {\pda \gb}}
  {\pda {(\pda \ga \iprodi \pda \gb)}}
$
\item $\newrepf {+left} : \forall \ga.\forall \gb.\iarrow \ga 
                            {\isum \ga \gb}$
\item $\newrepf {+right} : \forall \ga.\forall \gb.\iarrow \gb {\isum \ga \gb}$
\item $\newpdf {+left} : \forall \ga, \gb.\iarrow {\pda \ga} 
  {\ipty {(\isum {\pda \ga}{\pda \gb})}}$
\item $\newpdf {+right} :\forall  \ga, \gb. \iarrow {\pda \gb} 
                            {\ipty {(\isum {\pda \ga} {\pda \gb})}}$
\item $\newrepf {\&} : \forall \ga,\gb.\iarrow {\iprod \ga \gb} {\iprod \ga \gb}$
\item $\newpdf {\&} : 
\forall \ga,\gb.
  \pda \ga \iprodi
  \pda \gb \iarrowi 
         {\ipty {(\pda \ga \iprodi \pda \gb)}}
$.
\item $\newrepf {con} : \forall \ga.\iprod \iboolty \ga 
  \iarrowi {\isum \ga \ga}$
\item $\newpdf {con} :\forall  \ga. \iprod \iboolty \iarrow {\pda \ga} {\ipty {\pda \ga}}$
\item $\newrepf {seq\_init} : \forall \ga.\iarrow \iunitty {\iintty \iprodi \iseq \ga}$
\item $\newpdf {seq\_init} : \forall \ga. \iarrow \ioffty {\iapty {\pda\ga}}$
\item $\newrepf {seq} : \forall \ga.\iarrow
  {(\iintty \iprodi \iseq \ga) \iprodi \ga}
  {\iintty \iprodi \iseq \ga}$
\item $\newpdf {seq} :\forall  {\ga_{elt}},{\ga_{sep}}. 
  (\iapty {\pda {\ga_{elt}}}) \iprodi
  \pda {\ga_{sep}} \iprodi 
  \pda {\ga_{elt}} \iarrowi \\
  \iapty {\pda {\ga_{elt}}}$
\item $\newrepf {compute} : \forall \ga.\iarrow \ga \ga$
\item $\newpdf {compute} : \iarrow \ioffty {\ipty \iunitty}$
\item $\newrepf {absorb} : \forall \ga.\iarrow {\pda \ga} {\isum
    \iunitty \invty}$
\item $\newpdf {absorb} :\forall  \ga. \iarrow {\pda \ga} {\ipty
    \iunitty}$
\item $\newrepf {scan} : \forall \ga.\iarrow \ga {\isum \ga \invty}$
\item $\newpdf {scan} :\forall  \ga. \iarrow {\iprod \iintty {\pda \ga}}
  {\ipty {(\isum {\iprod \iintty {\pda \ga}} \iunitty)}}$
\item $\newrepf {scan\_err} : \forall \ga.\iarrow \iunitty {\isum \ga \invty}$
\item $\newpdf {scan\_err} :\forall  \ga. \iarrow \ioffty
  {\ipty {(\isum {\iprod \iintty \ga} \iunitty)}}$
\end{itemize}  
\end{lemma}

\begin{proof}
  By typing rules of \fomega.
\end{proof}

To prove our type correctness theorem by induction, we must account
for the fact that any free type variables in
a \ddc{} type $\ty$ will become free function variables
in $\trans[\ty,,]$.  

To that end, we define the function $\ptyc \pctxt$, 
which maps type-variable contexts
$\pctxt$ to typing contexts $\ctxt$:
\vskip -1.5ex
{\small
\[
\begin{array}{l}
  \ptyc{\cdot} = \cdot \\[1ex]
  \ptyc{\pctxt,\ptyvar{:}\kty} = \ptyc \pctxt,\codefont{f_\ptyvar}{:}\kTrans[\kty,\ptyvar]
\end{array}
\]
}

\begin{theorem}[Type Correctness]
\label{thm:type-correctness}
  If $\ddck[\ty,{\pctxt;\ctxt},\gk,{}]$ then
  $\stsem[{\trans[\ty,,]},{\fotyc \pctxt, \ctxt,\ptyc \pctxt},
            {\kTrans[\kind,\ty]}]$.
\end{theorem}

\begin{proof}
  By induction on the height of the kinding derivation.
\end{proof}

\begin{corollary}[Type Correctness of Closed Types]
  If $\ddck[\ty,,\gk,\con]$ then
  $\stsem[{\trans[\ty,,]},,\kTrans[\kind,\ty]]$.  
\end{corollary}


%%% Local Variables: 
%%% mode: latex
%%% TeX-master: "paper"
%%% End: 
