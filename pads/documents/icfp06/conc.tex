\section{Future Work}
\label{sec:future}

\padsml{} is already an effective, working system for data description
and processing.  However, there are a number of ways we plan to make it 
even better.

First, there are a number of properties of data descriptions
a programmer might want to infer or verify.  For example, it is not hard to
write a non-terminating data description by accident.  It
is also possible to write a description with completely redundant
subparts (dead parser code).  While these problems can be caught 
through testing,
we would prefer to catch them at compile time.  It is also often useful
for programmers to know the ``size'' of any data that matches 
a description.  For instance, programmers describing network packets can use
such properties to check
their work. Consequently,
we plan to explore development a \padsml{} ``type checker'' 
to infer description properties and catch obvious errors.

A second long-term goal is to build a collection of
higher-level, format-independent data analysis tools.  By
``higher-level'' tools, we mean tools that perform semantic data
analysis as opposed to simpler, low-level syntactic transformation
(such as \xml\ conversion) and analysis.  
Tools in this category include tools for
content-based search, clustering, statistical data modeling, data 
generation and machine learning.  We believe that if we can automatically
generate stand-alone,
end-to-end tools that perform these functions over arbitrary data, 
we can have a substantial impact on the
productivity of many researchers.  We plan, in particular,  
to target computational biologists.  The research challenge
in these tasks is not likely to be more language design, but rather
understanding the needs of specific domains 
(such as computational biology) and engineering end-to-end solutions
that may be used completely off-the-shelf
by researchers with limited time or programming skills.  We plan
to provide access to scientists with limited programming skills
through the LaunchPADS data visualization 
environment~\cite{launchpads:planx,launchpads:sigmod}, which currently
only interfaces with \padsc{}.

As mentioned in Section~\ref{sec:intro}, ad hoc data sources are often
very large scale.  Large data volumes often require that the data be
processed without loading it into main memory all at once.  The
\padsc{} language accommodates efficient processing of very
large-scale data~\cite{fisher+:pldi05} by supporting multiple-entry
point parsing, which permits a user to write tools that have fixed
memory requirements and that can yield a result in one scan of the
data source.  We plan to explore similar techniques in \padsml{}. 

\section{Conclusions}
\label{sec:conc}

Despite the advent of standard data representation languages such as
\xml, vast quantities of important information continue to exist
in ad hoc data formats.  In order to deal with this data,
programmers desperately need reliable, high-level tools to 
help them document, parse, 
analyze, transform, query and visualize it.  

In this paper, we describe \padsml{}, 
a high-level, domain-specific language and system
designed to improve the productivity of programmers who 
manipulate ad hoc data.  The \padsml{} 
language is inspired by the type structure of modern functional
programming languages, and uses 
polymorphic, dependent and recursive data types to document both the
basic syntax and the semantic properties of any data source.  
We have simplified and extended the Data Description Calculus (DDC) 
developed in
our earlier work~\cite{fisher+:next700ddl} to account for these features.
We have also demonstrated that this specification language
is compact, expressive, and 
capable of describing data from many domains, ranging from
networking to computational biology to finance to cosmology.

The \padsml{} compiler uses a ``types as modules'' compilation strategy
in which every \padsml{} type definition is compiled into
an \ocaml{} module containing types for data representations
and functions for parsing and data processing.  Functional programmers
can use the generated modules to write clear and concise {\em format-dependent}
data processing programs;
we give several simple examples in this paper.  Our system design
also allows external
tool developers to write new {\em format-independent} tools
simply by supplying a module that matches the appropriate generic
signature.  In both cases, the design improves substantially over
our earlier work on \padsc{}.

The \padsml{} compilation strategy also provides a stimulating and
practical test case for researchers studying functional language
design.  Since recursive types are compiled into recursive modules and
parameterized types are compiled into functors, \padsml{} pushes the
limits of the most expressive modern module systems.  It also suggests
a collection of problems for researchers studying type-directed
programming.  We encountered these limits in our work implementing
\padsml{} in
\ocaml, but rather than changing
our basic compilation strategy, which we feel is very natural and elegant,
we have left certain combinations of features (recursion and polymorphism)
unimplemented.
We challenge functional language designers to extend their favourite
language to meet the demands of our application.


