\newcommand{\pvalue}{???}
\section{Describing Data in \padsmlbig{}}
\label{sec:padsml-overview}

{\em
To do:
\begin{itemize}
\item compare w/ pads/c: key thing is to find compelling
  comparisons. I think Newick is one (short and sweet). I suspect that
  Darkstar/Regulus, which uses polymorphism heavily, should be the other.
\end{itemize}
}

A \padsml{} description specifies the physical layout and semantic
properties of an ad hoc data source.  
These descriptions are composed of types: 
base types describe atomic data, while structured
types describe compound data built from simpler pieces.  Examples of
base types include 8-bit unsigned integers (\cd{Puint8}), 32-bit
integers (\cd{Pint32}), binary 32-bit integers (\cd{Pbint32}), dates
(\cd{Pdate}), strings (\cd{Pstring}), and IP addresses (\cd{Pip}).
Semantic conditions for such base types include checking that the
resulting number fits in the indicated space, \ie, 16-bits for
\cd{Pint16}.

Base types (and other types) may be parameterized by values.  This
mechanism serves both to reduce the number of base types and to permit
the format and properties of later portions of the data to depend upon
earlier portions.  For example, the base type \cd{Puint16_FW(3)}
specifies an unsigned two byte integer physically represented by
exactly three characters. The base type \cd{Pstring} takes 
a character indicating the \textit{terminator character} -- the character in the
source that immediately follows the string. 

To describe more complex data, \padsml{} provides a collection of
type constructors derived from the type structure of functional
programming languages such as Haskell and ML.  The following
subsections will explain these structured types through a series 
of real-world examples.  Readers eager to see the complete syntax
of types should flip forward to Appendix~\ref{app:syntax-dd}.

\subsection{Simple Structured Descriptions}

The bread and butter of any \padsml{} description are the simple
structured types: tuple, record and array types for expressing
sequences of data; sum types, realized here as datatypes, for
expressing alternatives; and singleton types for expressing the
placement of particular characters in the data.  In this section, we
explain each of these items by building a description of the
\dibbler{} data presented in \figref{figure:dibblerml}.

The \dibbler{} data source is used to record summaries of phone orders
produced at AT\&T.  Each summary involves a date and one record per
order.  Each order record contains a header followed by a sequence of
events.  The header has 13 pipe separated fields: the order number,
AT\&T's internal order number, the order version, four different
telephone numbers associated with the order, the zip code of the
order, a billing identifier, the order type, a measure of the
complexity of the order, an unused field, and the source of the order
data.  Many of these fields are optional, in which case nothing
appears between the pipe characters.  The billing identifier may not
be available at the time of processing, in which case the system
generates a unique identifier, and prefixes this value with the string
``no\_ii'' to indicate the number was generated. The event sequence
represents the various states a service order goes through; it is
represented as a new-line terminated, pipe separated list of state,
timestamp pairs.  There are over 400 distinct states that an order may
go through during provisioning.  The sequence is sorted in order of
increasing timestamps.
% \figref{figure:dibbler-records} shows a small example of this
% format.  156 different states for one order -rw-r--r-- 1 angusm
% dibbler 2187472314 Jun 9 2003 /fs/dibblerd/tlf/data/out_sum.stream
% 2171.364u 31.379s 40:41.54 90.2% 0+0k 2+0io 2pf+0w
% 53 had trailing t or } after zip code It may be apparent from this
% paragraph that English is a poor language for describing data
% formats!


\begin{figure*}
{\scriptsize
\begin{verbatim}
0|1005022800
9152|9152|1|9735551212|0||9085551212|07988|no_ii152272|EDTF_6|0|APRL1|DUO|
9153|9153|1|0|0|0|0||152268|LOC_6|0|FRDW1|DUO|
\end{verbatim}
}
  \caption{Miniscule example of \dibbler{} data.}
  \label{figure:dibbler-records}
\end{figure*}

\suppressfloats

\begin{figure}
  \begin{code}\scriptsize
\kw{ptype} Semicolon = Pcharlit(';')
\kw{ptype} Vbar = Pcharlit('|')
\mbox{}
\kw{ptype} Pn\_t = Puint64
\kw{ptype} Pzip = Puint32
\mbox{}
\kw{ptype} Summary\_header = "0|" * Puint32 * Peor
\mbox{}
\kw{pdatatype} Dib\_ramp = 
  Ramp of Pint 
| GenRamp of "no\_ii" * Pint
\mbox{}
\kw{ptype} Order\_header = \{ 
    order\_num : Pint;  
'|'; att\_order\_num : [i:Pint | i < order\_num];  
'|'; ord\_version : Pint;  
'|'; service\_tn : Pn\_t Popt;
'|'; billing\_tn : Pn\_t Popt;  
'|'; nlp\_service\_tn : Pn\_t Popt;  
'|'; nlp\_billing\_tn : Pn\_t Popt;  
'|'; zip\_code : Pzip Popt;  
'|'; ramp : Dib\_ramp;  
'|'; order\_sort : Pstring('|');  
'|'; order\_details : Pint;
'|'; unused : Pstring('|');  
'|'; stream : Pstring('|'); 
'|'
\} 
\mbox{}
\kw{ptype} Event  = Pstring('|') * '|' * Puint32
\kw{ptype} Events = (Event,Vbar,Peor) Plist
\mbox{}
\kw{ptype} Order  = Order\_header * Events
\kw{ptype} Orders = (Order, Peor, Peof) Plist
\mbox{}
\kw{ptype} Source = Summary\_header * Orders\end{code}

  \caption{\padsml{} description for \dibbler{} provisioning data.}
  \label{figure:sirius_pml}
\end{figure}


\figref{figure:sirius_pml} gives a \padsml{} description for the
\dibbler{} phone order summaries in our syntax.  Overall, the
description is a sequence of type definitions. It is probably easiest
to understand the data source by reading these definitions bottom up.

The last type definition \cd{Source} is intended to be a definition of
an entire \dibbler{} data source.  It states that a \cd{Source} is a
\cd{Summary\_header} followed by a sequence of objects made up of an
\cd{Order\_header} followed by \cd{Events}.  The tuple type
constructor \cd{T1 * T2} and the list type constructor
(\cd{(Telt,Tsep,Tterm) Plist}) both specify sequences of objects in a
data source.  The \cd{Plist} type -- defined in \padsml{} itself --
depends upon three parameters, (\cd{Telt},\cd{Tsep} and \cd{Tterm}).
The first parameter describes the element type of the list. The second
parameter describes the syntactic separators -- described with a type
-- that may be found between elements of the list.  In this case
\cd{Peor}, the {\em end-of-record} type, may be found between each
element of the array.  In this case, the end-of-record marker should
be set to a newline character as there is one record per line in the
file. Howeve, this marker is not set in the description itself.
Rather, the end-of-record is left abstract, and set later by the
client program that uses the generated libraries. The third parameter
is the terminator for the list.  In this case, the terminator is
\cd{Peof}, a type that describes the special end-of-file marker.  It
is often necessary to have additional termination conditions for
lists, such as termination dependent on the number of elements read so
far, but we omitted this additional option for expository purposes.

The definition of \cd{Events} indicates that this part of the
\dibbler{} data will contain a sequence of \cd{Event}s separated by
vertical bars and terminated by an end-of-recorde marker.  Each
\cd{Event} is a string terminated by a vertical bar, followed by a
vertical bar and ending with an unsigned 32-bit integer.  The
interesting part of this sequence is the presence of the type
\cd{'|'}.  In type-theoretic terms, this is a {\em singleton type}.
It states that one should expect exactly the character \cd{'|'} in the
input stream at this point.  Other singletons appear in the summary
header type as \cd{"0|"} and \cd{Peor}.  Any expression that appears
embedded in a type as a singleton must be effect-free.

The type \cd{Order\_header} is a record type that indicates the data
format involves the sequence of items described by the fields of the
record.  Notice that there are two different sorts of fields:
anonymous fields (a second form of singleton type) containing
directives to parse a particular character (\cd{'|'}) or string
(\cd{"0|"}) and fields with names.

The second named field, \cd{att\_order\_num}, reveals two other
features of \padsml: dependency and constraints.  Here,
\cd{att\_order\_num} is constrained to be less than \cd{order\_num},
the value parsed in an earlier field.  This is a relatively simple
constraint on the correctness of the ad hoc data format.  In practice,
constraints can become very rich involving properties such as
sortedness of records in an array, definitions of expected characters,
restrictions on date and time ranges, constraints on IP address
domains, restrictions on phone number area codes and virtually
infinite variety of other possibilities. In general, constrained
types have the form \cd{[x:T | e]} where \cd{e} is an arbitrary pure
boolean expression.  

The last interesting feature in the \dibbler{} example is the datatype
definition of \cd{Dib\_ramp}.  It describes two alternatives for a
portion of data, either an integer alone or the fixed string
\cd{"no\_ii"} followed by an integer.  To parse data in this format,
the parser will attempt to parse the first branch and only if it fails
will it attempt to parse the second branch.  Notice how this semantics
differs from similar constructs in regular expressions and
context-free grammars, which non-deterministically choose between
alternatives.  Fortunately, we have yet to come across an ad hoc data
source where we wish we had nondeterministic choice.\footnote{\padsc{}
  can recognize string data based on regular expressions.
  Non-determinism here has been useful, but as it has been confined to
  parsing elements of the \cd{Pstring} base type, it has had no impact
  on the overall parsing algorithm.}

\begin{figure}
\begin{code}\scriptsize
\kw{ptype} Semicolon = Pcharlit(';')
\kw{ptype} Rparen= Pcharlit(')')
\mbox{}
\kw{ptype} Entry = \{
  name : Pstring(':'); ':'; dist : Pfloat32
\}
\mbox{}
\kw{pdatatype} Tree =
  Interior of '(' * (Tree, Semicolon, Rparen) Plist * ')'
| Leaf of Entry\end{code}

%%% Local Variables: 
%%% mode: latex
%%% TeX-master: "paper"
%%% End: 

Tiny fragment of Newick data:

{\scriptsize
\begin{verbatim}
(((erHomoC:0.28006,erCaelC:0.22089):0.40998,(erHomoA:0.32304,
(erpCaelC:0.58815,((erHomoB:0.5807,erCaelB:0.23569):0.03586,
erCaelA:0.38272):0.06516):0.03492):0.14265):0.63594,
(TRXHomo:0.65866,TRXSacch:0.38791):0.32147,TRXEcoli:0.57336)
\end{verbatim}
}
  \caption{Simplified tree-shaped Newick data}
  \label{fig:newick}
\end{figure}

\subsection{Recursive Descriptions}

Another powerful feature of \padsml{} is the ability to describe data
sources with recursive structure.  A representative example of
recursive, biological data comes courtesy of Steven Kleinstein,
program coordinator of Princeton's Picasso project for
interdisciplinary research in computational sciences.  Kleinstein is
in the process of building a simulator to study the proliferation of B
lymphocytes during an immune response.  Data needed for his
simulations is represented in a variant of the Newick format, which is
a flat representation of trees used by many biologists~\cite{newick}.
In Newick, leaves of the tree are string labels followed by a colon
and a number.  A parent node in the tree introduces a collection of
children by placing a sequence of trees within parens.  Following the
parens is a colon and a number, as is the case for the leaf node.  The
numbers represent the ``distance'' that separates the child from the
parent.  In Kleinstein's case, the distance is the number of mutations
that occur in the antibody receptor genes of B lymphocytes.  Each line
of a file may contain a different tree, terminated by a semi-colon.

\figref{fig:newick} gives a description of Newick and a short fragment 
of example data.  Despite the relative complexity of the structure of the data,
the description is remarkably concise and elegant.  Notice that the data type
definition of \cd{tree} is recursive --- we have not been able to find any
effective description of this data source without it.

\begin{figure*}
  \centering
\begin{code}\scriptsize
 2:3004092508||5001|dns1=abc.com;dns2=xyz.com|c=slow link;w=lost packets|INTERNATIONAL
 3:|3004097201|5074|dns1=bob.com;dns2=alice.com|src_addr=192.168.0.10;
 dst_addr=192.168.23.10;start_time=1234567890;end_time=1234568000;cycle_time=17412|SPECIAL
\end{code}  
  \caption{Simplified network monitoring data.  This 
data containts two alarm records.  Extra newlines 
were inserted mid-record so the data would fit on a page.}
  \label{fig:darkstar-records}
\end{figure*}

Another use of recursion is to define homogenous, arbitrary-length
sequences. Such sequences can be described in \padsc{}with the
built-in\kw{Parray} type constructor. In \padsml{}, building such a
construct into the language, then, becomes a matter of efficiency, not
expressive power. The only thing lacking, then, is the ability to
define a list construct for any element. We meet this need with the
polymorphic descriptions, which we describe next.
  
\subsection{Polymorphic Descriptions}

The last key feature of \padsml{} is the ability to define
parameterized descriptions.  In \padsc, description parameters were
limited to values.  In \padsml, descriptions maybe parameterized by
both values and types.  Type parameterization allows descriptions to
be reused at a variety of different types.  This feature helps make
data descriptions more concise and allows programmers to define
convenient libraries of reusable description components. For example,
the \cd{Plist} type that appears in both the \dibbler{} and Newick
formats, is such a reuseable component.

Our final example, shown in Figure~\ref{fig:darkstar-records}, will
illustrate the usefulness of descriptions parameterized by both types
and values and introduce a couple of other useful features as well.
This example displays alarm data recorded by a network link monitor
used by the \darkstar{} project at AT\&T.  Each alarm signals some
problem with a network link.

Now, at this point, we could write out an English description
of the \darkstar{} data, and indeed, in an earlier version of this paper
we did so.  However, English is a terrible specification language
for ad hoc data, and every description we have attempted to give has not 
only been imprecise, it has also made for some very 
boring reading.  Hence, we dispense with the informal
English and head straight for the \padsml{} description,
which is shown in \figref{fig:darkstar-ml}.

\begin{figure}
  \centering
  \begin{code}\scriptsize
(* Pstring terminated by ';' or '|'. *)
\kw{ptype} SVString = Pstring\_SE("/;|\\\\|/")
\mbox{}
(* Generic name value pair. Accepts predicate 
   to validate name as argument.� *)
\kw{ptype} (Alpha) Pnvp(p : string -> bool) =
      \{ name : [name : Pstring('=') | p name]; 
            '='; 
       value : Alpha \}
\mbox{}
(* Name value pair with name specified. *)
\kw{ptype} (Alpha) Nvp(name:string) = 
   Alpha Pnvp(fun s -> s = name)
\mbox{}
(* Name value pair with any name. *)
\kw{ptype} Nvp\_a = SVString Pnvp(fun \_ -> true)
\mbox{}
\kw{ptype} Details = \{
      source      : Pip Nvp("src\_addr");
';';  dest        : Pip Nvp("dest\_addr");
';';  start\_time  : Ptimestamp Nvp("start\_time");
';';  end\_time    : Ptimestamp Nvp("end\_time");
';';  cycle\_time  : Puint32 Nvp("cycle\_time")
\}

\kw{pdatatype} Info(alarm\_code : int) =
  match alarm\_code with
    5074 -> Details of Details
  | \_    -> Generic of Nvp\_a Plist(';','|')

\kw{pdatatype} Service = 
    DOMESTIC      of "DOMESTIC" 
  | INTERNATIONAL of "INTERNATIONAL" 
  | SPECIAL       of "SPECIAL"

\kw{ptype} Alarm = \{
       alarm    : [i : Puint32 | i = 2 or i = 3];
 ':';  start    : Ptimestamp Popt;
 '|';  clear    : Ptimestamp Popt;
 '|';  code     : Puint32;
 '|';  src\_dns  : SVString Nvp("dns1");
 ';';  dest\_dns : SVString Nvp("dns2");
 '|';  info     : Info(code);
 '|';  service  : Service
\} 

\kw{ptype} Source = Alarm Plist('\\n',\kw{peof})\end{code}
%
%\kw{let} checkCorr ra = ...
%\kw{ptype} Alarm = [x:Raw\_alarm | checkCorr x]



  \caption{Description of \darkstar{} data.}
  \label{fig:darkstar-ml}
\end{figure}

One of the interesting facets of this data source is the fact that it
contains multiple different types of name-value pairs.  In the second
definition from the top, \cd{Pnvp}, we take advantage of both the type
and value parameterization of types to encode all the different
variations.\footnote{This fragment of code contains both types
  \cd{Pstring} and \cd{string}.  These types are subtly different as
  the first is the type of a ??? containing both a string
  representation and a parse descriptor whereas the second is an
  ordinary string.  The reader can safely ignore the distinction for
  now.}  The type parameters are specified to the left of the type
name, as is customary in ML.  To the right of the type name in
parentheses, we give the value parameter and its type.  In the case of
\cd{Pnvp}, there is a type parameter name \cd{Alpha} and a value
parameter named \cd{p}.  Informally, \cd{Alpha pnvp(p)} is a
name-value pair where the value is described by \cd{Alpha} and the
name must satisfy the predicate \cd{p}.  More precisely, \cd{Pnvp} is
defined to be a record with three fields.  The first field of the
record has been given a constrained type. Data \cd{d} satisfies this description if it
satisfies \cd{T} and boolean \cd{e} evaluates to true when the parsed
representation of \cd{d} is substituted for \cd{x}.  If the boolean
evaluates to false, the data contains a semantic error.

The \cd{Pnvp} type is used in both of the following type definitions.
In the case of the \cd{Nvp} definition, the predicate is instantiated
with a test for a specific string but the type parameter remains 
abstract.  In the \cd{Nvp\_a} 
definition, the name can be arbitrary, but the value must have 
type string. Later in the description, \cd{Nvp}'s
are used with ip addresses, timestamps and integers. 

% The source type is an array of \cd{alarm}s, where each alarm is a
% \cd{raw\_alarm}, constrained to ensure that the alarm number is
% properly correlated with the timestamps.  We check this correlation
% with the function \cd{checkCorr}.  The type \cd{raw\_alarm} closely
% follows the description above. We highlight a few important features.
% First, we note that the type of the field \cd{info} depends on the
% alarm code, reflecting the text above. More interestingly, the type
% \cd{info} is implemented with a switched datatype, deciding how to
% parse based on the parameter \cd{alarm\_code}.  Next, we note that the
% description includes five different types of name-value pairs. We take
% advantage of both the type and value parameterization of types to
% encode all of these pair types based on one common description,
% \cd{pnvp}. This type is polymorphic in the type of the value and takes
% an arbitrary constraint \cd{c} as an argument. The type \cd{nvp} is
% polymorphic in the type of the value, but takes the expected name of
% the string as an argument. 

The \darkstar{} description also introduces a new form of datatypes,
appearin in the \cd{Info} type.  This datatype is parameterized by an
\cd{alarm\_code}.  Rather than determine the branch of the datatype to
choose based on the data about to be parsed, the decision is made
based on the \cd{alarm\_code}, data which has likely been extracted
from some previous point in the source.  More specifically, if the
alarm code is \cd{5074}, the format specification given by the
\cd{Details} constructor will be used to parse the current data.
Otherwise, the format given by the \cd{Generic} constructor will be
used to parse the current data. Such descriptions are called
\textit{switched} datatypes.

Another construct new to the \darkstar{} description is the type qualifier
\cd{omit}. We use \cd{omit} in the \cd{Service} datatype to specify
that the parsed string literal should be omitted from the parsed
data. We can do this because we can discern from the datatype constructor
which string was found in the data source.

%%% Local Variables: 
%%% mode: latex
%%% TeX-master: "paper"
%%% End: 
