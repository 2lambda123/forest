\documentclass[11pt]{article}
\usepackage{times,url}
\usepackage{code} 
%\usepackage{proof}
%\usepackage{newcode}
%\usepackage{epsfig}

\newcommand{\cut}[1]{}

\newcommand{\appref}[1]{Appendix~\ref{#1}}
\newcommand{\secref}[1]{Section~\ref{#1}}
\newcommand{\tblref}[1]{Table~\ref{#1}}
\newcommand{\figref}[1]{Figure~\ref{#1}}
\newcommand{\listingref}[1]{Listing~\ref{#1}}
%\newcommand{\pref}[1]{{page~\pageref{#1}}}

\newcommand{\eg}{{\em e.g.}}
\newcommand{\cf}{{\em cf.}}
\newcommand{\ie}{{\em i.e.}}
\newcommand{\etc}{{\em etc.\/}}
\newcommand{\naive}{na\"{\i}ve}
\newcommand{\role}{r\^{o}le}
\newcommand{\forte}{{fort\'{e}\/}}
\newcommand{\appr}{\~{}}

\newcommand{\bftt}[1]{{\ttfamily\bfseries{}#1}}
\newcommand{\kw}[1]{\bftt {#1}}
\newcommand{\Pthen}{\kw{Pthen}}
\newcommand{\pads}{\textsc{pads}}
\newcommand{\padsl}{\textsc{padsl}}
\newcommand{\padst}{\textsc{pads/t}}
\newcommand{\datatype}{\textsc{PADS/T}}
%\newcommand{\datatype}{\textsc{DataType}}
\newcommand{\C}{\textsc{C}}
\newcommand{\perl}{\textsc{Perl}}
\newcommand{\ml}{\textsc{ml}}
\newcommand{\sml}{\textsc{sml}}
\newcommand{\smlnj}{\textsc{sml/nj}}
\newcommand{\java}{\textsc{java}}
\newcommand{\ddl}{\textsc{ddl}}
\newcommand{\xml}{\textsc{xml}}
\newcommand{\datascript}{\textsc{DataScript}}
\newcommand{\packettypes}{\textsc{PacketTypes}}
\newcommand{\erlang}{\textsc{Erlang}}

\newcommand{\Core}{Ad hoc}
\newcommand{\core}{ad hoc}
\newcommand{\pvalue}{\core{} value}
\newcommand{\ppat}{\core{} pattern}
\newcommand{\ptype}{\core{} type}

\newcommand{\padsc}{\textsc{pads}/\C{}}
\newcommand{\padsml}{\textsc{pads}/\ml{}}

\newcommand{\dibbler}{Sirius}
\newcommand{\ningaui}{Altair}
\newcommand{\darkstar}{Regulus}

\newcommand{\pdgood}{{\tt G}}
\newcommand{\pdbad}{{\tt B}}
\newcommand{\pdnest}{{\tt N}}
\newcommand{\pdsem}{{\tt S}}
\newcommand{\ptypes}{T}
\newcommand{\patreadpd}[2]{{\tt #1<<#2>>}}
\newcommand{\btm}{\cd{BOT}}


\newcommand{\lsem}{{[\![}}
\newcommand{\rsem}{{]\!]}}


\newcommand{\figHeight}[4]{\begin{figure}[tb]
	\centerline{
	            \epsfig{file=#1,height=#4}}
	\caption{#2}
	\label{#3}
	\end{figure}}

%% Environment for typesetting BNF grammars. Uses display math mode.
\newenvironment{bnf}
     {%% local command definitions:
        %% BNF definition symbol
      \def\->{\rightarrow}
%%      \def\::={{::=} &}
      \def\::={\bnfdef &}
      \def\|{\bnfalt}
      \newcommand{\name}[1]{\text{##1}}
        %% non-terminal
      \newcommand{\nont}[1]{{##1}}
      \newcommand{\meta}[1]{& ##1 &}
      \newcommand{\descr}[1]{& \text{// ##1}}
      \newcommand{\opt}[1]{ [##1] }
      \newcommand{\opnon}[1]{\opt{\nont{##1}}}
      \newcommand{\none}{\epsilon}
      \newcommand{\nwln}{\\ &&&}
      \newcommand{\nlalt}{\\ && \| &}
      \[\begin{array}{lrlll}
     }
     {\end{array}\]}

\newcommand{\mcd}[1]{\mathtt{#1}}
\newcommand{\ppair}[3]{#1{:}#2 \mathrel{**} #3}
\newcommand{\parray}[3]{#1\;\mcd{Parray}(#2,#3)}
\newcommand{\pset}[3]{\{#1{:}#2\,|\,#3\}}
\newcommand{\pstream}[1]{#1\;\mcd{stream}}
\newcommand{\precord}[1]{\{\{#1\}\}}


%%%%%%%%%%%%%%%%%%%%%%%%%%%%%%%%%%%%%%%%%%%%%%%%%%%%%%%%%%%%%%%%%%%%%%%%%%%%
\voffset             0in    %  top vertical offset
\hoffset             0in    %  left horizontal offset
\oddsidemargin       0pt    %  Left margin on odd-numbered pages.
\evensidemargin      0pt    %  Left margin on even-numbered pages.
\topmargin           0pt    %  Nominal distance from top of page to top of
\headheight          0pt    %  Height of box containing running head.
\headsep             0pt    %  Space between running head and text.
\textwidth         6.5in    %  Width of text on page
\textheight          9in    %  Height of text on page
\setlength{\parskip}{.05in}
\renewcommand{\floatpagefraction}{.9}
\renewcommand{\textfraction}{0.1}
\renewcommand{\baselinestretch}{1.01}
%\input{macro}  %%%  other macro definitions

%%%%%%%%%%%%%%%%%%%%%%%%%%%%%%%%%%%%%%%%%%%%%%%%%%%%%%%%%%%%%%%%%%%%%%%%%%%%
\begin{document}
\setcounter{page}{1}
\pagenumbering{roman}
\appendix
%%%%%%%%%%%%%%%%%%%%%%%%%%%%%%%%%%%%%%%%%%%%%%%%%%%%%%%%%%%%%%%%%%%%%%%%%%%%
\section{NSF 04-528 Project Summary:  \\
Automatic Tool Generation for Ad Hoc Scientific Data}
%%%%%%%
%%%%%%%%%%%%%%%%%%%%%%%%%%%%%%
% \centerline{
% \begin{tabular}{cc}
%           \\[-1ex]
% \multicolumn{2}{c}{Automatic Tool Generation for Ad Hoc Scientific Data} \\
%           & \\[-1ex]
% David Walker (PI)\\
% Princeton University\\[2ex]
% \end{tabular}}
%%%%%%%%%%%%%%%%%%%%%%%%%%%%%%

\paragraph*{Intellectual Merits:} 
In every scientific discipline, researchers are digitizing their knowledge and
beginning to use computational methods to categorize,
query, filter, search, diagnose, and visualize their data.  
While this effort is leading to remarkable advances,
it is also generating enormous amounts of {\em ad hoc data}.
Ad hoc data is any data for which standard data processing tools
such as query engines, statistical packages, graphing tools, parsers, printers,
transformers or programming libraries are not readily available.
This data, which is often unpredictable, poorly documented,
filled with errors, high volume and unwieldy,
poses tremendous challenges to its users and the software
that manipulates it.  We cannot maximize the productivity of top 
computational scientists unless we can maximize the efficiency and 
accuracy with which they deal with this data.

Our overall goal is to alleviate the burden, risk and confusion
associated with ad hoc data.  Our overall strategy is (1) to develop a
specification language capable of precisely describing any ad hoc data
format at an easy-to-understand, high level of abstraction and (2) to
automatically generate useful data processing tools including
programming libraries, a query engine, format converters, a
statistical summarizer, a histogram generator and others.
 
We will accomplish our task by building upon our preliminary data
description and processing system, \pads{}, created by the PI and his
collaborators.  This preliminary system demonstrates the basic
feasibility of the \pads{} approach, but there is much research to do.
While the work to date demonstrates the feasibility of the \pads{}
approach, the \pads{} design and implementation are still in their
infancy: We have discovered many common, real-world data formats that
the current PADS infrastructure is incapable of describing, parsing or
analyzing.  To address these deficiencies, we propose to explore
innovative ways of extending the basic \pads{} specification language
and related infrastructure.  The second thrust of our research
involves improving the automatic tool generation infrastructure in the
\pads{} system.  The current tool generation system is brittle,
inflexible and suffers from subpar performance.  We will develop a
novel extension to \pads{} that allows users to augment \pads{}
descriptions with application-specific attributes and customizations.
The attributes convey semantic information to the tool generators and
the customizations improve error-handling and performance.  Our new
architecture has the promise of being substantially more robust,
flexible and efficient.  Finally, in order to improve the reliability
of the infrastructure, we will formalize \pads{} and prove important
correctness properties of our formal model.  Overall, this proposal
involves challenging research in language design, efficient systems
implementation and semantic analysis, all aimed at solving real-world
data processing problems.

\paragraph*{Broader Impacts:}  We are collaborating with Kathleen Fisher
and Mary Fernandez at
AT\&T who will be able to use our tools to address real problems such
as telephone fraud detection.  In addition, our tools will be freely
available to researchers and scientists over the web.  Moreover, part
of our mission will be to work with specific biologists and
physicists at Princeton and the broader community to help them with
their data processing needs.  We have already been meeting with Olga
Troyanskaya, who works in Princeton's Lewis-Sigler Institute for
Integrative Genomics on pathway modeling and analysis of
protein-protein interactions, and with Rachel Mandelbaum, Ph.D. candidate
in physics who analyzes cosmology data.  Many of our proposed extensions to
\pads{} were inspired directly by their data processing needs.  

The collaboration between Computer Science and Genomics will also be an
excellent platform for developing interdisciplinary undergraduate
research projects.  The PI has a proven track-record 
for following through with undergraduate and graduate
educational plans:  Last year, his undergraduate student advisee, Rob Simmons,
won the Princeton Computer Science Senior Thesis Award;
last year and the year before he organized two NSF-sponsored summer schools
on secure and reliable computing.


\newpage
% %%%%%%%%%%%%%%%%%%%%%%%%%%%%%%%%%%%%%%%%%%%%%%%%%%%%%%%%%%%%%%%%%%%%%%%%%%%%
% \section{Table of Contents}
% \tableofcontents\
% \vspace*{.1in} 
% \begin{center}
% {\Large\bf\em{}This is a place-holder. Fastlane generates this page automatically.}
% \end{center} 
% \newpage
%%%%%%%%%%%%%%%%%%%%%%%%%%%%%%%%%%%%%%%%%%%%%%%%%%%%%%%%%%%%%%%%%%%%%%%%%%%%
\pagenumbering{arabic}
\setcounter{page}{1}
%%%%%%%%%%%%%%%%%%%%%%%%%%%%%%%%%%%%%%%%%%%%%%%%%%%%%%%%%%%%%%%%%%%%%%%%%%%%
\section{Project Description}

\subsection{Introduction}
\label{ssec:intro}

An {\em ad hoc data format} is any nonstandard data format for which
parsing, querying, analysis or transformation tools are not readily
available.  \xml{} is not an ad hoc data format --- there are hundreds
of tools for reading, querying, and transforming \xml{}.  However,
network administrators,
data analysts, computer scientists,
biologists, chemists, and physicists deal with ad hoc
data in a myriad of complex formats on a daily basis.
This data, which is often unpredictable, poorly documented,
and filled with errors
poses tremendous challenges to its users and the software
that manipulates it.  
Our goal is to alleviate the burden, risk and confusion associated
with ad hoc data by developing a universal data processing system
capable of 

\begin{enumerate}
\item concisely and accurately describing any ad hoc data source at an 
easy-to-understand, high-level of abstraction, and
\item automatically generating a suite of highly reliable and
efficient data processing tools that make understanding, manipulating 
and transforming ad hoc data an effortless task.
\end{enumerate}

Together with his collaborators, the PI has already 
begun to develop a system called \pads{} (Processing
Ad hoc Data Sources)~\cite{padsproj,fisher+:700,launchpads} that 
helps to address some of these concerns.
It includes a high-level, declarative language for describing
ad hoc data formats and it is possible to generate tools for
parsing, printing, querying, generating histograms and 
statistical summaries of data, and transforming 
data into \xml.  An online demo~\cite{padsproj} illustrates
our progress to date.
Our preliminary work has demonstrated
there is much research to do in this area.  In particular, if funded we 
will pursue research on three fronts:

\begin{enumerate}
\item Extend the preliminary system with new features we have discovered are indispensible to
scientists who work with ad hoc data.  In particular, the preliminary system only operates
on single files.  We will extend it so it may operate over and integrate collections of files,
either local or distributed across a network.  We will also augment PADS so that
programmers may write simple, high level and easy-to-understand PADS transforms that can fix data errors
and filter, standardize, coerce, compress, or santize data fields.
\item  Improve automatic tool generation.  We will research new tool-generation architectures
that allow anyone to add new tools to the tool-set, which is currently is currently fixed.
We will also provide mechanisms that allow data descriptions to communicate useful semantic information
to downstream tools.
\item Develop the semantic theory of PADS.  We have a partial PADS parsing semantics based on dependent type
theory but we will extend this semantics to cover our new features and generalize it so it may
describe other components of the systems including printers and other tools.
\item In order to have a broad impact across research in the sciences,
we will develop descriptions and tools for researchers in
physics, biology and chemistry.
More specifically, we have already developed a partnership with
Olga Troyanskaya in Princeton's Lewis-Sigler Institute for 
Integrative Genomics.  Her Magic database system~\cite{magic}
reads several data sources
and applies a Baysian analysis in order to discover gene function.
Through the use of PADS, we will transform Magic into a truly generic
system by allowing Magic to import any data source -- to add new sources,
all a user will have to do is write a simple, high-level PADS description.
\item To extend and support our relationship with the Genomics Institute,
and to have a broad impact on interdisciplinary education,
we plan to develop undergraduate research projects in which
computer science majors use \pads{} to help biologists 
with their data processing problems.  
\end{enumerate}


Overall, our research combines novel language design, high-performance
systems engineering and theoretical analysis.  It will substantially
increase the overall productivity of data analysts, researchers and
software architects who deal with ad hoc data regularly, and it will
improve the security and reliability of the software they produce.
Finally, our research will also have a broad impact on research in the
natural sciences, where ad hoc data is pervasive, and on
interdisciplinary computer science.

In the rest of this introduction, we will explain the immense
challenges and risks posed by ad hoc data in more detail.  We will
explain how our proposed system as a whole can deal with them.  We
will then describe some of the features of \pads{} we have already
implemented.  After this introductory section, we will describe the
key components of our new research agenda and explain how our research
will make a broad impact on disciplines well outside the bounds of
traditional computer science and on interdisciplinary education.

\subsubsection{The Challenges of Ad Hoc Data}

There are vast amounts of useful data stored in
traditional databases and \xml{} formats, but there is just as much in
ad hoc formats.  \figref{figure:data-sources} provides some information
on ad hoc data formats from several different domains ranging from genomics
to cosmology to networking to finance to internal corporate billing information.  
They include ASCII, binary, and Cobol data formats, with
both fixed and variable-width records, ranging in size from
relatively small files through network applications which process over
a gigabyte per second.  Common errors include undocumented data,
corrupted data, missing data, and multiple missing-value
representations.


\begin{figure*}
\begin{center}
\begin{tabular}{@{}|l|l|l|l|l|}
\hline
Name: Use                           & Representation    & Processing Problems \\ \hline\hline
Gene Ontology (GO)~\cite{geneontology}:                  & Variable-width    & White-space ambiguities \\
Gene Product Information 	      & ASCII records &  \\ \hline
SDSS/Reglens Data~\cite{mandelbaum+:reglens}:                & Floating point numbers, & Repeated multiplicative error \\
Weak gravitational lensing analysis   & among others & \\ \hline
Web server logs (CLF):                & Fixed-column      & Race conditions on log entry\\ 
Measuring web workloads               & ASCII records     & Unexpected values\\ \hline
AT\&T Call detail data:                          & Fixed-width       & Undocumented data\\
Phone call fraud detection            & binary records  & \\ \hline 
AT\&T billing data:                 & Cobol             &  Unexpected values\\ 
Monitoring billing process          &                   & Corrupted data feeds \\ \hline
Newick:   Immune                    & Fixed-width ASCII records & None \\ 
system response simulation          & in tree-shaped hierarchy &\\ \hline                                
OPRA:                               & Mixed binary \& ASCII records 
                                                       & 100-page informal \\
Options-market transactions         & with data-dependent unions & documentation \\ \hline
Palm PDA:                           & Mixed binary \& character & No high-level  \\
Device synchronization              & with data-dependent constraints & documentation available\\ \hline
\end{tabular}
\caption{Selected ad hoc data sources.}
\label{figure:data-sources}
\end{center}
\end{figure*}


Processing ad hoc data is challenging for a variety of further reasons. 
First, ad hoc data typically arrives ``as is'': the analysts
who receive it can only say ``thank you,'' not request a more
convenient format.  Second, documentation for the format may not exist
at all, or it may be out of date.  A common phenomenon is for a field
in a data source to fall into disuse.  After a while, a new piece of
information becomes interesting, but compatibility issues prevent data
suppliers from modifying the shape of their data, so instead they
hijack the unused field, often failing to update the documentation in
the process.

Third, such data frequently contain errors, for a variety of reasons:
malfunctioning equipment, race conditions on log entry~\cite{wpp},
non-standard values to indicate ``no data available,'' human error in
entering data, unexpected data values, \etc{} The appropriate response
to such errors depends on the application.  Some applications require
the data to be error free: if an error is detected, processing needs
to stop immediately and a human must be alerted.  Other applications
can repair the data, while still others can simply discard erroneous
or unexpected values.  For some applications, errors in the data can
be the most interesting part because they can signal where two systems
are failing to communicate.

% A fourth challenge is that ad hoc data sources can be high volume:
% AT\&T's call-detail stream contains roughly 300~million calls per day
% requiring approximately 7GBs of storage space. Although this data is
% eventually archived in a database, analysts mine it profitably before
% such archiving~\cite{kdd98,kdd99}. More challenging, the \ningaui{}
% project at AT\&T accumulates billing data at a rate of 250-300GB/day,
% with occasional spurts of 750GBs/day. Netflow data arrives from Cisco
% routers at rates over a gigabyte per second~\cite{gigascope}! Such
% volumes mean it must be possible to process the data without loading
% it all into memory at once.

Finally, before anything can be done with an ad hoc data source,
someone has to produce a suitable parser for it.  Today, people tend
to use \C{} or \perl{} for this task.  Unfortunately, writing parsers
this way is tedious and error-prone, complicated by the lack of
documentation, convoluted encodings designed to save space, the need
to produce efficient code, and the need to handle errors robustly to
avoid corrupting down-stream data.  Moreover, the parser writers'
hard-won understanding of the data ends up embedded in parsing code,
making long-term maintenance difficult for the original writers and
sharing the knowledge with others nearly impossible.

\subsubsection{\pads{}:  Taking on the Challenge of Ad Hoc Data}

The \pads{} system makes life easier for data analysts by addressing
each of the concerns outlined in the previous section.
It provides a declarative data description
language that permits analysts to describe the physical layout of
their data, \textit{as it is}.  The language also permits analysts to
describe expected semantic properties of their data so that deviations can
be flagged as errors. The intent is to allow analysts to capture in a
\pads{} description all that they know about a given data source
and to provide the analysts with a library of useful routines in exchange. 


\pads{} descriptions are intended to be concise enough to serve as
documentation and flexible enough to describe many ASCII, binary,
Cobol, and mixed data formats.  In addition, useful software tools
can be generated from the descriptions and this feature provides
strong incentive for keeping the descriptions current, allowing them
to serve as living documentation.  

Given a \pads{} description, the \pads{} compiler produces
customizable \C{} libraries and tools for parsing, manipulating, and
summarizing the data.  The core \C{} library includes functions for
reading the data, writing it back out in its original form, writing it
into a canonical \xml{} form, pretty printing it in forms suitable for
loading into a relational database, and accumulating statistical
properties.  An auxiliary library provides an instance of the data API
for Galax~\cite{galax,galaxmanual}, an implementation of XQuery.  This
library allows users to query data with a \pads{} description as if
the data were in \xml{} without having to convert to \xml, which often
can avoid an 8-10 times blow-up in space requirements.  

The declarative nature of \pads{} descriptions facilitates the
insertion of error handling code.  The generated parsing code checks
all possible error cases: system errors related to the input file,
buffer, or socket; syntax errors related to deviations in the physical
format; and semantic errors in which the data violates user
constraints.  Because these checks appear only in generated code, they
do not clutter the high-level declarative description of the data
source.  Moreover, since tools are generated
automatically by a compiler rather than written by hand, 
they are far more likely to be robust
and far less likely to have dangerous vulnerabilities such as
buffer overflows.  Moreover, as formats change and are updated,
one can make a small change in the high-level description
and the compiler will takes care of the rest.  It is extremely unlikely
new vulnerabilities or buffer overflows will be added to the code
base over time as formats change.

The result of a parse is a pair consisting of a canonical
in-memory {\em representation} of the data and a {\em parse descriptor}. 
The parse
descriptor precisely characterizes both the syntactic and the semantic
errors that occurred during parsing.  This structure allows analysts
to choose how to respond to errors in application-specific ways.

% With such huge datasets, performance is critical. The \pads{} system
% addresses performance in a number of ways.  First, we compile the
% \pads{} description rather than simply interpret it to reduce run-time
% overhead.  Second, the generated parser provides multiple entry
% points, so the data consumer can choose the appropriate level of
% granularity for reading the data into memory to accommodate very large
% data sources.  Finally, we parameterize library functions by
% \textit{masks}, which allow data analysts to choose which semantic
% conditions to check at run-time, permitting them to specify all known
% properties in the source description without forcing all users of that
% description to pay the run-time cost of checking them.

% Given the importance of the problem, it is perhaps surprising that
% more tools do not exist to solve it.  \xml{} and relational databases
% only help with data already in well-behaved formats.  Lex and Yacc are
% both over- and under- kill.  Overkill because the division into a
% lexer and a context free grammar is not necessary for many ad hoc data
% sources, and under-kill in that such systems require the user to build
% in-memory representations manually, support only ASCII sources, and
% don't provide extra tools.  ASN.1~\cite{asn} and related
% systems~\cite{asdl} allow the user to specify an in-memory
% representation and generate an on-disk format, but this doesn't help
% when given a particular on-disk format.  Existing ad hoc description
% languages~\cite{gpce02,sigcomm00,erlang} are steps in the right
% direction, but they focus on binary, error-free data and they do not
% provide auxiliary tools.

% We propose to extend and improve the current \pads{}\ prototype by engaging
% in three intertwined research efforts:

% \begin{enumerate}
% \item Our experience with real-world data has revealed that
% the current \pads{}\ system is unable to process many common
% data formats.  We propose a number of extensions to \pads{} 
% that will greatly improve its expressive power and enable
% us to handle many more important ad hoc formats.
% \item While \pads{}\ can automatically generate tools for providing
% statistical data summaries, querying ad hoc data and generating
% \xml, the tool generation process is extremely brittle, inflexible
% and results in low-performance software.  We propose to redesign
% the automatic tool generation process using a principled methodology
% based upon a novel \pads{}-attribute system that can communicate
% semantic information across tools.  Our new methodology will
% improve the flexibility and reliability of the \pads{} system.
% \item There is no formal description of the \pads{} language
% and no semantics for the compiler or tools.  We propose to
% formalize the language and analyze the semantics of the
% compiler.  Our formal analysis may find bugs or flaws in the
% \pads{} system and will improve our confidence in the implementation.
% \end{enumerate}

% \noindent
% In addition, part of our mission will be to use \pads{} to provide
% data processing tools for researchers in the natural sciences including
% biology, chemistry and physics.  

% In building the \pads{} system, we faced challenges on two fronts:
% designing the data description language and
% crafting the generated libraries and tools.
% In the rest of the paper, 
% we describe each of these designs.
% We also give examples to show how useful they have been at AT\&T.


\subsubsection{The Current \pads{} Language Infrastructure}

% \figref{figure:data-sources} summarizes some of the sources we have
% worked with.  They include ASCII, binary, and Cobol data formats, with
% both fixed and variable-width records, ranging in size from
% relatively small files through network applications which process over
% a gigabyte per second.  Common errors include undocumented data,
% corrupted data, missing data, and multiple missing-value
% representations.


% \subsection{Common Log Format}
% Web servers use the Common Log Format (CLF) to log client
% requests~\cite{wpp}.  Researchers use such logs to measure
% properties of web workloads and to evaluate protocol changes
% by "replaying" the user activity recorded in the log.
% This ASCII format consists of a sequence of
% records, each of which has seven fields: the host name or IP address
% of the client making the request, the account associated with the
% request on the client side, the name the user provided for
% authentication, the time of the request, the actual request, the
% \textsc{http} response code, and the number of bytes returned as a
% result of the request.  The actual request has three parts: the
% request method (\eg, \texttt{GET}, \texttt{PUT}), the requested
% \textsc{uri}, and the protocol version.  In addition, the second and
% third fields are often recorded only as a '-' character to indicate
% the server did not record the actual data.  \figref{figure:clf-records}
% shows a couple of typical records.


In order to understand further how \pads{} can be used,
will take a look at an example of ad hoc data:
a tiny fragment of data from the Gene Ontology Project~\cite{geneontology}.  
Data in this format is widely used by molecular biologists to
analyze gene products produced by various organisms.
\figref{figure:dibbler-records} shows a tiny sample of
the format.  Cursory examination of the example data reveals
first of all that this is an ASCII format with two 
logical parts, a header portion (that portion beginning with
\texttt{format-version} and ending with the first blank line)
and a main portion, which includes a sequence of 
information blocks separated by blank lines (only one such block 
is shown in the example).  The format
designers call these information blocks {\em stanzas}.
Each stanza begins with a stanza tag, which may be either
\texttt{[Term]}, as shown in the example data, or
\texttt{[Typeref]}.  On the following line,
the first element of each stanza, is
the string \texttt{id}, followed by a colon, followed by a
GO identifier.  A GO identifier is the word GO, followed
by another colon, followed by a string.  
After the first mandatory id field, the rest of the stanza includes
any number of identifier, colon, field entries.
% While this format is actually remarkably simple, and
% we have left out several details in this coarse description,
% it should be apparent that English is a poor
% language for describing data formats!

\begin{figure*}
\begin{small}
%\begin{center}
\begin{verbatim}
format-version: 1.0
date: 11:11:2005 14:24
saved-by: midori
auto-generated-by: DAG-Edit 1.419 rev 3
default-namespace: gene_ontology
remark: cvs version: $Revision: 1.1 $
subsetdef: goslim_goa "GOA and proteome slim"
subsetdef: goslim_yeast "Yeast GO slim"
subsetdef: goslim_plant "Plant GO slim"
subsetdef: goslim_generic "Generic GO slim"
subsetdef: gosubset_prok "Prokaryotic GO subset"

[Term]
id: GO:0000001
name: mitochondrion inheritance
namespace: biological_process
def: "The distribution of mitochondria\, including the mitochondrial 
genome\, into daughter cells after mitosis or meiosis\, mediated by 
interactions between mitochondria and the cytoskeleton." [PMID:10873824, 
PMID:11389764, SGD:mcc]
is_a: GO:0048308 ! organelle inheritance
is_a: GO:0048311 ! mitochondrion distribution

\end{verbatim}
\caption{Tiny example of Gene Ontology data. (Some lines broken to
present data on this page.)}
\label{figure:dibbler-records}
%\end{center}
\end{small}
\end{figure*}

Now, we can examine how to use \pads{} to describe 
the {\em physical layout} and 
{\em semantic properties} of our ad hoc data source. 
The language provides a type-based model:
basic types describe atomic data such as integers, characters, 
strings, dates, urls, \etc, while
structured types describe compound data built from simpler pieces.
\suppressfloats

In a bit more detail,
the \pads{} library provides a collection of broadly useful base
types.  Examples include 8-bit unsigned integers (\cd{Puint8}), 32-bit
integers (\cd{Pint32}), dates (\cd{Pdate}), strings (\cd{Pstring}),
and IP addresses (\cd{Pip}).  Semantic conditions for such base types
include checking that the resulting number fits in the indicated
space, \ie, 16-bits for \cd{Pint16}.  By themselves, these base types
do not provide sufficient information to allow parsing because they do
not specify how the data is coded, \ie{}, in ASCII, EBCDIC, or binary.
To resolve this ambiguity, \pads{} uses the \textit{ambient} coding,
which the programmer can set.  By default, \pads{} uses ASCII.  
% To
% specify a particular coding, the description writer can select base
% types which indicate the coding to use.  Examples of such types
% include ASCII 32-bit integers (\cd{Pa_int32}), binary bytes
% (\cd{Pb_int8}), and EBCDIC characters (\cd{Pe_char}).  In addition to
% these types, users can define their own base types to specify more
% specialized forms of atomic data.

To describe more complex data, \pads{} provides a collection of
structured types loosely based on \C{}'s type structure.  In
particular, \pads{} has \kw{Pstruct}s, \kw{Punion}s, and \kw{Parray}s
to describe record-like structures, alternatives, and sequences,
respectively.  \kw{Penum}s describe a fixed collection of literals,
while \kw{Popt}s provide convenient syntax for optional data.  Each of
these types can have an associated predicate that indicates whether a
value calculated from the physical specification is indeed a legal
value for the type.  For example, a predicate might require that two
fields of a \kw{Pstruct} are related or that the elements of a
sequence are in increasing order.  Programmers can specify such
predicates using \pads{} expressions and functions, written using a
\C{}-like syntax.  Finally, \pads{} \kw{Ptypedef}s can be used to
define new types that add further constraints to existing types.

\pads{} types can be parameterized by values.  This mechanism serves
both to reduce the number of base types and to permit the format and
properties of later portions of the data to depend upon earlier
portions.  For example, the base type \cd{Puint_FW(:x:)} specifies
an unsigned integer physically represented by exactly \cd{x}
characters, where \cd{x} is a value that has been read earlier in the
parse.  The type \cd{Pstring(:COLON:)} describes a string
terminated by a colon (when \texttt{COLON} is defined to be \texttt{':'}).  
Parameters can be used with compound types like arrays and unions to
specify the size of an array or which branch of a union should be
taken.  This parameterization is what makes PADS a {\em dependently-typed}
language and substantially different from languages based on
context-free grammars or regular expressions.

\figref{figure:dibbler} gives an abbreviated \pads{} description 
for the Gene Ontology
data.  
We will use this example to illustrate some of the basic
features of the current \pads{} language.  
In \pads{} descriptions, types are declared before they are used, 
so the type that describes the totality of the data source appears 
at the bottom of the description.  In this case,
the type \texttt{OBO\_file} describes the the entirety of the
GO data source (the \texttt{Psource} type qualifier indicates
this fact explicitly).  \texttt{OBO\_file} is a \kw{Pstruct} type with
two fields, a \texttt{hdr} field, with type  \texttt{OBO\_header}
and a \texttt{stanzas} field with type \texttt{OBO\_stanza[]},
an array of \texttt{OBO\_stanza}s of arbitrary length.
In general, \kw{Pstruct}s describe fixed sequences of data with 
unrelated types.  A little further up the description, there is
a slightly more complex \kw{Pstruct}:  \texttt{OBO\_stanza\_tag}.
This struct contains two kinds of fields, named fields, like
before, and unnamed fields \texttt{'['} and \texttt{']'}.
These unnamed fields match literal characters in the data, in this
case square brackets.  In 
\texttt{OBO\_tag\_value\_pair}, another unnamed field involving 
a regular expression (introduced by \texttt{Pre}) matches a colon
followed by any number of spaces.  Named fields are included in
the internal representation of the data, while unnamed fields are
not.  

Both \texttt{OBO\_tag\_value\_pair} and
\texttt{OBO\_stanza\_tag} mentioned above are preceded by the
\texttt{Precord} type qualifier.  
The notion of a record varies depending upon the data encoding.  
ASCII data typically uses new-line characters to delimit 
records, binary sources tend to have fixed-width records, while 
COBOL sources usually store the length of each record before the actual data.
\pads{} supports each of these encodings of records and allows users to define
their own encodings.  By default, \pads{} assumes records are new-line terminated.  The constant \texttt{Peor} is also set to this
\texttt{e}nd-\texttt{o}f-\texttt{r}ecord marker, and may be used anywhere in a
description, as we did in the
type \texttt{OBO\_tag\_value\_pair}.
\texttt{Precords} have error-recovery semantics -- if errors 
in the data cause the parser to become seriously confused,
it will attempt to recover to a record boundary.
In practice, we have found this to be a very robust recovery mechanism
for ad hoc data.  

The \texttt{OBO\_stanza} structure 
illustrates the use of some simple semantic constraints.
For instance, following the \texttt{id} field is the constraint
\texttt{hastag(id,"id")}, which is defined earlier in the file
as a function in defined in C.  In general, such constraints should
be pure (have no externally visible effects), but otherwise may be
arbitrary C expressions.  Notice also that the first argument to the constraint
is \texttt{id}, the name of the current field.  More generally,
constraints may refer to any type parameters in scope and any previous field.
The type parameters in particular, allow information concerning data
just parsed to be passed arbitrarily far forward in a description.
The second semantic constraint in \texttt{OBO\_stanza} is
\texttt{Pterm(Peor)}, which is a built-in constraint for 
controlling array termination.  In this case, it states that
the array terminates when it sees an end-of-record marker 
\texttt{Peor} following any array element.  
In general, \pads{} provides a rich collection
of array-termination conditions: reaching a maximum size, finding 
a terminating literal 
or satisfying a user-supplied predicate over the already-parsed portion of 
the \kw{Parray}.  \pads{} also has convenient syntax for 
specifying separator symbols that
appear between elements of an array and declaring inter-element
constraints including sorting.

One last feature of note in this description is the \texttt{Penum}
\texttt{OBO\_stanza\_type}.  \texttt{Penum}s specify that
the data must be one of several fixed strings, in this case, 
either \texttt{Term} or \texttt{Typeref}.  \pads{} also admits
\texttt{Punion}s of several forms for more general sorts of alternatives
in data formats.

Other examples of \pads{} descriptions and an online demo may be
found at \url{www.padsproj.org}.

\begin{figure}[t]

%\begin{code}
/* www.geneontology.org OBO flat file format. */
\mbox{}
#define COLON ':'
\mbox{}
Precord Pstruct OBO\_tag\_value\_pair\{
  Pstring(:COLON:) tag;  
  Pre "/: */"; 
  Pstring\_ME(:Peor:) val;
\};
\mbox{}
int hasTag(OBO\_tag\_value\_pair p,char *tag)\{
  return Pstring\_eq\_cstr(&p.tag,tag); 
\};
\mbox{}
Precord Pstruct OBO\_header\{
  ...
\};
\mbox{}
Penum OBO\_stanza\_type\{ Term, Typeref \};
\mbox{}
Precord Pstruct OBO\_stanza\_tag \{
    '['; OBO\_stanza\_type s\_type; ']';
\};
\mbox{}
Precord Pstruct OBO\_stanza\{
  OBO\_stanza\_tag       tag;
  OBO\_tag\_value\_pair   id      : hasTag(id,"id");
  OBO\_tag\_value\_pair[] tvpairs : Pterm(Peor);
\};
\mbox{}
Psource Pstruct OBO\_file\{
  OBO\_header hdr;
  OBO\_stanza[] stanzas;
\};
\end{code}


Insert weblog description here.

\caption{Abbreviated \pads{} description for Web Log data.}
\label{figure:dibbler}
\end{figure}


% Returning to the CLF description in \figref{figure:wsl}, the \kw{Penum} type \cd{method_t} describes
% a collection of data literals.  During parsing, \pads{} interprets these
% constants using the ambient character encoding.  The \kw{Ptypedef} 
% \cd{response_t} describes possible server response codes in CLF data by adding
% the constraint that the three-digit integer must be between 100 and 600.

% \setlength{\floatsep}{0pt}
% \setlength{\dblfloatsep}{0pt}
% \setcounter{totalnumber}{4}
% \setcounter{dbltopnumber}{2}
% \begin{figure*}[t!]
% \begin{small}
% \begin{center}
% \begin{code}
% \chapter{Common log format example}
\section{pads description}
\section{c header file}
% \end{code}
% \vskip -2ex
% \caption{Tiny example of web server log data.}
% \label{figure:clf-records}
% \end{center}
% \end{small}
% \end{figure*}


% \begin{figure}
% \begin{code}
\kw{Punion} client\_t \{
  Pip       ip;      /- 135.207.23.32
  Phostname host;    /- www.research.att.com
\};
\mbox{}
\kw{Punion} auth\_id\_t \{
  Pchar unauthorized : unauthorized == '-';
  Pstring(:' ':) id;
\};
\mbox{}
\kw{Pstruct} version\_t \{
  "HTTP/";
  Puint8 major; '.';
  Puint8 minor;
\};
\mbox{}
\kw{Penum} method\_t \{
    GET,    PUT,  POST,  HEAD,
    DELETE, LINK, UNLINK
\};
\mbox{}
bool chkVersion(version\_t v, method\_t m) \{
  \kw{if} ((v.major == 1) && (v.minor == 1)) \kw{return} true;
  \kw{if} ((m == LINK) || (m == UNLINK)) \kw{return} false;
  \kw{return} true;
\};
\mbox{}
\kw{Pstruct} request\_t \{
  '\\"';   method\_t       meth;
  ' ';    Pstring(:' ':) req\_uri;
  ' ';    version\_t      version :
                  chkVersion(version, meth);
  '\\"';
\};
\mbox{}
\kw{Ptypedef} Puint16\_FW(:3:) response\_t :
         response\_t x => \{ 100 <= x && x < 600\};
\mbox{}
\kw{Precord} \kw{Pstruct} entry\_t \{
         client\_t       client;
   ' ';  auth\_id\_t      remoteID;
   ' ';  auth\_id\_t      auth;
   " ["; Pdate(:']':)   date;
   "] "; request\_t      request;
   ' ';  response\_t     response;
   ' ';  Puint32        length;
\};
\mbox{}
\kw{Psource} \kw{Parray} clt\_t \{
  entry\_t [];
\}
\end{code}

% \caption{\pads{} description for web server log data.}
% \label{figure:wsl}
% \end{figure}

\subsection{Overview of Planned Research}
\label{ssec:sow}

Our central research agenda is divided into three broad subsections,
which we will discuss here; our broader impacts will be discussed in the next 
section.  First, our experience with real-world data has revealed that
the current \pads{}\ system is unable to process certain kinds of
data sources, several of which are quite prevalent.  We 
propose a number of extensions to \pads{} 
that will greatly improve its expressive power and enable
us to handle many more important ad hoc formats.  Second,
while \pads{}\ can automatically generate tools for providing
statistical data summaries, querying ad hoc data and generating
\xml, the tool generation process is brittle and inflexible.  We 
propose to redesign
the automatic tool generation process using a principled methodology
based upon type-directed programming paradigms and
a novel \pads{}-attribute system that can communicate
semantic information from descriptions to tools.  
Third, while we have already begun a
semantic analysis of \pads{} that has turned to be extremely useful in 
practice, we only analyzed a small portion of the PADS system.  
We propose to formalize further elements language, particularly
output processors (printers) and prove critical \pads{} correctness
properties including the coherence of parsing and printing.  
Overall, our research combines novel language
design, high-performance systems engineering and theoretical analysis,
all aimed at solving crucial data processing problems.

\subsubsection{Towards a Universal Data Description System}

In this
section, we discuss a number of the limitations of the current \pads{}\ 
system and suggest extensions that we believe will be able to rectify
these deficiencies.  

\paragraph*{Data Transformation, Abstraction and Error-correction}
Many data sources require simple transformations of various kinds
immediately before or after parsing.  On output from a system,
inverse transformations are applied after or before printing.  
Examples of pre-parsing routines include decryption for
security-sensitive data and decompression for high-volume scientific data.  
The natural inverses for printing routines are encryption and compression.
After parsing, many data sources require, or at least benefit from,
a variety of simple data transformations.  For example,
sometimes ad hoc data will have multiple different representations
of the same concept -- dates in different formats, several different strings
to represent ``no value'' etc.  Simple transforms can be used to convert
these representations into a canonical form that facilitates down-stream
processing.  As another example, many data sources have privacy-sensitive parts
that should be ``sanitized'' in some way or another, possibly by scrubbing or
filtering data fields before passing them to down-stream applications.
Ironically, in October 2005, we asked the Princeton Computer Science
Department Technical Staff for access to web logs for some experiments
with \pads, but they refused until they had written scripts to
sanitize the data for us.  Medical data is another example of highly
privacy-sensitive data.
Lastly, almost all ad hoc data may contain errors, but sometimes there will be simple data-specific heuristics such as substitution of default values
that can be used to fix the most common problems.

It is currently impossible to code multi-stage processing and transformation
directly in PADS.  One solution
to this problem might be to let auxiliary passes through the data
remain outside the PADS system as prepasses or postpasses.  
Unfortunately, this solution is completely unsatisfactory for a
number of reasons.  First, doing so
will often leave us in a situation in which the \pads{} description 
is not a self-contained
definition of the data format in question.  Consequently some of the value
of \pads{} as documentation is lost.  
Second, programmers must do
more work themselves to produce PADS applications.  They cannot
simply run our automatic generators and receive a well-packaged 
query engine or statistical analyzer for the raw data.
Moreover, when coding transformation directly in C, the host language
for \pads{}, they must program at a much lower level of abstraction
than we might provide by supplying specialized domain-specific
transformers directly in \pads.
Since our goal is to maximize the productivity of scientists 
who use ad hoc data, ease-of-use is a key constraint.
Finally, some data formats and
tasks are not well-suited for implementation as a separate pre- or post-pass.
For example, some formats use non-uniform compression or encryption 
schemes~\cite{korn+:delta,korn+:data-format}.
Moreover, tasks such as data sanitization and error correction may be
data-dependent and directed by the structure of the data.  In other words,
they may involve just the sort of data analysis that PADS was built for
and should be integrated directly into the specification mechanism.

To address these difficulties,
we plan to research mechanisms that facilitate multi-stage data processing 
directly in PADS.  Since any PADS description must be able to generate 
both data
input tools {\em and} data output tools, we currently believe that each
data processing stage, or layer, should be specified as a pair of
transformations.  For instance, if data is decompressed on the way in,
it must be compressed on the way out.  If a field of the data is ``sanitized''
or filtered on the way in, perhaps a default value of the correct form must
be written back out to preserve the syntactic structure of the data format.

To achieve this functionality, we will begin by considering a
transformation specification with the general form 
\texttt{Ptransform \{ i,o :  Tcon <-> Tabs \}}.
Here, {\tt Tcon} is the PADS type of the external or {\tt con}crete
representation and {\tt Tabs} is the PADS type of the internal or 
{\tt abs}tract representation.  The internal representation of the current
phase may in turn serve as the external representation for the next
phase of the transformation.  The functions {\tt i} and {\tt o} are
user-defined functions that transform data from {\tt Tcon} to 
{\tt Tabs} and vice versa.  For example, {\tt i} may implement
decompression and {\tt o} may implement compression.

We intend to add these transformations as first-class
descriptions/types to the system.  In other words, these
transformation may be nested inside or otherwise composed with any
other form of PADS description.  When so nested, the transformation
will only apply to the appropriate specific subcomponent of the
format.  Therefore, transformations will be useful for simple
subcomponent error correction or representation casting as well as
full data source transformations such as decompression.  In addition,
transformations with compatible types will be composable.  For
example, if a data format is compressed and potentially contains
errors, a decompression transform may be composed with an
error-correction transform.

The meat of certain transformations such as compression and encryption
are probably best written as ordinary program code that is
subsequently included in the PADS description.  However, for smaller
scale, local transformations, we will investigate adding domain-specific
programming support.  This support would allow programmers to write
transforms quickly at an easy-to-understand and high level
abstraction.  It would also ensure that the output transformations are
proper inverses of the inputs.  In particular, inspired by recent work
by Foster et al.~\cite{foster+:lens}, we will explore how to develop a
library of {\em combinators}, simple composable functions, that can be
combined in a myriad of ways to produce the transforms {\em and their
inverses} at the same time.  Foster et al. used such bi-directional transforms to
solve a synchronization problem on error-free tree-shaped data. 
While we will exploit some of their
ideas, our application and
context are different: we are parsing, transforming and printing ad
hoc data.  As we do so, we are specifically interested in uncovering,
representing and handling error-filled data.  One of the critical
challenges for us will be to design the combinators so that they deal
with \pads{} parse descriptors correctly
and conveniently.


\paragraph*{Multi-source Data Processing}
Often, a single logical data source is represented
as several distinct, concrete repositories.   This is the case
in the GO data source we have examined, where data is split into four disjoint
files: a molecular function file, a biological process file,
a cellular component file and a term definitions file.
However, the size and number of repositories that make up a single source
may vary widely.  At the other end of the spectrum are systems involving
continuous monitoring of widescale phenomena that automatically
produce new data at phenominal rates.  For example, in order to monitor
the health of PlanetLab, a distributed network of several hundred machines
spread across the world, the CoMon monitoring system~\cite{comon} attempts to 
drag a small data file from each of the machines to a centralized repository once every
five minutes.  In the
centralized repository, the data is split into two new formats, each designed to accumulate
information about a certain aspect of the system.  In all, CoMon produces
approximately 22,000 files/day.  Unfortunately, the current \pads{} 
implementation is limited to processing a single data source.
We propose to extend our specification language to enable 
automatic generation of tools
that process multiple data sources, either on one local machine or distributed across a wide area network.  Doing so will require investigation of the 
following features.

\begin{enumerate}
\item {\bf Multi-source Specifications}  
We propose to extend the PADS specification language with a new first-class
type constructor, \texttt{Pget\{T;e\}}.  Here \texttt{e} specifies
a protocol for locating and accessing a data source, and \texttt{T} is a PADS type specifying the
structure of its contents.  The specification language for protocols should be completely flexible
so it can handle the largest possible collection of applications.  For instance,
CoMon accesses data via telnet and authentication; other data is available over the web
through FTP and HTTP.  And of course, the most common application will be to assemble
multiple sources through the local file system.  Though the construct \texttt{Pget \{T;e\}},
is simple (which, incidentally, is an explicit design choice)
we believe that when it interacts with the rest of the PADS specification language
it will provide powerful access possibilities.  In particular, due to the dependency 
already present in PADS, it will possible to begin reading the local file that contains information
concerning where and how to acquire further data.  Then that new information may be later used
in a \texttt{Pget} directive to locate distributed data.  For instance, to implement
a CoMon-like monitoring service, a local machine might contain a list of all currently active
PlanetLab nodes.  A PADS specification might read that list and use the information processed to
acquire information from all machines on the list.  This way, machines may be added or removed
from the network and the PADS specification remains unchanged.

% \item {\bf Concurrent Execution}  
% In order to support access to many sources, possibly distributed across a wide area, it will be necessary
% to process multiple repositories concurrently.  As data from each new source is requested
% through the \texttt{Pget} command, we plan on launching a new thread to read the source.
% However, the dependency in the language may require synchronization between threads.
% More research is required to understand the optimal implication strategy.
% However, the fact that PADS is a pure, declarative language should simplify
% dependency analysis greatly and improve opportunities to hide latency through concurrency.

\item {\bf Temporal Specifications}
In order to support applications that monitor ongoing phenomena, we plan to investigate how to add
temporal specifications to PADS.  These temporal specifications might indicate periodic arrival of
new data (once every five minutes), arrival of data at some specific time in the future (next Tuesday)
or upon receiving an explicit signal from some external source.  As is the case with our 
spatial specifications \texttt{Pget}, we will strive for a design that combines simplicity with
generality.  It is too early in our research to know exactly what form these temporal specifications 
will take.  However, we plan to pursue the design space diligently.

% BS alert!!

\item {\bf Modular Specifications}
As specifications begin to get larger and encompass multiple different data repositories,
possibly with different data formats, standard software engineering practice suggests
introduction of features to enable modular development of specifications and to allow multiple
programmers to collaborate on specifications.  The first necessary addition to PADS along these 
lines is a namespace mechanism to allow programmers to control their type names in a disciplined fashion.
A second extension we will explore is introduction of more general interfaces that allow programmers
to make type definitions abstract.  This feature should make it easier to evolve large specifications as
formats change.

\end{enumerate}

\subsubsection{Automatic Tool Generation}

The current \pads{}\ system can automatically generate tools for
creating statistical summaries of data sources, querying
data sources and outputting data in standard formats such as
\xml.  Unfortunately, it is not easy to extend the system with new generic tools,
nor does the current tool generation process allow all tools to be generated for all formats.
In this section, we propose several innovative ways of improving our
automatic tool generation infrastructure.

\paragraph*{Generic Interface Generation}
Currently, in order to generate any tool, the \pads{} compiler will 
produce a new version of a parser (or printer) that reads (or writes)
data in the format in question.  As control traverses the ad hoc data, within the parser (printer) code,
parts of the data are passed to explicit processing functions that 
convert to XML, gather statistical data, generate histogram buckets, etc.
As it stands, there is different traversal code for each different tool.
In order to add a new tool to the PADS tool suite, it is necessary to modify the compiler
to generate a new sort of traversal.  Hence, the barrier to adding new tools
is very high.  Unfortunately, this prevents independent researchers from adding more and more
useful tools to the suite of tools that apply to any PADS description.

We plan to study software architectures that will allow the compiler to generate
flexible and highly efficient, but tool-independent traversals for a particular data format.  
These tool-independent traversals will match a generic tool interface.  If designed correctly, 
tool implementers will be able to write separate tool-specific code and then link that code
against a given tool-independent traversal.  Our revised design will take advantage of
recent research in the functional programming languages community on type-directed 
programming~\cite{crary+:intensional-types,LPJ03,syb2,SYB3}
as we have observed that each of our tools is actually an example 
of a type-directed program -- in other words, the structure of the PADS type directs run-time 
execution of the tool.

\paragraph*{Attribute-based Tool Generation}
One important reason that automatic tool generation is brittle
is that many of the tools must assume that the data passed to them fits a generic
shape: it must be an optional {\em header} followed by
a {\em body} that consists of a sequence of records 
terminated by the end of source.  
When the data does not fit the assumed shape, automatic
tool generation cannot be used.  Moreover, in order to specify 
the header and body sections, a programmer must write a complicated series of
C macros that specialize the tool.  This process is obviously highly 
error-prone and extremely inflexible.  We need a more flexible 
tool generation paradigm that allows data analysts to manipulate
any data format they might come across, and also to
analyze any subparts of a data set they deem important.

We propose to develop a more principled and substantially more robust
approach to automatic tool generation by extending \pads{}\ with a
high-level attribute system.  {\em Attributes} are tokens that may be
attached to \pads{}\ specifications and that communicate semantic
information to tools such as the statistical analyzer and the XML generator.  
For example, if a data analyst wanted to use the statistical
summary tool, he or she might attach the {\cd{summary}} attribute to
the specific parts of a PADS description that describe data that must
be summarized.  Likewise, to generate \xml{}\ from a portion of the
data, the analyst might attach the {\cd{xml}} attribute to the
appropriate component of the \pads{} description.  In addition to
enabling more robust tool generation, attributes will make our
generated tools much more flexible.  Analysts will be able to use
attributes to select portions of the data that they wish to analyze or
manipulate.

In order to make this new attribute-based tool generation system work,
we must do considerable research in language design and we must
re-architect some of the current tool base.  The language design is
nontrivial because attribute specifications must be separated from the
basic data format description.  The attributes need to the separated
because they change frequently and depend upon the current needs of
the data analyst, whereas the basic data format description is
independent of any given tool or any given analyst, and ideally
persists indefinitely.

% \paragraph*{Exploiting Semantic Constraints}
% In addition to being brittle, the current \pads{}\ tools suffer from
% performance penalties because there is no automatic way to communicate
% semantic information from the data source to the tool.  The query
% engine tool, in particular, could benefit tremendously from semantic
% information such as the property that a given field acts as a {\em
% key} for a certain record (\ie\  each record of this type in the data
% source has a unique key field) or the property that an array of
% records is {\em sorted}.

% We plan to extend the basic attribute system described above with
% user-defined attributes that are associated with semantic constraints
% such as the \cd{key} attribute or the \cd{sorted} attribute.  When we
% generate tools from a \pads{} description and associated attributes,
% the generated parser will check the semantic constraints as it reads
% new data.  On the other hand, a downstream tool such as the query
% engine will be able to assume that the semantic constraints hold
% (provided the parse descriptor indicates the data is error-free) and
% it will be able to exploit this knowledge to optimize its query plan.
% Overall, we believe attributes with semantic constraints will provide
% an efficient and robust means of communicating information between
% tools.  However, once again, we must do much more research to develop
% the right language design and effective tool interfaces.


%\label{subsec:general}

% We start by showing in \figref{figure:dibbler-filter} a simple use of 
% the core library to clean and normalize \dibbler{} data. After initializing
% the \pads{} library handle and opening the data source, the code sets
% the mask to check all conditions in the \dibbler{} description except the
% sorting of the timestamps.  We have omitted from the figure the code to read and write the header. 
% The code then echoes error records to one file and cleaned ones to another.
% The raw data has two different representations of unavailable phone numbers:
% simply omitting the number altogether, which corresponds to the \cd{NONE}
% branch of the \kw{Popt}, or having the value \cd{0} in the data.  
% The function \cd{cnvPhoneNumbers} unifies these two representations 
% by converting the zeroes into \cd{NONE}s.  The function \cd{entry_t_verify}
% ensures that our computation hasn't broken any of the semantic properties
% of the in-memory representation of the data.
% \begin{figure}[t]
% \begin{small}
% \begin{center}
% \begin{centercode}
P\_t                  *p;
entry\_t              entry;
entry\_t\_pd           pd;
entry\_t\_m            mask;
    ...
P\_open(&p, 0, 0);
P\_io\_fopen(p, "sirius/data/2004.11.11");
    ...
entry\_t\_m\_init(p, &mask, P\_CheckAndSet);
mask.events.compoundLevel = P\_Set;
    ...
\kw{while} (!P\_io\_at\_eof(p)) \{
  entry\_t\_read(p, &mask, &pd, &entry);
  \kw{if} (pd.nerr > 0) \{
    entry\_t\_write2io(p, ERR\_FILE, &pd, &entry);
  \} \kw{else} \{
    cnvPhoneNumbers(&entry);
    \kw{if} (entry\_t\_verify(&entry)) \{
      entry\_t\_write2io(p, CLEAN\_FILE, &pd, &entry);
    \} \kw{else} \{
      error(2, "Data transform failed.");
    \}
  \}
\}
\end{centercode}

% \caption{Code fragment to filter and normalize \dibbler{} data.}
% \label{figure:dibbler-filter}
% \end{center}
% \end{small}
% \end{figure}

%\paragraph*{Transforming Data from One Ad Hoc Format to Another}

\subsubsection{Formalization and Semantic Analysis}

\begin{quote}
The languages people use to communicate with computers
differ in their intended aptitudes, towards either a particular
application area, or a particular phase of computer use
(high level programming, program assembly, job scheduling,
etc). They also differ in physical appearance, and more
important, in logical structure. The question arises, do the
idiosyncrasies reflect basic logical properties of the situations
that are being catered for? Or are they accidents of history
and personal background that may be obscuring fruitful
developments? This question is clearly important if we are
trying to predict or influence language evolution.

To answer it we must think in terms, not of languages,
but of families of languages. That is to say we must systematize
their design so that a new language is a point chosen
from a well-mapped space, rather than a laboriously devised
construction.

$\qquad$ --- P. J. Landin, {\em The Next 700 Programming Languages}, 
1966~\cite{landin:700}.
\end{quote}

Landin asserts that principled programming language design involves
thinking in terms of ``families of languages'' and choosing from a
``well-mapped space.''  However, when it comes to the domain of
processing ad hoc data, there has been too little work on mapping this
crucial space and understanding the family of languages that
inhabit it.  We have recently begun to tackle this problem
as Landin suggested, by developing a semantic
framework for defining, comparing, and contrasting languages
in our domain~\cite{fisher+:700}.  This semantic framework revolves around the
definition of a Data Description Calculus (DDC),
based on the guiding principles of dependent type theory.  
We show how to give a denotational semantics
to DDC by interpreting DDC
types as parsing functions that map external representations (bits)
to data structures in a typed lambda calculus.  More precisely,
these parsers produce both 
internal representations of the external data and
parse descriptors that pinpoint errors in the original source.
We also simultaneously interpret DDC types as the types for
representations of the internal data and as the types for parse descriptors.
We have already reaped a number of important, practical benefits from this calculus:
\begin{itemize}
\item We have shown how to map the features from several different data description languages including \pads, \packettypes{}~\cite{sigcomm00} and
\datascript{}~\cite{gpce02} into the calculus.  This allows us to give
precise comparisons between these languages without worrying about
superficial syntactic details.  It also helps reveal how the useful features
from one language may be assimilated by another.
\item We have used the calculus to help us devise design
principles for this class of languages.  In particular, we
have developed the principle of ``error correlation,'' that formalizes
the correspondence between parsed data and parse descriptors that mark
data errors.  We have proven DDC-produced parsers satisfy this principle
and a number of other important criteria.
\item Analysis of the \pads{} implementation with semantics in hand
has uncovered several bugs and ill-advised design decisions
that have been subsequently fixed.
\item Study of the formal type theory has lead to new features we have 
subsequently implemented including recursion, which requires a subtle 
``contractiveness'' constraint for correctness.  It has also lead us
to reconsider our design of several other features.
\end{itemize}

However, despite these advances, there is a great deal of work to do.

\begin{enumerate}
\item {\bf Formalization of output}.  The current DDC only formalizes the
input or parsing semantics of languages like \pads.  There is
a great need to study and formalize the inverse ``printing'' transformation 
from internal data to external format.  In fact, we have recently
uncovered practical problems with output processors that have pushed us in 
this direction. For example, \pads{} descriptions may use regular expressions
to match portions of data and, rather than include that data in the
internal representation, the data is thrown away (\figref{figure:dibbler}
includes an example of this feature).  On output, since the matched
data has been thrown away, what do we put in its place?  The 
current implementation is broken in this respect and must be fixed.
Similar problems occur when parsing erroneous data and in a couple of other
places.  To understand the full range of problems and propose
a principled solution, we require a sound, rigorous semantic study
of the output processors.

\item {\bf Coherence of input and output}  Once we have defined a semantics
for output, we can use that to find discrepancies between implementation
and idealization.  We can also once again
use it to compare and contrast different programming languages
in the family of ad hoc language processors.  Still, 
how will we judge that the semantics is indeed ``good''?
What general criteria and principles will we use in the design?  The
principles should be sufficiently general that as the family of languages
grows, the basic principles endure and provide guidance through that 
growth.  We require much research in this area, but one principal
we have identified at an intuitive level is that input and output
should be ``coherent'' in some sense.  One might suggest that 
if we parse then print data $D$, the result will be exactly $D$.
However, this principle precludes having the parser fix obvious errors
in the data, or standardize representations of dates, times or other
ad hoc values, for instance.  Hence, we are currently considering
defining coherence between input and output
by requiring that parsers and printers form
a {\em Galois connection}.  This definition would allow parsers and printers to
``improve'' data as they work.  
% The properties of Galois connections
% also tell us that parsing followed by printing followed by parsing is
% equal to parsing.  Hence, while parsers and printers may improve data,
% they reach a stable (or fixed) point after one iteration.

% \item {\bf Set-theoretic semantics} The semantics of classic parsing
% formalisms is normally given in terms of sets of strings.  For example,
% regular expressions, general context free grammars, 
% LL(k) and LR(k) grammars are all usually discussed in terms of
% the set of strings they recognize.  The semantics we have developed so far
% for out dependent type theory is different -- it is given in terms of 
% parsing functions from external data
% into internal data.  One reason we chose to develop our semantics differently
% is that, the dependency in our type theory, which is not present
% in these other formalisms appears to necessitate some degree of translation.
% However, we propose to explore techniques for providing a set-theoretic
% semantics to our language in addition.  Once again, the goal is to better 
% understand the relationship between our language and these traditional
% compiler techniques and view all such languages as a coherent family,
% in Landin's sense.  In doing so we hope to improve
% the effectiveness of data description languages for specific tasks
% such as ad hoc scientific data or, alternatively, program language parsing.

\item {\bf Semantics for tools}  \pads{} does not merely give programmers
a parser and a printer.  It also automatically
generates a collection of value-added tools for the particular format including 
an XML converter, a statistical analyzer, a histogram generator and
query engine.  The fact that \pads{} can generate all these different tools
from a single data specification is one of the major strengths of our 
approach.  So far, we had given semantics to our parsing library
and we proposed to give semantics to the printing library.  A more
challenging task still is to provide a {\em useful} semantics for 
these higher-level tools.  Doing so will require us to devise
a semantic framework in which types are interpreted as functions or
relations between ad hoc data and abstract models of that data.  For instance,
to understand our histogram generation tool, will have to relate
ad hoc data to histograms.  To give semantics to graphing functions, 
we will have to relate
ad hoc data to graphs.  To convert to XML, we will have to relate
ad hoc data to XML.  Ideally, we will be able to devise one 
reusable semantic framework that may be instantiated differently to
obtained semantics for these different tools.  

\item {\bf Semantics for new proposed features}
In this proposal, we have proposed to develop a number of new
features for the \pads{} implementation.  As we do so, we will formalize
each feature and engage in rigorous semantic analysis to find bugs
in our implementation, relate our design choices to those of others and 
guarantee our language maintains the crucial properties
that currently make it effective.
\end{enumerate}

\subsection{Broader Impacts}
\label{ssec:impact}

If funded, this project will have two major broad impacts.

\paragraph*{Supporting Research in the Natural Sciences}
In order to have a broad impact across research in the sciences,
we will develop descriptions and tools for researchers in
physics, biology and chemistry.
More specifically, we have already developed a partnership with
Olga Troyanskaya in Princeton's Lewis-Sigler Institute for 
Integrative Genomics.  Her Magic database system~\cite{magic}
reads several data sources
and applies a Baysian analysis in order to discover gene function.
The end goal of Magic is to provide critical information about how
our genes work that may be used in the treatment of widespread diseases,
including cancer.
In the past, Troyanskaya and her students have spent
substantial blocks of time building parsers to collect and integrate
this data.  This wastes her valuable time, which could otherwise be
used on the challenging and important algorithmic problems for which
she is a world expert.  It also limits the scope of Magic --- there are
only so many parsers one can program and maintain by hand.
Through the use of PADS, we will transform Magic into a truly generic
system by allowing Magic to import any data source.  To add new sources,
all a user will have to do is write a simple, high-level PADS description.
By adding data sources to the collection Magic
currently analyses, Magic will be able to make additional and possibly more
accurate predictions.

More generally, part of our mission will be
to provide support for the data processing needs of biologists,
chemists and physicists at Princeton University and 
the broader natural sciences community.  
Part of our contribution will be a series of PADS
descriptions for these formats and the analysis and querying tools we
can generate automatically from PADS.  A second important contribution
will be a visual interface built on top of PADS that allows scientists
to browse data represented in ad hoc data formats without having to
know anything about programming or parsing.  All of our software will
be freely available to academics via the Web.

\paragraph*{Undergraduate Research}
Princeton University has a very active undergraduate research program.
Princeton students usually do at least one semester of
Junior Independent Work and one semester of Senior Independent Work. 
Many students also do a year-long
Junior and Senior Theses.  We believe that PADS is an excellent
platform for undergraduate independent work and we plan to link
undergraduate independent work with our interdisciplinary effort to
produce tools for biologists.  We have already recruited Mark Daly,
a Princeton Senior who is doing his undergraduate senior thesis
on a PADS user interface and automatic format inference.
We hope encourage other undergraduates to help us build
specific data processing tools for the biological data formats
used in Princeton's Genomics Institute.
This will introduce students both to interesting and novel
computer science research and also allow them to build useful
tools for biological applications. The PI has a proven track record
when it comes to advancing graduate and undergraduate education
as he has organized two summer schools (2004, 2005) on technology for
secure and reliable programming and mentored several undergraduates,
the latest of which, Rob Simmons, won the Princeton Computer Science Department
Senior Thesis Award.

\subsection{Comparison with Other Research}
\label{ssec:related}

The oldest tools for describing data formats are parser generators such as
Lex and Yacc.  While excellent for parsing programming languages, Lex and Yacc
are too heavyweight for parsing the simpler ad hoc data formats one
runs into in the sciences.   
Unlike PADS, whose syntax is based on types from the well-known C language,
the syntax of Lex and Yacc is somewhat foreign.  Perhaps more importantly,
users must write a lexer, write a
grammar, and construct the in-memory representations by hand.  In
addition, they only work for ASCII data, they do not easily
accommodate data-dependent parsing, and they do not provide auxiliary
services such as automatic XML conversion, stastical analysis and
others.  Some more modern parser generators such as ANTLR~\cite{antlr} alleviate
a few of these problems, but they still do not automatically generate auxiliary tools
useful in processing ad hoc data nor do they provide good support for generating
rich, well-typed in-memory representations (ANTLR's in-memory representations
are very limited when compared with PADS and the extensions we propose).

There are parallels between PADS types and some of the elements of parser
combinator libraries found in languages like
Haskell~\cite{burge:parser-combinators,hutton+:parser-combinators}. 
However, as with most other general-purpose parsing tools, one cannot
simply put together a collection of Haskell's parser combinators and
automatically generate domain-specific programs such as 
an XML converter or a histogram generator, for instance.  

A somewhat different class of languages includes
\textsc{ASN.1}~\cite{asn} and \textsc{ASDL}~\cite{asdl}. 
Both of these systems specify the {\em logical\/} in-memory representation of data
and then automatically generate a {\em physical\/} on-disk representation.
Although useful for many purposes, this technology does not help
process data that arrives in predetermined, ad hoc formats.
Another language in this category is the Hierarchical Data Format 5 
(HDF5)~\cite{hdf5}.  This file format allows users to store scientific
data, but it does not help users deal with legacy ad hoc formats like PADS does.

The networking community has developed a number of domain-specific
languages~\cite{sigcomm00,bfd,gpce02} for parsing and printing binary data, particularly packets
from common networking protocols such as \textsc{TCP/IP} and also
\java{} jar-files.  Like \pads{}, these languages have a type-directed
approach to describing ad hoc data and permit the user to define
semantic constraints.  In contrast to our work, these systems handle
only binary data and assume the data is error-free or halt parsing if
an error is detected.  Parsing non-binary data poses additional
challenges because of the need to handle delimiter values and to
express richer termination conditions on sequences of data. These
systems also focus exclusively on the parsing/printing problem,
whereas we have leveraged the declarative nature of our data
descriptions to build additional useful tools.  

Currently, the Global Grid Forum is working on a standard
data-format description language for describing ad hoc data formats,
called DFDL~\cite{dfdl-proposal,dfdl-primer}.  Like \pads{},
DFDL{} has a rich collection of base types and supports a variety of
ambient codings.  Unlike \pads{}, DFDL{} does not support semantic
constraints on types nor dependent types, \eg{}, it is not possible to
specify that the length of an array is determined by some previously parsed field in the
data.  Our practical experience indicates that many ad hoc formats,
particularly binary formats, absolutely require dependent types in their
specifications.  DFDL{} is an annotated subset of XML{} Schema, which means
that the XML{} view of the ad hoc data is implicit in a DFDL{}
description.  DFDL{} is still being specified, so no DFDL-aware
parsers or data analyzers exist yet.  

% There are probably hundreds of tools that one might use if their data were
% in \xml.  However, the point of PADS is to allow scientists whose data is {\em not}
% already in \xml to get work done, particularly when that data contains errors,
% as ad hoc data often does.  Since many processes, machines, programs and other devices
% currently output data and a whole most of

XSugar~\cite{brabrand+:xsugar2005} allows user to specify an alternative non-XML
syntax for XML languages using a context-free grammar.  This tool
automatically generates conversion between XML and non-XML 
syntax. It also guarantees that
conversion will be invertable.  However, it does not use theory of Galois
connections, but instead introduces a notion of ``ignorable'' grammar
components (inferred based on properties of grammar) and proves
bijection modulo these ignorable elements.  More importantly,
since the basis of interconversion is a context free grammar, unlike PADS,
formats that required dependency may not be expressed.

XDTM~\cite{zhao+:sigmod05,xdtm} uses XML Schema to describe the locations of a collection
of sources spread across a local file system or distributed
across a network of computers.
However, XDTM has no means of specifying the contents of files,
so XDTM and PADS solve complementary problems.  Nevertheless, the XDTM design 
may provide ideas to us as we extend PADS from a single-source system to a
multi-source system. The METS schema~\cite{mets} is similar to XDTM as it describes 
metadata for objects in a digital library,
including a hierarchy such objects. 

Commercial database products provide support for
parsing data in external formats so the data can be imported into
their database systems, but they typically support a limited number of
formats, \eg{}, COBOL copybooks.  Also, no declarative description of the
original format is exposed to the user for their own use, and they
have fixed methods for coping with erroneous data.  For these reasons,
PADS is complementary to database systems.  We strongly believe that
in the future, commercial database systems could and should support a 
PADS-like description language that allows users to import information from
almost any format.  We hope that our research on PADS will make a broad
impact in this area.

On the theoretical front, the scientific community's understanding of type-based languages for data description
is much less mature.  To the best of our knowledge, our work on
the DDC is the first to provide a formal interpretation of dependent 
types as parsers and to study the properties of these parsers including error correctness and
type safety.  Regular expressions and context-free grammars, the basis for Lex and Yacc
have been well-studied, but they do not have dependency, a key feature necessary for expressing
constraints and parsing ad hoc scientific data.
{\em Parsing Expression Grammars} (PEGs),
studied in the early seventies~\cite{birman+:parsing}, revitalized more 
recently by Ford~\cite{ford:pegs} and implemented using
``packrat parsing'' techniques~\cite{ford:packrat,grimm:packrat}, 
are somewhat more similar to PADS recursive descent parsers. However, PADS does
not use packrat parsing techniques as the space overhead is too high for
large scientific data sets.  Moreover, our multiple interpretations of types in the DDC
makes our theory substantially different from the theory of PEGs.

%\subsection{Old Related}
%\section{Related Work}
\label{sec:related}

\edcom{Small summary followed by reference to thesis.}



\subsection{Results from Prior NSF Support}
\label{ssec:results}

\paragraph*{David Walker, PI} (NSF CCR-0238328 CAREER: Programming Languages for Secure and Reliable Component Software
Systems, PI)
The goal of Walker's career award is to develop new programming language
technology that will improve the security and reliability of component software systems.
More specifically, Walker and his students have begun to develop new techniques for defining,
implementing and reasoning about run-time enforcement of program properties.
For example, he developed a rich new theory of security monitors as formal
automata that transform untrusted program behavior as they 
execute~\cite{ligatti+:edit-automata}.
This theory allows security
architects to model a variety of different sorts of run-time
enforcement mechanisms, to prove that certain mechanisms can or cannot
enforce various security properties, and to compare the power of
different classes of security monitors.    Recently, Walker 
has used the theory to prove the surprising new result that powerful run-time
program monitors can enforce certain kinds of liveness properties~\cite{ligatti+:renewal}
and build an implementation for Java~\cite{bauer+:polymer}.

Walker's security monitoring language is a form of aspect-oriented programming language.
In order to better understand aspect-oriented technologies and their
potential impact on security, Walker formalized and proved
safe the {\em first} higher-order, strongly-typed calculus of 
aspects~\cite{walker+:aspects}, built an implementation and extended it with facilities for polymorphic
and type-directed programming~\cite{dantas+:polyaml}.  This calculus defines
both static typing rules and the execution behavior of aspect-oriented
programs.  Consequently, it may
serve as a starting point for analysis of deeper properties of programs.
Recently, he has used the calculus to study the design of a
program analysis that determines the effect of security monitors on
the code they monitor~\cite{dantas+:harmless-advice,dantas+:harmless-popl}.   The analysis
can guarantee that a security monitor does not interfere with the 
original application, which greatly increases a user's incentive
to apply security patches.

To complement his work on run-time monitoring programs, Walker has also
developed several type systems to ensure basic type and memory safety conditions
for low-level programs.  Basic type- and memory-safety guarantees provide a foundation on which
richer security mechanisms can be implemented.  More specifically, he
has extended his earlier work on typed assembly language (TAL)~\cite{morrisett+:tal,morrisett+:journal-stal} with
logic-based type systems that can detect memory errors involving
stack-allocated data~\cite{ahmed+:stack,jia+:stack} and heap or region-allocated
data~\cite{ahmed+:hierarchical-storage}.  In addition to studying memory safety
properties, Walker has shown how to use related type-theoretic and logical techniques
to verify programs~\cite{jia+:ilc} and enforce general software
protocols~\cite{mandelbaum+:refinements}.  

Finally, Walker's career grant allowed him to be a leader in
programming languages and security education. In 2004 and 2005 he organized
a 10-day summer school on software security and reliable computing,
attended by over 100 participants combined~\cite{summerschool04,summerschool05}.  He has also written a
chapter of a new textbook on type systems~\cite{walker:attapl}.  Also in 2005, his
undergraduate research advisee, Rob Simmons, won the Princeton Computer Science Department
Senior Thesis Award.



\newpage
%%%%%%%%%%%%%%%%%%%%%%%%%%%%%%%%%%%%%%%%%%%%%%%%%%%%%%%%%%%%%%%%%%%%%%%%%%%%
\section{References Cited}

{\bibliographystyle{abbrv}
\bibliography{pads-long,pads,galax,padsdave}
}
%{\bibliographystyle{abbrv}
% \small\bibliography{pads}
%} 

\end{document}


