\section{Introduction}
\label{sec:intro}

System implementers and adminstrators manipulate a wide
variety of system logs on a daily basis.
Common tasks include data mining, querying, performing statistical analysis,
detecting errors, and transforming the data to standard formats.
Because many of these logs are in non-standard formats,
there are often no ready-made tools to help process these logs.
As a result, system engineers have to resort to writing one-off
parsers, typically in Perl or Python, to ingest these data sources,
a tedious, error-prone and costly process. 

To facilitate working with such log files, we developed \pads
\cite{fisher+:pads,fisher+:popl06,padsweb}, a high-level declarative
specification language for describing the physical formats of ad hoc
data. A \pads{} description for a data source precisely documents the
format of the data, and the \pads{} system compiles such descriptions
into a suite of useful processing tools including an XML-translator, a
query engine, a statistical analyzer, and programmatic libraries and
interfaces. Analysts can then either use these generated tools to
manage the logs or write custom tools using the generated libraries.

A significant impediment to using \pads{} is the time and 
expertise required to write a \pads{} description for a new data
source.  If such a source is well-documented, writing a \pads{}
description is straightforward and requires time proportional to the
existing documentation.  Often, however, such data sources are not
well documented and the user must adopt an iterative process to produce
a description: write a partial description, use this description to
parse the data, flag segments of the data that do not match the
description, refine the description to cover these cases, and repeat.
This process is time-consuming, often requiring days for complex
formats. 

As a first step towards addressing this problem, we developed the
\learnpads{} system \cite{Fisher+:dirttoshovels,xi08:tokenization},
which automatically infers a \pads{} description from sample data, and
thus eliminates the need for hand-written descriptions. The
\learnpads{} system successfully produces correct descriptions for a
range of small data sources, but it cannot handle larger ones, because
the system includes a memory-intensive algorithm designed to process
the entire data source at once.

In this paper, we take the next step towards automatically inferring
descriptions of system log files by adapting \learnpads{} to work
\textit{incrementally}.  With this modification, the system takes as input
an initial description and a new batch of data. It returns a modified
description that extends the initial description and covers the new
data as well. The initial description
can be supplied by the user or the system can use the original
\learnpads{} system to infer it.  This iterative architecture also
allows the user to take the output of the system, make revisions such
as replacing generated field names like \cd{IP_1} with more meaningful
names like \cd{src}, and then use the refined description as the
basis for the next round of automatic revision.

In the rest of the paper, we give a brief overview of \pads{} and the
original inference system (\secref{sec:review}).  We then describe
the incremental inference algorithm (\secref{sec:algo}), discuss its
implementation and give some experimental results (\secref{sec:imp}),
and finally conclude (\secref{sec:conclude}).

% - system implementers and admins need tools to help them manage, query, extract info from,
%   detect errors in, and program with a variety of system logs. 
%
% - PADS: high level declarative specifications of physical formats
%
% - generates processing tools and programatic libraries and interfaces
%
% - automatically, incrementally learn PADS specs from sample data
