\section{Related Work}
\label{sec:related}

There is a long history of research in {\em grammar induction},
the process of discovering grammars from example data.  
Vidal~\cite{vidal:gisurvey} and
De La Higuera~\cite{higuera01current} both give surveys
of research in the area.  One way preliminary way research in this area
may be subdivided is by analyzing what kinds of inputs
the various algorithms require.  Some algorithms
require positive and negative examples to discover a grammar, some
algorithms require manual labelling of example data, some algorithms
require answers to various kinds of queries.  Our system only
requires positive examples (negative examples of ad hoc data sources 
are not available in practice), does not ask users to answer queries,
and does not require labelling (files are too simply too
big and change form at arbitrarily deep points within).  

Given that we work only with positive examples, there
are still a number of possible approaches we could take.
Our work on the core learning algorithm
borrows key ideas from two different places.
First, the structure discovery phase of our algorithm, which
generates an initial candidate grammar from a set of examples,
is a variant of Arasu and Garcia-Molina's algorithm 
for inferring the structure of web pages for the purpose
of information extraction~\cite{arasu+:sigmod03}.
Second, our system makes use of various grammar
rewriting algorithms that seek to optimize an information-theoretic
Minimum Description Length (MDL) score.  MDL rewriting has been
used many times before; we recommend Gr\"{u}nwald's book~\cite{mdlbook}
as a starting point for reading about this topic.
Given this background of basic algorithms, the key contribution
of the work is more applied:  we have borrowed basic algorithms to get
started, modified them so they work effectively on an important,
understudied domain (ad hoc system logs),
proven it is possible to scale the algorithms up
to the point that they may be applied to massive industrial data
sets, and empirically evaluated the results.

The incremental elements of our algorithm design are partly
inspired by traditional compiler error detection and correction
techniques.  In particular, the idea of using synchronizing tokens
as a means for accumulating chunks of unknown/unparseable data
has long been used in parsers from programming languages
(see Appel's text~\cite{appel:modern-compiler} for an
introduction to such techniques).  This heuristic appears to
work well in our domain of systems logs as these logs are
usually structured around punctuation symbols (commas, semi-colons,
verticle bars, parens, newlines, {\em etc.}) that act as
field-terminators and hence work well as synchronizing tokens.

Other incremental algorithms for learning grammars from example data
has been developed in the past.  For example, Parekh and 
Honavar~\cite{parekh+:incremental} have developed and proven correct
an incremental interactive algorithm for inferring
regular grammars from positive examples and membership queries.
This algorithm works quite differently than ours:  it operates over
automata and it uses membership queries, which ours does not.
More broadly speaking, Parekh and Honavar and many other related algorithms
work provide beautiful theoretical guarantees.  In contrast, we have
not studied the theoretical properties of our algorithms, but have
instead focused on implementation, empirical 
evaluation and scaling to support massive data sets.

Another place in which grammar induction is used is in information
extraction from web pages.  One example (amongst many, many others) is
work by Chidlovskii {\em et al.}~\cite{chidlovskii+:wrapper-generation},
which seeks to learn wrappers ({\em i.e.,} data extraction functions)
by using a modified edit distance algorithm.  Our algorithm also
uses edit distance in its guts to measure similarity
between chunks of data.  However, the edit distance metric we use is just
one element of a larger induction algorithm related to Arasu and
Garcina-Molina's recursive descent algorithm mentioned above.
Chidlovskii's algorithm is also incremental -- it
integrates one new record of data at a time into a grammar.
Our algorithm integrates batches of new data at a time.  One
reason we chose a batch-oriented approach is that processing
data in batches helps
disambiguate between various possibilities for both token
definitions and tree structure.  The tagged tree-structure of
XML or HTML documents eliminates many of the ambiguities that
appear in log files where the separators or tags are not known
a priori.  Our ad hoc data sets also appear different
from the web-based data studied by Chidlovskii in terms of their
scale:  Chidlovskii's algorithms take up to
30 seconds on up to 30 kilobytes of data; our algorithms 
take hundreds of times longer on data a million times larger.

Other related information-extraction efforts
are those that attempt to identify tabular data 
either from free-form text~\cite{Ng+:texttables,Pinto+:texttables} or
from web pages~\cite{Lerman+:webtables}.  These approaches typically
use hand-labelled examples to train machine learning systems to
identify the tables.  They then use heuristics specific to tabular
data to extract the tuples contained within those tables.  The portion
of this work related to identifying structured data from within more
free-form documents is complementary to ours.  The portion responsible
for deconstructing the identified tables uses more specific
domain-knowledge related to the form of tables than we do.

Many researchers have studied the problem of learning
a schema such as a DTD or XSchema from a collection 
of XML
documents~\cite{bex+:dtd-inference,bex+:inferring-xml-schema,fernau:learning-xml,garofalakis+:xtract}.  
At a high level, this task is similar to the kind of format inference
we are attempting to do, but
the details differ because, as mentioned above in reference to
Chidlovskii's work, XML has different characteristics
from ad hoc data:  XML documents have a well-nested tree 
shape, with obvious delimiters defining the structure
and the XML tags help with tokenization.  
As a result of these differences,
XML inference algorithms cannot be used ``off-the-shelf'' for understanding
the structure of ad hoc data.  They must be modified, tuned and
empirically evaluated on this new task.
%% One line of research on schema inference for XML makes use of the 
%% observation that 99\% of the content models for XML nodes are defined as
%% SOREs or CHAREs~\cite{martens+:expressiveness-xml-schema}. 
%% %(recall, these
%% %are heavily restricted forms of regular expressions).  
%% This observation allows \cite{bex+:dtd-inference} to define
%% an efficient algorithm for inferring concise DTDs.  Later 
%% \cite{bex+:inferring-xml-schema} build on this work 
%% by showing how to infer $k$-local XML Schema definitions also based on
%% SORES.  A $k$-local definition allows node content to depend on the parent
%% tag, grandparent tag, etc. (up to $k$ levels for some fixed $k$).
%% As mentioned earlier, hand-written PADS descriptions do not generally obey
%% the SOREs or CHAREs restriction, nor are they generally arranged with a nesting
%% structure that suggests $k$-local inference will be particularly useful.
%% The successful application of these techniques to XML data reinforces 
%% the idea that the ad hoc data we analyze has quite different characteristics
%% from XML, and therefore the ad hoc data inference problem merits study
%% independent of the XML inference problem.
Having made this point,
one of the more closely related XML schema inference systems
is XTRACT~\cite{garofalakis+:xtract}.  
It operates in three phases: generalization,
factoring and MDL optimization.  The first phase plays a role similar to
our structure discovery phase in that it generates a
collection of candidate structures from a series of XML examples.
This generalization phase searches for patterns in XML
data; it is tuned using the authors' knowledge of common DTD
structures.  Factoring decreases the size of generated candidate DTDs;
some of the factoring rules resemble our rewriting rules.
Finally, they tackle the MDL optimization problem by mapping the
problem into an instance of the NP-complete Facility Location Problem,
which they solve using a quadratic approximation algorithm.
Our MDL-guided rewriting problem considers a more general set of
rewriting rules and hence we cannot reuse this particular technique.

Finally, Potter's Wheel~\cite{raman+:potterwheel} is a system that attempts to
help users find and purge errors from
relational data sources.  It does so through the use of a spread-sheet
style interface, but in the background, a grammar inference algorithm
infers the structure of the input data, which may be ``ad hoc,'' 
somewhat like ours.  This inference algorithm operates by
enumerating all possible sequences of base types that appear
in the training data.  Like our work, Potter's Wheel is interested
in large-scale data processing problems and is designed
to process data incrementally and interactively.
Since Potter's Wheel is aimed at processing
relational data, they only infer \cd{struct} types
as opposed to enumerations, arrays, switches or unions.  

%% The TSIMMIS project~\cite{chawathe+:tsimmis} aims to
%% allow users to manage and query collections of heterogeneous, ad hoc
%% data sources.  TSIMMIS sits on top of the Rufus
%% system~\cite{shoens+:rufus}, which supports automatic classification
%% of data sources based on features such as the presence of certain
%% keywords, magic numbers appearing at the beginning of files and file
%% type.  
%% %The sources are classified using categories such as ``email''
%% %and ``C program.''  
%% This sort of classification is materially
%% different from the syntactic analysis we have developed.
