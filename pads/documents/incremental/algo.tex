\section{The Algorithm}
\label{sec:algo}
Intuitively, the incremental learning of a description from a large data source works
as follows. We first divide the data source into a sequence of smaller, manageable chunks.
For the first chunk, we learn a description $d_0$ using the format inference
algorithm descripted in section \ref{sec:review}. Then for each of the remaining chunk,
an {\em incremental learning step}, defined below, is performed.  

\begin{definition}[Incremental learning step]
Given an initial description $d$, and some new data lines $xs$, infer a new description $d'$ 
which is as similar to $d$ as possible such that $d'$ subsumes $d$ and describes $xs$.
\end{definition}

The new description $d'$ has to describe both the new data $xs$ and
all the old data which was described by $d$. 
In addition, we prefer $d'$ to be as similar to $d$ as possible because, 
first, the whole learning process could begin with a presumably higher quality description 
initially written by a human expert. We would like to retain the its shape as much as possible.
Second, descriptions often serve as documentation of the data sources, and they are
read and modified by users from time to time. Keeping the documentation relatively stable increases
the ease of management. The rest of the section will focus on an algorithm that
performs one incremental learning step.
We first give the inductive definitions of
a description $D$, a data representation $R$, and an aggregate structure $A$. 

{\small 
\begin{verbatim}
Descriptions:
Base ::= Pint | PstringME(re)
D ::=   
  Base           (Base token)
| Sync s         (Synchronizing token) 
| x:D1 * D2      (Dependent pair)
| D1 + D2        (Union)
| D array(s, t)  (Array)

Data representation:
BaseR ::= Str s | Int i | Error
SyncR ::= Good | Fail | Recovered s 
R ::=
  BaseR
| SyncR
| (R1, R2)
| inl R | inr R
| arrayR (R list, SyncR list, SyncR)

Aggregation structure:
A :: = 
  Base
| Sync s
| (A1, A2)
| unionA(A_l, A_r)
| arrayA (A_elem, A_sep, A_term)
| opt A
| l [s]
\end{verbatim}
}

For ease of presentation, we present the description language as a simplified abstract
syntax tree that resembles a subset of \pads{} features. We assume just two types of base tokens: 
an integer token, and a string token that matches a
regular expression. Synchronizing tokens, or sync token in short, are constant strings. 
Synchronization tokens often
serves as delimiters in the data, such as white spaces or punctuations.
A dependent pair binds
the first component to variable $x$, and then usess $x$ in its second component. It is 
straight-forward to extend the pairs and binary unions to n-ary structs and unions.
A array has a element which is described by $D$, and a separator string $s$ and a
terminator string $t$. 

The data representation is the result of parsing an input string using a description $d$.
For example, parsing a base token can result in a string, an integer or an error.
Besides {\tt Good} and {\tt Fail}, parsing a sync token can also result in 
a recovered mode, where $s$ is the string up to, but not including the good parse of the
sync token. In other words, $s$ is the ``skipped'' data when trying to parse a sync token.
The parse of a dependent pair is a pair of representations, and the
parse of a union is the left or right injection of the represention of parsing the left or
right branch of the union. The parse of an array includes a list of parses for the element,
a list of parses for the separator and a parse for the terminator which appears at the end of
the array.

The aggregation structure is the {\em accumulation} of the parsed data over data lines.
It ressembles the structure of the description AST, with two additional nodes: an option
node, and a learn node. The option node indicates that the description underneath this node is
optional, while the learn node $l$ accumulates all the ``skipped'' data in the recovered mode.
The data strings associated with the learn nodes are ``extraneous'' data that need be learned to
get its description. And such additional descriptions will be added to the original description
to get a new description. Essentially, the aggregate contains all the necessary information
to update the previous description.

\begin{figure}[t]
\begin{codebox}
incremental_step(d, xs) =
  as := [\kw{init_aggregate}(d)];
  foreach x in xs \{
    rs := \kw{parse_all}(d, x);
    as' := [];
    foreach r in rs \{
      foreach a in as \{
        a' := \kw{aggregate}(a, r); 
        as' := as' @ [a']
      \}
    \}
    as := as'
  \} 
  best_a := \kw{select_best}(as);
  d' = \kw{update_desc}(d, best_a);  
  return d'
\end{codebox}
\caption{A pseudo-code for the incremental learning step}
\label{fig:inc-learning}
\end{figure}

Figure \ref{fig:inc-learning} shows the overall algorithm for the incremental step in
pseudo-code. The {\tt init\_aggregate} function initializes an empty aggregate
structure according to a description $d$. The two tree structures are essentially the
same. Then for each data line $x$, we parse it into a list of data representations $rs$
using the {\tt parse\_all} function. We then call the {\tt aggregate} function to merge
each parse in $rs$ with each $a$ in the current list of aggregates. We use `@' to denote
concatenation of two lists. When we finish
parsing all the data lines and obtain a final list of aggregates $as$, we select
the best aggregate according to some criterion, and finally update the previous description
$d$ into $d'$ using the best aggregate. Next, we discuss the {\tt parse\_all},
{\tt aggregate}, {\tt select\_best} and {\tt update\_desc} function in more details.

%\begin{codebox}
%parse_all (d, x) =
%  switch (d) \{
%    case Pint =>  
%      (s, remainder) := match_prefix(x, "[0-9+\-]+");
%      if s != "" then return (Int s, remainder)
%      else return [(Error, x)];
%    case PstringME(re) => 
%      (s, remainder) := match_prefix(x, re);
%      if s != "" then return (Str s, remainder)
%      else return [(Error, x)];
%    case Sync s => 
%      (s', prefix, remainder) := match(x, s);
%      if s' = s and prefix = "" then return (Good, remainder)
%      elseif s' = "" then return (Fail, remainder)
%      else return [(Recovered prefix, remainder)]
%    case (x:d1, d2) =>
%      rs1 := parse_all (d1, x);
%      rs2 := [];
%      foreach (r1, remainder) in rs1 \{
%        (r2, remainder2) := parse_all (d2, remainder);
%        rs2 := rs2 + [((r1, r2), remainder2)]
%      \}
%    case (d1 + d2) => 
%	parse_all(d1, x) @ parse_all(d2, x)
%    case d array(sep, term) =>
%  \} 
%\end{codebox}    
%
The {\tt parse\_all} function takes a description $d$ and an input string $x$, and returns
a list of all possible parses along with their respective ending position in the input. 
This function implements a standard recursive descent parser which recursively matches the
description structure (and sub-structures) with the input. Figure \ref{fig:parse_base}
shows the {\tt parse\_base} function which parses a base token. The {\tt match\_prefix}
function matches the prefix of an input string with a regular expression and returns
the matched string and the remainder in the input.

\begin{figure}[t]
\begin{codebox}
parse_base (b, x) =
  switch (b) \{
    case Pint => 
      (s, suffix) := match_prefix(x, "[0-0+\-]+");
      if s <> "" then return [(Int s, suffix)];
      else return [(Error, x)]
    case PstringME(re) => 
      (s, suffix) := match_prefix(x, re);
      if s <> "" then return [(Str s, suffix)];
      else return [(Error, x)];
  \}
\end{codebox}
\caption{Function to parse a base token} \label{fig:parse_base}
\end{figure}

As an example, let $d$ be {\tt (Pint, Sync "|") + (PstringME "[a-z]+", Sync "|")}, 
and $x$ be ``\verb#abc|#''. {\tt parse\_all(d, x)} gives the following two possible parses:
{\small
\begin{verbatim}
  inl (Error, Recovered "abc")
  inr (Str "abc", Good)
\end{verbatim}
}

The {\tt aggregate} function adds a parse into an existing aggregate structure. When there is
no errors in the parse, it makes no changes to the aggregate. If the parse contains 
an error or failure for parsing token $b$, then the aggregate component 
$b$ is transformed to $opt b$, to indicate that $b$ node is optional. 
If a parse contains a recovered data $Recovered~ r$, then
a optional learn node will be created before the sync node. And the new aggregate component will be
$(opt (l [r]),~ Sync~ s)$. If the aggregate structure already contains the learn node before this
sync node, then recovered data $r$ will be added to the list under $l$.

The {\tt select\_best} function computes a score by counting the total number of $opt$ and $l$ nodes
in each of the aggregates, and returns the one with the smallest number. The idea is that the
aggregate with the smallest number of added nodes is more likely to represent a new description
that is the closest to the original description. 

Finally, the {\tt update\_desc} function does two things. First, it invokes the format inference
algorithm to learn a sub-description for the data strings collected at each of the learn nodes
and replace the learn nodes with these new sub-descriptions. Then, it uses a number of rewriting
rules into improve the overall structure. We will discuss some of these rewriting rules in more
details in section \ref{sec:imp}.


% - problem definition (as close to the previous description as possible) (but we don't
%   have a metric to measure how close yet, do we want to mention tree edit distance??)
% - overview of algorithm: parsing + aggregating + rewriting
% - parsing algo (parse rep, score metric, pseudo-code)
% - aggregating algo (in pseudo code)
% - selection of top aggregates
% - update original description
% - rewriting rules (data independent, data dependent, OptsTable)

