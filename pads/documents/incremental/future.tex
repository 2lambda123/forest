\section{Future Work}
\label{sec:future}
Despite the various optimization efforts we put in
the incremental learning system, parsing can still
be inefficient, especially for very complex descriptions
- those involving many unions, nested arrays and also
large number of short constant string fields in a struct.
These characteristics cause the
number of parses to go up very quickly. Below
are some of the ideas which might improve the
efficiency of our system.

First, our experiences told us that
the quality of the final output and the efficiency
of parsing is very sensitive to the initial
description. Therefore, coming up with a good
initial description is essential to the success
of the system. We may need better array-finding and
blob-finding algorithms in \learnpads{}. Blob-finding
is a procedure that looks for ``text-like'' data in the
log which is too complex to analyze and creates a blob
structure for it. This helps reduce the complexity of
the description. Incorporating human modifications 
before each incremental step is also one of the solutions.
In fact, this is one of the reasons why we wanted to
have incremental learning!

Second, we are interested in the techniques that
convert a \pads{} description into a finite-state
automaton, and parse the data using the automaton.
The advantage of doing this is that our base token
definitions are in regular expression and parsing 
these already requires the translation from regular
expression to DFA. Having the whole description in
DFA form can reduce the amount of work done on parsing
these base tokens by leveraging dynamic programming
techniques. However, there is the issue of a potential
exponential blow-up when once converts an NFA into
a DFA. We have to study the practical trade-off here.
 
Last but not least, it is conceivable that 
{\tt parse\_all} function can be parallelized. In
particular, whenever a choice point is made during
parsing, {\tt parse\_all}
can be called for all the alternatives in
parallel. The thread that finds the perfect
parse first commits and kills the other running threads.
This is an idea inspired by the {\em generalized committed 
choice} \cite{JaffarYZ07:gcc}. 
 

%  - efficiency in parsing
%  - concurrent parsing (GCC)
%  - quality of initial description
