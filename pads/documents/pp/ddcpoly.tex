\section{A Data Description Calculus}
\label{sec:ddc}

At the heart of our work is a polymorphic data description calculus (\ddc{}),
designed to capture the core features of \padsml.

\subsection{\ddc{} Syntax}
\begin{figure}
{\small
\begin{bnf}
  \name{Kinds} \meta{\gk} \::= \kty \| \ity \-> \gk 
                               \| \kty \-> \gk \\
  \name{Types} \meta{\ty} \::= 
    \ptrue\| \pfalse \| \pbase{e} \| 
    \plam{\var}{\ity}{\ty} \| \papp{\ty}{e} \nlalt
    \psig x \ty \ty \| \psum \ty e \ty \| \pand \ty \ty \|
    \pset x \ty e \| \pseq \ty \ty {\pterm e \ty} \nlalt
    \ptyvar       \| \pmu{\ptyvar}{\gk}{\ty} \| \pcompute e \ity \| 
     \pabsorb \ty \| \pscan{\ty} 
    \nlalt \ptylam{\ptyvar}{\kty}{\ty} \| \ptyapp{\ty}{\ty}
    \nlalt \ptry{\tau} \| \ptransform{e}{\ty}{\ty}{e} \| \pkleene{\ty}{e}
\end{bnf}
}
\caption{\ddc{} Syntax}
\label{fig:ddc-syntax}
\end{figure}
\figref{fig:ddc-syntax} shows the syntax of \ddc{}. 

\subsection{\Implang{} Language}
\label{sec:host-lang}
\begin{figure}[tp]
\small
\begin{bnf}
%   \name{Variables} \meta{f,x,y} \\
%   \name{Bit}   \meta{b}   \::= 0 \| 1 \\ 
  \name{Bits}  \meta{B}   \::= \cdot \| 0\,B \| 1\,B \\ 
  \name{Constants} \meta{c} \::=
      () \| \itrue \| \ifalse \| 0 \| 1 \| -1 \| \dots \nlalt
      \ierr \| \data \| \off \| \iok \| \iecerr \| \iecpc \| \ldots \\

  \name{Values} \meta{v} \::= 
      \const \| % \ilam{\nrm \var}{\ity}{e} \| 
      \ifun {\nrm f} {\nrm x} e \| \ipair v v \nlalt
      \iinld{\ity}{v} \| \iinrd{\ity}{v} \|
      \iarr{\vec{v}} \\

  \name{Operators} \meta{op} \::= 
      = \; \| \; < \; \| \inotop % \| \isizeofop
      \| \ldots \\

  \name{Expressions} \meta{e} \::= 
      \const \| \var \| \iop{e} \|
%      \ilam {\nrm \var} \ity e \| 
      \ifun {\nrm f} {\nrm x} e \| 
      \iapp e e \nlalt
%      \iletfun {\nrm f} {\nrm x} e \; \iin \; e' \| 
      \ilet {\nrm x} e \; e \|
      \iif e \; \ithen e \; \ielse e \nlalt
      \ipair{e}{e} \| \ipi {\nrm i}{e} \|
      \iinld{\ity}{e} \| \iinrd{\ity}{e} \nlalt
      \icaseg{e}{\nrm x}{e}{\nrm x}{e} \nlalt
      \iarr{\vec e} \| \iappend e e \| \isub e {\nrm e}
      \\
      
  \name{Base Types} \meta{a} \::= 
      \iunitty \| \iboolty \| \iintty  \| 
      \invty \nlalt  \ibitsty \| \ioffty \| \iecty
  \\
  \name{Types} \meta{\ity} \::= 
      \ibasety \| \ityvar \| \iarrow \ity \ity \| \iprod \ity \ity \|
      \isum \ity \ity \nlalt
      \iseq \ity \| \forall \ityvar{:}\kappa.\ity  \|
      \imu \ityvar \ity   
      \| \lambda \alpha{:}\kappa.\ity 
      \| \ity \; \ity
  \\
  \name{Kinds} \meta{\kappa} \::= \kty \| \kappa \rightarrow \kappa
  
\end{bnf}
\caption{\Implang{} Language: F$^\omega$}
\label{fig:implang-syntax}
\end{figure}

In \figref{fig:implang-syntax}, we present the host language of \ddc{}, an
extension of the simple-typed polymorphic lambda calculus. 
We use this host language both to encode the parsing semantics of \ddc{} 
and to write the expressions that can appear within \ddc{} itself.

As the calculus is largely standard, we highlight only its
unusual features. The constants include bitstrings $\data$; offsets $\off$,
representing locations in bitstrings; and error codes $\iok$,
$\iecerr$, and $\iecpc$, indicating success, success with errors and
failure, respectively. We use the constant $\ierr$ to indicate a
failed parse.  
Because of its specific meaning, we forbid its use in user-supplied expressions
appearing in \ddc{} types. 
%We include a special $\isizeofop$ operator, which returns the size in the data
%source of its input. 
Our expressions include arbitrary length
sequences $\iarr{\vec e}$, sequence append $\iappend e
{e'}$, and sequence indexing $\isub e {\nrm i}$.

The type $\invty$ is the singleton type of the constant $\ierr$.
Types $\iecty$ and $\ioffty$ classify error codes and bit string
offsets, respectively. The remaining types have standard
meanings: function types, product types, sum types, sequence types
$\iseqty \ty$; polymorphic types $\forall \ityvar.\ity$ and type
variables $\ityvar$; and recursive types $\imu \ityvar \ity$.

We extend the formal syntax with some syntactic sugar 
for use in the rest of the paper: anonymous functions
$\ilam {\nrm x} \ity e$ for $\ifun {\nrm f} {\nrm x} e$, with $f
\not\in {\rm FV}(e)$; function bindings $\iletfun {\nrm f} {\nrm x} e
\; \iin \; e'$ for $\ilet {\nrm f} {\ifun {\nrm f} {\nrm x} e} \; e'$;
$\ispty$ for $\iprod \ioffty \ioffty$.  We often use
pattern-matching syntax for pairs in place of explicit projections, as
in $\lampair{\codefont e}$ and $\ilet {\itup{\off,r,p}} e\; e'$.  Although
we have no formal records with named fields, we use a dot notation for
commonly occuring projections. For example, for a pair $\mathtt x$ of
rep and PD, we use $\codefont{x.rep}$ and $\codefont{x.pd}$ for the
left and right projections of $\codefont{x}$, respectively. \cut{Generally,
the particular projection intended should be apparent from context,
and will be specified when it is not.} Also, sums and products are
right-associative. 

We use standard judgments for the static semantics
($\stsem[e,{\ctxt},\ity]$) and operational semantics ($e
\stepsto e'$) of the \implang{} language. Details appear in \appref{app:host-lang}.

\section{\ddc{} Semantics}
\label{sec:ddc-sem}

The primitives of \ddc{} are deceptively simple.  Each captures a
simple concept, often familiar from type theory. However, in reality,
each primitive is multi-faceted. Each simultaneously describes a
collection of valid bit strings, two datatypes in the host language --
one for the data representation itself and one for its parse
descriptor -- and a transformation from bit strings, including
invalid ones, into data and corresponding meta-data.
We give semantics to \ddc{} types using three semantic functions, each
of which precisely conveys a particular facet of a type's meaning.
The functions $\itsem[\cdot]$ and $\itpdsem[\cdot]$ describe the {\it
  representation semantics} of \ddc{}, detailing the types of the
data's in-memory representation and parse descriptor. The function
$\trans[\cdot,,]$ describes the {\it parsing semantics} of \ddc{},
defining a \implang{} language function for each type that parses bit
strings to produce a representation and parse descriptor. We define
the set of valid bit strings for each type to be those strings for
which the PD indicates no errors when parsed.

We begin with a kinding judgment that checks if
a type is well formed. We then formalize
the three-fold semantics of \ddc{} types.

\begin{table}
  \begin{center}
    \renewcommand{\arraystretch}{1.35}
    \begin{tabular}{l l}
      $\ddck[\ty,{\pctxt;\rctxt;\ctxt},\kind,\mcon]$ & {\it \ddc{}-type
        kinding}\\
      $\itsem[\ty] = \ity$ & {\it representation types of \ddc{} types}\\
      $\itpdsem[\ty] = \ity$ & {\it pd types of \ddc{} types}\\
      $\trans[\ty,\ctxt,\gk] = e$   & {\it \ddc{}-type parser semantics} \\
      $\transpp[\ty,\ctxt,\gk] = e$   & {\it \ddc{}-type pretty printer semantics} \\
      $\kTrans[\gk,\ty] = \ity$     & {\it parser type} \\
      $\kTransPP[\gk,\ty] = \ity$     & {\it pretty printer type} \\
      $\ptyc \rctxt = \ctxt$     & {\it context parser type}\\
      $\stsem[e,{\pctxt;\ctxt},\ity]$ & {\it \implang expression typing} \\
      $e \stepsto e'$ & {\it \implang expression evaluation}
    \end{tabular}
    \caption{Translations and Judgments}
    \label{tab:judg-list}
  \end{center}
\end{table}
For reference, we provide in
\tblref{tab:judg-list} a listing of all the functions and judgments
defined in this section and a brief description of each.  


\subsection{\ddc{} Kinding}
\label{sec:ddc-kinding}

\begin{figure*}[t]
\small
\fbox{$\ddck[\ty,\pctxt;\rctxt;\ctxt,\kind,\mcon]$}\\[-2ex]
\[
\infer[\text{Unit}]{
    \ddck[\ptrue,\pctxt;\rctxt;\ctxt,\kty,\con]
  }{\wfd {} \ctxt}
\quad 
\infer[\text{Bottom}]{
    \ddck[\pfalse,\pctxt;\rctxt;\ctxt,\kty,\con]
  }{\wfd {} \ctxt}
\quad 
\infer[\text{Const}]{
    \ddck[\pbase{e},\pctxt;\rctxt;\ctxt,\kty,\con]
  }{
    \begin{semcond}
      \stsem[e,\ctxt,\ity] &
      (\vlet {\ity \iarrowi \kty} {\Ikind(C)})
    \end{semcond}
  }
\]

\[
\infer[\text{Abs}]{
    \ddck[\plam{\var}{\ity}{\ty},
         \pctxt;\rctxt;\ctxt,\ity \iarrowi \kind,\mcon]
  }{
    \ddck[\ty,\pctxt;\rctxt;\ectxt{\var{:}\ity},\kind,\mcon]
  }
\quad
\infer[\text{App}]{
  \ddck[\papp{\ty}{e},\pctxt;\rctxt;\ctxt,\gk,\mcon]
}{
  \ddck[\ty,\pctxt;\rctxt;\ctxt,\ity \iarrowi \gk,\mcon] &
  \stsem[e,\ctxt,\ity]
}
\]

\[
\infer[\text{Prod}]{
    \ddck[\psig{x}{\ty}{\ty'},\pctxt;\rctxt;\ctxt,\kty,\con]
  }{       
    \ddck[\ty,\pctxt;\rctxt;\ctxt,\kty,\mcon] &
    \ddck[\ty',\pctxt;\rctxt;
          \ectxt {x{:}\iprod {\itsem[\asub \rctxt \ty]} 
              {\itpdsem[\asub \rctxt \ty]}},
          \kty,\mcon']
  }
\]

\[
\infer[\text{Sum}]{
    \ddck[\psum{\ty}{e}{\ty'},\pctxt;\rctxt;\ctxt,\kty,\con]
  }{
    \ddck[\ty,\pctxt;\rctxt;\ctxt,\kty,\mcon] & \ddck[\ty',\pctxt;\rctxt;\ctxt,\kty,\mcon'] 
  }
\quad
  \infer[\text{Intersection}]{
    \ddck[\pand \ty {\ty'},\pctxt;\rctxt;\ctxt,\kty,\con]
  }{
    \ddck[\ty,\pctxt;\rctxt;\ctxt,\kty,\mcon] & \ddck[\ty',\pctxt;\rctxt;\ctxt,\kty,\mcon'] 
  }
\]

\[
  \infer[\text{Con}]{
    \ddck[\pset x \ty e,\pctxt;\rctxt;\ctxt,\kty,\con]
  }{ 
    \ddck[\ty,\pctxt;\rctxt;\ctxt,\kty,\mcon] & 
    \stsem[e,
    \ectxt{x{:}\iprod{\itsem[\asub \rctxt \ty]} 
      {\itpdsem[\asub \rctxt \ty]}},\iboolty]
  }
\]

\[\infer[\text{Seq}]{
    \ddck[\pseq \ty {\ty_s} {\pterm e {\ty_t}},\pctxt;\rctxt;\ctxt,\kty,\con]
  }{
    \begin{array}{c}
    \ddck[\ty,\pctxt;\rctxt;\ctxt,\kty,\mcon] \qquad
    \ddck[{\ty_s},\pctxt;\rctxt;\ctxt,\kty,\mcon_s] \qquad
    \ddck[{\ty_t},\pctxt;\rctxt;\ctxt,\kty,\mcon_t] \\
    \stsem[e,\ctxt,
    \iprod {\itsem[{\ty_m}]}      
    {\itpdsem[{\ty_m}]}
    \iarrowi \iboolty]
    \quad (\ty_m = \asub \rctxt {\pseq \ty {\ty_s} {\pterm e {\ty_t}}})
    \end{array}
  }
\]

\[
  \infer[\text{RecVar}]{
    \ddck[\ptyvar,{\pctxt;\rctxt;\ctxt},\kty,\ncon]
  }{\wfd {} \ctxt \quad \ptyvar {=} \pmu \ptyvar \kty \ty \in \dom \rctxt}
\quad
  \infer[\text{Rec}]{
    \ddck[\pmu \ptyvar \kty \ty,\pctxt;\rctxt;\ctxt,\kty,\con]
  }{
    \ddck[\ty,{\rctxt,\ptyvar {=} \pmu \ptyvar \kty \ty;\ctxt},\kty,\con]
  }
\]

\[
  \infer[\text{Compute}]{       
    \ddck[\pcompute{e}{\ity},\pctxt;\rctxt;\ctxt,\kty,\con]
  }{
    \stsem[e,\ctxt,\ity]
  }      
\quad
\infer[\text{Absorb}]{
    \ddck[\pabsorb{\ty},\pctxt;\rctxt;\ctxt,\kty,\con]
  }{
    \ddck[\ty,\pctxt;\rctxt;\ctxt,\kty,\mcon]
  }
\quad
  \infer[\text{Scan}]{
    \ddck[\pscan{\ty},\pctxt;\rctxt;\ctxt,\kty,\con]
  }{
    \ddck[\ty,\pctxt;\rctxt;\ctxt,\kty,\mcon]
  }
\]

\[
  \infer[\text{TyVar}]{
    \ddck[\ptyvar,{\pctxt;\rctxt;\ctxt},\kty,\ncon]
  }{\wfd {} \ctxt \quad \tyvar{:}\kty \in \pctxt}
\quad
\infer[\text{TyAbs}]{
    \ddck[\ptylam{\tyvar}{\kty}{\ty},
         \pctxt;\rctxt;\ctxt,\kty \iarrowi \kind,\mcon]
  }{
    \ddck[\ty,{\pctxt,\tyvar{:}\kty;\rctxt;\ectxt},\kind,\mcon]
  }
\quad
\infer[\text{TyApp}]{
  \ddck[\ptyapp{\ty_1}{\ty_2},\pctxt;\rctxt;\ctxt,\gk,\mcon]
}{
  \ddck[\ty_1,\pctxt;\rctxt;\ctxt,\kty \iarrowi \gk,\mcon] &
  \ddck[\ty_2,\pctxt;\rctxt;\ctxt,\kty,\mcon]
}
\]

\[
  \infer[\text{Try}]{       
    \ddck[\ptry{\ty},\pctxt;\rctxt;\ctxt,\kty,\con]
  }{
    \ddck[\ty,\pctxt;\rctxt;\ctxt,\kty,\mcon]
  }      
\quad
\infer[\text{Transform}]{
    \ddck[\ptransform{e_1}{\ty_1}{\ty_2}{e_2},\pctxt;\rctxt;\ctxt,\kty,\con]
  }{
    \begin{array}{c}
    \ddck[\ty_1,\pctxt;\rctxt;\ctxt,\kty,\mcon]\\
    \ddck[\ty_2,\pctxt;\rctxt;\ctxt,\kty,\mcon]\\
    \stsem[e_1,\ctxt,
    \iprod {\itsem[{\asub \rctxt {\ty_1}}]}      
           {\itpdsem[{\asub \rctxt {\ty_1}}]}
    \iarrowi \iprod {\itsem[{\asub \rctxt {\ty_2}}]}      
           {\itpdsem[{\asub \rctxt {\ty_2}}]}]\\
    \stsem[e_2,\ctxt,
    \iprod {\itsem[{\asub \rctxt {\ty_2}}]}      
           {\itpdsem[{\asub \rctxt {\ty_2}}]}
    \iarrowi \iprod {\itsem[{\asub \rctxt {\ty_1}}]}      
           {\itpdsem[{\asub \rctxt {\ty_1}}]}]
    \end{array}
  }
\]

\[
  \infer[\text{Kleene}]{
    \ddck[\pkleene{\ty}{e},\pctxt;\rctxt;\ctxt,\kty,\con]
  }{
    \begin{array}{c}
    \ddck[\ty,\pctxt;\rctxt;\ctxt,\kty,\mcon]\\
    \stsem[e,\ctxt,
    \iprod {\iseq{\itsem[{\asub \rctxt \ty}]}}
    {\iseq{\itpdsem[{\asub \rctxt \ty}]}}
    \iarrowi \iboolty]
    \end{array}
  }
\]


\caption{\ddc{} Kinding Rules}
\label{fig:ddc-kinding}
\end{figure*}

The kinding judgment defined in \figref{fig:ddc-kinding} determines
well-formed \ddc{} types, assigning kind $\kty$ to basic types and
kind $\ity \iarrowi \kind$ to type abstractions.  We use two contexts to express our kinding judgment:
\[
\begin{array}{ll}
\ctxt  & \mathrel{::=} \cdot \bnfalt \ctxt,{\var{:}\ity}\\
\rctxt & \mathrel{::=} \cdot \bnfalt \rctxt,\ptyvar{=}\pmu \ptyvar \gk \ty\\
\pctxt  & \mathrel{::=} \cdot \bnfalt \pctxt,\tyvar{:}\kty
\end{array}
\]
Context $\Gamma$ is a finite partial map that binds expression
variables to their types.
Context $\rctxt$ is an ordered list of mappings between type variables and recursive types.
This context serves two purposes: first, to ensure the well-formedness
of types with free type variables; and second, to provide mappings
between recursive type variables and their associated types. 
This second purpose leads us to consider a context $\rctxt$ to be a
substitution from type variables to types. Application of such a
substitution has the form $\asub \rctxt \ty$.
Context $\rctxt$ is a list of type variables with kind $\kty$.

To ensure that recursive types have properly-shaped parse descriptors
with a valid PD header (a condition necessary for the type safety of
generated parsers), we disallow types such as $\pmu \ptyvar
\ptyvar$. More generally, we ensure that
recursive type variables are separated from their binder by at least
one basic primitive, such as a product or sum, a condition called {\it contractiveness}. To this end, we annotate every judgment with a contractiveness
indicator, one of $\con$, $\ncon$, or $\mcon$. A
$\con$ indicates the type is contractive, an $\ncon$
indicates it is not, and a $\mcon$ indicates it may be either. 
We consider $\ncon < \con$. 

As the rules are otherwise mostly straightforward, we highlight
just two of them. We use the function $\Ikind$ to assign kinds to base types.
While their kind does not differentiate them from type
abstractions, base types are not well formed when not applied.  
Once applied, all base types have kind $\kty$. The product rule
shows that the name of the first component is bound to a pair of a representation and corresponding PD.
The semantic functions defined in the next section determine the type of this pair.
Note that we apply $\rctxt$ to the type of the first component before
translation, thereby closing it,
 as open \ddc{}
types do not translate into well-formed \implang{} types.

\subsection{Representation Semantics}
\label{sec:intty-sem}

\begin{figure}
\fbox{$\itsem[\ty] = \ity$}
\[
\begin{array}{lcl} 
\itsem[\ptrue] & = & \iunitty \\
\itsem[\pfalse] & = & \invty \\
\itsem[\pbase{e}] & = & \isum {\Irty(C)} \invty   \\
\itsem[\plam{\var}{\ity}{\ty}] & = & \itsem[\ty] \\
\itsem[\papp \ty e] & = & \itsem[\ty] \\
\itsem[\psig \var {\ty_1} {\ty_2}]  & = & \iprod {\itsem[\ty_1]} {\itsem[\ty_2]}    \\
\itsem[\psum {\ty_1} e {\ty_2}]     & = & \isum {\itsem[\ty_1]} {\itsem[\ty_2]} \\
\itsem[\pand {\ty_1} {\ty_2}]  & = & \iprod {\itsem[\ty_1]}{\itsem[\ty_2]}\\
\itsem[\pset x \ty e] & = & \isum {\itsem[\ty]}{\itsem[\ty]}\\
% field names: length, elts
\itsem[\pseq \ty {\ty_{\text{sep}}} {\pterm e {\ty_{\text{term}}}}] & = & 
    \iprod \iintty {(\iseq{\itsem[\ty]})}             \\
%% \itsem[\pcase e c {\ty_1} {\ty_2}]       & = & \isum {\itsem[\ty_1]} {\itsem[\ty_2]}\\
\itsem[\ptyvar] & = & \ptyvar_\repname \\
\itsem[\pmu{\ptyvar}{\gk}{\ty}] & = & \imu{\ptyvar_\repname}{\itsem[\ty]} \\
\itsem[\pcompute e \ity]                 & = & \ity \\
\itsem[\pabsorb \ty]                     & = & \isum \iunitty \invty \\
\itsem[\pscan \ty] & = & \isum {\itsem[\ty]} \invty \\
%% \pext{
%% \itsem[\ptransform e e \ty]              & = & \itsem[\ty]\\
%% }
\itsem[\lambda \ptyvar.\ty]       & = & \lambda \ptyvar_\repname.\itsem[\ty] \\
\itsem[\ty_1 \ty_2]              & = & \itsem[\ty_1] \itsem[\ty_2] \\
\itsem[\ptry{\ty}]    & = & \itsem[\ty] \\
\itsem[\ptransform{e_1}{\ty_1}{\ty_2}{e_2}] & = & \itsem[\ty_2]\\
\itsem[\pkleene{\ty}{e}] & = & \iseq{\itsem[\ty]}
\end{array}
\]
\caption{Representation Types}
\label{fig:rep-tys}
\end{figure}

In Figure~\ref{fig:rep-tys}, we present the representation type
of each \ddc{} primitive. While the primitives are
dependent types, the mapping to the \implang{} language erases the dependency because the \implang{} language does not have dependent types. For \ddc{} types in which expressions appear,
the translation drops the expressions to remove the dependency.
With these expressions gone, variables become useless, so we drop 
variable bindings as well, as in product and constrained types.
Similarly, as type abstraction and application are only relevant for
dependency, we translate them according to their underlying types.

In more detail,
the \ddc{} type $\ptrue$ consumes no input and produces only
the $\iunitty$ value.  Correspondingly, $\pfalse$ consumes no input,
but uniformly fails, producing the value $\invty$. The
function $\Irty$ maps each base type to a representation for
successfully parsed data. Note that this representation does not depend
on the argument expression. As base type parsers can fail, we sum this type
with $\invty$ to produce the actual representation type.
Intersection types produce a pair of values, one for each sub-type,
because the representations of the subtypes need not be identical nor
even compatible. 
Constrained types produce sums, where a left branch indicates the data
satisfies the constraint and the right indicates it does not. In
the latter case, the parser returns the offending data rather than
$\ierr$ because the error is semantic rather than syntactic.
Sequences produce a \implang{} language sequence paired with its
length.  Recursive types generate recursive representations. Note that the \implang{} type uses the same variable name
as the \ddc{} type, and so the type corresponding to the type variable
$\ptyvar$ is exactly $\ityvar$.
The output of a $\pcomputen$ is exactly the computed value, and
therefore shares its type.  The output of $\pabsorbn$ is a sum
indicating whether parsing the underlying type succeeded or failed.
The type of $\pscann$ is similar, but also returns an element of the
underlying type in case of success.

\begin{figure}
\fbox{$\itpdsem[\ty] = \ity$}
\[ 
\begin{array}{lcl} 
%% %% example: \ua.(int * a) + None
%% %%          pd = \ua.pd_hdr  * ((pd_hdr * ([int]_pd * [a]_pd)) + [None]_pd)
%% %%             = \ua.pd_hdr  * ((pd_hdr * ([int]_pd * a)) + [None]_pd)
\itpdsem[\ptrue] & = & \ipty \iunitty \\                                                  
\itpdsem[\pfalse] & = & \ipty \iunitty \\                                                  
\itpdsem[\pbase{e}] & = & \ipty \iunitty\\
\itpdsem[\plam \var \ity \ty] & = & \itpdsem[\ty] \\
\itpdsem[\papp \ty e] & = & \itpdsem[\ty] \\
\itpdsem[\psig \var {\ty_1} {\ty_2}] & = & 
               \ipty {\iprod {\itpdsem[\ty_1]} {\itpdsem[\ty_2]}} \\
\itpdsem[\psum {\ty_1} e {\ty_2}] & = & 
               \ipty {(\isum {\itpdsem[\ty_1]} {\itpdsem[\ty_2]})} \\
\itpdsem[\pand {\ty_1} {\ty_2}] & = & \ipty {\iprod {\itpdsem[\ty_1]} {\itpdsem[\ty_2]}}    \\
\itpdsem[\pset x \ty e] & = & \ipty {\itpdsem[\ty]} \\
\itpdsem[\pseq \ty {\ty_{\text{sep}}} {\pterm e {\ty_{\text{term}}}}] & = & 
  \iapty {\itpdsem[\ty]} \\
\itpdsem[\ptyvar] & = & \ipty{\ptyvar_\pdname} \\
\itpdsem[\pmu \ptyvar \kty \ty] & = & 
  \ipty{\imu{\ptyvar_\pdname}{\itpdsem[\ty]}} \\
\itpdsem[\pcompute e \ity]            & = & \ipty \iunitty \\
\itpdsem[\pabsorb \ty]                & = & \ipty \iunitty \\
\itpdsem[\pscan{\ty}] & = & \ipty {(\isum {(\iprod \iintty
    {\itpdsem[\ty]})} \iunitty)} \\
\itpdsem[\lambda \ptyvar.\ty]      
     & = & \lambda \ptyvar_\pdname.\itpdsem[\ty] \\
\itpdsem[\ty_1 \ty_2]            & = & \itpdsem[\ty_1] \itpdsemstrip[\ty_2] \\
\itpdsem[\ptry{\ty}]    & = & \ipty{\itpdsem[\ty]} \\
\itpdsem[\ptransform{e_1}{\ty_1}{\ty_2}{e_2}] & = & \itpdsem[\ty_2]\\
\itpdsem[\pkleene{\ty}{e}] & = & \iapty {\itpdsem[\ty]}
\end{array}
\]

\fbox{$\itpdsemstrip[\ty] = \ity$}

\[
\begin{array}{lcl} 
\itpdsemstrip[\ty] & = & \ity \ \ \mbox{where}\ \itpdsem[\ty] = \ipty{\ity}
\end{array}
\]
\caption{Parse Descriptor Types}
\label{fig:pd-tys}
\end{figure}

In \figref{fig:pd-tys}, we give the parse descriptor
type for each \ddc{} type. Each PD type has a header and body.
This common shape allows us to define functions that polymorphically
process PDs based on their headers. Each header stores the number of
errors encountered during parsing, an error code indicating the degree
of success of the parse -- success, success with errors, or failure --
and the span of data described by the descriptor.  Formally, the type
of the header  ($\tyface{pd\_hdr}$) is $\iintty \iprodi \iecty \iprodi
\ispty$.  Each body consists of subdescriptors corresponding to the
subcomponents of the representation and any type-specific meta-data. For types with neither subcomponents nor special meta-data, we
use $\iunitty$ as the body type.

We discuss a few of the more complicated parse descriptors in detail.
The parse descriptor body for sequences contains the parse descriptors of its elements,
the number of element errors, and the sequence length. Note that the
number of element errors is distinct from the number of sequence
errors, as sequences can have errors that are not related to their
elements (such as errors reading separators).  We introduce an
abbreviation for array PD body types, $\iaptyname \; \ity =
\iintty \iprodi \iintty \iprodi (\iseq \ity)$.
\trversion{The $\pcomputen$ parse descriptors have no subelements because the
data they describe is not parsed from the data source.}
The $\pabsorbn$ PD
type is $\iunitty$ as with its representation. We assume that just as
the user does not want the represenation to be kept, so too the parse
descriptor.  The $\pscann{}$ parse descriptor is either $\iunitty$, in case
no match was found, or records the number of bits skipped before the
type was matched along with the type's corresponding parse descriptor.


\begin{figure}
\small
\fbox{$\kTrans[\gk,\ty] = \ity$} 
    
\begin{align*}
  &\kTrans[\kty,\ty] = \extdom * \offdom \iarrowi \offdom * \itsem[\ty] * \itpdsem[\ty]
   \\
   &\kTrans[\ity \iarrowi \gk,\ty] = \ity \iarrowi \kTrans[\gk,\ty]
   \\
   &\kTrans[\kty \iarrowi \gk,\ty] = 
      \forall\tyvar_\repname.\forall\tyvar_\pdname.
         \kTrans[\kty,\tyvar] \iarrowi \kTrans[\gk,\ty \tyvar] 
\end{align*}  
  \caption{\Implang{} Language Types for Parsing Functions}
  \label{fig:parser-types}
\end{figure}

\subsection{Parsing Semantics of the \ddc{}}
\label{sec:parse-sem}

\begin{figure*}
\small
\fbox{$\trans[\ty,\ctxt,\gk] = e$} 

\[
\begin{array}{l}
  %% None 
\trans[\ptrue,\ctxt,\kty] =
  \lampair{\spair<\off,\newrep{unit}{},\newpd{unit}{\off}>}
\\[3pt] %\\
%% False 
\trans[\pfalse,,] =
  \lampair{\spair<\off,\newrep {bottom}{},\newpd {bottom}{\off}>}
\\[3pt] %\\ 
%% Const 
\trans[\pbase{e},\ctxt,\kty] =
  \lampair{\iapp {\iapp {\Iimp(C)} (e)} {\itup {\idata,\off}}}
\\[3pt] %\\
%% Abs 
\trans[\plam{\var}{\ity}{\ty},,] =
   \sfn{\nrm\var}{\ity}{\trans[\ty,\ectxt{\var{:}\ity},\kind]}
\\[3pt] %\\
%% App 
\trans[\papp{\ty}{e},\ctxt,\gk] =
  \trans[\ty,,] \sapp e  
\\[3pt]
%% Prod 
%\begin{array}{l}
\trans[\psig{x}{\ty}{\ty'},\ctxt,\kty] = \\
  \begin{array}{l}  
    \lampair{} \\
    \quad  \ilet {\spair<\off',r,p>} 
    {{\trans[\ty,,]} \sapp \spair<\idata,\off>} \\
    \quad  \ilet x {\ictup{r,p}}\\
    \quad  \ilet {\spair<\off'',r',p'>} 
    {{\trans[\ty',,]} \sapp \spair<\idata,\off'>} \\
    \quad \spair<\off'',\newrep {\gS}{r,r'},\newpd {\gS}{p,p'}>
  \end{array}  
%\end{array}
\\
%% Sum 
%\begin{array}{l}
  \trans[\psum{\ty}{e}{\ty'},,] = \\
  \begin{array}{l}  
  \lampair{} \\
  \quad \ilet {\itup{\off',r,p}}{\trans[\ty,,] \sapp \spair<\idata,\off>} \\
  \quad \iif {\pdok p} \; \ithen {
    \def \r {\newrep {+left}{r}}
    \def \p {\newpd {+left}{p}}
    \spair<\off',\r,\p>} \\
  \quad \ielse {\ilet {\itup{\off',r,p}}{\trans[\ty',,] \sapp \spair<\idata,\off>}} \\
  \quad 
  {  % begin scope
    \def \r {\newrep {+right}{r}}
    \def \p {\newpd {+right}{p}}
    %% 
    \spair<\off',\r,\p>
  }\\ % end scope
  \end{array}
\\
%% Intersection 
  \trans[\pand{\ty}{\ty'},,] = \\
  \begin{array}{l}  
     \lampair{} \\
     \quad \ilet {\itup{\off',r,p}} {\trans[\ty,,] \sapp \spair<\idata,\off>} \\
     \quad \ilet {\itup{\off'',r',p'}} {\trans[\ty',,] \sapp \spair<\idata,\off>} \\
     \quad {\spair<\codefont{max}(\off',\off''),\newrep {\&}{r,r'},\newpd {\&}{p,p'}>}
   \end{array}
\\
%\quad
%% Set 
  \trans[\pset{x}{\ty}{e},\ctxt,\kty] = \\
  \begin{array}{l}  
    \lampair{} \\
    \quad \ilet {\itup{\off',r,p}}{\trans[\ty,,] \sapp \spair<\idata,\off>} \\
    \quad \ilet x {\ictup{r,p}}\\
    \quad \ilet c e \\
    \quad \spair<\off',\newrep {con} {c,r},\newpd {con} {c,p}>
  \end{array}
\\
%% Array 
\trans[\pseq{\ty}{\ty_s}{\pterm e {\ty_t}},,] = \\
  \begin{array}{l}  
    \lampair{}\\
      \quad \iletfun {isDone}{\itup{\off,r,p}}{\\
        \qquad \ior {\eofpred {\idata,\off}} {e\codefont {\sapp
          \spair<r,p>}} \iori \\
        \qquad \ilet {\itup{\off',r',p'}}{\trans[\ty_t,,] \spair<\idata,\off>}\\
        \qquad \pdok{p'}
      }\\
      \quad \iin \\
      \quad \iletfun {continue} {\itup{\off,\off',r,p}} {\\
        \qquad \iif  {\off = \off' \iori \isdone {\off',r,p}} \; \ithen {\itup{\off',\codefont{r,p}}} \\
        \qquad \ielse {
          \ilet {\itup{\off_s,r_s,p_s}}{\trans[\ty_s,,] \sapp \spair<\idata,\off'>}}\\
        \qquad \ilet {\itup{\off_e,r_e,p_e}}{\trans[\ty,,] \sapp \ictup{\idata,\off_s}}\\
        \qquad \mathtt{continue} \sapp \ictup{
            \off,\off_e,\newrep {seq} {r,r_e}, \newpd {seq} {p, p_s, p_e}
        }}\\
      \quad \iin
   \end{array}\\
  \begin{array}{l}  
      \quad \ilet {\mathtt{r}} {\newrep {seq\_init}{}}\\
      \quad \ilet {\mathtt{p}} {\newpd {seq\_init}{\off}}\\
      \quad \iif {\isdone{\off,r,p}} \; \ithen {\itup{\off,\codefont{r,p}}}\\
      \quad \ielse {\ilet {\itup{\off_e,r_e,p_e}}{\trans[\ty,,] \sapp
          \spair<\idata,\off>}} \\
      \quad \mathtt{continue} \sapp \ictup{\off,\off_e,
        \newrep {seq} {r,r_e}, \newpd {seq} {p, \newpd {unit} \off, p_e}}      
  \end{array}  
\\
\end{array}
%\quad
\begin{array}{l}
%% Var
\trans[\ptyvar,,] = \codefont{f_\ptyvar}
\\[3pt]
%% Mu
\trans[\pmu \ptyvar \gk \ty,,] = \\
  \begin{array}{l}
  \ifun {f_\ptyvar} {\itup{\data,\off}} {}\\
  \quad \ilet {\itup{\off',r,p}} 
  {\trans[\ty,,] \iappi \ictup{\data,\off}} \\ 
  \qquad \ictup{\off',r,\newpd{mu}{p}}
  \end{array}  
\\[3pt]
%% Compute
\trans[\pcompute e \ity,,] = \\
  \quad \lampair{\itup{\off,\newrep {compute} {\nrm e},\newpd {compute} \off}}
\\[3pt]
%% Absorb
\trans[\pabsorb \ty,,] = \\
  \begin{array}{l}  
    \lampair{}\\
    \quad \ilet {\itup {\off',r,p}} {\trans[\ty,,] \sapp \spair<\idata,\off>}\\
    \quad \itup{\off',\newrep {absorb} p,\newpd {absorb} p}   
  \end{array}  
\\
%% Scan
\trans[\pscan \ty,,] = \\
  \begin{array}{l}  
    \lampair{}\\
    \quad \iletfun {try} {i} {\\
      \qquad \ilet {\itup{\off',r,p}} {\trans[\ty,,] \sapp
        \codefont{\spair<\data,\off + i>}} \\
      \qquad \iif {\pdok p}\; \ithen \\
      \qquad {\ictup{\off',\newrep {scan} r,
        \newpd {scan} {i,p}}}\; \ielse {}\\
      \qquad \iif {\codefont{i = scanMax}}\; \ithen \\
      \qquad {\ictup{\off,\newrep {scan\_err} {},
        \newpd {scan\_err} {\off}}}\; \ielse {}\\
      \qquad \codefont {try \sapp (i+1)}
   }\\
   \quad \iin \sapp \codefont{try \sapp 0} \\
  \end{array}  
\\
%% lambda \alpha
\trans[\lambda\tyvar . \ty,,] = %\\
%  \begin{array}{l}
    \Lambda \tyvar_\repname. 
    \Lambda \tyvar_\pdname. \lambda \codefont{f_\ptyvar}. \trans[\ty,,]
%  \end{array}  
\\
%% t1 t2
\trans[\ty_1 \ty_2,,] = 
    \trans[\ty_1,,]\; [\itsem[\ty_2]]\; [\itpdsemstrip[\ty_2]]\; \trans[\ty_2,,]

\\
%% Try ty
\trans[\ptry \ty,,] = \\
  \begin{array}{l}  
    \lampair{}\\
    \quad \ilet {\itup {\off',r,p}} {\trans[\ty,,] \sapp \spair<\idata,\off>}\\
    \quad \itup{\off,r,\newpd {try} p}   
  \end{array}  
\\
%% Transform e1 ty1 ty2 e2
\trans[\ptransform{e_1}{\ty_1}{\ty_2}{e_2},,] = \\
  \begin{array}{l}  
    \lampair{}\\
    \quad \ilet {\itup {\off',r,p}} {\trans[{\ty_1},,] \sapp \spair<\idata,\off>}\\
    \quad \ilet {\itup {r',p'}}{e_1\, r\, p}\\
    \quad \itup{\off',r', p'}   
  \end{array}  
\\
%% Kleene 
\trans[\pkleene{\ty}{e},,] = \\
  \begin{array}{l}  
    \lampair{}\\
      \quad \iletfun {isDone}{\itup{\off',r,p}}{\\
        \qquad \ior {\eofpred {\idata,\off'}} {e\codefont {\sapp
          \spair<r,p>}} 
      }\\
      \quad \iin \\
      \quad \iletfun {continue} {\itup{\off,\off',r,p}} {\\
        \qquad \iif  {\off = \off' \iori \isdone {\off',r,p}} \; \ithen {\itup{\off',\codefont{r,p}}} \\
        \qquad \ielse {
               \ilet {\itup{\off_e,r_e,p_e}}{\trans[\ty,,] \sapp \ictup{\idata,\off'}}}\\
        \qquad \mathtt{continue} \sapp \ictup{
            \off,\off_e,\newrep {kleene} {r,r_e}, \newpd {kleene} {p, p_e}
        }}\\
      \quad \iin
   \end{array}\\
  \begin{array}{l}  
      \quad \ilet {\mathtt{r}} {\newrep {kleene\_init}{}}\\
      \quad \ilet {\mathtt{p}} {\newpd {kleene\_init}{\off}}\\
      \quad \iif {\isdone{\off,r,p}} \; \ithen {\itup{\off,\codefont{r,p}}}\\
      \quad \ielse {\ilet {\itup{\off_e,r_e,p_e}}{\trans[\ty,,] \sapp
          \spair<\idata,\off>}} \\
      \quad \mathtt{continue} \sapp \ictup{\off,\off_e,
        \newrep {kleene} {r,r_e}, \newpd {kleene} {p, p_e}}      
  \end{array}  
\\
\end{array}
\]
%\caption{\ddc{} Semantics (cont.)}
\caption{\ddc{} Semantics}
\label{fig:ddc-sem}
\end{figure*}

\begin{figure}
\small
\begin{itemize}
\renewcommand{\labelitemi}{}

%\item %[Unit:]
\item $\ifun {R_{unit}} \iuval \iuval$
\item $\ifun {P_{unit}} \off {\itup{\itup{0,\iok,\ipair \off \off},\iuval}}$

%\item %[Bottom:]
\item $\ifun {R_{bottom}} \iuval \ierr$
\item $\ifun {P_{bottom}} \off ((1,\iecpc,\ipair \off \off),())$

\item %[Pair:]
\item $\ifun {R_{\gS}} {\ipair {r_1} {r_2}} {\itup {\codefont{r_1,r_2}}}$
\item $\ifun{H_{\gS}} {\ictup{h_1,h_2}}{}$ \\
  $\begin{array}{l}
    \ilet {nerr} {\codefont{pos \itup{{h_1}.{nerr}} + pos \itup{{h_2}.{nerr}}}}\\
    \ilet {ec} {\iif {\codefont{h_2.ec} = \iecpc}\; \ithen {\iecpc}\\
    \quad \ielse {\codefont{max\_ec} \iappi \codefont{h_1.ec} \iappi \codefont{h_2.ec}}} \\
    \ilet {sp} {\ictup{h_1.sp.begin, h_2.sp.end}} \\
    \quad \ictup {nerr,ec,sp}
  \end{array}$

\item $\ifun {P_{\gS}} {\ictup{p_1, p_2}} {\ictup {H_{\gS} \itup{p_1.h,p_2.h},\itup{p_1,p_2}}}$

\end{itemize}
\caption{Selected Constructor Functions. 
The type of PD headers is $\iintty
  \iprodi \iecty \iprodi \ispty$. 
  We refer to the projections using
  dot notation as $\codefont{nerr}$, $\codefont{ec}$ and
  $\codefont{sp}$, respectively. A span is a pair of offsets, referred
  to as $\codefont{begin}$ and $\codefont{end}$, respectively.  The full collection of such constructor functions appears in \appref{app:asst-functions}.}
\label{fig:cons-funs}
%\caption{Constructor Functions (cont.)}
\end{figure}


The parsing semantics of a type $\tau$ is a function that transforms some amount of input into a pair of a representation and a parse descriptor, the types of which are determined by $\tau$.
\figref{fig:parser-types} specifies the \implang{} language types of the parsers generated from well-kinded \ddc{} types.  Note that parameterized \ddc{} types require their arguments before they can parse any input.

\figref{fig:ddc-sem} shows the parsing semantics function.  For each
type, the input to the corresponding parser is a bit string and an
offset which indicates the point in the bit string at which parsing
should commence.  The output is a new offset, a representation of the
parsed data, and a parse descriptor. As the bit string input is
never modified, it is not returned as an output.  In addition
to specifying how to handle correct data, each function describes how
to transform corrupted bit strings, marking detected errors in
a parse descriptor. The semantics function is partial, applying only
to well-formed \ddc{} types.

For any type, there are three steps to parsing: parse the
subcomponents of the type (if any), assemble the resultant representation, and
tabulate meta-data based on subcomponent meta-data
(if any). For the sake of clarity, we have factored the latter two
steps into separate representation and PD constructor functions which we define for
each type. For some types, we additionally factor the PD header
construction into a separate function. For example, the representation 
and PD constructors for $\ptrue$ are $\newrepf {unit}$ and $\newpdf
{unit}$, respectively, and the header constructor for products is
${\codefont{H_{\gS}}}$. Selected constructors are shown in
\figref{fig:cons-funs}. We have also factored out some commonly
occuring code into ``built-in'' functions, explained as needed and
defined formally in \appref{app:asst-functions}.

The PD constructors determine the error code and
calculate the error count.  There are three possible error codes:
$\iok$, $\iecerr$, and $\iecpc$, corresponding to the three possible results of a parse: 
it can succeed, parsing the data without errors; it can succeed,
but discover errors in the process; or, it can find an
unrecoverable error and fail.
\trversion{
Note that the the purpose of the $\iecpc$ code is to indicate to any
higher level elements that some form of error recovery is required.
Hence, the whole parse is marked as failed exactly when the parse ends
in failure.}
The error count is determined by subcomponent error counts and any errors associated directly with the type
itself.  
\trversion{
If a subcomponent has errors then the error count is
increased by one; otherwise its not increased at all. We use the
function $\codefont {pos}$, which maps all positive numbers to 1
(leaving zero as is), to assist in calculating the contribution of
subcomponents to the total error count.  Errors at the level of the
element itself - such as constraint violation in constrained types - are
generally counted individually.}

With this background, we can now discuss selected portions of the semantics.
%With this background, we can now understand the semantics. 
The semantics of $\ptrue$ and $\pfalse$ show that they do not consume any input, \ie{}, they do not change the offset. 
A look at their constructors shows that the parse
descriptor for $\ptrue$ always indicates no errors and a corresponding
$\iok$ code, while that of $\pfalse$ always indicates failure with an
error count of one and the $\iecpc$ error code. The semantics of base
types applies the implementation of the base type's parser, provided
by the function $\Iimp$, to the appropriate arguments.  Abstraction
and application are defined directly in terms of \implang language
abstraction and application.  Dependent pairs read the first element
at $\off$ and then the second at $\off'$, the offset returned from
parsing the first element.  Notice that we bind the pair of the
returned representation and parse descriptor to the variable $\codefont{x}$
before parsing the second element, implicitly mapping the 
\ddc{} variable $x$ to the \implang{} language variable $\codefont{x}$ in the process.
Finally, we combine the results
using the constructor functions, returning $\off''$ as the final
offset of the parse.

\trversion{
Sums first attempt to parse according to the left type, returning the resulting
value if it parses without errors. Otherwise, it parses according to
the right type. Intersections read both types starting at the same
point. They advance the stream to the maximum of the two offsets
returned by the component parsers. The construction of the parse
descriptor is similar to that of products. For constrained types, we call the
parser for the underlying type $\ty$, bind $\var$ to the resulting rep
and PD, and check whether constraint is satisfied. The result
indicates whether the data has a semantic error and is used in
constructing the representation and PD. For example, the PD constructor will add
one to the error count if the constraint is not satisfied. Notice that
we advance the stream independent of whether the constraint was
satisfied.
}
Sequences have the most complicated semantics because the number of subcomponents depends upon a combination of the data, the termination predicate, and the terminator type. Consequently, the sequence parser uses mutually
recursive functions $\codefont{isDone}$ and $\codefont{continue}$ to implement this open-ended semantics. 
Function $\codefont{isDone}$ determines if the parser
should terminate by checking whether the end of the source has been
reached, the termination condition $e$ has been satisfied, or the
terminator type can be read from the stream without errors at
$\off$.
Function $\codefont{continue}$ takes four
arguments: two offsets, a sequence representation, and a sequence PD.  The two
offsets are the starting and ending offset of the previous round of
parsing. They are compared to determine whether the parser is
progressing in the source, a check that is critical to ensuring that
the parser terminates. Next, the parser checks whether the sequence is
finished, and if so, terminates. Otherwise, it attempts to read a
separator followed by an element and then continues parsing the
sequence with a call to $\codefont{continue}$.

\trversion{
Finally, the body of the parser creates an initial sequence representation and PD and
then checks whether the sequence described is empty. If not, it reads
an element and creates a new rep and PD for the sequence.  Note that
it passes the PD for $\ptrue$ in place of a separator PD, as no
separator is read before the first element.  Finally, it continues
reading the sequence with a call to $\codefont{continue}$.

Because of  the iterative nature of sequence parsing, the representation and PD are constructed incrementally. The parser first creates an empty representation and PD
and then adds elements to them with each call to
$\codefont{continue}$. The error count for an array is the sum of the
number of separators with errors plus one if there were any element
errors. Therefore, in function ${\codefont{H_{seq}}}$ we first check
if the element is the first with an error, setting $\codefont{eerr}$
to one if so. Then, the new error count is a sum of the old,
potentially one for a separator error, and $\codefont{eerr}$. In
$\newpdf{seq}$ we calculate the element error count by unconditionally
adding one if the element had an error.
}
We translate recursive types into
recursive functions with a special
function name corresponding to the name of the 
bound type   variable.
Recursive type variables translate to these special names.
\trversion{We note that the body of the recursive function is somewhat
  redundant. However, the simpler encoding of $\ifun {f_\ptyvar}
  {\itup{\data,\off}} {\trans[\ty,,] \; \itup{\data,\off}}$ would have
  complicated the meta-theory.}

The $\pscann$ type attempts to parse the underlying type from the
stream at an increasing scan-offset, $i$, from the original offset
$\off$, until success is achieved or a predefined maximum scan-offset
(\cd{scanMax}) is reached.  In the semantics we give here, offsets are
incremented one bit at a time --- a practical implementation would choose
some larger increment ({\it e.g.,} 32 bits at a time).

\trversion{
The definition of $\pcomputen$ just calls the compute constructors.
The representation constructor returns the value computed by $e$, while the PD
records no errors and reports a span of length 0, as no data is
consumed by the computation. The $\pabsorbn$ parser first parses the
underlying type and then calls the absorb constructors, passing only
the PD, which is needed by the rep constructor to determine whether an
error occured while parsing the underlying type. If so, the value
returned is a $\ierr$. Otherwise, it is $\iunitty$.  The absorb parse
descriptor duplicates the error information of its underlying type.


The $\pscann$ type attempts to parse the underlying type from the
stream at an increasing scan-offset, $i$, from the original offset
$\off$, until success is achieved or a predefined maximum scan-offset
(\cd{scanMax}) is reached. Note that, upon success, $i$ is passed to
the PD constructor function, which both records it in the PD and sets
the error code based on it. It is considered a semantic error for the
value to be found at a positive $i$, whereas it is a syntactic error
for it not to be found at all.  }

\section{Meta-theory}
\label{sec:meta-theory}

\cut{This paragraph belongs somewhere in the TR:
A few things to note regarding variable names. First, all variable
names introduced in translation are by definition not equal to the
target of the substitution, nor present in the free variables of the
term being substituted. Second, for those types with bound variables,
we note that the potential alpha-conversion when performing a
substitution on the type exactly parallels any alpha-conversion of the
same variable where it appears in the translation of the type. Last,
all constructors, support functions and base-type parsers are closed
with respect to user-defined variable names.}

One of the most difficult, and perhaps most interesting, challenges of our
work on \ddc{} was determining what
properties we wanted to hold. What are the ``correct''
invariants of data description languages? While there are many
well-known desirable invariants for programming languages, the
meta-theory of data description languages has been
uncharted.
%
We present the following two properties as critical invariants of
our theory. We feel that they should hold, in some form, for any data
description language.
\begin{itemize}
\trversion{
\item {\bf Translation Completeness}: The semantic function
  $\trans[\cdot,,]$ is a total function of well-formed \ddc{} types.
}
\item {\bf Parser Type Correctness}: For a \ddc{} type $\ty$, the
  representation and PD output by the parsing function of $\ty$ will
  have the types specified by $\itsem[\ty]$ and
  $\itpdsem[\ty]$, respectively.
  
\item {\bf Parser Error Correlation}: For any representation and PD output by a
  parsing function, the errors reported in the PD
  will be correlated with the errors present in the representation.
\end{itemize}
%
\trversion{
Before presenting the formal statement of these properties, we specify
some assumptions that we make about \ddc{} base types, their kinds,
types, and implementations, that are necessary for the satisfaction of
these properties. In essence, we assume that the properties we desire of
the rest of the calculus hold for the base types.
}
% In formalizing these properties, we assume that \ddc{} base types, their kinds, representation and parse descriptor types, and parsers satisfy the properties we desire to hold of the rest of the calculus.  
% \appref{app:meta-theory} contains a formal statement of these assumptions.

\trversion{
\begin{theorem}[Translation Completeness]
\label{thm:translation-completeness}
  If \ $\ddck[\ty,,\kind,\mcon]$ then $\trans[\ty,,] = e$.
\end{theorem}

\begin{proof}
  By induction on kinding derivations. For constant case, use the
  first item of Condition~\ref{cond:base-types}.
\end{proof}
}
\trversion{
We continue by stating and proving that parsers are type correct.
However, to do so, we must first establish some typing properties of the Rep
and PD constructors, as at least one of them appears in most
parsing functions. 

With this lemma we now formally state that each constructor
produces a value whose type corresponds to its namesake \ddc{} type.
Note that all universally quantified \ddc{} types $\ty$ are assumed
to be well formed.

\begin{lemma}[Types of Constructors]
\label{lem:nice-types-of-constructors}
\begin{itemize}
\item $\newrepf {true} : \iarrow \iunitty \itsem[\ptrue]$
\item $\newpdf  {true} : \iarrow \ioffty {\itpdsem[\ptrue]}$
\item $\newrepf {false} : \iarrow \iunitty {\itsem[\pfalse]} $
\item $\newpdf  {false} : \iarrow \ioffty {\itpdsem[\pfalse]}$
\item $\forall \ty_1,\ty_2,x.\newrepf {\gS} : \iarrow {\iprod
    {\itsem[\ty_1]}{\itsem[\ty_2]}} {\itsem[\psig x {\ty_1}{\ty_2}]}$
\item $\forall \ty_1,\ty_2,x.\newpdf {\gS} : 
  \iarrow {\iprod {\itpdsem[\ty_1]}{\itpdsem[\ty_2]}}
  {\itpdsem[\psig x {\ty_1}{\ty_2}]}
$
\item $\forall \ty_1,\ty_2.\newrepf {+left} : \iarrow {\itsem[\ty_1]} 
                            {\itsem[\psum {\ty_1}{}{\ty_2}]}$
\item $\forall \ty_1,\ty_2.\newrepf {+right} : \iarrow {\itsem[\ty_2]} 
                            {\itsem[\psum {\ty_1}{}{\ty_2}]}$
\item $\forall \ty_1,\ty_2.\newpdf {+left} : \iarrow {\itpdsem[\ty_1]} 
                            {\itpdsem[\psum {\ty_1}{}{\ty_2}]}$
\item $\forall \ty_1,\ty_2.\newpdf {+right} : \iarrow {\itpdsem[\ty_2]} 
                            {\itpdsem[\psum {\ty_1}{}{\ty_2}]}$
\item $\forall \ty_1,\ty_2.\newrepf {\&} : \iarrow {\iprod
    {\itsem[\ty_1]}{\itsem[\ty_2]}} {\itsem[\pand {\ty_1}{\ty_2}]}$
\item $\forall \hdtvs {\ty_1}{\ty_2}.\newpdf {\&} : 
  \iarrow {\iprod {\itpdsem[\ty_1]}{\itpdsem[\ty_2]}}
  {\itpdsem[\pand {\ty_1}{\ty_2}]}$.
\item $\forall \ty,x,e.\newrepf {set} : \iprod \iboolty {\itsem[\ty]} 
  \iarrowi \itsem[\pset x \ty e]$
\item $\forall \ty,x,e.\newpdf {set} : \iprod \iboolty {\itpdsem[\ty]} 
  \iarrowi \itpdsem[\pset x \ty e]$
\item $\forall \ty,\ty_1.
  \newrepf {seq\_init} : \iarrow \iunitty {\itsem[\pseq \ty
      {\ty_1} {\_}]}$
\item $\forall \ty,\ty_1.
  \newpdf {seq\_init} : \iarrow \ioffty {\itpdsem[\pseq \ty
      {\ty_1} {\_}]}$
\item $\forall \ty,\ty_1.
  \newrepf {seq} : \iarrow {\itsem[\pseq \ty
      {\ty_1} {\_}] \iprodi \itsem[\ty]} 
  {\itsem[\pseq \ty
      {\ty_1} {\_}]}$
\item $\forall \ty,\ty_1.
  \newpdf {seq} : 
  \itpdsem[\pseq \ty {\ty_1} {\_}] \iprodi
   \itpdsem[\ty_1] \iprodi \itpdsem[\ty] \iarrowi \\
  \itpdsem[\pseq \ty {\ty_1} {\_}]$
% \item $\forall \ty,\ty_1,\ty_2,e.
%   \newrepf {seq\_init} : \iarrow \iunitty {\itsem[\pseq \ty
%       {\ty_1} {\pterm e {\ty_2}}]}$
% \item $\forall \ty,\ty_1,\ty_2,e.
%   \newpdf {seq\_init} : \iarrow \ioffty {\itpdsem[\pseq \ty
%       {\ty_1} {\pterm e {\ty_2}}]}$
% \item $\forall \ty,\ty_1,\ty_2,e.
%   \newrepf {seq} : \iarrow {\itsem[\pseq \ty
%       {\ty_1} {\pterm e {\ty_2}}] \iprodi \itsem[\ty]} 
%   {\itsem[\pseq \ty
%       {\ty_1} {\pterm e {\ty_2}}]}$
% \item $\forall \ty,\ty_1,\ty_2,e.
%   \newpdf {seq} : 
%   \itpdsem[\pseq \ty {\ty_1} {\pterm e {\ty_2}}] \iprodi
%    \itpdsem[\ty_1] \iprodi \itpdsem[\ty] \iarrowi \\
%   \itpdsem[\pseq \ty {\ty_1} {\pterm e {\ty_2}}]$
\item $\forall e.\newrepf {compute} : \forall \ga.\iarrow \ga {\itsem[\pcompute
    e \ga]}$
\item $\forall e.\newpdf {compute} : \forall \ga.\iarrow \ioffty
  {\itpdsem[\pcompute e \ga]}$
\item $\forall \ty.\newrepf {absorb} : \iarrow {\itpdsem[\ty]}
  {\itsem[\pabsorb \ty]}$
\item $\forall \ty.\newpdf {absorb} : \iarrow {\itpdsem[\ty]}
  {\itpdsem[\pabsorb \ty]}$
\item $\forall \ty.\newrepf {scan} : \itsem[\ty] \iarrowi
  \itsem[\pscan \ty]$
\item $\forall \ty.\newpdf {scan} : \iintty \iprodi \itpdsem[\ty] \iarrowi
  \itpdsem[\pscan \ty]$
\item $\forall \ty.\newpdf {scan} : \iintty \iprodi \itpdsem[\ty] \iarrowi
  \itpdsem[\pscan \ty]$
\item $\forall \ty.\newrepf {scan\_err} : \iunitty \iarrowi
  \itsem[\pscan \ty]$
\item $\forall \ty.\newpdf {scan\_err} : \ioffty \iarrowi
  \itpdsem[\pscan \ty]$
\end{itemize}  
\end{lemma}

\begin{proofsketch}
  By inspection of code. First we infer (by hand) the (polymorphic)
  type of each function. Then we verify that it matches the types
  specified above.
\end{proofsketch}
}

\trversion{
\begin{figure}
{\small
\fbox{$\ptyc{\rctxt} = \ctxt$} 
    
\[
  \ptyc{\cdot} = \cdot \qquad\qquad
  \ptyc{\rctxt,\ptyvar{=}\pmu \ptyvar \ty} = \ptyc \rctxt,\codefont{f_\ptyvar}{:}
  \kTrans[\kty,\asub \rctxt {\pmu \ptyvar \ty}]
\]}
  \caption{Recursive Parser \Implang{} Types}
  \label{fig:rec-parser-types}
\end{figure}
}

To prove our type correctness theorem by induction, we must account
for the fact that any free recursive type variables in
a \ddc{} type $\ty$ will become free function variables
in $\trans[\ty,,]$.  
\trversion{
To that end, we define in \figref{fig:rec-parser-types}
the function $\ptyc \rctxt$, 
which maps recursive variable contexts
$\rctxt$ to typing contexts $\ctxt$. 
}
\poplversion{
To that end, we define the function $\ptyc \rctxt$, 
which maps recursive variable contexts
$\rctxt$ to typing contexts $\ctxt$:
\vskip -1.5ex
{\small
\[
\begin{array}{l}
  \ptyc{\cdot} = \cdot \\[1ex]
  \ptyc{\rctxt,\ptyvar{=}\pmu \ptyvar \ty} = \ptyc \rctxt,\codefont{f_\ptyvar}{:}
  \kTrans[\kty,\asub \rctxt {\pmu \ptyvar \ty}]
\end{array}
\]%
}%
}%
\noindent
We also apply $\rctxt$ to $\ty$
to close any open references to recursive types before
determining the corresponding parser type. 
\trversion{  We do not explicitly require that $\ctxt$
be well formed in the premises, as it follows from the fact that $\ty$
is well formed in $\ctxt$.
}

\begin{theorem}[Type Correctness]
\label{thm:type-correctness}
  If  $\wfd \ctxt \rctxt$ and
   $\ddck[\ty,{\pctxt;\rctxt;\ctxt},\gk,\mcon]$ then
  $\stsem[{\trans[\ty,,]},{\ctxt,\ptyc \rctxt},
            {\kTrans[\kind,\asub \rctxt \ty]}]$.
\end{theorem}

\begin{proof}
  By induction on the height of the second derivation.
\end{proof}

\begin{corollary}[Type Correctness of Closed Types]
  If $\ddck[\ty,,\gk,\con]$ then
  $\stsem[{\trans[\ty,,]},,\kTrans[\kind,\ty]]$.  
\end{corollary}

We start our formalization of the error-correlation property by
defining representation and PD correlation.
Informally, a representation and a PD are correlated when the number
of errors recorded in the PD is at least as many as the number of
errors in the representation and semantic errors, \ie{},
constraint violations, are properly reported.  Formally, we define
correlation using two mutually recursive definitions.  The first,
$\corrkl \ty r p$, defines error correlation between a representation
$r$ and a parse descriptor $p$ at type $\ty$.  It does so by
computing a weak-head normal form $\tyval$ for $\ty$ and then
using the subsidiary relation $\corr \tyval r p$, which is defined for
all weak-head normal types $\tyval$ with base kind $\kty$.
Types with higher kind such as abstractions are excluded from this definition
as they cannot directly produce representations and PDs.
\figref{fig:revised-ddc-syntax} defines the weak-head
normal types $\tyval$ and give normalization rules while the
following definitions specify error correlation.   
Below, we abbreviate $p.h.{nerr}$ as $p.{nerr}$.
and use $\mathtt{pos}$ to denote the function which returns zero when
passed zero and one otherwise.



\begin{figure}
\small
\begin{bnf}
%   \name{Kinds} \meta{\gk} \::= \kty \| \ity \-> \gk 
%                                \pext{\| \gk \-> \gk} \\
  \mname{Normalized\\ Types}{2} \meta{\tyval} \::= 
    \ptrue\| \pfalse \| \pbase{e} \| \plam{\var}{\ity}{\ty} \|
%  \nlnalt{Types}
%%        \|
%%       \pext{\plam{\ptyvar}{\gk}{\ty} \| \papp{\ty}{\ty} \nlalt}
%%       \pxpd{\ty}{e}
%%       \pext{\nlalt
%%         \ptransform{e}{e}{\ty} \| 
%%       }
    \psig x \ty \ty  \nlalt
    \psum \ty e \ty  \| \pand \ty \ty \|
    \pset x \ty e \|
    \pseq \ty \ty {\pterm e \ty} \nlalt
    \pcompute e \ity \| \pabsorb \ty \| \pscan{\ty} 
    \\
  \name{Types} \meta{\ty} \::= \tyval \| \papp{\ty}{e} \|
    \ptyvar \| \pmu{\ptyvar}{\ty} 
\end{bnf}  
\[
  \infer{
    \papp {\ty} {e} \stepsto \papp {\ty'} {e}
  }{
    \ty \stepsto \ty'
  }
\quad
  \infer{
    \papp {\tyval} {e} \stepsto \papp {\tyval} {e'}
  }{
    e \stepsto e'
  }
\quad
  \infer{
    \papp {(\plam x {} \ty)} {v} \stepsto \ty[v/x]
  }{}
\quad
  \infer{
    \pmu \ptyvar \ty \stepsto \ty[\pmu \ptyvar \ty/\ptyvar]
  }{}
\]
  \caption{\ddc{} Weak-Head Normal Types and Normalization}
  \label{fig:ddc-reduction-rules}
  \label{fig:revised-ddc-syntax}
\end{figure}

\begin{definition}
$\corrkl \ty r p$ iff if $\ty \kstepsto \tyval$ then $\corr \tyval r p$.
\end{definition}

\begin{definition}[Representation and PD Correlation Relation]
$\corr \tyval r p$ iff exactly one of the following is true:
  \begin{itemize}
  \item $\tyval = \ptrue$ and $r = \iuval$ and $p.{nerr} = 0$.
  \item $\tyval = \pfalse$ and $r = \ierr$ and $p.{nerr} = 1$.
  \item $\tyval = \pbase{e}$ and $r = \iinld \ity \const$ and $p.{nerr} = 0$.
  \item $\tyval = \pbase{e}$ and $r = \iinrd \ity \ierr$ and $p.{nerr} = 1$.
  \item $\tyval = \psig x {\ty_1} {\ty_2}$ and $r =\ipair {r_1} {r_2}$ and $p =
    \ipair h {\ipair {p_1} {p_2}}$ 
    and $h.{nerr} = \mathtt{pos}(p_1.{nerr}) + \mathtt{pos}(p_2.{nerr})$, $\corrkl
    {\ty_1} {r_1} {p_1}$ and $\corrkl {\ty_2[(r,p)/x]} {r_2} {p_2}$.
  \item $\tyval = \psum {\ty_1} e {\ty_2}$ and $r =\iinld {\ity}{r'}$
    and $p = \ipair h {\iinld {\ity}{p'}}$
    and $h.{nerr} = \mathtt{pos}(p'.{nerr})$ and $\corrkl
    {\ty_1} {r'} {p'}$.
  \item $\tyval = \psum {\ty_1} e {\ty_2}$ and $r =\iinr {r'}$
    and $p = \ipair h {\iinr {p'}}$
    and $h.{nerr} = \mathtt{pos}(p'.{nerr})$ and $\corrkl
    {\ty_2} {r'} {p'}$.
  \item $\tyval = \pand {\ty_1} {\ty_2}$, $r = \ipair {r_1} {r_2}$ and $p =
    \ipair h {\ipair {p_1}{p_2}}$, 
    and $h.{nerr} = \mathtt{pos}(p_1.{nerr}) + \mathtt{pos}(p_2.{nerr})$, 
    $\corrkl {\ty_1} {r_1} {p_1}$ and $\corrkl {\ty_2} {r_2} {p_2}$.
  \item $\tyval = \pset x {\ty'} e$, $r = \iinld \ity {r'}$ and $p =
    \ipair h {p'}$, 
    and $h.{nerr} = \mathtt{pos}(p'.{nerr})$, $\corrkl {\ty'}{r'}{p'}$
    and $e[(r',p')/x] \kstepsto\itrue$.
  \item $\tyval = \pset x {\ty'} e$, $r = \iinrd \ity {r'}$
    and $p = \ipair h {p'}$,
    and $h.{nerr} = 1 + \mathtt{pos}(p'.{nerr})$,
    $\corrkl {\ty'}{r'}{p'}$ and $e[(r',p')/x] \kstepsto \ifalse$.
  \item $\tyval = \pseq {\ty_e}{\ty_s}{\pterm {e,\ty_t}}$, 
    $r = \ipair {len} {\iarr{\vec {r_i}}}$, $p = \itup{h,\itup{{neerr},{len}',\iarr {\vec {p_i}}}}$,
    ${len} = {len}'$, ${neerr} = \sum_{i=1}^{len}
    \mathtt{pos}(p_i.{nerr})$, $\corrkl {\ty_e}
    {r_i} {p_i}$, (for $i=1 \ldots {len}$), and
    $h.{nerr} \geq \mathtt{pos}({neerr})$.
%   \item $\tyval = \pmu \ptyvar {\ty'}$ and 
%     $\corrkl {\ty'[\pmu \ptyvar {\ty'}/\ptyvar]} r p$.
  \item $\tyval = \pcompute e \ity$ and $p.{nerr} = 0$.
  \item $\tyval = \pabsorb {\ty'}$, $r = \iinl \iuval$, and $p.nerr = 0$.
  \item $\tyval = \pabsorb {\ty'}$, $r = \iinr \ierr$, and $p.nerr > 0$.
  \item $\tyval = \pscan {\ty'}$, $r =\iinl {r'}$,
      $p = \ipair h {\iinl {\ipair i {p'}}}$,
      $h.nerr = \mathtt{pos}(i) + \mathtt{pos}(p'.nerr)$, and
      $\corrkl {\ty'}{r'}{p'}$.
  \item $\tyval = \pscan {\ty'}$,
      $r =\iinr \ierr$,
      $p = \ipair h {\iinr \iuval}$, and
      $h.{nerr} = 1$.
  \end{itemize}
\end{definition}

\trversion{
Once again, we first need to prove error correlation properties of
the Rep and PD constructors, as expressed in the following lemma.

\begin{lemma}[Correlation Properties of Constructors]
  \label{lem:cons-props}
  \begin{itemize}
  \item $\corr \ptrue {\newrep {true} {}} {\newpd {true} \off}$.
  \item $\corr \pfalse {\newrep {false} {}} {\newpd {false} \off}$.
  \item If $\corrkl {\ty_1} {r_1} {p_1}$ and $\corrkl {\ty_2[(r,p)/x]} {r_2} {p_2}$
    then\\ $\corr {\psig x {\ty_1} {\ty_2}}
    {\newrep {\gS} {r_1,r_2}}{\newpd {\gS} {p_1,p_2}}$.
  \item If $\corrkl \ty r p$ then $\corr {\psum \ty e {\ty'}} 
      {\newrep {+left} r} {\newpd {+left} p}$.
  \item If $\corrkl \ty r p$ then $\corr {\psum {\ty'} e \ty} 
      {\newrep {+right} r} {\newpd {+right} p}$.
  \item If $\corrkl {\ty_1} {r_1} {p_1}$ and $\corrkl {\ty_2} {r_2} {p_2}$
    then\\ $\corr {\pand {\ty_1} {\ty_2}}
    {\newrep {\&} {r_1,r_2}}{\newpd {\&} {p_1,p_2}}$.
  \item If $\corrkl \ty r p$ and $e[(r,p)/x] \kstepsto c$ then\\ $\corr {\pset x \ty e} 
    {\newrep {set} {c,r}} {\newpd {set} {c,p}}$
  \item $\corr {\pseq \ty {\ty_s}{\pterm e {\ty_t}}} 
    {\newrep {seq\_init} {}} {\newpd {seq\_init} \off}$.
  \item If $\corr {\pseq \ty {\ty_s} {\pterm e {\ty_t}}} r p$ and
    $\corrkl \ty {r'} {p'}$ then for any $p''$, $\corr {\pseq \ty {\ty_s}{\pterm e {\ty_t}}}
    {\newrep {seq} {r,r'}} {\newpd {seq} {p,p'',p'}}$.    
  \item $\corr {\pcompute e \ity} {\newrep {compute} {e}} {\newpd {compute} \off}$.
  \item $\corr {\pabsorb \ty} {\newrep {absorb} p} {\newpd {absorb} p}$.
  \item If $\corrkl {\ty} r p$ then $\corr {\pscan \ty} 
      {\newrep {scan} r} {\newpd {scan} {i,p}}$.
  \item $\corr {\pscan \ty} 
      {\newrep {scan\_err} {}} {\newpd {scan\_err} \off}$.
  \end{itemize}
\end{lemma}

\begin{proof}
  By inspection of code. 
\cut{
  \reminder{Fix the following or drop it: }
  Array case is most complicated, in particular proving the clause
  $h.{nerr} \geq \mathtt{pos}({neerr})$. To do so, you must prove that
  $H_{seq}$ maintains this invariant. The first case of the match is the
  hard one, as ${nerr}$ is 0 (if its $1$, then it must be greater than
  or equal to $\mathtt{pos}(n)$, for any $n$).  First, as $h.{nerr} =
  0$, so too must $neerr$. Next, note that in this first case,
  $h1.{nerr} = 0$. Now, the new value of ${neerr}$ is just the sum of
  the original ${neerr}$ and $\mathtt{pos}(h1.{nerr})$, that is, $0 +
  0$.}
\end{proof}
}

Definition~\ref{def:err-corr} specifies the exact property we require
of parsing functions. At base kind, we require that any representation and PD
returned by the parser must be correlated. At higher kind, we require
that the function preserve the property of error correlation. Hence,
the definition is a simple form of logical relation.
Lemma~\ref{lem:err-corr-at-T} states that any well-formed type of base
kind is error-correlated.

\begin{definition}[Error Correlation Relation]
\label{def:err-corr}
$\ecpred \ty \kind$ iff exactly one of the following is true:
  \begin{itemize}
  \item $\kind = \kty$ and if $\trans[\ty,,] \sapp \spair<B,\off> \kstepsto
  \spair<\off',r,p>$ then $\corrkl \ty r p$
  \item $\kind = \ity \iarrowi \kind'$ and if $\stsem[v,,\ity]$
    then $\ecpred {\ty \sapp v} {\kind'}$
  \end{itemize}
\end{definition}

\begin{lemma}[Error Correlation at Base Kind]
\label{lem:err-corr-at-T}
If $\ddck[\ty,,\kty,\con]$ and $\trans[\ty,,] \sapp \spair<B,\off> \kstepsto
  \spair<\off',r,p>$ then $\corrkl \ty r p$.
\end{lemma}

\begin{proof}
  By induction on the height of the second derivation.
\end{proof}

\begin{theorem}[Error Correlation]
\label{thm:err-corr}
If $\ddck[\ty,,\kind,\con]$ then $\ecpred \ty \kind$.
\end{theorem}

\begin{proof}
  By induction on the size of the kind $\kind$. For $\kind = \kty$, we
  use Lemma~\ref{lem:err-corr-at-T}.
\end{proof}

\trversion{
We conclude this section with a useful property of correlated representations
and PDs. If the PD reports no errors, then there are no syntactic
errors in the representation data structure {\it at any level}. 
We formally define
{\it clean} (error-free) values next, followed by the statement of the
property itself.

\begin{definition}[Clean Value]
$\noerr v$ iff exactly one of the following is true:
\begin{itemize}
\item $v = c$ and $c \neq \ierr$.
\item $v = \ifun f x e$.
\item $v = \ipair {v_1} {v_2}$ and $\noerr {v_1}$ and $\noerr {v_2}$.
\item $v = \iinl {v'}$ and $\noerr {v'}$.
\item $v = \iinr {v'}$ and $\noerr {v'}$.
\item $v = \iarr{v_1 \cdots v_n}$ and $\noerr {v_i}$ for $i=1\ldots n$.
\end{itemize}
\end{definition}

\begin{lemma}
  If $\corrkl \ty r p$ and $p.h.nerr = 0$ then $\noerr r$.
\end{lemma}

\begin{proof}
  By induction on the structure of r.
\end{proof}
}

\begin{corollary}
  If $\corrkl \ty r p$ and $p.h.nerr = 0$ then there are no syntactic
  or semantic errors in the representation data structure $r$.
\end{corollary}


%%% Local Variables: 
%%% mode: latex
%%% TeX-master: "semantics"
%%% End: 
\cut{
=======
\section{The Semantics of PADS/ML}
\label{sec:ddc}
% \emph{What is a good name for types with kind T?}

% {\em Should we use a different meta-variable for fomega contexts given
%   that they contain both types and variables?}

% \emph{Make change from single \fomega context to split context, for
%   whole section.}

% \emph{Do we want to show an example from pads/ml shown in DDC.
%   (e.g. name-value pairs?). It think it would be nice, but I don't
%   know that there is space.  }

\trversion{\em Modify a\_pd to a\_pd\_body, everywhere. This name is more
  appropriate.  Note that the new scheme for translating type
  variables will affect WF rules. a will be in D but a\_rep and
  a\_pd\_body could appear in a sigma

  Consider a new convention: for compute type, any references to type
  interepretations should be done with the type interpretation
  functions rather than being hard coded. Otherwise, type substitution gets
  messed up,e.g., when unfolding a recursive type. 
  For example, should be compute(e:[a]\_rep) instead of
  compute(e:a\_rep). Then, substitution on DDC types will burrow into
  the type annotation $\gs$ of compute types. To support, need new
  syntax for type annotations $\gs$ and need to explicitly translate
  annotation types into F-omega types.
  
}

% In this section, we discuss the formal semantics of \padsml{}. This
% semantics allows us to reason formally about the system, and prove
% properties that we feel should hold. In addition, it allows us to
% convey the core ideas of the system to other researchers, language
% implementers, and even users.

In this section, we introduce \ddc{}, a calculus of simple, orthogonal
type constructors, which serves to give a semantics
to the main features of \padsml.  \ddc{} is an extension
and revision of our previous work on 
\ddcold{}~\cite{fisher+:next700ddl}.  The main new feature is 
the ability to define functions from types to types, 
which are needed to model \padsml's polymorphic data types.
In the process of adding these new functions, which we call {\em type
abstractions} (as opposed to {\em value abstractions}, which are functions 
from values to types), we simplified our overall semantics
by making a couple of subtle technical changes.  For example, we were
able to eliminate the complicated ``contractiveness'' constraint from our
earlier work.

The main practical benefit of the calculus has been as a guide for our
implementation. Before working through the formal semantics, we
struggled to disentangle the invariants related to polymorphism. After
we had defined the calculus, we were able to implement type
abstractions as \ocaml{} functors in approximately a week.  We hope
the calculus will serve as a guide for implementations of {\sc PADS} in
other host languages.
In the remainder of this section, we give an overview of the calculus.
\appref{app:ddc-semantics} contains a complete formal specification.

% designed a semantic framework for understanding a range
% of data description languages and guiding us in developing future
% ones. At the core of this framework is a
% low-level data description language \ddcold{}, intended to capture the
% essential features used in data description.  Syntactically, \ddcold{}
% is a calculus of dependent types extended with a number of type
% constructors specific to the domain of data description. Semantically,
% however, rather than describing the properties of expressions,
% \ddcold{} types describe data sources. The base types describe atomic
% fragments of data, while the type constructors capture compound
% structures within a source.

% We we would like to use \ddcold{} to specify the semantics of
% \padsml{}. However, the polymorphic types of \padsml{} cannot be
% expressed in \ddcold{}. Therefore, we extend \ddcold{} with type
% abstractions $\ptylam{\ptyvar}{\kty}{\ty}$ and type application
% $\ptyapp{\ty}{\ty}$ in order to capture the semantics of polymorphic
% types. We refer to the extended calculus as \ddc{}.

% There are a number of different data description languages, as we
% discuss in more detail in our section on related work. Therefore, the
% utility of \ddc{} is not limited to explaining \padsml{}, but should
% be useful for any data description language with similar features.
% Indeed, in the remainder of this section, we focus exclusively on
% \ddc{}, its syntax, semantics, and meta-theory and leave the
% connection to \padsml{} implicit.

% A final note before we begin. The benefits of \ddc{} have not been
% hypothetical alone. All of the essential design of polymorphic types
% occured within the setting of \ddc{}. Using \ddc{} and \ocaml{} functors,
% the implementation of polymophic types in \padsml{} took about a week.

\subsection{\ddc{} Syntax}
\begin{figure}
{\small
\begin{bnf}
  \name{Kinds} \meta{\gk} \::= \kty \| \kty \-> \gk
                               \| \ity \-> \gk  \\
  \name{Types} \meta{\ty} \::= 
    \ptrue\| \pfalse \| \pbase{e} \| 
    \plam{\var}{\ity}{\ty} \| \papp{\ty}{e} \nlalt
    \psig x \ty \ty \| \psum \ty e \ty \|
    \pset x \ty e \nlalt
    \ptyvar \| \pmu{\ptyvar}{\gk}{\ty} \| 
    \ptylam{\ptyvar}{\kty}{\ty} \| \ptyapp{\ty}{\ty} \| ...
\end{bnf}
}
\caption{\ddc{} syntax, selected constructs}
\label{fig:ddc-syntax}
\end{figure}

\figref{fig:ddc-syntax} summarizes the syntax of the \ddc.
% A portion of the syntax of \ddc{} is shown in \figref{fig:ddc-syntax}.
% We leave out a number of primitives from \ddcold{} that do not play a
% critical role in \padsml{} and are unaffected by the extensions of
% \ddc{}. While they are included in \ddc{}, we leave them out of the
% discussion in this section, so as not to confuse the reader with
% unnecessary details. The interested reader is encouraged to consult
% our previous paper for a full discussion of the omitted primitives. In
% addition, all \ddc{}-related figures are included in their complete
% form in the appendix.
The interpretation of a type with kind $\kty$ is a parser that maps
data from an external form into an internal one.  
A type with kind $\kty \rightarrow \gk$ is a function mapping 
a parser to the interpretation of a type with kind $\gk$.
Finally, types with kind $\ity \rightarrow \gk$  map values
with host language type $\ity$ to the interpretation of
types with kind $\gk$.  For concreteness, we adopt \fomega{} as our
host language.

The atomic types include $\ptrue$, which consumes no input and reports
no errors,  and $\pfalse$, which consumes no input but reports an
error.   $\pbase{e}$ is a base type parameterized
by an expression $e$ from the host language. 
The \padsml{} type {\tt Pstring} is an example of such a base type.

Structured types include value abstraction $\plam{\var}{\ity}{\ty}$
and application $\papp{\ty}{e}$, which allow us to parameterize types
by host language values. 
\cut{
 Any type in the language may be parameterized by a value using
 value abstraction $\plam{\var}{\ity}{\ty}$.  
 We do not include the type $\sigma$ of the variable---the reader will
 have to reconstruct it from context. 
 If $\ty$ is such an abstraction,
 the parameter $\var$ may be instantiated
 using value application $\papp{\ty}{e}$. 
}
The dependent sum type, $\psig x \ty \ty$, describes a pair of values,
where the value of the first element of the pair can be referenced
when describing the second element.  Variation in a data source can be
described with the sum type $\psum \ty e \ty$, which deterministically
describes a data source that either matches the first type, or fails
to match the first branch but does match the second one.
%Intersections $\pand \ty \ty$
%describe data sources which can be described in two ways
%simultaneously. 
We specify semantic constraints over a data source
with type $\pset x \ty e$, which describes any value $x$ that satisfies the
description $\ty$ and the predicate $e$. Type variables $\ptyvar$ are
abstract descriptions; they are introduced by recursive types and type
abstractions. Recursive types $\pmu{\ptyvar}{\gk}{\ty}$ describe
recursive formats, like lists and trees. Type abstraction
$\ptylam{\ptyvar}{\kty}{\ty}$ and application $\ptyapp{\ty}{\ty}$
allow us to parameterize types by other types.  Type variables $\ptyvar$
always have kind $\kty$.

To specify the well-formedness of types, 
we use a kinding judgment of the form
$\ddck[\ty,\pctxt;\ctxt,\kind,\mcon]$,
where $\pctxt$ maps type variables 
to kinds and $\ctxt$ maps host language value variables to host language 
types. In our original work~\cite{fisher+:next700ddl}, these kinding rules
were somewhat unorthodox, but we have sinced simplifed them.  
Details appear in \appref{app:ddc-semantics}.

\subsection{\Implang{} Language}
\label{sec:host-lang}
The host language of \ddc{} is a straightforward 
extension of \fomega{} with recursion and a variety of useful constants
and operators.  
% We use \fomega{} 
% both to write the expressions that can appear within
% \ddc{} itself and to encode the parsing semantics of \ddc{}.
For reference, the grammar appears in \appref{app:host-lang}.
The constants include bitstrings $\data$; offsets $\off$, representing
locations in bitstrings; and error codes $\iok$, $\iecerr$, and
$\iecpc$, indicating success, success with errors, and failure,
respectively. We use the constant $\ierr$ to indicate a failed parse.
Because of its specific meaning, we forbid its use in user-specified
expressions appearing in \ddc{} types.
Our base types include the type $\invty$, the singleton type of the
constant $\ierr$, and types $\iecty$ and $\ioffty$, which classify
error codes and bit string offsets, respectively.

We extend the formal syntax with some syntactic sugar for use in the
rest of this section: anonymous functions $\ilam {\nrm x} \ity e$ for
$\ifun {\nrm f} {\nrm x} e$, with $f \not\in {\rm FV}(e)$; $\ispty$
for $\iprod \ioffty \ioffty$.  We often use pattern-matching syntax
for pairs in place of explicit projections, as in $\lampair{\codefont
  e}$ and $\ilet {\itup{\off,r,p}} e\; e'$.  Although we have no
formal records with named fields, we use a dot notation for commonly
occuring projections. For example, for a pair $\mathtt x$ of rep and
PD, we use $\codefont{x.rep}$ and $\codefont{x.pd}$ for the left and
right projections of $\codefont{x}$, respectively. Also, sums and
products are right-associative.  Finally, we only specify type
abstraction over terms and application when we feel it will clarify
the presentation. Otherwise, the polymorphism is implicit.  We also
omit the usual type and kind annotations on $\lambda$, with the
expectation the reader can construct them from context.  

The static semantics ($\stsem[e,{\pctxt;\ctxt},\ity]$), operational
semantics ($e \stepsto e'$), and type 
equality ($\ity \equiv \ity'$) are those of \fomega{} extended with
recursive functions and recursive types and are entirely standard.
See, for example, Pierce's text~\cite{pierce:tapl}.

\subsection{\ddc{} Semantics}
\label{sec:ddc-sem}

The primitives of \ddc{} each have three interpretations: two
datatypes in the host language---one for the data representation
itself and one for its parse descriptor---and a transformation from
bit strings into data and corresponding
meta-data.  We therefore specify the semantics of \ddc{} types using three semantic
functions, each of which precisely conveys a particular facet of the
meaning of a type.  The functions $\itsem[\cdot]$ and $\itpdsem[\cdot]$
describe the type of the data's in-memory representation and 
parse descriptor, respectively. The
function $\trans[\cdot,,]$ describes the parsing semantics of
\ddc{}.

% In our previous work~\cite{fisher+:next700ddl}, we discuss the
% semantics of \ddcold{} in detail. Here, we will focus only on those
% aspects of the semantics that are new or modified from the original
% presentation.
\trversion{
\begin{table}
  \begin{center}
    \renewcommand{\arraystretch}{1.35}
    \begin{tabular}{l l}
      $\ddck[\ty,{\pctxt;\ctxt},\kind,\mcon]$ & {\it \ddc{}-type
        kinding}\\
      $\itsem[\ty] = \ity$ & {\it representation types of \ddc{} types}\\
      $\itpdsem[\ty] = \ity$ & {\it pd types of \ddc{} types}\\
      $\trans[\ty,\ctxt,\gk] = e$   & {\it \ddc{}-type semantics} \\
      $\kTrans[\gk,\ty] = \ity$     & {\it parser type} \\
      $\ptyc \pctxt = \ctxt$     & {\it parser-type context }\\
      $\fotyc \pctxt = \pctxt$     & {\it \fomega version of poly. context }\\
      $\fortyc \pctxt = \pctxt$     & {\it Rep. type variables in $\fotyc \pctxt$ }\\
      $\fopdtyc \pctxt = \pctxt$     & {\it PD type variables in $\fotyc \pctxt$ }\\
      $\stsem[e,\pctxt;\ctxt,\ity]$ & {\it \fomega expression typing} \\
    \end{tabular}
    \caption{Translations and Judgments}
    \label{tab:judg-list}
  \end{center}
\end{table}

For reference, we provide in
\tblref{tab:judg-list} a listing of all the functions and judgments
defined in this section and a brief description of each.  
}

% \subsubsection{\ddc{} Kinding}
% \label{sec:ddc-kinding}

% \begin{figure*}[t]
% \small
% \fbox{$\ddck[\ty,\pctxt;\ctxt,\kind,\mcon]$}\\[-2ex]
% \[
% \infer[\text{Unit}]{
%     \ddck[\ptrue,\pctxt;\ctxt,\kty,\con]
%   }{\wfd {} {\fotyc \pctxt,\ctxt}}
% \quad 
% \infer[\text{Bottom}]{
%     \ddck[\pfalse,\pctxt;\ctxt,\kty,\con]
%   }{\wfd {} {\fotyc \pctxt, \ctxt}}
% \quad 
% \infer[\text{Const}]{
%     \ddck[\pbase{e},\pctxt;\ctxt,\kty,\con]
%   }{
%       \wfd {} {\fotyc \pctxt,\ctxt} &
%       \stsem[e,{\fotyc \pctxt,\ctxt},\ity] &
%       \Ikind(C) = {\ity \iarrowi \kty}
%   }
% \]

% \[
% \infer[\text{Abs}]{
%     \ddck[\plam{\var}{\ity}{\ty},
%          \pctxt;\ctxt,\ity \iarrowi \kind,\mcon]
%   }{
%     \ddck[\ty,\pctxt;\ectxt{\var{:}\ity},\kind,\mcon]
%   }
% \quad
% \infer[\text{App}]{
%   \ddck[\papp{\ty}{e},\pctxt;\ctxt,\gk,\mcon]
% }{
%   \ddck[\ty,\pctxt;\ctxt,\ity \iarrowi \gk,\mcon] &
%   \stsem[e,{\fotyc \pctxt,\ctxt},\ity]
% }
% \]

% \[
% \infer[\text{Prod}]{
%     \ddck[\psig{x}{\ty}{\ty'},\pctxt;\ctxt,\kty,\con]
%   }{       
%     \ddck[\ty,\pctxt;\ctxt,\kty,\mcon] &
%     \ddck[\ty',\pctxt;
%           \ectxt {x{:}\iprod {\itsem[\ty]} 
%               {\itpdsem[\ty]}},
%           \kty,\mcon']
%   }
% \quad
% \infer[\text{Sum}]{
%     \ddck[\psum{\ty}{e}{\ty'},\pctxt;\ctxt,\kty,\con]
%   }{
%     \ddck[\ty,\pctxt;\ctxt,\kty,\mcon] & \ddck[\ty',\pctxt;\ctxt,\kty,\mcon'] 
%   }
% \quad
%   \infer[\text{Con}]{
%     \ddck[\pset x \ty e,\pctxt;\ctxt,\kty,\con]
%   }{ 
%     \ddck[\ty,\pctxt;\ctxt,\kty,\mcon] & 
%     \stsem[e,
%      {\fotyc \pctxt,
%     \ectxt{x{:}\iprod{\itsem[\ty]} 
%       {\itpdsem[\ty]}}},\iboolty]
%   }
% \]

% \[
%   \infer[\text{TyVar}]{
%     \ddck[\ptyvar,{\pctxt;\ctxt},\kty,\ncon]
%   }{\wfd {}{\fotyc \pctxt, \ctxt} \quad \tyvar{:}\kty \in \pctxt}
% \quad
%   \infer[\text{Rec}]{
%     \ddck[\pmu \ptyvar \kty \ty,\pctxt;\ctxt,\kty,\con]
%   }{
%     \ddck[\ty,{\pctxt,\ptyvar{:}\kty;\ctxt},\kty,\con]
%   }
% \quad
% \infer[\text{TyAbs}]{
%     \ddck[\ptylam{\tyvar}{\kty}{\ty},
%          \pctxt;\ctxt,\kty \iarrowi \kind,\mcon]
%   }{
%     \ddck[\ty,{\pctxt,\tyvar{:}\kty;\ctxt},\kind,\mcon]
%   }
% \quad
% \infer[\text{TyApp}]{
%   \ddck[\ptyapp{\ty_1}{\ty_2},\pctxt;\ctxt,\gk,\mcon]
% }{
%   \ddck[\ty_1,\pctxt;\ctxt,\kty \iarrowi \gk,\mcon] &
%   \ddck[\ty_2,\pctxt;\ctxt,\kty,\mcon]
% }
% \]
% \caption{Partial listing of \ddc{} kinding rules}
% \label{fig:ddc-kinding}
% \end{figure*}

% The kinding judgment defined in \figref{fig:ddc-kinding} determines
% well-formed \ddc{} types, assigning kind $\kty$ to actual (i.e. unparameterized) types, kinds
% $\ity \iarrowi \kind$ to value abstractions, and kinds $\kty \iarrowi
% \kind$ to type abstractions. We use two contexts to express our
% kinding judgment:
% \[
% \begin{array}{ll}
% \ctxt  & \mathrel{::=} \cdot \bnfalt \ctxt,{\var{:}\ity}\\
% \pctxt  & \mathrel{::=} \cdot \bnfalt \pctxt,\tyvar{:}\kty
% \end{array}
% \]

% Context $\Gamma$ is a finite partial map that binds expression
% variables to their types.
% Context $\pctxt$ is a finite partial map that binds type
% variables to their kinds. We provide the following translations from
% type contexts $\pctxt$ to \fomega type-variable contexts $\pctxt$.
% \begin{align*}
% \fortyc {\cdot} &= \cdot &
% \fortyc {\pctxt,\ptyvar{:}\kty} &= \fortyc \pctxt, \ptyvar_\repname
% {\mathrel{::}} \kty \\
% \fopdtyc {\cdot} &= \cdot &
% \fopdtyc {\pctxt,\ptyvar{:}\kty} &= \fopdtyc \pctxt, \ptyvar_\pdbname
% {\mathrel{::}} \kty
% \end{align*}
% The translation $\fortyc \pctxt$ and $\fopdtyc \pctxt$ respectively map type
% variables of \ddc{} to the corresponding representation-type and parse
% descriptor-type variables in \fomega. Translation
% $\fotyc \pctxt$ simply a combines of the two ($\fotyc \pctxt = \fortyc \pctxt, \fopdtyc \pctxt$).  These
% translations are used when checking the well-formedness of contexts
% $\ctxt$ with open type variables.

% An essential difference in the kinding rules between \ddcold{} and
% \ddc{} is that we replace the context $M$ of \ddcold{}, which mapped
% recursive-type variables to their definitions, with a simpler context
% $\pctxt$ which merely assigns a kind (always $\kty$) to open type
% variables. Recursive-type variables are now treated as abstract, just
% like the type variables arising from type abstractions.
% Correspondingly, the rule for type variables (TyVar) now has an
% entirely standard form, and the premise of the rule for recursive types
% (Rec) is now identical to the premise of the rule for type abstractions
% (TyAbs). The rule for type application (TyApp) is almost standard,
% except that parameters are restricted to having kind $\kty$. 

\begin{figure}
\fbox{$\itsem[\ty] = \ity$}
\[
\begin{array}{lcl} 
\itsem[\ptrue] & = & \iunitty \\
\itsem[\pfalse] & = & \invty \\
\itsem[\pbase{e}] & = & \isum {\Irty(C)} \invty   \\
\itsem[\plam{\var}{\ity}{\ty}] & = & \itsem[\ty] \\
\itsem[\papp \ty e] & = & \itsem[\ty] \\
\itsem[\psig \var {\ty_1} {\ty_2}]  & = & \iprod {\itsem[\ty_1]} {\itsem[\ty_2]}    \\
\itsem[\psum {\ty_1} e {\ty_2}]     & = & \isum {\itsem[\ty_1]} {\itsem[\ty_2]} \\
\itsem[\pset x \ty e] & = & \isum {\itsem[\ty]}{\itsem[\ty]}\\
\itsem[\ptyvar] & = & \ptyvar_\repname \\
\itsem[\pmu{\ptyvar}{\gk}{\ty}] & = & \imu{\ptyvar_\repname}{\itsem[\ty]} \\
\itsem[\lambda \ptyvar.\ty]       & = & \lambda \ptyvar_\repname.\itsem[\ty] \\
\itsem[\ty_1 \ty_2]              & = & \itsem[\ty_1] \itsem[\ty_2] \\
\end{array}
\]
\caption{Representation type translation, selected constructs}
\label{fig:rep-tys}
\end{figure}


\paragraph*{\ddc{} representation types.}
\label{sec:intty-sem}
In Figure~\ref{fig:rep-tys}, we present the representation type
of selected \ddc{} primitives. While the primitives are
dependent types, the mapping to the \implang{} language 
erases the dependency because the \implang{} language 
does not have dependent types. This involves erasing all host language
expressions that appear in types as well as 
value abstractions and applications.
A type variable $\ptyvar$ in \ddc{} is mapped to a corresponding
type variable $\ptyvar_\repname$ in \fomega{}.
Recursive types generate recursive representation types with the type
variable named appropriately. Polymorphic types and their application 
become \fomega{} type constructors and type application, respectively.

\begin{figure}
\fbox{$\itpdsem[\ty] = \ity$}
\[ 
\begin{array}{lcl} 
%% %% example: \ua.(int * a) + None
%% %%          pd = \ua.pd_hdr  * ((pd_hdr * ([int]_pd * [a]_pd)) + [None]_pd)
%% %%             = \ua.pd_hdr  * ((pd_hdr * ([int]_pd * a)) + [None]_pd)
\itpdsem[\ptrue] & = & \ipty \iunitty \\                                                  
\itpdsem[\pfalse] & = & \ipty \iunitty \\                                                  
\itpdsem[\pbase{e}] & = & \ipty \iunitty\\
\itpdsem[\plam \var \ity \ty] & = & \itpdsem[\ty] \\
\itpdsem[\papp \ty e] & = & \itpdsem[\ty] \\
\itpdsem[\psig \var {\ty_1} {\ty_2}] & = & 
               \ipty {\iprod {\itpdsem[\ty_1]} {\itpdsem[\ty_2]}} \\
\itpdsem[\psum {\ty_1} e {\ty_2}] & = & 
               \ipty {(\isum {\itpdsem[\ty_1]} {\itpdsem[\ty_2]})} \\
\itpdsem[\pset x \ty e] & = & \ipty {\itpdsem[\ty]} \\
\itpdsem[\ptyvar] & = & \ipty{\ptyvar_\pdbname} \\
\itpdsem[\pmu \ptyvar \kty \ty] & = & 
  \ipty{\imu{\ptyvar_\pdbname}{\itpdsem[\ty]}} \\
\itpdsem[\lambda \ptyvar.\ty]      
     & = & \lambda \ptyvar_\pdbname.\itpdsem[\ty] \\
\itpdsem[\ty_1 \ty_2]            & = & \itpdsem[\ty_1] \itpdsemstrip[\ty_2] \\
\end{array}
\]

\fbox{$\itpdsemstrip[\ty] = \ity$}

\[
\begin{array}{lcl} 
\itpdsemstrip[\ty] & = & \ity \ \ \mbox{where}\ \itpdsem[\ty] \equiv \ipty{\ity}
\end{array}
\]
\caption{Parse-descriptor type translation, selected constructs}
\label{fig:pd-tys}
\end{figure}

\paragraph*{\ddc{} parse descriptor types.}
\figref{fig:pd-tys} gives the types of the parse descriptors
corresponding to selected \ddc{} types.  The translation reveals that
all parse descriptors share a common structure, consisting of two
components, a header and a body.  The header reports on the
corresponding representation as a whole. It stores the number of
errors encountered during parsing, an error code indicating the degree
of success of the parse---success, success with errors, or
failure---and the span of data (location in the source)  
described by the descriptor.  To be precise, the type
of the header ($\tyface{pd\_hdr}$) is $\iintty \iprodi \iecty \iprodi
\ispty$. The body contains parse descriptors for the subcomponents of
the representation. For types without subcomponents, we use $\iunitty$
as the body type.  As with the representation types, dependency is
uniformaly erased.
% For types with base kind, the corresponding parse
% descriptor will always have a header and body. 

% In \figref{fig:pd-tys}, we give the types of the parse descriptors
% corresponding to each of the selected \ddc{} types. The majority of
% the types shown are products of the type $\tyface{pd\_hdr}$ and
% another host-language type. This common form reflects the fact that all
% parse descriptors consist of two components, a {\it header} and a {\it
%   body}.  The header describes the representation as a whole. It
% provides the number of errors encountered during parsing, an error
% code indicating the degree of success of the parse -- success, success
% with errors, or failure -- and the span of data described by the
% descriptor.  Formally, the type of the header ($\tyface{pd\_hdr}$) is
% $\iintty \iprodi \iecty \iprodi \ispty$. 

% The parse-descriptor body consists of parse descriptors for the
% subcomponents of the representation. For types without subcomponents,
% we use $\iunitty$ as the body type.

Like other types, \ddc{} type variables $\ptyvar$ are translated into 
a pair of header and a body.  The body has abstract type 
$\ptyvar_\pdbname$.
This translation makes it possible for polymorphic parsing code to examine the
header of a PD, even though it does not know the \ddc{} type it is parsing.
\ddc{} abstractions are translated into \fomega\ type constructors that
abstract the body of the PD (as opposed to the entire PD)
and \ddc{} applications are translated into \fomega\ type applications
where the argument type is the PD body type.

% PD-type constructors are
% parameterized over the PD \emph{body} type, rather than over the PD
% type itself. Correspondingly, type application $\papp {\ty_1}{\ty_2}$ applies
% the PD interpretation of $\ty_1$ to the body 

% This common shape enables us to write
% polymorphic functions that treat parse descriptors in a generic
% mannner. Indeed, the parsing semantics of types rely on this
% structure, as discussed in \secref{sec:parse-sem}.

% The PD types without the common shape are those corresponding to
% abstractions and applications. Value abstractions and applications are
% translated in an identical manner for PD types as for representation
% types.  Type abstractions and applications, however, are translated in
% a subtly different manner. As with the rep-type interpretation, type
% abstractions correspond to parse decriptor-type constructors in
% \fomega. The major difference is that PD-type constructors are
% parameterized over the PD \emph{body} type, rather than over the PD
% type itself. Correspondingly, type application $\papp {\ty_1}{\ty_2}$ applies
% the PD interpretation of $\ty_1$ to the body portion of the PD
% interpretation of $\ty2$ (specified formally with the $\itbdsem[\ty]$
% function). Similarly, type variables are translated to tuples of a
% header and a type variable, rather than to a type variable alone. 

% The reason for parameterization over PD-body types (rather than
% parse-descriptor types) in the translation scheme will be explained in
% the next section, once we have discussed the parsing semantics of type
% abstractions and applications. Note, however, that an important result
% of this design is that the PD interpretation is not defined for all
% types. The problem lies with the interpretation of type application.
% It requires that $\itbdsem[\ty_2]$ be defined, which, in turn,
% requires that $\itpdsem[\ty_2] \equiv \ipty{\ity}$, for some $\ity$.
% Yet, this requirement is not mebt by some types; for example, $\lambda
% \ptyvar.\ty$.


\begin{figure}
\small
\fbox{$\kTrans[\gk,\ty] = \ity$} 
    
\begin{align*}
  &\kTrans[\kty,\ty] & = & \quad \extdom * \offdom \iarrowi \offdom * \itsem[\ty] * \itpdsem[\ty]
   \\
   &\kTrans[\ity \iarrowi \gk,\ty] & = & \quad \ity \iarrowi \kTrans[\gk,\ty e],
   \; \mbox{for any e}.
   \\
   &\kTrans[\kty \iarrowi \gk,\ty] & = & \quad 
      \forall\tyvar_\repname.\forall\tyvar_\pdbname.
         \kTrans[\kty,\tyvar] \iarrowi \kTrans[\gk,\ty \tyvar] \\
         & & & \quad (\ptyvar_\repname, \ptyvar_\pdbname \not \in \ftv \kind \cup
         \ftv \ty)
\end{align*}  
  \caption{\Implang{} language types for parsing functions}
  \label{fig:parser-types}
\end{figure}

\begin{figure}
\small
\fbox{$\trans[\ty,\ctxt,\gk] = e$} 

\[
\begin{array}{l}
  %% None 
\trans[\ptrue,\ctxt,\kty] =
  \lampair{\spair<\off,\newrep{unit}{},\newpd{unit}{\off}>}
\\[3pt] %\\
%% False 
\trans[\pfalse,,] =
  \lampair{\spair<\off,\newrep {bottom}{},\newpd {bottom}{\off}>}
\\[3pt] %\\ 
%% Const 
\trans[\pbase{e},\ctxt,\kty] =
  \lampair{\iapp {\iapp {\Iimp(C)} (e)} {\itup {\idata,\off}}}
\\[3pt] %\\
%% Abs 
\trans[\plam{\var}{\ity}{\ty},,] =
   \sfn{\nrm\var}{\ity}{\trans[\ty,\ectxt{\var{:}\ity},\kind]}
\\[3pt] %\\
%% App 
\trans[\papp{\ty}{e},\ctxt,\gk] =
  \trans[\ty,,] \sapp e  
\\[3pt]
%% Prod 
%\begin{array}{l}
\trans[\psig{x}{\ty}{\ty'},\ctxt,\kty] = \\
  \begin{array}{l}  
    \lampair{} \\
    \quad  \ilet {\spair<\off',r,p>} 
    {{\trans[\ty,,]} \sapp \spair<\idata,\off>} \\
    \quad  \ilet x {\ictup{r,p}}\\
    \quad  \ilet {\spair<\off'',r',p'>} 
    {{\trans[\ty',,]} \sapp \spair<\idata,\off'>} \\
    \quad \spair<\off'',\newrep {\gS}{r,r'},\newpd {\gS}{p,p'}>
  \end{array}  
%\end{array}
\\[3pt]
%% Sum 
%\begin{array}{l}
  \trans[\psum{\ty}{e}{\ty'},,] = \\
  \begin{array}{l}  
  \lampair{} \\
  \quad \ilet {\itup{\off',r,p}}{\trans[\ty,,] \sapp \spair<\idata,\off>} \\
  \quad \iif {\pdok p} \; \ithen {
    \def \r {\newrep {+left}{r}}
    \def \p {\newpd {+left}{p}}
    \spair<\off',\r,\p>} \\
  \quad \ielse {\ilet {\itup{\off',r,p}}{\trans[\ty',,] \sapp \spair<\idata,\off>}} \\
  \quad 
  {  % begin scope
    \def \r {\newrep {+right}{r}}
    \def \p {\newpd {+right}{p}}
    %% 
    \spair<\off',\r,\p>
  }\\ % end scope
  \end{array}
\\[3pt]
%\quad
%% Set 
  \trans[\pset{x}{\ty}{e},\ctxt,\kty] = \\
  \begin{array}{l}  
    \lampair{} \\
    \quad \ilet {\itup{\off',r,p}}{\trans[\ty,,] \sapp \spair<\idata,\off>} \\
    \quad \ilet x {\ictup{r,p}}\\
    \quad \ilet c e \\
    \quad \spair<\off',\newrep {con} {c,r},\newpd {con} {c,p}>
  \end{array}
\\[3pt]
%% Var
\trans[\ptyvar,,] = \codefont{f_\ptyvar}
\\[3pt]
%% Mu
\trans[\pmu \ptyvar \gk \ty,,] = \\
  \begin{array}{l}
  \ifun {f_\ptyvar} {\itup{\data,\off} :
    \ioffty * \pmu \ptyvar \gk \itsem[\ty] 
                    * (\ipty{\pmu \ptyvar \kty \itpdsem[\ty]}) } {}\\
  \quad \ilet {\itup{\off',r,p}} 
   {\trans[\ty,,] \iappi \ictup{\data,\off}} \\ 
  \quad \ictup{\off',\iroll{r}{\pmu \ptyvar \gk \itsem[\ty]},
     (p.h,\iroll{p}{\pmu \ptyvar \kty \itpdsem[\ty]})}
%}}
  \end{array}  
\\[3pt]
%% lambda \alpha
\trans[\lambda\tyvar . \ty,,] = %\\
%  \begin{array}{l}
    \Lambda \tyvar_\repname. 
    \Lambda \tyvar_\pdbname. \lambda \codefont{f_\ptyvar}. \trans[\ty,,]
%  \end{array}  
\\[3pt]
%% t1 t2
\trans[\ty_1 \ty_2,,] = 
    \trans[\ty_1,,]\; [\itsem[\ty_2]]\; [\itpdsemstrip[\ty_2]]\; \trans[\ty_2,,]
\\
\end{array}
\]
%\caption{\ddc{} Semantics (cont.)}
\caption{\ddc{} parsing semantics, selected constructs}
\label{fig:ddc-sem-sum}
\end{figure}

% \begin{figure}
% \small
% \begin{itemize}
% \renewcommand{\labelitemi}{}

% %\item %[Unit:]
% \item $\ifun {R_{unit}} \iuval \iuval$
% \item $\ifun {P_{unit}} \off {\itup{\itup{0,\iok,\ipair \off \off},\iuval}}$

% %\item %[Bottom:]
% \item $\ifun {R_{bottom}} \iuval \ierr$
% \item $\ifun {P_{bottom}} \off ((1,\iecpc,\ipair \off \off),())$

% \item %[Pair:]
% \item $\ifun {R_{\gS}} {\ipair {r_1} {r_2}} {\itup {\codefont{r_1,r_2}}}$
% \item $\ifun{H_{\gS}} {\ictup{h_1,h_2}}{}$ \\
%   $\begin{array}{l}
%     \ilet {nerr} {\codefont{pos \itup{{h_1}.{nerr}} + pos \itup{{h_2}.{nerr}}}}\\
%     \ilet {ec} {\iif {\codefont{h_2.ec} = \iecpc}\; \ithen {\iecpc}\\
%     \quad \ielse {\codefont{max\_ec} \iappi \codefont{h_1.ec} \iappi \codefont{h_2.ec}}} \\
%     \ilet {sp} {\ictup{h_1.sp.begin, h_2.sp.end}} \\
%     \quad \ictup {nerr,ec,sp}
%   \end{array}$

% \item $\ifun {P_{\gS}} {\ictup{p_1, p_2}} {\ictup {H_{\gS} \itup{p_1.h,p_2.h},\itup{p_1,p_2}}}$

% \end{itemize}
% \caption{Selected Constructor Functions. 
% The type of PD headers is $\iintty
%   \iprodi \iecty \iprodi \ispty$. 
%   We refer to the projections using
%   dot notation as $\codefont{nerr}$, $\codefont{ec}$ and
%   $\codefont{sp}$, respectively. A span is a pair of offsets, referred
%   to as $\codefont{begin}$ and $\codefont{end}$, respectively.  The full collection of such constructor functions appears in \appref{app:asst-functions}.}
% \label{fig:cons-funs}
% %\caption{Constructor Functions (cont.)}
% \end{figure}


\paragraph*{\ddc{} parsing semantics.}
\label{sec:parse-sem}
The parsing semantics of a type $\tau$ with kind $\kty$ is a function that
transforms some amount of input into a pair of a representation and a
parse descriptor, the types of which are determined by $\tau$.  The
parsing semantics for types with higher kind are functions that
construct parsers, or functions that construct functions that
construct parsers, \etc{} \figref{fig:parser-types} specifies
the host-language types of the functions generated from well-kinded
\ddc{} types.

For each (unparameterized) type, the input to the corresponding parser
is a bit string to parse and an offset at which to begin parsing.  
The output is a new offset,
a representation of the parsed data, and a parse descriptor.

For any type, there are three steps to parsing: parse the
subcomponents of the type (if any), assemble the resultant
representation, and tabulate meta-data based on subcomponent meta-data
(if any). For the sake of clarity, we have factored the latter two
steps into separate representation and PD constructor functions which
we define for each type. For some types, we additionally factor the PD
header construction into a separate function. For example, the
representation and PD constructors for $\ptrue$ are $\newrepf {unit}$
and $\newpdf {unit}$, respectively, and the header constructor for
products is ${\codefont{H_{\gS}}}$.  We have also factored out some commonly
occuring code into auxiliary functions.  These constructors and functions
appear in \appref{app:asst-functions}.

The PD constructors determine the error code and calculate the error
count.  There are three possible error codes: $\iok$, $\iecerr$, and
$\iecpc$, corresponding to the three possible results of a parse: it
can succeed, parsing the data without errors; it can succeed, but
discover errors in the process; or, it can find an unrecoverable error
and fail.  The error count is determined by subcomponent error counts
and any errors associated directly with the type itself.

In \figref{fig:ddc-sem-sum}, we specify the parsing semantics of
selected portion of \ddc{}. We explain the interpretations of select
types, from which the interpretation of the remaining types may be
understood. The full semantics appears in \appref{app:ddc-semantics}.
A dependent sum parses the data according to the first type, binding
the resulting representation and PD to $x$ before parsing the
remaining data according to the second type. It then bundles the
results using the dependent sum constructor functions.

A type variable translates to an expression variable whose name
corresponds directly to the name of the type variable. These
expression variables are bound in the interpretations of recursive
types and type abstractions. We interpret each recursive type as a
recursive function whose name corresponds to the name of the recursive
type variable. For clarity, we annotate the recursive function with its
type.

We interpret type abstraction as a function over other parsing
functions. Because those parsing functions can have
arbitrary \ddc{} types (of kind $\kty$), the interpretation must be a
polymorphic function, parameterized by the representation and PD-body
type of the \ddc{} type parameter.  For clarity, we present
this type parameterization explicitly.  Type application $\papp
{\ty_1}{\ty_2}$ simply becomes the application of the interpretation
of $\ty_1$ to the representation-type, PD-type, and parsing
interpretations of $\ty_2$.

% Now, we can explain why the PD and parsing interpretations
% of type abstraction and application involve the PD-body interpretation
% rather than just the PD interpretation. Notice that the constructor
% functions for dependent sums (and for many other types as well) are
% polymorphic. No matter the types of the subcomponent types $\ty$ and
% $\ty'$, we use the same functions $\newrepf \Sigma$ and $\newpdf
% \Sigma$. The PD constructor is not fully polymorphic, though, in that
% it assumes that the parse descriptors given to it have the familiar PD
% shape of header and body.  Therefore, in order to judge the type of
% this function, we must be sure that the PD types of the subcomponents
% $\ty$ and $\ty'$ satisfy this constraint.

% What happens, then, when we have $\ptyvar$ as a subcomponent? We must
% ensure that its PD type has the appropriate shape. To do so, we
% specify the translation of type variables explicitly as a product of a
% header and a type variable for the body, rather than simply as a type
% variable for the whole PD type. Therefore, all parameterization
% related to PD types must be over the PD-body type, rather than the PD
% type as a whole.

\subsection{Meta-theory}
\label{sec:meta-theory}
Our key theoretical result is that the various semantic functions we
have defined are coherent.  In particular, we show that for any 
well-kinded \ddc{} type $\tau$, the corresponding parser is
well typed, returning a pair of the corresponding representation and
parse descriptor.  

To prove this theorem, we first need to characterize the shape of
parse descriptors for arbitrary \ddc{} types.   
Such a characterization is necessary because we require for type
application that the argument parse descriptor be a pair of a PD
header and an (arbitrary) body.
The most straightforward characterization of this property is
too weak to prove directly, so we instead characterize it
as a logical relation in Definition~\ref{def:pd-props}.
Lemma~\ref{lemma:pd-log-rel} establishes that the logical relation
holds of all well-formed \ddc{} types by induction on kinding
derivations, and the desired characterization follows as a corollary.

\begin{definition}
\label{def:pd-props}
\begin{itemize}
\item $\hhpred \ty \kty$ iff $\ \exists\,\ity$ s.t. $\itpdsem[\ty] \equiv
  {\ipty \ity}$.
\item $\hhpred \ty {\kty \iarrowi \kind}$ iff $\ \exists\,\ity$
  s.t. $\itpdsem[\ty] \equiv \ity$ and whenever $\hhpred
  {\ty'}{\kty}$, we have $\hhpred {\papp \ty {\ty'}}{\kind}$.
\item $\hhpred \ty {\ity \iarrowi \kind}$ iff $\ \exists\,\ity'$
  s.t. $\itpdsem[\ty] \equiv \ity'$ and $\hhpred{\papp \ty e}{\kind}$
  for any expression $e$.
\end{itemize}
\end{definition}
\cut{
We can now prove the following lemma by induction on kinding derivations
(with $\ty[\ty_1/\ptyvar_1,...,\ty_n/\ptyvar_n]$ denoting simultaneous
capture-avoiding substitution defined in the standard way):
}
\begin{lemma}
\label{lemma:pd-log-rel}
   If
   $\ddck[\ty,{\pctxt,\ptyvar_1{:}\kty,...,\ptyvar_n{:}\kty;\ctxt},\kind,{}]$ 
   and $\hhpred {\ty_i} \kty$, then \\ 
   $\hhpred {\ty[\ty_1/\ptyvar_1,...,\ty_n/\ptyvar_n]} \kind$.
\end{lemma}

\begin{lemma}
\label{lemma:pd-props}
  \begin{itemize}
  \item If $\ddck[\ty,\pctxt;\ctxt,\kind,{}]$ then $\exists
     \ity.\itpdsem[\ty] = \ity$.
   \item If $\ddck[\ty,\pctxt;\ctxt,\kty,{}]$ then $\exists
     \ity.\itpdsem[\ty] \equiv \ipty \ity$.
  \end{itemize}
\end{lemma}


\trversion{
\begin{definition}
$\pda \ptyvar = \ipty \ptyvar$
\end{definition}

\begin{lemma}[Types of Constructors]
\label{lem:types-of-constructors}
\begin{itemize}
\item $\newrepf {unit} : \iarrow \iunitty \iunitty$
\item $\newpdf  {unit} : \iarrow \ioffty {\ipty \iunitty}$
\item $\newrepf {bottom} : \iarrow \iunitty \invty$
\item $\newpdf  {bottom} : \iarrow \ioffty {\ipty \iunitty}$
\item $\newrepf {\gS} : \forall \ga,\gb.\iarrow {\iprod \ga \gb} {\iprod \ga \gb}$
\item $\newpdf {\gS} : \forall \ga,\gb. 
  \iarrow {\iprod {\pda \ga} {\pda \gb}}
  {\pda {(\pda \ga \iprodi \pda \gb)}}
$
\item $\newrepf {+left} : \forall \ga.\forall \gb.\iarrow \ga 
                            {\isum \ga \gb}$
\item $\newrepf {+right} : \forall \ga.\forall \gb.\iarrow \gb {\isum \ga \gb}$
\item $\newpdf {+left} : \forall \ga, \gb.\iarrow {\pda \ga} 
  {\ipty {(\isum {\pda \ga}{\pda \gb})}}$
\item $\newpdf {+right} :\forall  \ga, \gb. \iarrow {\pda \gb} 
                            {\ipty {(\isum {\pda \ga} {\pda \gb})}}$
\item $\newrepf {\&} : \forall \ga,\gb.\iarrow {\iprod \ga \gb} {\iprod \ga \gb}$
\item $\newpdf {\&} : 
\forall \ga,\gb.
  \pda \ga \iprodi
  \pda \gb \iarrowi 
         {\ipty {(\pda \ga \iprodi \pda \gb)}}
$.
\item $\newrepf {con} : \forall \ga.\iprod \iboolty \ga 
  \iarrowi {\isum \ga \ga}$
\item $\newpdf {con} :\forall  \ga. \iprod \iboolty \iarrow {\pda \ga} {\ipty {\pda \ga}}$
\item $\newrepf {seq\_init} : \forall \ga.\iarrow \iunitty {\iintty \iprodi \iseq \ga}$
\item $\newpdf {seq\_init} : \forall \ga. \iarrow \ioffty {\iapty {\pda\ga}}$
\item $\newrepf {seq} : \forall \ga.\iarrow
  {(\iintty \iprodi \iseq \ga) \iprodi \ga}
  {\iintty \iprodi \iseq \ga}$
\item $\newpdf {seq} :\forall  {\ga_{elt}},{\ga_{sep}}. 
  (\iapty {\pda {\ga_{elt}}}) \iprodi
  \pda {\ga_{sep}} \iprodi 
  \pda {\ga_{elt}} \iarrowi \\
  \iapty {\pda {\ga_{elt}}}$
\item $\newrepf {compute} : \forall \ga.\iarrow \ga \ga$
\item $\newpdf {compute} : \iarrow \ioffty {\ipty \iunitty}$
\item $\newrepf {absorb} : \forall \ga.\iarrow {\pda \ga} {\isum
    \iunitty \invty}$
\item $\newpdf {absorb} :\forall  \ga. \iarrow {\pda \ga} {\ipty
    \iunitty}$
\item $\newrepf {scan} : \forall \ga.\iarrow \ga {\isum \ga \invty}$
\item $\newpdf {scan} :\forall  \ga. \iarrow {\iprod \iintty {\pda \ga}}
  {\ipty {(\isum {\iprod \iintty {\pda \ga}} \iunitty)}}$
\item $\newrepf {scan\_err} : \forall \ga.\iarrow \iunitty {\isum \ga \invty}$
\item $\newpdf {scan\_err} :\forall  \ga. \iarrow \ioffty
  {\ipty {(\isum {\iprod \iintty \ga} \iunitty)}}$
\end{itemize}  
\end{lemma}

\begin{proof}
  By typing rules of \fomega.
\end{proof}
}
With this lemma, we can establish the type correctness of the
generated parsers. We prove the theorem using a more general induction
hypothesis that applies to open types.
This hypothesis must account for the fact
that any free type variables in a \ddc{} type $\ty$ will become free
function variables in $\trans[\ty,,]$. To that end, we define the function $\ptyc \pctxt$, 
which maps type-variable contexts
$\pctxt$ to typing contexts $\ctxt$:
\vskip -1.5ex
{\small
\[
  \ptyc{\cdot} = \cdot \qquad
  \ptyc{\pctxt,\ptyvar{:}\kty} = \ptyc \pctxt,\codefont{f_\ptyvar}{:}\kTrans[\kty,\ptyvar]
\]
}

\begin{lemma}[Type Correctness Lemma]
\label{thm:type-correctness}
  If $\ddck[\ty,{\pctxt;\ctxt},\gk,{}]$ then
  $\stsem[{\trans[\ty,,]},{\fotyc \pctxt, \ctxt,\ptyc \pctxt},
            {\kTrans[\kind,\ty]}]$.
\end{lemma}

\begin{proof}
  By induction on the height of the kinding derivation.
\end{proof}

\begin{theorem}[Type Correctness of Closed Types]
  If $\ddck[\ty,,\gk,\con]$ then
  $\stsem[{\trans[\ty,,]},,\kTrans[\kind,\ty]]$.  
\end{theorem}

A practical implication of this theorem is that
it is sufficient to check data descriptions (i.e. \ddc{} types) for
well-formedness to ensure that the generated types and
functions are well formed. This property is sorely lacking in many
common implementations of Lex and YACC, for which users must examine
generated code to debug compile-time errors in
specifications.


%%% Local Variables: 
%%% mode: latex
%%% TeX-master: "paper"
%%% End: 
}