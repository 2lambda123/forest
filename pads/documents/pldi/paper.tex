\documentclass{sig-alternate}
%\documentclass{acm-sigplan}

\usepackage{times,url}
\usepackage{code} 
\newcommand{\cut}[1]{}

\newcommand{\appref}[1]{Appendix~\ref{#1}}
\newcommand{\secref}[1]{Section~\ref{#1}}
\newcommand{\tblref}[1]{Table~\ref{#1}}
\newcommand{\figref}[1]{Figure~\ref{#1}}
\newcommand{\listingref}[1]{Listing~\ref{#1}}
%\newcommand{\pref}[1]{{page~\pageref{#1}}}

\newcommand{\eg}{{\em e.g.}}
\newcommand{\cf}{{\em cf.}}
\newcommand{\ie}{{\em i.e.}}
\newcommand{\etc}{{\em etc.\/}}
\newcommand{\naive}{na\"{\i}ve}
\newcommand{\role}{r\^{o}le}
\newcommand{\forte}{{fort\'{e}\/}}
\newcommand{\appr}{\~{}}

\newcommand{\bftt}[1]{{\ttfamily\bfseries{}#1}}
\newcommand{\kw}[1]{\bftt {#1}}
\newcommand{\Pthen}{\kw{Pthen}}
\newcommand{\pads}{\textsc{pads}}
\newcommand{\padsl}{\textsc{padsl}}
\newcommand{\padst}{\textsc{pads/t}}
\newcommand{\datatype}{\textsc{PADS/T}}
%\newcommand{\datatype}{\textsc{DataType}}
\newcommand{\C}{\textsc{C}}
\newcommand{\perl}{\textsc{Perl}}
\newcommand{\ml}{\textsc{ml}}
\newcommand{\sml}{\textsc{sml}}
\newcommand{\smlnj}{\textsc{sml/nj}}
\newcommand{\java}{\textsc{java}}
\newcommand{\ddl}{\textsc{ddl}}
\newcommand{\xml}{\textsc{xml}}
\newcommand{\datascript}{\textsc{DataScript}}
\newcommand{\packettypes}{\textsc{PacketTypes}}
\newcommand{\erlang}{\textsc{Erlang}}

\newcommand{\Core}{Ad hoc}
\newcommand{\core}{ad hoc}
\newcommand{\pvalue}{\core{} value}
\newcommand{\ppat}{\core{} pattern}
\newcommand{\ptype}{\core{} type}

\newcommand{\padsc}{\textsc{pads}/\C{}}
\newcommand{\padsml}{\textsc{pads}/\ml{}}

\newcommand{\dibbler}{Sirius}
\newcommand{\ningaui}{Altair}
\newcommand{\darkstar}{Regulus}

\newcommand{\pdgood}{{\tt G}}
\newcommand{\pdbad}{{\tt B}}
\newcommand{\pdnest}{{\tt N}}
\newcommand{\pdsem}{{\tt S}}
\newcommand{\ptypes}{T}
\newcommand{\patreadpd}[2]{{\tt #1<<#2>>}}
\newcommand{\btm}{\cd{BOT}}


\newcommand{\lsem}{{[\![}}
\newcommand{\rsem}{{]\!]}}


\newcommand{\figHeight}[4]{\begin{figure}[tb]
	\centerline{
	            \epsfig{file=#1,height=#4}}
	\caption{#2}
	\label{#3}
	\end{figure}}

%% Environment for typesetting BNF grammars. Uses display math mode.
\newenvironment{bnf}
     {%% local command definitions:
        %% BNF definition symbol
      \def\->{\rightarrow}
%%      \def\::={{::=} &}
      \def\::={\bnfdef &}
      \def\|{\bnfalt}
      \newcommand{\name}[1]{\text{##1}}
        %% non-terminal
      \newcommand{\nont}[1]{{##1}}
      \newcommand{\meta}[1]{& ##1 &}
      \newcommand{\descr}[1]{& \text{// ##1}}
      \newcommand{\opt}[1]{ [##1] }
      \newcommand{\opnon}[1]{\opt{\nont{##1}}}
      \newcommand{\none}{\epsilon}
      \newcommand{\nwln}{\\ &&&}
      \newcommand{\nlalt}{\\ && \| &}
      \[\begin{array}{lrlll}
     }
     {\end{array}\]}

\newcommand{\mcd}[1]{\mathtt{#1}}
\newcommand{\ppair}[3]{#1{:}#2 \mathrel{**} #3}
\newcommand{\parray}[3]{#1\;\mcd{Parray}(#2,#3)}
\newcommand{\pset}[3]{\{#1{:}#2\,|\,#3\}}
\newcommand{\pstream}[1]{#1\;\mcd{stream}}
\newcommand{\precord}[1]{\{\{#1\}\}}

\newcommand{\dibbler}{Sirius}
\newcommand{\ningaui}{Altair}
\newcommand{\darkstar}{Regulus}

\begin{document}
\title{PADS:\\ A domain-specific language for processing ad hoc data}
\numberofauthors{2}

\author{
\alignauthor Kathleen Fisher\\
       \affaddr{AT\&T Labs Research}\\
       \affaddr{180 Park Ave., E244}\\
       \affaddr{Florham Park, NJ}\\
       \email{kfisher@research.att.com}
\alignauthor Robert Gruber\titlenote{Work carried out while at AT\&T
                                     Labs Research.}\\
       \affaddr{Google}\\
       \affaddr{1600 Amphitheatre Pkwy}\\
       \affaddr{Mountain View, CA}\\
       \email{gruber@google.com}
}

\date{\today}
\maketitle
\begin{abstract}
\pads{} is a declarative data description language that allows data
analysts to describe both the physical layout of ad hoc data sources
and semantic properties of that data.  From such descriptions, the
\pads{} compiler generates libraries and tools for manipulating the
data, including parsing routines, statistical profiling tools,
translation programs to produce well-behaved
formats such as \xml{} or those required for loading relational
databases, and tools for running XQueries over raw \pads{} data sources.
The descriptions are concise enough to serve as ``living'' documentation
while flexible enough to describe most of the ASCII, binary, and
Cobol formats that we have seen in practice.  The generated parsing
library provides for robust, application-specific error handling.
\end{abstract}

\section{Introduction}
Vast amounts of useful data are stored and processed in ad hoc formats.
Traditional databases and \xml{} systems provide rich infrastructure
for processing well-behaved data, but are of little help when dealing with ad hoc formats.
Examples that we face at AT\&T include call detail data~\cite{hancock-toplas}, 
%compiler traces~\cite{chilimbi},
web server logs~\cite{wpp}, 
netflows capturing internet traffic~\cite{netflow}, 
log files characterizing IP backbone resource utilization,
wire formats for legacy telecommunication billing systems, 
\etc{}
Such data may simply require processing before it can be loaded into a
data management system, or it may be too large or too transient to
make such loading cost effective.

Processing ad hoc data can be challenging for a variety of
reasons. First, ad hoc data typically arrives ``as is'': the analysts
who receive it can only say ``thank you,'' not request a more convenient format. 
Second, documentation for the format may not exist at all, or it may be
out of date.  A common phenomenon is for a field in a
data source to fall into disuse.  After a while, a new piece of
information becomes interesting, but compatibility issues prevent 
data suppliers from modifying the
shape of their data, so instead they hijack the unused field, often
failing to update the documentation in the process.

Third, such data frequently contain errors, for a variety of
reasons: malfunctioning equipment, race conditions on log
entry~\cite{wpp}, non-standard values to indicate ``no data
available,'' human error in entering data, unexpected data
values, \etc{} The appropriate response to such errors depends on the application. Some applications require the data to be error free: 
if an error is detected, processing needs to stop immediately and a human
must be alerted.  Other applications can repair the data, while still
others can simply discard erroneous or unexpected values.  
For some applications,
errors in the data can be the most interesting part  because
they can signal where two systems are failing to communicate.

A fourth challenge is that ad hoc data sources can be high volume:
AT\&T's call-detail stream contains roughly 300~million calls per day
requiring approximately 7GBs of storage space. Although this data is
eventually archived in a database, analysts mine it profitably before
such archiving~\cite{kdd98,kdd99}. More challenging, the \ningaui{} project at AT\&T
accumulates billing data at a rate
of 250-300GB/day, with occasional spurts of 750GBs/day. Netflow data
arrives from Cisco routers at rates over a gigabyte per
second~\cite{gigascope}! Such volumes mean it must be possible to
process the data without loading it all into memory at once.

Finally, before anything can be done with an ad hoc data source,
someone has to produce a suitable parser for it.
Today, people tend to use \C{} or \perl{} for this task.
Unfortunately, writing
parsers this way is tedious and error-prone, complicated by the lack
of documentation, convoluted encodings designed to save space, 
the need to produce efficient code,
and the need to handle errors robustly to avoid corrupting down-stream data.
Moreover, the parser writers' hard-won understanding of the data
ends up embedded in parsing code, making long-term maintenance
difficult for the original writers and sharing the knowledge with
others nearly impossible.

The \pads{} system makes life easier for data analysts by addressing
each of these concerns.\footnote{
  \pads{} is short for Processing Ad hoc Data Sources.
}
It provides a declarative data description
language that permits analysts to describe the physical layout of
their data, \textit{as it is}.  The language also permits analysts to
describe expected semantic properties of their data so that deviations can
be flagged as errors. The intent is to allow analysts to capture in a
\pads{} description all that they know about a given data source
and to provide the analysts with a library of useful routines in exchange. 


\pads{} descriptions are concise enough to
serve as documentation and flexible enough to describe most of
the data formats we have seen in practice, including ASCII, binary,
Cobol, and mixed data formats.  The fact that useful software
artifacts are generated from the descriptions provides strong
incentive for keeping the descriptions current, allowing them to serve
as living documentation.  

Given a \pads{} description, the \pads{} compiler produces customizable \C{} libraries
and tools for parsing, manipulating, and summarizing the data. 
The core \C{} library includes functions for reading the data, writing it 
back out in its original form, writing it into a canonical \xml{} form, pretty printing
it in forms suitable for loading into a relational database, and accumulating  
statistical properties.  An auxiliary library provides 
an instance of the data API for Galax~\cite{galax,galaxmanual}, an implementation of XQuery.  This 
library allows users to query data with a \pads{} description as if the data were
in \xml{} without having to convert to \xml{}.  In addition to these libraries,
the \pads{} system provides wrappers that build tools to 
summarize the data, format it,  or convert it to \xml{}.

The declarative nature of \pads{} descriptions facilitates the
insertion of error handling code.
The generated parsing code checks all possible error cases: system
errors related to the input file, buffer, or socket; syntax errors
related to deviations in the physical format; and semantic errors in
which the data violates user constraints.  Because these checks appear
only in generated code, they do not clutter the high-level declarative
description of the data source.
The result of a parse is a pair consisting of a canonical in-memory
representation of the data and a parse descriptor. The parse
descriptor precisely characterizes both the syntactic and the semantic
errors that occurred during parsing.  This structure allows analysts
to choose how to respond to errors in application-specific ways.  

With such huge datasets, performance is critical. The \pads{} system addresses performance in a number of ways.
First, we compile the \pads{} description rather than simply interpret it to reduce run-time overhead.
Second, the generated parser provides  multiple entry points, so the data consumer can choose 
the appropriate level of granularity for reading the data into memory to accommodate very large data sources.
Finally, we parameterize library functions by \textit{masks}, which allow data analysts to 
choose which semantic conditions to check at run-time, permitting them to specify
all known properties in the source description without forcing all users of that 
description to pay the run-time cost of checking them.  

Given the importance of the problem, it is perhaps surprising
that more tools do not exist to solve it.  \xml{} and relational databases
only help with data already in well-behaved formats.  Lex and Yacc are both
over- and under- kill.  Overkill because the division into a lexer and a context free grammar is not necessary for many ad hoc data sources, and under-kill in that such systems require the user to build in-memory representations manually,
support only ASCII sources, and don't provide extra tools.  ASN.1~\cite{asn} and related systems~\cite{asdl} allow the user to specify an in-memory representation and generate an on-disk format, but this doesn't help when given a particular on-disk format.  
Existing ad hoc description languages~\cite{gpce02,sigcomm00,erlang} are steps 
in the right direction, but they focus on binary, error-free data and they do not provide auxiliary tools.

In building the \pads{} system, we faced challenges on two fronts:
designing the data description language and
crafting the generated libraries and tools.
In the rest of the paper, 
we describe each of these designs.
We also give examples to show how useful they have been at AT\&T.


\section{Example data sources}

Before discussing the \pads{} design, we describe a selection of data
sources, focusing on those we use as running examples.
\figref{figure:data-sources} summarizes some of the sources we have
worked with.  They include ASCII, binary, and Cobol data formats, with
both fixed and variable-width records, ranging in size from
relatively small files through network applications which process over
a gigabyte per second.  Common errors include undocumented data,
corrupted data, missing data, and multiple missing-value
representations.

\begin{figure*}
\begin{center}
\begin{tabular}{|l|l|l|l|l|}
\hline
Name \& Use   &  Representation              &Size           & Common Errors \\ \hline\hline
\textbf{Web server logs (CLF)}: &  Fixed-column ASCII records & $\leq$12GB/week & Race conditions on log entry\\ 
Measuring web workloads                   &                             &                             & Unexpected values\\ \hline
\textbf{AT\&T provisioning data (\dibbler{})}: & Variable-width ASCII records & 2.2GB/week & Unexpected values \\ 
Monitoring service activation &                              &            & Corrupted data feeds \\ \hline
Call detail: Fraud detection  &  Fixed-width binary records  &\appr{}7GB/day &  Undocumented data\\  \hline 
AT\&T billing data (\ningaui{}): & Various Cobol data formats  & \appr{}4000 files/day, & Unexpected values\\ 
Monitoring billing process   &                             & 250-300GB/day    & Corrupted data feeds \\ \hline
IP backbone data (\darkstar{})  & ASCII  & $\ge$ 15 sources  & Multiple missing-value representations \\
Monitoring network performance  &        & \appr{}15 GB/day              & Undocumented data \\ \hline
Netflow       & Data-dependent number of     & $\ge$1Gigabit/second  & Missed packets\\ 
Monitoring network performance              &  fixed-width binary record   &                       & \\ \hline

\end{tabular}
\caption{Selected ad hoc data sources.  We will use the \textbf{bold} data sources in our examples. }
\label{figure:data-sources}
\end{center}
\end{figure*}


\subsection{Common Log Format}
Web servers use the Common Log Format (CLF) to log client
requests~\cite{wpp}.  Researchers use such logs to measure
properties of web workloads and to evaluate protocol changes
by "replaying" the user activity recorded in the log.
This ASCII format consists of a sequence of
records, each of which has seven fields: the host name or IP address
of the client making the request, the account associated with the
request on the client side, the name the user provided for
authentication, the time of the request, the actual request, the
\textsc{http} response code, and the number of bytes returned as a
result of the request.  The actual request has three parts: the
request method (\eg, \texttt{GET}, \texttt{PUT}), the requested
\textsc{uri}, and the protocol version.  In addition, the second and
third fields are often recorded only as a '-' character to indicate
the server did not record the actual data.  \figref{figure:clf-records}
shows a couple of typical records.


\subsection{Provisioning data}
In the telecommunications industry, the term \textit{provisioning} refers to the steps necessary to convert an order for phone service into the actual 
service.  
To track AT\&T's provisioning process, the \dibbler{} project compiles
weekly summaries of the state of certain types of phone service orders.  
These ASCII summaries store the summary date and one record per order.
Each order record contains a header followed by a sequence of events.
The header has 13 pipe separated fields: the order number, AT\&T's
internal order number, the order version, four different telephone
numbers associated with the order, the zip code of the order, a
billing identifier, the order type, a measure of the complexity of the
order, an unused field, and the source of the order data.  Many of
these fields are optional, in which case nothing appears between the
pipe characters.  The billing identifier may not be available at the
time of processing, in which case the system generates a unique
identifier, and prefixes this value with the string ``no\_ii'' to
indicate the number was generated. The event sequence represents the
various states a service order goes through; it is represented as a
new-line terminated, pipe separated list of state, timestamp pairs.
There are over 400 distinct states that an order may go through during
provisioning.  The sequence is sorted in order of increasing timestamps. \figref{figure:dibbler-records} shows a small example of
this format.
%156 different states for one order
%-rw-r--r--    1 angusm   dibbler   2187472314 Jun  9  2003 /fs/dibblerd/tlf/data/out_sum.stream
%2171.364u 31.379s 40:41.54 90.2% 0+0k 2+0io 2pf+0w
%53 had trailing t or } after zip code


It may be apparent from these paragraphs that English is a poor
language for describing data formats!


\section{Pads Design}

A \pads{} description specifies the physical layout and 
semantic properties of an ad hoc data source. 
The language provides a type-based model:
basic types describe atomic data such as integers, strings, dates, \etc{}, while
structured types describe compound data built from simpler pieces.
\suppressfloats

The \pads{} library provides a collection of broadly useful base types.
Examples include
8-bit unsigned integers (\cd{Puint8}),
32-bit integers (\cd{Pint32}),
dates (\cd{Pdate}), strings (\cd{Pstring}), and IP addresses (\cd{Pip}).
Semantic conditions for such base types include checking that the
resulting number fits in the indicated space, \ie, 16-bits for
\cd{Pint16}.
By themselves, these base types do not provide sufficient information to allow parsing
because they do not specify how the data is coded, \ie{}, in ASCII, EBCDIC, or binary.  
To resolve this ambiguity, \pads{} uses the \textit{ambient} coding, which the programmer can set.  By default,
\pads{} uses ASCII.  To specify a particular coding, the description writer can select
base types which indicate the coding to use.  Examples of such types include
ASCII 32-bit integers (\cd{Pa_int32}), binary bytes (\cd{Pb_int8}), and
EBCDIC characters (\cd{Pe_char}).  
In addition to these types,  users can define their own base types to specify more
specialized forms of atomic data.  

To describe more complex data, \pads{} provides a collection of 
structured types loosely based on \C{}'s type structure.
In particular, \pads{} has 
\kw{Pstruct}s, \kw{Punion}s, and \kw{Parray}s to describe
record-like structures, alternatives, and sequences, respectively.
\kw{Penum}s describe a fixed collection of literals, while \kw{Popt}s 
provide convenient syntax for optional data.
Each of these
types can have an associated predicate that indicates whether a
value calculated from the physical specification is indeed a legal
value for the type.  For example, a predicate might require that two
fields of a \kw{Pstruct} are related or that the elements
of a sequence are in increasing order.  Programmers can specify such
predicates using \pads{} expressions and functions, 
written using a \C{}-like syntax.
Finally, \pads{} \kw{Ptypedef}s can be used
to define new types that add further constraints to existing types.

\pads{} types can be parameterized by values.
This mechanism
serves both to reduce the number of base types and to permit the
format and properties of later portions of the data to depend upon earlier portions.
For example, 
the base type \cd{Puint16_FW(:3:)} specifies an unsigned two byte integer
physically represented by exactly three characters, while the type
\cd{Pstring(:' ':)} 
describes a string terminated by a space.  Parameters can be 
used with compound types to specify the size of an array or which
branch of a union should be taken.




\figref{figure:wsl} gives the \pads{} description for CLF web server logs, 
while \figref{figure:dibbler} gives the description for the \dibbler{} 
provisioning data.  We will use these two examples to illustrate various 
features of the \pads{} language.  In \pads{} descriptions, types are declared before they are used, so the type that describes the totality of the data source appears at the bottom of the description.

\kw{Pstruct}s describe fixed sequences of data with unrelated types.
In the CLF description, the type declaration for
\cd{version_t} illustrates a simple \kw{Pstruct}. It starts with a 
string literal that matches the constant \cd{HTTP/} in the data source.  It 
then has two unsigned integers recording the major and minor version numbers
separated by the literal character \kw{'.'}.  \pads{} supports character, string,
and regular expression literals, which are interpreted with the ambient character 
encoding. The type \cd{request_t} 
similarly describes the request portion of a CLF record.  In addition
to physical format information, this \kw{Pstruct} includes a semantic constraint
on the \cd{version} field.  Specifically, it requires that obsolete methods
\cd{LINK} and \cd{UNLINK} occur only under HTTP/1.1.  This constraint illustrates
the use of predicate functions and the fact 
that earlier fields are in scope during the processing of later fields, as the 
constraint
refers to both the \cd{meth} and \cd{version} fields in the \kw{Pstruct}.

\kw{Punion}s describe variation in the data source.  For example, the
\cd{client_t} type in the CLF description indicates that the first field 
in a CLF record can be either an IP address or a hostname.  During parsing, 
the branches of a \kw{Punion} are tried in order; the first branch that 
parses without error is taken.  The \cd{auth_id_t} type illustrates the use
of a constraint: the branch \cd{unauthorized} is chosen only if the parsed
character is a dash.  \pads{} also supports a \textit{switched} union that uses a selection expression to determine the branch to parse.  Typically, this expression depends upon already-parsed portions of the data source.

\pads{} provides \kw{Parray}s to describe varying-length sequences of data all 
with the same type.  The \cd{eventSeq_t} declaration in the \dibbler{} data description
uses a \kw{Parray} to characterize the sequence of events an
order goes through during processing.  This declaration indicates that the elements
in the sequence have type \cd{event_t}.  It also specifies that the elements will
be separated by vertical bars, and that the sequence will be terminated by 
an end-of-record marker (\kw{Peor}).  In general, \pads{} provides a rich collection
of array-termination conditions: reaching a maximum size, finding a terminating
literal (including end-of-record and end-of-source),
or satisfying a user-supplied predicate over the already-parsed portion of 
the \kw{Parray}.  Finally, this type declaration includes a \kw{Pwhere} clause
to specify that the sequence of timestamps must be in sorted order.
It uses the \kw{Pforall} construct to express this constraint.
In general, the body of a \kw{Pwhere} clause can be any boolean expression.
In such a context for arrays, the pseudo-variable \cd{elts} is bound to the in-memory representation of the sequence and \cd{length} to its length.

Returning to the CLF description in \figref{figure:wsl}, the \kw{Penum} type \cd{method_t} describes
a collection of data literals.  During parsing, \pads{} interprets these
constants using the ambient character encoding.  The \kw{Ptypedef} 
\cd{response_t} describes possible server response codes in CLF data by adding
the constraint that the three-digit integer must be between 100 and 600.

The \cd{order_header_t} type in the \dibbler{} data description contains several
anonymous uses of the \kw{Popt} type.  This type is syntactic sugar for a 
stylized use of a \kw{Punion} with two branches: the first with the indicated type, and the second with the ``void'' type, which  
always matches but never consumes any input.

\setlength{\floatsep}{0pt}
\setlength{\dblfloatsep}{0pt}
\setcounter{totalnumber}{4}
\setcounter{dbltopnumber}{2}
\begin{figure*}[t!]
\begin{small}
\begin{center}
\begin{code}
\chapter{Common log format example}
\section{pads description}
\section{c header file}
\end{code}
\vskip -2ex
\caption{Tiny example of web server log data.}
\label{figure:clf-records}
\end{center}
\end{small}
\end{figure*}

\begin{figure*}
\begin{small}
\begin{center}
\begin{verbatim}
0|1005022800
9152|9152|1|9735551212|0||9085551212|07988|no_ii152272|EDTF_6|0|APRL1|DUO|10|1000295291
9153|9153|1|0|0|0|0||152268|LOC_6|0|FRDW1|DUO|LOC_CRTE|1001476800|LOC_OS_10|1001649601
\end{verbatim}
\caption{Tiny example of \dibbler{} provisioning data.}
\label{figure:dibbler-records}
\end{center}
\end{small}
\end{figure*}


\begin{figure}
\begin{code}
\kw{Punion} client\_t \{
  Pip       ip;      /- 135.207.23.32
  Phostname host;    /- www.research.att.com
\};
\mbox{}
\kw{Punion} auth\_id\_t \{
  Pchar unauthorized : unauthorized == '-';
  Pstring(:' ':) id;
\};
\mbox{}
\kw{Pstruct} version\_t \{
  "HTTP/";
  Puint8 major; '.';
  Puint8 minor;
\};
\mbox{}
\kw{Penum} method\_t \{
    GET,    PUT,  POST,  HEAD,
    DELETE, LINK, UNLINK
\};
\mbox{}
bool chkVersion(version\_t v, method\_t m) \{
  \kw{if} ((v.major == 1) && (v.minor == 1)) \kw{return} true;
  \kw{if} ((m == LINK) || (m == UNLINK)) \kw{return} false;
  \kw{return} true;
\};
\mbox{}
\kw{Pstruct} request\_t \{
  '\\"';   method\_t       meth;
  ' ';    Pstring(:' ':) req\_uri;
  ' ';    version\_t      version :
                  chkVersion(version, meth);
  '\\"';
\};
\mbox{}
\kw{Ptypedef} Puint16\_FW(:3:) response\_t :
         response\_t x => \{ 100 <= x && x < 600\};
\mbox{}
\kw{Precord} \kw{Pstruct} entry\_t \{
         client\_t       client;
   ' ';  auth\_id\_t      remoteID;
   ' ';  auth\_id\_t      auth;
   " ["; Pdate(:']':)   date;
   "] "; request\_t      request;
   ' ';  response\_t     response;
   ' ';  Puint32        length;
\};
\mbox{}
\kw{Psource} \kw{Parray} clt\_t \{
  entry\_t [];
\}
\end{code}

\caption{\pads{} description for web server log data.}
\label{figure:wsl}
\end{figure}

\begin{figure}
\begin{code}
\kw{Precord} \kw{Pstruct} summary\_header\_t \{
  "0|";
  Puint32       tstamp;
\};
\mbox{}
\kw{Pstruct} no\_ramp\_t \{
  "no\_ii";
  Puint64 id;
\};
\mbox{}
\kw{Punion} dib\_ramp\_t \{
  Pint64     ramp;
  no\_ramp\_t  genRamp;
\};
\mbox{}
\kw{Pstruct} order\_header\_t \{
       Puint32             order\_num;
 '|';  Puint32             att\_order\_num;
 '|';  Puint32             ord\_version;
 '|';  \kw{Popt} pn\_t           service\_tn;
 '|';  \kw{Popt} pn\_t           billing\_tn;
 '|';  \kw{Popt} pn\_t           nlp\_service\_tn;
 '|';  \kw{Popt} pn\_t           nlp\_billing\_tn;
 '|';  \kw{Popt} Pzip           zip\_code;
 '|';  dib\_ramp\_t          ramp;
 '|';  Pstring(:'|':)      order\_type;
 '|';  Puint32             order\_details;
 '|';  Pstring(:'|':)      unused;
 '|';  Pstring(:'|':)      stream;
 '|';
\};
\mbox{}
\kw{Pstruct} event\_t \{
  Pstring(:'|':) state;   '|';
  Puint32        tstamp;
\};
\mbox{}
\kw{Parray} eventSeq \{
  event\_t[] : \kw{Psep}('|') && \kw{Pterm}(\kw{Peor});
\} \kw{Pwhere} \{
  \kw{Pforall} (i \kw{Pin} [0..length-2] :
           (elts[i].tstamp <= elts[i+1].tstamp));
\};
\mbox{}
\kw{Precord} \kw{Pstruct} entry\_t \{
  order\_header\_t  header;
  eventSeq        events;
\};
\mbox{}
\kw{Parray} entries\_t \{
  entry\_t[];
\};
\mbox{}
\kw{Psource} \kw{Pstruct} out\_sum\{
  summary\_header\_t  h;
  entries\_t         es;
\};
\end{code}

\caption{\pads{} description for \dibbler{} provisioning data.}
\label{figure:dibbler}
\end{figure}


Finally, the \kw{Precord} and \kw{Psource} annotations deserve comment.  The first
indicates that the annotated type constitutes a record,
while the second means that the type constitutes the totality of a data source.  
The notion of a record varies depending upon the data encoding.  
ASCII data typically uses new-line characters to delimit 
records, binary sources tend to have fixed-width records, while 
COBOL sources usually store the length of each record before the actual data.
\pads{} supports each of these encodings of records and allows users to define
their own encodings.  By default, \pads{} assumes records are new-line terminated.
Before parsing, however, the user can direct \pads{} to use a different record
definition.

More information about the \pads{} language may be found in the
\pads{} manual~\cite{padsmanual}.

\section{Generated library}
From a description, the \pads{} compiler generates a \C{} library
for parsing and manipulating the associated data source.  We chose \C{}
as the target language for pragmatic reasons: there were 
libraries that made building the compiler and run-time libraries easier,
our target users are comfortable with \C{}, and it can serve 
as a lingua franca in that essentially all languages have provisions for 
calling \C{} libraries.  Nothing about the \pads{} language mandates compiling
to \C{}, however, and we envision eventually building alternate bindings.

From each type in a \pads{} description, the compiler generates 
\begin{itemize}
\setlength{\itemsep}{0ex plus0.2ex}
\item an in-memory representation, 
\item a mask, which allows users to customize generated functions,
\item a parse descriptor, which describes syntactic and
semantic errors detected during parsing, 
\item parsing and printing functions, and 
\item a broad collection of utility functions.
\end{itemize}
%
\setcounter{totalnumber}{1}
\setcounter{dbltopnumber}{1}
\renewcommand{\topfraction}{0.85}
\renewcommand{\textfraction}{0.1}
\renewcommand{\floatpagefraction}{0.75}
\begin{figure*}
\begin{tiny}
\begin{figure*}
\begin{code}
\kw{typedef} \kw{struct} \{
  Pbase\_m compoundLevel;   // Struct-level controls, eg., check Pwhere clause
  order\_header\_t\_m h;
  eventSeq\_t\_m events;
\} entry\_t\_m;
\mbox{}
\kw{typedef} \kw{struct} \{
  Pflags\_t pstate;         // Normal, Partial, or Panicking 
  Puint32 nerr;            // Number of detected errors.
  PerrCode\_t errCode;      // Error code of first detected error
  Ploc\_t loc;              // Location of first error
  order\_header\_t\_pd h;     // Nested header information
  eventSeq\_t\_pd events;    // Nested event sequence information
\} entry\_t\_pd;
\mbox{}
\kw{typedef} \kw{struct} \{
  order\_header\_t h;
  eventSeq\_t events;
\} entry\_t;
\end{code}

\begin{code}
/* Core parsing library */
Perror\_t entry\_t\_read (P\_t *pads,entry\_t\_m *m,entry\_t\_pd *pd,entry\_t *rep);
ssize\_t entry\_t\_write2io (P\_t *pads,Sfio\_t *io,entry\_t\_pd *pd,entry\_t *rep);
\end{code}

\begin{code}
/* Selected utility functions */
\kw{void} entry\_t\_m\_init (P\_t *pads,entry\_t\_m *mask,Pbase\_m baseMask);
\kw{int} entry\_t\_verify (entry\_t *rep);
\end{code}

\begin{code}
/* Selected accumulator functions */
Perror\_t entry\_t\_acc\_init (P\_t *pads,entry\_t\_acc *acc);
Perror\_t entry\_t\_acc\_add (P\_t *pads,entry\_t\_acc *acc,entry\_t\_pd *pd,entry\_t *rep);
Perror\_t entry\_t\_acc\_report (P\_t *pads,\kw{char} \kw{const} *prefix,\kw{char} \kw{const} *what,
                             \kw{int} nst,entry\_t\_acc *acc);
\end{code}

\begin{code}
/* Formatting */
ssize\_t entry\_t\_fmt2io (P\_t *pads,Sfio\_t *io,\kw{int} *requestedOut,
                        \kw{char} \kw{const} *delims,entry\_t\_m *m,entry\_t\_pd *pd,entry\_t *rep);
\end{code}

\begin{code}
/* Conversion to XML */
ssize\_t entry\_t\_write\_xml\_2io (P\_t *pads,Sfio\_t *io,entry\_t\_pd *pd,
                               entry\_t *rep,\kw{char} \kw{const} *tag,\kw{int} indent);
\end{code}

\begin{code}
/* Galax Data API */
PDCI\_node\_t *entry\_t\_node\_new (PDCI\_node\_t *parent,\kw{char} \kw{const} *name,\kw{void} *m,
                               \kw{void} *pd,\kw{void} *rep,\kw{char} \kw{const} *kind,\kw{char} \kw{const} *whatfn);
PDCI\_node\_t *entry\_t\_node\_kthChild (PDCI\_node\_t *self,PDCI\_childIndex\_t idx);
\end{code}

\label{figure:library}
\caption{Selected portions of generated library for \texttt{entry\_t}
  declaration from Dibbler data description.}
\end{figure*}
\caption{Selected portions of the library generated for the \texttt{entry\_t}
  declaration from \dibbler{} data description.}
\label{figure:library}
\end{tiny}
\end{figure*}
To give a feeling for the library that \pads{} generates, 
\figref{figure:library} includes selected portions of the generated 
library for the \dibbler{} \cd{entry_t} declaration.

The \C{} declarations for the in-memory representation, the mask, 
and the parse descriptor all share the structure of the \pads{}
type declaration.  The mapping to \C{} for each is straightforward: 
\kw{Pstruct}s map to \C{} structs with appropriately mapped fields, 
\kw{Punion}s map to tagged unions coded as \C{} structs with a tag field 
and an embedded 
union, \kw{Parray}s map to a \C{} struct with a length field and an 
embedded sequence, \kw{Penums} map to \C{} enumerations, \kw{Poptions} 
to tagged unions, and \kw{Ptypedef}s to \C{} typedefs.  Masks include
auxiliary fields to control behavior at the level of a structured
type, and parse descriptors include extra fields to record the 
state of the parse, the number of detected errors, 
the error code of the first detected error, and the location of that error.

The parser takes a mask as an argument and returns an
in-memory representation and a parse descriptor.  
The mask allows the user to specify 
which constraints the parser should check and which portions of the
in-memory representation it should fill in.  This control allows the
description-writer to specify all known constraints about the data
without worrying about the run-time cost of verifying potentially
expensive constraints for time-critical applications.

Appropriate error-handling can be as important as processing
error-free data.  The parse descriptor marks which portions of the
data contain errors and characterizes the detected errors.
Depending upon the nature of the errors and the desired application,
programmers can take the appropriate action: halting the program,
discarding parts of the data, or repairing the errors.
If the mask requests
that a data item be verified and set, and if the parse descriptor
indicates no error, then the in-memory representation satisfies the
semantic constraints on the data.

Because we generate a parsing function for each type in a \pads{} description,
we support multiple-entry point parsing, which allows us to 
accommodate larger-scale data.
For a small file, programmers can define a \pads{} type that describes
the entire file and use that type's parsing function to read the whole
file with one call.  For larger-scale data, programmers can sequence
calls to parsing functions that read manageable portions of the file,
\eg{}, reading a record at a time in a loop.  The parsing code generated
for \kw{Parrays} allows users to choose between reading the entire array
at once or reading it one element at a time, again to support parsing
and processing very large data sources.

The ratio of the size of the data description to the size of the generated code gives a rough measure of the leverage of the
declarative description.  For the 
68~line \dibbler{} data description, the compiler yields a 1432~\texttt{.h} file
and a 6471~\texttt{.c} file.  This expansion comes from the extensive error checking in the generated parser and the number of generated utility functions.

We discuss details of the generated library in the following section
as we describe its uses.

\section{Pads in practice}
Because the \pads{} language is declarative, we can leverage \pads{}
descriptions to build additional tools besides the core parsing and printing
library.  As a result, \pads{} provides direct support for a number
of different uses: computing with the data, developing a high-level
understanding of the data through statistical summaries, converting the data into a more standard form, and querying.  In this section, we give a survey of some of these uses.

\subsection{General computation}
\subsubsection{Normalizing data}
\label{subsec:general}
We start by showing in \figref{figure:dibbler-filter} a simple use of 
the core library to clean and normalize \dibbler{} data. After initializing
the \pads{} library handle and opening the data source, the code sets
the mask to check all conditions in the \dibbler{} description except the
sorting of the timestamps.  We have omitted from the figure the code to read and write the header. 
The code then echoes error records to one file and cleaned ones to another.
The raw data has two different representations of unavailable phone numbers:
simply omitting the number altogether, which corresponds to the \cd{NONE}
branch of the \kw{Popt}, or having the value \cd{0} in the data.  
The function \cd{cnvPhoneNumbers} unifies these two representations 
by converting the zeroes into \cd{NONE}s.  The function \cd{entry_t_verify}
ensures that our computation hasn't broken any of the semantic properties
of the in-memory representation of the data.
\begin{figure}[t]
\begin{small}
\begin{center}
\begin{centercode}
P\_t                  *p;
entry\_t              entry;
entry\_t\_pd           pd;
entry\_t\_m            mask;
    ...
P\_open(&p, 0, 0);
P\_io\_fopen(p, "sirius/data/2004.11.11");
    ...
entry\_t\_m\_init(p, &mask, P\_CheckAndSet);
mask.events.compoundLevel = P\_Set;
    ...
\kw{while} (!P\_io\_at\_eof(p)) \{
  entry\_t\_read(p, &mask, &pd, &entry);
  \kw{if} (pd.nerr > 0) \{
    entry\_t\_write2io(p, ERR\_FILE, &pd, &entry);
  \} \kw{else} \{
    cnvPhoneNumbers(&entry);
    \kw{if} (entry\_t\_verify(&entry)) \{
      entry\_t\_write2io(p, CLEAN\_FILE, &pd, &entry);
    \} \kw{else} \{
      error(2, "Data transform failed.");
    \}
  \}
\}
\end{centercode}

\caption{Code fragment to filter and normalize \dibbler{} data.}
\label{figure:dibbler-filter}
\end{center}
\end{small}
\end{figure}

\subsubsection{Hancock streams}
Another programmatic use of the core library is to define  
Hancock streams.  Hancock is a domain-specific language for
building persistent profiles of entities described in transaction 
streams~\cite{hancock-toplas}.  Data analysts at AT\&T use Hancock 
programs over streams of call-detail records to build 
profiles of phone numbers, capturing behaviors like the frequency with which a given number makes a phone call.  They uses these profiles
to detect fraud. 
Defining the input streams turned out to be one of the most
difficult parts of writing Hancock programs because the parsing
code had to handle erroneous values and decipher complex encodings.  
Also, the analysts wanted to have one stream description for their
many applications, but each application could only afford to check for the errors immediately relevant to it.  Hence they parameterized the 
stream descriptions by flags to toggle various error checking code.
This application provided the initial motivation for the \pads{} project
in general, and for masks in particular.

\subsection{Statistical summaries}
Before using a data source, analysts must develop an understanding 
of both the layout and the meaning of the data.  
Because documentation is usually
incomplete or out-of-date, this understanding must be developed 
through exploring the data itself.  Typical questions include:
how complete is the description of the syntax of the data source,
how many different representations for ``data not available" are there,
what is the distribution of values for particular fields, \etc{}
\pads{} addresses these kinds of questions with the notion of an accumulator. 
For each type in a \pads{} description, accumulators track the number of good values, the number of bad values, and the 
distribution of legal values.  Selected functions from this portion of the library appear in \figref{figure:library}.  

We can of course use these functions by hand to write
a program to compute the statistical profile of any \pads{} data source.
However, ad hoc sources are often simply a sequence of records, perhaps prefixed by a header, so we can create a complete accumulator program from minimal extra information.  Both the web server log and the \dibbler{} data sources exhibit this pattern.\footnote{
  Note that a data format that can be read in one bulk read also fits this pattern.
} 
Consequently, given only the names of the optional
header type and the record type, the \pads{} system will generate 
an accumulator program.  The accumulator report for the length field
of the web server data looks as follows when run on a data set used
in several studies of web traffic~\cite{clf-cluster, clf-adaptation}:

\begin{small}
\begin{verbatim}
<top>.length : uint32
+++++++++++++++++++++++++++++++++++++++++++
good: 53544   bad: 3824    pcnt-bad: 6.666
min: 35  max: 248591  avg: 4090.234
top 10 values out of 1000 distinct values:
tracked 99.552% of values
 val:  3082 count:  1254  %-of-good:  2.342
 val:   170 count:  1148  %-of-good:  2.144
 val:    43 count:  1018  %-of-good:  1.901
 val:  9372 count:   975  %-of-good:  1.821
 val:  1425 count:   896  %-of-good:  1.673
 val:   518 count:   893  %-of-good:  1.668
 val:  1082 count:   881  %-of-good:  1.645
 val:  1367 count:   874  %-of-good:  1.632
 val:  1027 count:   859  %-of-good:  1.604
 val:  1277 count:   857  %-of-good:  1.601
. . . . . . . . . . . . . . . . . . . . . . 
 SUMMING    count:  9655  %-of-good: 18.032
\end{verbatim}
\end{small}
%
By default, accumulators track the first 1000 distinct
values seen in the data source and report the frequency
of the top ten values.  In this particular run, 99.552\%
of all values were tracked.  When generating the accumulator
program (or when using the library directly), \pads{} users can specify 
the number of distinct values to track and the number 
of values to print in the report.

Perhaps surprisingly, the report shows that 6.66\% of the length
fields contained errors.  A glance at the error log generated
by the program (which contains all records flagged as errors) 
reveals that web servers occasionally store the '-' character
rather than the actual number of bytes returned, a possibility
not mentioned in the documentation~\cite{wpp}.

Accumulators can also be used to profile data sources automatically.
Indeed, this application motivated the initial design of accumulators.
AT\&T's \ningaui{} project receives roughly 4000~data files per day in various
Cobol formats.  This volume makes looking at each file by hand 
prohibitively expensive.  However, accumulator profiles can be used to 
automatically determine
which profiles have high percentages of errors and which have significantly
different statistical profiles than earlier versions of the same file.  
To support this usage, we built a tool that automatically translates
Cobol copybooks into \pads{} descriptions.

In practice at AT\&T, accumulators have proven themselves very useful for
data exploration.  The \darkstar{} project uses \pads{} accumulator programs
to find all the different representations of ``data not available,"  typical 
examples of which include \cd{0}, a blank, \cd{NONE}, and \cd{Nothing}.
An accumulator program revealed the two representations
of missing phone numbers in the \dibbler{} data that the example program
in \secref{subsec:general} repaired.
Accumulators also often serve as a quick tool for iteratively
refining a \pads{} description until only genuine errors remain.

\subsection{Format Conversion}
Another common need is to convert ad hoc data into a more well-behaved format, such as delimited fields suitable for loading into a spreadsheet or relational database, or into XML.  
\subsubsection{Formatting}
To support converting ad hoc data into a delimited format, the \pads{}
library generates a formatting function for each type.  This function,
an example of which appears in \figref{figure:library}, takes a delimiter
list as an argument.  At each field boundary, it prints the first delimiter.
At each nested type boundary, it advances the delimiter list unless the list
is exhausted, in which case it reuses the last delimiter.  The mask argument
allows the user to suppress printing of portions of the data.  Programmers
can use the library directly to write formatting programs by hand.  However, 
as in the accumulator case, \pads{} can generate a formatting program for 
commonly occurring data patterns given only the header type (optional), record type, and a delimiter string.  Users can further customize the generated program by specifying an output format for dates and mask values.   Given the delimiter
string \cd{"|"} and the output date format \cd{"\%D:\%T"}, the generated
web server log formatting program yields
the output shown in \figref{figure:clf-records-formatted} when applied to the
sample data in \figref{figure:clf-records}. 
\begin{figure*}
\begin{small}
\begin{center}
\begin{code}
\chapter{Common log format example}
\section{pads description}
\section{c header file}
\end{code}
\caption{Formatted CLF records.}
\label{figure:clf-records-formatted}
\end{center}
\end{small}
\end{figure*}
To support customization, \pads{} allows users to provide their own formatting functions for any type.

AT\&T's \darkstar{} project uses generated formatting programs to convert
various data sources into a format suitable for loading into a relational
database.

\subsubsection{XML Generation}
\pads{} also supports converting ad hoc data into XML by providing a canonical mapping from \pads{} descriptions into XML.  This mapping is quite natural, as both \pads{} and XML are languages for describing semi-structured data.
One interesting aspect of the mapping is that we embed not just the in-memory representation of \pads{} values, but also the parse descriptors in cases where the data was buggy.  This choice allows users to explore the error portions
of their data sources, which can be the most interesting parts of the data.
Given a \pads{} specification, the \pads{} compiler generates an XML Schema describing the canonical embedding for that data source.  As an example, 
the following is the portion of the generated XML Schema for the \cd{eventSeq} type in the \dibbler{} data description.

\begin{small}
\begin{verbatim}
<xs:complexType name="eventSeq_pd">
<xs:sequence>
<xs:element name="pstate" type="Pflags_t"/>
<xs:element name="nerr" type="Puint32"/>
<xs:element name="errCode" type="PerrCode_t"/>
<xs:element name="loc" type="Ploc_t"/>
<xs:element name="neerr" type="Puint32"/>
<xs:element name="firstError" type="Puint32"/>
<xs:element name="elt" type="Puint32" 
    minOccurs="0" maxOccurs="unbounded"/>
</xs:sequence>
</xs:complexType>

<xs:complexType name="eventSeq">
<xs:sequence>
<xs:element name="elt" type="event" 
    minOccurs="0" maxOccurs="unbounded"/>
<xs:element name="length" type="Puint32"/>
<xs:element name="pd" type="eventSeq_pd" 
    minOccurs="0" maxOccurs="1"/>
</xs:sequence>
</xs:complexType>
\end{verbatim} 
\end{small}
The \pads{} compiler generates a \cd{write_xml_2io} function for each type, an example of which is shown in \figref{figure:library}.  Given a specification of the top level type, \pads{} can also automatically generate a conversion program, the output of which conforms to the generated XML Schema.

\subsection{Queries}
Our final use-scenario is querying data.
Given a data source, a natural desire is to ask questions about the data, a desire which led to SQL and its many variants for relational data and XQuery for XML data~\cite{boag03XQueryDraft}.  Analysts working with ad hoc data would also like to query their
data, but the lack of tools generally means they code their queries in an imperative fashion in languages such as \textsc{awk}, \perl{}, or \C{}.
Indeed, the analyst working with the \dibbler{} data took this approach.
He coded queries such as ``Select all orders starting within a certain time window," ``Count the number of orders going through a particular state," and ``What is the average time required to go from a particular state to another
particular state" in a mixture of \textsc{awk} and \textsc{perl}.  He was
able to get the answers to his questions, but he had to code the queries explicitly, and the query-related code ended up embedded in his already-brittle parsing code.

We wanted to support declarative querying over ad hoc sources, but we didn't want to invent an entirely new query language, which led us to examine existing languages.  Because XQuery is designed to manipulate semi-structured data, its expressiveness matched our data sources well.  We were able to code all the \dibbler{}-related queries in XQuery.  For example, the XQuery
\begin{code}
\begin{small}
{ $sirius/sirius/order[event[1]
    [timeStamp >= xs:date("2002-04-14") and 
     timeStamp <= xs:date("2002-05-25") ]] }
\end{small}
\end{code}
asks for all orders starting within the given time window.  

Happy with XQuery's expressiveness, we worked with the designers of the Galax~\cite{galax} open-source implementation of XQuery to define a data API~\cite{galaxmanual}. 
This API presents the source as a tree to Galax. With this architecture, Galax can incorporate any data source accessible through an instance of the data API.  We then extended the \pads{} system to produce such instances.  We were able to define the bulk of the API generically, having to generate on a per type basis only a handful of functions. 
\figref{figure:library} contains the key generated functions for the \cd{entry_t} type from the \dibbler{} data.    The  \cd{node_new} function creates a node in the tree representation of the data, storing the supplied name, mask, parse descriptor, and in-memory representation.  It makes the argument node the parent of the newly created node.
The \cd{node_kthChild} function takes a tree
corresponding to an \cd{entry_t} node and a child index and returns the appropriate child.  For the \cd{entry_t} type, possible children are the header, the event sequence, or a parse descriptor. 
At the moment, it is possible to use the resulting system to query ad hoc data sources that can be loaded entirely into memory, and a version that allows the data to be read lazily is well underway.
How best to optimize Xqueries over ad hoc data sources is an open research area.

\section{Implementation}

From a \pads{} description, the \pads{} compiler generates \texttt{.h}
and \texttt{.c} files that together implement the data structures and
operations necessary for manipulating the types declared in the source
file.  The \pads{} run-time library implements all shared
functionality, including file operations, regular expression
manipulation, memory management, the provided base types, and the
generic portions of the Galax data API.  \pads{} generates a recursive
descent parser that makes it easy to provide multiple entry points.

We used the CKIT library~\cite{ckit} to implement the \pads{} compiler.
This library greatly facilitated the construction of the compiler as it
provides a framework for extending \C{}, typechecking the extension, and then pretty printing the result.  Because of CKIT, we were able to have a working version of the \pads{} compiler very quickly.
Currently, our compiler consists of 10,000 lines of \smlnj{} code layered on top of CKIT.

The \pads{} run-time library comprise approximately 30,000 lines of \C{} code, built on top of the AST~\cite{ast} and SFIO~\cite{sfio} libraries.
These libraries provide support for regular expressions, container data types, a date parsing library, and various I/O routines.  
 
To make the collection of base types user-extensible, the compiler reads all base type specifications from files.  
At compile time, the user can provide a list of such files to augment the provided base types.
A base type specification
declares the names of the in-memory representation, mask, and parse descriptor data structures and the names of the functions that implement the parsing, writing, formatting, \etc{} operations.  Support for accumulators is optional, as is support for the Galax data API.  The user is responsible for providing a \C{} library with implementations of all the named types and functions. 
More information about user-defined base types appears in the \pads{} manual~\cite{padsmanual}.

The \pads{} implementation is currently going through AT\&T's software
release process.  When the process is finished, hopefully in the next few weeks, the source code will be available from the \pads{} website with a non-commercial use license:
\begin{center}
\url{http://www.research.att.com/projects/pads}
\end{center}
 
\section{Performance}
To measure the performance of \pads{}, we compare the parser \pads{} generates for the \dibbler{} data description against a hand-written \perl{} program, as \perl{} is the language that our user base has typically used. We compare the performance of the two approaches on two different tasks: vetting \dibbler{} data in a fashion similar to the filter program of
\figref{figure:dibbler-filter} and collecting the order number of all records that ever pass through a particular state.  For the first task, we check all the specified properties on the data, including the constraint that the timestamps in the events appear in sorted order.  For the second task, we turn off all error checking and simply output the desired order numbers on standard out.
We pass as input to the selection programs the cleaned data file produced by the vetter programs. 
We attempted to write the equivalent \perl{} programs in as efficient a manner possible, given the specified tasks.  For each record, the \perl{} vetter uses the built-in \cd{split} operator to produce an in-memory array of the pipe-separated fields.  The \perl{} selection program uses \perl{}'s regular expression pattern matcher to find lines with the desired state in any 
position after the 13th field.  \figref{figure:perl-re} shows the relevent regular expression, looking for orders going through state \cd{\$STATE}.
\perl{} compiles this pattern and applies the compiled pattern to each line.  The \perl{} vetter is 323~lines of well-commented \perl{} code, while the selection program is 66~lines.  
The \pads{} vetter is 153~lines of well-commented \C{} code, while the \pads{} selection program is 120~lines. 

\begin{figure*}
\begin{center}
\begin{verbatim}
                   qr/^(\d+)\|(?:[^|]*\|){12}(?:[^|]*\|[^|]*\|)*$STATE\|/;
\end{verbatim}
\vskip -2ex
\end{center}
\caption{Regular expression at the heart of the \perl{} selection program.}
\label{figure:perl-re}
\end{figure*}

We used a \dibbler{} data file to exercise the programs.  This 2.2GB file contained 11,773,843 records.  The minimum number of states for an order was one, the maximum number was 156, and the average was 5.5.  One of these records violated the expected sorting order on event timestamps, and 53 of them contained a syntax error.
(These statistics are courtesy of the generated \pads{} accumulator program, a nice side-benefit of the \pads{} description.)  

We conducted our experiments on one processor of an SGI Origin 2000 running Irix 6.5.
The processor, a 500~Mhz R14000, has split 32KB instruction and data caches and an 8MB secondary cache.  The machine has more than 20GB of main memory.
We used \perl{} version 5.6.1 to execute the \perl{} scripts,
and \cd{gcc} version 3.2.2 with optimization level O2 to compile the \pads{} programs.
We ran each program three times.  \figref{figure:performance} reports the elapsed running times using the Unix \cd{time} utitlity.
For comparison, a \perl{} program that simply counts the number of records takes on average~124 seconds.  The corresponding \pads{} program takes~81 seconds.

\begin{figure}
\begin{center}
\begin{tabular}{|l@{\hskip 2ex}l|}
\hline
pads\_vet  & perl\_vet.pl \\ \hline
1619.1   & 3310.8       \\
1600.7   & 3300.9       \\
1627.5   & 3205.2       \\ \hline 
\end{tabular}
\begin{tabular}{|l@{\hskip 2ex}l|}
\hline
pads\_select  & perl\_select.pl \\ \hline
426.4     & 548.7       \\
419.8     & 518.4       \\
418.4     & 492.1        \\  \hline
\end{tabular}
\end{center}
\vskip -2ex
\caption{Elapsed time in seconds of \perl{} and \pads{} vetting and selection programs. We report the times for all three runs.}
\label{figure:performance}
\end{figure}

The \pads{} vetter was more than twice as fast as the \perl{} vetter, while the \pads{} selection program outperformed the \perl{} selection program by a smaller margin.
We have not yet devoted significant time to optimizing the generated parser
or the base type library,
so we expect improvements are possible.

\section{Related work}
There are many tools for describing data formats. For example,
\textsc{ASN.1}~\cite{asn} and \textsc{ASDL}~\cite{asdl} are both
systems for declaratively describing data and then generating
libraries for manipulating that data.  In contrast to \pads{},
however, both of these systems specify the {\em logical\/} representation
and automatically generate a {\em physical\/} representation.
Although useful for many purposes, this technology does not help
process data that arrives in predetermined, ad hoc formats.

Lex and yacc-based tools generate parsers from declarative descriptions, but 
they require users to write both a lexer and a grammar and to construct the in-memory representations by hand.  In addition, they only work for ASCII data,  they do not easily accommodate data-dependent parsing, and they do not provide auxiliary services. 

More closely related work includes \erlang{}'s bit syntax~\cite{erlang} and
the \packettypes{}~\cite{sigcomm00} and
\datascript{} languages~\cite{gpce02}, 
all of which allow declarative descriptions of physical data.  These projects were motivated by parsing protocols,
\textsc{TCP/IP} packets, and \java{} jar-files, respectively.  Like
\pads{}, these languages have a type-directed approach to
describing ad hoc data and permit the user to define semantic constraints.
In contrast to our
work, these systems handle only binary data and assume the data is
error-free or halt parsing if an error is detected. 
Parsing non-binary data poses additional challenges because of the need
to handle delimiter values and to express richer termination conditions
on sequences of data. These systems also
focus exclusively on the parsing/printing problem, whereas we have 
leveraged the declarative nature of
our data descriptions to build additional useful tools.


Recently, a standardization effort has been started whose stated goals are quite similar to those of the \pads{} project~\cite{dfdl}. The description
language seems to be \xml{} based, but at the moment, more details are 
not available.

\section{Future work}
\pads{} already defines a rich collection of tools for ad hoc data processing.
However, there are a number of directions for future research and possibilities for extension. 

\textbf{Language expressiveness.}  
We intend to add bit-field 
and overlay constructs to \pads{} to support binary data sources better,
in a fashion similar to that already supported in the \datascript{} and \packettypes{} languages.
We also plan to generalize the switched union construct to permit arbitrary lookahead to better support
parsing for complex ASCII formats such the HTTP packet specification~\cite{http}.  Further, we need to design a mechanism
for specifying character encodings so that we may support 
Unicode~\cite{unicode} data sources.

\textbf{Generated artifacts.} 
For research purposes, it would be very useful to be able to generate random data that conforms to a given specification, or 
deviates from it in specified ways, particularly when the real data is proprietary and cannot be exposed outside of AT\&T. 
Given a \pads{} specification and a characterization of the desired distribution, it should be possible to generate such data.  We also plan to augment the statistical profiling library with
functions that use randomized and approximate techniques to create
small summaries such as histograms~\cite{histograms,histograms-wavelets}, wavelet summaries~\cite{histograms-wavelets},
or quantile summaries\cite{quantiles}.
Another useful tool we 
would like to generate from \pads{} descriptions is a graphical binary data editor. 

\textbf{Semantics.}  We are in the process of developing a formal semantics for the \pads{} language so that we may have a declarative specification of what a \pads{} specification means.

\textbf{Application-specific customization.}
Ideally, \pads{} specifications describe everything about a given
data source, but nothing about how the data is to be used. This
division allows a single description of a given source to be used
for multiple applications.  However, different applications have different
performance, verification, and error-handling needs.  Currently,
\pads{} allows users to customize the behavior of library
functions at run-time by supplying appropriate masks and setting various error handlers.  However, this leaves a run-time overhead.  We would like to 
design a mechanism whereby users can specify application-specific information.  The \pads{} system would then use this information to generate an 
application-specific instance of the library.  Conceptually, this would
amount to partially evaluating the current \pads{} library by specifying mask arguments at binding time.

\textbf{Language bindings.}  Nothing about the \pads{} language is specific to \C{}.  It would be very interesting to develop a binding for \pads{} in a higher-order functional language because the pattern-matching constructs of such languages are extremely adept at expressing transformations, the natural next step after parsing data.


\section{Conclusions}
\pads{} is a declarative data description language that allows 
its users to describe ad hoc data sources as they are, capturing
layout information and semantic constraints.  The language
is expressive enough to describe almost all of the ASCII, Cobol, 
and binary data formats we have seen in practice at AT\&T, while being 
concise enough to serve as living documentation for those data sources.  

The 
advantages of a declarative language for describing data are
significant.  First, the user is spared from having to write
the parser in the first place, which is an inherently tedious
process.   Second, because the parser is machine
generated, it can check all the error conditions necessary to
guard against corrupting downstream data without cluttering
user code.  Third, because of its conciseness, the \pads{}
description can serve as documentation for a data source.  Given
that the parser is generated from the description, there is 
strong incentive to maintain the description and hence the documentation
as the data evolves.  Finally, because we have a declarative
specification, we can generate not just a parser, but also
a validator, a printer, a statistical profiler, various formatting
tools, query support, \etc{}  In other words, we can leverage the declarative nature of the specification to build a large number of useful tools.

\pads{} has already been quite useful at AT\&T, and as the collection of generated tools grows, it will only become more so.  We look forward to developing a larger user base to get more feedback to further improve the system. 


\section{Acknowledgments}
We would like to thank Mary Fern\'andez and J\'er\^ome Sim\'eon for 
developing the Galax data API and our summer
students Ricardo Medel and Yitzhak Mandelbaum for their work on implementing
the XML-related portions of the \pads{} system.
Mary Fern\'andez, Yitzhak Mandelbaum, David Walker, Rick Schlichting,
and John Launchbury made many helpful comments on an earlier draft.


\bibliographystyle{abbrv}
\small
\bibliography{pads} 

\end{document}














