\documentclass{sig-alternate}
\usepackage{times,url}
\usepackage{code} 
\newcommand{\cut}[1]{}

\newcommand{\appref}[1]{Appendix~\ref{#1}}
\newcommand{\secref}[1]{Section~\ref{#1}}
\newcommand{\tblref}[1]{Table~\ref{#1}}
\newcommand{\figref}[1]{Figure~\ref{#1}}
\newcommand{\listingref}[1]{Listing~\ref{#1}}
%\newcommand{\pref}[1]{{page~\pageref{#1}}}

\newcommand{\eg}{{\em e.g.}}
\newcommand{\cf}{{\em cf.}}
\newcommand{\ie}{{\em i.e.}}
\newcommand{\etc}{{\em etc.\/}}
\newcommand{\naive}{na\"{\i}ve}
\newcommand{\role}{r\^{o}le}
\newcommand{\forte}{{fort\'{e}\/}}
\newcommand{\appr}{\~{}}

\newcommand{\bftt}[1]{{\ttfamily\bfseries{}#1}}
\newcommand{\kw}[1]{\bftt {#1}}
\newcommand{\Pthen}{\kw{Pthen}}
\newcommand{\pads}{\textsc{pads}}
\newcommand{\padsl}{\textsc{padsl}}
\newcommand{\padst}{\textsc{pads/t}}
\newcommand{\datatype}{\textsc{PADS/T}}
%\newcommand{\datatype}{\textsc{DataType}}
\newcommand{\C}{\textsc{C}}
\newcommand{\perl}{\textsc{Perl}}
\newcommand{\ml}{\textsc{ml}}
\newcommand{\sml}{\textsc{sml}}
\newcommand{\smlnj}{\textsc{sml/nj}}
\newcommand{\java}{\textsc{java}}
\newcommand{\ddl}{\textsc{ddl}}
\newcommand{\xml}{\textsc{xml}}
\newcommand{\datascript}{\textsc{DataScript}}
\newcommand{\packettypes}{\textsc{PacketTypes}}
\newcommand{\erlang}{\textsc{Erlang}}

\newcommand{\Core}{Ad hoc}
\newcommand{\core}{ad hoc}
\newcommand{\pvalue}{\core{} value}
\newcommand{\ppat}{\core{} pattern}
\newcommand{\ptype}{\core{} type}

\newcommand{\padsc}{\textsc{pads}/\C{}}
\newcommand{\padsml}{\textsc{pads}/\ml{}}

\newcommand{\dibbler}{Sirius}
\newcommand{\ningaui}{Altair}
\newcommand{\darkstar}{Regulus}

\newcommand{\pdgood}{{\tt G}}
\newcommand{\pdbad}{{\tt B}}
\newcommand{\pdnest}{{\tt N}}
\newcommand{\pdsem}{{\tt S}}
\newcommand{\ptypes}{T}
\newcommand{\patreadpd}[2]{{\tt #1<<#2>>}}
\newcommand{\btm}{\cd{BOT}}


\newcommand{\lsem}{{[\![}}
\newcommand{\rsem}{{]\!]}}


\newcommand{\figHeight}[4]{\begin{figure}[tb]
	\centerline{
	            \epsfig{file=#1,height=#4}}
	\caption{#2}
	\label{#3}
	\end{figure}}

%% Environment for typesetting BNF grammars. Uses display math mode.
\newenvironment{bnf}
     {%% local command definitions:
        %% BNF definition symbol
      \def\->{\rightarrow}
%%      \def\::={{::=} &}
      \def\::={\bnfdef &}
      \def\|{\bnfalt}
      \newcommand{\name}[1]{\text{##1}}
        %% non-terminal
      \newcommand{\nont}[1]{{##1}}
      \newcommand{\meta}[1]{& ##1 &}
      \newcommand{\descr}[1]{& \text{// ##1}}
      \newcommand{\opt}[1]{ [##1] }
      \newcommand{\opnon}[1]{\opt{\nont{##1}}}
      \newcommand{\none}{\epsilon}
      \newcommand{\nwln}{\\ &&&}
      \newcommand{\nlalt}{\\ && \| &}
      \[\begin{array}{lrlll}
     }
     {\end{array}\]}

\newcommand{\mcd}[1]{\mathtt{#1}}
\newcommand{\ppair}[3]{#1{:}#2 \mathrel{**} #3}
\newcommand{\parray}[3]{#1\;\mcd{Parray}(#2,#3)}
\newcommand{\pset}[3]{\{#1{:}#2\,|\,#3\}}
\newcommand{\pstream}[1]{#1\;\mcd{stream}}
\newcommand{\precord}[1]{\{\{#1\}\}}


\begin{document}
\title{PADS\\ A domain specific language for processing ad hoc data}
\numberofauthors{2}

\author{
\alignauthor Kathleen Fisher\\
       \affaddr{AT\&T Labs Research}\\
       \affaddr{180 Park Ave., E244}\\
       \affaddr{Florham Park, NJ}\\
       \email{kfisher@research.att.com}
\alignauthor Robert Gruber\titlenote{Work carried out while at AT\&T
                                     Labs Research}\\
       \affaddr{Google}\\
       \affaddr{1600 Amphitheatre Pkwy}\\
       \affaddr{Mountain View, CA}\\
       \email{gruber@google.com}
}

\date{\today}
\maketitle
\begin{abstract}
\pads{} is a declarative data description language that allows data
analysts to describe both the physical layout of ad hoc data sources
and semantic constraints over that data.  From such descriptions, the
\pads{} compiler generates libraries and tools for manipulating the
data, including parsing routines, statistical profiling tools,
translation programs to produce well-behaved
formats such as \xml{} or those required for loading relational
databases, and an instance of the Galax data API, which permits
running XQueries over raw \pads{} data sources.
The descriptions are concise enough to serve as ``living'' documentation
while flexible enough to describe most of the ASCII, binary, and
EBCDIC formats that we have seen in practice.  The generated parsing
library provides for robust and application-specific error handling.
\end{abstract}

\section{Introduction}


Although traditional databases and \xml{} systems provide rich
infrastructure for processing well-behaved data, vast amounts of
useful data are stored and processed neither in databases nor in
\xml{}, but rather in ad hoc data formats. 
Examples include call detail data~\cite{kdd00}, 
%compiler traces~\cite{chilimbi},
web server logs~\cite{wpp}, 
netflows capturing internet traffic\cite{netflow}, 
log files characterizing IP backbone resource utilization,
wire formats for legacy telecommunication billing systems, 
etc{}. 
Such data may simply require processing before it can be loaded into a
data management system, or it may be too large or too transient to
make such loading cost effective.

Processing ad hoc data can be quite challenging, for a variety of
reasons. First, ad hoc data typically arrives ``as is'': the analysts
who receive it can only say ``thank you'' to the people sending the
data, not request a more convenient format. 
Second, documentation for the format may not exist at all, or it may be
significantly out of date.  A common phenomenon is for a field in a
data source to fall into disuse because the data is no longer
relevant.  After a while though, people find an intereting new piece of
information.  Compatibility issues prevent them from modifying the
shape of the data, so instead they hijack the unused field, often
failing to update the documentation in the process.

Third, such data frequently contain errors, for a variety of
reasons: malfunctioning equipment, race conditions on log
entry~\cite{wpp}, non-standard values to indicate ``no data
available,'' human error in entering data, undocumented data
encodings, \etc{} The appropriate response to such errors is
application-specific. Some applications require the data to be error free: 
if an error is detected processing needs to stop immediately and a human
needs to be alerted.  Other applications can repair the data, while still
others can simply discard erroneous or unexpected values.  
For some applications,
errors in the data can be the most interesting part  because
they can signal where two systems are failing to communicate.

A fourth challenge is that such data sources can be high volume:
AT\&T's call-detail stream contains roughly 300~million calls per day
requiring approximately 7GBs of storage space. Although this data is
eventually archived in a database, analysts mine it profitably before
such archiving~\cite{kdd98,kdd99}. The Ningaui project at AT\&T
accumulates data related to provisioning business services at a rate
of 250-300GB/day, with occasional spurts of 750GBs/day.  Netflow data
arrives from Cisco routers at rates over a gigabyte per
second~\cite{gigascope}. Such volumes mean it must be possible to
process the data without loading it all into memory at once.

Finally, before anything can be done with an ad hoc data source,
someone has to produce a suitable parser for it, typically in \C{} or \perl{}. 
Unfortunately, writing
such parsers is tedious and error-prone. It is complicated by the lack
of documentation, convoluted encodings designed to save space, 
the need to produce efficient code to cope with the scale of the data,
and the need to handle errors robustly to avoid corrupting precious data.
In fact, such parsers tend to skip checking all possible errors
because doing so requires writing so much extra code. 
Often, the hard-won understanding of the data
ends up embedded in parsing code, making long-term maintenance
difficult for the original writer and sharing the knowledge with
others nearly impossible.

The \pads{} system facilitates ad hoc data processing by addressing
each of these concerns.  It provides a declarative data description
language that permits analysts to describe the physical layout of
their data, \textit{as it is}.  The language also permits analysts to
describe expected semantic properties of their data so deviations can
be flagged as errors. The intent is to allow analysts to capture in a
\pads{} description all that they know about a given data source.
These descriptions are concise enough to
serve as documentation and flexible enough to describe most of
the data formats we have seen in practice, including ASCII, binary,
EBCDIC, and mixed data formats.  The fact that useful software
artifacts are generated from the descriptions provides strong
incentive for keeping the descriptions current, allowing them to serve
as living documentation.  

From such a description, the \pads{} compiler produces tunable \C{} libraries
and tools for parsing, manipulating, and summarizing the data,
obviating the need for a hand-written parser and leveraging the
declarative specification to produce additional useful tools. 
The core \C{} library includes functions for reading the data, writing it 
back out in its original form, writing it into a canonical \xml{} form, pretty printing
it in forms suitable for loading into a relational database, and accumulating  
statistical properties.  An auxiliary library provides 
an instance of the data API for Galax~\cite{galax,galaxmanual}, an implementation of XQuery.  This 
library allows users to query data with a \pads{} description as if the data were
in \xml{} without materializing the data as \xml{}.  In addition to these libraries,
the \pads{} system provides wrappers that build tools to 
summarize the data, format it,  or convert it to \xml{}.

The declarative nature of \pads{} descriptions facilitates error handling.
The generated parsing code checks all possible error cases: system
errors related to the input file, buffer, or socket; syntax errors
related to deviations in the physical format; and semantic errors in
which the data violates user constraints.  Because these checks appear
in generated code, they do not clutter the high-level declarative
description of the data source.
In addition, the result of a parse is a pair consisting of a canonical in-memory
representation of the data and a parse descriptor. The parse
descriptor precisely characterizes both the syntactic and the semantic
errors that occured during parsing.  This structure allows analysts
to respond to errors in application-specific ways.  

The \pads{} system addresses performance issues in a number of ways.
First, we compile the \pads{} description rather than interpret it to reduce run-time overhead.
The generated parser provides  multiple entry points, so the data consumer can choose 
the appropriate level of granularity for reading the data into memory to accommodate very large data sources.
We parameterize library functions by \textit{masks}, which allow data analysts to 
choose which semantic conditions to check at run-time, permitting them to specify
all known properties in the source description without forcing all users of that 
description to pay the run-time cost of checking them.  


\section{Example data sources}

Before discussing the \pads{} design, we describe selected data
sources, focusing on those we will use as running examples.
\figref{figure:data-sources} summarizes some of the sources we have
worked with.  They include ASCII, binary, and Cobol data formats, with
both fixed and variable-width records.  They range in size from
relatively small files through network applications which process over
a gigabyte per second.  Common errors include undocumented data,
corrupted data, missing data, and multiple missing-value
representations.

\begin{figure*}
\begin{center}
\begin{tabular}{|l|l|l|l|l|}
\hline
Name          &  Representation              &Size           & Common Errors \\ \hline\hline
Call detail   &  Fixed width binary records  &\appr{}7GB/day &  Undocumented data\\  \hline 
\textbf{Web server logs (CLF)} &  Fixed-column ASCII records & $\leq$12GB/week & Race conditions on log entry\\ \hline
\textbf{AT\&T provisioning} & Variable-width ASCII records & 2.2GB/week & Unexpected values \\ 
\textbf{summary data (Dibbler)} &                              &            & Corrupted data feeds \\ \hline
AT\&T provisioning  & Various Cobol data formats  & \appr{}4000 files/day, & Unexpected values\\ 
raw data            &                             & 250-300GB/day    & Corrupted data feeds \\ \hline
IP backbone monitoring data  & ASCII  & Feeds from \appr{}15 sources  & Multiple missing-value representations \\
                             &        & \appr{}15 GB/day              & Undocumented data \\ \hline
Netflow       & Data-dependent number of     & $\ge$1Gigabit/second  & Missed packets\\ 
              &  fixed-width binary record   &                       & \\ \hline

\end{tabular}
\label{figure:data-sources}
\caption{Selected ad hoc data sources.  We will use \textbf{bold} data sources in our examples. }
\end{center}
\end{figure*}

\subsection{Common Log Format}
Web servers use the Common Log Format (CLF) to log client
requests~\cite{wpp}.  This ASCII format consists of a sequence of
records, each of which has seven fields: the host name or IP address
of the client making the request, the account associated with the
request on the client side, the name the user provided for
authentication, the time of the request, the actual request, the
\textsc{http} response code, and the number of bytes returned as a
result of the request.  The actual request has three parts: the
request method (\eg, \texttt{GET}, \texttt{PUT}), the requested
\textsc{uri}, and the protocol version.  In addition, the second and
third fields are often recorded only as a '-' character to indicate
the server didn't record the actual data.  \figref{figure:clf-records}
shows a couple of typical records.

\begin{figure*}
\begin{small}
\begin{center}
\begin{verbatim}
207.136.97.49 - - [15/Oct/1997:18:46:51 -0700] "GET /tk/p.txt HTTP/1.0" 200 30
tj62.aol.com - - [16/Oct/1997:14:32:22 -0700] "POST /scpt/dd@grp.org/confirm  HTTP/1.0" 200 941
\end{verbatim}
\label{figure:clf-records}
\caption{Typical CLF records.}
\end{center}
\end{small}
\end{figure*}


\subsection{Provisioning summary data}

To track AT\&T's provisioning process, the Dibbler project compiles
weekly summaries of the state of each order for phone service.  
These ASCII summaries store the processing date and one record per order.
Each order record contains a header followed by a sequence of events.
The header has 13 pipe separated fields: the order number, AT\&T's
internal order number, the order version, four different telephone
numbers associated with the order, the zip code of the order, a
billing identifier, the order type, a measure of the complexity of the
order, an unused field, and the source of the order data.  Many of
these fields are optional, in which case nothing appears between the
pipe characters.  The billing identifier may not be available at the
time of processing, in which case the system generates a unique
identifier, and prefixes this value with the string "no\_ii" to
indicate the number was generated. The event sequence represents the
various states a service order goes through; it is represented as a
new-line terminated, pipe separated list of state, timestamp pairs.
There are over 400 distinct states that an order may go through during
provisioning.  The sequence is sorted in order of increasing time
stamps. \figref{figure:dibbler-records} shows a small example of
this format.
%156 different states for one order
%-rw-r--r--    1 angusm   dibbler   2187472314 Jun  9  2003 /fs/dibblerd/tlf/data/out_sum.stream
%2171.364u 31.379s 40:41.54 90.2% 0+0k 2+0io 2pf+0w
%53 had trailing t or } after zip code
\begin{figure*}
\begin{small}
\begin{center}
\begin{verbatim}
0|1005022800
9152|9152|1|9735551212|0||9085551212|07988|no_ii152272|EDTF_6|0|APRL1|DUO|10|1000295291
9153|9153|1|0|0|0|0||152268|LOC_6|0|FRDW1|DUO|LOC_CRTE|1001476800|LOC_OS_10|1001649601
\end{verbatim}
\label{figure:dibbler-records}
\caption{Sample provisioning summary data (vastly truncated).}
\end{center}
\end{small}
\end{figure*}


It may be apparent from these paragraphs that English is a poor
language for describing data formats!


\section{Pads Design}
\begin{figure}
\begin{code}
\kw{Punion} client\_t \{
  Pip       ip;      /- 135.207.23.32
  Phostname host;    /- www.research.att.com
\};
\mbox{}
\kw{Punion} auth\_id\_t \{
  Pchar unauthorized : unauthorized == '-';
  Pstring(:' ':) id;
\};
\mbox{}
\kw{Pstruct} version\_t \{
  "HTTP/";
  Puint8 major; '.';
  Puint8 minor;
\};
\mbox{}
\kw{Penum} method\_t \{
    GET,    PUT,  POST,  HEAD,
    DELETE, LINK, UNLINK
\};
\mbox{}
bool chkVersion(version\_t v, method\_t m) \{
  \kw{if} ((v.major == 1) && (v.minor == 1)) \kw{return} true;
  \kw{if} ((m == LINK) || (m == UNLINK)) \kw{return} false;
  \kw{return} true;
\};
\mbox{}
\kw{Pstruct} request\_t \{
  '\\"';   method\_t       meth;
  ' ';    Pstring(:' ':) req\_uri;
  ' ';    version\_t      version :
                  chkVersion(version, meth);
  '\\"';
\};
\mbox{}
\kw{Ptypedef} Puint16\_FW(:3:) response\_t :
         response\_t x => \{ 100 <= x && x < 600\};
\mbox{}
\kw{Precord} \kw{Pstruct} entry\_t \{
         client\_t       client;
   ' ';  auth\_id\_t      remoteID;
   ' ';  auth\_id\_t      auth;
   " ["; Pdate(:']':)   date;
   "] "; request\_t      request;
   ' ';  response\_t     response;
   ' ';  Puint32        length;
\};
\mbox{}
\kw{Psource} \kw{Parray} clt\_t \{
  entry\_t [];
\}
\end{code}

\caption{\pads{} description for web server log data}
\label{figure:wsl}
\end{figure}

\begin{figure}
\begin{code}
\kw{Precord} \kw{Pstruct} summary\_header\_t \{
  "0|";
  Puint32       tstamp;
\};
\mbox{}
\kw{Pstruct} no\_ramp\_t \{
  "no\_ii";
  Puint64 id;
\};
\mbox{}
\kw{Punion} dib\_ramp\_t \{
  Pint64     ramp;
  no\_ramp\_t  genRamp;
\};
\mbox{}
\kw{Pstruct} order\_header\_t \{
       Puint32             order\_num;
 '|';  Puint32             att\_order\_num;
 '|';  Puint32             ord\_version;
 '|';  \kw{Popt} pn\_t           service\_tn;
 '|';  \kw{Popt} pn\_t           billing\_tn;
 '|';  \kw{Popt} pn\_t           nlp\_service\_tn;
 '|';  \kw{Popt} pn\_t           nlp\_billing\_tn;
 '|';  \kw{Popt} Pzip           zip\_code;
 '|';  dib\_ramp\_t          ramp;
 '|';  Pstring(:'|':)      order\_type;
 '|';  Puint32             order\_details;
 '|';  Pstring(:'|':)      unused;
 '|';  Pstring(:'|':)      stream;
 '|';
\};
\mbox{}
\kw{Pstruct} event\_t \{
  Pstring(:'|':) state;   '|';
  Puint32        tstamp;
\};
\mbox{}
\kw{Parray} eventSeq \{
  event\_t[] : \kw{Psep}('|') && \kw{Pterm}(\kw{Peor});
\} \kw{Pwhere} \{
  \kw{Pforall} (i \kw{Pin} [0..length-2] :
           (elts[i].tstamp <= elts[i+1].tstamp));
\};
\mbox{}
\kw{Precord} \kw{Pstruct} entry\_t \{
  order\_header\_t  header;
  eventSeq        events;
\};
\mbox{}
\kw{Parray} entries\_t \{
  entry\_t[];
\};
\mbox{}
\kw{Psource} \kw{Pstruct} out\_sum\{
  summary\_header\_t  h;
  entries\_t         es;
\};
\end{code}

\caption{\pads{} description for telecom provisioning summary data}
\label{figure:dibbler}
\end{figure}


A \pads{} description specifies the physical layout and 
semantic properties of an ad hoc data source. 
The language provides a type-based model:
basic types describe atomic data such as integers, strings, dates, \etc{}, while
structured types describe compound data built from simpler pieces.
The \pads{} library provides a collection of broadly useful base types.
In addition,  users can define their own base types to specify more
specialized forms of atomic data.  Examples of built-in types include
ASCII 32-bit integers (\cd{Pa_int32}), binary bytes (\cd{Pb_int8}), and
EBCDIC characters (\cd{Pe_char}).  
We also support base types that don't specify an explicit encoding, 
\eg{} 64-bit integers (\cd{Pint64}).  
During parsing, \pads{} uses the ambient encoding to interpret such
types, which the library user can set at run-time.  
We currently support ASCII, binary, and EBCDIC encodings.  
By default, the ASCII encoding is selected.
Semantic conditions for such base types include checking that the
resulting number fits in the indicated space, \ie, 16-bits for
\cd{Pint16}.  All built-in \pads{} types start with "\cd{P}".

To described more complex data, \pads{} provides a collection of 
structured types loosely based on \C{}'s type structure.
In particular, \pads{} has 
\kw{Pstruct}s, \kw{Punion}s, and \kw{Parray}s to describe
record-like structures, alternatives, and sequences, respectively.
\kw{Penum}s describe a fixed collection of literals, while \kw{Popt}s 
provide convenient syntax for optional data.
Each of these
types can have an associated predicate that indicates whether a
value calculated from the physical specification is indeed a legal
value for the type.  For example, a predicate might require that two
fields of a \kw{Pstruct} are related or that the elements
of a sequence are in increasing order.  Programmers can specify such
predicates using \pads{} expressions and functions, 
written using a \C{}-like syntax.
Finally, \pads{} \kw{Ptypedef}s can be used
to define new types that add further constraints to existing types.

\pads{} types can be parameterized by values.
This mechanism
serves both to reduce the number of base types and to permit the
format and properties of later portions of the data to depend upon earlier portions.
For example, 
the base type \cd{Puint16_FW(:3:)} specifies an unsigned two byte integer
physically represented by exactly 3 characters, while the type
\cd{Pstring(:' ':)} 
describes a string terminated by a space.  Parameters can be 
used with compound types to specify the size of an array or which
branch of a union should be taken.

\figref{figure:wsl} gives the \pads{} description for CLF web server logs, 
while \figref{figure:dibbler} gives the description for the Dibbler 
provisioning data.  We will use these two examples to illustrate various 
features of the \pads{} language. 

\kw{Pstruct}s describe fixed sequences of data with unrelated types.
In the CLF description, the type declaration for
\cd{version_t} illustrates a simple \kw{Pstruct}. It starts with a 
string literal that matches the constant \cd{HTTP/} in the data source.  It 
then has two unsigned integers recording the major and minor version numbers
separated by the literal character \kw{'.'}.  \pads{} supports character, string,
and regular expression literals, which are interpreted with the ambient character 
encoding. The type \cd{request_t} 
similarly describes the user request portion of a CLF record.  In addition
to physical format information, this \kw{Pstruct} includes a semantic constraint
on the \cd{version} field.  Specifically, it requires that obsolete methods
\cd{LINK} and \cd{UNLINK} occur only with under HTTP/1.1.  This constraint illustrates
the use of predicate functions and the fact 
that earlier fields are in scope during the processing of later fields, as the 
constraint
refers to both the \cd{meth} and \cd{version} fields in the \kw{Pstruct}.

\kw{Punion}s describe variation in the data source.  For example, the
\cd{client_t} type in the CLF description indicates that the first field 
in a CLF record can be either an IP address or a hostname.  During parsing, 
the branches of a \kw{Punion} are tried in order; the first branch that 
parses without error is taken.  The \cd{auth_id_t} type illustrates the use
of a constraint: the branch \cd{unauthorized} is chosen only if the parsed
character is a dash.  

\pads{} provides \kw{Parray}s to describe varying-length sequences of data all 
with the same type.  The \cd{eventSeq_t} declaration in the Dibbler data description
uses a \kw{Parray} to characterize the sequence of events an
order goes through during processing.  This declaration indicates that the elements
in the sequence have type \cd{event_t}.  It also specifies that the elements will
be separated by vertical bars, and that the sequence will be terminated by 
an end-of-record marker (\kw{Peor}).  In general, \pads{} provides a rich collection
of array-termination conditions: reaching a maximum size, finding a terminating
literal (including end-of-record and end-of-source),
or satisfying a user-supplied predicate over the already-parsed portion of 
the \kw{Parray}.  Finally, this type declaration includes a \kw{Pwhere} clause
to specify that the sequence of time stamps must be in sorted order.
It uses the \kw{Pforall} construct to express this constraint.
In general, the body of a \kw{Pwhere} clause can be any boolean expression.
In such a context for arrays, the pseudo-variable \cd{elts} is bound to the in-memory representation of the sequence and \cd{length} to its length.

Returning to the CLF description, the \kw{Penum} \cd{method_t} describes
a collection of data literals.  During parsing, \pads{} interprets these
constants using the ambient character encoding.  The \kw{Ptypedef} 
\cd{response_t} describes possible server response codes in CLF data by adding
the constraint that the three-digit integer must be between 100 and 600.

The \cd{order_header_t} type in the Dibbler data description contains several
anonymous uses of the \kw{Popt} type.  This type is syntactic sugar for a 
stylized use of a \kw{Punion} with two branches: the first with the indicated type, and
the second with what might be thought of as the void type.  It always matches but
never consumes any input.

Finally, the \kw{Precord} and \kw{Psource} annotations deserve comment.  The first
indicates that the annotated type constitutes a record,
while the second means that the type constitutes the totality of a data source.  
The notion of a record varies depending upon the data encoding.  
ASCII data typically uses new-line characters to delimit 
records, binary sources tend to have fixed-width records, while 
COBOL sources usually store the length of each record before the actual data.
\pads{} supports each of these encodings of records and allows users to define
their own encodings.  By default, \pads{} assumes records are new-line terminated.
Before parsing, however, the user can direct \pads{} to use a different record
definition.

\section{Generated library}
From a \pads{} description, the \pads{} compiler generates a \C{} library
for parsing and manipulating the associated data source.  We selected \C{}
as the target language for pragmatic reasons: there were locally available
libraries that made building the compiler and run-time libraries easier,
our target users are comfortable wth \C{}, and it can serve most easily
as a lingua franca in that essentially all languages have provisions for 
calling \C{} libraries.  Nothing about the \pads{} language mandates compiling
to \C{}, however, and we envision eventually building alternate bindings.

From each type in a \pads{} description, the compiler generates 
\C{} declarations for
\begin{itemize}
\item an in-memory representation, 
\item a mask, which allows users to customize generated functions
on a per-type basis,
\item a parse descriptor, which we use to describe syntactic and
semantic errors detected during parsing, 
\item parsing and unparsing functions, and 
\item a broad collection of utility functions.
\end{itemize}
\begin{figure*}
\begin{code}
\kw{typedef} \kw{struct} \{
  Pbase\_m compoundLevel;   // Struct-level controls, eg., check Pwhere clause
  order\_header\_t\_m h;
  eventSeq\_t\_m events;
\} entry\_t\_m;
\mbox{}
\kw{typedef} \kw{struct} \{
  Pflags\_t pstate;         // Normal, Partial, or Panicking 
  Puint32 nerr;            // Number of detected errors.
  PerrCode\_t errCode;      // Error code of first detected error
  Ploc\_t loc;              // Location of first error
  order\_header\_t\_pd h;     // Nested header information
  eventSeq\_t\_pd events;    // Nested event sequence information
\} entry\_t\_pd;
\mbox{}
\kw{typedef} \kw{struct} \{
  order\_header\_t h;
  eventSeq\_t events;
\} entry\_t;
\end{code}

\begin{code}
/* Core parsing library */
Perror\_t entry\_t\_read (P\_t *pads,entry\_t\_m *m,entry\_t\_pd *pd,entry\_t *rep);
ssize\_t entry\_t\_write2io (P\_t *pads,Sfio\_t *io,entry\_t\_pd *pd,entry\_t *rep);
\end{code}

\begin{code}
/* Selected utility functions */
\kw{void} entry\_t\_m\_init (P\_t *pads,entry\_t\_m *mask,Pbase\_m baseMask);
\kw{int} entry\_t\_verify (entry\_t *rep);
\end{code}

\begin{code}
/* Selected accumulator functions */
Perror\_t entry\_t\_acc\_init (P\_t *pads,entry\_t\_acc *acc);
Perror\_t entry\_t\_acc\_add (P\_t *pads,entry\_t\_acc *acc,entry\_t\_pd *pd,entry\_t *rep);
Perror\_t entry\_t\_acc\_report (P\_t *pads,\kw{char} \kw{const} *prefix,\kw{char} \kw{const} *what,
                             \kw{int} nst,entry\_t\_acc *acc);
\end{code}

\begin{code}
/* Formatting */
ssize\_t entry\_t\_fmt2io (P\_t *pads,Sfio\_t *io,\kw{int} *requestedOut,
                        \kw{char} \kw{const} *delims,entry\_t\_m *m,entry\_t\_pd *pd,entry\_t *rep);
\end{code}

\begin{code}
/* Conversion to XML */
ssize\_t entry\_t\_write\_xml\_2io (P\_t *pads,Sfio\_t *io,entry\_t\_pd *pd,
                               entry\_t *rep,\kw{char} \kw{const} *tag,\kw{int} indent);
\end{code}

\begin{code}
/* Galax Data API */
PDCI\_node\_t *entry\_t\_node\_new (PDCI\_node\_t *parent,\kw{char} \kw{const} *name,\kw{void} *m,
                               \kw{void} *pd,\kw{void} *rep,\kw{char} \kw{const} *kind,\kw{char} \kw{const} *whatfn);
PDCI\_node\_t *entry\_t\_node\_kthChild (PDCI\_node\_t *self,PDCI\_childIndex\_t idx);
\end{code}

\label{figure:library}
\caption{Selected portions of generated library for \texttt{entry\_t}
  declaration from Dibbler data description.}
\end{figure*}
\figref{figure:library} includes selected portions of the generated 
library for the Dibbler \cd{entry_t} declaration.

The \C{} declarations for the in-memory representation, the mask, 
and the parse descriptor all share the structure of the \pads{}
type declaration.  The mapping to \C{} for each is straightforward: 
\kw{Pstruct}s map to \C{} structs with appropriately mapped fields, 
\kw{Punion}s map to tagged unions coded as \C{} structs with a tag field 
and an embedded 
union, \kw{Parray}s map to a \C{} struct with a length field and an 
embedded sequence, \kw{Penums} map to \C{} enumerations, \kw{Poptions} 
to tagged unions, and \kw{Ptypedef}s to \C{} typedefs.  Masks include
auxiliary fields to control behavior at the level of a structured
type, and parse descriptors include extra fields to record the 
state of the parse, the number of detected errors, 
the error code of the first detected error, and the location of that error.

The parse function takes a mask, an
in-memory representation, and a parse descriptor as arguments.  
The mask allows the user to specify with fine granularity
which constraints the parser should check and which portions of the
in-memory representation it should fill in.  This control allows the
description-writer to specify all known constraints about the data
without worrying about the run-time cost of verifying potentially
expensive constraints for time-critical applications.

Appropriate error-handling can be as important as processing
error-free data.  The parse descriptor marks which portions of the
data contain errors and characterizes the detected errors.
Depending upon the nature of the errors and the desired application,
programmers can take the appropriate action: halting the program,
discarding parts of the data, or repairing the errors.

The parser maintains the invariant that if the mask requests
that a data item be verified and set, and if the parse descriptor
indicates no error, then the in-memory representation satisfies the
semantic constraints on the data.

By supporting multiple entry-point parsing, we accommodate larger-scale data.
For a small file, programmers can define a \pads{} type that describes
the entire file and use that type's parsing function to read the whole
file with one call.  For larger-scale data, programmers can sequence
calls to parsing functions that read manageable portions of the file,
\eg{}, reading a record at a time in a loop.  The parsing code generated
for \kw{Parrays} allows users to choose between reading the entire array
at once or reading it one element at a time, again to support parsing
and processing very large data sources.

We discuss details of the generated library in the following section
as we describe its uses.

\section{Pads in practice}
Because the \pads{} language is declarative, we can leverage \pads{}
descriptions to build additional tools besides the core parsing/unparsing
library.  As a result, \pads{} provides direct support for a number
of different uses: computing with the data, developing a high-level
understanding of the data through statistical summaries, converting the data into a more standard form, and querying.  

\subsection{General programs}
We start by showing a simple use of 
the core library to filter and normalize Dibbler data
in \figref{figure:dibbler-filter}. After initializing
the \pads{} library handle and opening the data source, the code sets
the mask to check all conditions in the Dibbler description except the
sorting of the time stamps.  Omitted code reads and writes the header. 
The code then echoes error records to one file and cleaned ones to another.
The raw data has two different representations of unavailable phone numbers:
simply omitting the number altogether, which corresponds to the \cd{NONE}
branch of the \kw{Popt}, or having the value \cd{0} in the data.  
The function \cd{cnvPhoneNumbers} unifies these two representations 
by converting the zeroes into \cd{NONE}s.  The function \cd{entry_t_verify}
ensures that our computation hasn't broken any of the semantic properties
of the in-memory representation of the data.
\begin{figure*}
\begin{small}
\begin{center}
\begin{centercode}
P\_t                  *p;
entry\_t              entry;
entry\_t\_pd           pd;
entry\_t\_m            mask;
    ...
P\_open(&p, 0, 0);
P\_io\_fopen(p, "sirius/data/2004.11.11");
    ...
entry\_t\_m\_init(p, &mask, P\_CheckAndSet);
mask.events.compoundLevel = P\_Set;
    ...
\kw{while} (!P\_io\_at\_eof(p)) \{
  entry\_t\_read(p, &mask, &pd, &entry);
  \kw{if} (pd.nerr > 0) \{
    entry\_t\_write2io(p, ERR\_FILE, &pd, &entry);
  \} \kw{else} \{
    cnvPhoneNumbers(&entry);
    \kw{if} (entry\_t\_verify(&entry)) \{
      entry\_t\_write2io(p, CLEAN\_FILE, &pd, &entry);
    \} \kw{else} \{
      error(2, "Data transform failed.");
    \}
  \}
\}
\end{centercode}

\label{figure:dibbler-filter}
\caption{Code fragment for filtering and normalizing Dibbler data.}
\end{center}
\end{small}
\end{figure*}

Another programmatic use of the core library is to define  
Hancock streams.  Hancock is a domain-specific language for
building persistent profiles of the entities described in transaction 
streams~\cite{hancock-toplas}.  Data analysts at AT\&T use Hancock 
applications to detect fraud and target marketing. 
The observation that defining the transaction
streams was one of the most difficult parts of writing Hancock programs
motivated the \pads{} project. 


\subsection{Statistical summaries}
Before using a data source, analysts must develop an understanding 
of the semantics of that source.  Because documentation is usually
incomplete or out-of-date, this understanding must be developed 
through playing with the data itself.  Typical questions include:
how complete is my understanding of the syntax of the data source,
how many different representations for "data not available" are there,
what is the distribution of values for particular fields, \etc{}.
To address this type of question, we added the notion of an accumulator
to the generated library

\subsection{Format Conversion}
\subsection{Queries}
\section{Implementation}
\section{Performance}
\section{Related work}
\section{Conclusions}
\bibliographystyle{abbrv}
\bibliography{pads} 

\end{document}