\subsection{A Traffic Sniffer with Automated Data Characterization}

It is entirely reasonable to ask whether automated monitoring,
including format inference and anomaly detection, is even
possible. Given that monitoring systems typically require explicit
configuration, automatic monitoring would present a significant
advance over the state of the art.

We have good reason to believe that our approach to automated
monitoring is possible for a wide variety of applications, especially
those that one would expect to encounter in a non-adversarial setting.
To understand the process, consider the following steps individually:

\begin{enumerate}
\item Isolating traffic flows and associating them with applications
\item Deriving formats and deciphering data within flows
\item Detecting anomalous traffic
\end{enumerate}

\paragraph*{Isolating traffic flows} Assuming an appropriate network
vantage point that can see all relevant traffic, application flows can
be determined by noting that for most well-behaved applications, port
selection is not random for one end of the connection. In a
client-server model, the client may choose a randomly-numbered
ephemeral port for communication, but the server's port is likely to
be something that is fixed across communications. This endpoint can be
determined either by observing the TCP handshaking process or by
observing port usage over time -- the fixed port is more likely to
appear across multiple flows than the random port. While some
applications, such as Skype or BitTorrent, may hop across ports to
reduce the chance of detection, our goal is to monitor the custom
network applications that are unlikely to have commercial monitoring
systems available. In contrast, VOIP or BitTorrent traffic can already
be tracked using commercially-available systems. Even other
applications that may have legitimate reasons for hopping ports may be
identifiable by deep packet inspection, which is part of our
system. Given the identification of flows, associating flows with
applications can be performed by having small stub programs running on
the monitored servers. These stubs would use data from \texttt{netstat}
to identify which processes are associated with particular ports and
could examine the process tree in the /proc pseudo-filesystem to trace
the processes to the applications that spawned them. Without these
stubs, the system would still be usable, but could only report traffic
types and port numbers, rather than application names.

\paragraph*{Deciphering Data} Once application flows have been isolated, the
contents of the flows can be examined to begin understanding the data
formats and deciphering their contents. The main observation here is that
data sent on the wire is not likely to be random, but instead stem
from application needs. Applications may send strings, integers or
floating-point numbers, and each of these formats has
characteristic bit patterns. Strings are likely to contain
values representing characters rather than non-printing control
sequences, integers are likely to be popular numbers,
and IEEE floating-point numbers are likely to have specific patterns
in the mantissa and exponent.  Likewise, applications are not likely
to invent entirely new transport formats, so they may transfer raw
binary data for C programs, or marshalled serialized data for Java
programs, or data in particular representations (e.g., XDR) for
specific RPC protocols. Given enough examples of the data from the
applications, and given the space of known data formats, it is
possible to derive likely candidates for the format of the application
flows. These candidate formats can be exported to the operator for
refinement and naming with human-friendly labels, or they can be used
as-is. We intend to use the PADS system to describe the data flows,
which will present a high-level representation of the data which the
operator can modify as needed. By using PADS as the representation,
the rest of the PADS toolchain can be invoked, which will
automatically generate safe parsers for the data and generate
stand-alone tools that can be used to query the archives.

\paragraph*{Detecting anomalies} Once the data streams can be parsed,
volumes of data can be collected for analysis with the understanding
that applications are typically running without incident. By profiling
the data, statistical models can be derived for each field to
understand its typical behavior. Correlations between fields can also
be tested, and these expected values can be presented to the operator
in the same manner as the stream format. Over time,
tighter confidence intervals can be determined for the data, and when
the monitor encounters new data that is outside of the expected range,
it can raise alerts based on the magnitude of the deviation and the
sensitivity defined by the operator. The analysis can include not only 
the data contained within the streams, but also the type of traffic
being handled. For example, if the monitor detects that a certain
record format that normally represents 1\% of the traffic is appearing
much more frequently, an alert can be raised on the frequency of the
record itself, rather than just on the data within the record. Where
possible, operators or application developers can provide
information on the expected range of values and when alerts should be
raised.  However, even without user-provided augmentation, the
montoring system can determined baselines for each
record. Note that this anomaly 
detection is not guaranteed to be complete, but given the choice
between no monitoring or automatic inference, the inference may
provide a wealth of information that would otherwise be lost.

