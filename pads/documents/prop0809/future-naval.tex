\subsection{Future Naval Relevance}
\label{sec:naval}

Future naval information systems will involve a complex, heterogenous
infrastructure with world-wide reach and a vast ecosystem of
interacting, distributed applications.  In order to keep this
ecosystem alive, future networked systems and applications will have
to be monitored to proactively find problems, record/archive system
status, oversee operations, debug new applications and detect malicious
processes.  However, as systems and applications diversify, building
appropriate defensive monitoring infrastructure becomes more
difficult, time-consuming, error-prone, and, ultimately, more costly.

We believe that the best hope for building trustworthy,
high reliability systems in such complex future environments
is to develop new technology capable of partially or fully automating
the construction of monitoring software and systems.  In order to
do this, we have proposed a multi-pronged approach that combines
ideas from the programming languages and systems community:

\begin{itemize}
\item Prong 1:  Develop a new programming language capable of specifying the 
form and properties of data produced by networked systems and 
applications.  {\em Automatically} compile those specifications into 
components for parsing, traversal, transformation, error detection,
statistical analysis and querying needed by monitoring systems.

\item Prong 2:  Provide {\em automatic} support for generating the data 
specifications by inferring them directly from available data.

\item Prong 3:  Build new systems support for {\em automatically}
isolating live traffic flows, associating them with their applications,
learning normal behavior, and detecting anomalous traffic.

\item Prong 4:  Wrap all previous components up in a general, scalable,
and adaptive monitoring system capable of
presenting new and evolving data sources to operators and allowing them to
analyze and visualize data across multiple domains.
\end{itemize}

Together, these four prongs form a blueprint for development of the
next generation of monitoring infrastructure.