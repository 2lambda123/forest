
\begin{quote}
The languages people use to communicate with computers
differ in their intended aptitudes, towards either a particular
application area, or a particular phase of computer use
(high level programming, program assembly, job scheduling,
etc). They also differ in physical appearance, and more
important, in logical structure. The question arises, do the
idiosyncrasies reflect basic logical properties of the situations
that are being catered for? Or are they accidents of history
and personal background that may be obscuring fruitful
developments? This question is clearly important if we are
trying to predict or influence language evolution.

To answer it we must think in terms, not of languages,
but of families of languages. That is to say we must systematize
their design so that a new language is a point chosen
from a well-mapped space, rather than a laboriously devised
construction.

$\qquad$ --- P. J. Landin, {\em The Next 700 Programming Languages}, 
1966~\cite{landin:700}.
\end{quote}

Landin asserts that principled programming language design involves
thinking in terms of ``families of languages'' and choosing from a
``well-mapped space.''  However, when it comes to the domain of
processing ad hoc data, there has been too little work on mapping this
crucial space and understanding the family of languages that
inhabit it.  We have recently begun to tackle this problem
as Landin suggested, by developing a semantic
framework for defining, comparing, and contrasting languages
in our domain~\cite{fisher+:700}.  This semantic framework revolves around the
definition of a Data Description Calculus (DDC),
based on the guiding principles of dependent type theory.  
We show how to give a denotational semantics
to DDC by interpreting DDC
types as parsing functions that map external representations (bits/formatted
data)
to data structures in a typed lambda calculus.  More precisely,
these parsers produce both 
internal representations of the external data and
parse descriptors that pinpoint errors in the original source.
We also simultaneously interpret DDC types as the types for
representations of the internal data and as the types for parse descriptors.
We have already reaped a number of important, practical benefits from this calculus:
\begin{itemize}
\item We have shown how to map the features from several different data 
description languages including \pads, \packettypes{}~\cite{sigcomm00} and
\datascript{}~\cite{gpce02} into the calculus.  This allows us to give
precise comparisons between these languages without worrying about
superficial syntactic details.  It also helps reveal how the useful features
from one language may be assimilated by another.
\item We have used the calculus to help us devise design
principles for this class of languages.  In particular, we
have developed the principle of ``error correlation,'' that formalizes
the correspondence between parsed data and parse descriptors that mark
data errors.  We have proven DDC-produced parsers satisfy this principle
and a number of other important criteria.
\item Analysis of the \pads{} implementation with semantics in hand
has uncovered several bugs and ill-advised design decisions
that have been subsequently fixed.
\item Study of the formal type theory has lead to new features we have 
subsequently implemented including recursion, which requires a subtle 
``contractiveness'' constraint for correctness.  It has also lead us
to reconsider our design of several other features.
\end{itemize}

However, despite these advances, there is a great deal of work to do.

\begin{enumerate}
\item {\bf Formalization of output}.  The current DDC only formalizes the
input or parsing semantics of languages like \pads.  There is
a great need to study and formalize the inverse printing/archiving 
transformation from internal data to external format.  In fact, 
we have recently
uncovered practical problems with output processors that have pushed us in 
this direction. For example, \pads{} descriptions may use regular expressions
to match portions of data and, rather than include that data in the
internal representation, the data is thrown away (\figref{figure:clf}
includes an example of this feature).  On output, since the matched
data has been thrown away, what do we put in its place?  The 
current implementation is broken in this respect and must be fixed.
Similar problems occur when parsing erroneous data and in a couple of other
places.  To understand the full range of problems and propose
a principled solution, we require a sound, rigorous semantic study
of the output processors.

\item {\bf Coherence of input and output}  Once we have defined a semantics
for output, we can use that to find discrepancies between implementation
and idealization.  We can also once again
use it to compare and contrast different programming languages
in the family of ad hoc language processors.  Still, 
how will we judge that the semantics is indeed ``good''?
What general criteria and principles will we use in the design?  The
principles should be sufficiently general that as the family of languages
grows, the basic principles endure and provide guidance through that 
growth.  We require much research in this area, but one principal
we have identified at an intuitive level is that input and output
should be ``coherent'' in some sense.  One might suggest that 
if we parse then print data $D$, the result will be exactly $D$.
However, this principle precludes having the parser fix obvious errors
in the data, or standardize representations of dates, times or other
ad hoc values, for instance.  Hence, we are currently considering
defining coherence between input and output
by requiring that parsers and printers form
a {\em Galois connection}.  This definition would allow parsers and printers to
``improve'' data as they work.  
% The properties of Galois connections
% also tell us that parsing followed by printing followed by parsing is
% equal to parsing.  Hence, while parsers and printers may improve data,
% they reach a stable (or fixed) point after one iteration.

% \item {\bf Set-theoretic semantics} The semantics of classic parsing
% formalisms is normally given in terms of sets of strings.  For example,
% regular expressions, general context free grammars, 
% LL(k) and LR(k) grammars are all usually discussed in terms of
% the set of strings they recognize.  The semantics we have developed so far
% for out dependent type theory is different -- it is given in terms of 
% parsing functions from external data
% into internal data.  One reason we chose to develop our semantics differently
% is that, the dependency in our type theory, which is not present
% in these other formalisms appears to necessitate some degree of translation.
% However, we propose to explore techniques for providing a set-theoretic
% semantics to our language in addition.  Once again, the goal is to better 
% understand the relationship between our language and these traditional
% compiler techniques and view all such languages as a coherent family,
% in Landin's sense.  In doing so we hope to improve
% the effectiveness of data description languages for specific tasks
% such as ad hoc scientific data or, alternatively, program language parsing.

\item {\bf Semantics for monitoring tools}  
From any format specification, the
\pads{} compiler will generate a collection of monitoring tools,
including distributed data aggregation, data querying and data
presentation tools.  The fact that \pads{} will be able to 
generate all these different value tools
from high-level data specifications is the major strengths of our 
approach.  A challenging task is to provide a {\em useful} semantics for 
these higher-level tools.  Doing so will require us to devise
a semantic framework in which types are interpreted as functions or
relations between monitoring data and abstract models of that data.  
For instance, to give semantics to our queries, we must relate
ad hoc systems data to query results.  
To give semantics to graphing functions, we will have to relate
ad hoc data to graphs.  To give semantics to HTML table generators,
we will have to relate ad hoc data to the HTML tables.  
Ideally, we will be able to devise one 
reusable semantic framework that may be instantiated differently to
obtained semantics for these different tools.  Doing so
will take substantial research.
\end{enumerate}
