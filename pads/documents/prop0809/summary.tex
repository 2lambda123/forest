%%%%%%%%%%%%%%%%%%%%%%%%%%%%%%
% \centerline{
% \begin{tabular}{cc}
%           \\[-1ex]
% \multicolumn{2}{c}{Automatic Tool Generation for Ad Hoc Scientific Data} \\
%           & \\[-1ex]
% David Walker (PI)\\
% Princeton University\\[2ex]
% \end{tabular}}
%%%%%%%%%%%%%%%%%%%%%%%%%%%%%%

% \paragraph*{Intellectual Merits:} 

Complex systems must be {\em monitored} to proactively find problems,
record/archive system health, oversee system operation, detect
malicious processes or security violations and perform a myriad of
other tasks.  Unfortunately, application developers may lack the time
or discipline to create monitoring tools and update them as necessary
as the underlying applications change. As a result, complex systems
are often under-monitored, and can fail in ways that the monitoring
system cannot observe and help diagnose.

A unified {\em monitoring infrastructure} to perform such tasks should
be able to include embedded sensors that monitor physical processes,
machine-level monitors to manage server infrastructure, intrusion
detection systems that monitor network traffic, application-level
monitors that can observe applications within a system. Monitoring
should be able to span multiple such systems, spread across a facility
or multiple locations.

The goal of our research is to perform whole-system monitoring for
complex systems, including the development of automated application
monitoring and anomaly detection, for both small-scale and wide-area
systems. Our approach consists of three components: (1) a
non-intrusive application traffic sniffer that performs automated
characterization and analysis, (2) a high-level language, called PADS,
capable of specifying the data that monitoring systems accumulate,
archive, and present to users, and (3) a scalable monitoring system,
CoMon, that can adaptively present new data sources and allow
operators to analyze and visualize data across multiple domains.

This approach addresses a number of shortcomings of modern monitoring
systems. Automated format inference and parsing of ad-hoc data allows
application data to be sniffed on the wire, without having to modify
applications. PADS provides a comprehensive system for describing data
in its native formats, and automatically generating tools to access
the data, both for live and archival data. CoMon works with PADS to
interactively present the data to operators, allowing them to query
it, view it graphically, and analyze trends on the data, without having
to provide manual configuration of the database.

Moving away from manual parsing simplifies the system and provides the
flexibility and security that hand-written parsers do not.  Given a
high-level specification, either provided by the operator or inferred
from the data, PADS will automatically generate a collection of
reliable, secure, and high-performance libraries as well as all of the
monitoring components, such as concurrently {\em ingesting/processing}
data from any number of distributed sources, {\em archiving}
(self-describing) data for later analysis, {\em querying} data to
troubleshoot problems, and {\em displaying} statistical data summaries
so users can monitor system health in real time.  As the system
changes and evolves, implementers can change the high-level
specifications and recompile to automatically obtain an improved
monitoring system.  In addition, since code is automatically
generated, as opposed to hand-written, it will not contain
vulnerabilities that make other systems susceptible to buffer
overflows and related attacks.  Finally, since PADS can describe the
format of any data source, users will be able to automatically
generate monitoring tools that interoperate with legacy software,
data, and devices.  Hence, our research can have an
immediate impact on the productivity of systems implementers by
helping produce the next generation of monitoring systems.  Overall,
our research will combine principled and innovative language design
with high-performance systems engineering, all aimed at solving
pervasive systems monitoring and measurement problems.

% \paragraph*{Broader Impacts:}  Kathleen Fisher, senior personnel, 
% will work with other researchers at AT\&T to transfer our monitoring
% technology to industry.  In addition, we will develop tools for
% monitoring the health of PlanetLab, a global network research testbed
% with 400-450 nodes and 200-250 network experiments running at any
% given time.  Not only will our troubleshooting and diagnostic tools
% will provide feedback to PlanetLab users and administrators, thereby
% improving PlanetLab as a research facility for the entire networking
% community, but these same tools will easily usable by individual
% researchers to monitor their own experiments on PlanetLab. In effect,
% we can provide a complete system to archive history, analyze data, and
% present demonstrations, all from a simple data description.
% 
% PADS will also make a broad impact outside the networking community.
% The kind of {\em ad hoc data} found in monitoring systems also appears
% across the natural and social sciences, including biology, chemistry,
% physics and economics.  The PADS specification language will be used to
% specify the formats of these other data sources and to generate the
% querying and visualization tools that help improve
% the productivity of computational scientists.  To jumpstart this
% research, we have already been meeting with Olga Troyanskaya,
% professor in Princeton's Lewis-Sigler Institute for Integrative
% Genomics, who does computational analysis of protein-protein
% interactions, and with Rachel Mandelbaum, Ph.D. candidate in physics
% who analyzes cosmology data.  It is clear that if funded, the PADS system
% will make a broad impact on their research --- PADS will free them to use
% their world-class skills on their {\em science} 
% as opposed to labouring over development of data processing tools.
% The collaboration between computer science, genomics and physics will
% also be an excellent platform for developing interdisciplinary
% undergraduate research projects.  The PI has a proven track-record for
% following through with undergraduate and graduate educational plans.
% In 2004 and 2005, he organized NSF-sponsored summer schools on secure
% and reliable computing.  Last year, his undergraduate student advisee,
% Rob Simmons, won the Princeton Computer Science Senior Thesis Award.
% 

\section*{Major Tasks and Subtasks by Year}

We have four main tasks, with each task containing a number of
subtasks, such as 1.a, 2.a, etc.  These tasks and subtasks are
presented below, first arranged by major task, and then broken into
years.

\begin{itemize}
\item Task 1 -- PADS compiler development
\item Task 2 -- Classification of Live Data
\item Task 3 -- Application Log File Format Inference
\item Task 4 -- Monitoring system integration
\end{itemize}


\noindent
{\bf Year 1}
\begin{itemize}
\item 1.a New algorithms for compositional data-dependent parsing
\item 2.a Collection of (live) wire data from multiple sources
\item 2.b Tracking of data by application, ports, wire formats
\item 3.a Collection of log files from popular applications
\item 3.b Application and evaluation of prototype inference engine to application data
\item 3.c New algorithms for chunking, tokenization, structure discovery, rewriting of individual logs
\item 4.a Automatic configuration of monitoring system from inferred data format
\end{itemize}

\noindent
{\bf Year 2}
\begin{itemize}
\item 1.b Language support for repository/archive specification
\item 2.c Classification of documented formats from RFCs, file converters, etc
\item 2.d Analysis of wire formats by language/runtime 
\item 3.d Design and implementation of new incremental inference algorithms that can scale to handle log files in range of millions of lines
\item 4.b Algorithms for appropriate classification, trending, and anomaly detection using live data
\end{itemize}

\noindent
{\bf Year 3}
\begin{itemize}
\item 1.c Tool generation for forensic repository analysis
\item 2.e Entropy-based matching of unknown data to existing formats
\item 3.e New algorithms for format inference for collections of log files
\end{itemize}


