\section{Background}
\label{sec:Background}

%%\item Background
%%\begin{itemize}
%%\item Forest Language (primitives, load, store)
%%\item Running example 
%%\item Problems with naive semantics
%%\item Transactions to the rescue
%%\end{itemize}

To explain the design of TxForest, we use a running example drawn from
the field of environmental science, specifically SWAT.  This modeling
tool allows researchers to explore tradeoffs related to land use in a
given watershed.  The model incorporates data related to the topology
of the watershed, current land use, historic precipitation and
temperature levels, measurements of water purity at various locations,
\textit{etc}.  This data is stored in a large collection of files and
directories in the file system.  A typical SWAT installation contains
thousands of files.  The data and required formats are documented in a
654~page manual available on the web~\cite{SWAT-IO-Documentation}. We
call such a structured use of the file system a \textit{filestore}.

The Forest language was designed to describe filestores such as SWAT.
The language includes primitives for describing files, directories,
symbolic links, and associated meta-data such as names, owners,
groups, permissions, sizes, and timestamps.  File contents may be
represented as unstructured strings in a variety of encodings or as
structured data using Pads
descriptions~\cite{fisher+:pads,fisher-walker:icdt}. Forest is
implemented using Haskell's quasi-quotation
mechanism~\cite{Mainland:quasi}.

\begin{figure}
\begin{code}
[txforest|
  \kw{data} Preamble\_f  = File Preamble
  data BSN\_f       = File SwatFile
  data PCP\_f       = File PCP
  data TMP\_f       = File TMP

  data Flow\_f      = File Flow                  
  data RCH\_f       = File RCH                   
  data Deviation\_f = File Deviation     
  ...                                        
  data Swat\_d = Directory 
     \{ cio \kw{is} "file.cio" :: Preamble\_f      
     , basin is <| getBasinPath cio |> :: BSN\_f          
     , pcps  is [f :: PCP\_f | f <- matches (GL "*.pcp") ] 
     , tmps  is [f :: TMP\_f | f <- matches (GL "*.tmp") ] 
     , ...
     \}
|]
\end{code}
\caption{(Partial) TxForest SWAT description. }
\label{fig:SWAT-description}
\end{figure}


\figref{fig:SWAT-description}
shows a (partial) Forest description of the SWAT filestore that we use
as a running example in this paper.
The TxForest declarations appear within \texttt{[txforest|...|]}
quasiquotation brackets, allowing us to embed TxForest within Haskell
code files.  The SWAT description starts by declaring a number of
different file formats by applying the \texttt{File} constructor to
a PADS format description.  (For brevity, we omit the PADS
descriptions).   For example, the TxForest file format
\texttt{Preamble\_f}, which contains SWAT configuration information,
is described by the PADS type \texttt{Preamble}.
The other TxForest file formats in the figure use PADS types to
describe the format of files storing 
physical information about the watershed basin (\texttt{BSN\_f}),
daily precipitation data (\texttt{PCP\_f}),
daily temperature data (\texttt{TMP\_f}),
measured waterflow out of the basin (\texttt{Flow\_f}),
modeled waterflow out of the basin (\texttt{RCH\_f}),
and the deviation between the measured and modeled outflows (\texttt{Deviation\_f}).

Given these file formats, the description specifies the root SWAT
directory, \texttt{Swat\_d}.  This directory includes the master
watershed file, which has internal name \texttt{cio}, on-disk name
\texttt{file.cio}, and a file format described by
\texttt{Preamble\_f}.  The next component of the directory is the
\texttt{basin} field, whose format is described by the \texttt{BSN\_f}
file type.  The on-disk name of the \texttt{basin} component is given
in the \texttt{cio} file.  The Haskell code \texttt{getBasinPath cio},
included in TxForest using the anti-quotation notation
\texttt{<|...|>} extracts this path information from the configuration
file.  The next component in the SWAT directory is internally named
\texttt{pcps}; this component stores daily precipitation information.
It is stored on disk as a collection of files each of which has the
\texttt{PCP\_f} file format and the \texttt{.pcp} file extension.  The
SWAT description uses TxForest's file comprehension and Glob matching
facilities to bind the \texttt{pcp} internal name to the list of files
matching this specification.  Similarly, the \texttt{tmps} component
specifies the collection of files that contain daily temperature
information.

Given such a description, the TxForest compiler generates Haskell data
structures for storing in-memory representations of the on-disk data
and associated metadata, including any discrepancies detected between
the TxForest description and the actual filestore.  TxForest also
generates functions for (lazily) loading the contents of the filestore
into the generated Haskell data structures and for writing the
information in these Haskell data structures back to the datastore in
the required formats.

\cut{Writing structures to disk is a two-step
process.  In step one, a \textit{manifest} function writes the
structure into a temporary space and notes any errors.  In step two, a
\textit{store} function copies the temporary store into the correct
location.  This two-step process allows TxForest to detect errors
without corrupting the mainline file store and lets users determine
whether the errors should halt the writing process.}

The original Forest implementation did not prevent concurrent loads
and stores from causing inconsistencies or corruption.  As the number
of users manipulating a filestore grows, such a laissez-faire approach
becomes untenable.  In this paper, we address this weakness by
integrating transactions into Forest.  Transactional Forest (TxForest)
ensures that all accesses to the filestore mediated by a Forest
description will see and maintain a consistent view by aborting and
restarting transactions that would otherwise have created a
conflict. The design of the description language is unchanged: the
TxForst description in \figref{fig:SWAT-description} is also a valid
Forest description.  The difference between TxForest and Forest lies
in the generated code for loading and storing files and in how the
user accesses the in-memory representations.

The process that SWAT users follow to calibrate their models
illustrates the utility of transactions.  The goal of calibration is
to tune various model parameters stored in the filestore so that the
outflow predicated by SWAT matches the measured outflow as closely as
possible.  Because SWAT is a black-box executable developed over
decades, researchers perform this calibration by searching through the
parameter space: they set the parameters in the filestore, run SWAT,
calculate the deviation between the modeled and predicted value, and
repeat until they find a set of parameters with an acceptable
deviation.  Using TxForest, we are able to parallelize this search.
Each thread atomically copies the SWAT datastore to thread-local
space, sets the parameters, invokes SWAT on the thread-local
filestore, checks if the calculuated deviation is the best found by
any thread so far, and if so, atomically writes the new parameter
values back into the SWAT filestore.


\cut{
An example query that researchers using this tool might ask is ``what
type of land use assignment to a given area of a watershed keeps corn
yield above a threshold, maintains housing capacity above another
threshold, and minimizes nitrate levels in nearby streams.''  The SWAT
approach to solving such queries involves a concurrent black-box
optimization process in which each thread reads the current values of
all relevant parameters from the file system, computes the current
value of the optimization function, and makes local changes, and
re-runs the optimization function. If the new result is higher than
the old one, the tool writes those changes back into the file system.
Figure ~\ref{fig:SWAT-opt-code} shows the key TxForest code that
replicates this process.
}

\begin{figure}
\begin{code}
iterateSWAT :: Path -> IO ()
iterateSWAT workingDir = do
  atomically (copySWATData workingDir)
  newBasin <- modifyBasinParams workingDir
  runSWAT workingDir
  newDeviation <- getNewDeviation workingDir
  atomically (updateDeviation newBasin newDeviation)
\end{code}
\caption{TxForest code to determine if new model parameters reduce the
deviation between modeled and measured outflows.}
\label{fig:SWAT-opt-code}
\end{figure}



