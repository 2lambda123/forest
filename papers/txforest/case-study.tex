\section{SWAT: A Case Study}
\label{sec:swat}
In this section, we describe a case study we have built to show the
use of TxForest in a real-world application.  As mentioned earlier,
SWAT is a tool used by environmental scientists to study the impact of
land use.  Each SWAT datastore contains data related to a single
watershed.  The datastore includes information about topology, land
use, vegetation, precipitation, temperatures, measured waterflows,
\etc{}.  \figref{fig:SWAT-description} gives a TxForest description of
a portion of a SWAT datastore.  

Before environmental scientists can use SWAT to answer questions about
possible crop yields or pollution effects resulting from various land
use policies, they must first calibrate the system so that predicted
values match measured ones.  A key portion of the model that must be
calibrated is the predicted outflow from the watershed.  Basin
parameters that are particularly important in determining outflow
include \texttt{SURLAG}, which is the 
 \textit{surface runoff lag coefficient} that models how much surface
 water reaches the main basin channel 
and \texttt{ESCO}, which is the 
 \textit{soil evaporation compensation factor} that models how ground
 characteristics might impede or enhance evaporation.

Scientists use SWAT as a black box, so to perform calibration, they
perform a search over the parameter space, seeking to minimize the
deviation between the measured outflow and the predicted outflow.
The measured outflow is available in a data file supplied by the
United States Geological Survey.  The predicted outflow is computed by
running SWAT over the datastore with the parameters set to the desired
values.  If the deviation is higher than desired, the scientists
randomly perturb the parameters and rerun the calibration.  

We used TxForest to parallelize this process.  


Each thread searches a
portion of the parameter space.  The threads communicate via a shared file
(with format \texttt{Deviation\_f}) that stores the best deviation
found so far.   

The calibration process starts by initializing the deviation to 
sentinel value that will be replaced by the first thread to finish its
search. 

\begin{figure}
\begin{code}
\hscomment{Start numThread optSWAT threads.}
main :: IO ()
main = do
  setDeviation deviationMax deviationFile  
  replicateM_ numThreads \$ forkIO \$ optSWAT

\mbox{}
\hscomment{Each thread runs iterateSwat numIteration
   times in a thread-local space.}
optSWAT :: IO ()
optSWAT = do
  workingDir <- getThreadTempDirectory
  replicateM_ numIterations \$ iterateSWAT workingDir

\mbox{}
\hscomment{Each iteration atomically copies the filestore
   to its working space, modifies the Basin parameters,
   runs SWAT on the working space, calculates the new 
   deviation, and updates the global deviation atomically.}
iterateSWAT :: Path -> IO ()
iterateSWAT workingDir = do
  atomically (copySWATData workingDir)
  newBsn <- modifyBsnParams workingDir
  runSWAT workingDir
  newDeviation <- getNewDeviation workingDir
  atomically (updateDeviation newBsn newDeviation)

\mbox{}
\hscomment{Copy all data from the SWAT datastore to a
   temporary directory.}
copySWATData :: FilePath -> FTM TxVarFS ()
copySWATData tmpDir = do
    (orig :: Universal_d TxVarFS) <- new () swatDataDir
    (copy :: Universal_d TxVarFS) <- new () tmpDir
    copyOrError orig copy ``Failed to copy SWAT data.''

\mbox{}
modifyBsnParams :: Path -> IO SwatLines
modifyBsnParams workingDir = ...

\mbox{}
getNewDeviation :: Path -> IO Double
getNewDeviation workingDir = ...

\mbox{}
runSWAT :: FilePath -> IO ()
runSWAT path = do
  setCurrentDirectory path
  callProcess swatExe []

\mbox{}
updateDeviation :: SwatLines -> Double -> FTM TxVarFS ()
updateDeviation newBsn newDeviation =  do
  (dInfo :: Deviation_f TxVarFS) <- new () deviationFile
  (devMd, Deviation currentDeviation) <- read dInfo
  guard (currentDeviation > newDeviation)
  (swatRep :: Swat_d TxVarFS) <- new () \$ swatDataDir
  (dirMd, dir) <- read swatRep
  (bsnMd, _  ) <- read \$ basin dir
  writeOrError devInfo     (devMd, Deviation newDeviation) 
               ``Failed to write new deviation.''
  writeOrError (basin dir) (bsnMd, SwatFile  newBsn)       
               ``Failed to update SWAT Basin data.''


\end{code}
\caption{TxForest code.}
\label{fig:SWAT-code}
\end{figure}


\note{To be fixed later: Why do we store the deviation in a file
  instead of in the program doing the search?}

\note{We need to have some results about running this process.  How
  fast is it?  How does it compare with what the scientists were doing
previosly?   Ie, is it in the same ballpark?}
