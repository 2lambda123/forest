We present a core calculus for working with POSIX, which we call IMPOSIX.
We start by discussing some of the reasons why one might want a core calculus
rather than a more advanced, Sewell-esque formalism followed by some
subtleties arising from using POSIX as the underlying filesystem model for 
Forest (\ref{subsec:posix-discussion}).
We then describe the semantics of our core calculus and some of the differences
from standard POSIX (\ref{subsec:posix-semantics}).
In \ref{subsec:posix-translation} we define a translation from TxForest to IMPOSIX 
before finally proving serializability in fully Forested programs (\ref{subsec:posix-proof}).

\subsection{Discussion}
\label{subsec:posix-discussion}

The motivation for using a core calculus rather than a Sewell-esque formalism
is largely simplicity. Sewell-esque formalism are fantastic for large
real-world proofs of complicated systems, but in some situations you don't
need the level of detail or even accuracy that they offer and the simplicity
of dealing with just a core calculus of POSIX is beneficial.

In choosing POSIX as the filesystem model underlying Forest, we
believe we have the large benefit of being immediately applicable to
real-world systems since many filesystems do use POSIX as the core
model. A toy filesystem may have offered more power, but would also have a much
higher barrier to adoption, which goes against one of Forest's core creeds.

However, this also presents a number of subtle problems. For example,
POSIX solely offers advisory file locking (as opposed to mandatory).
There are a variety of arguably good reasons why they choose to do this,
but the effect on Transactional Forest is that we can't offer
transactionality with respect to arbitrary processes making changes on the
filesystem. We could be transactional w.r.t. those adhering to advisory
file locks and we are inherently transactional within Forest threads.

\subsection{Semantics}
\label{subsec:posix-semantics}

In our core calculus, we include the POSIX operations,
open, close, read, readdir, write, remove, test, and lockf.
In most cases these work similarly to POSIX, but with some simplifications,
particularly in regards to errors. For example open cannot fail and
read can only get one type of error whether it fails due to the argument
not being a file descriptor or the file descriptor not pointing to a file.
lockf works on whole files instead of pieces of them and simply allow 
locking and unlocking while test is slightly more powerful in that it
can not only tell you if a path has a file, a directory, or nothing,
but also if it is a directory, whether or not it is empty. 
This is largely done for simplicity, but in practice one could
simply check by opening and trying to read a directory instead. The
exact semantics are described in Figure~\ref{fig:posix-semantics}
along with a standard IMP-like construction for command and
expression evaluation.

\begin{figure}
\caption{POSIX semantics goes here}
\label{fig:posix-semantics}
\end{figure}

\subsection{Translation to IMPOSIX}
\label{subsec:posix-translation}



\subsection{Proof of Serializability}
\label{subsec:posix-proof}

We define a compilation function, which runs on the IMPOSIX translation
and turns it into our transactional code.



Now we move on to proving that compiled atomic statements exhibit mutual
serializability.
