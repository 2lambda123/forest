%
% File systems are a thing
%
Many applications today use file systems to store persistent
data. There are numerous reasons that programmers choose to use file
systems instead of systems specifically designed for managing data,
such as key-value stores or relational databases. File systems are
ubiquitous and have a low barrier to entry, being bundled with all
major operating systems. Programs can manipulate the file system
directly, through standard APIs such as POSIX, and do not have to
first perform tasks such as setting up database accounts, creating
tables, defining schemeas, or loading data into the system. File
system data is portable across operating systems and can be easily
replicated and transferred between them, since it is not ``locked in''
to a custom database representation. For this reason, a large number
of data-intensive users such as scientists have come to rely on file
systems for storing their data.

%
% But they kind of suck
%
At the same time, file systems have several serious limitations that
create practical hurdles in many programs. First, many applications
divide their data across multiple files and directories on the file
system. Having to write explicit code to open, read, and write these
files is tedious for programmers and complicates the logic of many
applications. For example, consider a system that stores web server
logging data in files whose name records the year and month when the
data was recorded and whose contents records the actual requests
received by the web server. Even a simple task such as tabulating up
the total number of requests in a given date range will require
opening and reading a large number of files. Second, APIs such as
POSIX lack constructs for documenting assumptions about the structure
of the file system. Applications whose correctness depends on specific
sets of files and directories being organized in a particular way
essentially have no way to declare and enforce those constraints
(except by writing code to manually traverse the file system and
verify the presence of certain files). Simple errors such as a
misnamed directory or a missing file can easily lead to
application-level errors, but can be difficult diagnose. Third, few
file systems offer constructs for ensuring consistency in the presence
of multiple concurrent users. Although POSIX file locks can be used to
control access to specific files, these primitives are complicated to
use and, like any pessimistic scheme, reduce the degree of concurrency
that can be achieved by the system. 

%
% Forest 1.0
%
Previous work on Forest developed type-based abstractions for file
systems. The programmer writes a high-level specification that
describes the expected structure of the file system, and the Forest
compiler automatically generates an in-memory representation for the
data as well as accompanying ``load'' and ``store'' functions. In
addition, the framework provides generic tools for visualizing,
summarizing, and validating file system data described in
Forest. Forest solves the first two issues discussed above: it
streamlines applications, since they can be written against high-level
datatypes rather than having to use low-level file system APIs; and it
also provides mechanisms for automatically detecting situations when
assumptions about the structure of the file system have been
violated. However, Forest's ``load'' and ``store'' functions are
strictly best effort and do not provide any guarantees about
consistency. When there are multiple concurrent users of the file
system, low-level file system operations may be interleaved
arbitrarily, leading to inconsistent results.

%
% SWAT
%
As an example to illutrate the ways that consistency failures can lead
to incorrect results, consider the following real-world
application. The Soil and Water Assessment Tool (SWAT) is a widely
tool that models the impact of various land management practices in
large watersheds. An environmental scientist might use SWAT to predict
the effects on local rivers and streams that would result from
changing the type of crops and fertilizers used on farms, or by
replacing farmland with a new suburban development. SWAT represents
data about the watershed in a structured directory with a large
collection of files, each recorded in a master ``index'' file and
stored under a specific name and extension in the directory and with a
specific, although varying structure. An attractive feature of SWAT is
that it includes tools and interfaces for connecting to other datasets
and frameworks for modeling watershed dynamics, which allows
scientists calibrate their results and crosscheck predictions.

%
% SWAT Workflow
%
In a typical workflow using SWAT, a scientist might wish to modify
some of the parameters stored in certain data files within certain
parameters, until the overall model is calibrated to a given
theoretical model or external dataset. Another common workflow is
finding the value of a set of input parameters (e.g., land use) to
optimize a set of output parameters (e.g., phosporus runoff). In both
of these cases, the scientist needs to explore the set of parameters
to optimize some value---concretely, this entails modifying the input
parameters stored in ASCII text files, running the SWAT binary to
compute derived data, and then comparing the output parameters, which
are also stored in ASCII text files. Although this is a
straightforward optimization problem, it is not generally feasible to
use techniques such as gradient descent since the function being
computed is encoded as a black-box function and may not even be
convex. Hence, typically scientists solve these problems by exhaustive
trial-and-error search.

%
% Consistency failure
%
Ideally, to improve performance, the search for optimal parameters
could be done in parallel, using multiple threads to explore the
search space. However, since the data is stored on the file system,
this is not safe in general since writes to the file system performed
by multiple threads may be interleaved in arbitrary order, which can
easily lead to incorrect results. For example, if the optimal value
terminates first but a later thread writes its parameters into the
SWAT files, then the computation would halt with sub-optimal
parameters.

%
% Transactions to the rescue!
%
It is not hard to see that the root cause of this problem is Forest's
``best effort'' approach to loading and storing data on the file
system. This paper develops a new version of Forest, Transactional
Forest (TxForest) that remedies this problem by providing a more
powerful abstractions that offer strong consistency
guarantees. TxForest provides an ``atomic'' construct and a new
monadic API for accessing data on the file system. By using these
features, programmers gain the guarantee that all of the low-level
read and write operations needed to implement a transaction will be
executed as if they ran from start to finish in isolation, or not at
all---i.e., TxForest ensures serializability. This has all of the
usual benefits in that programmers can reason locally about the
behavior of each transaction without worrying about harmful
interference from other transactions that might be executing
concurrently.

%
% Implementation
%
There are many possible ways of implementing serializable transactions
including pessimistic schemes based on locking, optimistic schemes
based on buffering reads and writes using a log, and emerging schemes
such as coordination-free transactions. In principle, any of these
could be used as the basis for an implementation of TxForest. We have
built a full working prototype in Haskell based on an optimistic
scheme. It leverages the power of Haskell's type system to track file
system effects in a monad, and provides constructs for executing those
effects atomically, failing if some other transaction has caused their
assumptions to become invalid.

%
% Formalization
%
To establish the correctness of our implementation, we formalized the
subset of POSIX used by the TxForest compiler and run-time system in a
simple calculus and proved that it guarantees serializability.
Although this calculus only models a subset of the overall POSIX API,
it captures a number of its essential features. Hence, we hope that it
might be useful for other work on file system based abstractions.

%
% Evaluation
%
Through a collaboration with environmental scientists, we have built a
TxForest description that captures SWAT data files and used it to
automate the task of searching for optimal parameters, as described
above. This application demonstrates that even our unoptimized version
of TxForest is a useful tool that can be used to solve practical
real-world problems. 

%
% Contributions
%
Overall, the contributions of this paper are as follows:
\begin{itemize}
\item We make the case for developing language-based file system
  abstractions with strong consistency guarantees.
\item We describe the design of TxForest, a particular language that
  realizes these goals in a Haskell DSL.
\item We formalize the key elements of TxForest and POSIX in a core
  calculus and prove that our implementation provides strong
  consistency.
\item We present a prototype implementation of TxForest, and discuss
  using it to build a real-world application for representing and
  optimizing SWAT data. 
\end{itemize}
%
The rest of this paper is structured as follows. The next section
reviews Forest and provides further motivation for our
design. Section~\ref{sec:txforest} presents our design for
TxForest. Section~\ref{sec:formalization} develops a formal model of
POSIX and establishes the correctness of a reference implementation
based on optimistic transactions. Section~\ref{sec:implementation}
discusses our implementation. Section~\ref{sec:swat} describes our
experience building a SWAT application in TxForest. We discuss related
work in Section~\ref{sec:related} and conclude in
Section~\ref{sec:conclusion}.
