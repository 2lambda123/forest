\newif\ifdraft\drafttrue
\newif\ifcolor\colortrue

% For per-person control of tex'ing, put commands like \twocolfalse
% in a file called texdirectives.tex, which we read at this point (if
% it exists).  Note that this file should be left out of the SVN
% repository. 
\makeatletter \@input{texdirectives} \makeatother

\documentclass[nocopyrightspace]{sigplanconf}

\usepackage{alltt}
\usepackage{balance}
\usepackage{amsmath}
\usepackage{amsthm}
\usepackage{amssymb}
\usepackage{code}
\usepackage{color}
\usepackage{tikz}
\usepackage[normalem]{ulem}
\usepackage{url}

\newcommand{\bftt}[1]{{\ttfamily\bfseries{}#1}}
\newcommand{\kw}[1]{\bftt{#1}}
\title{Transactional Forest}

\authorinfo{Submission \#XXX}{}{\vspace*{-4cm}}

\begin{document}

\maketitle


\begin{abstract}
Many applications rely on the file system to store persistent data,
but current programming languages lack convenient constructs for
manipulating file system data. Previous work on the Forest language
developed a type-based abstraction for file systems in which the
programmer writes a high-level specification describing the expected
structure of the file system, and the compiler generates an in-memory
representation for the data and accompanying ``load'' and ``store''
functions. Unfortunately Forest does not provide any consistency
guarantees so if multiple applications are manipulating the file
system concurrently---by far the common case---it can produce
incorrect results.

This paper presents Transactional Forest: an extension to Forest that
enriches the language with seralizable transactions. We present the
design of the language, which is based on a new ``atomic'' construct
and a monad that tracks effects. We formalize the semantics of POSIX
file systems in a simple core calculus and prove the correctness of
our implementation. We discuss our implementation in Haskell and
illustrate its use on a substantial case study: the Soil and Water
Assessment Tool (SWAT), which is a modeling tool used by numerous
hydrologists and environmental scientists.
\end{abstract}

\begin{itemize}
\item Introduction
\begin{itemize}
\item PADS/Forest manifesto  (include term filestore)
\item Problems with Forest 1.0
\item Running example informally
\item Our approach
\item Challenges
\item Contributions and Outline
\end{itemize}
\item Background
\begin{itemize}
\item Forest Language (primitives, load, store)
\item Running example 
\item Problems with naive semantics
\item Transactions to the rescue
\end{itemize}
\item TxForest Language
\begin{itemize}
\item Atomic construct 
\item Transaction monad
\item Varieties of failure
\item Revised running example
\item Guarantees
\end{itemize}
\item Featherweight POSIX
\begin{itemize}
\item Discussion of Sewell-eque formalism vs. core calculi  
\item Showcase subtlety (e.g., weak locking primitives?)
\item Define translation from TxForest to IMPOSIX (defines TxForest's semantics)
\item Reference implementations (data structure locks, lockf, etc.)
\item Prove serializability for fully Forested programs
\end{itemize}
\item Implementation
\item SWAT Case Study
\item Related Work
\item Conclusion
\end{itemize}

\section{Background}
\label{sec:Background}

%%\item Background
%%\begin{itemize}
%%\item Forest Language (primitives, load, store)
%%\item Running example 
%%\item Problems with naive semantics
%%\item Transactions to the rescue
%%\end{itemize}

The Forest language includes primitives for describing files,
directories, symbolic links, and associated meta-data.  Meta-data
includes names, owners, permissions, sizes, and timestamps.  File
contents may be represented as simple strings or as structured data
using Pads descriptions~\cite{fisher+:pads,fisher-walker:icdt}. Forest has been implemented using
Haskell's quasi-quotation mechanism~\cite{Mainland:quasi}.
Figure~\ref{fig:SWAT-description} shows a simple Forest
description of the running example we will use throughout this paper.

Given such a description, the Forest compiler generates functions for
(lazily) loading the contents of the file store into a Haskell data
structure and for writing a Haskell data structure back to disk.
Writing structures to disk is a two-step process.  In step one, a
\textit{manifest} function writes the structure into a temporary space
and notes any errors.  In step two, a \textit{store} function copies
the temporary store into the correct location.  This two-step process
allows Forest to detect errors without corrupting the mainline
file store and lets users determine whether the errors should halt the
writing process.

The initial version of Forest did not attempt to ensure that 
concurrent loads and stores did not cause inconsistencies or
corruption.  As the number of users manipulating the file store grows,
this laissez-faire approach becomes untenable.  To address this
weekness, this paper integrates transactions into Forest.  With
Transactional Forest (TxForest) we ensure that all accesses to the
file store mediated by a Forest description will see and maintain a
consistent view by aborting and restarting transactions that would
otherwise have observed a conflict.

To explain the design of TxForest, we will use the following running
example, drawn from the field of agriculture science.  Specifically,
there is a large community centered on SWAT, a Soil and Water
Assessment Tool~\cite{SWAT}.  Members of
this community use SWAT to explore tradeoffs related to different uses
of land in a given watershed.  The model includes data related to the
topology of the watershed, current land use of regions within the
watershed, historic precipitation and temperature levels, measurements
of water purity at various locations, etc.  This data is stored in a
large collection of files and directories in the file system (XXX:how
many files? how much data?).

An example query that researchers using this tool might ask is
``what type of land use assignment to a given area of a watershed
keeps corn yield above a threshold, maintains housing capacity above
another threshold, and minimizes nitrate levels in nearby streams.''
The SWAT approach to solving such queries involves a concurrent black-box
optimization process in which each thread reads the current values of
all relevant parameters from the file system, computes the current
value of the optimization function, and makes local changes, and re-runs
the optimization function. If the new result is higher than the old
one, the tool writes those changes back into the file system.  Figure
~\ref{fig:SWAT-opt-code} shows Forest code that replicates this process.

\begin{figure}
\begin{code}
Update with relevant parts of SWAT forest description   
Ideally we should only have to change the quotation
forest to txforest to port this code snippet
[forest|
 \kw{type} Stats = \kw{Directory}
   \{ last :: File Last, topk :: File Topk \}
 \kw{type} Dat   = [ s :: Site | s <- \kw{matches} site ]
 \kw{type} Site  = [ d :: Log  | d <- \kw{matches} time ]
 \kw{data} Log = \kw{Directory}
   \{ log \kw{is} coralwebsrv :: Gzip (File CoralLog) \} |]
\end{code}
\caption{Forest SWAT description. }
\label{fig:SWAT-description}
\end{figure}

\begin{figure}
\begin{code}
Update with relevant parts of SWAT optimization code
Not portable, relevant to illustrate the design differences?
\end{code}
\caption{Forest SWAT description. }
\label{fig:SWAT-opt-code}
\end{figure}





\bibliographystyle{plain} 
\balance  
\bibliography{main}

\end{document}
