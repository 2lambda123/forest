\subsection*{Type definitions}

\begin{figure*}
\begin{minipage}[t]{\linewidth}
\begin{displaymath}
\begin{array}{l@{\quad}l@{\quad}l@{\quad}l@{\quad}l@{\quad}l@{\quad}l@{\quad}l@{}}
n \in \mathbb{N} & b \in \mathbb{B} & p \in \textbf{Path} & x \in \textbf{Var} &
ch \in \textbf{Character} & w \in \textbf{String} &  F \in \textbf{FileStore} 
\end{array}
\end{displaymath}
\centerline{\textbf{Metavariable Conventions}}
\end{minipage}
\begin{minipage}[t]{0.24\linewidth}
\begin{displaymath}
\begin{array}{l@{~}l@{\,}ll@{}}
 \text{recType} & ::= & Read(n,c,b) \\
 & \mid & Write(n,w) \\
 & \mid & Remove(p) \\
 & \mid & Test(p)
\end{array}
\end{displaymath}
\centerline{\textbf{Recording Type}}
\end{minipage}
\begin{minipage}[t]{0.24\linewidth}
\begin{displaymath}
\begin{array}{l@{~}l@{\,}ll@{}}
 \text{testCond} & ::= & TFile \\
 & \mid & TDir(b) \\
 & \mid & TNone \\
 & \mid & TUnknown
\end{array}
\end{displaymath}
\centerline{\textbf{Test Condition Type}}
\end{minipage}
\begin{minipage}[t]{0.24\linewidth}
\begin{displaymath}
\begin{array}{l@{~}l@{\,}ll@{}}
 \text{readCond} & ::= & RKnown \\
 & \mid & RUnknown \\
 & \mid & RPartial \\
 & &
\end{array}
\end{displaymath}
\centerline{\textbf{Read Condition Type}}
\end{minipage}
\begin{minipage}[t]{0.24\linewidth}
\begin{displaymath}
\begin{array}{l@{~}l@{\,}ll@{}}
 \text{vFST} & ::= & Rem \\
 & \mid & Fi(w) \\
 & \mid & D(l,l',b) \\
 & &
\end{array}
\end{displaymath}
\centerline{\textbf{Virtual FS Type}}
\end{minipage}
\begin{minipage}[t]{0.5\linewidth}
\begin{displaymath}
\begin{array}{l@{\quad}l@{~}l@{\,}ll@{}}
 \text{Values} & v  & ::= & n \mid p \mid w \mid (v_1,\dots,v_n) \mid l \mid \text{fun } x \rightarrow e \\
 \text{Options} & o  & ::= & Some(v) \mid None \\
 \text{Lists} & l  & ::= & v :: l \mid [] \\
 \text{FS Types} & lf & ::= & File(w) \mid Dir(l) \mid Lock(lf) \\
 \text{Expressions} & e  & ::= & x \mid v \mid f~e \mid e_1~e_2 \\ 
 \text{Threads} & t  & ::= & <H,c> \\ 
 \text{Thread Bag} & M  & ::= & \{t_1,\dots,t_n\} \\
 \text{Heap} & H & \in &\textbf{Var} \mapsto v
\end{array}
\end{displaymath}
\centerline{\textbf{More Types}}
\end{minipage}
\begin{minipage}[t]{0.25\linewidth}
\begin{displaymath}
\begin{array}{l@{~}l@{\,}ll@{}}
 c & ::= & \text{skip} \\
 & \mid & c_1;c_2 \\
 & \mid & x := e \\
 & \mid & \ifte{e}{c_1}{c_2}\\
 & \mid & \wdo{e}{c} \\
 & \mid & \text{atomic } c \\ \\ \\
\end{array}
\end{displaymath}
\centerline{\textbf{Commands}}
\end{minipage}
\begin{minipage}[t]{0.25\linewidth}
\begin{displaymath}
\begin{array}{l@{~}l@{\,}ll@{}}
 \text{logT} & ::= & Test(p) \\
 & \mid & RF(p,ch,n) \\
 & \mid & RD(p,w,n) \\
 & \mid & W(p,w,n)  \\
 & \mid & Rm(p)     \\ \\ \\ \\
\end{array}
\end{displaymath}
\centerline{\textbf{Log Type}}
\end{minipage}
\caption{Lots of type definitions}
\end{figure*}

\subsection*{Evaluation Semantics}

%Evaluation Semantics
\begin{figure*}
%TODO: Note notational conventions somewhere
\begin{comment}
(* Expression evaluation: (F,H,e) - o -> (F',H',e') *)

(* Notational convention: 
   write 
     (F,H,e) - None -> (F',H',e') 
   as 
     (F,H,e) --> (F',H',e') 
*)  

(* Command evaluation: (F,H,c) - o -> (F',H',c') *)
C[t] needs to be well-formed...

(* Multi-step command evaluation: (F,H,c) - l -> (F',H',c') *)
\end{comment}
\begin{minipage}[t]{\linewidth}
\begin{displaymath}
\begin{array}{l@{\quad}l@{~}l@{\,}ll@{}}
 \text{Expressions} & E  & ::= & . \mid f~E \mid E~e \mid v~E \\
 \text{Commands} & C  & ::= & . \mid x := C \mid \ifte{C}{c_1}{c_2} \mid \wdo{C}{c} \mid C ; c \\
 \text{Hole type?} & t & ::=  & e \mid c
\end{array}
\end{displaymath}
\end{minipage}
\centerline{\textbf{Evaluation Contexts}}

\begin{minipage}{0.5\linewidth}
\infrule[E-var]
{H(x) = v}
{(F,H,x) \rightarrow (F,H,v)}
\vspace{8pt}

\infrule[E-Call1]
{(F',H',v',o) = [[ f ]]~(F,H,v)}
{(F,H,f~v) \opstep{o} (F',H',v')}
\vspace{8pt}

\infrule[E-Call2]
{(F',H',v') = [[ f ]]~(F,H,v)}
{(F,H,f~v) \rightarrow (F',H',v')}
\end{minipage}
\hfill
\begin{minipage}{0.5\linewidth}
\infrule[E-Lambda]
{v_1 = \text{fun } x \rightarrow e \\
 e'  = e[x \mapsto v_2]}
{(F,H,v_1~v_2) \rightarrow (F,H,e'})
\vspace{8pt}

\infrule[E-Context]
{(F,H,e) \opstep{o} (F',H',e')}
{(F,H,E[e]) \opstep{o} (F',H',E[e'])}
\end{minipage}
\centerline{\textbf{Expression Evaluation}}

\begin{minipage}[t]{0.5\linewidth}
\infrule[C-IfFalse]
{}
{(F,H,\ifte{0}{c_1}{c_2}) \rightarrow (F,H,c_2)}
\vspace{8pt}

\infrule[C-IfTrue]
{n <> 0}
{(F,H,\ifte{n}{c_1}{c_2}) \rightarrow (F,H,c_1)}
\vspace{8pt}

\infrule[C-WhileFalse]
{}
{(F,H,\wdo{0}{c}) \rightarrow (F,H,skip)}
\vspace{8pt}

\infrule[C-WhileTrue]
{n <> 0}
{(F,H,\wdo{n}{c}) \rightarrow (F,H,c;\wdo{n}{c})}
\end{minipage}
\hfill
\begin{minipage}[t]{0.5\linewidth}
\infrule[C-Context]
{(F,H,e) \opstep{o} (F',H',e')}
{(F,H,E[e]) \opstep{o} (F',H',E[e'])}
\vspace{22pt}

\infrule[C-SkipSeq]
{}
{(F,H,skip;c_2) \rightarrow (F,H,c_2)}
\vspace{22pt}

\infrule[C-Assgn]
{H' = H[x \mapsto v]}
{(F,H,x:=v) \rightarrow (F,H',skip)}
\end{minipage}
\centerline{\textbf{Command Evaluation}}

\begin{minipage}[t]{0.5\linewidth}
\infrule[T-Refl]
{}
{(F,H,t)  \opsteps{[]} (F,H,t)}
\vspace{18pt}

\infrule[T-step]
{(F,H,t) \opsteps{l} (F',H',t') \\
(F',H',t') \opstep{o} (F'',H'',t'') }
{(F,H,t) \opsteps{o :: l} (F'',H'',t'')}
\centerline{\textbf{Multi-Step Command Evaluation}}
\end{minipage}
\hfill
\begin{minipage}[t]{0.5\linewidth}
\infrule[M-skip]
{}
{(F,M \uplus \mset{ <H,skip> }) \rightarrow (F,M)}
\vspace{8pt}

\infrule[M-step]
{(F,H,c) \rightarrow (F',H',c')}
{(F,M \uplus \mset{ <H,c> }) \rightarrow (F',M \uplus \mset{ <H',c'> })}
\vspace{8pt}

\infrule[M-atomic]
{(F,H,c) \rightarrow^* (F',H',skip)}
{(F,M \uplus \mset{ <H,atomic c> }) \rightarrow (F',M)} %In txt we have C[atomic c]. Not sure why.
\centerline{\textbf{Concurrent Command Evaluation}}
\end{minipage}

\caption{Operational Semantics}
\end{figure*}

\begin{figure*}
\begin{minipage}[t]{0.5\linewidth}
\begin{displaymath}
\begin{array}{l}
\evalpat{open} = \FUN (F,H,p) \rightarrow \\
\quad \LET S = H(\pfds) \IN \\
\quad \LET n = gensym(S) \IN \\
\quad \LET H' = H[\pfds := S \cup \GOOSE{n \mapsto (p,0)}] \IN \\
\quad (F,H',n,\NONE) \\ \\
\\
\evalpat{read} = \FUN (F,H,n) \rightarrow \\
\quad \IF n \in H(\pfds) \THEN \\
\qquad \LET (p,i) = H(\pfds)(n) \IN \\
\qquad \MWITH{F(p)} \\
\qquad \mid~\SOME(Lock(File(w))) \\
\qquad \mid~\SOME(File(w)) \rightarrow \\
\qqquad \LETIN{H'}{H[\pfds(n) := (p,i+1)]} \\
\qqquad \LETIN{c}{w[i,i+1]} \\
\qqquad (F,H',c,\SOME(RF(p,c,i))) \\
\qquad \mid~\_  \rightarrow (F,H,\ERR,\NONE) \\
\quad \ELSE (F,H,\ERR,\NONE) \\
\\
\evalpat{write} = \FUN (F,H,(n,w)) \rightarrow \\
\quad \IF n \in H(\pfds) \THEN \\
\qquad \LET (p,i) = H(\pfds)(n) \IN \\
\qquad \MWITH{F(p)} \\
\qquad \mid~\SOME(Lock(File(u))) \rightarrow \\
\qqquad \LETIN{F'}{F[p := Lock(File(u[i] := w))]} \\
\qqquad \LETIN{H'}{H[\pfds(n) := (p,i + \abs{w})]} \\
\qqquad (F',H',0,\SOME(W(p,w,i))) \\
\qquad \mid~\SOME(File(u)) \rightarrow \\
\qqquad \LETIN{F'}{F[p := File(u[i] := w)]} \\
\qqquad \LETIN{H'}{H[\pfds(n) := (p,i + \abs{w})]} \\
\qqquad (F',H',0,\SOME(W(p,w,i))) \\
\qquad \mid~\_  \rightarrow (F,H,\ERR,\NONE) \\
\quad \ELSE (F,H,\ERR,\NONE) \\
\\
\evalpat{test} = \FUN (F,H,p) \rightarrow \\
\quad \LET b = \\
\qquad \MWITH{F(p)} \\
\qquad \mid~\SOME(Lock(File(\_))) \\
\qquad \mid~\SOME(File(\_)) \qquad \rightarrow 0 \\
\qquad \mid~\SOME(Lock(Dir([]))) \\
\qquad \mid~\SOME(Dir([])) \qquad \rightarrow 2 \\
\qquad \mid~\SOME(Lock(Dir(\_))) \\
\qquad \mid~\SOME(Dir(\_)) \qquad ~ \rightarrow 1 \\
\qquad  \mid~\_ \qqqquad \qqquad ~~~ \rightarrow ERR \\
\quad \FIN \\
\quad (F,H,b,\SOME(Test(p)))
\end{array}
\end{displaymath}
\end{minipage}
\hfill
\begin{minipage}[t]{0.5\linewidth}
\begin{displaymath}
\begin{array}{l}
\evalpat{close} = \FUN (F,H,n) \rightarrow \\
\quad \LET S = H(\pfds) \IN \\
\quad \IF n \in dom(S) \THEN \\
\qquad \LET H' = H[\pfds := S \BS n] \IN \\
\qquad (F,H',0,\NONE) \\
\quad \ELSE (F,H,\ERR,\NONE) \\
\\
\evalpat{readdir} = \FUN (F,H,n) \rightarrow \\
\quad \IF n \in H(\pfds) \THEN \\
\qquad \LET (p,i) = H(\pfds)(n) \IN \\
\qquad \MWITH{F(p)} \\
\qquad \mid~\SOME(Lock(Dir(l))) \\
\qquad \mid~\SOME(Dir(l)) \rightarrow \\
\qqquad \LETIN{H'}{H[\pfds(n) := (p,i+1)]} \\
\qqquad \LETIN{f}{l[i,i+1]} \\
\qqquad (F,H',f,\SOME(RD(p,f,i))) \\
\qquad \mid~\_  \rightarrow (F,H,\ERR,\NONE) \\
\quad \ELSE (F,H,\ERR,\NONE) \\
\\
\evalpat{lockf} = \FUN (F,H,(p,n)) \rightarrow \\
\quad \IF n \THEN \\
\qquad \MWITH{F(p)} \\
\qquad \mid~\NONE \\
\qquad \mid~\SOME(Lock(\_)) \rightarrow (F,H,\ERR,\NONE) \\
\qquad \mid~\SOME(x) \rightarrow \\
\qqquad \LETIN{F'}{F[p := Lock(x)]} \\
\qqquad (F',H,0,\NONE) \\
\quad \ELSE \\
\qquad \MWITH{F(p)} \\
\qquad \mid~\SOME(Lock(x)) \rightarrow \\
\qqquad \LETIN{F'}{F[p := x]} \\
\qqquad (F',H,0,\NONE) \\
\qquad \mid~\_ \rightarrow (F,H,\ERR,\NONE) \\
\\
\evalpat{remove} = \FUN (F,H,p) \rightarrow \\
\quad \MWITH{F(p)} \\
\quad \mid~\SOME(Lock(File(\_))) \\
\quad \mid~\SOME(Lock(Dir([]))) \\
\quad \mid~\SOME(File(\_)) \\
\quad \mid~\SOME(Dir([])) \rightarrow \\
\qquad \LETIN{F'}{F \BS p} \\
\qquad (F',H,0,\SOME(Rm(p))) \\
\quad \mid~\_  \rightarrow (F,H,\ERR,\SOME(Test(p)))
\end{array}
\end{displaymath}
\end{minipage}
\caption{POSIX Function Semantics}
\end{figure*}


\begin{figure*}
\begin{minipage}[t]{0.5\linewidth}
\begin{displaymath}
\begin{array}{l}
\evalpat{\LL{R}} = \FUN (F,H,n) \rightarrow \\
\quad \LET (p,i) = H(\pfds)(n) \IN \\
\quad \LET b = \\
\qquad \MWITH{H(logFS)(p)} \\
\qquad \mid~\SOME(Fi(\_)) \\
\qquad \mid~\SOME(D(\_,\_,true)) \rightarrow RKnown \\
\qquad \mid~\SOME(D(\_,\_,false)) \rightarrow RPartial \\
\qquad \mid~\SOME(Rm) \\
\qquad \mid~\NONE \rightarrow RUnknown \\
\quad \FIN \\
\quad (F,H,b) \\
\\ %TODO: I really need to double check and make sure that this works
   %It seems sketchy to me everytime.
\evalpat{read\_and\_check} = \FUN (F,H,(n,e)) \rightarrow \\
\quad \LET (p,i) = H(\pfds)(n) \IN \\
\quad \LETR get\_correct~e~last~l~l' = \\
\qquad \MWITH{l,l'} \\
\qquad \mid~[],[] \rightarrow (F,H,e,RD(p,e,i-1)) \\
\qquad \mid~hd :: tl,\_ \rightarrow \\
\qqquad \IFTHEN{hd > last \AND hd < e} \\
\qqqquad (F,H[\pfds(n) := (p,i-1)],hd,RD(p,hd,i-1)) \\
\qqquad \ELSE \IFTHEN{hd = e} \\
\qqqquad (F,H,hd,RD(p,hd,i-1)) \\
\qqquad \ELSE \\
\qqqquad get\_correct~e~last~tl~l' \\
\qquad \mid~[],hd :: tl \rightarrow \\
\qqquad \IFTHEN{hd = e} \\
\qqqquad (F,H,e,\NONE) \\
\qqquad \ELSE \\
\qqqquad get\_correct~e~last~[]~tl \\
\quad \FIN \\
\quad \MWITH{H(logFS)(p)} \\
\quad \mid~\SOME(D(l,l',false)) \rightarrow \\
\qquad \BEGIN \\
\qquad \MWITH{H(log)} \\
\qquad \mid~RD(p',last,\_) :: \_ \WHERE p=p' \rightarrow get\_correct~e~last~l~l' \\
\qquad \mid~\_ \rightarrow get\_correct~e~``"~l~l' \\
\qquad \END \\
\quad \mid~\_ \rightarrow \FW{Can't happen} \\
\end{array}
\end{displaymath}
\end{minipage}
\hfill
\begin{minipage}[t]{0.5\linewidth}
\begin{displaymath}
\begin{array}{l}
\evalpat{\LL{E}} = \FUN (F,H,p) \rightarrow \\
\quad \LET b = \\
\qquad \MWITH{H(logFS)(p)} \\
\qquad \mid~\SOME(Fi(\_)) \rightarrow TFile\\
\qquad \mid~\SOME(D(l,\_,\_)) \WHEN l <> [] \rightarrow TDir(false) \\
\qquad \mid~\SOME(D(\_,\_,false))  \\
\qquad \mid~\NONE \rightarrow TUnknown \\
\qquad \mid~\SOME(D(\_)) \rightarrow TDir(true) \\
\qquad \mid~\SOME(Rm) \rightarrow TNone \\
\quad \FIN \\
\quad (F,H,b) \\
\\
\evalpat{read\_log} = \FUN (F,H,n) \rightarrow \\
\quad \LET (p,i) = H(\pfds)(n) \IN \\
\quad \LETIN{H'}{H[\pfds(n) := (p,i+1)]}
\quad \LET (c,x) = \\
\qquad \MWITH{H(logFS)(p)} \\
\qquad \mid~\SOME(Fi(w)) \rightarrow w[i],Some(RF(p,w[i],i))\\
\qquad \mid~\SOME(D(w,\_,true)) \rightarrow  w[i],Some(RD(p,w[i],i)) \\
\qquad \mid~\_ \rightarrow \FW{read\_log shouldn't have been called} \\
\quad \FIN \\
\quad (F,H',c,x) \\
\\
\evalpat{record} = \FUN (F,H,t) \rightarrow \\
\quad \LET H' = \\
\qquad \MWITH{t} \\
\qquad \mid~Read(fd,v,d) \rightarrow \\
\qqquad \LETIN{(p,i)}{H(\pfds)(fd)} \\
\qqquad \LETIN{index}{\IFTHEN{H(real\_index) = 0 \AND i > 0} ~ i-1 ~ \\
\qqqquad \qqqquad ~ \ELSE H(real\_index)-1} \\
\qqquad \LETIN{ele}{\IFTHEN{d}~ RD(p,v,index) ~ \ELSE RF(p,v,index)} \\
\qqquad \LETIN{H'}{\IFTHEN{v=\ERR}~ H[real\_index := 0] ~ \ELSE H} \\
\qqquad H'[log := ele :: H(log)] \\
\qquad \mid~Write(fd,w) \rightarrow \\
\qqquad \LETIN{(p,i)}{H(\pfds)(fd)} \\
\qqquad H[log := W(p,w,i) :: H(log)] \\
\qquad \mid~Remove(p) \rightarrow H[log := Rm(p) :: H(log)] \\
\qquad \mid~Test(p) \rightarrow H[log := Test(p) :: H(log)] \\
\quad \FIN \\
\quad \LETIN{logFS}{build\_fs (rev H(log))} \\
\quad \LETIN{H'}{H'[logFS := logFS]} \\
\quad (F,H',0)
\end{array}
\end{displaymath}
\end{minipage}
\caption{Built-in Helper Function Semantics}
\end{figure*}

\begin{comment}
\begin{figure*}
\begin{minipage}[t]{0.5\linewidth}
\begin{displaymath}
\begin{array}{l}
\evalpat{\LL{R}} = \FUN (F,H,n) \rightarrow \\
\quad \LET (p,i) = H(\pfds)(n) \IN \\
\quad \LET b = \\
\qquad \MWITH{H(logFS)(p)} \\
\qquad \mid~\SOME(Fi(\_)) \\
\qquad \mid~\SOME(D(\_,\_,true)) \rightarrow RKnown \\
\qquad \mid~\SOME(D(\_,\_,false)) \rightarrow RPartial \\
\qquad \mid~\SOME(Rm) \\
\qquad \mid~\NONE \rightarrow RUnknown \\
\quad \FIN \\
\quad (F,H,b) \\
\end{array}
\end{displaymath}
\end{minipage}
\hfill
\begin{minipage}[t]{0.5\linewidth}
\begin{displaymath}
\begin{array}{l}
\evalpat{\LL{E}} = \FUN (F,H,p) \rightarrow \\
\quad \LET b = \\
\qquad \MWITH{H(logFS)(p)} \\
\qquad \mid~\SOME(Fi(\_)) \rightarrow TFile\\
\qquad \mid~\SOME(D(l,\_,\_)) \WHEN l <> [] \rightarrow TDir(false) \\
\qquad \mid~\SOME(D(\_,\_,false))  \\
\qquad \mid~\NONE \rightarrow TUnknown \\
\qquad \mid~\SOME(D(\_)) \rightarrow TDir(true) \\
\qquad \mid~\SOME(Rm) \rightarrow TNone \\
\quad \FIN \\
\quad (F,H,b) \\
\\
\evalpat{record} = \FUN (F,H,t) \rightarrow \\
\quad \LET H' = \\
\qquad \MWITH{t} \\
\qquad \mid~Read(fd,v,d) \rightarrow \\
\qqquad \LETIN{(p,i)}{H(\pfds)(fd)} \\
\qqquad \LETIN{index}{\IFTHEN{H(real\_index) = 0 \AND i > 0} ~ i-1 ~ \\
\qqqquad \qqqquad ~ \ELSE H(real\_index)-1} \\
\qqquad \LETIN{ele}{\IFTHEN{d}~ RD(p,v,index) ~ \ELSE RF(p,v,index)} \\
\qqquad \LETIN{H'}{\IFTHEN{v=\ERR}~ H[real\_index := 0] ~ \ELSE H} \\
\qqquad H'[log := ele :: H(log)] \\
\qquad \mid~Write(fd,w) \rightarrow \\
\qqquad \LETIN{(p,i)}{H(\pfds)(fd)} \\
\qqquad H[log := W(p,w,i) :: H(log)] \\
\qquad \mid~Remove(p) \rightarrow H[log := Rm(p) :: H(log)] \\
\qquad \mid~Test(p) \rightarrow H[log := Test(p) :: H(log)] \\
\quad \FIN \\
\quad \LETIN{logFS}{build\_fs (rev H(log))} \\
\quad \LETIN{H'}{H'[logFS := logFS]} \\
\quad (F,H',0)
\end{array}
\end{displaymath}
\end{minipage}
\caption{Transaction Helper Function Semantics}
\end{figure*}
\end{comment}

\begin{figure}
\begin{code}
let build\_fs l =
  let rec helper lst map acc =
  match lst with
    | []       -> map
    | hd :: tl ->
      let map,acc =
      (match hd with
      | RF(p,c,i)  ->
        if c = ERR
        then add p Fi(acc) map
        else map
      | RD(p,d,i)  -> 
        if d = ERR
        then add p D(acc,[],true) map
        else map
      | W(p,w,_)   ->
        begin
        match map(dir(p)) with
        | None            -> 
          add dir(p) D([p],[],false) (add p Fi(w) map)
        | Some(D(l,l',b)) -> 
          add dir(p) D(souq (p :: l),rem p l',b) (add p Fi(w) map)
        | \_              -> 
          failwith "Path directory isn't a directory?"
        end
      | Rm(p)      -> 
        begin
        match map(dir(p)) with
        | None            -> 
          add dir(p) D([],[p],false) (add p Rm map)
        | Some(D(l,l',b)) -> 
          add dir(p) D(rem p l,souq (p::l'),b)) (add p Rm map)
        | \_               -> 
          failwith "Path directory isn't a directory?"
        end
      | \_        -> map
      ),(match hd with
      | RF(p,c,i) 
      | RD(p,c,i)  -> 
        if c = EOF
        then []
        else c :: acc
      | \_        -> acc
      )
      in
      helper tl map acc
  in 
  helper l empty ""
\end{code}
\caption{Builds up the virtual FS based on the log}
\end{figure}

\subsection*{Notes to put somewhere}
The heap H has reserved values \textbf{log, logFS, posix\_fds,} and \textbf{real\_index} working as follows: \\
\begin{itemize}
\item H(log) is a log of FS relevant events generated by running compiled functions. 
It is initialized as [].
\item H(logFS) is generated from H(log) and is represents a partial virtual filesystem.
It is a mapping from paths to vFST. It is initialized as \GOOSE{}.
\item H(\pfds) is a set of file descriptors. 
It's a mapping from integers to path,integer (index) pairs.
It is initialized as \GOOSE{}.
\item H(real\_index) is the real index we are currently at when reading.
For various reasons, the real index may be different from the one stored in
the file descriptor set.
It is initialized as 0.
\end{itemize}
The filesystem F has the reserved value \textbf{glog}, which is a list of (time,path) pairs. It starts empty.
