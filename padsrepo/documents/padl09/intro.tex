\section{Introduction}\label{sec:intro}

% \begin{verbatim}

% - problem in general : tools
% - problem specified in POPL paper : Token Ambiguity Problem
% - solution
%   - statistical methods to deal with it
%   - re-work to build algorithm
%     - multiple token sequences: DAG
%     - most likely sequences (with certain constraints to make sure 
%       the token sequences are legal) 
%        => histograms
%     - subdivision algorithm
%       - to select less likely sequences if the most likely sequence 
%         doesn't satisfy some constraints, or it's broken by cutting the DAG
%     - blob finding
% \end{verbatim}

An {\em ad hoc data format} is any data format for which useful data processing
tools do not exist.  Examples of ad hoc data formats include web server logs,
genomic data sets, astronomical readings, financial transaction reports, 
agricultural data and more.  
%Almost everywhere you look, there is some new, 
%semistructured data file someone has created to store or communicate
%information about their domain.

\pads~\cite{fisher+:pads,padsweb} is a declarative 
language that describes the syntax and semantics of ad hoc data
formats.  The \pads{} compiler, developed in \ml{},
reads these declarative descriptions
and produces a series of programming libraries (parser, printer, validator
and visitor) and end-to-end tools (\xml{} translator, query engine, reformatter,
error monitor, \etc{}).  Consequently, \pads{} can
dramatically improve the  
productivity of data analysts who work with ad hoc data.
However, \pads{} is not (yet) a silver bullet.  It takes time for new users
to learn the language syntax and even experienced users can take hours or 
days to develop descriptions for complex formats.  Hence, to further improve
programmer productivity, we have developed a system 
called \learnpads{} that automatically generates end-to-end data
processing tools directly from example 
data~\cite{fisher+:dirttoshovels,fisher+:sigmod08}.  
%\learnpads{} helps users with
%one of the most common applications of the \pads{} system --- the task of
%taking a huge pile of unknown data, understanding its structure and 
%translating it into a known format
%such as \xml{} or CVS for further analysis.
It uses machine learning techniques to infer 
a \pads{} description and then it passes that description on to the
\pads{} compiler.  The compiler will in turn produce its suite of 
custom data processing tools.  Hence \pads{} now serves as a declarative
intermediate language in the tool generation process.

Our past experiments \cite{fisher+:dirttoshovels} have shown that \learnpads{} is highly effective
when the set of tokens it uses matches the tokens used in the unknown
data set.  For instance, when the unknown data set contains 
URLs, dates and messages the inference system will work very well when
its tokenizer contains the correct corresponding definitions
for URLs, dates and messages used in the file.  If the
tokenizer does not contain these elements, inference is still possible,
but the inferred descriptions are generally much more complex
than they would be otherwise. 
% The excessive complexity interferes with
% the overall goal of using the system to understand the data, analyze and 
% transform it.

The challenge then is to develop a general-purpose tokenizer containing
a wide variety of abstractions like URLs, dates, messages,
phone numbers, file paths and more.  The key problem is that
when using the conventional approach 
to building a tokenizer ({\em i.e.,} regular expressions), as we did
in our previous work, the definitions of basic tokens overlap
tremendously.  For example, 
``{\tt January 24, 2008}'' includes a word made up of letters, a couple 
of numbers,
some spaces and English-like punctuation such as the ``{\tt ,}''.  Does that mean 
this string should be treated as an arbitrary text fragment or 
is it a date? Perhaps ``{\tt January}'' an element of an string-based enumeration 
unconnected to integers {\tt 24} and {\tt 2008}?  Perhaps the entire phrase
should be merged with surrounding characters rather than treated
in isolation?  Doing a good job of format
inference involves identifying that the string of characters
{\tt J-a-n-...-0-8} should be treated as an indivisible token and that it is
in fact a date.  More generally, an effective format inference engine
for ad hoc data solves the {\em Token Ambiguity Problem}
-- the problem of determining which substrings of a data file correspond
to which token definitions in the presence of syntactic 
ambiguity.

In this paper, we describe our attempts to solve the token ambiguity problem.
In particular, we make the following contributions:
\begin{itemize}
\item We redesign our format inference algorithm~\cite{fisher+:dirttoshovels} 
to take advantage of information 
generated from an arbitrary statistical token model.
This advance allows the algorithm to process a set of
ambiguous parses, selecting the most likely parses that 
match global criteria.
\item We instantiate the arbitrary statistical token model with
Hidden Markov Models (HMMs), Hierarchical Maximum Entropy Models
(HMEMs) and Support Vector Machines (SVMs) and evaluate 
their relative effectiveness empirically.  We also compare the effectiveness of
these models to our previous approach, which used regular expressions
and conventional prioritized, longest match for disambiguation.
\item We augment our algorithm with an additional phase to
analyze the complexity of inferred descriptions and to simplify
them when description complexity exceeds a threshold relative to
the underlying data complexity.
\end{itemize}

% \paragraph*{Outline.}
% The following section further defines the problem we seek to solve
% by introducing examples of relevant, real-world ad hoc data formats.
% Section~\ref{sec:algo} describes a new algorithm for ad hoc format
% inference that incorporates a generic statistical token model.
% Section~\ref{sec:stats} describes the three statistical models with
% which we have
% instantiated our algorithm.  Section~\ref{sec:eval} evaluates the
% effectiveness of each statistical model using several different
% metrics and Section~\ref{sec:conclude} concludes.

