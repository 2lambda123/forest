\section{Related Work}
\label{sec:related-work}

% Given the importance of ad hoc data, it is perhaps surprising that
% more tools do not support it.  Many, many tools simply assume that the
% data they manipulate is in the right format from the beginning.  If it
% is not, it is up to the user to get the data in the correct format
% themselves --- he or she receives little or no help with the problem.
% For instance, \xml{} and relational databases expect their inputs are
% already in \xml{} or standard formats such as CSV (Comma-Separated
% Values).  The development of \datatype{} is completely complementary
% to research on relational and semi-structured databases as \datatype
% will be explicitly designed to help with the problem (among others) of
% loading one of these databases with data that is not currently in the
% expected format.

As \datatype{} supports both data description and transformation, we
will divide related work between these two functions. It is
interesting to note that, to the best of our knowledge, there are no
other languages that synthesize these two functions as \datatype{}
does.

%\noindent
\subsection{Data Description}
%One might wonder why we do not choose to base our descriptions on
%regular expressions or context-free grammars. First, r
Regular
expressions and context-free grammars, while excellent formalisms for
describing programming language syntax, are not ideal for describing
the sort of ad hoc data we have discussed in this paper.  The main
reason for this is that regular expressions and context free grammars
do not support polymorphism, dependency or semantic constraints ---
key features for describing many ad hoc data formats.

ASN.1~\cite{asn} and related systems~\cite{asdl} allow the user to
specify the {\em logical} in-memory representation and
automatically generate some 
{\em physical} on-disk format. 
Unfortunately, this doesn't help in the slightest when the user is
given a fixed, physical on-disk format and needs to parse or transform
that specific format.  \datatype{} helps solve
the latter problem.

More closely related work includes \erlang{}'s bit
syntax~\cite{erlang} as well as languages like \packettypes~\cite{sigcomm00},
\datascript~\cite{gpce02}, and \blt~\cite{eger:blt}. 
All these systems allow users to write declarative
descriptions of physical data.  These projects were motivated by
parsing networking protocols, \textsc{TCP/IP} packets, and \java{} 
jar-files.  Like \datatype, these languages have a type-directed
approach to describing ad hoc data and permit the user to define
semantic constraints.  In contrast to our work, these systems 
do not have recursion or polymorphism, handle
only binary data and assume the data is error-free.
In addition, they are designed for imperative or object-oriented host
languages, while we have focused here on data descriptions appropriate for a
functional setting.

%\noindent
\subsection{Data Transformation}
There are a number of languages that focus on transforming \xml{}
data including XDuce~\cite{hosoya+:xduce-journal}, 
Cduce~\cite{benzaken+:cduce}, and 
Xtatic~\cite{gapeyev+:XtaticRuntime}, to name just a few.
The closest work to our own is the XDuce
language~\cite{hosoya+:xduce-journal} as it considers
\xml-processing in the context of a
statically typed, functional language with pattern matching. 
However, the types needed for describing \xml{} are quite different
from the types needed to describe ad hoc data.  Moreover,
these languages simply reject ill-formed \xml.  On the whole,
we view \datatype{} as completely complementary to this work ---
one can easily imagine a system in which \datatype{} is used to translate
ad hoc data into \xml{} and then one of these other tools takes over.
 
The Harmony project~\cite{foster+:lenses} is also engaged in 
data transformation.  This time for the purpose of synchronizing
disparate views of the same logical data. 
However, Harmony operates at a higher 
level of abstraction than \datatype.  Once again, the relationship 
with Harmony appears more cooperative than competitive:  One can 
imagine using \datatype{} as a front end that translates data into 
a format Harmony can understand whenceforth Harmony uses
its technology for synchronization.  

% ... yes there are similarities and we want to take advantage of them but, I
% would argue we **do** have a new approach -- one founded on {\em error-aware
% computing}.  All these XML languages assume that their input is error-free
% and simply throw up their hands if not.  This is one feature that makes ad
% hoc data completely different from XML.  Our language focuses on handling
% this novel form of complexity.  This is a major new area of research...

% And of course our language of types and patterns is different in the details
% as well.  

% The tone of the writing should emphasize that the similarities, where they
% exist, are positive (no need to be defensive at all).  We will study these
% other languages and extract every bit of information we can from them.  We
% do not have to worry about being defensive about overlap because there are
% clearly so many novel elements of the design.  We just must make the novelty
% clear.

%%% Local Variables: 
%%% mode: latex
%%% TeX-master: "paper"
%%% End: 
