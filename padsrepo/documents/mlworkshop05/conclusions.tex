\section{Conclusions and Future Work}
\label{sec:conclusion}

In this paper, we have described the importance and pervasiveness of
ad hoc data, as well as the many difficulties in dealing with it.  We
have enumerated a set of principles for any language targeted at the
problem of transforming such data.  These principles include
obliviousness, reification, and soundness. We term languages that
support these properties ``error-aware.''  Based on these principles,
we have sketched the design of \datatype{}, an \ml{}-like language
with a declarative data description sublanguage and an error-aware
transformation language.  We have illustrated the design of each
through a number of practically-motivated examples.

The work described in this paper is very much ``work in progress.''
Consequently, much work remains, including formally specifying the
type system and semantics for the language and building a prototype
implementation.  We intend to leverage the existing \pads{} system to
implement our parsers, and to define our transform language as an
extension of \sml{} so that we may ``compile'' it by translation into
\sml{} proper.



