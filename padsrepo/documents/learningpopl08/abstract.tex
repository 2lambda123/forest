An {\em ad hoc data source} is any semistructured data source for
which useful data analysis and transformation tools are not 
readily available.  Such data must be queried, transformed and displayed by
systems administrators, computational biologists, financial analysts
and hosts of others on a regular basis.  
%These tasks are
%normally tedious and time-consuming.  To improve this situation, we have
%developed a new framework to improve the productivity of data
%analysts and thereby cut the cost of ubiquitous data processing tasks.
In this paper, we demonstrate that it is possible to generate a suite
of useful data processing tools, including a semi-structured query
engine, several format converters, a statistical analyzer and data
visualization routines directly from the ad hoc data itself, 
without any human intervention.  
The key technical contribution of the work is a multi-phase algorithm
that automatically infers the structure of an ad hoc data source and
produces a format specification in the \pads{} data description
language.  
Programmers wishing to implement custom data analysis tools
can use such descriptions to generate printing and parsing libraries
for the data.  Alternatively,  our software infrastructure will
push these descriptions through the \pads{} compiler, creating
format-dependent modules that, when linked with format-independent
algorithms for analysis and transformation, result in
fully functional tools.  We evaluate the performance of
our inference algorithm, showing it scales linearly
in the size of the training data --- completing in seconds, as opposed
to the hours or days it takes to write a description by hand.
We also evaluate the correctness of the algorithm, demonstrating that 
generating accurate descriptions often requires less than 5\% of the
available data.