\chapter{Accumulators}
\label{chap:accumulators}
\cutname{accumulators.html}

Accumulators summarize data inserted into them.  They are useful for
quickly computing a ``bird's-eye'' view of a given data source.  For
each piece of a \pads{} description, the accumulator summarizes the
percentage of errors seen and reports the most frequently seen values.
For example, 
when run on sample web server log data, 
the accumulator report for the length field contains the information
shown in \figref{figure:wsl-accum-report-2}.
%
\begin{figure*}
\begin{small}
\begin{verbatim}
<top>.length : uint32
+++++++++++++++++++++++++++++++++++++++++++
good: 53544   bad: 3824    pcnt-bad: 6.666
min: 35  max: 248591  avg: 4090.234
top 10 values out of 1000 distinct values:
tracked 99.552% of values
 val:  3082 count:  1254  %-of-good:  2.342
 val:   170 count:  1148  %-of-good:  2.144
 val:    43 count:  1018  %-of-good:  1.901
 val:  9372 count:   975  %-of-good:  1.821
 val:  1425 count:   896  %-of-good:  1.673
 val:   518 count:   893  %-of-good:  1.668
 val:  1082 count:   881  %-of-good:  1.645
 val:  1367 count:   874  %-of-good:  1.632
 val:  1027 count:   859  %-of-good:  1.604
 val:  1277 count:   857  %-of-good:  1.601
. . . . . . . . . . . . . . . . . . . . . . 
 SUMMING    count:  9655  %-of-good: 18.032
\end{verbatim}
\end{small}
\caption{Portion of accumulator report for length field of web server
  log data.}
\label{figure:wsl-accum-report-2}
\end{figure*}
%
By default, accumulators track the first 1000 distinct
values seen in the data source and report the frequency
of the top ten values.  In this particular run, 99.552\%
of all values were tracked. 

\section{Operations}
\figref{figure:accumulators} shows the accumulator type
declaration and associated functions for a \pads{} type.
\begin{figure}
\inputCode{code/accumulator}
\caption{Accumulator functions generated for the \texttt{entry\_t}  type.}
\label{figure:accumulators}
\end{figure}
%
These functions have the following behaviors:
\begin{description}
\item[\cd{entry\_t\_acc\_init}] Initializes accumulator data
  structure. This function must be called before any data can be added
  to the accumulator.
\item[\cd{entry\_t\_acc\_reset}] Reinitializes accumulator data
  structure, erasing all information previously stored.
\item[\cd{entry\_t\_acc\_cleanup}] Deallocates all memory associated
  with accumulator.
\item[\cd{entry\_t\_acc\_add}] Inserts argument in-memory
  representation and parse descriptor into argument
  accumulator.  The parse descriptor allows the accumulator to track
  errors as well as legal values.
\item[\cd{entry\_t\_acc\_report2io}] Writes summary report for
  accumulator \cd{acc} to open
  SFIO stream \cd{outstr}.  The argument \cd{prefix} is a descriptive
  string, usually the path to the data being accumulated. If
  \cd{NULL}, the string \literal{\cd{"<top>"}} is used. In
  the accumulator snippet in \figref{figure:wsl-accum-report-2}, this
  path is \texttt{<top>.length}.  The argument \cd{what} is a string
  describing the kind of data.  If \cd{NULL}, a short for of the
  accumulator is used as a default, \eg{} \cd{uint32} for
  \cd{Puint32}. The argument \cd{nst} indicates the nesting
  level. Level zero should be used for a top-level call.  Reporting
  routines bump the nesting level for recrsive report calls that
  describe sub-parts.  Nesting level \literal{\cd{-1}} indicates a
  minimal prefix header should be output, \ie{}, just the prefix
  without any adornment.
  
\item[\cd{entry\_t\_acc\_report}] Writes summary report for
  accumulator \cd{acc} to standard error.  The other arguments are the
  same as for \cd{entry\_t\_acc\_report2io}
\end{description}
\figref{figure:wsl-accum-hand} illustrates a sample use of accumulator
functions for printing a summary of CLF \cd{entry_t}s.  
\begin{figure}
\inputCode{code/wsl-accum-hand}
\caption{Simple use of  accumulator functions for the
  \texttt{entry\_t} type from CLF data.}
\label{figure:wsl-accum-hand}
\end{figure}

\section{Customization}
\label{sec:accumulators-customization}
The \pads{} discpline allows users to customize various aspects of
accumulation by setting the appropriate field in the discpline.  If
\cd{pads} is an active \pads{} handle, then \cd{pads->disc} provides
access to the discipline, which contains the following accumulator
related fields: 

\begin{description}
\item[\cd{acc\_max2track}] is a \cd{Puint64} denoting the default maximum number of distinct values
   for accumulators to track. Setting this field to \cd{P\_MAX_UINT64}
   indicates no limit. Note that the higher the value, the more memory
   accumulators will consume.  By default, the \pads{} system sets
   this value to \cd{1000}. When an \cd{acc\_init} function is
   called on a base-type accumulator \cd{a}, the field
   \cd{a.max2track} is set to \cd{pads->disc->acc_max2track}.
   The value \cd{a.max2track} may be modified by hand after this call
   to force the accumulator \cd{a} to use a non-default value.

\item[\cd{acc\_max2rep}] is a \cd{Puint64} denoting the default number of tracked values for
  accumulators to describe in detail in the generated report. Setting
  this field to \cd{P\_MAX\_UINT64} indicates no limit on the tracked values
  to display. By default, the \pads{} system sets this value to
  ten. When an \cd{acc\_init} function is called on a 
  base-type accumulator \cd{a}, \cd{a.max2rep} is set to
  \cd{pads->disc->acc\_max2rep}. The value \cd{a.max2rep} can be 
  modified by hand after this call to force the accumulator \cd{a} to
  use a non-default value. 

\item[\cd{acc\_pcnt2rep}] is a \cd{Pfloat} denoting the default percent of values for
  accumulators to describe in detail in the generated report. Setting this field to \cd{100.0}
  indicates no limit on the set of tracked values to display.  By
  default, \pads{} sets this value to \cd{100.0}.
  Upon calling an \cd{acc\_init} function on some base-type accumulator \cd{a},
  \cd{a.pcnt2rep} is set to \cd{pads->disc->acc_pcnt2rep}.
  \cd{a.pcnt2rep} can be modified by hand after this call to force
  the accumulator \cd{a} to use a non-default value. 

\end{description}

Note that both \cd{acc\_max2rep} and \cd{acc\_pcnt2rep} set a limit on
the number of tracked values to display.  The reporting stops when
either limit occurs.

Generated accumulators have components that are base-type
accumulators.  Thus, after initializing some generated accumulator
\cd{a}, one could modify \cd{a.foo.bar.max2track} or
\cd{a.foo.bar.max2rep} to change the tracking or reporting of the
\cd{foo.bar} component \cd{a}.

\section{Template Program}
Because generating an accumulator report from a \pads{} description is
a very routine task, \pads{} provides a template program to automate
the task for common data formats.  In particular, the template applies
to data that can be viewed as an optional header followed by a
sequence of records.  Note that any data source that can be read
entirely into memory fits this pattern by considering the source to
have no header and a single body record.

When instantiated, the template program takes an optional command-line
argument specifying the path to the data source. If no argument is
given, it uses a default location for the data specified by the
template user.
The template first reads the optional header, then
reads each record and inserts the resulting value into an
accumulator until the data source is exhuasted, at which point it
prints the accumulator report to standard error.
The code in \figref{figure:wsl-accum}
illustrates using the accumulator template
\cd{template/accum\_report.h}. This template is a \C{} header file
parameterized by a number of macros that permit the user to customize
the template by defining appropriate values for these macros.  For
example, in the code in \figref{figure:wsl-accum}, the user defines the
macros \cd{DEF\_INPUT\_FILE}, \cd{PADS\_TY}, and \cd{IO\_DISC\_MK} to
indicate the default input file, the type of the repeated record in
the data source, and the IO discipline.
The following list describes these and the other macros used by the
accumulator template:

\begin{description}
\cut{
\item[\cd{COPY\_STRINGS}] If defined, this macros indicates that
  string values should be copied into the in-memory representation
  rather than being shared.
  \secref{sec:library-customization-copy-strings} describes this
  choice in more detail.  }
\item[\cd{DATE\_IN\_FMT}] If defined, this macro sets the default
  input format for dates described by \cd{Pdate}.  See
  \secref{sec:library-customization-input-formats} for more
  information.
\item[\cd{DATE\_OUT\_FMT}] If defined, this macro sets the default
  output format for \cd{Pdate} and \cd{Pdate\_explicit}.  See
  \secref{sec:library-customization-output-formats} for more information.
\item[\cd{DEF\_INPUT\_FILE}] If defined, this macros specifies a
  string representation of the path to the default data source.  If no
  path to the data is supplied at the command-line, this is the
  location used for input data.
\item[\cd{EXTRA\_BAD\_READ\_CODE}] If defined, this macro points to a \C{}
  statement that will be executed after any body record containing an
  error.
\item[\cd{EXTRA\_BEGIN\_CODE}] If defined, this macro points to a \C{}
  statement that will be executed after all initialization code is
  performed, but before the optional header is read.
\item[\cd{EXTRA\_DECLS}] This optional macro defines additional \C{}
  declarations that proceed all accumulator code.
\item[\cd{EXTRA\_DONE\_CODE}] If defined, this macro points to a \C{}
  statement that will be executed after generating the accumulator report.
\item[\cd{EXTRA\_GOOD\_READ\_CODE}] If defined, this macro points to a \C{}
  statement that will be executed after any body record not containing an
  error.
\item[\cd{EXTRA\_HEADER\_READ\_ARGS}] If the type of the header record
  was parameterized, this macro allows the user to supply
  corresponding  parameters. 
\item[\cd{EXTRA\_READ\_ARGS}] If the type of the repeated record was
  parameterized, this macro allows the user to supply corresponding
  parameters. 
\item[\cd{IN\_TIME\_ZONE}] If set, this macro specifies the input time
  zone of date types that do not include time zone information.
  See \secref{sec:library-customization-input-time-zone} for more detail.
\item[\cd{IO\_DISC\_MK}] If defined, this macro specifies the
  interpretation of \Precord{} by indicating which IO discpline the
  system should install.  It specifies the discipline by naming the
  function to create the discipline. \secref{sec:io-discipline}
  describes the available IO discipline creation functions.  If the
  user does not define this macro, the system installs the IO
  discipline corresponding to  new-line terminated ASCII records.
\item[\cd{MAX\_RECS}] If defined, this macro specifies an integer that
  limits the number of repeated records that the accumulator program
  should read.
\item[\cd{OUT\_TIME\_ZONE}] If set, this macro specifies the output
  time zone of date types.
  See \secref{sec:library-customization-output-time-zone} for more detail.
\item[\cd{PADS\_HDR\_TY}]  Intuitively, this macro defines the 
  type of the header record in the data source.  This macro need only
  be defined if the data source has a header record.
  It defines a function used by the template
  program to generate the various function and type names derived from
  the name of the header record type, \ie{}, the type of the associated
  in-memory representation, mask, parse descriptor, read function,
  \etc{}
\item[\cd{PADS\_TY}]  Intuitively, this macro defines the 
  type of the repeated record in the data source, \ie{}, the type of
  the value to be accumulated.  This macro must be defined to use the
  accumulator template.  It defines a function used by the template
  program to generate the various function and type names derived from
  the name of the record type, \ie{}, the type of the associated
  in-memory representation, mask, parse descriptor, read function,
  \etc{}
\item[\cd{READ\_MASK}] This macro specifies the mask to use in reading
  the repeated record.  If not defined by the user, the template uses
  the value \cd{P\_CheckAndSet}.
\item[\cd{TIME\_IN\_FMT}] If defined, this macro sets the default
  input format for \cd{Ptime}.  See
  \secref{sec:library-customization-input-formats} for more
  information.
\item[\cd{TIME\_OUT\_FMT}] If defined, this macro sets the default
  output format for \cd{Ptime} and \cd{Ptime\_explicit}.  See
  \secref{sec:library-customization-output-formats} for more information.
\item[\cd{TIMESTAMP\_IN\_FMT}] If defined, this macro sets the default
  input format for \cd{Ptimestamp}.  See
  \secref{sec:library-customization-input-formats} for more
  information.
\item[\cd{TIMESTAMP\_OUT\_FMT}] If defined, this macro sets the default
  output format for the \pads{} types \cd{Ptimestamp} and \cd{Ptimestamp\_explicit}.  See
  \secref{sec:library-customization-output-formats} for more information.
\item[\cd{WSPACE\_OK}] If defined, this macro indicates that leading
  white space for variable-width ASCII integers is okay, as well as
  leading and trailing white space for fixed-width ASCII integers.

\end{description}
