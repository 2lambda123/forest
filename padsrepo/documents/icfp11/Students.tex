The following is the initial portion of a student record, shown here
to illustrate the format.
\begin{code}
KESSEL, PHIL	   BSE   '11
- - - - - - - - - - - - - - - - - - -
Type    Yr  Course     Grade
         1             A+ to F
d        2             P  (  Pass )
t  D  p  3             INC
o  .  .  4  Dept  xxx  N  (Not Avail)
- - - - - - - - - - - - - - - - - - -
d  .  .  1  COS   101  C
o  .  .  1  HOC   101  A
o  .  .  1  GOL   599  A+
...
\end{code}

\begin{code}
\textcolor{red}{-- Auxiliary Haskell functions for PADS description}
ws   = RE "[ \t]+"
ows  = RE "[ \t]*"
junk = RE ".*"
space = ' '
quote = '\''
comma = ','
\end{code}

\begin{code}
\textcolor{red}{-- PADS description of Princeton CS Student Record Format}
[pads| 
  \kw{type} Grade = StringME '[ABCD][+-]?|F|AUD|N|INC|P'
\mbox{}
  \kw{data} Course = Course
    \{ sort         :: StringME '[dto]',      ws
    , departmental :: StringME '[.D]',       ws
    , passfail     :: StringME '[.p]',       ws
    , level        :: StringME '[1234]',          ws
    , department   :: StringME '[A-Z][A-Z][A-Z]', ws
    , number       :: Int \kw{where} <| 100 <= number && number < 600 |>, ws
    , grade        :: Grade,                 junk                               
    \} 
\mbox{}
  \kw{data} MiddleName = MiddleName \{space, middle :: StringME '[a-zA-Z]+[.]?' \}           
\mbox{} 
  \kw{data} FullName(myname::String) = FullName
    \{ lastname   :: StringME '[a-zA-Z]*'  \kw{where} <| lastname ==  myname |>,  comma, ows     
    , firstname  :: StringME '[a-zA-Z]*' 
    , middlename :: Maybe MiddleName
    \}
\mbox{}
  \kw{data} School = AB | BSE
\mbox{}
  \kw{data} Person (myname::String) =
    \{ fullname   :: Student_Name myname,    ws
    , school     :: School,                 ws, quote
    , year       :: StringME '[0-9][0-9]'
    \}
\mbox{}
  \kw{type} Junk    = Line (StringME <|RE ".*"|>)
  \kw{type} Header  = [Junk] \kw{length} 7 
  \kw{type} Trailer = [Junk] \kw{terminator} EOF 
\mbox{}
  \kw{data} Student (name::String) = 
    \{ person  :: Line (Person name)
    , Header  
    , courses :: [Line Course]
    , Trailer
    \}
|]
\end{code}

\begin{code}
\textcolor{red}{-- Auxiliary Haskell functions for Forest description}
template s = s `elem` ["SSSS.txt", "SSS.txt", "sxx.txt", "sss.txt", "ssss.txt"]
not_template = not . template
\mbox{}
getYear :: String -> Integer
getYear s = read (reverse (take 2 (reverse s)))
toStrN i n = (replicate  (n - length (show i)) '0') ++ (show i)
mkClass y = "classof" ++ (toStrN y 2)
\mbox{}
transferRE  = RE "TRANSFER|Transfer"
leaveRE     = RE "LEAVE|Leave"
withdrawnRE = RE "WITHDRAWN|WITHDRAWAL|Withdrawn|Withdrawal|WITHDREW"
cRE         = RE "classof[0-9][0-9]" 
txt         = GL "*.txt"
\end{code}


\begin{code}
\textcolor{red}{-- FOREST description of Princeton CS Department Database}
[forest|
  \textcolor{red}{-- Root of the hierarchy}
  \kw{type} PrincetonCS (y::Integer) = \kw{Directory}
    \{ notes     \kw{is} "README" :: Text
    , seniors   \kw{is} <|mkClass y      |> :: Class y
    , juniors   \kw{is} <|mkClass (y + 1)|> :: Class <| y + 1 |>
    , graduates :: Grads
    \}
\mbox{}
  \textcolor{red}{-- Collection of directories containing graduated students}
  \kw{type} Grads = 
     Map [ c :: Class <| getYear c |> | c <- matches cRE ] 
\mbox{}
  \textcolor{red}{-- Directory containing all students in a particular year}
  \kw{type} Class (y :: Integer) = \kw{Directory}
    \{ bse \kw{is} <|"BSE" ++ (toStrN y 2)|> :: Major
    , ab  \kw{is} <|"AB"  ++ (toStrN y 2)|> :: Major   
    , transfer  \kw{matches} transferRE  :: Maybe Major 
    , withdrawn \kw{matches} withdrawnRE :: Maybe Major 
    , leave     \kw{matches} leaveRE     :: Maybe Major 
    \}
\mbox{}
  \textcolor{red}{-- Collection of files containing all students in a particular major.}
  \kw{type} Major = Map 
    [ s :: File (Student <| dropExtension s |>) 
    | s <- \kw{matches} txt,  <| (not . template) s |>  ]  
|]
\end{code}
\vfill{}

\subsubsection{Generated Description}

Here follows a description generated from a small sample of the
student directory data using the description inference tool.

%\begin{figure}
\begin{code}
\kw{data} transfer = \kw{Directory} \{
\}

\kw{data} wITHDREW = \kw{Directory} \{
    fingertxt \kw{is} "finger.txt" :: File Text
\}

\kw{data} tRANSFER = \kw{Directory} \{
    bEAUCHEMINtxt \kw{is} "BEAUCHEMIN.txt" :: File Text,
    vERSTEEGtxt \kw{is} "VERSTEEG.txt" :: File Text
\}

\kw{data} bSE11 = \kw{Directory} \{
    transfer \kw{is} "transfer" :: transfer,
    bOZAKtxt \kw{is} "BOZAK.txt" :: File Text,
    kESSELtxt \kw{is} "KESSEL.txt" :: File Text,
    ssstxt \kw{is} "sss.txt" :: File Text
\}

\kw{data} aB11 = \kw{Directory} \{
    kADRItxt \kw{is} "KADRI.txt" :: File Text,
    mACARTHERtxt \kw{is} "MACARTHER.txt" :: File Text,
    oRRtxt \kw{is} "ORR.txt" :: File Text,
    sSSStxt \kw{is} "SSSS.txt" :: File Text
\}

\kw{data} classof11 = Directory \{
    aB11 \kw{is} "AB11" :: aB11,
    bSE11 \kw{is} "BSE11" :: bSE11,
    tRANSFER \kw{is} "TRANSFER" :: tRANSFER,
    wITHDREW \kw{is} "WITHDREW" :: wITHDREW
\}
\end{code}
%\caption{Generated description. Type \texttt{File Text} is synonymous
%  with \texttt{TextFile}.}
%\label{fig:generated-description}
%\end{figure}
%\vfill