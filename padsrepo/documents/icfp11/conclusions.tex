\section{Conclusions}
\label{sec:conclusion}

\reviewer{ What about concurrent updates by other Forest programs or
  by users hand-editing the filestore?  Can "invariants" become
  violated as a result of such concurrency?}

In this paper, we propose the idea of extending a modern, high-level
programming language with tightly integrated features for
processing coherent file system fragments, which we call
\filestores{}.  To demonstrate the
potential of this idea, we designed \forest{}, a
domain-specific language 
embedded in \haskell{}
for describing and managing \filestores{}.  

The \forest{} design has been informed by both
theoretical analysis and practical experience.
On the theoretical side, 
we developed a formal
calculus that models the core \forest{} functionality and
we proved that our calculus obeys round-tripping laws
derived from previous work on bi-directional programming paradigms.
On the practical side, 
we illustrated the utility of our design
by describing several example \filestores{}, and showing how
to use these descriptions
to build simple \haskell{} scripts that query,  
analyze, and transform the example data in useful ways.
We also provided evidence that
\forest{} has effective support for building
generic, description-directed tools by 
implementing a number of such 
tools ourselves, including a filestore visualizer,
a generic query interface, an access control checker, and
(circularly) a simple description inference engine.
An ancillary benefit of this 
engineering work is that it
serves as an extensive case study in domain-specific
language design, and, as such, inspired changes in the design of
\template{}.  

For further information about \forest{}, we direct readers to
the \forest{} web site~\cite{forest-web-site}, where they may find
our open source implementation and a number of
additional examples.

