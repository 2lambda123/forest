\section{Conclusions}
\label{sec:conc}

% Our work on \ddc\ has suggested a number of possible directions for
% future work, of which we will briefly describe two. First, we have
% begun to consider which properties we might expect to hold of the
% interaction between the parser and printer of any given description.

% {\em Add more detail here.}

% %\edcom{M: Scan is primitive. speculate on using fancier algorithm.}
% Second, we would like to enhance our support for expressing error
% recovery mechanisms in \ddc. The $\pscann$ type provides a very simple
% error recovery mechanism that is similar to the {\em local} error
% recovery mechanisms of many early versions of the
% \yacc{} parser generator.  Yet, more advanced error
% recovery mechanisms exist that take a substantially different approach
% to error recovery, foor example, {\em global error repair}~\cite{appel:mci}.
% We would like to support such error recovery mechanisms in the \ddc\
% framework. 

% We hypothesize that support for a global mechanism would likely
% require that we parameterize the parsing semantics itself by an error
% recovery mechanism. Furthermore, as the exact operation of the error
% repair, including the choice of which tokens to insert or delete,
% depends on the particular description, we expect that the error
% recovery mechanism itself will be best be specified as an
% interpretation of \ddc.


The \ddc{}, a dependent data description calculus, now has both
parsing and printing semantics.  Moreover, these semantics have been
proven to satisfy important type correctness and canonical forms
theorems.  In the future, we hope to provide semantics for other
tools generated by data description languages in general, and
the \pads{} family of languages in particular.  The ease with which 
we were able to augment our existing semantic framework with a printing
semantics suggests this goal is within our reach.
