%%%%%%%%%%%%%%%%%%%%%%%%%%%%%%
% \centerline{
% \begin{tabular}{cc}
%           \\[-1ex]
% \multicolumn{2}{c}{Automatic Tool Generation for Ad Hoc Scientific Data} \\
%           & \\[-1ex]
% David Walker (PI)\\
% Princeton University\\[2ex]
% \end{tabular}}
%%%%%%%%%%%%%%%%%%%%%%%%%%%%%%

\paragraph*{Intellectual Merits:} 
Complex systems must be {\em monitored} to proactively find problems,
record/archive system health, oversee system operation, detect
malicious processes or security violations and perform a myriad of other tasks.
The {\em monitoring subsystems} 
that perform these tasks take a wide range of forms, 
from embedded sensors that monitor physical processes to
intrusion detection systems that monitor a single network connection
to large-scale systems
that monitor the health and performance of hundreds or thousands of nodes
on the Grid.  
The goal of our research is to improve the reliability, security,
performance, ease-of-construction, and maintenance of system monitors, 
whether they
be for monitoring small-scale single-node processes or wide-area, 
distributed systems.  We will achieve our goal by developing a 
high-level language, called PADS, capable of specifying
the data that monitoring systems accumulate, 
archive, and present to users.  Given a high-level specification,
the PADS compiler will automatically generate a collection of
reliable, secure, and high-performance libraries as well as stand-alone
tools that perform all of the basic functions of a monitoring system
including concurrently {\em fetching} data from any number of distributed
sources, {\em archiving} (self-describing) data for later analysis,
{\em querying} data to troubleshoot problems, and {\em displaying}
statistical data summaries so users can monitor system health in real time.
Moreover, as system requirements change and evolve, implementers may make
simple changes to the high-level specifications and recompile to 
automatically obtain an improved monitoring system.  In addition,
since code is automatically generated, as opposed to hand-written, 
it will not contain vulnerabilities
that make other systems susceptible to buffer overflows and related attacks.  
Finally, since
PADS will have the ability to describe the format of 
any data source, users will be able
to automatically generate monitoring tools that interoperate with
legacy software, legacy data and legacy devices.  
Hence, our
research has the potential to have an immediate impact on the productivity of
systems implementers and to produce the next generation of monitoring systems.
Overall, our research will combine principled and innovative language design 
with high-performance systems engineering, all aimed at solving
pervasive systems monitoring and measurement problems.

\paragraph*{Broader Impacts:}  Kathleen Fisher, senior personnel, 
will work with other researchers at AT\&T to transfer our monitoring
technology to industry.  In addition, we will develop tools for
monitoring the health of PlanetLab, a global network research testbed
with 400-450 nodes and 200-250 network experiments running at any
given time.  Not only will our troubleshooting and diagnostic tools
will provide feedback to PlanetLab users and administrators, thereby
improving PlanetLab as a research facility for the entire networking
community, but these same tools will easily usable by individual
researchers to monitor their own experiments on PlanetLab. In effect,
we can provide a complete system to archive history, analyze data, and
present demonstrations, all from a simple data description.

PADS will also make a broad impact outside the networking community.
The kind of {\em ad hoc data} found in monitoring systems also appears
across the natural and social sciences, including biology, chemistry,
physics and economics.  The PADS specification language will be used to
specify the formats of these other data sources and to generate the
querying and visualization tools that help improve
the productivity of computational scientists.  To jumpstart this
research, we have already been meeting with Olga Troyanskaya,
professor in Princeton's Lewis-Sigler Institute for Integrative
Genomics, who does computational analysis of protein-protein
interactions, and with Rachel Mandelbaum, Ph.D. candidate in physics
who analyzes cosmology data.  It is clear that if funded, the PADS system
will make a broad impact on their research --- PADS will free them to use
their world-class skills on their {\em science} 
as opposed to labouring over development of data processing tools.
The collaboration between computer science, genomics and physics will
also be an excellent platform for developing interdisciplinary
undergraduate research projects.  The PI has a proven track-record for
following through with undergraduate and graduate educational plans.
In 2004 and 2005, he organized NSF-sponsored summer schools on secure
and reliable computing.  Last year, his undergraduate student advisee,
Rob Simmons, won the Princeton Computer Science Senior Thesis Award.

