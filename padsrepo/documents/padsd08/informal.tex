The \padsd{} language allows users to describe streams of data and
meta-data that we refer to as {\em feeds}.  
To introduce the central features of the language,
we work through a series of examples 
drawn from the CoMon and \vizGems{} monitoring systems.

\begin{figure}[t]
\begin{code}
\kw{let} sites = 
  [
    "http://pl1.csl.utoronto.ca:3121";
    "http://plab1-c703.uibk.ac.at:3121";
    "http://planet-lab1.cs.princeton.edu:3121"
  ] 
\kw{feed} simple_comon =
  \kw{base} \{|
    \kw{sources}  = \kw{all} sites;
    \kw{schedule} = every 5 min, starting now, 
               timeout 60.0 sec; 
    \kw{format}   = Comon_format.Source;  
  |\}
\end{code}
\vskip -1.5ex
\caption{Simple CoMon feed (\texttt{simple\_comon.fml}).}
\label{fig:simplecomon}
\end{figure}

\begin{figure}[t]
\begin{code}
\kw{feed} comon_1 =
  \kw{base} \{|
    \kw{sources}  = \kw{any} sites;
    \kw{schedule} = every 5 min, lasting 2 hours;
    \kw{format}   = Comon_format.Source;
  |\}
\end{code}
\vskip -2ex
\caption{Description fragment for data from one of many sites (\texttt{sites.fml}).}
\label{fig:comon_1}
\end{figure}


\subsection{CoMon Feeds}
\figref{fig:simplecomon} presents a simple CoMon statistics feed.
This description 
specifies the \cd{simple\_comon} feed
using the \kw{base} feed constructor.  The \kw{sources} field
indicates that data for the feed comes from \kw{all} of the locations
listed in \cd{sites}.  The \kw{schedule} field specifies that relevant
data is available from each source every five minutes, starting immediately.
When trying to fetch such data, the system may occasionally fail,
either because a remote machine is down or because of network
problems. To manage such errors, the schedule specifies that the
system should try to collect the data from each source for 60 seconds.
If the data does not arrive within that window, the system should give
up. 

The last field in a base feed constructor is the \kw{format} field,
which specifies the syntax of the fetched data by supplying a parser
for it.  In this case, \cd{Comon_format.Source} is  
actually a parser generated from a
\padsml{}~\cite{mandelbaum+:pads-ml} specification file  
(\cd{comon_format.pml}), which we have omitted because of space
constraints.  While it is not strictly necessary for
\padsd{} programmers to use \padsml{} specifications in their descriptions,
and the key ideas in this paper can be understood without a deep knowledge
of \padsml{}, the two languages have been designed to fit together elegantly.
Moreover, several of our generated tools exploit the common underlying 
infrastructure to enable useful data analyses and transformations over
feeds whose formats are specified by \padsml{} descriptions.

% \padsml{}~\cite{mandelbaum+:pads-ml} description named
% \cd{Source} defined in the file \cd{comon_format}.

A simple variation of our first description appears in \figref{fig:comon_1}.
In contrast to \cd{simple\_comon}, which returns data from
{\em all} sites per time slice, \cd{comon\_1} returns data from just 
{\em one} site per time slice. 
This difference between the two is
specified using the \kw{any} constructor instead of the \kw{all}.
%
%
This feature is particularly useful when monitoring the
behavior of replicated systems, such as those using
state machine replication, consensus protocols, or even
loosely-coupled ones such as Distributed Hash Tables (DHTs) 
\cite{Balakrishnan+03:dht}.
In these systems, the same data will be available from any
of the functioning nodes, so receiving results from the first
available node is sufficient. 
%These kinds of monitoring systems
%are useful in the face of partial network unreachability or
%machine failure. Specifying this behavior at the language level
%provides a simpler implementation than network-centric approaches such
%as anycast \cite{anycast}. 
%
The schedule for \cd{comon\_1} indicates the system should fetch data
every five minutes for two hours, using the \cd{lasting} field to indicate
the duration of the feed.  It omits the \cd{starting} and
\cd{timeout} specifications, causing the system to use default
%settings of \cd{now} for the start time and 30 seconds for the
%timeout.  
settings for the start time and the timeout window.  


\begin{figure}
\begin{code}
(* Ocaml helper values and functions *)
\kw{let} config_locations = 
  ["http://summer.cs.princeton.edu/status/ \\
    tabulator.cgi?table=slices/ \\
    table_princeton_comon&format=nameonly"]

(* Feed of nodes to query *)
\kw{feed} nodes =  
  \kw{base} \{|
    \kw{sources}  = \kw{all} config_locations;
    \kw{schedule} = every 5 min;
    \kw{format}   = Nodelist.Source;
  |\}

\kw{let} makeURL (Nodelist.Data x) = 
     "http://" ^ x ^ ":3121"

\kw{let} old_locs = ref []
\kw{let} current list_opt =
  \kw{match} list_opt \kw{with}
    Some l ->  old_locs := l; l
  | None   -> !old_locs

(* Dependent CoMon feed of node statistics *)
\kw{feed} comon =
  \kw{foreach} nodelist \kw{in} nodes 
  \kw{create}
    \kw{base} \{|
      \kw{sources}  = \kw{all} (List.map makeURL 
                     (List.filter Nodelist.is_node 
                     (current (value nodelist))));
      \kw{schedule} = once, timeout 60.0 sec; 
      \kw{format}   = Comon_format.Source;
    |\}
\end{code}
\vskip -2ex
\caption{Node location feed drives
  data collection (\texttt{comon.fml}).}
\label{fig:feedcomon}
\shrink
\end{figure}



\begin{figure}
\begin{code}
\kw{ptype} nodeitem =
  Comment of '#' * pstring_SE(peor)
| Data of pstring_SE(peor)

\kw{let} is_node item = 
  \kw{match} item \kw{with}
  Data _ -> true
  | _ -> false

\kw{ptype} source = 
    nodeitem precord plist (No_sep, No_term)
\end{code}
\vskip -2ex
\caption{\padsml{} description (\texttt{nodelist.pml}) for CoMon configs, 
with one host name per uncommented line.}
\label{fig:nodepml}
\shrink
\end{figure}


So far, our examples have hard-coded the set of locations from
which to gather data.  In reality, however, the CoMon system has an
Internet-addressable configuration file that contains a list of hosts
to be queried, one per non-comment line. This list is periodically
updated to reflect the set of active nodes in PlanetLab. 
\figref{fig:feedcomon} specifies a version of the \cd{comon} feed that
depends upon this configuration information.  To do so, the
description includes an auxiliary feed called \cd{nodes} that describes the
configuration information: it is available from the
\cd{config_location}, it should be fetched every five minutes, and its
format is described by the \padsml{} description \cd{source} given in
the file \cd{nodelist.pml}, which appears in \figref{fig:nodepml}.

The \padsml{} description in \figref{fig:nodepml} 
specifies that \cd{source} is a list (\cd{plist}) of
new-line terminated records (\cd{precord}) each containing a \cd{nodeitem}.  In
turn, a \cd{nodeitem} is either a \cd{'#'} character followed by a
comment string, which should be tagged with the \cd{Comment}
constructor, or a host name, which should be tagged as
\cd{Data}. The description also defines a helper function \cd{is_node},
which returns true if the data item in question is a host name
rather than a comment.  Given this specification, the \cd{nodes} feed
logically yields a list of host names and comments every five minutes.
In fact, because of the possibility of errors, the feed actually
delivers a {\em list option} every five minutes: \cd{Some} if the list is
populated with data, \cd{None} if the data was unavailable at the
given time-slice.  Furthermore, to record provenance information, each
element in the feed is actually a pair of meta-data and the payload
value. 

Given the \cd{nodes} specification, we can define 
the \cd{comon} feed using the notation 
\kw{foreach} \cd{nodelist} \kw{in} \cd{nodes} \kw{create} \cd{...}.
In this declaration, each element of \cd{nodes} is bound in turn to
the variable \cd{nodelist} for use in generating the new feed
declared in ``\cd{...}'' 
The final result of the \kw{foreach} is the union of all such newly
generated feeds.  Both the payload data {\em and} the provenance
meta-data of \cd{nodelist}
may be used in creating the dependent feed.  In this example, 
we use the function \cd{value} to select only the payload portion,
ignoring the meta-data. The complementary function \cd{meta} provides
access to the provenance information.

To complete the construction of the \cd{comon} feed,
a small amount of functional programming allows the user 
to manage errors and strip out comment fields.  Any such simple
transformations may be written directly in \ocaml{}, the host language 
into which we have embedded \padsd{}.  In particular, 
the \cd{current} function checks if
the \cd{nodelist} value is \cd{Some l}, in which case it caches \cd{l}
before returning it as a result.  Otherwise, if the \cd{nodelist} value
is \cd{None} (indicating an error), the most recently cached list of nodes 
is used instead.  The
rest of the \kw{sources} specification filters out comment fields and 
converts the host names to URLs with the required port using the
auxiliary function \cd{makeURL}.

% The \kw{foreach ... create} construct merges the resulting data from
% each machine into a single feed.  As
% before, the format of data fetched from each node matches the
% description \cd{Comon\_format.Source}.  

With this specification, we
expect to get data from all the active machines listed in the
configuration file every five minutes.  We further expect the system to
notices changes in the configuration file within five minutes.


The previous examples all showcased feeds containing a single type
of data.  \padsd{} also provides a datatype mechanism so we may
construct compound feeds containing data of different sorts.  As an
example of where such a construct is useful, the CoMon system includes a
number of administrative data sources.  One such source is a collection of
node profiles, collecting the domain name, IP address, physical
location, \etc, for each node in the cluster.  A second such source is a
list of authentication information for logging into the machines.
These two data sources have different formats, locations, and update
schedules, but system administrators want to keep a combined archive
of the administrative information present in these sources.  If
\cd{sites\_mime} is a feed description of the profile
information and \cd{sites\_keyscan\_mime} is a feed of authentication
information, then the declaration
%
\begin{code}
\kw{feed} sites = 
    Locale \kw{of} sites_mime
  | Keyscan \kw{of} sites_keyscan_mime
\end{code}
%
creates a feed with elements drawn from each of the two 
feeds.  The constructors \cd{Locale} and \cd{Keyscan} tag each item in
the compound feed to indicate its source. 

% \subsection{\vizGems{} Example}
% We now shift to an example drawn from AT\&T's \vizGems{} project.  Like
% the earlier CoMon example, the \cd{stats} feed in \figref{fig:pulse}
% monitors a collection of machines described in a configuration file.
% The \cd{hostList} description has the same form as the \cd{nodes} feed
% we saw earlier, except it draws the data from a local file and only
% once a day.  Unlike the earlier example, we use a \textit{feed
% comprehension} to clean up this feed before creating the \cd{stats} feed.
% The comprehension filters
% the list of hosts to remove comments using the \cd{is_node} function
% and we use the built-in \cd{flatten} function to convert the feed of
% lists of hosts into a simple feed of hosts.  We use the \cd{mk\_host}
% function to remove the \cd{Data} constructor around the hostnames to
% simplify down-stream processing.

% The \cd{stats} feed depends upon the \cd{hosts} feed.  For each host
% \cd{h} in \cd{hosts}, it creates a schedule consisting of a single
% time corresponding to ``now'' with a timeout of one minute.  It uses
% this schedule \cd{s} to describe a compound feed, which pairs two base
% feeds: the first uses the Unix command \cd{ping} to collect network
% statistics about the remote machine while the second performs a remote
% shell invocation using \cd{ssh} to find out statistics about how long
% the machine has been up.  Both of these feeds use the \kw{proc}
% constructor in the \kw{sources} field to compute the data on the fly,
% rather than reading it from a file.  The argument to \kw{proc} is a
% string that is executed in a freshly constructed shell.  The pairing
% constructor for feeds takes a pair of feeds and returns a feed of
% pairs, with elements sharing the same scheduled fetch time being
% paired. This semantics is very convenient in this case, as it produces
% a feed that for each host returns a pair of its ping and uptime
% statistics, grouping together the information for each host. 
% Of course, the full \vizGems{} monitoring application
% uses many more tools than just ping and uptime to probe the remote
% machine; the corresponding feed description has many more branches than this simplified version.


% \begin{figure}
% \begin{code}
% \kw{let} config_locations = 
%     [("file:///arrakis/config/machine_list")];

% \kw{feed} hostList =  
%   \kw{base} \{|
%    \kw{sources}  = \kw{all} config_location;
%    \kw{schedule} = every 1 day;
%    \kw{format}   = Nodelist.Source;
%  |\}

% \kw{let} mk_host (Nodelist.Data h) = h

% \kw{feed} hosts = 
%     \{| mk_host n | n <- (flatten hostList), 
%                    Nodelist.is_node n |\}

% \kw{feed} stats =
%   \kw{foreach} h \kw{in} hosts \kw{create}
%   \kw{let} s = once, timeout 1 min \kw{in}
%   (
%    \kw{base} \{| 
%       \kw{sources}  = \kw{proc} ("ping -c 1 " ^ h);   
%       \kw{format}   = Ping.Source;  
%       \kw{schedule} = s; |\},
%    \kw{base} \{| 
%       \kw{sources}  = \kw{proc} ("ssh " ^ h ^ " uptime");  
%       \kw{format}   = Uptime.Source;  
%       \kw{schedule} = s; |\}
%   ) 
% \end{code}
% \vskip -2ex
% \caption{\texttt{arrakis.fml} Simplified version of \vizGems{} feed.}
% \label{fig:pulse}
% \end{figure}


\subsection{\vizGems{} Example}
\label{sec:vizgems}
We now shift to an example drawn from AT\&T's \vizGems{} project.  Like
the earlier CoMon example, the \cd{stats} feed in \figref{fig:pulse2}
monitors a collection of machines described in a configuration file.
Before we discuss the \cd{stats} feed itself, we first explain some
auxiliary feeds that we use in its definition.   

The \cd{raw\_hostLists} description has the same form as the \cd{nodes}
feed we saw earlier, except it draws the data from a local file once a
day.  We use a \textit{feed comprehension} to define a clean
version of the feed, \cd{host\_lists}.  In the comprehension, the
built-in predicate \cd{is\_good} verifies that no errors occurred in
fetching the current list of machines \cd{hl}, as would be expected
for a local file.  The function \cd{get\_hosts} takes \cd{hl} and uses
the built-in function \cd{get\_good} to extract the payload data from
the provenance and error infrastructure, an operation that is
guaranteed to succeed because of the \cd{is\_good} guard. The function
\cd{get_hosts} then selects the non-comment entries and unwraps them to
produce a list of unadorned host names.   

We next define a feed generator \cd{gen_stats} that yields an
integrated feed of performance statistics for each supplied host.
When given a host \cd{h}, \cd{gen_stats} creates an every-five-minute
schedule lasting twenty-four hours with a one minute timeout. It uses
this schedule to describe a compound feed that pairs two base feeds:
the first uses the Unix command \cd{ping} to collect network
statistics about the route to \cd{h} while the second performs a
remote shell invocation using \cd{ssh} to gather statistics about how
long the machine has been up.  Both of these feeds use the \kw{proc}
constructor in the \kw{sources} field to compute the data on the fly,
rather than reading it from a file.  The argument to \kw{proc} is a
string that the system executes in a freshly constructed shell.  The
pairing constructor for feeds takes a pair of feeds and returns a feed
of pairs, with elements sharing the same scheduled fetch-time being
paired. This semantics produces a compound feed that for each host
returns a pair of its ping and uptime statistics, conveniently
grouping the information for each host.  Of course, the full
\vizGems{} monitoring application uses many more tools than just \cd{ping}
and \cd{uptime} to probe remote machines so the full feed description 
has many more components than this simplified version.


%% For each list of hosts in the
%% \cd{host_lists} feed, we generate a {\em feed of lists} using 
Finally, we define the feed \cd{stats}.  The most interesting piece
of this declaration is the {\em list feed comprehension}, given in
square brackets, that we use to generate a feed of lists. 
Given a host list element \cd{hl}, the right-hand side of the comprehension
uses the \cd{value} function to extract the payload from the meta-data and
then considers each host \cd{h} from the resulting list in turn.  The left-hand side of
the comprehension uses the \cd{gen_stats} feed generator to construct
a feed of the statistics for \cd{h}.  The list feed comprehension then
takes this collection of statistics feeds and converts them into a
single feed, where each entry is a list of the statistics for
the machines in \cd{hl} at a particular scheduled fetch-time.  
We call each such entry a \textit{snapshot} of the system.
The resulting feed makes it easy for down-stream users to perform
actions over snapshots, relieving them of the burden of having to
implement their own multi-way synchronization.
Given the list feed comprehension, the \kw{foreach...create} construct
generates a feed of snapshots from the feed of host lists.  
%Whenever a
%new host list arrives, the \kw{foreach...update} construct
%terminates the snapshot feed from the old host list and starts
%generating a new snapshot feed from the new host list.  

% Note the difference between the \kw{foreach...create} construct 
% from the CoMon example and the \kw{foreach...update} construct.  The
% create form generates a collection of feeds and merges their
% contents into a single all-inclusive feed.  The update form
% generates a collection of feeds and produces a single feed by
% concatenating the collection, stopping one feed as soon as the next is
% generated. 
% We have found the create form to be useful when the actual arrival
% times of the argument feed are regular because the regularity means we
% can give a finite schedule for the dependent feed.  In contrast, the
% update form is useful when the argument feed is irregular and we must
% give an infinite schedule for the dependent feed to ensure we get the
% desired values.  We define precise semantics for the create and update
% forms in \secref{sec:semantics}.


% construct then merges this list of
% feeds into a single feed with data for all the hosts.  This merged
% feed delivers values for a given host list \cd{hl} until the next host
% list \cd{hl}' arrives from the feed \cd{host\_lists}, at which point
% it stops collecting data from the feeds generated from \cd{hl} and
% starts gathering data from the machines in \cd{hl}'.

% With this specification, we expect the system to return data for each
% machine listed in the configuration file every five minutes.  We will
% notice changes in the configuration file once a day.

\begin{figure}
\begin{code}
\kw{let} config_locations =
  [("file:///arrakis/config/machine_list")];

\kw{feed} raw_hostLists =  
  \kw{base} \{|
   \kw{sources}  = \kw{all} config_locations;
   \kw{schedule} = every 24 hours; 
   \kw{format}   = Hosts.Source;   |\}

\kw{let} get_host (Hosts.Data h) = h
\kw{let} get_hosts hl =
    List.map get_host 
     (List.filter Hosts.is_node hl)

\kw{feed} host_lists = 
  \{| get_hosts (get_good hl) | 
     hl <- raw_hostLists, is_good hl |\}

\kw{feed} gen_stats (string h) = 
  \kw{let} s = every 5 mins, 
          timeout 1 min, 
          lasting 24 hours \kw{in}
  (
   \kw{base} \{| 
     \kw{sources}  = \kw{proc} ("ping -c 1 " ^ h);   
     \kw{format}   = Ping.Source;  
     \kw{schedule} = s; |\},
   \kw{base} \{| 
     \kw{sources}  = \kw{proc} ("ssh " ^ h ^ " uptime");  
     \kw{format}   = Uptime.Source;  
     \kw{schedule} = s; |\}
  )

\kw{feed} stats =
  \kw{foreach} hl \kw{in} host_lists \kw{create}
     [ gen_stats (h) | h <- value hl ]
\end{code}
\caption{Simplified version of \vizGems{} feed (\texttt{arrakis.fml}).}
\label{fig:pulse2}
\shrink
\end{figure}
