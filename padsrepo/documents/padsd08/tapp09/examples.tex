

%In this section, we describe two examples that we will use throughout
%the paper to motivate and explain our system.

The CoMon~\cite{comon} system, developed at Princeton, 
monitors the health and status of
PlanetLab~\cite{planetlab} by attempting to
fetch data from each of PlanetLab's 800+ nodes every 5 minutes.  
This data ranges from
the node uptime to memory usage to kernel version.  
%The CoMon system takes this
%raw data and transforms it into two different forms, one of which is a
%per-node collection of statistics and the other of which is a
%per-slice ({\em i.e.,} per-application) collection of statistics.
CoMon displays the data to users in tabular form and allows them to
perform a number of simple queries to find, for instance, lightly
loaded nodes, nodes with drifting clocks or nodes with little
remaining disk space.  CoMon also monitors nodes for various
sorts of problems and generates reports of deviant machines or user
programs.
Finally, the data is archived so PlanetLab users can perform their own
custom analyses of historical data.

AT\&T provides a web hosting service.  The infrastructure for this
service includes a variety of hardware components such as routers,
firewalls, load balancing machines, actual web servers, and
databases, replicated and geographically distributed.  Hence, a given
web site may be distributed across a variety of machines running a
variety of operating systems in a variety of locations.  When a
customer signs up for AT\&T's hosting service, part of the contract
specifies what kinds of monitoring AT\&T will provide for the site.
The \vizGems{} infrastructure provides this monitoring
service.  It tracks a variety of resources using a wide array of
measures, including network
bandwidth, packet loss, cpu utilization, disk utilization, memory
usage, load averages, \etc{} For each machine in the hosting service
and for each such resource, the monitoring system archives the values at
regular intervals and issues alerts when the values exceed resource-
and contract-specific levels.  The archive is used to track long-term
behavior of the service, allowing engineers to determine when more
resources need to be provisioned, for example, adding cpus,
memory, or disk space.  It also allows engineers to understand the
``normal'' behavior for a particular site such as daily or seasonal
cycles for a particular site.


%% Notes on the visgems example.
%smaug:/fs/swift/proj/vg/4yitzhak
% inventory file:
% labems-test-inv.txt
%    for each asset, defines its type: linux, ip address, password
%    url1, url2, ip,
%    systype(url,url-win,win32.i386,vmware, solaris.sun4,linux.i386,cisco,cisco3750, alteonsw, alteon,...), 
%    user, password, 
%    snmpcommunity(public,CompuLert,monitor, MT1HostingMgmt!,R1cd4Win+g1A, private), 
%    sysfunc(client,ems), 
%    servicelevel(os,man,mon,soss), 
%    need_tags(eastcoast), nets(ip/port), weight (1000,300),
%    ticketmodel(keep)
%    realid(esxhost-122)
%    scopeinv_port22(22), scopeinv_port443(443), scopeinv_port80(80)
%    implappend_protSNMP(version=1)
% class file:
% parameter.txt
%    for each asset, what type of info to collect and how
%    including what kind of scope machine (windows, linux) to use
%    bindings from inventory file are in scope in single brackets
%    what are double brackets: [[scopeinv_cpu]]?
%       the single brackets mean if there's an inventory entry with
%       that key, find it and replace the thing in brackets with the
%       value. if there's no entry, abort processing that metric
%       rule. the double brackets are similar except that the thing in
%       brackets is assumed to be a prefix. so in the above, the tool
%       searches the inventory for entries with key == scopeinv_cpu*
%       and for each one found, it generates a metric collection entry
%       in the schedule. this is how monitoring of multiple
%       filesystems, or multiple cpus is implemented. the scopes query
%       the assets and collect info about filesystems, and cpus which
%       are sent back to the main server that adds them to the
%       inventory. 
%          scopeinv_cpu
%          scopeinv_fs
%          scopeinv_iface
%          scopeinv_port

%    what are counts: count=10, count=5?
%       these are collection type specific. for example, in PING
%       rules, it means send 5 packets.  in calls to vmstat / mpstat /
%       etc, means collect 5 samples. 


%    what are inst parameters (inst=_total), etc
%       inst goes with the 'var' attribute: var=cpu_used and inst=0
%       would collect data for cpu usage on cpu #0 and return it as
%       metric: cpu_used.0 

%    what are labels used for?
%      CPU Used ([[scopeinv_cpu!All]])
%      CPU System
%      CPU User
%      Number of Threads
%      Pages In
%      Pages Out
%      Run Queue
%      Swap In
%      Swap Out
%      Used Memory
%      they are used for tools like WMI where it's simpler to override
%      the label of the returned stats instead of generating them on
%      the scope. 


%    what are val fields
%       val=* */%v *
%       collection specific, in this case it's a regular expression
%       that means the value is the text after a '/' and before a
%       space. 

%    What are file fields
%       file=loadavg
%       file=vmstat
%       file=stat
%       collection specific, in this case it tells the tool to look for /proc/loadavg etc

%    what are exclude fields?
%       exclude=*:top
%       collection specific, in this case it tells the top tool to not
%       include itself in the top process discovery. 


%    pipe separated
%    servicelevel: man, os, soss, mon, colo
%       monitor fewer things for less expensive levels of support
%    asset machine type: linux.i386
%    scope machine type: linux.i386
%    collected info
%       ping_loss (_main)
%       ping_time (_main)
%       cpu_free
%       cpu_sys
%       cpu_used
%       cpu_usr
%       cpu_wait
%       fs_used
%       memory_free (_total)
%       memory_total (_total)
%       memory_used (_total)
%       os_loadavg (_main)
%       os_nproc (_total)
%       os_nthread (_total)
%       os_nuser (_total)
%       os_pagein (_total)
%       os_pageout (_total)
%       os_runqueue (_total)
%       os_swapin (_total)
%       os_swapout (_total)
%       proc_topcpu (1)
%       swap_free (_total)
%       swap_total (_total)
%       swap_used (_total)
%       tcpip_inpkt
%       tcpip_outpkt
%       tcpip_inerrpkt
%       tcpip_outerrpkt
%       url_avail (_main)
%       url_time (_main)
%       port_avail
%       port_time
%       log.hardware
%       log.console
%       log.application
%       log.system
%       host_cpuused,....
%       pool_cpumax,...
%       guest_numvcpu,...
%       collection

%       any instance starting with '_' is meant to be special, as in
%       'overall' or 'main' metric instance. so you may have
%       cpu_used.0, cpu_used.1, ..., for each cpu and also
%       cpu_used._total that is the average of the individual ones.


%    y/n
%       the y/n is a boolean that says to report or not report the
%       stat value back to the server. you'd set it to 'n' when the
%       metric is't important, but you either need it to generate
%       another metric (using the CALC methods), or to generate an
%       alarm. for example, for network interfaces, we don't really
%       care to chart the in/out errors and discards since they are
%       usually 0. but we still monitor them and when errors do occur
%       we create an alarm. 

%    command: 
%      what is distinction between raw, cooked, and embedded?
%         - raw means run a simple command and return the output,
%         e.g. collect SNMP oid .a.b.c.d and return its value.
%         - cooked means runs a more elaborate tool that interacts with
%         the remote side. for example, most SSH collections are like
%         that because they run either multiple commands or need to
%         parse the results and perform calculations. 
%         - embedded is similar to cooked except that the remote end
%         is assumed to not be a full POSIX shell environment, so the
%         mechanism for collection needs to be a little
%         different. this happens for network switches that support a
%         limited shell type environment. 

%      PING:..., 
%        loss, time
%      SSH:...
%        mpstat, df, free, uptime, top, proc, uptime, netstat, sar,
%        swap, ibmhmc, vmwarei, vmwarevires
%      CALC:...
%        [[!scopeinv_cpu]]
%      PORT:...
%      URL:
%         url=
%      WMI:
%      NOOP
%      SNMP:
%         community
%         version
%         var, label, unit, helper, unique
%    units:
%        %,ms,GB,<empty>, pkts, mbps
%    number, counter
%    alarm spec:  >=100:1:2/2:1/3600:CLEAR:5:2/2
%       <vrange>:<severity>:<m hits/n collections>:<alarm refresh count/time>
%     or
%       CLEAR:<severity>:<m clears/n collections>
%     vrange can be >= v, <= v, [v1,v2], (v1,v2) (inclusive / exclusive intervals)
%     1/3600 means resend this alarm once every 3600 secs, e.g. 1hr.
%     so the above means: alarm if the value is >= 100 for 2 consecutive intervals,
%     refresh the alarm every hour while the condition persists, and clear the alarm
%     if you get 2 intervals < 100.



% scopemgr script assigns a scope based on inventory and class files
% and generates a schedule for the asset.  
% schedules are grouped by customer and scope
%  asset schedule file:
%  labems-test-sched.scope3.txt
% all schedules for a scope are concatenated it single schedule for
% scope

% vg_collector invoked with segment of scheduler for given asset
% examples:
%  ssh-schedule.txt
%  snmp-schedule.txt
% "XML code between <cfg> tags
% "vars" table with variables to collect and how to do it
% "alarms" table with threshold limits for these variables.

% scopes come in linux and windows flavors because hard to monitor
% windows machines from non-windows machines.

% scopes are assigned based on network reachability and grouping by
% tags:
%  a scope will be assigned to an asset if their tag sets intersect
%  assignment also considers "cost" which accounts for bandwidth and
%  load issues.

% if a scope fails, tasks are reassigned to other scopes until it
% comes back on-line
