%\subsection{The ``Off the Shelf'' User}
%\label{sec:tools}
The \padsd{} system provides a suite of ``off-the-shelf'' tools to
help users cope with standard data archiving and
administration needs.  After writing
a \padsd{} description, users can customize these tools by
writing simple {\em configuration files}, such as shown in
\figref{fig:toolconfigs}.   
Each configuration file includes a feed declaration header 
and a sequence of tool specifications. The header specifies the path to the 
feed description file (\cd{comon.fml}) and the name of the feed to be created (\cd{comon}).
Each tool specification starts with the keyword \cd{tool} followed by the
name of the tool (\eg{}, \cd{feedaccum} and \cd{rss}). The body of each tool specification 
lists name-value pairs, where values are \ocaml{} expressions. Some attributes are 
optional, and the compiler fills in a default value for every omitted attribute.
\padsd{} compiles a configuration file into an \ocaml{} program that creates and archives the specified 
feed, configures the specified tools, and applies them to the feed in parallel.
% Multiple 
%tools can simultaneously consume the same feed possible because feeds in
%\padsd{} are functional and immutable. 
In the following paragraphs, we describe the tools we have implemented.

\begin{figure}[t]
\centering
\begin{codebox}
\kw{feed} comon.fml/comon
\kw{tool} feedaccum \{
  minalert  = true;
  maxalert  = true;
  lesssig   = 3;
  moresig   = 3;
  slicesize = 10;
  slicefile = "slice.acc";
  totalfile = "total.acc";
\}
... configs for other tools ...
\end{codebox}
\caption{\texttt{comon.tc}: Example tool configuration file.}
\label{fig:toolconfigs}
\end{figure}

%  path  = ["[?]"; "Mem_info"];


% The most straightforward use of \padsd{} is to create a feed 
% based on a \padsd{} description, and then apply one or more built-in tools 
% to the feed. \padsd{} provides a {\it tool configurations} ({\it configs} for short) to accomplish this task. 
% Figure~\ref{fig:toolconfigs} presents an example config.



 

%without any further programming.
%We now briefly introduce them and discuss their uses. 

%\edcom{Move this bit somewhere?}
%A tool is typically an \ocaml{} module with a standard interface: a tool specification, a feed representation or prefeed compiled from the feed description, and a feed of interest. 

\cut{%%%%%%%%%%
\begin{figure}[t]
\centering
\begin{scodebox}
======================================================
Summary of network transmission errors
======================================================
ErrCode: 1      ErrMsg: Misc HTTP error Count: 12
ErrCode: 5      ErrMsg: Bad message     Count: 27
ErrCode: 6      ErrMsg: No reply        Count: 2

======================================================
Top 10 locations with most network errors
======================================================
Loc: http://planetlab01.cnds.unibe.ch:3121    Count: 2
Loc: http://pepper.planetlab.cs.umd.edu:3121  Count: 2
Loc: http://planetlab3.cs.uchicago.edu:3121   Count: 2
... omitted ...
\end{scodebox}
\caption{Fragment of accumulator output.} \label{fig:acc}
\end{figure} 
}%%%%%%%%%

%===========================================================================
%Summary of network transmission errors
%===========================================================================
%ErrCode: 1      ErrMsg: Misc HTTP error Count: 12
%ErrCode: 5      ErrMsg: Bad message     Count: 27
%ErrCode: 6      ErrMsg: No reply        Count: 2

%===========================================================================
%Top 10 locations with most network errors
%===========================================================================
%Loc: http://planetlab01.cnds.unibe.ch:3121    Count: 2
%Loc: http://pepper.planetlab.cs.umd.edu:3121  Count: 2
%Loc: http://planetlab3.cs.uchicago.edu:3121   Count: 2
%... omitted ...
%===========================================================================
%<top>.<listitem>.Cpu_hog.elt1.Some.elt2 : string
%===========================================================================
%good vals:      554     bad vals:       0       pcnt-bad:       0.00%
%min val: 4      max val: 22     avg val: 11     std dev: 5.29   

%=> Distribution of the top 10 values out of 47 distinct values:
%    Val: "nyu_d"                   Count: 111     pcnt-of-good-vals: 20.04%
%    Val: "ethzcs_q"                Count: 86      pcnt-of-good-vals: 15.52%
%    Val: "princeton_coblitz"       Count: 68      pcnt-of-good-vals: 12.27%
%    ... omitted ...
%    Val: "root"                    Count: 12      pcnt-of-good-vals: 2.17%
%. . . . . . . . . . . . . . . . . . . . . . . . . . . . . . . . . . . . . .
%    SUMMING                        Count: 434     pcnt-of-good-vals: 78.34%

%% ======================================================
%% <top>.<listitem>.Cpu_hog.elt1.Some.elt2 : string
%% ======================================================
%% good vals: 554        bad vals: 0      pcnt-bad: 0.00%
%% min val: 4   max val: 22   avg val: 11   std dev: 5.29   

%% Distribution of top 10 values of 47 distinct values:
%%   Val                Count        Percent of good vals
%% "nyu_d"              111                        20.04%
%% "ethzcs_q"           86                         15.52%
%% "princeton_coblitz"  68                         12.27%
%% ... omitted ...
%% "root"               12                          2.17%
%% . . . . . . . . . . . . . . . . . . . . . . . . . . . 
%% SUMMING              434                        78.34%

    
%% \begin{figure}
%% {\small
%% \begin{verbatim}
%% path :: =  
%%   "<top>"
%% | path.ID  (field/variant name)
%% | path.INT (branch number (from 1) of a tuple)
%% | path.[?] (any one element of array/table) 
%% | path.[*] (all elements of array/table) 
%% | path.[INT] (nth element of array (from 0))
%% | path.[Key] (a table entry indexed by the Key)
%% \end{verbatim}
%% }
%% \caption{Selector path language.}
%% \label{fig:selector-lang}
%% \end{figure}


\textbf{\textit{Archiver.}} The archiver saves the data fetched by a feed 
in the local file system, organizing it according to the structure
of the feed, with one directory per base feed.  
%One directory is created per base feed, in which the archiver stores all the files fetched by that feed. 
It places a catalog in each directory documenting the provenance of 
the data include its location of origin, its scheduled arrival time and the
actual arrival time. The archiver will optionally compress
files.

\textbf{\textit{Printer.}} The printer outputs the contents of a
feed.  If configured to print to a single file, the tool concatenates
successive items with a specified separator. If configured to print to
multiple files, it outputs the contents of each base feed into a
separate file. 


\textbf{\textit{Profiler.}} The profiler monitors performance, reporting
throughput, average network latency and average system latencies
over a period of time. Users can specify in the configuration 
when to profile and for how long.  
%We used this tool to produce
%some of the experimental results in \secref{sec:experiments}.

\textbf{\textit{Accumulator.}} The accumulator
maintains statistical profiles for feeds, including their error characteristics. For
numeric data, the accumulator keeps aggregates such
as averages, max/min values and standard deviations. For other
data (\eg{}, strings, URLs and IP addresses), it keeps the
frequency of the top $N$ most common values. For all data, it
tracks error rates, the most common error values and their
sources.  The user can configure the accumulator to profile entire
feeds at once, or incrementally. The latter option 
allows users to continuously monitor feeds
and compare their current behavior with historical statistics.  The
accumulator can output either plain text or XML. 
%\figref{fig:acc} shows portions of text-based accumulator output
%for the CoMon example.

\textbf{\textit{Alerter.}} The alerter allows users to register boolean 
functions which generate notifications when they evaluate to false on
feed items. The tool appends these notifications to a file, which can
be piped into other tools.  The system provides a library of 
common alerters such as exceeding max/min thresholds or deviating from the norm
({\em i.e.,} trigger an alert when a selected value strays more than $k$
standard deviations from its historical value). 
Users can supply their own conditions by giving \ocaml{} predicates
in the configuration file.  These predicates can consider both feed values 
and meta-data about the feed when determining whether to
trigger an alert.
%This tool has also been integrated with the accumulator described above. 
%The alerter notifies users of violated conditions by
%appending to a file, which can then be piped to other tools such as an email
%client.

\textbf{\textit{Database loader.}} This tool allows users to load numerical
data from a feed into a Round Robin Database
(RRD)~\cite{rrdtool}. Users specify a function to transform
feed items into numeric values and RDD parameters such as data source 
type and sampling rate. RRD indexes the data by arrival time.  It
periodically discards old data to make space for new. The tool
supports time-indexed queries and displays historical data as graphs.

\textbf{\textit{Selector.}} The selector allows users to choose
subcomponents of feed elements using path expressions. It returns a
feed of the selected subcomponents, which may then be fed into other
tools. 
%\figref{fig:selector-lang} shows the path expression language,
%partly inspired by \xpath{}~\cite{xpath}.


\textbf{\textit{RSS feed generator.}} The RSS feed generator converts a
\padsd{} feed to an RSS feed. Users specify the title, link (source),
description, update schedule and contents of the RSS feed.  Content
specifications are written in the path expression language.


%We apply the tools in parallel to
%achieve maximum throughput and also because some of the tools, such as feedaccum,
%operate on a per feed basis, rather than a per item basis. And thus running the tools
%in pipeline is not optimal.

%In summary, the tool configuration in \padsd{} is a very simple
%programming paradigm where minimum \ocaml{} language knowledge is
%required. Feed designers and system administrators who hardly know any
%functional programming can still perform day-to-day data management
%tasks. For users who do know \ocaml{} programming, more control is
%available by programming directly against the \padsd{} tools library.

