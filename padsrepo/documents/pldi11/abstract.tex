Many applications use the file system as a simple persistent data
store.  This approach is expedient, but not robust.  
The correctness of such an application depends on the
collection of files, directories, and symbolic links having a
precise organization. Furthermore these components must have
acceptable values for a variety of file system attributes 
such as ownership, permissions, and timestamps. Unfortunately, current
programming languages do not support documenting assumptions about the
file system. In 
addition, actually loading data from disk requires writing tedious
boilerplate code.

This paper describes \forest{}, a new domain-specific language
embedded in \haskell{} for describing directory structures. \forest{}
descriptions use a type-based metaphor to specify portions of the file
system in a simple, declarative manner.  \forest{} makes it easy to
connect data on disk to an isomorphic representation in memory that
can be manipulated by programmers as if it were any other data
structure in their program.  \forest{} generates metadata that
describes to what degree the files on disk conform to the
specification, making error detection easy. As a result, the system greatly lowers
the divide between on-disk and in-memory representations of
data. \forest{} leverages \haskell{}'s powerful generic programming
infrastructure to make it easy for third-party developers to build
tools that work for any \forest{} description.  We illustrate the use
of this infrastructure to build a number of useful tools, including a
visualizer, permission checker, and description-specific replacements for a
number of standard shell tools. \forest{} has a formal semantics based
on classical tree logics.

\cut{
We present the design for \forest{} and describe the implementation of
a full working prototype. From a single compact description, the
\forest{} compiler generates a collection of \haskell{} types and
functions for validating and analyzing file system data.  In addition,
\forest{} generates type class instance declarations that make it
possible to exploit powerful generic programming paradigms that allow
third-party developers to build tools for querying, visualizing, and
debugging on-disk data in a generic way. We present examples
illustrating the use of \forest{} on a number of real-world directory
structures and programming tasks, including description-specific
replacements for a number of standard shell tools. Finally, we
formalize the core elements of the language as a simple calculus based
on classical tree logics.}