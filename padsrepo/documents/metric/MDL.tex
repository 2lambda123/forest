\section{Minimum Description Length}

We present the basic concept of \textit{minimum description length} for
reference. This is an excerpt from
\textit{A Tutorial Introduction to the Minimum Description Length Principle}
by Peter Gr\"{u}nwald.

\fbox{
\parbox{12cm}{
\textbf{Crude two-part version of MDL principle:}

Let $\calH^{(1)}, \calH^{(2)}, \ldots$ be a list of candidate models (e.g.
$\calH^{(2)}$ is the set of $k$-th degree polynomials), each containing
a set of point hypotheses (e.g. individual polynomials).
The best point hypothesis $H \in \calH^{(1)} \cup \calH^{(2)} \ldots$
to explain the data $D$ is the one which minimizes $L(H) + L(D|H)$, where
\begin{itemize}

\item
$L(H)$ is the length, in bits, of the description of the hypothesis; and

\item
$L(D|H)$ is the length, in bits, of the description of the data when
encoded with the help of the hypothesis.

\end{itemize}

The best \textit{model} to explain $D$ is the smallest model containing
the selected $H$.
}
}
