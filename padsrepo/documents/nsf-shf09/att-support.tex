\documentclass{letter}
\usepackage{suresh-attletter-belanger}
\usepackage{times}
% common LaTeX macros used in Moby papers
%
% Last modified: 03-24-2003
%

\usepackage{amssymb} % for \pitchfork

\newcommand{\NOTE}[1]{%
  \par\leavevmode\noindent\textbf{[[ #1 ]]}\par\leavevmode\noindent}
\newcommand{\CUT}[1]{}

\newcommand{\appref}[1]{Appendix~\ref{#1}}
\newcommand{\chapref}[1]{Chapter~\ref{#1}}
\newcommand{\secref}[1]{Section~\ref{#1}}
\newcommand{\tblref}[1]{Table~\ref{#1}}
\newcommand{\figref}[1]{Figure~\ref{#1}}
\newcommand{\listingref}[1]{Listing~\ref{#1}}
\newcommand{\pref}[1]{{page~\pageref{#1}}}
\newcommand{\defref}[1]{Definition~\ref{#1}}
\newcommand{\ruleref}[1]{Rule~\ref{#1}}

\newcommand{\eg}{{\em e.g.}}
\newcommand{\cf}{{\em cf.}}
\newcommand{\ie}{{\em i.e.}}
\newcommand{\etc}{{\em etc.\/}}
\newcommand{\naive}{na\"{\i}ve}
\newcommand{\ala}{{\em \`{a} la\/}}
\newcommand{\etal}{{\em et al.\/}}
\newcommand{\role}{r\^{o}le}
\newcommand{\vs}{{\em vs.}}
\newcommand{\forte}{{fort\'{e}\/}}

%
% language names
\newcommand{\Cplusplus}{\mbox{C\hspace{-.05em}\raisebox{.4ex}{\tiny\bf ++}}}
\newcommand{\Cmm}{\mbox{C\hspace{-.05em}\raisebox{.4ex}{\small\textbf{{-}{-}}}}} 
\newcommand{\csharp}{\textsc{C\#}}
\newcommand{\C}{\textsc{C}}
\newcommand{\Ckit}{\textsc{Ckit}}
\newcommand{\java}{\textsc{Java}}
\newcommand{\loom}{\textsc{Loom}}
\newcommand{\moby}{\textsc{Moby}}
\newcommand{\minimoby}{\textsc{MiniMoby}}
\newcommand{\micromoby}{\textsc{microMoby}}
\newcommand{\MOC}{\textsc{MOC}}
\newcommand{\ml}{\textsc{ML}}
\newcommand{\sml}{\textsc{SML}}
\newcommand{\smlnj}{\textsc{SML/NJ}}
\newcommand{\mlj}{\textsc{MLj}}
\newcommand{\cml}{\textsc{CML}}
\newcommand{\pml}{\textsc{PML}}
\newcommand{\ocaml}{\textsc{OCaml}}
\newcommand{\mlkk}{\textsc{ML2000}}
\newcommand{\haskell}{\textsc{Haskell}}
\newcommand{\mltwok}{\textsc{ML2000}}
\newcommand{\scheme}{\textsc{Scheme}}
\newcommand{\unix}{\textsc{Unix}}
\newcommand{\smalltalk}{\textsc{Smalltalk}}
\newcommand{\self}{\textsc{Self}}

%
% font commands
\providecommand{\bftt}[1]{{\ttfamily\bfseries{}#1}}
\providecommand{\ittt}[1]{{\ttfamily\itshape{}#1}}
\providecommand{\kw}[1]{\bftt{#1}}
\providecommand{\nt}[1]{{\rmfamily\itshape{#1}}}
\providecommand{\term}[1]{{\sffamily{#1}}}
%
% math-mode versions
\providecommand{\mkw}[1]{\ensuremath{\text{\kw{#1}}}}
\providecommand{\mnt}[1]{\ensuremath{\text{\nt{#1}}}}
\providecommand{\mterm}[1]{\ensuremath{\text{\term{#1}}}}

% braces (in math mode)
\newcommand{\LCB}{\mkw{\{}}
\newcommand{\RCB}{\mkw{\}}}

% underscore
\newcommand{\US}{\char`\_}

%%%%%
% Some common math notation
%

% double brackets
\newcommand{\LDB}{\ensuremath{[\mskip -3mu [}}
\newcommand{\RDB}{\ensuremath{]\mskip -3mu ]}}

\newcommand{\dom}{\ensuremath{\mathrm{dom}}}
\newcommand{\rng}{\ensuremath{\mathrm{rng}}}

% sets
\newcommand{\SET}[1]{\ensuremath{\{#1\}}}
\newcommand{\Fin}{\textrm{Fin}}		% finite power set
\newcommand{\DISJOINT}[2]{\ensuremath{#1 \pitchfork #2}}
\newcommand{\finsubset}{\mathrel{\stackrel{\textrm{fin}}{\subset}}}


% finite maps
\newcommand{\finmap}{\mathrel{\stackrel{\textrm{fin}}{\rightarrow}}}
\newcommand{\MAP}[2]{\SET{#1 \mapsto #2}}
\newcommand{\EXTEND}[2]{\ensuremath{#1{\pm}#2}}
\newcommand{\EXTENDone}[3]{\EXTEND{#1}{\MAP{#2}{#3}}}
\newcommand{\SUBST}[3]{\ensuremath{#1[#2\mapsto{}#3]}}


% timestamp
\newcount\timeHH
\newcount\timeMM
\timeHH=\time
\divide\timeHH by 60
\timeMM=\time
\count255=\timeHH
\multiply\count255 by -60 \advance\timeMM by \count255
\newcommand{\timestamp}{%
  \today{} --- 
  \ifnum\timeHH<10 0\fi\number\timeHH\,:\,\ifnum\timeMM<10 0\fi\number\timeMM}

\signature{David Belanger\\
           Vice President-Information, Software, and Systems Research\\
           Chief Scientist}
\begin{document}
\begin{letter}{
}

\opening{Dear NSF Program Officer,} 

AT\&T Labs-Research is committed to participating in the research
initiative set forth in David Walker's (Princeton University)
NSF CCF proposal (NSF solicitation 09-555) 
entitled {\em SHF:Small:Language Support for Ad Hoc Data Processing}.

One of the biggest challenges in the telecommunications industry is
effectively manipulating the vast quantities of data associated with
running the various networks. Properly understanding this data is of
critical importance, both to detect fraud and to monitor the
performance of these networks for outages, denial of service attacks,
provisioning errors, etc. These tasks are complicated by legacy and ad
hoc data formats, the volume of data to be processed, poor
documentation, and errors in the data. The research described in this
proposal addresses each of these problems by providing a declarative
description language for describing ad hoc data sources as they are
and a system that converts such descriptions into an efficient suite
of tools for understanding and transforming the associated data in a
principled way.

The work outlined is quite challenging in terms of design,
implementation, and theory. At the design level, the general question is how
to provide simply and concisely sufficient expressiveness to permit
analysts to describe the many unruly formats they encounter in
practice. Implementation-wise, the challenge is to produce powerful
but efficient tools that scale to processing gigabytes worth of
data. Theoretically, the goal is to establish precise semantics for
the data description language and the associated tools so the system
can guarantee that the output of the tools is consistent with the
semantic properties asserted in the data description. The semantically
sound, efficient tools produced by this project will provide a strong
platform upon which to build novel applications for processing ad hoc
data. The research described in this proposal is a wonderful example
of using an innovative language design to solve a critical industry
problem.

AT\&T Labs-Research commits to supporting this activity should NSF
fund the proposal. Kathleen Fisher, a senior researcher, will fully
and actively participate in the research at no cost to NSF, including
project meetings and graduate student supervision. AT\&T Labs-Research
will where appropriate also make available experimental data to be
used to evaluate the project. Finally, we will explore supporting
summer internship opportunities at AT\&T Labs-Research for graduate
students involved in the project. All these add up to a substantial
commitment of AT\&T resources, and indicate our enthusiasm for the
work outlined in this proposal.

\closing{Yours Sincerely,}

\end{letter}

\end{document}


